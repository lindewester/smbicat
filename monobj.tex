\section{Monoidal objects in locally cubical bicategories}
\label{sec:mono-objects}

We now move on to define an appropriate abstract sort of ``monoidal object'' that will be preserved by the product-preserving functor $\cH$, and that specializes to monoidal double categories and to monoidal bicategories.
It would be nice if we could stay entirely in the world of iconic tricategories (that is, \Icon-enriched bicategories); but unfortunately the usual composition of monoidal functors between monoidal bicategories is not strictly associative, so they do not form an iconic tricategory.

However, they do form a more general structure, namely a bicategory enriched over \cDbl; in~\cite{gg:ldstr-tricat} this is called a \textbf{locally cubical bicategory}.
Since any bicategory can be regarded as a double category with only identity 1-morphisms, any iconic tricategory can be regarded as a locally cubical bicategory, but the latter are more general.
In particular, in a locally cubical bicategory the composition of 1-morphisms is associative only up to an invertible (tight) 2-morphism.
And indeed, one of the results of~\cite{gg:ldstr-tricat} is that monoidal bicategories form a locally cubical bicategory; here we will generalize this to monoidal objects, perhaps braided and symmetric, in any iconic tricategory with finite products --- and indeed, in any locally cubical bicategory with finite products.

Since \cDbl\ is also a cartesian monoidal 2-category, we can define what it means for a locally cubical bicategory to have finite products, and this property is preserved when regarding an iconic tricategory as a locally cubical bicategory.
In particular, this applies to \cDblf\ and to \cBicat\ --- but actually, in place of the iconic tricategory \cBicat\ considered up until now we will focus instead on the locally cubical bicategory of bicategories constructed in~\cite{gg:ldstr-tricat}, whose ``locally loose part'' is \cBicat, but whose tight 2-cells are \emph{icons}.
We denote this by \fBicat; it is easy to see that it also has products preserved by the inclusion $\cBicat\to \fBicat$, so that the composite functor $\cH : \cDblf \to\fBicat$ still preserves products.

We now define symmetric, braided and monoidal structures on objects, 1-cells, 2-cells, and 3-cells internal to a locally cubical bicategory with products, by taking the definitions of monoidal, braided, and symmetric structure for bicategories given in~\cite{nick:tricatsbook},~\cite{mccrudden:bal-coalgb}, and~\cite{gg:ldstr-tricat} and regarding the data of bicategories, functors, pseudonatural transformations, and modifications abstractly as objects, 1-cells, 2-cells, and 3-cells in a locally cubical bicategory.

Note that under this translation pseudonatural transformations become \emph{loose} 2-cells and modifications become globular 3-cells.
The loose 2-cells in \cDblf\ (which has no nonidentity tight 2-morphisms) are the (tight) transformations, while those in \fBicat\ are exactly the pseudonatural transformations (its tight 2-morphisms are icons).

Before we give the definitions of monoidal cells, we recall the structure of a locally cubical bicategory. Locally cubical bicategories have three types of composition. As a locally cubical category is a specific kind of intercategory, we will adopt the notation introduced for intercategories in~\cite{gp:intercategories-i}. Firstly, we have loose composition "$\horc$" within the hom- double categories. This gives us composition of loose 2-cells along a 1-cell boundary and of 3-cells along a tight 2-cell bounday. We write this composition in the order of diagrammatic composition: $\alpha \horc \beta$, meaning "$\beta$ after $\alpha$". The identity for loose composition is written as $\looseid$. Loose composition is weakly associative and loose identities hold up to isomorphism. We write $U_{f}$ and $U_{\alpha}$ for the loose identity on a 1-cell $f$ and a tight 2-cell $\alpha$, respectively. 
Secondly, we have tight composition "$\verc$" in the hom-double categories. This gives us composition of tight 2-cells along a 1-cell boundary and tight composition of 3-cells along a loose 2-cell boundary, written in the conventional order: $f \verc g$ denoting "$f$ after $g$". The identity for tight composition is given by $\tightid$. Tight composition is strictly associative and has strict identities.
Thirdly, there is composition "$\comp$" of 1-cells, 2-cells, and 3-cells along a 0-cell boundary, given by the enriched structure. We write this composition in the conventional order: $f \comp g$ meaning "$f$ after $g$". When it is clear from the context, we omit the composition symbol "$\comp$",  and write the juxtaposition of 1-cells instead. The identity for this composition is given by $\transid$. This composition is weak with weak identities. We write $\compI$ for the unit double functor. Note that by functoriality, the image of $\compI_A$ is given by some 1-cell $\compI_A$, the identity tight 2-cell $\tightid_{\compI_A}$, some loose 2-cell $\compI_{\compI_A}$ which is naturally isomorphic to the identity for loose composition $\looseid_{\compI_A}$, and the identity 3-cell  $\id_{\compI_{\compI_A}}$.
As two of the compositions, "$\horc$", and "$\comp$" are weak, we sometimes use special notation to distinguish between their associators and unitors, given by the composition symbol in superscript: $\hora, \horr$, $\horl$, and $\compa, \compr$, $\compl$, respectively.

Let \fB\ be a locally cubical bicategory with products.

\begin{defn}
A {\bf monoidal object} in \fB\ is an object $A$, equipped with 1-cells $\otimes: A \times A \onecell A$ and $I_A: * \onecell A$, and loose 2-cells
\begin{itemize} 
\item $\alpha: \ten  (\id \times \tens) \looseRightarrow{} \ten (\tens \times \id)$
\item $l: \ten (I \times \transid) i_2 \looseRightarrow{} \transid$ and $r:\ten (\transid \times I) i_1 \looseRightarrow{} \transid$ 
\end{itemize}

where $i_1$ and $i_2$ are the morphisms defining the product $A \times A$. Finally, it must be equipped with the invertible globular 3-cells $\pi, \mu, \lambda, \rho$, relating the two different ways around the Mac Lane pentagon and the three other coherence diagrams given in Definition 4.1 of~\cite{nick:tricatsbook}, which satisfy the appropriate three axioms.

A monoidal object is {\bf braided} if in addition there is a loose 2-cell $\sigma_A: \tens \looseRightarrow{} \ten \tau$, where $\tau: A \times A \rightarrow A \times A$ interchanges the two copies of $A$; and if there are invertible globular 3-cells 

\begin{equation}
  \begin{aligned}
\begin{tikzpicture}[xscale=0.9]
\node (t) at (2,3) {$\ten (\ten \times \transid)$};
\node (tl) at (0,2) {$\ten(\ten \times \transid)$};
\node (bl) at (0,1) {$\ten (\transid \times \ten)$};
\node (b) at (2,0) {$\ten (\transid \times \ten)$};
\node (tr) at (4,2) {$\ten(\transid \times \ten)$};
\node (br) at (4,1) {$\ten (\ten \times \transid)$};
\draw[doubleloose] (t) to node [above,xshift=10pt, yshift=-2] {$\alpha$} (tr);
\draw[doubleloose] (tr) to node [right] {$\sigma$} (br);
\draw[doubleloose] (br) to node [below,xshift=10pt, yshift=2pt] {$\alpha$} (b);
\draw[doubleloose] (t) to node [above, xshift=-10pt, yshift=-2pt] {$\sigma \ten \looseid$} (tl);
\draw[doubleloose] (tl) to node [left] {$\alpha$} (bl);
\draw[doubleloose] (bl) to node [below,xshift=-10pt,yshift=2pt] {$\looseid \ten \sigma$} (b);
\node at (2,1.5) {$\DDownarrow R \iso$};
\end{tikzpicture}
  \end{aligned}
\hspace{5pt}\mbox{and} \hspace{5pt}
\begin{aligned}
\begin{tikzpicture}[xscale=0.9]
\node (t) at (2,3) {$\ten(\transid \times \ten)$};
\node (tl) at (0,2) {$\ten(\transid \times \ten)$};
\node (bl) at (0,1) {$\ten(\ten \times \transid)$};
\node (b) at (2,0) {$\ten(\ten \times \transid)$};
\node (tr) at (4,2) {$\ten(\ten \times \transid)$};
\node (br) at (4,1) {$\ten(\transid \times \ten)$};
\draw[doubleloose] (tr) to node [above,xshift=10pt, yshift=-2] {$\alpha$} (t);
\draw[doubleloose] (tr) to node [right] {$\sigma$} (br);
\draw[doubleloose] (b) to node [below,xshift=10pt, yshift=2pt] {$\alpha$} (br);
\draw[doubleloose] (t) to node [above,xshift=-10pt, yshift=-2pt] {$\mbox{id} \ten \sigma$} (tl);
\draw[doubleloose] (tl) to node [left] {${\alpha}^{-1}$} (bl);
\draw[doubleloose] (bl) to node [below,xshift=-10pt,yshift=2pt] {$\sigma \ten \mathid$} (b);
\node at (2,1.5) {$\DDownarrow S \iso$};
\end{tikzpicture}
\end{aligned}
\end{equation}
satisfying the axioms (BA1), (BA2), (BA3), and (BA4) given in~\cite[p136--139]{mccrudden:bal-coalgb} . 
It is {\bf sylleptic} when there exists an invertible globular 3-cell

 \[
 \begin{tikzpicture}
 \node (tl) at (-2,1) {$\ten$};
 \node (tr) at (2,1) {$\ten$};
 \node (b) at (0,-.25) {$\tens \tau$};
 \draw[double] (tl)  -- (tr);
 \draw[doubleloose] (tl) to node[left, yshift=-5pt]{$\sigma$} (b);
 \draw[doubleloose] (b) to node[right, yshift=-5pt] {$\sigma$}(tr);
 \node at (0,0.5) {\footnotesize $\DDownarrow \upsilon \iso$}; 
 \end{tikzpicture}
 \]
  satisfying the axioms (SA1), (SA2) on~\cite[p144--145]{mccrudden:bal-coalgb}. It is {\bf symmetric} if in addition, it satisfies the axiom given on~\cite[p91]{mccrudden:bal-coalgb}.
\end{defn}

By construction, these definitions give the expected results in \fBicat.
In \cDblf, where there are no nonidentity 3-cells, they reduce to the definitions from section~\ref{sec:symm-mono-double}; and in particular every syllepsis is a symmetry.

\begin{defn}
Let $A,B$ be monoidal objects in \fB. A 1-cell $f:A \onecell B$ is {\bf lax monoidal} when it is equipped with the following loose 2-cells:
\begin{itemize}
\item $\chi: \ten (f \times f) \looseRightarrow{} f  \otimes$
\item $\iota: I_B \looseRightarrow{} f I_A$
\end{itemize}
As well as globular invertible 3-cells 

\begin{align*}
& \omega: \looseid_f \alpha_A \horc \chi(\looseid_{\tens \times \tightid} \horc \looseid_{\tens}(\chi \times \looseid_f) \RRightarrow \chi \looseid_{\tightid \times \tens} \horc \looseid_{\tens}(\looseid \times \chi) \horc \alpha_B\looseid_{f \times f \times f} \\
 &\gamma: \looseid_{\tens}(\iota_f \times \looseid_f) \looseid_{i_2} \horc \chi \looseid_{I \times \transid} \looseid_{i_2} \horc \looseid_f l\RRightarrow l \looseid_f \\
 &\delta: \looseid_f r \horc \chi \looseid_{\tightid \times I} \horc \looseid_{\tens} (\looseid_{f}\times \iota) \RRightarrow r \looseid_f
\end{align*}

 given in Definition 4.10 of~\cite{nick:tricatsbook}, expressing the usual associativity and unitality conditions, which satisfy the three given commutativity axioms.
A monoidal 1-cell is called {\bf braided}, when $A$ and $B$ are braided and there is a globular 3-cell $u: \sigma_B \looseid_{f \times f} \verc \chi  \looseid_{\tau} \looseRightarrow \chi \horc (\looseid_f \sigma_A)$, satisfying the braiding axioms analogous to (BHA1) and (BHA2) given in  \cite[p141-142]{mccrudden:bal-coalgb}. It is {\bf symmetric} when $A$ and $B$ are symmetric and the 3-cells defining the braided monoidal structure of $f$ satisfy the additional axiom analogous to  (SHA1) given in   \cite[p145]{mccrudden:bal-coalgb}.

When the 2-cells go in the opposite direction, the morphism $f$ is {\bf oplax monoidal}, and when they are adjoint equivalences, it is {\bf strong monoidal}.
\end{defn}



\begin{defn}\label{Def:monverttrans}
Let $f, g:A \onecell B$ be lax monoidal 1-cells in \fB. A {\bf lax monoidal 2-cell} $\beta: f \looseRightarrow g$ is a loose 2-cell in \fB\ that is equipped with globular 3-cells
\begin{itemize}
\item $\Pi: \chi_f \horc (\beta \comp \looseid_{\ten}) \RRightarrow{} \looseid_{\ten}(\beta \times \beta) \horc \chi_g$
\item $M: \iota_f \horc (\beta \comp \looseid_{\ten}) \RRightarrow{} \looseid_{I_A} \horc \iota_g$
\end{itemize}

Such that coherence equations \ref{eq:monicon1}, \ref{eq:monicon2}, and \ref{eq:monicon3} below hold.

Let $f, g:A \onecell B$ be oplax monoidal 1-cells in \fB. An {\bf oplax monoidal 2-cell} $\beta: f \looseRightarrow{} g$ is a loose 2-cell in \fB\ that is equipped with globular 3-cells
\begin{itemize}
\item $\Pi': \chi_f \horc \looseid_{\ten}(\beta \times \beta)  \RRightarrow{} (\beta \comp \looseid_{\ten}) \horc \chi_g $
\item $M':   \iota_f \horc \looseid_{I} \RRightarrow{} (\beta \comp \looseid_{\ten}) \horc \iota_g $
\end{itemize}
\end{defn}

 Such that coherence equations analogous to \ref{eq:monicon1}, \ref{eq:monicon2}, and \ref{eq:monicon3} hold.

If $g,f$ are strong monoidal, we call $\beta$ a {\bf strong monoidal 2-cell} when it is equipped with $M, \Pi, M '$ and $\Pi '$, which correspond to each other in pairs as mates under the adjoint equivalence structure on $\chi$ and $\iota$.

A monoidal 2-cell is {\bf braided} or {\bf symmetric} when $f,g$ are braided or symmetric, and in addition the coherence axiom~\ref{eq:bricon} holds. 

\begin{equation}\label{eq:monicon1}
\begin{aligned}
\begin{tikzpicture}[xscale=3, yscale=1.5]
\node (t0) at (0,2) {\small $\tens(I_B \times f)i_2$};
\node (t1) at (1,2) {\small $\tens(f I_A \times f)i_2$};
\node (t2) at (2,2) {\small $f \tens(I_A \times \transid)i_2$};
\node (t3) at (3,2) {\small $f$};
\node (t4) at (4,2) {\small $g$};
\node (m0) at (0,1) {\small $\tens(I_B \times \transid)i_2f$};
\node (m3) at (3,1) {\small $f$};
\node (m4) at (4,1) {\small $g$};
\node (b0) at (0,0) {\small $\tens(I_B \times \transid)i_2f$};
\node (b3) at (3,0) {\small $\tens (I_B \times \transid)i_2g$};
\node (b4) at (4,0) {\small $g$};
\draw[doubleloose] (t0) to node[above]{\small $\looseid_{\tens}(\iota_f \times \looseid_f)\looseid_{i_2}$} (t1);
\draw[doubleloose] (t1) to node[above]{\small $\chi (\looseid_{I \times \transid})\looseid_{i_2}$} (t2);
\draw[doubleloose] (t2) to node[above]{\small $\looseid_f l$} (t3);
\draw[doubleloose] (t3) to node[above]{\small $\beta$} (t4);
\draw[doubleloose] (m0) to node[above]{\small $l \looseid_f$} (m3);
\draw[doubleloose] (m3) to node[above]{\small $\beta$} (m4);
\draw[doubleloose] (b0) to node[above]{\small $\looseid_{\tens}(\beta \times \looseid_I)\looseid_{i_2}$} (b3);
\draw[doubleloose] (b3) to node[above]{\small $l \looseid_g$} (b4);
\draw[doubletighteq] (t0) to (m0);
\draw[doubletighteq] (m0) to (b0);
\draw[doubletighteq] (t3) to (m3);
\draw[doubletighteq] (t4) to (m4);
\draw[doubletighteq] (m4) to (b4);
\node at (1.5,1.5) {\small $\DDownarrow \gamma^f$};
\node at (3.5,1.5) {\small $\DDownarrow \tightid_{\beta}$};
\node at (2,0.5) {\small $\iso$};
\end{tikzpicture}
\end{aligned}
\end{equation}
\[
=
\]
\begin{equation*}
\begin{aligned}
\begin{tikzpicture}[xscale=3, yscale=1.5]
\node (04) at (0,4) {\small $\tens(I_B \times f)i_2$};
\node (14) at (1,4) {\small $\tens(f I_A \times I_A)i_2$};
\node (24) at (2,4) {\small $\tens(I_A\times \transid_A)i_2$};
\node (34) at (3,4) {\small $f$};
\node (44) at (4,4) {\small $g$};
%%%%%
\node (03) at (0,3) {\small $\tens(I_B \times f)i_2$};
\node (13) at (1,3) {\small $\tens(f I_A\times f)i_2$};
\node (23) at (2,3) {\small $f \tens(I_A \times \transid_A)i_2$};
\node (33) at (3,3) {\small $g \tens(I_A \times \transid_A)i_2$};
\node (43) at (4,3) {\small $g$};
%%%%%%
\node (02) at (0,2) {\small $\tens(I_B \times f)i_2$};
\node (12) at (1,2) {\small $\tens(f I_A \times f)i_2$};
\node (22) at (2,2) {\small $\tens(g I_A \times g)i_2$};
\node (32) at (3,2) {\small $g \tens (I_A\times \transid_A)i_2$};
\node (42) at (4,2) {\small $g$};
%%%%%% 
\node (01) at (0,1) {\small $\tens(I_B \times f)i_2$};
\node (11) at (1,1) {\small $\tens(I_B \times g)i_2$};
\node (21) at (2,1) {\small $\tens(g I_A \times g)i_2$};
\node (31) at (3,1) {\small $g \tens (I_A \times \transid_A)i_2$};
\node (41) at (4,1) {\small $g$};
%%%%%%%
\node (00) at (0,0) {\small $\tens(I_B \times \transid)i_2 f$};
\node (10) at (1,0) {\small $\tens(I_B \times \transid)i_2 g$};
\node (40) at (4,0) {\small $g$};
%%%%%%%
\draw[doubleloose] (04) to node[above]{\small $\looseid_{\tens}(\iota_f \times \looseid_f)\looseid_{i_2}$} (14);
\draw[doubleloose] (14) to node[above]{\small $\chi_f \looseid_{I \times \transid}\looseid_{i_2}$} (24);
\draw[doubleloose] (24) to node[above]{\small $\looseid_{f}l$} (34);
\draw[doubleloose] (34) to node[above]{\small $\beta$} (44);
%%%%
\draw[doubleloose] (13) to node[above]{\small $\chi (\looseid_{I \times \transid})\looseid_{i_2}$} (23);
\draw[doubleloose] (23) to node[above]{\small $\beta \looseid_{\tens}(\looseid_{I\times \transid})\looseid_{i_2}$} (33);
\draw[doubleloose] (33) to node[above]{\small$\looseid_g l$} (43);
%%%%
\draw[doubleloose] (02) to node[above]{\small $\looseid_{\tens} (\iota_f  \times \looseid_f)\looseid_{i_2}$} (12);
\draw[doubleloose] (12) to node[above]{\small $\looseid_{\tens} (\beta \looseid_I \times \beta)\looseid_{i_2} $} (22);
\draw[doubleloose] (22) to node[above]{\small $\chi_g \looseid_{I \times \id}\looseid_{i_2}$} (32);
\draw[doubleloose] (32) to node[above]{\small $\looseid_g l$} (42);
%%%%%%
\draw[doubleloose] (01) to node[above]{\small $\looseid_{\tens} (\looseid_I \times \beta)\looseid_{i_2}$} (11);
\draw[doubleloose] (11) to node[above]{\small $\looseid_{\tens} (\iota_g \times \looseid_g) \looseid_{i_2}$} (21);
\draw[doubleloose] (21) to node[above]{\small $\chi \looseid_{I_A \times \transid}\looseid_{i_2}$} (31);
\draw[doubleloose] (31) to node[above]{\small $\looseid_g l$} (41);
%%%%%%
\draw[doubleloose] (00) to node[above]{\small $\looseid_{\tens} (\looseid_I  \times \looseid)\looseid_{i_2} \beta$} (10);
\draw[doubleloose] (10) to node[above]{\small $l \looseid_g $} (40);
%%%%%%
\draw[doubletighteq] (04) to (03);
\draw[doubletighteq] (14) to (13);
\draw[doubletighteq] (24) to (23);
\draw[doubletighteq] (44) to (43);
%%%%%%
\draw[doubletighteq] (03) to (02);
\draw[doubletighteq] (13) to (12);
\draw[doubletighteq] (33) to (32);
\draw[doubletighteq] (43) to (42);
%%%%%%
\draw[doubletighteq] (02) to (01);
\draw[doubletighteq] (22) to (21);
\draw[doubletighteq] (42) to (41);
%%%%%%
\draw[doubletighteq] (01) to (00);
\draw[doubletighteq] (11) to (10);
\draw[doubletighteq] (41) to (40);
%%%%%%%%
\node at (.5,3) {\small $=$};
\node at (1.5,3.5) {\small $=$};
\node at (3,3.5) {\small $\iso$};
\node at (2,2.5) {\small $\DDownarrow \overline{\Pi^{\beta}\tightid_{\looseid}}$};
\node at (3.5,2.5) {\small $=$};
\node at (1,1.5) {\small $\DDownarrow \overline{\tightid_{\looseid} (M^{\beta} \times (\horl \verc {\horr}^{-1}) }$};
\node at (3,1.5) {\small $=$};
\node at (.5,.5) {\small $=$};
\node at (2.5,0.5) {\small $\DDownarrow \gamma^g$};
\end{tikzpicture}
\end{aligned}
\end{equation*}

%%%%%%%%%%%%%%%%%%%%%%%%%%%%%%%%%

\begin{equation}\label{eq:monicon2}
\begin{aligned}
\begin{tikzpicture}[xscale=3, yscale=1.5]
\node (t0) at (0,2) {\small $\tens(f\times I_B)i_1$};
\node (t1) at (1,2) {\small $\tens(f\times fI_A)i_1$};
\node (t2) at (2,2) {\small $\tens(\id \times I_A)i_1$};
\node (t3) at (3,2) {\small $f$};
\node (t4) at (4,2) {\small $g$};
\node (m0) at (0,1) {\small $\tens(\transid \times I_B)i_1f$};
\node (m3) at (3,1) {\small $f$};
\node (m4) at (4,1) {\small $g$};
\node (b0) at (0,0) {\small $\tens(\transid \times I_B)i_1f$};
\node (b3) at (3,0) {\small $\tens (\transid \times I_B)i_1g$};
\node (b4) at (4,0) {\small $g$};
\draw[doubleloose] (t0) to node[above]{$\small \looseid_{\tens}(\looseid_f \times \iota_f)\looseid_{i_1}$} (t1);
\draw[doubleloose] (t1) to node[above]{\small $\chi (\id \times \looseid_I)\looseid_{i_1}$} (t2);
\draw[doubleloose] (t2) to node[above]{\small $\looseid_{f}r$} (t3);
\draw[doubleloose] (t3) to node[above]{\small $\beta$} (t4);
\draw[doubleloose] (m0) to node[above]{\small $r \looseid_f$} (m3);
\draw[doubleloose] (m3) to node[above]{\small $\beta$} (m4);
\draw[doubleloose] (b0) to node[above]{\small $\looseid_{\tens}(\beta \times \looseid_I)\looseid_{i_1}$} (b3);
\draw[doubleloose] (b3) to node[above]{\small $r \looseid_g$} (b4);
\draw[doubletighteq] (t0) to (m0);
\draw[doubletighteq] (m0) to (b0);
\draw[doubletighteq] (t3) to (m3);
\draw[doubletighteq] (t4) to (m4);
\draw[doubletighteq] (m4) to (b4);
\node at (1.5,1.5) {\small $\DDownarrow \delta^f$};
\node at (3.5,1.5) {\small $\DDownarrow \tightid_{\beta}$};
\node at (2,0.5) {\small $\iso$};
\end{tikzpicture}
\end{aligned}
\end{equation}
\[
=
\]
\begin{equation*}
\begin{aligned}
\begin{tikzpicture}[xscale=3, yscale=1.5]
\node (04) at (0,4) {\small $\tens(f\times I_B)i_1$};
\node (14) at (1,4) {\small $\tens(f\times fI_A)i_1$};
\node (24) at (2,4) {\small $f \tens(\transid \times I_A)i_1$};
\node (34) at (3,4) {\small $f$};
\node (44) at (4,4) {\small $g$};
%%%%%
\node (03) at (0,3) {\small $\tens(f\times I_B)i_1$};
\node (13) at (1,3) {\small $\tens(f\times f I_A)i_1$};
\node (23) at (2,3) {\small $f \tens(\transid \times I_A)i_1$};
\node (33) at (3,3) {\small $g \tens(\transid \times I_A)i_1$};
\node (43) at (4,3) {\small $g$};
%%%%%%
\node (02) at (0,2) {\small $\tens(f\times I_B)i_1$};
\node (12) at (1,2) {\small $\tens(f\times f I_A)i_1$};
\node (22) at (2,2) {\small $\tens(g\times g I_A)i_1$};
\node (32) at (3,2) {\small $g\tens(\transid \times I_A)i_1$};
\node (42) at (4,2) {\small $g$};
%%%%%%
\node (01) at (0,1) {\small $\tens(f\times I_B)i_1$};
\node (11) at (1,1) {\small $\tens(g\times  I_B)i_1$};
\node (21) at (2,1) {\small $\tens(g\times g I_A)i_1$};
\node (31) at (3,1) {\small $g \tens (\transid \times I_A)i_1$};
\node (41) at (4,1) {\small $g$};
%%%%%%%
\node (00) at (0,0) {\small $\tens(\transid \times I_B)i_1f$};
\node (10) at (1,0) {\small $\tens(\transid \times  I_B)i_1g$};
\node (40) at (4,0) {\small $g$};
%%%%%%%
\draw[doubleloose] (04) to node[above]{\small $\looseid_{\tens}(\looseid_f \times \iota_f)\looseid_{i_1}$} (14);
\draw[doubleloose] (14) to node[above]{\small $\chi \looseid_{\transid \times I})\looseid_{i_1}$} (24);
\draw[doubleloose] (24) to node[above]{\small $\looseid_{f}r$} (34);
\draw[doubleloose] (34) to node[above]{\small $\beta$} (44);
%%%%
\draw[doubleloose] (13) to node[above]{\small $\chi (\looseid_{\transid \times I})\looseid_{i_1}$} (23);
\draw[doubleloose] (23) to node[above]{\small $\beta \looseid_{\tens}(\looseid_{\transid \times I})\looseid_{i_1}$} (33);
\draw[doubleloose] (33) to node[above]{\small $\looseid_g r$} (43);
%%%%
\draw[doubleloose] (02) to node[above]{\small $\looseid_{\tens} (\looseid_f  \times \iota_f) \looseid_{i_1}$} (12);
\draw[doubleloose] (12) to node[above]{\small $\looseid_{\tens} (\beta \times \beta) \looseid_{\transid \times I}\looseid_{i_1}$} (22);
\draw[doubleloose] (22) to node[above]{\small $\chi \looseid_{\transid \times I}\looseid_{i_1}$} (32);
\draw[doubleloose] (32) to node[above]{\small $\looseid_g r$} (42);
%%%%%%
\draw[doubleloose] (01) to node[above]{\small $\looseid_{\tens} (\beta \times \looseid_I)\looseid_{i_1}$} (11);
\draw[doubleloose] (11) to node[above]{\small $\looseid_{\tens} (\looseid_g \times \iota_g) \looseid_{i_1}$} (21);
\draw[doubleloose] (21) to node[above]{\small $\chi \looseid_{\transid \times I}\looseid_{i_1}$} (31);
\draw[doubleloose] (31) to node[above]{\small $\looseid_g r$} (41);
%%%%%%
\draw[doubleloose] (00) to node[above]{\small $\looseid_{\tens} (\transid  \times \looseid_I)\looseid_{i_1} \beta$} (10);
\draw[doubleloose] (10) to node[above]{\small $r \looseid_g $} (40);
%%%%%%
\draw[doubletighteq] (04) to (03);
\draw[doubletighteq] (14) to (13);
\draw[doubletighteq] (24) to (23);
\draw[doubletighteq] (44) to (43);
%%%%%%
\draw[doubletighteq] (03) to (02);
\draw[doubletighteq] (13) to (12);
\draw[doubletighteq] (33) to (32);
\draw[doubletighteq] (43) to (42);
%%%%%%
\draw[doubletighteq] (02) to (01);
\draw[doubletighteq] (22) to (21);
\draw[doubletighteq] (42) to (41);
%%%%%%
\draw[doubletighteq] (01) to (00);
\draw[doubletighteq] (11) to (10);
\draw[doubletighteq] (41) to (40);
%%%%%%%%
\node at (.5,3) {\small $=$};
\node at (1.5,3.5) {\small $=$};
\node at (3,3.5) {\small $\iso$};
\node at (2,2.5) {\small $\DDownarrow \overline{\Pi^{\beta}\looseid_{\tightid \times I_A}}$};
\node at (3.5,2.5) {\small $=$};
\node at (1,1.5) {\small $\DDownarrow \overline{\tightid_{\looseid_{\tens}} (\horl \verc {\horr}^{-1}) \times M^{\beta}}$};
\node at (3,1.5) {\small $=$};
\node at (.5,.5) {\small $=$};
\node at (2.5,0.5) {\small $\DDownarrow \delta^g$};
\end{tikzpicture}
\end{aligned}
\end{equation*}

%%%%%%%%%%%%%%%%%%%%%%%%%%%%%%%%%
\begin{equation}\label{eq:monicon3}
\begin{aligned}
\begin{tikzpicture}[xscale=3.5, yscale=1.5]
\node (04) at (0,4) {\small$ \tens( \tens \times \transid)(f \times f \times f)$};
\node (14) at (1,4) {\small $ \tens(f \tens \times f)$};
\node (24) at (2,4) {\small $f \tens(\tens \times \transid)$};
\node (34) at (3,4) {\small $f\tens (\transid \times \tens)$};
\node (44) at (4,4) {\small $g \tens (\transid \times \tens)$};
\node (03) at (0,3) {\small $\tens( \tens \times \transid)(f \times f \times f)$};
\node (13) at (1,3) {\small $\tens( \transid \times \tens)(f \times f \times f)$};
\node (23) at (2,3) {\small $\tens (f \times f \tens)$};
\node (33) at (3,3) {\small $f \tens (\transid \times  \tens)$};
\node (43) at (4,3) {\small $g \tens (\transid \times  \tens)$};
\node (02) at (0,2) {\small $\tens( \tens \times \transid)(f \times f \times f)$};
\node (12) at (1,2) {\small $\tens( \transid \times \tens)(f \times f \times f)$};
\node (22) at (2,2) {\small $\tens (f \times f \tens)$};
\node (32) at (3,2) {\small $\tens (g \times g \tens)$};
\node (42) at (4,2) {\small $g \tens (\transid \times  \tens)$};
%%%%%%%
\node (01) at (0,1) {\small $\tens( \tens \times \transid)(f \times f \times f)$};
\node (11) at (1,1) {\small $\tens( \transid \times \tens)(f \times f \times f)$};
\node (21) at (2,1) {\small $\tens (\transid \times \tens) (g \times g \times g)$};
\node (31) at (3,1) {\small $\tens (g \times g \tens)$};
\node (41) at (4,1) {\small $g \tens (\transid \times  \tens)$};
%%%%%%%
\node (00) at (0,0) {\small $\tens( \tens \times \transid)(f \times f \times f)$};
\node (10) at (1,0) {\small $\tens( \transid \times \tens)(g \times g \times g)$};
\node (20) at (2,0) {\small $\tens (\transid \times \tens) (g \times g \times g)$};
\node (30) at (3,0) {\small $\tens (g \times g \tens)$};
\node (40) at (4,0) {\small $g \tens (\transid \times  \tens)$};
%%%%%%%
\draw[doubleloose] (04) to node[above]{\small $\looseid_{\tens}(\chi_f \times \looseid_f)$} (14);
\draw[doubleloose] (14) to node[above]{\small $\chi_f (\looseid_{\tens \times \transid})$} (24);
\draw[doubleloose] (24) to node[above]{\small $\looseid_{f}\alpha$} (34);
\draw[doubleloose] (34) to node[above]{\small $\beta \looseid_{\tens} \looseid_{\transid \times \tens}$} (44);
%%%%%%%%
\draw[doubleloose] (03) to node[above]{\small $\alpha \looseid_{f \times f \times f}$} (13);
\draw[doubleloose] (13) to node[above]{\small $\looseid_{\tens} (\looseid_{\transid} \times \chi_f)$} (23);
\draw[doubleloose] (23) to node[above]{\small $\chi_f \looseid_{\transid \times \tens}$} (33);
\draw[doubleloose] (33) to node[above]{\small $\beta \looseid_{\tens} \looseid_{\transid \times \tens}$} (43);
%%%%%%%%
\draw[doubleloose] (12) to node[above]{\small $\looseid_{\tens} (\looseid_{\transid} \times \chi_f)$} (22);
\draw[doubleloose] (22) to node[above]{\small $\looseid_{\tens} (\beta \times \beta) \looseid_{\transid \times \tens}$} (32);
\draw[doubleloose] (32) to node[above]{\small $\chi_g \looseid_{\transid \times \tens}$} (42);
%%%%%%%%
\draw[doubleloose] (01) to node[above]{\small $\alpha \looseid_{f \times f \times f}$} (11);
\draw[doubleloose] (11) to node[above]{\small $\looseid_{\tens} \looseid_{\transid \times \tens} (\beta \times \beta \times \beta)$} (21);
\draw[doubleloose] (21) to node[above]{\small $\looseid_{\tens} (\looseid_{\transid} \times \chi_g)$} (31);
%%%%%%%%
\draw[doubleloose] (00) to node[above]{\small $\looseid_{\tens} (\looseid_{\tens} \times \looseid_{\transid})(\beta \times \beta \times \beta)$} (10);
\draw[doubleloose] (10) to node[above]{\small $\alpha \looseid_{g \times g \times g}$} (20);
\draw[doubleloose] (20) to node[above]{\small $\looseid_{\tens} (\looseid_{\transid} \times \chi_g)$} (30);
\draw[doubleloose] (30) to node[above]{\small $\chi_g \looseid_{\transid \times \tens}$} (40);
%%%%%%%%
\draw[doubletighteq] (04) to (03);
\draw[doubletighteq] (34) to (33);
\draw[doubletighteq] (44) to (43);
%%%%%%%%%
\draw[doubletighteq] (03) to (02);
\draw[doubletighteq] (13) to (12);
\draw[doubletighteq] (23) to (22);
\draw[doubletighteq] (43) to (42);
%%%%%%%%%
\draw[doubletighteq] (02) to (01);
\draw[doubletighteq] (12) to (11);
\draw[doubletighteq] (32) to (31);
\draw[doubletighteq] (42) to (41);
%%%%%%%%%
\draw[doubletighteq] (01) to (00);
\draw[doubletighteq] (21) to (20);
\draw[doubletighteq] (31) to (30);
\draw[doubletighteq] (41) to (40);
%%%%%%%%%
\node at (1.5,3.5) {\small $\DDownarrow \omega^f$};
\node at (3.5,3.5) {\small $=$};
\node at (0.5,2) {\small $=$};
\node at (1.5,2.5) {\small $=$};
\node at (3,2.5) {\small $\DDownarrow \overline{\Pi^{\beta} \tightid_{\looseid_{\transid \times \tens}}}$};
\node at (2,1.5) {\small $\iso$};
\node at (3.5,1) {\small $=$};
\node at (1,0.5) {\small $\DDownarrow ({\horl}^{-1} \verc \horr) ({\horr}^{-1} \verc \horl)$};
\node at (2.5,0.5) {\small $=$};
\end{tikzpicture}
\end{aligned}
\end{equation}
\[
=
\]
\begin{equation*}
\begin{aligned}
\begin{tikzpicture}[xscale=3.75, yscale=1.5]
\node (04) at (0,4) {$\tens( \tens \times \transid)(f \times f \times f)$};
\node (14) at (1,4) {$\tens(f \tens \times f)$};
\node (24) at (2,4) {$f \tens(\tens \times \transid)$};
\node (34) at (3,4) {$f\tens (\transid \times \tens)$};
\node (44) at (4,4) {$g \tens (\transid \times \tens)$};
%%%%%%%%%
\node (03) at (0,3) {$\tens( \tens \times \transid)(f \times f \times f)$};
\node (13) at (1,3) {$\tens( f \tens \times f)$};
\node (23) at (2,3) {$f \tens ( \tens \times \transid)$};
\node (33) at (3,3) {$g \tens (\tens \times  \transid)$};
\node (43) at (4,3) {$g \tens (\transid \times \tens)$};
\node (02) at (0,2) {$\tens( \tens \times \transid)(f \times f \times f)$};
\node (12) at (1,2) {$\tens(f \tens \times f )$};
\node (22) at (2,2) {$\tens (g \tens \times g)$};
\node (32) at (3,2) {$g \tens (\tens \times \transid)$};
\node (42) at (4,2) {$g \tens (\transid \times  \tens)$};
%%%%%%%
\node (01) at (0,1) {$\tens( \tens \times \transid)(f \times f \times f)$};
\node (11) at (1,1) {$\tens( \tens \times \transid)(g \times g \times g)$};
\node (21) at (2,1) {$\tens (g \tens \times g)$};
\node (31) at (3,1) {$g \tens ( \tens \times \transid )$};
\node (41) at (4,1) {$g \tens (\transid \times  \tens)$};
%%%%%%%
\node (00) at (0,0) {$\tens( \tens \times \transid)(f \times f \times f)$};
\node (10) at (1,0) {$\tens( \transid \times \tens)(g \times g \times g)$};
\node (20) at (2,0) {$\tens (\transid \times \tens) (g \times g \times g)$};
\node (30) at (3,0) {$\tens (g \times g \tens)$};
\node (40) at (4,0) {$g \tens (\transid \times  \tens)$};
%%%%%%%
\draw[doubleloose] (04) to node[above]{$\looseid_{\tens}(\chi_f \times \looseid_f)$} (14);
\draw[doubleloose] (14) to node[above]{$\chi_f (\looseid_{\tens \times \transid})$} (24);
\draw[doubleloose] (24) to node[above]{$\looseid_{f}\alpha$} (34);
\draw[doubleloose] (34) to node[above]{$\beta \looseid_{\tens} \looseid_{\transid \times \tens}$} (44);
%%%%%%%%
\draw[doubleloose] (13) to node[above]{$\chi_f \looseid_{\tens \times \transid})$} (23);
\draw[doubleloose] (23) to node[above]{$\beta \looseid_{\tens} \looseid_{\tens \times \transid}$} (33);
\draw[doubleloose] (33) to node[above]{$ \looseid_{g} \alpha$} (43);
%%%%%%%%
\draw[doubleloose] (02) to node[above]{$\looseid_{\tens}(\chi_f \times \looseid_f)$} (12);
\draw[doubleloose] (12) to node[above]{$\looseid_{\tens} (\beta \times \beta) \looseid_{\tens \times \transid}$} (22);
\draw[doubleloose] (22) to node[above]{$\chi_g \looseid_{\tens \times \transid}$} (32);
%%%%%%%%
\draw[doubleloose] (01) to node[above]{$\looseid_{\tens} \looseid_{\tens \times \transid}(\beta \times \beta \times \beta)$} (11);
\draw[doubleloose] (11) to node[above]{$\looseid_{\tens} (\chi_g \times \looseid_g)$} (21);
\draw[doubleloose] (21) to node[above]{$\chi_g \looseid_{\tens \times \transid}$} (31);
\draw[doubleloose] (31) to node[above]{$\looseid_g \alpha$} (41);
%%%%%%%%
\draw[doubleloose] (00) to node[above]{$\looseid_{\tens} (\looseid_{\tens \times \transid})(\beta \times \beta \times \beta)$} (10);
\draw[doubleloose] (10) to node[above]{$\alpha \looseid_{g \times g \times g}$} (20);
\draw[doubleloose] (20) to node[above]{$\looseid_{\tens} (\looseid_{\transid} \times \chi_g)$} (30);
\draw[doubleloose] (30) to node[above]{$\chi_g \looseid_{\transid \times \tens}$} (40);
%%%%%%%%
\draw[doubletighteq] (04) to (03);
\draw[doubletighteq] (14) to (13);
\draw[doubletighteq] (24) to (23);
\draw[doubletighteq] (44) to (43);
%%%%%%%%%
\draw[doubletighteq] (03) to (02);
\draw[doubletighteq] (13) to (12);
\draw[doubletighteq] (33) to (32);
\draw[doubletighteq] (43) to (42);
%%%%%%%%%
\draw[doubletighteq] (02) to (01);
\draw[doubletighteq] (22) to (21);
\draw[doubletighteq] (32) to (31);
\draw[doubletighteq] (42) to (41);
%%%%%%%%%
\draw[doubletighteq] (01) to (00);
\draw[doubletighteq] (11) to (10);
\draw[doubletighteq] (41) to (40);
%%%%%%%%%
\node at (.5,3) {$=$};
\node at (1.5,3.5) {$=$};
\node at (3,3.5) {$\DDownarrow ({\horl}^{-1} \verc \horr) ({\horr}^{-1} \verc \horl)$};
\node at (2,2.5) {$\DDownarrow \overline{\Pi^{\beta} \tightid_{\looseid_{\tens \times \transid}}}$};
\node at (2.5,1.5) {$=$};
\node at (.5,.5) {$=$};
\node at (2.5,0.5) {$\DDownarrow \omega^g$};
\node at (1,1.5) {$\DDownarrow \overline{\tightid_{\looseid_{\tens}} \Pi^{\beta} \times  ({\horr}^{-1} \verc \horl)}$};
\node at (3.5,2) {$=$};
\end{tikzpicture}
\end{aligned}
\end{equation*}

For simplicity of notation, we have omited the identities $\looseid_{i_1}$ and $\looseid_{i_2}$.

\begin{equation}\label{eq:bricon}
\begin{aligned}
\begin{tikzpicture}[xscale=3, yscale=1.5]
\node (03) at (0,3) {\small $\tens(f \times f)$};
\node (13) at (1,3) {\small $\tens \tau (f \times f)$};
\node (23) at (2,3) {\small $\tens(f \times f) \tau$};
\node (33) at (3,3) {\small $f \tens \tau$};
\node (43) at (4,3) {\small $g \tens \tau$};
%%%%%%
\node (02) at (0,2) {\small $\tens(f \times f)$};
\node (12) at (1,2) {\small $f \tens$};
\node (32) at (3,2) {\small $f \tens \tau$};
\node (42) at (4,2) {\small $g \tens \tau$};
%%%%%% 
\node (01) at (0,1) {\small $\tens(f \times f)$};
\node (11) at (1,1) {\small $f \tens$};
\node (31) at (3,1) {\small $g \tens $};
\node (41) at (4,1) {\small $g \tens \tau$};
%%%%%%%
\node (00) at (0,0) {\small $\tens(f \times f)$};
\node (10) at (1,0) {\small $\tens(g \times g)$};
\node (30) at (3,0) {\small $g \tens$};
\node (40) at (4,0) {\small $g \tens \tau$};
%%%%%%%
\draw[doubleloose] (03) to node[above]{\small $\sigma \looseid_{f \times f}$} (13);
\draw[double] (13) to (23);
\draw[doubleloose] (23) to node[above]{\small $\chi \looseid_{\tau}$} (33);
\draw[doubleloose] (33) to node[above]{\small $\beta \looseid_{\tens \tau}$} (43);
%%%%
\draw[doubleloose] (02) to node[above]{\small $\chi$} (12);
\draw[doubleloose] (12) to node[above]{\small $\looseid_f \sigma$} (32);
\draw[doubleloose] (32) to node[above]{\small $\beta \looseid_{\tens \tau}$} (42);
%%%%%%
\draw[doubleloose] (01) to node[above]{\small $\chi$} (11);
\draw[doubleloose] (11) to node[above]{\small $\beta \looseid_{\tens}$} (31);
\draw[doubleloose] (31) to node[above]{\small $ \looseid_g \sigma$} (41);
%%%%%%
\draw[doubleloose] (00) to node[above]{\small $\looseid_{\tens} (\beta \times \beta)$} (10);
\draw[doubleloose] (10) to node[above]{\small $\chi $} (30);
\draw[doubleloose] (30) to node[above]{\small $\looseid_g \sigma $} (40);
%%%%%%
\draw[doubletighteq] (03) to (02);
\draw[doubletighteq] (33) to (32);
\draw[doubletighteq] (43) to (42);
%%%%%%
\draw[doubletighteq] (02) to (01);
\draw[doubletighteq] (12) to (11);
\draw[doubletighteq] (42) to (41);
%%%%%%
\draw[doubletighteq] (01) to (00);
\draw[doubletighteq] (31) to (30);
\draw[doubletighteq] (41) to (40);
%%%%%%%%
\node at (1.5,2.5) {\small $\DDownarrow u$};
\node at (3.5,2.5) {\small $=$};
\node at (.5,1.5) {\small $=$};
\node at (2.5,1.5) {\small $\iso$};
\node at (1.5,.5) {\small $\DDownarrow \Pi^{\beta}$};
\node at (3.5,.5) {\small $=$};
\end{tikzpicture}
\end{aligned}
\end{equation}
\[=\]
\begin{equation*}
\begin{aligned}
\begin{tikzpicture}[xscale=3, yscale=1.5]
\node (03) at (0,3) {\small $\tens(f \times f)$};
\node (13) at (1,3) {\small $\tens \tau (f \times f)$};
\node (23) at (2,3) {\small $\tens(f \times f) \tau$};
\node (33) at (3,3) {\small $f \tens \tau$};
\node (43) at (4,3) {\small $g \tens \tau$};
%%%%%%
\node (02) at (0,2) {\small $\tens(f \times f)$};
\node (12) at (1,2) {\small $\tens \tau (f \times f)$};
\node (22) at (2,2) {\small $\tens (f \times f) \tau $};
\node (32) at (3,2) {\small $\tens (g \times g) \tau$};
\node (42) at (4,2) {\small $g \tens \tau$};
%%%%%% 
\node (01) at (0,1) {\small $\tens(f \times f)$};
\node (11) at (1,1) {\small $\tens (g \times g)$};
\node (21) at (2,1) {\small $\tens \tau (g \times g)$};
\node (31) at (3,1) {\small $\tens (g \times g) \tau $};
\node (41) at (4,1) {\small $g \tens \tau$};
%%%%%%%
\node (00) at (0,0) {\small $\tens(f \times f)$};
\node (10) at (1,0) {\small $\tens(g \times g)$};
\node (30) at (3,0) {\small $g \tens$};
\node (40) at (4,0) {\small $g \tens \tau$};
%%%%%%%
\draw[doubleloose] (03) to node[above]{\small $\sigma \looseid_{f \times f}$} (13);
\draw[double] (13) to (23);
\draw[doubleloose] (23) to node[above]{\small $\chi \looseid_{\tau}$} (33);
\draw[doubleloose] (33) to node[above]{\small $\beta \looseid_{\tens \tau}$} (43);
%%%%
\draw[doubleloose] (02) to node[above]{\small $\sigma \looseid_{f \times f}$} (12);
\draw[double] (12) to  (22);
\draw[doubleloose] (22) to node[above]{\small $\looseid_{\tens} (\beta \times \beta) \looseid_{\tau}$} (32);
\draw[doubleloose] (32) to node[above]{\small $\chi \looseid_{\tau}$} (42);
%%%%%%
\draw[doubleloose] (01) to node[above]{\small $\looseid_{\tens} (\beta \times \beta)$} (11);
\draw[doubleloose] (11) to node[above]{\small $\sigma \looseid_{g \times g}$} (21);
\draw[double] (21) to (31);
\draw[doubleloose] (31) to node[above]{\small $ \chi \looseid_{\tau}$} (41);
%%%%%%
\draw[doubleloose] (00) to node[above]{\small $\looseid_{\tens} (\beta \times \beta)$} (10);
\draw[doubleloose] (10) to node[above]{\small $\chi $} (30);
\draw[doubleloose] (30) to node[above]{\small $\looseid_g \sigma $} (40);
%%%%%%
\draw[doubletighteq] (03) to (02);
\draw[doubletighteq] (13) to (12);
\draw[doubletighteq] (43) to (42);
%%%%%%
\draw[doubletighteq] (02) to (01);
\draw[doubletighteq] (32) to (31);
\draw[doubletighteq] (42) to (41);
%%%%%%
\draw[doubletighteq] (01) to (00);
\draw[doubletighteq] (11) to (10);
\draw[doubletighteq] (41) to (40);
%%%%%%%%
\node at (.5,2.5) {\small $=$};
\node at (2.5,2.5) {\small $\DDownarrow \overline{\Pi^{\beta} \looseid_{\tau}}$};
\node at (1.5,1.5) {\small $\iso$};
\node at (3.5,1.5) {\small $=$};
\node at (.5,.5) {\small $=$};
\node at (2.5,.5) {\small $\DDownarrow u$};
\end{tikzpicture}
\end{aligned}
\end{equation*}


As remarked above, we will actually construct a locally cubical bicategory of monoidal objects.
The monoidal 2-cells will be the loose 2-cells therein; we now define the tight 2-morphisms.

\begin{defn}\label{Def:monicon}
  Let $f, g:A \rightarrow B$ be lax monoidal 1-cells in \fB.
  A \textbf{lax monoidal icon} $\beta: f \Rightarrow g$ is a (tight) 2-morphism in \fB\ that is equipped with 3-cells
\begin{equation}
\begin{aligned}
 \begin{tikzpicture}[scale=2]
 \node (tl) at (0,1) {$I_B$};
 \node (tr) at (1,1) {$f I_A$};
 \node (bl) at (0,0) {$I_B$};
 \node (br) at (01,0) {$g I_A$}; 
 \draw[doubleloose] (tl)  to node[above]{$\iota_f$} (tr);
 \draw[doubleeq] (tl) to (bl);
 \draw[doubleloose] (bl) to node[below] {$\iota_g$}(br);
  \draw[doubletight] (tr) to node[right] {$\beta \tightid_I$}(br);
 \node at (0.5,0.5) {\footnotesize $\DDownarrow N^{\beta}$}; 
 \end{tikzpicture}
 \end{aligned}
 \hspace{.5cm}
 \begin{aligned}
  \begin{tikzpicture}[scale=2]
 \node (tl) at (0,1) {$\ten (f \times f)$};
 \node (tr) at (1,1) {$f \ten$};
 \node (bl) at (0,0) {$\ten(g \times g)$};
 \node (br) at (01,0) {$g  \ten$}; 
 \draw[doubleloose] (tl)  to node[above]{$\chi_f$} (tr);
 \draw[doubletight] (tl) to node[left]{$\tightid_{\ten} (\beta \times \beta)$} (bl);
 \draw[doubleloose] (bl) to node[below] {$\chi_g$}(br);
  \draw[doubletight] (tr) to node[right] {$\beta \tightid_{\ten}$}(br);
 \node at (0.5,0.5) {\footnotesize $\DDownarrow \Sigma^{\beta}$}; 
 \end{tikzpicture}
\end{aligned}
\end{equation}

such that the coherence axioms (TA2), (TA3), and (TA4) analogous to definition 2 of~\cite{gg:ldstr-tricat} hold.
An {\bf oplax monoidal icon} between oplax monoidal 1-cells, is a 2-cell with the same data. 
If $g,f$ are strong monoidal, by a \textbf{strong monoidal icon} we mean a lax and a colax one whose structure morphisms are inverse to each other in the loosel direction (modulated by the up-to-isomorphism invertibility of $\chi$ and $\iota$).

A monoidal icon is {\bf braided} or {\bf symmetric} when $f,g$ are braided or symmetric, and in addition the coherence axiom analogous to (BTA1) of~\cite[p143]{mccrudden:bal-coalgb} holds. Note that here the author writes $\rho$ for our braiding 1-cell $\sigma$.
\end{defn}

\begin{defn}
  Let $f,g,f',g' A \rightarrow B$ be monoidal 1-cells, let $\alpha: f \looseRightarrow{} g$, $\beta: f' \looseRightarrow{} g'$ be monoidal 2-cells, and let $\gamma: f \Rightarrow f'$, $\delta: g \Rightarrow g'$ be monoidal icons. A \textbf{monoidal 3-cell} is a 3-cell 
  
   \[
 \begin{tikzpicture}[scale=2]
 \node (tl) at (0,1) {$f$};
 \node (tr) at (1,1) {$g$};
 \node (bl) at (0,0) {$f'$};
 \node (br) at (01,0) {$g'$}; 
 \draw[doubleloose] (tl)  to node[above]{$\alpha$} (tr);
 \draw[doubletight] (tl) to node[left]{$\gamma$} (bl);
 \draw[doubleloose] (bl) to node[below] {$\beta$}(br);
  \draw[doubletight] (tr) to node[right] {$\delta$}(br);
 \node at (0.5,0.5) {\footnotesize $\DDownarrow \Gamma$}; 
 \end{tikzpicture}
 \]
 
 Such that the two equalities below hold.
 
 \begin{equation}
\begin{aligned}
 \begin{tikzpicture}[scale=2]
 \node (tm) at (0,1) {$f  I_A$};
 \node (tr) at (1,1) {$g  I_A$};
 \node (bm) at (0,0) {$f' I_A$};
 \node (br) at (01,0) {$g' I_A$}; 
 \draw[doubleloose] (tm)  to node[above]{$\alpha \looseid_I$} (tr);
 \draw[doubletight] (tm) to node[right, yshift=8] {$\gamma \tightid_I$} (bm);
 \draw[doubleloose] (bm) to node[above] {$\beta \looseid_I$}(br);
  \draw[-implies, double equal sign distance] (tr) to node[right] {$\delta \tightid_I$}(br);
 \node at (0.5,0.5) {\footnotesize $\DDownarrow \Gamma \tightid_{\looseid}$}; 
 \node (tl) at (-1,1) {$I_B$};
 \node (bl) at (-1,0) {$I_B$};
 \draw[doubleloose] (tl)  to node[above]{$\iota_f$} (tm);
 \draw[doubleeq] (tl) to (bl);
 \draw[doubleloose] (bl) to node[above]{$\iota_{f'}$}(bm);
 \node at (-0.5,.5) {\footnotesize $\DDownarrow N^{\gamma}$};
\node (bl1) at (-1,-.7){$I_B$};  
 \node (bm1) at (0,-.7) {$I_B$};
  \node (br1) at (1,-.7) {$g' I_A$}; 
 \draw[doubleloose] (bl1)  to node[above]{$\looseid_{I}$} (bm1);
 \draw[doubleloose] (bm1) to  node[above]{$\iota_{g'}$}(br1);
  \draw[doubleeq] (bl)  to (bl1);
    \draw[doubleeq] (br)  to (br1);
 \node at (0,-0.35) {\footnotesize $\DDownarrow M^{\beta}$}; 
 \end{tikzpicture}
\end{aligned}
 =
 \begin{aligned}
  \begin{tikzpicture}[scale=2]
 \node (ml) at (0,1) {$I_B$};
 \node (mm) at (1,1) {$I_B$};
 \node (bl) at (0,0) {$I_B$};
 \node (bm) at (01,0) {$I_B$}; 
 \draw[doubleloose] (ml)  to node[above]{$ \looseid_{I}$}(mm);
 \draw[doubleeq] (ml) to  (bl);
 \draw[doubleloose] (bl) to  node[above]{$ \looseid_{I}$}(bm);
 \draw[doubleeq] (mm) to (bm);
 \node at (0.5,0.5) {\footnotesize $=$}; 
 \node (tl) at (0,1.7) {$I_B$};
 \node (tm) at (1,1.7) {$f I_A$};
 \node (tr) at (2,1.7) {$g I_A$};
 \node (mr) at (2,1) {$g I_A$};
 \draw[doubleloose] (tl)  to node[above]{$\iota_f$} (tm);
 \draw[doubleloose] (tm) to node[above]{$\alpha \looseid_I$} (tr);
 \draw[doubleloose] (mm) to node[above]{$\iota_{g}$}(mr);
 \node at (1,1.35) {\footnotesize $\DDownarrow M^{\alpha}$};
  \node (br) at (2,0) {$g' I$};
 \draw[doubleloose] (bm)  to node[above]{$\iota_{g'}$} (br);
 \draw[doubletight] (mr) to  node[right]{$\delta \tightid_I$}(br);
 \draw[doubleeq] (tr) to (mr);
  \draw[doubleeq] (tl) to (ml);
 \node at (1.5,.5) {\footnotesize $\DDownarrow N^{\delta}$}; 
 \end{tikzpicture}
 \end{aligned}
\end{equation}

 \begin{equation}
\begin{aligned}
 \begin{tikzpicture}[yscale=2, xscale=2.5]
 \node (tm) at (0,1) {$f\ten$};
 \node (tr) at (1,1) {$g \ten$};
 \node (mm) at (0,0) {$f' \ten$};
 \node (mr) at (01,0) {$g' \ten$}; 
 \draw[doubleloose] (tm)  to node[above]{$\alpha  \looseid_{\ten}$} (tr);
 \draw[doubletight] (tm) to node[right, yshift=8]{$\gamma \tightid_{\ten}$} (mm);
 \draw[doubleloose] (mm) to node[above, xshift=1pt, yshift=-1pt] {$\beta \looseid_{\ten}$}(mr);
  \draw[doubletight] (tr) to node[right] {$\delta \tightid_{\ten}$}(mr);
 \node at (0.5,0.5) {\footnotesize $\DDownarrow \Gamma \tightid$}; 
 \node (tl) at (-1,1) {$\ten  (f\times f)$};
 \node (ml) at (-1,0) {$\ten  (f'\times f')$};
 \draw[doubleloose] (tl)  to node[above]{$\chi^f$} (tm);
 \draw[doubletight] (tl) to node[left]{$\tightid_{\ten} (\gamma \times \gamma)$} (ml);
 \draw[doubleloose] (ml) to node[above]{$\chi^{f'}$}(mm);
 \node at (-0.5,0.5) {\footnotesize $\DDownarrow \Sigma^{\gamma}$};
 \node (bl) at (-1,-.7) {$\ten (f'\times f')$};
  \node (bm) at (0,-.7) {$\ten (g'\times g')$};
  \node (br) at (1,-.7) {$g' \ten$};
  \draw[doubleeq] (ml) to (bl);
 \draw[doubleloose] (bl)  to node[above]{$\looseid_{\ten} (\beta \times \beta)$} (bm);
 \draw[doubleloose] (bm) to  node[above]{$\chi^{g'}$}(br);
   \draw[doubleeq] (mr) to (br);
 \node at (0,-0.35) {\footnotesize $\DDownarrow \Pi^{\beta}$}; 
 \end{tikzpicture}
\end{aligned}
 =
 \begin{aligned}
  \begin{tikzpicture}[yscale=2, xscale=2.5]
 \node (ml) at (0,1) {$\ten (f\times f)$};
 \node (mm) at (1,1) {$\ten (g\times g)$};
 \node (bl) at (0,0) {$\ten (f'\times f')$};
 \node (bm) at (01,0) {$\ten (g'\times g')$}; 
 \draw[doubleloose] (ml)  to node[above]{$\looseid_{\ten} (\alpha \times \alpha)$} (mm);
 \draw[doubletight] (ml) to node[left]{$\tightid_{\ten} (\gamma \times \gamma)$}  (bl);
 \draw[doubleloose] (bl) to node [below] {$\looseid_{\ten} (\beta \times \beta)$} (bm);
  \draw[doubletight] (mm) to node[above] {$\tightid_{\ten} (\delta \times \delta)$} (bm);
 \node at (0.5,0.5) {\footnotesize $\DDownarrow \tightid (\Gamma \times \Gamma)$}; 
 \node (tl) at (0,1.7) {$ \ten (f \times f$)};
 \node (tm) at (1,1.7) {$f \ten$};
 \node (tr) at (2,1.7) {$g \ten$};
   \node (mr) at (2,1) {$g \ten$};
   \node(br) at (2,0) {$g' \ten$};
 \draw[doubleloose] (tl)  to node[above]{$\chi^f$} (tm);
 \draw[doubleloose] (tm) to node[above]{$\alpha \looseid_{\ten}$} (tr);
 \draw[doubletick] (mm) to node[above]{$\chi^{g}$}(mr);
 \node at (1,1.35) {\footnotesize $\DDownarrow \Pi^{\alpha}$};
 \draw[doubleloose] (bm)  to node[below]{$\chi^{g'}$} (br);
 \draw[doubletight] (mr) to  node[right]{$\delta \tightid_{\ten}$}(br);
 \draw[doubleeq] (tr) to (mr);
  \draw[doubleeq] (tl) to (ml);
 \node at (1.5,.5) {\footnotesize $\DDownarrow \Sigma^{\delta}$}; 
 \end{tikzpicture}
 \end{aligned}
\end{equation}

Let $f,g,f',g' A \rightarrow B$ be braided, sylleptic or symmetric monoidal 1-cells, let $\alpha: f \looseRightarrow{} g$, $\beta: f' \looseRightarrow{} g'$ be braided, sylleptic, or symmetric monoidal 2-cells, and let $\gamma: f \Rightarrow f'$, $\delta: g \Rightarrow g'$ be braided, sylleptic, or symmetric monoidal icons. A \textbf{braided, sylleptic, or symmetric monoidal 3-cell} $\Gamma$ as depicted above, is simply a monoidal 3-cell. 
\end{defn}

\begin{prop}\label{prop:dc}
Let $A,B$ be monoidal objects in a locally cubical bicategory. The lax monoidal 1-cells, monoidal 2-cells, monoidal icons, and monoidal 3-cells in $\fB (A,B)$ form a double category $\cM on\cB (A,B)$. The same statement holds for colax and strong monoidal objects and cells and if $A,B$ are braided, sylleptic or symmetric, it holds for braided, sylleptic, and symmetric cells, respectively.
\end{prop}

\begin{proof}
The first step of this proof is to show that lax monoidal 1-cells and icons form a category. 
For each monoidal 1-cell $f:A \rightarrow B$, the identity tight 2-cell $\tightid_f$ is a monoidal icon with the 3-cells $N^{\tightid_f} := \tightid_{\iota_f}$ and $\Sigma^{\tightid_f} := \tightid_{\chi_f}$. This is well-defined, because the functor "$\comp$" preserves identities. The coherence equations are trivially satisfied.  For each two monoidal 1-cells $f,g$ and monoidal icons $\alpha, \beta: f \Rightarrow g$, the composite tight 2-cell $\beta \circ \alpha$ can be equiped with the monoidal structure given by the composites $N^{\beta \circ \alpha} := N^{\beta} \circ N^{\alpha}$ and $\Sigma^{\beta\circ \alpha} : = \Sigma^{\beta} \circ \Sigma^{\alpha}$.  We have a strict interchange law between $\verc$ and $\comp$, induced by functoriality of $\comp$, so these 3-cells are well-defined. The coherence conditions (TA1)-(TA4) hold by componentwise application of the coherence equalities for $N^{\beta \verc \alpha}$ and $\Sigma^{\beta \verc \alpha}$. The same argument holds for oplax and strong monoidal 1-cells and icons.
When $f$ and $g$ are braided, the same data satisfies the coherence equation for braided monoidal icons. Hence, braided monoidal 1-cells and braided monoidal icons form a category, and the same is true for sylleptic or symmetric monoidal 1-cells and icons.

In addition, we need to show that lax monoidal 2-cells and monoidal 3-cells form a category. For every lax monoidal 2-cell $\alpha: f \looseRightarrow{} g$, the identity 3-cell $\tightid_{\alpha}$ in $\cB$  is monoidal, since the required two equations are trivially satisfied.
For any two monoidal 3-cells $L: \alpha \Rightthreecell \beta$, $K:\beta \Rightthreecell \gamma$, the composition $K \circ L$ in $\cB$ is a monoidal 3-cell. The equations for monoidal 3-cells hold by sequential application of the respective equations for $L$ and $K$. The same is true for oplax monoidal 2-cells, and hence for strong monoidal 2-cells. As braided, sylleptic, and symmetric monoidal 3-cells are not required to satisfy additional data, it follows that braided, sylleptic, or symmetric monoidal 2-cells and 3-cells form a category.

Now we describe the loose structure.
Let $f$ be a lax monoidal 1-cell. The loose identity 2-cell $\looseid_f$ is a monoidal 2-cell with monoidal structure given below. 

\begin{equation}
M^{\looseid_f}:=
\begin{aligned}
 \begin{tikzpicture}[yscale=1.5, xscale=3]
 \node (tl) at (0,1) {$I_B$};
\node (tr) at (1,1) {$f   I_A$};
 \node (tm) at (.5,1) {$f  I_A$};
 \node (bl) at (0,0) {$I_B$};
 \node (bm) at (.5,0) {$I_B$};
 \node (br) at (1,0) {$f I_A$}; 
 \draw[doubleloose] (tl)  to node[above]{$\iota_f$} (tm);
  \draw[doubleloose] (tm)  to node[above]{$\looseid_f \looseid_I$} (tr);
 \draw[doubleeq] (tl) to (bl);
  \draw[doubleloose] (bl) to node[below] {$\looseid_I$}(bm);
 \draw[doubleloose] (bm) to node[below] {$\iota_f$}(br);
  \draw[doubletight] (tr) to node[right] {$\tightid_f \comp \tightid_{I}$}(br);
 \node at (0.5,0.5) {\footnotesize ${\horl_{\iota_f}}^{-1}\verc \horr_{\iota_f} \DDownarrow \iso $}; 
 \end{tikzpicture}
 \end{aligned}
 \hspace{.5cm}
 \Pi^{\looseid_f}:=
 \begin{aligned}
  \begin{tikzpicture}[yscale=1.5, xscale=5]
 \node (tl) at (0,1) {$\ten  (f \times f)$};
 \node (tr) at (1,1) {$f  \ten$};
 \node (bl) at (0,0) {$\ten  (f \times f)$};
 \node (br) at (01,0) {$f \ten$}; 
 \node(tm) at (.5,1) {$f \ten$};
 \node (bm) at (.5,0) {$\ten (f\times f)$};
 \draw[doubleloose] (tl)  to node[above]{$\chi_f $} (tm);
  \draw[doubleloose] (tm)  to node[above]{$\looseid_{f \ten}$} (tr);
 \draw[doubletight] (tl) to node[left]{$\tightid_{\ten} (\tightid_f \times \tightid_f)$} (bl);
  \draw[doubleloose] (bl) to node[below] {$\looseid_{\ten(f \times f)}$}(bm);
 \draw[doubleloose] (bm) to node[below] {$\chi_f$}(br);
  \draw[doubletight] (tr) to node[right] {$\alpha \tightid_{\ten}$}(br);
 \node at (0.5,0.5) {\footnotesize ${\horl}^{-1}_{\chi_f} \verc \horr_{\chi_f}\DDownarrow \iso$}; 
 \end{tikzpicture}
\end{aligned}
\end{equation}

For every monoidal icon $\gamma$, the loose identity 3-cell $\looseid_{\gamma}$ is a monoidal 3-cell. The conditions for monoidal 3-cells follow from the naturality conditions of $\horl$ and $\horr$. 

Let $\alpha:f \looseRightarrow{} g$ and $\beta: g \looseRightarrow{} h$ be two monoidal 2-cells. Their composition $\alpha \horc \beta$ is monoidal with the following structure 3-cells.

\begin{equation}
M^{\alpha \horc \beta} := 
\begin{aligned}
 \begin{tikzpicture}[yscale=1.5, xscale=3]
 \node (tl) at (0,1) {$I_B$};
\node (tr) at (1,1) {$g   I_A$};
 \node (tm) at (.5,1) {$f  I_A$};
 \node (bl) at (0,0) {$I_B$};
 \node (bm) at (0.5,0) {$I_B$};
 \node (br) at (1,0) {$g I_A$}; 
 \node (trr) at (1.5,1) {$h I_A$};
 \node (brr) at (1.5,0) {$h I_A$};
 \node (bbr) at (1.5,-1) {$hI_A$};
  \node (bbm1) at (.5,-1) {$I_B$};
 \node (bbm) at (1,-1) {$I_B$};
 \node(bbl) at (0,-1) {$I_B$};
    \draw[doubleloose] (tm) to[in=120, out=60] node[above]{$(\alpha \horc \beta)\looseid_{I}$} (trr);
 \draw[doubletight] (brr) to node[right] {} (bbr);
 \draw[doubleeq] (bl) to (bbl);
  \draw[doubleloose] (bbl) to node [above]{$\looseid_{I}$} (bbm1);
    \draw[doubleloose] (bbm1) to node [above]{$\looseid_{I}$} (bbm);
 \draw[doubleloose] (bbm) to node [above]{$\iota_{h}$} (bbr);
 \draw[doubleloose] (tr) to node[above]{$\beta \looseid_I$} (trr);
  \draw[doubleloose] (br) to node[above]{$\beta \looseid_I$}(brr);
  \draw[doubleeq] (trr) to (brr);
 \draw[doubleloose] (tl)  to node[above]{$\iota_f$} (tm);
  \draw[doubleloose] (tm)  to node[above]{$\alpha \looseid_I$} (tr);
 \draw[doubleeq] (tl) to (bl);
  \draw[doubleloose] (bl) to node[below] {$\looseid_I$}(bm);
 \draw[doubleloose] (bm) to node[below] {$\iota_g$}(br);
 \draw[doubleloose] (bbl) to[in=220, out=-60] node[below]{$\looseid_I$} (bbm);
  \draw[doubleeq] (tr) to (br);
   \draw[doubleeq] (bm) to (bbm1);
 \node at (0.5,0.5) {\footnotesize $M^{\alpha} \DDownarrow  $}; 
  \node at (1,-.5) {\footnotesize $M^{\beta} \DDownarrow $}; 
 \node at (1.25,.5) {\footnotesize $=$}; 
 \node at (1,1.25) {$\iso$};
 \node at (0.5,-1.25) {$\iso$};
 \end{tikzpicture}
 \end{aligned}
\end{equation}
\begin{equation}
 \Pi^{\alpha \horc \beta}:=
 \begin{aligned}
  \begin{tikzpicture}[yscale=1.5, xscale=5]
 \node (tl) at (0,1) {$\ten  (f \times f)$};
 \node (tr) at (1,1) {$g \ten$};
 \node (bl) at (0,0) {$\ten  (f \times f)$};
 \node (br) at (01,0) {$g \ten$}; 
 \node(tm) at (.5,1) {$f \ten$};
 \node (bm) at (.5,0) {$\ten (g\times g)$};
 \node (trr) at (1.5,1) {$h \ten$};
  \node (brr) at (1.5,0) {$h \ten$};
  \node (bbl) at (0,-1) {$\ten (f \times f)$};
  \node (bbm) at (.5,-1) {$\ten (g \times g)$}; 
  \node (bbr) at (1,-1) {$\ten (h \times h)$};
  \node (bbrr) at (1.5,-1) {$h \ten $};
 \draw[doubleloose] (tl)  to node[above]{$\chi_f $} (tm);
  \draw[doubleloose] (tm)  to node[above]{$\alpha \looseid_{\ten}$} (tr);
 \draw[doubleeq] (tl) to (bl);
  \draw[doubleloose] (bl) to node[below] {$\looseid_{\ten} (\alpha \times \alpha)$}(bm);
 \draw[doubleloose] (bm) to node[below] {$\chi_g$}(br);
  \draw[doubleeq] (tr) to (br); 
 \draw[doubleeq] (trr) to (brr);
 \draw[doubleloose] (tr) to node[above]{$\beta \looseid_{\ten}$} (trr);
 \draw[doubleloose] (br) to node[above]{$\beta \looseid_{\ten}$} (brr);
 \draw[doubleloose] (bbr) to node[above]{$\chi_h$} (bbrr);
 \draw[doubleeq] (bl) to (bbl);
 \draw[doubleeq] (bm) to (bbm);
 \draw[doubleeq] (brr) to (bbrr);
 \draw[doubleloose] (bbl) to node[above]{$\looseid_{\ten} (\alpha \times \alpha)$} (bbm);
  \draw[doubleloose] (bbm) to node[above]{$\looseid_{\ten} (\beta \times \beta)$} (bbr);
   \draw[doubleloose] (tm) to[in=120, out=60] node[above]{$(\alpha \horc \beta)\looseid_{\ten}$} (trr);
   \draw[doubleloose] (bbl) to[in=220, out=-60] node[below]{$\looseid_{\ten} \comp (\alpha \horc \beta)\times (\alpha \horc \beta)$} (bbr);
    \node at (0.5,0.5) {\footnotesize $\DDownarrow  \Pi^{\alpha}$};
  \node at (1.25,0.5) {\footnotesize $=$};
  \node at (0.25,-.5) {\footnotesize $=$};
  \node at (1,-.5) {\footnotesize $\DDownarrow  \Pi^{\beta}$};
  \node at (1,1.2) {$\iso$};
 \node at (.5,-1.2) {$\iso$};
 \end{tikzpicture}
\end{aligned}
\end{equation}


The coherence equations are satisfied by sequential application of the respective equation for $\alpha$ and $\beta$, applications of the exchange law between loose and tight composition, together with simple manipulations of coherence cells.

Let $\Gamma$ and $\Delta$ be 3 monoidal 3-cells. Their composite $\Gamma \horc \Delta$ is again monoidal. Again, the conditions for monoidal 3-cells follow directly from the conditions on the monoidal 3-cells $\Gamma$ and $\Delta$, applications of the exchange law between loose and tight composition, and simple manipulations of coherence cells. 

The unitality and associativity constraints $\hora$, $\horl$, and $\horr$ are monoidal 3-cells. The conditions for monoidal 3-cells amount to 3-cells pasted together with coherence cells being equal to themselves, which follows from coherence of the functor $\horc$. It follows that lax monoida 1-cells, 2-cells, icons and 3-cells in the hom-categories $\cM on\cB(A,B)$ form a double category.
The oplax and strong case hold by a similar proof with arrows in the opposite directions.

Let $f$ be a braided monoidal 1-cell. The loose identity $\looseid_f$ is a braided monoidal 2-cell, as the coherence equation merely states that the 3-cell $u$ pasted with coherence and naturality 3-cells equals itself. Let $\alpha, \beta$ be braided monoidal 2-cells, the loose composition $\alpha \horc \beta$ is braided monoidal. It is easy to verify that~\ref{eq:bricon} holds by applying the exchange law between loose and tight composition, manipulation of coherence cells, and sequential application of the respective equations for $\alpha$ and $\beta$.  As braided monoidal 3-cells are simply monoidal 3-cells, it follows from the arguments above that braided, sylleptic and symmetric monoidal 1-cells, 2-cells, icons, and 3-cells form a double category.
\end{proof}

\begin{thm}\label{thm:lcbc}
  Monoidal objects, lax monoidal 1-cells, lax monoidal 2-cells, lax monoidal icons, and lax monoidal 3-cells in a locally cubical bicategory \fB\ form a locally cubical bicategory $\cM on\cB$. The same statement holds for colax and strong monoidal objects and cells, and for braided, sylleptic, and symmetric objects and cells.
\end{thm}

\begin{proof}
We have established that the respective homsets $\cM onB(A,B)$, $\cB r \cM onB(A,B)$, $\cS yl \cM onB(A,B)$, and $\cS ym \cM onB(A,B)$ form double categories in Proposition \ref{prop:dc}. It is left to show that monoidal objects and cells have the enriched structure of a locally cubical bicategory, and that the same is true for braided, sylleptic, and symmetric monoidal cells.

We need to check that the unit $I^{\comp}_A$ is a well-defined functor from the trivial double category $1$ to the respective hom-categories of lax, oplax and strong monoidal cells, as well as braided, sylleptic and symmetric ones. 
First of all, {\it we need to assume that the image of $I^{\comp}$ on the object and the loose 1-cell of the trivial double category is a lax monoidal 1-cell and lax monoidal loose 2-cell, respectively, and that the natural transformations $I^{\comp}_U$ and $I^{\comp}_{\odot}$ are lax monoidal 3-cells}. The image of $I^{\comp}$ on the other cells is lax, since by functoriality of $I^{\comp}_0$ and $I^{\comp}_1$, they are mapped to the identities. This also holds for oplax and strong cells, as well as braided, sylleptic, and symmetric cells under analogous assumptions. 

Next, we need to show that monoidal structure is preserved by the composition along a 0-cell boundary.
For any two lax monoidal 1-cells $f:A \rightarrow B$, $g:B \rightarrow C$, the composite $g \comp f$ is monoidal with 
\begin{align}
\chi_{g \comp f} &:= &(\chi_g \comp \looseid_{fI \times f}) \horc (\looseid_g \comp \chi_f \comp \looseid_{I \times \tightid}) \\
\iota_{g \comp f} & := & (\looseid_{\tens} \comp (\iota_f \times \looseid_{gf})) \horc (\looseid_{\tens} \comp (\iota_g \times \looseid_{gf}))
\end{align}

The structure 3-cell $\gamma$ is defined as

\begin{equation}
\gamma^{g \comp f} := 
\begin{aligned}
 \begin{tikzpicture}[yscale=1.5, xscale=5]
 \node (t0) at (0,2) {\small $\tens(I_C \times gf)i_2$};
 \node (t1) at (.5,2) {\small $\tens(gI_B \times gf)i_2$};
\node (t2) at (1,2) {\small $g \tens (I_B \times f)i_2$};
 \node (t3) at (1.5,2) {\small $g \tens (fI_A \times f)i_2$};
  \node (t4) at (2,2) {\small $gf \tens (I_A \times \transid)i_2$};
 \node (t5) at (2.5,2) {\small $gf$};
  \node (m0) at (0,1) {\small $\tens(I_C \times g)i_2f$};
 \node (m1) at (.5,1) {\small $\tens(gI_B \times g)i_2f$};
\node (m2) at (1,1) {\small $g \tens (I_B \times \transid)i_2f$};
 \node (m5) at (2.5,1) {\small $gf$};
  \node (b0) at (0,0) {\small $\tens(I_C \times \transid)i_2 gf$};
 \node (b5) at (2.5,0) {\small $gf$};
 %%%%%%%%%%%%%%%%
  \draw[doubleloose] (t0) to[in=120, out=60] node[above]{$\looseid_{\tens} (\iota_{gf} \times \looseid_{gf})\looseid_{i_2} \horc \chi_{gf}\looseid_{i_2}$} (t4);
  %%%%%%%%%%%%%%%%
 \draw[doubleloose] (t0)  to node[above]{\small $\looseid_{\tens}(\iota_g \times \looseid_{gf})\looseid_{i_2}$} (t1);
  \draw[doubleloose] (t1)  to node[above]{\small $\chi_g\looseid_{I_A \times f}\looseid_{i_2}$} (t2);
\draw[doubleloose] (t2) to node[above]{\small $\looseid_{\tens g}(\iota_f \times \looseid_{f})\looseid_{i_2}$} (t3);
  \draw[doubleloose] (t3) to node[above]{\small $\looseid_g \chi_f \looseid_{I_A \times \transid}\looseid_{i_2}$}(t4);
  \draw[doubleloose] (t4) to node[above]{\small $\looseid_{gf}l_I$}(t5);
  %%%%%%%%%%%%%%%%%%
  \draw[doubleloose] (m0)  to node[above]{\small $\looseid_{\tens}(\iota_g \times \looseid_{g})\looseid_f$} (m1);
  \draw[doubleloose] (m1)  to node[above]{\small $\chi_g\looseid_{i_2 f}$} (m2);
   \draw[doubleloose] (m2) to node[below]{\small $ \looseid_g l \looseid_f$}(m5); 
   %%%%%%%%%%%%%%%%%
    \draw[doubleloose] (b0) to node[above]{\small $ l \looseid_g \looseid_f$}(b5); 
       \draw[doubleloose] (b0) to[in=220, out=-60] node[above]{\small $l \looseid_{gf}$}(b5); 
    %%%%%%%%
  \draw[doubleeq] (t0) to (m0);
    \draw[doubleeq] (t2) to (m2);
  \draw[doubleeq] (t5) to (m5);
  \draw[doubleeq] (m0) to (b0);
    \draw[doubleeq] (m5) to (b5);
    \node at (.5,1.5) {\footnotesize $=$}; 
   \node at (1.75,1.5) {\footnotesize $\overline{ \tightid_g \gamma^f}$}; 
   \node at (1.25,.5) {\footnotesize $\overline{  \gamma^g \tightid_{\looseid}}$}; 
      \node at (1,2.35) {\footnotesize $\overline{ \iso}$}; 
  \node at (1.25,-.35) {\footnotesize $\overline{ \iso}$}; 
 \end{tikzpicture}
 \end{aligned}
\end{equation}

The 3-cells $\delta^{g \comp f}$ and $\omega^{g \comp f}$ are defined similarly, so is $u^{g \comp f}$ when $g, f$ are braided monoidal 1-cells. 

For oplax monoidal 1-cells, the the above gives the monoidal structure if the arrows are reversed.



Let $f,h: A \rightarrow B $ and $g,i: B \rightarrow C$ be lax monoidal 1-cells and let $\alpha: f \rightarrow h$, $\beta: g \rightarrow i$ be lax monoidal 2-cells, the composite $\beta \comp \alpha$ is lax monoidal with the following structure 3-cells

\begin{equation}
M^{\beta \comp \alpha} := 
\begin{aligned}
 \begin{tikzpicture}[yscale=1.5, xscale=4]
 \node (t0) at (0,1) {$I_C$};
\node (t2) at (1,1) {$g f  I_A$};
 \node (t4) at (2,1) {$i h I_A$};
 \node (m0) at (0,0) {$I_C$};
 \node (m1) at (.5,0) {$g I_B$}; 
\node (m2) at (1,0) {$h I_B$};
\node (m3) at (1.5,0) {$h f I_A$};
\node (m4) at (2,0) {$h k I_A$};
 \node (b0) at (0,-1) {$I_C$};
 \node (b1) at (.5,-1) {$h I_B$}; 
\node (b2) at (1,-1) {$h I_B$};
\node (b3) at (1.5,-1) {$h k I_A$};
\node (b4) at (2,-1) {$h k I_A$};
\node (bb0) at (0,-2) {$I_C$};
 \node(bb2) at (1,-2) {$I_C$};
   \node(bb4) at (2,-2) {$hk I_A$};
 \draw[doubleloose] (t0)  to node[above]{$\iota_{g \comp f}$} (t2);
  \draw[doubleloose] (t2)  to node[above]{$\beta \comp \alpha$} (t4);
\draw[doubleloose] (m0) to node[above]{$\iota_g \looseid_{fI \times f}I$} (m1);
  \draw[doubleloose] (m1) to node[above]{$\beta \looseid_{I}$}(m2);
  \draw[doubleloose] (m2) to node[above]{$\looseid_h \iota_f \looseid_{I \times \transid}$}(m3);
  \draw[doubleloose] (m3) to node[above]{$\looseid_h \alpha \looseid_{I}$}(m4);
  %%%%%%%%%%%%
  \draw[doubleloose] (b0) to node[above]{$\looseid$} (b1);
  \draw[doubleloose] (b1) to node[above]{$\iota_h$} (b2);
  \draw[doubleloose] (b2) to node[above]{$\looseid$}(b3);
  \draw[doubleloose] (b3) to node[above]{$\looseid_h \iota_k$}(b4);
 %%%%%%%%%
  \draw[doubleloose] (bb0)  to node[above]{$\looseid_{I}$} (bb2);
  \draw[doubleloose] (bb2)  to node[above]{$\iota_{hk}$} (bb4); 
   %%%%%%%%%% 
  \draw[doubleeq] (t0) to (m0);  
   \draw[doubleeq] (m0) to (b0);
      \draw[doubleeq] (b0) to (bb0);
    \draw[doubleeq] (t4) to (m4);  
   \draw[doubleeq] (m4) to (b4);
      \draw[doubleeq] (b4) to (bb4);
   \draw[doubleeq] (m2) to (b2);
 \node at (1,-1.5) {\footnotesize $\iso$}; 
  \node at (.5,-.5) {\footnotesize $\DDownarrow \Pi^{\beta} \tightid_{f \times f}$}; 
    \node at (1.5,-.5) {\footnotesize $\DDownarrow \tightid_{h} \Pi^{\alpha} $}; 
   \node at (1,.5) {\footnotesize $\iso$}; 
 \end{tikzpicture}
 \end{aligned}
\end{equation}


\begin{equation}
\Pi^{\beta \comp \alpha} := 
\begin{aligned}
  \begin{tikzpicture}[yscale=1.5, xscale=5]
 \node (t0) at (0,1) {$\tens (gf I_A \times gf)$};
\node (t2) at (1,1) {$gf \tens (I_A \times \transid)$};
 \node (t4) at (2,1) {$hk \tens (I_A \times \transid)$};
 \node (m0) at (0,0) {$\tens (gf I_A \times gf)$};
 \node (m1) at (.5,0) {$g \tens (f I_A \times f)$}; 
\node (m2) at (1,0) {$h \tens (f I_A \times f)$};
\node (m3) at (1.5,0) {$hf \tens ( I_A \times \transid)$};
\node (m4) at (2,0) {$hk \tens ( I_A \times \transid)$};
 \node (b0) at (0,-1) {$\tens (gf I_A \times gf)$};
 \node (b1) at (.5,-1) {$\tens (hf I_A \times hf)$}; 
\node (b2) at (1,-1) {$h \tens (f I_A \times f)$};
\node (b3) at (1.5,-1) {$h \tens (k I_A \times k)$};
\node (b4) at (2,-1) {$hk \tens ( I_A \times \transid)$};
\node (bb0) at (0,-2) {$\tens (gf I_A \times gf)$};
 \node(bb2) at (1,-2) {$\tens (hk I_A \times hk)$};
   \node(bb4) at (2,-2) {$hk \tens (I_A \times \transid)$};
 \draw[doubleloose] (t0)  to node[above]{$\chi_{gf} \looseid_{I \times \transid}$} (t2);
  \draw[doubleloose] (t2)  to node[above]{$\beta \alpha \looseid_{\tens(I\times \transid)}$} (t4);
\draw[doubleloose] (m0) to node[above]{$\chi_g \looseid_{fI \times f}$} (m1);
  \draw[doubleloose] (m1) to node[above]{$\beta \looseid_{\tens (fI \times f)}$}(m2);
  \draw[doubleloose] (m2) to node[above]{$\looseid_{h \tens} \chi_f \looseid_{I \times \transid}$}(m3);
  \draw[doubleloose] (m3) to node[above]{$\looseid_h \alpha \looseid_{\tens(I \times \transid)}$}(m4);
  %%%%%%%%%%%%
  \draw[doubleloose] (b0) to node[above]{$\looseid_{\tens} (\beta \times \beta) \looseid_{fI \times f}$} (b1);
  \draw[doubleloose] (b1) to node[above]{$\chi_h \looseid_{fI \times f}$} (b2);
  \draw[doubleloose] (b2) to node[above]{$\looseid_{h \tens} (\alpha \times \alpha) \looseid_{I \times \transid}$}(b3);
  \draw[doubleloose] (b3) to node[above]{$\looseid_h \chi_k \looseid_{I \times \transid}$}(b4);
 %%%%%%%%%
  \draw[doubleloose] (bb0)  to node[above]{$\looseid_{\tens} (\beta \alpha \times \beta \alpha) \looseid_{I \times \transid}$} (bb2);
  \draw[doubleloose] (bb2)  to node[above]{$\chi_{hk} \looseid_{I \times \transid}$} (bb4); 
   %%%%%%%%%% 
  \draw[doubleeq] (t0) to (m0);  
   \draw[doubleeq] (m0) to (b0);
      \draw[doubleeq] (b0) to (bb0);
    \draw[doubleeq] (t4) to (m4);  
   \draw[doubleeq] (m4) to (b4);
      \draw[doubleeq] (b4) to (bb4);
   \draw[doubleeq] (m2) to (b2);
 \node at (1,-1.5) {\footnotesize $\iso$}; 
  \node at (.5,-.5) {\footnotesize $\DDownarrow \Pi^{\beta} \tightid_{f \times f}$}; 
    \node at (1.5,-.5) {\footnotesize $\DDownarrow \tightid_{h} \Pi^{\alpha} $}; 
   \node at (1,.5) {\footnotesize $\iso$}; 
 \end{tikzpicture}
 \end{aligned}
\end{equation}

Let $f,h: A \rightarrow B $ and $g,i: B \rightarrow C$ be lax monoidal 1-cells and let $\alpha: f \rightarrow h$, $\beta: g \rightarrow i$ be lax monoidal icons, the composite $\beta \comp \alpha$ is lax monoidal with the following structure 3-cells.

\begin{equation}
N^{\beta \comp \alpha} := 
\begin{aligned}
 \begin{tikzpicture}[yscale=1.5, xscale=6]
 \node (tl) at (0,1) {$I_C$};
 \node (tm) at (.5,1) {$g I_B$};
\node (tr) at (1,1) {$g f  I_A$};
 \node (bl) at (0,0) {$I_C$};
\node (bm) at (.5,0) {$i I_B$};
\node (br) at (1,0) {$i h I_A$};
 \draw[doubleloose] (tl)  to node[above]{$\iota_g$} (tm);
  \draw[doubleloose] (tm)  to node[above]{$\looseid_g \iota_f$} (tr);
\draw[darrow] (tr) to node[right]{$\beta \alpha  \tightid_I$} (br);
   \draw[doubleloose] (bl) to node[above]{$\iota_i$}(bm);
  \draw[doubleloose] (bm) to node[above]{$\looseid_i \iota_h$}(br);
  \draw[doubletight] (tm) to node[left] {$\beta \tightid_I$} (bm);
    \draw[doubleeq] (tl) to (bl);
  \node at (.75,0.5) {$\looseid_{\beta} \comp N^{\alpha} \DDownarrow $}; 
    \node at (.25,.5) {$ N^{\beta} \DDownarrow $}; 
 \end{tikzpicture}
 \end{aligned}
\end{equation}


\begin{equation}
\Sigma^{\beta \comp \alpha} := 
\begin{aligned}
 \begin{tikzpicture}[yscale=1.5, xscale=8]
 \node (tl) at (0,1) {$\ten (g \times g) (f \times f)$};
 \node (tm) at (.5,1) {$g \ten (f \times f)$};
\node (tr) at (1,1) {$g f  \ten$};
  \node(bl) at (0,-1) {$\ten (g \times g) (f \times f)$};
  \node(bm) at (.5,-1) {$\ten (i \times i) (h \times h)$};
  \node (br) at (1,-1) {$i h \ten$};
 \draw[doubleloose] (tl)  to node[above]{$\chi_g \looseid_{f \times f}$} (tm);
  \draw[doubleloose] (tm)  to node[above]{$\looseid_g \chi_f$} (tr);
\draw[doubletight] (tr) to node[above]{$\beta \alpha  \tightid_{\ten}$} (br);
  \draw[doubletight] (tm) to node[above]{$\beta  \tightid_{\ten} (\alpha \times \alpha)$}(bm);
   \draw[doubletight] (tl) to node[above]{$\tightid_{\ten}(\beta \times \beta)(\alpha \times \alpha)$}(bl);
    \draw[doubleloose] (bl) to node[above]{$\chi_i \looseid_{h \times h}$}(bm);
  \draw[doubleloose] (bm) to node[above]{$\looseid_i \chi_h$}(br);
  \node at (.25,0.5) {$ \Sigma^{\beta} \looseid_{\alpha \times \alpha} \DDownarrow $}; 
    \node at (.75,.5) {$ \looseid_{\beta} \comp\Sigma^{\alpha} \DDownarrow $}; 
 \end{tikzpicture}
 \end{aligned}
\end{equation}


Let $\Gamma$ and $\Delta$ be two composable monoidal 3-cells. It is easy to see that the composition $\Gamma \comp \Delta$ satisfies the two equations for monoidal 3-cells. This is a matter of applying the equations for $\Gamma$ and $\Delta$ sequentially.

In all coherence equations between 3-cells for the monoidal structure of transversal compositionof monoidal above, each 3-cell consists of a conponent for the first composite  composed with the identity on the second composite, and a component for the second composite composed with the identity on the product of the first composite with itself. This means that the coherence equations for $g \comp f$  can be established by componentwise application of the equations for $g$ and $f$. Some 3-cells also contain coherence cells, but these equally break up in a part concerning the first, and a part concerning the second component. Manipulation of these coherence cells results in the required equalities. Note that rewriting of the 1-cells and composites of loose 2-cells is necessary in several of the steps. A similar argument holds for coherence equations for braided, syllectic and sylleptic cells.

\end{proof}

% Local Variables:
% TeX-master: "smbicat"
% End:
