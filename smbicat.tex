\documentclass{amsart}
\usepackage{tikz}
\usepackage{tikz-cd}
\usepackage[status=draft]{fixme}
\usepackage{standalone}
\usepackage{amssymb,amsmath,stmaryrd,txfonts,mathrsfs,amsthm}
\usepackage[all,2cell]{xy}
\usepackage[neveradjust]{paralist}
\usepackage{hyperref}
\usepackage{mathtools}
\usepackage{tikz}
\usetikzlibrary{trees}
\usetikzlibrary{topaths}
\usetikzlibrary{decorations}
\usetikzlibrary{decorations.pathreplacing}
\usetikzlibrary{decorations.pathmorphing}
\usetikzlibrary{decorations.markings}
\usetikzlibrary{matrix,backgrounds,folding}
\usetikzlibrary{chains,scopes,positioning,fit}
\usetikzlibrary{arrows,shadows}
\usetikzlibrary{calc} 
\usetikzlibrary{chains}
\usetikzlibrary{shapes,shapes.geometric,shapes.misc}
\usepackage{smbicat}


\makeatletter
\let\ea\expandafter

%% Defining commands that are always in math mode.
\def\mdef#1#2{\ea\ea\ea\gdef\ea\ea\noexpand#1\ea{\ea\ensuremath\ea{#2}}}
\def\alwaysmath#1{\ea\ea\ea\global\ea\ea\ea\let\ea\ea\csname your@#1\endcsname\csname #1\endcsname
  \ea\def\csname #1\endcsname{\ensuremath{\csname your@#1\endcsname}}}

% Script letters
\newcommand{\sA}{\ensuremath{\mathscr{A}}}
\newcommand{\sB}{\ensuremath{\mathscr{B}}}
\newcommand{\sC}{\ensuremath{\mathscr{C}}}
\newcommand{\sD}{\ensuremath{\mathscr{D}}}
\newcommand{\sE}{\ensuremath{\mathscr{E}}}
\newcommand{\sF}{\ensuremath{\mathscr{F}}}
\newcommand{\sG}{\ensuremath{\mathscr{G}}}
\newcommand{\sH}{\ensuremath{\mathscr{H}}}
\newcommand{\sI}{\ensuremath{\mathscr{I}}}
\newcommand{\sJ}{\ensuremath{\mathscr{J}}}
\newcommand{\sK}{\ensuremath{\mathscr{K}}}
\newcommand{\sL}{\ensuremath{\mathscr{L}}}
\newcommand{\sM}{\ensuremath{\mathscr{M}}}
\newcommand{\sN}{\ensuremath{\mathscr{N}}}
\newcommand{\sO}{\ensuremath{\mathscr{O}}}
\newcommand{\sP}{\ensuremath{\mathscr{P}}}
\newcommand{\sQ}{\ensuremath{\mathscr{Q}}}
\newcommand{\sR}{\ensuremath{\mathscr{R}}}
\newcommand{\sS}{\ensuremath{\mathscr{S}}}
\newcommand{\sT}{\ensuremath{\mathscr{T}}}
\newcommand{\sU}{\ensuremath{\mathscr{U}}}
\newcommand{\sV}{\ensuremath{\mathscr{V}}}
\newcommand{\sW}{\ensuremath{\mathscr{W}}}
\newcommand{\sX}{\ensuremath{\mathscr{X}}}
\newcommand{\sY}{\ensuremath{\mathscr{Y}}}
\newcommand{\sZ}{\ensuremath{\mathscr{Z}}}

% Calligraphic letters
\newcommand{\cA}{\ensuremath{\mathcal{A}}}
\newcommand{\cB}{\ensuremath{\mathcal{B}}}
\newcommand{\cC}{\ensuremath{\mathcal{C}}}
\newcommand{\cD}{\ensuremath{\mathcal{D}}}
\newcommand{\cE}{\ensuremath{\mathcal{E}}}
\newcommand{\cF}{\ensuremath{\mathcal{F}}}
\newcommand{\cG}{\ensuremath{\mathcal{G}}}
\newcommand{\cH}{\ensuremath{\mathcal{H}}}
\newcommand{\cI}{\ensuremath{\mathcal{I}}}
\newcommand{\cJ}{\ensuremath{\mathcal{J}}}
\newcommand{\cK}{\ensuremath{\mathcal{K}}}
\newcommand{\cL}{\ensuremath{\mathcal{L}}}
\newcommand{\cM}{\ensuremath{\mathcal{M}}}
\newcommand{\cN}{\ensuremath{\mathcal{N}}}
\newcommand{\cO}{\ensuremath{\mathcal{O}}}
\newcommand{\cP}{\ensuremath{\mathcal{P}}}
\newcommand{\cQ}{\ensuremath{\mathcal{Q}}}
\newcommand{\cR}{\ensuremath{\mathcal{R}}}
\newcommand{\cS}{\ensuremath{\mathcal{S}}}
\newcommand{\cT}{\ensuremath{\mathcal{T}}}
\newcommand{\cU}{\ensuremath{\mathcal{U}}}
\newcommand{\cV}{\ensuremath{\mathcal{V}}}
\newcommand{\cW}{\ensuremath{\mathcal{W}}}
\newcommand{\cX}{\ensuremath{\mathcal{X}}}
\newcommand{\cY}{\ensuremath{\mathcal{Y}}}
\newcommand{\cZ}{\ensuremath{\mathcal{Z}}}

% blackboard bold letters
\newcommand{\lA}{\ensuremath{\mathbb{A}}}
\newcommand{\lB}{\ensuremath{\mathbb{B}}}
\newcommand{\lC}{\ensuremath{\mathbb{C}}}
\newcommand{\lD}{\ensuremath{\mathbb{D}}}
\newcommand{\lE}{\ensuremath{\mathbb{E}}}
\newcommand{\lF}{\ensuremath{\mathbb{F}}}
\newcommand{\lG}{\ensuremath{\mathbb{G}}}
\newcommand{\lH}{\ensuremath{\mathbb{H}}}
\newcommand{\lI}{\ensuremath{\mathbb{I}}}
\newcommand{\lJ}{\ensuremath{\mathbb{J}}}
\newcommand{\lK}{\ensuremath{\mathbb{K}}}
\newcommand{\lL}{\ensuremath{\mathbb{L}}}
\newcommand{\lM}{\ensuremath{\mathbb{M}}}
\newcommand{\lN}{\ensuremath{\mathbb{N}}}
\newcommand{\lO}{\ensuremath{\mathbb{O}}}
\newcommand{\lP}{\ensuremath{\mathbb{P}}}
\newcommand{\lQ}{\ensuremath{\mathbb{Q}}}
\newcommand{\lR}{\ensuremath{\mathbb{R}}}
\newcommand{\lS}{\ensuremath{\mathbb{S}}}
\newcommand{\lT}{\ensuremath{\mathbb{T}}}
\newcommand{\lU}{\ensuremath{\mathbb{U}}}
\newcommand{\lV}{\ensuremath{\mathbb{V}}}
\newcommand{\lW}{\ensuremath{\mathbb{W}}}
\newcommand{\lX}{\ensuremath{\mathbb{X}}}
\newcommand{\lY}{\ensuremath{\mathbb{Y}}}
\newcommand{\lZ}{\ensuremath{\mathbb{Z}}}

% bold letters
\newcommand{\bA}{\ensuremath{\mathbf{A}}}
\newcommand{\bB}{\ensuremath{\mathbf{B}}}
\newcommand{\bC}{\ensuremath{\mathbf{C}}}
\newcommand{\bD}{\ensuremath{\mathbf{D}}}
\newcommand{\bE}{\ensuremath{\mathbf{E}}}
\newcommand{\bF}{\ensuremath{\mathbf{F}}}
\newcommand{\bG}{\ensuremath{\mathbf{G}}}
\newcommand{\bH}{\ensuremath{\mathbf{H}}}
\newcommand{\bI}{\ensuremath{\mathbf{I}}}
\newcommand{\bJ}{\ensuremath{\mathbf{J}}}
\newcommand{\bK}{\ensuremath{\mathbf{K}}}
\newcommand{\bL}{\ensuremath{\mathbf{L}}}
\newcommand{\bM}{\ensuremath{\mathbf{M}}}
\newcommand{\bN}{\ensuremath{\mathbf{N}}}
\newcommand{\bO}{\ensuremath{\mathbf{O}}}
\newcommand{\bP}{\ensuremath{\mathbf{P}}}
\newcommand{\bQ}{\ensuremath{\mathbf{Q}}}
\newcommand{\bR}{\ensuremath{\mathbf{R}}}
\newcommand{\bS}{\ensuremath{\mathbf{S}}}
\newcommand{\bT}{\ensuremath{\mathbf{T}}}
\newcommand{\bU}{\ensuremath{\mathbf{U}}}
\newcommand{\bV}{\ensuremath{\mathbf{V}}}
\newcommand{\bW}{\ensuremath{\mathbf{W}}}
\newcommand{\bX}{\ensuremath{\mathbf{X}}}
\newcommand{\bY}{\ensuremath{\mathbf{Y}}}
\newcommand{\bZ}{\ensuremath{\mathbf{Z}}}

% fraktur letters
\newcommand{\fa}{\ensuremath{\mathfrak{a}}}
\newcommand{\fb}{\ensuremath{\mathfrak{b}}}
\newcommand{\fc}{\ensuremath{\mathfrak{c}}}
\newcommand{\fd}{\ensuremath{\mathfrak{d}}}
\newcommand{\fe}{\ensuremath{\mathfrak{e}}}
\newcommand{\ff}{\ensuremath{\mathfrak{f}}}
\newcommand{\fg}{\ensuremath{\mathfrak{g}}}
\newcommand{\fh}{\ensuremath{\mathfrak{h}}}
\newcommand{\fj}{\ensuremath{\mathfrak{j}}}
\newcommand{\fk}{\ensuremath{\mathfrak{k}}}
\newcommand{\fl}{\ensuremath{\mathfrak{l}}}
\newcommand{\fm}{\ensuremath{\mathfrak{m}}}
\newcommand{\fn}{\ensuremath{\mathfrak{n}}}
\newcommand{\fo}{\ensuremath{\mathfrak{o}}}
\newcommand{\fp}{\ensuremath{\mathfrak{p}}}
\newcommand{\fq}{\ensuremath{\mathfrak{q}}}
\newcommand{\fr}{\ensuremath{\mathfrak{r}}}
\newcommand{\fs}{\ensuremath{\mathfrak{s}}}
\newcommand{\ft}{\ensuremath{\mathfrak{t}}}
\newcommand{\fu}{\ensuremath{\mathfrak{u}}}
\newcommand{\fv}{\ensuremath{\mathfrak{v}}}
\newcommand{\fw}{\ensuremath{\mathfrak{w}}}
\newcommand{\fx}{\ensuremath{\mathfrak{x}}}
\newcommand{\fy}{\ensuremath{\mathfrak{y}}}
\newcommand{\fz}{\ensuremath{\mathfrak{z}}}

% fraktur letters
\newcommand{\fA}{\ensuremath{\mathfrak{A}}}
\newcommand{\fB}{\ensuremath{\mathfrak{B}}}
\newcommand{\fC}{\ensuremath{\mathfrak{C}}}

\mdef\fahat{\hat{\fa}}

% Underline letters
\newcommand{\uA}{\ensuremath{\underline{A}}}
\newcommand{\uB}{\ensuremath{\underline{B}}}
\newcommand{\uC}{\ensuremath{\underline{C}}}
\newcommand{\uD}{\ensuremath{\underline{D}}}
\newcommand{\uE}{\ensuremath{\underline{E}}}
\newcommand{\uF}{\ensuremath{\underline{F}}}
\newcommand{\uG}{\ensuremath{\underline{G}}}
\newcommand{\uH}{\ensuremath{\underline{H}}}
\newcommand{\uI}{\ensuremath{\underline{I}}}
\newcommand{\uJ}{\ensuremath{\underline{J}}}
\newcommand{\uK}{\ensuremath{\underline{K}}}
\newcommand{\uL}{\ensuremath{\underline{L}}}
\newcommand{\uM}{\ensuremath{\underline{M}}}
\newcommand{\uN}{\ensuremath{\underline{N}}}
\newcommand{\uO}{\ensuremath{\underline{O}}}
\newcommand{\uP}{\ensuremath{\underline{P}}}
\newcommand{\uQ}{\ensuremath{\underline{Q}}}
\newcommand{\uR}{\ensuremath{\underline{R}}}
\newcommand{\uS}{\ensuremath{\underline{S}}}
\newcommand{\uT}{\ensuremath{\underline{T}}}
\newcommand{\uU}{\ensuremath{\underline{U}}}
\newcommand{\uV}{\ensuremath{\underline{V}}}
\newcommand{\uW}{\ensuremath{\underline{W}}}
\newcommand{\uX}{\ensuremath{\underline{X}}}
\newcommand{\uY}{\ensuremath{\underline{Y}}}
\newcommand{\uZ}{\ensuremath{\underline{Z}}}

% bars
\newcommand{\Abar}{\ensuremath{\overline{A}}}
\newcommand{\Bbar}{\ensuremath{\overline{B}}}
\newcommand{\Cbar}{\ensuremath{\overline{C}}}
\newcommand{\Dbar}{\ensuremath{\overline{D}}}
\newcommand{\Ebar}{\ensuremath{\overline{E}}}
\newcommand{\Fbar}{\ensuremath{\overline{F}}}
\newcommand{\Gbar}{\ensuremath{\overline{G}}}
\newcommand{\Hbar}{\ensuremath{\overline{H}}}
\newcommand{\Ibar}{\ensuremath{\overline{I}}}
\newcommand{\Jbar}{\ensuremath{\overline{J}}}
\newcommand{\Kbar}{\ensuremath{\overline{K}}}
\newcommand{\Lbar}{\ensuremath{\overline{L}}}
\newcommand{\Mbar}{\ensuremath{\overline{M}}}
\newcommand{\Nbar}{\ensuremath{\overline{N}}}
\newcommand{\Obar}{\ensuremath{\overline{O}}}
\newcommand{\Pbar}{\ensuremath{\overline{P}}}
\newcommand{\Qbar}{\ensuremath{\overline{Q}}}
\newcommand{\Rbar}{\ensuremath{\overline{R}}}
\newcommand{\Sbar}{\ensuremath{\overline{S}}}
\newcommand{\Tbar}{\ensuremath{\overline{T}}}
\newcommand{\Ubar}{\ensuremath{\overline{U}}}
\newcommand{\Vbar}{\ensuremath{\overline{V}}}
\newcommand{\Wbar}{\ensuremath{\overline{W}}}
\newcommand{\Xbar}{\ensuremath{\overline{X}}}
\newcommand{\Ybar}{\ensuremath{\overline{Y}}}
\newcommand{\Zbar}{\ensuremath{\overline{Z}}}
\newcommand{\abar}{\ensuremath{\overline{a}}}
\newcommand{\bbar}{\ensuremath{\overline{b}}}
\newcommand{\cbar}{\ensuremath{\overline{c}}}
\newcommand{\dbar}{\ensuremath{\overline{d}}}
\newcommand{\ebar}{\ensuremath{\overline{e}}}
\newcommand{\fbar}{\ensuremath{\overline{f}}}
\newcommand{\gbar}{\ensuremath{\overline{g}}}
%\newcommand{\hbar}{\ensuremath{\overline{h}}} % whoops, \hbar means something else!
\newcommand{\ibar}{\ensuremath{\overline{\imath}}}
\newcommand{\jbar}{\ensuremath{\overline{\jmath}}}
\newcommand{\kbar}{\ensuremath{\overline{k}}}
\newcommand{\lbar}{\ensuremath{\overline{l}}}
\newcommand{\mbar}{\ensuremath{\overline{m}}}
\newcommand{\nbar}{\ensuremath{\overline{n}}}
%\newcommand{\obar}{\ensuremath{\overline{o}}}
\newcommand{\pbar}{\ensuremath{\overline{p}}}
\newcommand{\qbar}{\ensuremath{\overline{q}}}
\newcommand{\rbar}{\ensuremath{\overline{r}}}
\newcommand{\sbar}{\ensuremath{\overline{s}}}
\newcommand{\tbar}{\ensuremath{\overline{t}}}
\newcommand{\ubar}{\ensuremath{\overline{u}}}
\newcommand{\vbar}{\ensuremath{\overline{v}}}
\newcommand{\wbar}{\ensuremath{\overline{w}}}
\newcommand{\xbar}{\ensuremath{\overline{x}}}
\newcommand{\ybar}{\ensuremath{\overline{y}}}
\newcommand{\zbar}{\ensuremath{\overline{z}}}

% tildes
\newcommand{\Atil}{\ensuremath{\widetilde{A}}}
\newcommand{\Btil}{\ensuremath{\widetilde{B}}}
\newcommand{\Ctil}{\ensuremath{\widetilde{C}}}
\newcommand{\Dtil}{\ensuremath{\widetilde{D}}}
\newcommand{\Etil}{\ensuremath{\widetilde{E}}}
\newcommand{\Ftil}{\ensuremath{\widetilde{F}}}
\newcommand{\Gtil}{\ensuremath{\widetilde{G}}}
\newcommand{\Htil}{\ensuremath{\widetilde{H}}}
\newcommand{\Itil}{\ensuremath{\widetilde{I}}}
\newcommand{\Jtil}{\ensuremath{\widetilde{J}}}
\newcommand{\Ktil}{\ensuremath{\widetilde{K}}}
\newcommand{\Ltil}{\ensuremath{\widetilde{L}}}
\newcommand{\Mtil}{\ensuremath{\widetilde{M}}}
\newcommand{\Ntil}{\ensuremath{\widetilde{N}}}
\newcommand{\Otil}{\ensuremath{\widetilde{O}}}
\newcommand{\Ptil}{\ensuremath{\widetilde{P}}}
\newcommand{\Qtil}{\ensuremath{\widetilde{Q}}}
\newcommand{\Rtil}{\ensuremath{\widetilde{R}}}
\newcommand{\Stil}{\ensuremath{\widetilde{S}}}
\newcommand{\Ttil}{\ensuremath{\widetilde{T}}}
\newcommand{\Util}{\ensuremath{\widetilde{U}}}
\newcommand{\Vtil}{\ensuremath{\widetilde{V}}}
\newcommand{\Wtil}{\ensuremath{\widetilde{W}}}
\newcommand{\Xtil}{\ensuremath{\widetilde{X}}}
\newcommand{\Ytil}{\ensuremath{\widetilde{Y}}}
\newcommand{\Ztil}{\ensuremath{\widetilde{Z}}}
\newcommand{\atil}{\ensuremath{\widetilde{a}}}
\newcommand{\btil}{\ensuremath{\widetilde{b}}}
\newcommand{\ctil}{\ensuremath{\widetilde{c}}}
\newcommand{\dtil}{\ensuremath{\widetilde{d}}}
\newcommand{\etil}{\ensuremath{\widetilde{e}}}
\newcommand{\ftil}{\ensuremath{\widetilde{f}}}
\newcommand{\gtil}{\ensuremath{\widetilde{g}}}
\newcommand{\htil}{\ensuremath{\widetilde{h}}}
\newcommand{\itil}{\ensuremath{\widetilde{\imath}}}
\newcommand{\jtil}{\ensuremath{\widetilde{\jmath}}}
\newcommand{\ktil}{\ensuremath{\widetilde{k}}}
\newcommand{\ltil}{\ensuremath{\widetilde{l}}}
\newcommand{\mtil}{\ensuremath{\widetilde{m}}}
\newcommand{\ntil}{\ensuremath{\widetilde{n}}}
\newcommand{\otil}{\ensuremath{\widetilde{o}}}
\newcommand{\ptil}{\ensuremath{\widetilde{p}}}
\newcommand{\qtil}{\ensuremath{\widetilde{q}}}
\newcommand{\rtil}{\ensuremath{\widetilde{r}}}
\newcommand{\stil}{\ensuremath{\widetilde{s}}}
\newcommand{\ttil}{\ensuremath{\widetilde{t}}}
\newcommand{\util}{\ensuremath{\widetilde{u}}}
\newcommand{\vtil}{\ensuremath{\widetilde{v}}}
\newcommand{\wtil}{\ensuremath{\widetilde{w}}}
\newcommand{\xtil}{\ensuremath{\widetilde{x}}}
\newcommand{\ytil}{\ensuremath{\widetilde{y}}}
\newcommand{\ztil}{\ensuremath{\widetilde{z}}}

% Hats
\newcommand{\Ahat}{\ensuremath{\widehat{A}}}
\newcommand{\Bhat}{\ensuremath{\widehat{B}}}
\newcommand{\Chat}{\ensuremath{\widehat{C}}}
\newcommand{\Dhat}{\ensuremath{\widehat{D}}}
\newcommand{\Ehat}{\ensuremath{\widehat{E}}}
\newcommand{\Fhat}{\ensuremath{\widehat{F}}}
\newcommand{\Ghat}{\ensuremath{\widehat{G}}}
\newcommand{\Hhat}{\ensuremath{\widehat{H}}}
\newcommand{\Ihat}{\ensuremath{\widehat{I}}}
\newcommand{\Jhat}{\ensuremath{\widehat{J}}}
\newcommand{\Khat}{\ensuremath{\widehat{K}}}
\newcommand{\Lhat}{\ensuremath{\widehat{L}}}
\newcommand{\Mhat}{\ensuremath{\widehat{M}}}
\newcommand{\Nhat}{\ensuremath{\widehat{N}}}
\newcommand{\Ohat}{\ensuremath{\widehat{O}}}
\newcommand{\Phat}{\ensuremath{\widehat{P}}}
\newcommand{\Qhat}{\ensuremath{\widehat{Q}}}
\newcommand{\Rhat}{\ensuremath{\widehat{R}}}
\newcommand{\Shat}{\ensuremath{\widehat{S}}}
\newcommand{\That}{\ensuremath{\widehat{T}}}
\newcommand{\Uhat}{\ensuremath{\widehat{U}}}
\newcommand{\Vhat}{\ensuremath{\widehat{V}}}
\newcommand{\What}{\ensuremath{\widehat{W}}}
\newcommand{\Xhat}{\ensuremath{\widehat{X}}}
\newcommand{\Yhat}{\ensuremath{\widehat{Y}}}
\newcommand{\Zhat}{\ensuremath{\widehat{Z}}}
\newcommand{\ahat}{\ensuremath{\hat{a}}}
\newcommand{\bhat}{\ensuremath{\hat{b}}}
\newcommand{\chat}{\ensuremath{\hat{c}}}
\newcommand{\dhat}{\ensuremath{\hat{d}}}
\newcommand{\ehat}{\ensuremath{\hat{e}}}
\newcommand{\fhat}{\ensuremath{\hat{f}}}
\newcommand{\ghat}{\ensuremath{\hat{g}}}
\newcommand{\hhat}{\ensuremath{\hat{h}}}
\newcommand{\ihat}{\ensuremath{\hat{\imath}}}
\newcommand{\jhat}{\ensuremath{\hat{\jmath}}}
\newcommand{\khat}{\ensuremath{\hat{k}}}
\newcommand{\lhat}{\ensuremath{\hat{l}}}
\newcommand{\mhat}{\ensuremath{\hat{m}}}
\newcommand{\nhat}{\ensuremath{\hat{n}}}
\newcommand{\ohat}{\ensuremath{\hat{o}}}
\newcommand{\phat}{\ensuremath{\hat{p}}}
\newcommand{\qhat}{\ensuremath{\hat{q}}}
\newcommand{\rhat}{\ensuremath{\hat{r}}}
\newcommand{\shat}{\ensuremath{\hat{s}}}
\newcommand{\that}{\ensuremath{\hat{t}}}
\newcommand{\uhat}{\ensuremath{\hat{u}}}
\newcommand{\vhat}{\ensuremath{\hat{v}}}
\newcommand{\what}{\ensuremath{\hat{w}}}
\newcommand{\xhat}{\ensuremath{\hat{x}}}
\newcommand{\yhat}{\ensuremath{\hat{y}}}
\newcommand{\zhat}{\ensuremath{\hat{z}}}

%% FONTS AND DECORATION FOR GREEK LETTERS

%% the package `mathbbol' gives us blackboard bold greek and numbers,
%% but it does it by redefining \mathbb to use a different font, so that
%% all the other \mathbb letters look different too.  Here we import the
%% font with bb greek and numbers, but assign it a different name,
%% \mathbbb, so as not to replace the usual one.
\DeclareSymbolFont{bbold}{U}{bbold}{m}{n}
\DeclareSymbolFontAlphabet{\mathbbb}{bbold}
\newcommand{\bbDelta}{\ensuremath{\mathbbb{\Delta}}}
\newcommand{\bbone}{\ensuremath{\mathbbb{1}}}
\newcommand{\bbtwo}{\ensuremath{\mathbbb{2}}}
\newcommand{\bbthree}{\ensuremath{\mathbbb{3}}}

% greek with bars
\newcommand{\albar}{\ensuremath{\overline{\alpha}}}
\newcommand{\bebar}{\ensuremath{\overline{\beta}}}
\newcommand{\gmbar}{\ensuremath{\overline{\gamma}}}
\newcommand{\debar}{\ensuremath{\overline{\delta}}}
\newcommand{\phibar}{\ensuremath{\overline{\varphi}}}
\newcommand{\psibar}{\ensuremath{\overline{\psi}}}
\newcommand{\xibar}{\ensuremath{\overline{\xi}}}
\newcommand{\ombar}{\ensuremath{\overline{\omega}}}

% greek with hats
\newcommand{\alhat}{\ensuremath{\hat{\alpha}}}
\newcommand{\behat}{\ensuremath{\hat{\beta}}}
\newcommand{\gmhat}{\ensuremath{\hat{\gamma}}}
\newcommand{\dehat}{\ensuremath{\hat{\delta}}}

% greek with checks
\newcommand{\alchk}{\ensuremath{\check{\alpha}}}
\newcommand{\bechk}{\ensuremath{\check{\beta}}}
\newcommand{\gmchk}{\ensuremath{\check{\gamma}}}
\newcommand{\dechk}{\ensuremath{\check{\delta}}}

% greek with tildes
\newcommand{\altil}{\ensuremath{\widetilde{\alpha}}}
\newcommand{\betil}{\ensuremath{\widetilde{\beta}}}
\newcommand{\gmtil}{\ensuremath{\widetilde{\gamma}}}
\newcommand{\phitil}{\ensuremath{\widetilde{\varphi}}}
\newcommand{\psitil}{\ensuremath{\widetilde{\psi}}}
\newcommand{\xitil}{\ensuremath{\widetilde{\xi}}}
\newcommand{\omtil}{\ensuremath{\widetilde{\omega}}}

% MISCELLANEOUS SYMBOLS
\mdef\del{\partial}
\mdef\delbar{\overline{\partial}}
\let\sm\wedge
\newcommand{\dd}[1]{\ensuremath{\frac{\partial}{\partial {#1}}}}
\newcommand{\inv}{^{-1}}
\newcommand{\dual}{^{\vee}}
\mdef\hf{\textstyle\frac{1}{2}}
\mdef\thrd{\textstyle\frac{1}{3}}
\mdef\qtr{\textstyle\frac{1}{4}}
\let\meet\wedge
\let\join\vee
\let\dn\downarrow
\newcommand{\op}{^{\mathit{op}}}
\newcommand{\co}{^{\mathit{co}}}
\newcommand{\coop}{^{\mathit{coop}}}
\let\adj\dashv
\SelectTips{cm}{}
\newdir{ >}{{}*!/-10pt/@{>}}    % extra spacing for tail arrows in XYpic
\newcommand{\pushoutcorner}[1][dr]{\save*!/#1+1.2pc/#1:(1,-1)@^{|-}\restore}
\newcommand{\pullbackcorner}[1][dr]{\save*!/#1-1.2pc/#1:(-1,1)@^{|-}\restore}
\let\iso\cong
\let\eqv\simeq
\let\cng\equiv
\mdef\Id{\mathrm{Id}}
\mdef\id{\mathrm{id}}
\alwaysmath{ell}
\alwaysmath{infty}
\alwaysmath{odot}
\def\frc#1/#2.{\frac{#1}{#2}}   % \frc x^2+1 / x^2-1 .
\mdef\ten{\mathrel{\otimes}}
\mdef\bigten{\bigotimes}
\mdef\sqten{\mathrel{\boxtimes}}
\def\pow(#1,#2){\mathop{\pitchfork}(#1,#2)} % powers and
\def\cpw{\mathop{\odot}}                    % copowers
\newcommand{\mathid}{\mbox{id}}
\newcommand{\cat}[1]{\ensuremath{\mathbf{#1}}}


%% OPERATORS
\DeclareMathOperator\lan{Lan}
\DeclareMathOperator\ran{Ran}
\DeclareMathOperator\colim{colim}
\DeclareMathOperator\coeq{coeq}
\DeclareMathOperator\eq{eq}
\DeclareMathOperator\Tot{Tot}
\DeclareMathOperator\cosk{cosk}
\DeclareMathOperator\sk{sk}
\DeclareMathOperator\im{im}
\DeclareMathOperator\Spec{Spec}
\DeclareMathOperator\Ho{Ho}
\DeclareMathOperator\Aut{Aut}
\DeclareMathOperator\End{End}
\DeclareMathOperator\Hom{Hom}
\DeclareMathOperator\Map{Map}

%% TIKZ ARROWS AND HIGHER CELLS
\makeatletter
\def\slashedarrowfill@#1#2#3#4#5{%
  $\m@th\thickmuskip0mu\medmuskip\thickmuskip\thinmuskip\thickmuskip
   \relax#5#1\mkern-7mu%
   \cleaders\hbox{$#5\mkern-2mu#2\mkern-2mu$}\hfill
   \mathclap{#3}\mathclap{#2}%
   \cleaders\hbox{$#5\mkern-2mu#2\mkern-2mu$}\hfill
   \mkern-7mu#4$%
}

\def\Rightslashedarrowfill@{%
  \slashedarrowfill@\Relbar\Relbar\Mapstochar\Rightarrow}
\newcommand\xslashedRightarrow[2][]{%
  \ext@arrow 0055{\Rightslashedarrowfill@}{#1}{#2}}
\def\hTo{\xslashedRightarrow{}}
\def\hToo{\xslashedRightarrow{\quad}}
\let\xhTo\xslashedRightarrow

\pagestyle{empty}

\newcommand{\Rightthreecell}{\RRightarrow}
\newcommand{\Rtwocell}{\Rightarrow}

\tikzstyle{doubletick}=[-implies, double equal sign distance, postaction={decorate},decoration={markings,mark=at position .5 with {\draw[-] (0,-0.1) -- (0,0.1);}}]

\tikzstyle{darrow}=[-implies, double equal sign distance]

\tikzstyle{doubleeq}=[double equal sign distance]


%% ARROWS
% \to already exists
\newcommand{\too}[1][]{\ensuremath{\overset{#1}{\longrightarrow}}}
\newcommand{\ot}{\ensuremath{\leftarrow}}
\newcommand{\oot}[1][]{\ensuremath{\overset{#1}{\longleftarrow}}}
\let\toot\rightleftarrows
\let\otto\leftrightarrows
\let\Impl\Rightarrow
\let\imp\Rightarrow
\let\toto\rightrightarrows
\let\into\hookrightarrow
\let\xinto\xhookrightarrow
\mdef\we{\overset{\sim}{\longrightarrow}}
\mdef\leftwe{\overset{\sim}{\longleftarrow}}
\let\mono\rightarrowtail
\let\leftmono\leftarrowtail
\let\cof\rightarrowtail
\let\leftcof\leftarrowtail
\let\epi\twoheadrightarrow
\let\leftepi\twoheadleftarrow
\let\fib\twoheadrightarrow
\let\leftfib\twoheadleftarrow
\let\cohto\rightsquigarrow
\let\maps\colon
\newcommand{\spam}{\,:\!}       % \maps for left arrows

\newsavebox{\DDownarrowbox}
\savebox{\DDownarrowbox}{\tikz[scale=1.5]{\draw[-implies,double equal
sign distance] (0,.1) -- (0,-.1); \draw (0,.1) -- (0,-.1);}}
\newcommand{\DDownarrow}{\mathrel{\raisebox{-.2em}{\usebox{\DDownarrowbox}}}}

\newsavebox{\RRightarrowbox}
\savebox{\RRightarrowbox}{\tikz[scale=1.5]{\draw[-implies,double equal
sign distance] (-.1,0) -- (.1,0); \draw (-.1,0) -- (.1,0);}}
\newcommand{\RRightarrow}{\mathrel{\raisebox{.2em}{\usebox{\RRightarrowbox}}}}

%\newsavebox{\Rightslashedarrowbox}
%\savebox{\Rightslashedarrowbox}{\tikz[scale=1.5]{\draw[Rightslashedarrow{}] (-.1,0) -- (1,0);}}
%\newcommand{\Rightslashedarrow}{\mathrel{\raisebox{-.2em}%{\usebox{\Rightslashedarrowbox}}}}


%% EXTENSIBLE ARROWS
\let\xto\xrightarrow
\let\xot\xleftarrow
% See Voss' Mathmode.tex for instructions on how to create new
% extensible arrows.
\def\rightarrowtailfill@{\arrowfill@{\Yright\joinrel\relbar}\relbar\rightarrow}
\newcommand\xrightarrowtail[2][]{\ext@arrow 0055{\rightarrowtailfill@}{#1}{#2}}
\let\xmono\xrightarrowtail
\let\xcof\xrightarrowtail
\def\twoheadrightarrowfill@{\arrowfill@{\relbar\joinrel\relbar}\relbar\twoheadrightarrow}
\newcommand\xtwoheadrightarrow[2][]{\ext@arrow 0055{\twoheadrightarrowfill@}{#1}{#2}}
\let\xepi\xtwoheadrightarrow
\let\xfib\xtwoheadrightarrow
% Let's leave the left-going ones until I need them.

%% EXTENSIBLE SLASHED ARROWS
% Making extensible slashed arrows, by modifying the underlying AMS code.
% Arguments are:
% 1 = arrowhead on the left (\relbar or \Relbar if none)
% 2 = fill character (usually \relbar or \Relbar)
% 3 = slash character (such as \mapstochar or \Mapstochar)
% 4 = arrowhead on the left (\relbar or \Relbar if none)
% 5 = display mode (\displaystyle etc)
\def\slashedarrowfill@#1#2#3#4#5{%
  $\m@th\thickmuskip0mu\medmuskip\thickmuskip\thinmuskip\thickmuskip
   \relax#5#1\mkern-7mu%
   \cleaders\hbox{$#5\mkern-2mu#2\mkern-2mu$}\hfill
   \mathclap{#3}\mathclap{#2}%
   \cleaders\hbox{$#5\mkern-2mu#2\mkern-2mu$}\hfill
   \mkern-7mu#4$%
}
% Here's the idea: \<slashed>arrowfill@ should be a box containing
% some stretchable space that is the "middle of the arrow".  This
% space is created as a "leader" using \cleader<thing>\hfill, which
% fills an \hfill of space with copies of <thing>.  Here instead of
% just one \cleader, we use two, with the slash in between (and an
% extra copy of the filler, to avoid extra space around the slash).
\def\rightslashedarrowfill@{%
  \slashedarrowfill@\relbar\relbar\mapstochar\rightarrow}
\newcommand\xslashedrightarrow[2][]{%
  \ext@arrow 0055{\rightslashedarrowfill@}{#1}{#2}}
\mdef\hto{\xslashedrightarrow{}}
\mdef\htoo{\xslashedrightarrow{\quad}}
\let\xhto\xslashedrightarrow

%% To get a slashed arrow in XYpic, do
% \[\xymatrix{A \ar[r]|-@{|} & B}\]

% ISOMORPHISMS
\def\xiso#1{\mathrel{\mathrlap{\smash{\xto[\smash{\raisebox{1.3mm}{$\scriptstyle\sim$}}]{#1}}}\hphantom{\xto{#1}}}}
\def\toiso{\xto{\smash{\raisebox{-.5mm}{$\scriptstyle\sim$}}}}

% SHADOWS
\def\shvar#1#2{{\ensuremath{%
  \hspace{1mm}\makebox[-1mm]{$#1\langle$}\makebox[0mm]{$#1\langle$}\hspace{1mm}%
  {#2}%
  \makebox[1mm]{$#1\rangle$}\makebox[0mm]{$#1\rangle$}%
}}}
\def\sh{\shvar{}}
\def\scriptsh{\shvar{\scriptstyle}}
\def\bigsh{\shvar{\big}}
\def\Bigsh{\shvar{\Big}}
\def\biggsh{\shvar{\bigg}}
\def\Biggsh{\shvar{\Bigg}}

%HIGHER CELLS



% THEOREM-TYPE ENVIRONMENTS, hacked to
%% (a) number all with the same numbers, and
%% (b) have the right names for autoref
\def\defthm#1#2{%
  \newtheorem{#1}{#2}[section]%
  \expandafter\def\csname #1autorefname\endcsname{#2}%
  \expandafter\let\csname c@#1\endcsname\c@thm}
\newtheorem{thm}{Theorem}[section]
\newcommand{\thmautorefname}{Theorem}
\defthm{cor}{Corollary}
\defthm{prop}{Proposition}
\defthm{lem}{Lemma}
\defthm{sch}{Scholium}
\defthm{assume}{Assumption}
\defthm{claim}{Claim}
\defthm{conj}{Conjecture}
\defthm{hyp}{Hypothesis}
\defthm{fact}{Fact}
\theoremstyle{definition}
\defthm{defn}{Definition}
\defthm{notn}{Notation}
\theoremstyle{remark}
\defthm{rmk}{Remark}
\defthm{eg}{Example}
\defthm{egs}{Examples}
\defthm{ex}{Exercise}
\defthm{ceg}{Counterexample}

% How to get QED symbols inside equations at the end of the statements
% of theorems.  AMS LaTeX knows how to do this inside equations at the
% end of *proofs* with \qedhere, and at the end of the statement of a
% theorem with \qed (meaning no proof will be given), but it can't
% seem to combine the two.  Use this instead.
\def\thmqedhere{\expandafter\csname\csname @currenvir\endcsname @qed\endcsname}

% Number numbered lists as (i), (ii), ...
\renewcommand{\theenumi}{(\roman{enumi})}
\renewcommand{\labelenumi}{\theenumi}

%% Labeling that keeps track of theorem-type names.  Superseded by
%% autoref from hyperref, as above, but we keep this in case we are
%% using a journal style file that is incompatible with hyperref.
% 
% \ifx\SK@label\undefined\let\SK@label\label\fi
% \let\your@thm\@thm
% \def\@thm#1#2#3{\gdef\currthmtype{#3}\your@thm{#1}{#2}{#3}}
% \def\xlabel#1{{\let\your@currentlabel\@currentlabel\def\@currentlabel
% {\currthmtype~\your@currentlabel}
% \SK@label{#1@}}\label{#1}}
% \def\xref#1{\ref{#1@}}

% Also number formulas with the theorem counter
\let\c@equation\c@thm
\numberwithin{equation}{section}

% Only show numbers for equations that are actually referenced (or
% whose tags are specified manually) - uses mathtools.
\mathtoolsset{showonlyrefs,showmanualtags}

%% Fix enumerate spacing with paralist.  This has two parts:
%%   1. enable mixing of "old spacing" lists with those adjusted by paralist
%%   2. allow us to specify a number based on which to adjust the spacing
%% For the first, use \killspacingtrue when you want the spacing
%% adjusted, then \killspacingfalse to turn adjustment off.  For the
%% second, use \maxenum=14 to set the widest number you want the
%% spacing to be calculated with.
\newlength\oldleftmargini       % save old spacing
\newlength\oldleftmarginii
\newlength\oldleftmarginiii
\newlength\oldleftmarginiv
\newlength\oldleftmarginv
\newlength\oldleftmarginvi
\newcount\maxenum
\maxenum=7
\newif\ifkillspacing
\def\@adjust@enum@labelwidth{%
  \advance\@listdepth by 1\relax
  \ifkillspacing                % do the paralist thing
    \csname c@\@enumctr\endcsname\maxenum
    \settowidth{\@tempdima}{%
      \csname label\@enumctr\endcsname\hspace{\labelsep}}%
    \csname leftmargin\romannumeral\@listdepth\endcsname
      \@tempdima
  \else                         % otherwise, reset it
    \csname fixspacing\romannumeral\@listdepth\endcsname
  \fi
  \advance\@listdepth by -1\relax}
% these commands, one for each enum level (I couldn't get a generic
% one to work), test whether oldleftmargin has been set yet, and if
% not, set it from leftmargin; otherwise, they reset leftmargin to
% it.  Just setting oldleftmargin to leftmargin in the preamble
% doesn't seem to work.
\def\fixspacingi{\ifnum\oldleftmargini=0\setlength\oldleftmargini\leftmargini\else\setlength\leftmargini\oldleftmargini\fi}
\def\fixspacingii{\ifnum\oldleftmarginii=0\setlength\oldleftmarginii\leftmarginii\else\setlength\leftmarginii\oldleftmarginii\fi}
\def\fixspacingiii{\ifnum\oldleftmarginiii=0\setlength\oldleftmarginiii\leftmarginiii\else\setlength\leftmarginiii\oldleftmarginiii\fi}
\def\fixspacingiv{\ifnum\oldleftmarginiv=0\setlength\oldleftmarginiv\leftmarginiv\else\setlength\leftmarginiv\oldleftmarginiv\fi}
\def\fixspacingv{\ifnum\oldleftmarginv=0\setlength\oldleftmarginv\leftmarginv\else\setlength\leftmarginv\oldleftmarginv\fi}
\def\fixspacingvi{\ifnum\oldleftmarginvi=0\setlength\oldleftmarginvi\leftmarginvi\else\setlength\leftmarginvi\oldleftmarginvi\fi}

%% Fix paralist references, so that we can refer to (1) instead of
%% just 1.
\def\pl@label#1#2{%
  \edef\pl@the{\noexpand#1{\@enumctr}}%
  \pl@lab\expandafter{\the\pl@lab\csname yourthe\@enumctr\endcsname}%
  \advance\@tempcnta1
  \pl@loop}
\def\@enumlabel@#1[#2]{%
  \@plmylabeltrue
  \@tempcnta0
  \pl@lab{}%
  \let\pl@the\pl@qmark
  \expandafter\pl@loop\@gobble#2\@@@
  \ifnum\@tempcnta=1\else
    \PackageWarning{paralist}{Incorrect label; no or multiple
      counters.\MessageBreak The label is: \@gobble#2}%
  \fi
  \expandafter\edef\csname label\@enumctr\endcsname{\the\pl@lab}%
  \expandafter\edef\csname the\@enumctr\endcsname{\the\pl@lab}%
  \expandafter\let\csname yourthe\@enumctr\endcsname\pl@the
  #1}


% GREEK LETTERS, ETC.
\alwaysmath{alpha}
\alwaysmath{beta}
\alwaysmath{gamma}
\alwaysmath{Gamma}
\alwaysmath{delta}
\alwaysmath{Delta}
\alwaysmath{epsilon}
\mdef\ep{\varepsilon}
\alwaysmath{zeta}
\alwaysmath{eta}
\alwaysmath{theta}
\alwaysmath{Theta}
\alwaysmath{iota}
\alwaysmath{kappa}
\alwaysmath{lambda}
\alwaysmath{Lambda}
\alwaysmath{mu}
\alwaysmath{nu}
\alwaysmath{xi}
\alwaysmath{pi}
\alwaysmath{rho}
\alwaysmath{sigma}
\alwaysmath{Sigma}
\alwaysmath{tau}
\alwaysmath{upsilon}
\alwaysmath{Upsilon}
\alwaysmath{phi}
\alwaysmath{Pi}
\alwaysmath{Phi}
\mdef\ph{\varphi}
\alwaysmath{chi}
\alwaysmath{psi}
\alwaysmath{Psi}
\alwaysmath{omega}
\alwaysmath{Omega}
\let\al\alpha
\let\be\beta
\let\gm\gamma
\let\Gm\Gamma
\let\de\delta
\let\De\Delta
\let\si\sigma
\let\Si\Sigma
\let\om\omega
\let\ka\kappa
\let\la\lambda
\let\La\Lambda
\let\ze\zeta
\let\th\theta
\let\Th\Theta
\let\vth\vartheta

\makeatother

% Tikz styles
\tikzstyle{tickarrow}=[->,postaction={decorate},decoration={markings,mark=at position .5 with {\draw[-] (0,-0.1) -- (0,0.1);}},line width=0.50]

% Local Variables:
% mode: latex
% TeX-master: ""
% End:

\let\cref\autoref
\UseAllTwocells
\title{Constructing symmetric monoidal bicategories functorially}
\author{Michael A.\ Shulman}
\author{Linde Wester Hansen}
\thanks{This material is based on research sponsored by The United States Air Force Research Laboratory under agreement number FA9550-15-1-0053.  The U.S.~Government is authorized to reproduce and distribute reprints for Governmental purposes notwithstanding any copyright notation thereon.  The views and conclusions contained herein are those of the authors and should not be interpreted as necessarily representing the official policies or endorsements, either expressed or implied, of the United States Air Force Research Laboratory, the U.S.~Government, or Carnegie Mellon University.}
\mdef\cMod{\mathcal{M}\mathit{od}}
\mdef\cCat{\mathcal{C}\mathit{at}}
\mdef\cTwocat{2\text{-}\mathcal{C}\mathit{at}}
\mdef\cBicat{\mathcal{B}\mathit{icat}}
\mdef\cCat{\mathcal{C}\mathit{at}}
\mdef\cVect{\mathcal{V}\mathit{ect}}
\mdef\cHilb{\mathcal{H}\mathit{ilb}}
\mdef\cProf{\mathcal{P}\mathit{rof}}
\mdef\cSpan{\mathcal{S}\mathit{pan}}
\mdef\cMon{\mathcal{M}\mathit{on}}
\mdef\cBr{\mathcal{B}\mathit{r}}
\mdef\cSyl{\mathcal{S}\mathit{yl}}
\mdef\cSym{\mathcal{S}\mathit{ym}}
\mdef\cH{\mathcal{L}}
\mdef\fBicat{\mathfrak{Bicat}}
\mdef\fDbl{\mathfrak{Dbl}}
\mdef\fH{\mathfrak{H}}
\mdef\fC{\mathfrak{T}}
\mdef\lMod{\mathbb{M}\mathsf{od}}
\mdef\lnCob{n\mathbb{C}\mathsf{ob}}
\let\ltwo\bbtwo
\mdef\lProf{\mathbb{P}\mathsf{rof}}
\mdef\lSpan{\mathbb{S}\mathsf{pan}}
\mdef\lMat{\mathbb{M}\mathsf{at}}
\mdef\cDbl{\mathcal{D}\mathit{bl}}
\mdef\cDblcf{\cDbl_{c\mathbf{f}}}
\mdef\cDbllf{\cDbl_{l\mathbf{f}}}
\mdef\cDblf{\cDbl_{\mathbf{f}}}
\mdef\fDblf{\mathfrak{Dbl}_{\mathbf{f}}}
\mdef\cMonDbll{\mathcal{M}\mathit{on}\mathcal{D}\mathit{bl}_l}
\mdef\cMonDblc{\mathcal{M}\mathit{on}\mathcal{D}\mathit{bl}_c}
\mdef\cMonDblp{\mathcal{M}\mathit{on}\mathcal{D}\mathit{bl}_p}
\mdef\fchk{\check{f}}
\mdef\conj{\Yleft}
\mdef\Conj{\mathcal{C}\mathit{onj}}
\mdef\Icon{\mathcal{I}\mathit{con}}
\mdef\id{\mathsf{id}}
\mdef\C{\mathbb{C}}
\mdef\D{\mathbb{D}}
\mdef\E{\mathbb{E}}
\mdef\F{\mathbb{F}}
\newcommand{\bAlg}{\mathbb{A}\mathsf{lg}}
\let\lAlg\bAlg
\newcommand{\cAlg}{\mathcal{A}\mathit{lg}}
\newcommand{\Alg}{\mathbb{A}\mathsf{lg}}
\newcommand{\tens}{\otimes}
\newcommand{\onecell}{\rightarrow}
\newcommand{\ver}{\cdot}
\newcommand{\hor}{\bullet}
\newcommand{\verc}{\cdot}
\newcommand{\horc}{\bullet}
\newcommand{\comp}{\circ}
\newcommand{\looseid}{\Id}
\newcommand{\tightid}{1}
\newcommand{\transid}{\id}
\newcommand{\mult}{\tinymult[gray dot]}
\newcommand{\unit}{\tinyunit[gray dot]}

\newcommand{\lop}{^{l\cdot\mathrm{op}}}
\newcommand{\ttop}{^{t\cdot\mathrm{op}}}
\newcommand{\tlop}{^{tl\cdot\mathrm{op}}}
\newcommand{\lco}{^{l\cdot\mathrm{co}}}
\newcommand{\tco}{^{t\cdot\mathrm{co}}}
\newcommand{\tlco}{^{tl\cdot\mathrm{co}}}
\newcommand{\lcoop}{^{l\cdot\mathrm{coop}}}

\newcommand{\hora}{a^{\horc}}
\newcommand{\horr}{r^{\horc}}
\newcommand{\horl}{l^{\horc}}
\newcommand{\compa}{a^{\comp}}
\newcommand{\compr}{r^{\comp}}
\newcommand{\compl}{l^{\comp}}
\newcommand{\compI}{I^{\comp}}


\newcounter{mondbl}             % For restarting enums manually

%\includeonly{}

\hyphenation{mon-oid-al}

\begin{document}

\begin{abstract}
  We present a method of constructing monoidal, braided monoidal, and symmetric monoidal bicategories from corresponding types of monoidal double categories that satisfy a lifting condition.
  Many important monoidal bicategories arise naturally in this way, and applying our general method is much easier than explicitly verifying the coherence laws of a monoidal bicategory for each example.
  Abstracting from earlier work in this direction, we express the construction as a functor between locally cubical bicategories that preserves monoid objects; this ensures that it also preserves monoidal functors, transformations, adjunctions, and so on.
  Examples include the monoidal bicategories of algebras and bimodules, categories and profunctors, sets and spans, open Markov processes, parametrized spectra, and various functors relating them.
\end{abstract}

\maketitle
\setcounter{tocdepth}{1}
\tableofcontents

\section{Introduction}
\label{sec:introduction}

Symmetric monoidal bicategories are important in many contexts.
However, the definition of even a monoidal bicategory
(see~\cite{gps:tricats,nick:tricats}), let alone a symmetric monoidal
one
(see~\cite{kv:2cat-zam,kv:bm2cat,bn:hda-i,ds:monbi-hopfagbd,crans:centers,mccrudden:bal-coalgb,gurski:brmonbicat}),
a monoidal functor between such (see~\cite{nick:tricatsbook,mccrudden:bal-coalgb}),
or a monoidal transformation or modification (see~\cite{sp:thesis})
is quite imposing, and time-consuming to verify in any example.

In this paper we describe a method for constructing (symmetric) monoidal
bicategories, as well as functors and transformations between them, which is hardly more difficult than constructing a pair
of ordinary (symmetric) monoidal categories.
While not universally applicable, this method applies in many cases of interest.
The underlying idea has often been implicitly used in particular cases, such as
bicategories of enriched profunctors, but to our knowledge the first
general statement was claimed in~\cite[Appendix B]{shulman:frbi}.
In the unpublished~\cite{shulman:smbicat}, the first author worked out the details for the construction of monoidal bicategories themselves.
Here we include that work and build on it further to construct monoidal functors, transformations, and so on between monoidal bicategories as well, making the entire construction into a functor.%
\footnote{See also~\cite[\S5]{gg:ldstr-tricat} which generalizes the construction in a different direction, showing that every sufficiently nice locally cubical bicategory has an underlying tricategory.}

The method relies on the fact that in many bicategories, the 1-cells
are not the most fundamental notion of `morphism' between the objects.
For instance, in the bicategory \cMod\ of rings, bimodules, and
bimodule maps, the more fundamental notion of morphism between objects
is a ring homomorphism. The addition of these extra morphisms promotes
a bicategory to a \emph{double category}, or a category internal to
\cCat.  The extra morphisms are usually stricter than the 1-cells in
the bicategory and easier to deal with for coherence questions; in
many cases it is quite easy to show that we have a \emph{symmetric
  monoidal double category}.  The central observation is that in most
cases (when the natural transformations have `loosely strong companions') we can then `lift' this
symmetric monoidal structure to the original bicategory.  That is, we
prove the following theorem:

\begin{thm}\label{thm:mondbl-monbi-intro}
  If \lD\ is a monoidal double category, of which the monoidal constraints have loosely strong companions, then its underlying bicategory $\cH(\lD)$ is a monoidal bicategory.  If \lD\ is braided
  or symmetric, so is $\cH(\lD)$.
\end{thm}

In~\cite{shulman:smbicat} this theorem was proven by explicitly constructing liftings of all the coherence data, but in the present paper we take a more functorial viewpoint.
Specifically, in addition to our interest in constructing functors and transformations between monoidal bicategories in addition to the monoidal bicategories themselves, we take a more conceptual approach to the proof of \cref{thm:mondbl-monbi-intro}.
We extend the operation $\cH$ that takes a double category to its underlying bicategory to a suitable sort of ``functor'', and show that this functor is product-preserving.
Thus, just as a product-preserving functor between ordinary categories automatically preserves not just internal monoids but also monoid homomorphisms, the functor $\cH$ preserves monoidal objects as well as functors, transformations, and so on between them.
In fact, we will show that $\cH$ induces another functor from a ``category'' of monoidal double categories to one of monoidal bicategories, thereby preserving all kinds of composition as well.

The tricky part is deciding into what kind of categorical structure we should assemble our double categories and bicategories, and thus what kind of functor $\cH$ should be.
To start with, double categories most naturally assemble into a strict 2-category, while bicategories most naturally form a tricategory; and since a 2-category can be considered a degenerate tricategory, we could work with tricategories all the way through.
However, tricategories are really too weak for our purposes.
On the one hand, manipulating all the coherences in a tricategory, let alone a functor between tricategories, is exceedingly difficult.
On the other hand, even the tricategory of bicategories is considerably stricter than an arbitrary tricategory.
In addition to suggesting that stricter alternatives are available, this also means that if we treated bicategories as forming a fully weak tricategory, then an internal notion of ``monoid'' in that tricategory would not coincide exactly with a monoidal bicategory as usually defined, but would have extra unnecessary coherences added, making for yet more work in relating such a definition to the now-accepted one.

The first alternative to tricategories one might consider is what was called in~\cite{shulman:psalg} an \emph{iconic tricategory}.
This is a tricategory-like structure whose coherences for composition along 0-cells are \emph{icons}~\cite{lack:icons} rather than fully general pseudonatural transformations; more precisely it is a bicategory enriched over the monoidal 2-category of bicategories, pseudofunctors, and icons.
Informally, an iconic tricategory is one where composition of 1-cells along 0-cells is strictly associative and unital (though composition of 2-cells along 0-cells need not be).
The tricategory of bicategories is indeed iconic, as of course is the strict 2-category of double categories regarded as a tricategory, and $\cH$ can be made into a product-preserving ``iconic functor'' between them.
However, while the notion of iconic tricategory suffices for our \emph{input} data, it is insufficient for our \emph{output} data: the tricategory of monoidal bicategories is not iconic.

There is, however, a stricter structure than tricategories that does encompass monoidal bicategories: a bicategory enriched over \emph{double} categories, introduced in~\cite{gg:ldstr-tricat}.
In addition to 0-cells and 1-cells, a locally cubical category has \emph{two} kinds of 2-cells (``vertical'' and ``horizontal''), as well as 3-cells inhabiting a square boundary of 2-cells.
Just as a bicategory can be regarded as a double category that is trivial in one direction, an iconic tricategory can be regarded as a locally cubical bicategory --- although in the case of bicategories, it is more natural to take the additional kind of 2-cells to be icons.
Similarly, it is shown in~\cite{gg:ldstr-tricat} that monoidal bicategories form a locally cubical bicategory, with an appropriate notion of ``monoidal icon'' as the additional 2-cells.

We will show that more generally, internal monoids in any\footnote{Actually, for technical simplicity we assume a 1-dimensional strictness property that is satisfied in both our main examples, and should be obtainable by a suitable strictification theorem.} locally cubical bicategory with products form another locally cubical bicategory, and that this reproduces the standard notions of monoidal double category and monoidal bicategory.
Moreover, we will show that any product-preserving functor between locally cubical bicategories with products induces another functor between their locally cubical bicategories of internal monoids (of all sorts).
In particular, this specializes to our desired statement:

\begin{thm}\label{thm:functor-intro}
  The assignment $\cH$ extends to a functor between the locally cubical bicategories of monoidal, braided, and symmetric double categories and bicategories.
  In particular, it preserves monoidal functors and monoidal transformations, and composites thereof.
\end{thm}

In fact, we actually prove several theorems of this sort, depending on whether the monoidal functors and transformations in question are chosen to be \emph{lax}, \emph{colax}, or \emph{strong}, i.e.\ whether they preserve the monoidal structure up to a transformation in one direction, the other direction, or an invertible transformation.
This distinction for functors is already known for ordinary monoidal categories; for monoidal bicategories such a threefold choice is also available for transformations.
Note that this laxity is only relative to the monoidal structure: on the underlying bicategories, all our functors and transformations will be strong/pseudo, preserving composition up to invertible transformations.

One might hope to incorporate both lax and colax functoriality in a single theorem.
For instance, as noted in~\cite{gp:double-adjoints,shulman:dblderived}, lax and colax morphisms \emph{themselves} form the horizontal and vertical morphisms in a double category!
A functoriality theorem at this level would have the advantage of also preserving ``mates'' in this double category, including for instance doctrinal adjunctions~\cite{kelly:doc-adjn} between lax and colax monoidal functors.
However, the kind of 3-dimensional structure that would be needed for such a theorem in our case seemingly does not exist in the literature (though~\cite{gp:intercategories-i} is a step towards it), so we do not pursue it here.

\begin{rmk}
We also expect similar theorems to be true in higher dimensions.  For
instance, Chris Douglas~\cite{douglas:tfttalk} has suggested that many
apparent tricategories are more naturally bicategories internal to
\cCat\ or categories internal to \cTwocat; and in most such cases
arising in practice, we can again `lift' the coherence to give a
tricategory.
%
We propose the term \textbf{$(n\times k)$-category}
(pronounced ``$n$-by-$k$-category'') for an $n$-category internal to
$k$-categories, which has $(n+1)(k+1)$ different types of
cells in an $(n+1)$ by $(k+1)$ grid.  Thus
double categories are \textbf{1x1-categories}, while in
place of tricategories we may consider 2x1-categories and
1x2-categories --- or even 1x1x1-categories, i.e.\ triple categories,
as in~\cite{gp:intercategories-i,gp:intercategories-ii}.%
\footnote{If we also generalize the traditional terminology to say that an $(n+1)$-category is a category \emph{enriched} in $n$-categories even when $n$ is a symbol more general than a natural number, we could say that locally cubical bicategories are ``$((1\times 1)+1)$-categories''.}
Any
$(n\times k)$-category with a suitable lifting property
should have an underlying $(n+k)$-category, but this discards an increasing amount of structure as $n$ and $k$ grow.

% , such as duality and
% trace~\cite{ps:traces} or autonomous
% structure~\cite{ds:monbi-hopfagbd,street:funct-calc,dms:antipodes},

There is a case to be made that often the extra cells should
\emph{not} be discarded.  But sometimes
it really is the underlying $(n+k)$-category one cares about; for
instance, the Baez-Dolan cobordism hypothesis is about the $(n+1)$-category of cobordisms, not
the $(n\times 1)$-category from which it is constructed
(see~\cite{lurie:tft}).  Thus we believe there is an indisputable value to
results such as \autoref{thm:mondbl-monbi-intro} and \autoref{thm:functor-intro}.
\end{rmk}

Proceeding to the contents of this paper, in
\S\ref{sec:symm-mono-double} we review the definition of symmetric
monoidal double categories, and in \S\ref{sec:comp-conj} we recall the
notions of `companion' and `conjoint' whose presence supplies the
necessary lifting property.
(Double categories with companions and conjoints have also been called ``framed bicategories''~\cite{shulman:frbi}, and are roughly equivalent to ``proarrow equipments''~\cite{wood:proarrows-i}).
Then in \S\ref{sec:1x1-to-bicat} we describe a functor from double categories with loosely strong companions to bicategories.
In order to prove that this functor preserves the monoidal structure, we define monoidal structures abstractly for elements of a locally cubical bicategory in \S\ref{sec:mono-objects}, and we show that the monoidal objects and cells form a new locally cubical bicategory. Furthermore, we prove that any product-preserving functor between locally cubical bicategories preserves monoidal objects and cells of all sorts, and indeed induces another functor of locally cubical bicategories.
There is a lot to check here, but the hardest part is writing down all the definitions in the appropriate generality! We specialize this to the functor from double categories with loosely strong companions and conjoints for all transformations to bicategories, yielding functorial constructions of monoidal, braided, and symmetric monoidal bicategories.

\fxnote{Rewrite this paragraph once the paper settles down.}

Finally, in \S\ref{sec:Alg} we illustrate our method by constructing the symmetric monoidal bicategory ${\cA}lg(\bC)$ of algebras, bimodules and bimodule homomorphisms, as well as that of dagger algebras, dagger bimodules, and dagger bimodule homomorphisms, in a monoidal category \bC, varying functorially in \bC.

We would like to thank Peter May, Tom Fiore, Stephan Stolz, Chris
Douglas, Nick Gurski, Jamie Vicary, and Julian Hedges for helpful discussions and comments.

% Local Variables:
% TeX-master: "smbicat"
% End:


\section{Symmetric monoidal double categories}
\label{sec:symm-mono-double}

In this section, we recall basic notions of double categories to fix our terminology and notation, and define monoidal double categories and functors between them in an explicit way.
Double categories go back originally to Ehresmann
in~\cite{ehresmann:cat-str}; a brief introduction can be found
in~\cite{ks:r2cats}.  Other references
include~\cite{multi_funct_i,gp:double-limits,gp:double-adjoints,aleiferi2018cartesian}.


\begin{defn}\label{def:dblcat}
  A \textbf{(pseudo) double category} \lD\ consists of a `category of
  objects' $\lD_0$ and a `category of arrows' $\lD_1$, with structure
  functors
  \begin{align*}
    U&\maps \lD_0\to \lD_1\\
    S,T&\maps \lD_1\rightrightarrows \lD_0\\
    \odot&\maps \lD_1\times_{\lD_0}\lD_1\to \lD_1
  \end{align*}
  (where the pullback is over
  $\lD_1\too[T]\lD_0\overset{S}{\longleftarrow} \lD_1$) such that
  \begin{alignat*}{2}
    S(U_A) &= A &\qquad
    S(M\odot N) &= SN\\
    T(U_A) &= A &\qquad
    T(M\odot N) &= TM
  \end{alignat*}
  naturally, and equipped with natural isomorphisms
  \begin{align*}
    \fa &: (M\odot N) \odot P \too[\iso] M \odot (N \odot P)\\
    \fl &: U_B \odot M \too[\iso] M\\
    \fr &: M \odot U_A \too[\iso] M
  \end{align*}
  such that $S(\fa)$, $T(\fa)$, $S(\fl)$, $T(\fl)$, $S(\fr)$, and
  $T(\fr)$ are all identities, and such that the standard coherence
  axioms for a monoidal category or bicategory (such as Mac Lane's
  pentagon; see~\cite{maclane}) are satisfied.
\end{defn}

Just as a bicategory can be thought of as a category weakly
\emph{enriched} over \cCat, a pseudo double category can be thought of
as a category weakly \emph{internal} to \cCat.  Since these are the
kind of double categories of most interest to us, we will usually drop
the adjective ``pseudo.''

We call the objects of $\lD_0$ \textbf{objects} or \textbf{0-cells},
and we call the morphisms of $\lD_0$ \textbf{tight 1-morphisms}
and write them as $f\maps A\to B$.  We call the objects of $\lD_1$
\textbf{loose 1-cells}; if $M$ is a 1-cell with $S(M)=A$ and
$T(M)=B$, we write $M\maps A\hto B$.  We call a morphism $\alpha\maps
M\to N$ of $\lD_1$ with $S(\alpha)=f$ and $T(\alpha)=g$ a
\textbf{2-morphism} and draw it as follows:
\begin{equation}\label{eq:square}
  \xymatrix@-.5pc{
    A \ar[r]|{|}^{M}  \ar[d]_f \ar@{}[dr]|{\Downarrow\alpha}&
    B\ar[d]^g\\
    C \ar[r]|{|}_N & D
  }.
\end{equation}


Note that composition of tight 1-morphisms is strictly associative and unital, while that of loose 1-cells is only weakly so.
The words ``tight'' and ``loose'', borrowed from~\cite{ls:limlax}, are intended to suggest one kind of morphism that is ``stricter'' than another.
Traditionally the two kinds of 1-arrow in a double category are called ``vertical'' and ``horizontal'' with reference to how they are drawn, but this creates confusion because some authors draw the tight morphisms (those with strictly associative composition) vertically and others draw them horizontally.
We usually draw the tight 1-morphisms vertically and the loose 1-cells horizontally, but the terminology ``tight'' and ``loose'' unambiguously identifies which arrows we are talking about, independently of our conventions about how to draw pictures.%
\footnote{In~\cite{shulman:smbicat} a similar effect was intended by simply distinguishing between ``1-morphisms'' (the tight ones) and ``1-cells'' (the loose ones), but since ``morphism'' and ``cell'' (and ``arrow'') are generally used interchangeably in higher category theory this was not very successful.}

We write the composition of tight 1-morphisms $A\too[f] B\too[g] C$
and the tight composition of 2-morphisms $M\too[\alpha]
N\too[\beta] P$ as $g\circ f$ and $\beta\circ\alpha$, or sometimes
just $gf$ and $\beta\alpha$.  We write the loose composition of
1-cells $A\xhto{M} B \xhto{N} C$ as $A\xhto{N\odot M} C$ and that of
2-morphisms
\[\vcenter{\xymatrix{ \ar[r]|-@{|}^-{} \ar[d] \ar@{}[dr]|{\Downarrow\alpha} &
     \ar[r]|-@{|}^-{} \ar[d] \ar@{}[dr]|{\Downarrow\beta} &\ar[d]\\
  \ar[r]|-@{|}_-{} & \ar[r]|-@{|}_-{} & }}\]
as
\[\vcenter{\xymatrix@C=4pc{ \ar[r]|-@{|}^-{} \ar[d] \ar@{}[dr]|{\Downarrow\;\be\odot\al} &  \ar[d]\\
  \ar[r]|-@{|}_-{} & }}\]


The two different compositions of 2-morphisms obey an interchange law,
by the functoriality of $\odot$:
\[(M_1\odot M_2) \circ (N_1\odot N_2) = (M_1\circ N_1)\odot (M_2\circ N_2).
\]
Every object $A$ has a tight identity $1_A$ and a loose unit
$U_A$, every loose 1-cell $M$ has an identity 2-morphism $1_M$, every
tight 1-morphism $f$ has a loose unit 2-morphism $U_f$, and we
have $1_{U_A} = U_{1_A}$ (by the functoriality of $U$).

Note that the tight composition $\circ$ is strictly associative and
unital, while the loose one $\odot$ is only weakly so.  This is
the case in the majority of examples, and is part of what makes monoidal double categories so much easier to construct than monoidal bicategories.
(It is, however, possible to define double categories that are weak in both directions~\cite{verity:base-change}.)

% \begin{rmk}\label{rmk:monglob}
%   In general, an $(n\times 1)$-category consists of 1-categories
%   $\lD_i$ for $0\le i\le n$, together with source, target, unit, and
%   composition functors and coherence isomorphisms.  We refer to the
%   objects of $\lD_i$ as \textbf{$i$-cells} and to the morphisms of
%   $\lD_i$ as \textbf{morphisms of $i$-cells} or \textbf{(tight)
%     $(i+1)$-morphisms}.  A formal definition can be found
%   in~\cite{batanin:monglob} under the name \emph{monoidal $n$-globular
%     category}.
% \end{rmk}

% We call $\lD_0$ the \textbf{vertical category} of \lD.  We say that two
% objects are isomorphic if they are isomorphic in $\lD_0$, and that two
% horizontal 1-cells are isomorphic if they are isomorphic in $\lD_1$.
% We will never refer to a horizontal 1-cell as an isomorphism.

A 2-morphism~\eqref{eq:square} where $f$ and $g$ are identities (such
as the constraint isomorphisms $\fa,\fl,\fr$) is called
\textbf{globular}.  Every double category \lD\ has a
\textbf{loose bicategory} $\cH(\lD)$ consisting of the objects,
loose 1-cells, and globular 2-morphisms.  In the literature, this is often called the ``horizontal'' or ``vertical'' bicategory of a double category (depending on conventions). Conversely, many naturally
occurring bicategories are actually the loose bicategory of some
naturally ocurring double category.  Here are just a few examples.

\begin{eg}
  The double category \lnCob\ has as objects closed $n$-manifolds, as
  tight 1-morphisms diffeomorphisms, as loose 1-cells cobordisms, and as
  2-morphisms diffeomorphisms between cobordisms.  Again $\cH(\lnCob)$
  is the usual bicategory of cobordisms.
\end{eg}

\begin{eg}
  The double category \lMod\ has as objects rings, as tight 1-morphisms ring
  homomorphisms, as loose 1-cells bimodules, and as 2-morphisms equivariant
  bimodule maps.  Its loose bicategory $\cMod = \cH(\lMod)$ is
  the usual bicategory of rings and bimodules. 
\end{eg}


\begin{eg}
  The double category \lProf\ has as objects categories, as
  tight 1-morphisms functors, as loose 1-cells \emph{profunctors} (a profunctor
  $A\hto B$ is a functor $B\op\times A\to \mathbf{Set}$), and as
  2-morphisms natural transformations.  Bicategories such as
  $\cH(\lProf)$ are commonly encountered in category theory,
  especially the enriched versions.
\end{eg}

In Section ~\ref{sec:Alg} we will discuss the symmetric monoidal bicategories categories, \lMod\, \lProf\ amongst other applications.

% If $A$ and $B$ are objects of \lD, we write $\lD(A,B)$ for the
% \emph{set} of vertical arrows from $A$ to $B$ and $\calD(A,B)$ for the
% \emph{category} of horizontal 1-cells and globular 2-cells from $A$ to
% $B$.  It is standard in bicategory theory to say that something holds
% \textbf{locally} when it is true of all hom-categories $\calD(A,B)$, and
% we will extend this usage to double categories.


As opposed to bicategories, which naturally form a tricategory, double
categories naturally form a \emph{2-category}, a much simpler object.

\begin{defn}
  Let \lD\ and \lE\ be double categories.  A \textbf{(pseudo double)
    functor} $F \maps \lD\to \lE$ consists of the following.
  \begin{itemize}
  \item Functors $F_0\maps \lD_0 \to \lE_0$ and $F_1\maps \lD_1 \to
    \lE_1$ such that $S\circ F_1 = F_0\circ S$ and $T\circ F_1 =
    F_0\circ T$.
  \item Natural transformations $F_\odot\maps F_1M \odot F_1N \to
    F_1(M\odot N)$ and $F_U\maps U_{F_0 A} \to F_1(U_A)$, whose
    components are globular isomorphisms, and which satisfy the usual
    coherence axioms for a monoidal functor or pseudofunctor
    (see~\cite[\S{}XI.2]{maclane}).
  \end{itemize}
\end{defn}

\begin{defn}\label{thm:dbl-transf}
  A \textbf{(tight) transformation} between two functors $\alpha:
  F\to G:\lD\to\lE$ consists of natural transformations $\alpha_0\maps
  F_0\to G_0$ and $\alpha_1\maps F_1\to G_1$ (both usually written as
  $\alpha$), such that $S(\alpha_{M}) = \alpha_{SM}$ and
  $T(\alpha_{M}) = \alpha_{TM}$, and such that
  \[\vcenter{\xymatrix@-.5pc{
      FA \ar@{=}[d] \ar[r]|{|}^{FM}
      \ar@{}[drr]|{\Downarrow F_\odot} &
      FB \ar[r]|{|}^{FN} &
      FC \ar@{=}[d]\\
      FA \ar[rr]|{F(N\odot M)} \ar[d]_{\alpha_A}
      \ar@{}[drr]|{\Downarrow \alpha_{N\odot M}} &&
      FC \ar[d]^{\alpha_C}\\
      GA \ar[rr]|{|}_{G(N\odot M)} && GC
    }} =
  \vcenter{\xymatrix@-.5pc{
      FA \ar[d]_{\alpha_A} \ar@{}[dr]|{\Downarrow \alpha_M} \ar[r]|{|}^{FM} &
      FB \ar[d]|{\alpha_B} \ar@{}[dr]|{\Downarrow \alpha_N} \ar[r]|{|}^{FN} &
      FC \ar[d]^{\alpha_C}\\
      GA \ar@{=}[d] \ar[r]|{|}_{GM} \ar@{}[drr]|{\Downarrow G_\odot} &
      GB \ar[r]|{|}_{GN} &
      GC \ar@{=}[d]\\
      GA \ar[rr]|{|}_{G(N\odot M)} && GC
    }}\]
  for all 1-cells $M\colon A\hto B$ and $N\colon B\hto C$, and
  \[\vcenter{\xymatrix@-.5pc{
      FA \ar[rr]|{|}^{U_{FA}} \ar@{=}[d]
      \ar@{}[drr]|{\Downarrow F_U} &&
      FA \ar@{=}[d]\\
      FA \ar[rr]|{F(U_A)} \ar[d]_{\alpha_A}
      \ar@{}[drr]|{\Downarrow \alpha_{U_A}} &&
      FA \ar[d]^{\alpha_A}\\
      GA \ar[rr]|{|}_{G(U_A)} && GA
    }} =
  \vcenter{\xymatrix@-.5pc{
      FA \ar[rr]|{|}^{U_{FA}} \ar[d]_{\alpha_A}
      \ar@{}[drr]|{\Downarrow U_{\alpha_A}} &&
      FA \ar[d]^{\alpha_A}\\
      GA \ar[rr]|{U_{GA}} \ar@{=}[d]
      \ar@{}[drr]|{\Downarrow G_U} &&
      GA \ar@{=}[d]\\
      GA \ar[rr]|{|}_{G(U_A)} && GA.
    }}\]
  for all objects $A$.
\end{defn}

We write \cDbl\ for the strict 2-category of double categories, functors, and transformations, and $\mathbf{Dbl}$ for its underlying 1-category. As pseudofunctors compose strictly associatively, this is well-defined. Note that a 2-cell $\al$ in \cDbl\ is an isomorphism just when each
$\al_A$, \emph{and} each $\al_M$, is invertible.

The 2-category \cDbl\ gives us an easy way to define what we mean by a
\emph{symmetric monoidal double category}.
We say that a strict 2-category $\cK$ has \emph{finite products} if it has a strictly terminal object, i.e.\ an object $1$ such that each category $\cK(D,1)$ is \emph{isomorphic} to the terminal category, and any two objects $D,E$ have a strict product, i.e.\ a span $D \ot D\times E \to E$ such that the induced functors $\cK(X,D\times E) \to \cK(X,D) \times \cK(X,E)$ is an \emph{isomorphism} of categories.
In any such 2-category with finite products there is a notion of a \emph{pseudomonoid} (perhaps braided or symmetric), which generalizes the notion of monoidal category in \cCat.
We omit the general definition, since we will give a more general one in a later section;\fxnote[author=MS]{update reference}
for now we just specialize it to \cDbl\ and obtain the following.

\begin{defn}\label{def:symmondoub}
  A \textbf{monoidal double category} is a double category equipped
  with functors $\ten\maps \lD\times\lD\to\lD$ and $I\maps * \to\lD$,
  and invertible transformations
  \begin{align*}
    \mathord{\otimes} \circ (\Id\times \mathord{\otimes})
    &\iso \mathord{\otimes} \circ (\mathord{\otimes} \times \Id)\\
    \mathord{\otimes} \circ (\Id\times I) &\iso \Id\\
    \mathord{\otimes} \circ (I\times \Id) &\iso \Id
  \end{align*}
  satisfying the usual axioms.  If it additionally has a braiding
  isomorphism
  \begin{align*}
    \mathord{\otimes} &\iso \mathord{\otimes} \circ \tau
  \end{align*}
  (where $\tau\maps \lD\times\lD\iso \lD\times\lD$ is the twist)
  satisfying the usual axioms, then it is \textbf{braided} or
  \textbf{symmetric}, according to whether or not the braiding is
  self-inverse.
\end{defn}

Unpacking this definition more explicitly, we see that a monoidal
double category is a double category together with the following
structure.
\begin{enumerate}
\item $\lD_0$ and $\lD_1$ are both monoidal categories.
\item If $I$ is the monoidal unit of $\lD_0$, then $U_I$ is the
  monoidal unit of $\lD_1$.\footnote{Actually, all the above
    definition requires is that $U_I$ is coherently \emph{isomorphic
      to} the monoidal unit of $\lD_1$, but we can always choose them
    to be equal without changing the rest of the structure.
    More precisely, we may choose the unit isomorphism
    $I_U : U_{I_{\lD_0}} \to I_{\lD_1}$ of the functor $I:* \to \lD$
    to be an identity.}
\item The functors $S$ and $T$ are strict monoidal, i.e.\ $S(M\ten N)
  = SM\ten SN$ and $T(M\ten N)=TM\ten TN$ and $S$ and $T$ also
  preserve the associativity and unit constraints.
\item \label{eq:mon1} We have globular isomorphisms derived from $\otimes_{\odot}$ and $\otimes_U$
  \[\fx\maps (M_1\ten N_1)\odot (M_2\ten N_2)\too[\iso] (M_1\odot M_2)\ten (N_1\odot N_2)\]
  and
  \[\fu\maps U_{A\ten B} \too[\iso] (U_A \ten U_B)\]
  such that the following diagrams commute, expressing that $\ten$ is a pseudo double functor:
\begin{equation}\label{eq:mondoub1}
\begin{aligned}
\begin{tikzpicture}[xscale=1.8, yscale=1.5]
\node (tl) at (0,2) {$((M_1 \tens N_1)\odot (M_2 \tens N_2)) \odot (M_3 \tens N_3)$};
\node (tr) at (4,2) {$((M_1 \odot M_2) \tens (N_1 \odot N_2)) \odot (M_3 \tens N_3)$};
\node (ml) at (0,1) {$(M_1 \tens N_1) \odot ((M_2 \tens N_2) \odot (M_3 \tens N_3))$};
\node (mr) at (4,1) {$((M_1 \odot M_2) \odot M_3) \tens ((N_1 \odot N_2) \odot N_3)$};
\node (bl) at (0,0) {$(M_1 \tens N_1) \odot ((M_2 \odot M_3) \tens (N_2 \odot N_3))$};
\node (br) at (4,0) {$(M_1 \odot (M_2 \odot M_3)) \tens (N_1 \odot (N_2 \odot N_3))$};
\draw[->] (tl) to node[above] {$\fx \odot \id$} (tr);
\draw[->] (tl) to node[left]{$\fa$} (ml);
\draw[->] (ml) to node[left]{$\id\odot \fx$} (bl);
\draw[->] (tr) to node[left]{$\fx$} (mr);
\draw[->] (mr) to node[left]{$\fa \odot \fa$} (br);
\draw[->] (bl) to node[above] {$\fx$} (br);
\end{tikzpicture}
    \end{aligned}
\end{equation}

%\begin{equation}\label{eq:mondoub1}
%\begin{aligned}
%  \xymatrix{
%    ((M_1\ten N_1)\odot (M_2\ten N_2)) \odot (M_3\ten N_3) \ar[r]\ar[d]
%    & ((M_1\odot M_2)\ten (N_1\odot N_2)) \odot (M_3\ten N_3) \ar[d]\\
%    (M_1\ten N_1)\odot ((M_2\ten N_2) \odot (M_3\ten N_3)) \ar[d] &
%    ((M_1\odot M_2)\odot M_3) \ten ((N_1\odot N_2)\odot N_3) \ar[d]\\
%    (M_1\ten N_1) \odot ((M_2\odot M_3) \ten (N_2\odot N_3))\ar[r] &
%    (M_1\odot (M_2\odot M_3)) \ten (N_1\odot (N_2\odot N_3))}
%    \end{aligned}
%\end{equation}

\begin{equation}\label{eq:mondoub2}
\begin{aligned}
\begin{tikzpicture}[xscale=2, yscale=1.5]
\node (tl) at (0,2) {$(M \tens N) \odot U_{C \tens D}$};
\node (tr) at (4,2) {$(M \tens N) \odot ( U_C \tens U_D)$};
\node (ml) at (0,1) {$M \tens N$};
\node (mr) at (4,1) {$(M \odot U_C) \tens (N \odot U_D)$};
\draw[->] (tl) to node[above] {$\id \odot \fu$} (tr);
\draw[->] (tl) to node[left]{$\fr$} (ml);
\draw[->] (tr) to node[left]{$\fx$} (mr);
\draw[->] (mr) to node[above] {$\fr \tens \fr$} (ml);
\end{tikzpicture}
    \end{aligned}
\end{equation}

%\begin{equation}\label{eq:mondoub2}
%\begin{aligned}    
%  \xymatrix{(M\ten N) \odot U_{C\ten D} \ar[r]\ar[d] &
%    (M\ten N)\odot (U_C\ten U_D) \ar[d]\\
%    M\ten N\ar@{<-}[r] & (M\odot U_C) \ten (N\odot U_D)}
%        \end{aligned}
%    \end{equation}
    
    \begin{equation}\label{eq:mondoub3}
\begin{aligned}
\begin{tikzpicture}[xscale=2, yscale=1.5]
\node (tl) at (0,2) {$U_{A \tens B} \odot (M \tens N) $};
\node (tr) at (4,2) {$(U_A \tens U_B) \odot (M \tens N) $};
\node (ml) at (0,1) {$M \tens N$};
\node (mr) at (4,1) {$(U_A \odot M ) \tens (U_B \odot N)$};
\draw[->] (tl) to node[above] {$\fu \odot \id$} (tr);
\draw[->] (tl) to node[left]{$\fl$} (ml);
\draw[->] (tr) to node[left]{$\fx$} (mr);
\draw[->] (mr) to node[above] {$\fl \tens \fl$} (ml);
\end{tikzpicture}
    \end{aligned}
\end{equation}

%    \begin{equation}\label{eq:mondoub3}
%    \begin{aligned}
%  \xymatrix{U_{A\ten B}\odot (M\ten N)  \ar[r]\ar[d] &
 %   (U_A\ten U_B)\odot (M\ten N) \ar[d]\\
 %   M\ten N\ar@{<-}[r] & (U_A \odot M) \ten (U_B\odot N)}
 %       \end{aligned}
%    \end{equation}
\item \label{eq:mon2}The following diagrams commute, expressing that the
  associativity isomorphism for $\ten$ is a transformation of double
  categories.
\begin{equation}\label{eq:mondoub4}
\begin{aligned}
\begin{tikzpicture}[xscale=1.8, yscale=1.5]
\node (tl) at (0,2) {$((M_1 \tens N_1) \tens P_1) \odot ((M_2 \tens N_2) \tens P_2) $};
\node (tr) at (4,2) {$(M_1 \tens (N_1 \tens P_1)) \odot (M_2 \tens (N_2 \tens P_2))$};
\node (ml) at (0,1) {$((M_1 \tens N_1) \odot (M_2 \tens N_2)) \tens (P_1 \odot P_2)$};
\node (mr) at (4,1) {$(M_1 \odot M_2)  \tens ((N_1 \tens P_1) \odot (N_2 \tens P_2))$};
\node (bl) at (0,0) {$((M_1 \odot M_2) \tens (N_1 \odot N_2)) \tens (P_1 \odot P_2)$};
\node (br) at (4,0) {$(M_1 \odot M_2) \tens ((N_1 \odot N_2) \tens (P_1 \odot P_2))$};
\draw[->] (tl) to node[above] {$\fa \odot \fa$} (tr);
\draw[->] (tl) to node[left]{$\fx$} (ml);
\draw[->] (ml) to node[left]{$ \fx \tens \id$} (bl);
\draw[->] (tr) to node[left]{$\fx$} (mr);
\draw[->] (mr) to node[left]{$\id \tens \fx$} (br);
\draw[->] (bl) to node[above] {$\fa$} (br);
\end{tikzpicture}
    \end{aligned}
\end{equation}
  
%     {\small 
%\begin{equation}\label{eq:mondoub4}
%\begin{aligned}
%  \xymatrix{
%    ((M_1\ten N_1)\ten P_1) \odot ((M_2\ten N_2)\ten P_2) \ar[r]\ar[d] &
 %   (M_1\ten (N_1\ten P_1)) \odot (M_2\ten (N_2\ten P_2)) \ar[d]\\
    %((M_1\ten N_1) \odot (M_2\ten N_2)) \ten (P_1\odot P_2) \ar[d] &
%    (M_1\odot M_2) \ten ((N_1\ten P_1)\odot (N_2\ten P_2))\ar[d] \\
   % ((M_1\odot M_2) \ten(N_1\odot N_2)) \ten (P_1\odot P_2) \ar[r] &
%    (M_1\odot M_2) \ten ((N_1\odot N_2)\ten (P_1\odot P_2))}
   %     \end{aligned}
      %  \end{equation}
        
        \begin{equation}\label{eq:mondoub5}
\begin{aligned}
\begin{tikzpicture}[xscale=1.8, yscale=1.5]
\node (tl) at (0,2) {$U_{(A \tens B) \tens C}$};
\node (tr) at (4,2) {$U_{A \tens (B \tens C)}$};
\node (ml) at (0,1) {$U_{A \tens B} \tens U_C$};
\node (mr) at (4,1) {$U_A \tens U_{B \tens C}$};
\node (bl) at (0,0) {$(U_A \tens U_B) \tens U_C$};
\node (br) at (4,0) {$U_A \tens (U_B \tens U_C)$};
\draw[->] (tl) to node[above] {$U_{\alpha_{A,B,C}}$} (tr);
\draw[->] (tl) to node[left]{$\fu$} (ml);
\draw[->] (ml) to node[left]{$\fu \tens \id$} (bl);
\draw[->] (tr) to node[left]{$\fu$} (mr);
\draw[->] (mr) to node[left]{$\id \tens \fu$} (br);
\draw[->] (bl) to node[above] {$\alpha_{U_A, U_B, U_C}$} (br);
\end{tikzpicture}
    \end{aligned}
\end{equation}
%    \begin{equation}\label{eq:mondoub5}
  %  \begin{aligned}
  %\xymatrix{
%    U_{(A\ten B)\ten C} \ar[r] \ar[d] & U_{A\ten (B\ten C)} \ar[d]\\
   % U_{A\ten B} \ten U_C \ar[d] & U_A\ten U_{B\ten C}\ar[d]\\
%    (U_A\ten U_B)\ten U_C \ar[r] & U_A\ten (U_B\ten U_C) }
   %     \end{aligned}
   % \end{equation}
    %}

\item \label{eq:mon3}The following diagrams commute, expressing that the unit
  isomorphisms for $\ten$ are transformations of double categories.
 \begin{equation}\label{eq:mondoub6}
\begin{aligned}
\begin{tikzpicture}[xscale=2, yscale=1.5]
\node (tl) at (0,2) {$(M \tens U_I) \odot (N \tens U_I) $};
\node (tr) at (4,2) {$(M \odot N) \tens ( U_I \odot U_I)$};
\node (ml) at (0,1) {$M \odot N$};
\node (mr) at (4,1) {$(M \odot N) \tens U_I$};
\draw[->] (tl) to node[above] {$\fx$} (tr);
\draw[->] (tl) to node[left]{$\rho_M \odot \rho_M$} (ml);
\draw[->] (tr) to node[left]{$\id \tens \rho_{U_I}$} (mr);
\draw[->] (mr) to node[above] {$\rho_{M \odot N}$} (ml);
\end{tikzpicture}
    \end{aligned}
\end{equation}
      
      \begin{equation}\label{eq:mondoub7}
\begin{aligned}
\begin{tikzpicture}[yscale=1.5]
\node (tl) at (0,2) {$U_{A\tens I}$};
\node (tr) at (4,2) {$U_A \tens  U_I$};
\node (mr) at (4,1) {$U_A$};
\draw[->] (tl) to node[above]{$\fu$} (tr);
\draw[->] (tl) to node[left]{$U_{\rho_A}$} (mr);
\draw[->] (tr) to node[right]{$\rho_{U_A}$} (mr);
\end{tikzpicture}
    \end{aligned}
\end{equation}

 \begin{equation}\label{eq:mondoub8}
\begin{aligned}
\begin{tikzpicture}[xscale=2, yscale=1.5]
\node (tl) at (0,2) {$(U_I \tens M) \odot ( U_I \tens N) $};
\node (tr) at (4,2) {$( U_I \odot U_I) \tens (M \odot N) $};
\node (ml) at (0,1) {$M \odot N$};
\node (mr) at (4,1) {$U_I \tens (M \odot N) $};
\draw[->] (tl) to node[above] {$\fx$} (tr);
\draw[->] (tl) to node[left]{$\lambda_M \odot \lambda_N$} (ml);
\draw[->] (tr) to node[left]{$\lambda_{U_I} \tens \id$} (mr);
\draw[->] (mr) to node[above] {$\lambda_{M \odot N}$} (ml);
\end{tikzpicture}
    \end{aligned}
\end{equation}
      
      \begin{equation}\label{eq:mondoub9}
\begin{aligned}
\begin{tikzpicture}[yscale=1.5]
\node (tl) at (0,2) {$U_{I\tens A}$};
\node (tr) at (4,2) {$U_I \tens  U_A$};
\node (mr) at (4,1) {$U_A$};
\draw[->] (tl) to node[above]{$\fu$} (tr);
\draw[->] (tl) to node[left]{$U_{\lambda_A}$} (mr);
\draw[->] (tr) to node[right]{$\lambda_{U_A}$} (mr);
\end{tikzpicture}
    \end{aligned}
\end{equation}

  \setcounter{mondbl}{\value{enumi}}
\end{enumerate}
Similarly, a \textbf{braided monoidal double category} is a monoidal double
category with the following additional structure.
\begin{enumerate}\setcounter{enumi}{\value{mondbl}}
\item $\lD_0$ and $\lD_1$ are braided monoidal categories.
\item The functors $S$ and $T$ are strict braided monoidal (i.e.\ they
  preserve the braidings).
\item \label{eq:braid1} The following diagrams commute, expressing that the braiding is
  a transformation of double categories. 
  \begin{equation}\label{eq:brmondoub1}
\begin{aligned}
\begin{tikzpicture}[xscale=2, yscale=1.5]
\node (tl) at (0,2) {$(M_1 \odot M_2) \tens (N_1 \odot N_2)$};
\node (tr) at (4,2) {$(N_1 \odot N_2) \tens (M_1 \odot M_2)$};
\node (ml) at (0,1) {$(M_1 \tens N_1) \odot (M_2 \tens N_2)$};
\node (mr) at (4,1) {$(N_1 \tens M_1) \odot (N_2 \tens M_2)$};
\draw[->] (tl) to node[above] {$\fs$} (tr);
\draw[->] (tl) to node[left]{$\fx$} (ml);
\draw[->] (tr) to node[left]{$\fx$} (mr);
\draw[->] (ml) to node[above] {$\fs \odot \fs$} (mr);
\end{tikzpicture}
    \end{aligned}
\end{equation}
  %\[\xymatrix{(M_1\odot M_2)\ten (N_1\odot N_2) \ar[r]^\fs\ar[d]_\fx &
  %  (N_1\odot N_2)\ten (M_1 \odot M_2)\ar[d]^\fx\\
   % (M_1\ten N_1)\odot (M_2\ten N_2) \ar[r]_{\fs\odot \fs} &
  %  (N_1\ten M_1) \odot (N_2 \ten M_2)}
 % \]
 \begin{equation}\label{eq:brmondoub2}
\begin{aligned}
\begin{tikzpicture}[xscale=2, yscale=1.5]
\node (tl) at (0,2) {$U_A \tens U_B$};
\node (tr) at (4,2) {$U_{A \tens B}$};
\node (ml) at (0,1) {$U_B \tens U_A$};
\node (mr) at (4,1) {$U_{B \tens A}$};
\draw[->] (tl) to node[above] {$\fu$} (tr);
\draw[->] (tl) to node[left]{$\fs$} (ml);
\draw[->] (tr) to node[left]{$U_{\fs}$} (mr);
\draw[->] (ml) to node[above] {$\fu$} (mr);
\end{tikzpicture}
    \end{aligned}
\end{equation}

 % \[\xymatrix{U_A \ten U_B \ar[r]^(0.55)\fu \ar[d]_\fs &
  %  U_{A\ten B} \ar[d]^{U_\fs}\\
  %  U_B\ten U_A \ar[r]_(0.55)\fu &
  %  U_{B\ten A}}.
  %\]
  \setcounter{mondbl}{\value{enumi}}
\end{enumerate}
Finally, a \textbf{symmetric monoidal double category} is a braided one such that
\begin{enumerate}\setcounter{enumi}{\value{mondbl}}
\item $\lD_0$ and $\lD_1$ are in fact symmetric monoidal.
\end{enumerate}
While there are a fair number of coherence diagrams in this definition, most of
them are fairly small, and in any given case most or all of them are
fairly obvious.  Thus, verifying that a given double category is
(braided or symmetric) monoidal is not a great deal of work.

%\fxnote[author=LW]{If we merge the examples with the final section, this should be included.}
\begin{eg}
  The examples \lMod, \lnCob, and \lProf\ are all easily seen to be
  symmetric monoidal under the tensor product of rings, disjoint union
  of manifolds, and cartesian product of categories, respectively.
\end{eg}

\begin{rmk}
  In a 2-category with finite products there is additionally the
  notion of a \emph{cartesian object}: one such that the diagonal
  $D\to D\times D$ and projection $D\to 1$ have right adjoints.  Any
  cartesian object is a symmetric pseudomonoid in a canonical way,
  just as any category with finite products is a monoidal category
  with its cartesian product.  Many of the ``cartesian bicategories''
  considered in~\cite{cw:cart-bicats-i,ckww:cartbicats-ii} are in
  fact the loose bicategory of some cartesian object in \cDbl,
  and inherit their monoidal structure in this way.
\end{rmk}

Two further general methods for constructing symmetric monoidal double
categories can be found in~\cite{shulman:frbi}.

\begin{rmk}
  The general yoga of internalization says that an $X$ internal to
  $Y$s internal to $Z$s is equivalent to a $Y$ internal to $X$s
  internal to $Z$s, but this is only strictly true when the
  internalizations are all strict.  We have defined a symmetric
  monoidal double category to be a (pseudo) symmetric monoid internal
  to (pseudo) categories internal to categories, but one could also
  consider a (pseudo) category internal to (pseudo) symmetric monoids
  internal to categories, i.e.\ a pseudo internal category in the
  2-category
  $\mathcal{S}\mathit{ym}\mathcal{M}\mathit{on}\mathcal{C}\mathit{at}$
  of symmetric monoidal categories and strong symmetric monoidal
  functors.  This would give \emph{almost} the same definition, except
  that $S$ and $T$ would only be strong monoidal (preserving $\ten$ up
  to isomorphism) rather than strict monoidal.  We prefer our
  definition, since $S$ and $T$ are strict monoidal in almost all
  examples, and keeping track of their constraints would be tedious.
\end{rmk}

Just as every bicategory is equivalent to a strict 2-category, it is
proven in~\cite{gp:double-limits} that every pseudo double category is
equivalent to a strict double category (one in which the associativity
and unit constraints for $\odot$ are identities).  Thus, from now on
we will usually omit to write these constraint isomorphisms (or
equivalently, implicitly strictify our double categories).  We
\emph{will} continue to write the constraint isomorphisms for the
monoidal structure $\ten$, since these are where the whole question
lies.

We now move on to define functors and transformations of monoidal double categories.
Like monoidal double categories themselves, these are also special cases of a notion that makes sense internal to any 2-category with products.

\begin{defn}\label{def:monfunc}
  Let $\D$, $\E$ be (braided/symmetric) monoidal double categories.  A {\bf (braided or symmetric) lax monoidal double functor} $F: \D \rightarrow \E$ is a pseudo double functor $F$, together with transformations $\phi : \otimes \circ (F,F) \rightarrow F \circ \otimes$ and $\phi_u:I_{\E}\rightarrow F \circ I_{\D}$ satisfying the usual axioms for (braided/symmetric) monoidal functors with respect to $\otimes$.
\end{defn}

Unfolding the definitions gives us:

\begin{enumerate}
\item $F_0$ and $F_1$ are (braided/symmetric) monoidal functors.
\item The equalities $F_0 \circ S_\D = S_\E \circ F_1$ and $F_0 \circ T_\D = T_\E \circ F_1$ are strict equalities of monoidal functors.
\item The following diagrams commute, expressing that $\phi$ is a transformation of double categories:

\begin{tikzpicture}[xscale=2, yscale=2]
\node (tl) at (0,2) {$(FN \otimes FL) \odot (FM \otimes FK)$};
\node (tr) at (4,2) {$F(N\otimes L) \odot F(M \otimes K)$};
\draw[->] (tl) to node[above] {$\phi \odot \phi$} (tr);
\node (ml) at (0,1) {$(FN \odot FM) \otimes (FL \odot FK)$};
\node (mr) at (4,1) {$F((N \otimes L) \odot (M \otimes K))$};
\node (bl) at (0,0) {$F(N \odot M) \otimes F(L \odot K)$};
\node (br) at (4,0) {$F((N \odot M)\otimes(L\odot K))$};
\draw[->] (tl) to node[left]{$\xi$} (ml);
\draw[->] (ml) to node[left]{$F_{\odot} \otimes F_{\odot}$} (bl);
\draw[->] (tr) to node[left]{$F_{\odot}$} (mr);
\draw[->] (mr) to node[left]{$F(\xi)$} (br);
\draw[->] (bl) to node[above] {$\phi$} (br);
\end{tikzpicture}

\begin{tikzpicture}[yscale=2]
\node (tl) at (0,2) {$U_{FA \otimes FB}$};
\node (tr) at (4,2) {$U_{F(A \otimes B)} $};
\draw[->] (tl) to node[above] {$U_{\phi}$} (tr);
\node (ml) at (0,1) {$U_{FA} \otimes U_{FB}$};
\node (mr) at (4,1) {$F(U_{A\otimes B}) $};
\node (bl) at (0,0) {$F(U_A) \otimes F(U_B)$};
\node (br) at (4,0) {$F(U_A \otimes U_B)$};
\draw[->] (tl) to node[left]{$u$} (ml);
\draw[->] (ml) to node[left]{$F_u \otimes F_u$} (bl);
\draw[->] (tr) to node[left]{$F_U$} (mr);
\draw[->] (mr) to node[left]{$F \circ u$} (br);
\draw[->] (bl) to node[above] {$\phi$} (br);
\end{tikzpicture}

\end{enumerate}

When the natural transformations are in the opposite direction, the functor is {\bf colax monoidal}, and when they are isomorphisms, the functor is {\bf strong monoidal}.

\begin{defn}\label{Def:monverttrans}
  Let $\D$, $\E$ be monoidal double categories and let $(F, \phi) ,(G,\psi): \D \rightarrow \E$ be monoidal double functors. A \textbf{monoidal transformation} $\alpha: F \rightarrow G$ is a tight transformation such that $\alpha_0$ and $\alpha_1$ are monoidal natural transformations.
  Explicitly, this means (in the lax case) that the following equalities hold:

\begin{equation}
\begin{aligned}
\begin{tikzpicture}
\node (tl) at (0,4) {$FA \otimes FB$};
\node (tr) at (4,4) {$FC \otimes FD$};
\node (ml) at (0,2) {$F(A\otimes B)$};
\node (mr) at (4,2) {$F(C \otimes D)$};
\node (bl) at (0,0) {$G(A \otimes B)$};
\node (br) at (4,0) {$G(C \otimes D)$};
\draw[style=tickarrow] (tl) to node [above] {$FM \otimes FN$} (tr);
\draw[style=tickarrow] (ml) to node [above] {$F(M\otimes N)$} (mr);
\draw[->] (tl) to node [left] {$\phi_{A,B}$} (ml);
\draw[->] (tr) to node [right] {$\phi_{C,D}$}(mr);
\draw[->] (ml) to node [left] {$\alpha_{A\otimes B}$} (bl);
\draw[->] (mr) to node [right] {$\alpha_{C \otimes D}$} (br);
\draw[style=tickarrow] (bl) to node [above] {$G(M \otimes N)$} (br);
\node at (2,3) {$\Downarrow \phi_{M,N}$};
\node at (2,1) {$\Downarrow \alpha_{M \otimes N}$};
\end{tikzpicture}
\end{aligned}
=
\begin{aligned}
\begin{tikzpicture}
\node (tl) at (0,4) {$FA \otimes FB$};
\node (tr) at (4,4) {$FC \otimes FD$};
\node (ml) at (0,2) {$GA \otimes GB$};
\node (mr) at (4,2) {$GC \otimes GD$};
\node (bl) at (0,0) {$G(A \otimes B)$};
\node (br) at (4,0) {$G(C \otimes D)$};
\draw[style=tickarrow] (tl) to node [above] {$FM \otimes FN$} (tr);
\draw[style=tickarrow] (ml) to node [above] {$GM \otimes GN$} (mr);
\draw[->] (tl) to node [left] {$\alpha_A \otimes \alpha_B$} (ml);
\draw[->] (tr) to node [right] {$\alpha_C \otimes \alpha_D$} (mr);
\draw[->] (ml) to node [left] {$\psi_{A,B}$} (bl);
\draw[->] (mr) to node [right] {$\psi_{C,D}$} (br);
\draw[style=tickarrow] (bl) to node [above] {$G(M \otimes N)$} (br);
\node at (2,3) {$\Downarrow \alpha_M \otimes \alpha_N$};
\node at (2,1) {$\Downarrow \psi_{M,N}$};
\end{tikzpicture}
\end{aligned}
\end{equation}

\begin{equation}
\begin{aligned}
\begin{tikzpicture}
\node (tl) at (0,4) {$I_{\mathbb{E}}$};
\node (tr) at (4,4) {$I_{\mathbb{E}}$};
\node (ml) at (0,2) {$F(I_{\mathbb{D}})$};
\node (mr) at (4,2) {$F(I_{\mathbb{D}})$};
\node (bl) at (0,0) {$G(I_{\mathbb{D}})$};
\node (br) at (4,0) {$G(I_{\mathbb{D}})$};
\draw[style=tickarrow] (tl) to node [above] {$U_{I_{\mathbb{E}}}$} (tr);
\draw[style=tickarrow] (ml) to node [above] {$F(U_{I_{\mathbb{D}}})$} (mr);
\draw[->] (tl) to node [left] {$\phi_{u_0}$} (ml);
\draw[->] (tr) to node [right] {$\phi_{u_1}$}(mr);
\draw[->] (ml) to node [left] {$\alpha_{I_{\mathbb{D}}}$} (bl);
\draw[->] (mr) to node [right] {$\alpha_{I_{\mathbb{D}}}$} (br);
\draw[style=tickarrow] (bl) to node [below] {$G(U_{I_{\mathbb{D}}})$} (br);
\node at (2,3) {$\Downarrow \phi_{u_1}$};
\node at (2,1) {$\Downarrow \alpha_{U_{I_{\mathbb{D}}}}$};
\end{tikzpicture}
\end{aligned}
=
\begin{aligned}
\begin{tikzpicture}
\node (tl) at (0,4) {$I_{\mathbb{E}}$};
\node (tr) at (4,4) {$I_{\mathbb{E}}$};
\node (ml) at (0,2) {$G(I_{\mathbb{D}})$};
\node (mr) at (4,2) {$G(I_{\mathbb{D}})$};
\draw[style=tickarrow] (tl) to node [above] {$U_{I_{\mathbb{E}}}$} (tr);
\draw[style=tickarrow] (ml) to node [below] {$G(U_{I_{\mathbb{D}}})$} (mr);
\draw[->] (tl) to node [left] {$\psi_{u}$} (ml);
\draw[->] (tr) to node [right] {$\psi_{u}$}(mr);
\node at (2,3) {$\Downarrow \psi_{u_1}$};
\end{tikzpicture}
\end{aligned}
\end{equation}


A {\bf braided or symmetric monoidal tight transformation} is a monoidal transformation between braided/symmetric monoidal functors.
\end{defn}

We have three strict 2-categories $\cMonDbll, \cMonDblc,\cMonDblp$ of monoidal double categories and lax, colax, or pseudo monoidal functors, respectively.
(More generally, we have three such 2-categories of pseudomonoids internal to any 2-category with finite products.)


% Local Variables:
% TeX-master: "smbicat"
% End:


\section{Companions and conjoints}
\label{sec:comp-conj}

Suppose that \lD\ is a monoidal double category; when does
$\cH(\lD)$ become a monoidal bicategory?  It clearly has a
unit object $I$, and the pseudo double functor $\ten\maps
\lD\times\lD\to\lD$ clearly induces a functor $\ten\maps
\cH(\lD)\times\cH(\lD)\to\cH(\lD)$.  However, the problem is that the
constraint isomorphisms such as $A\ten (B\ten C)\iso (A\ten B)\ten C$
are \emph{tight} 1-cells, which get discarded when we pass to
$\cH(\lD)$.  Thus, in order for $\cH(\lD)$ to inherit a symmetric
monoidal structure, we must have a way to make tight 1-cells
into loose ones.  Thus is the purpose of the following
definition.


\begin{defn}\label{def:companion}
  Let \lD\ be a double category and $f\maps A\to B$ a tight
  1-cell.  A \textbf{companion} of $f$ is a loose 1-cell
  $\fhat\maps A\hto B$ together with 2-morphisms
  \begin{equation*}
    \begin{array}{c}
      \xymatrix@-.5pc{
        \ar[r]|-@{|}^-{\fhat} \ar[d]_f \ar@{}[dr]|{\Downarrow \epsilon_{\hat{f}} }
        & \ar@{=}[d]\\
        \ar[r]|-@{|}_-{U_B} & }
    \end{array}\quad\text{and}\quad
    \begin{array}{c}
      \xymatrix@-.5pc{
        \ar[r]|-@{|}^-{U_A} \ar@{=}[d] \ar@{}[dr]|{\Downarrow \eta_{\hat{f}}}
        & \ar[d]^f\\
        \ar[r]|-@{|}_-{\fhat} & }
    \end{array}
  \end{equation*}
  such that the following equations hold.
  \begin{align}\label{eq:compeqn}
    \begin{array}{c}
      \xymatrix@-.5pc{
        \ar[r]|-@{|}^-{U_A} \ar@{=}[d] \ar@{}[dr]|{\Downarrow \eta_{\hat{f}}}
        & \ar[d]^f\\
        \ar[r]|-{\fhat} \ar[d]_f \ar@{}[dr]|
        {\Downarrow  \epsilon_{\hat{f}} }
        & \ar@{=}[d]\\
        \ar[r]|-@{|}_-{U_B} & }
    \end{array} &= 
    \begin{array}{c}
      \xymatrix@-.5pc{ \ar[r]|-@{|}^-{U_A} \ar[d]_f
        \ar@{}[dr]|{\Downarrow U_f} &  \ar[d]^f\\
        \ar[r]|-@{|}_-{U_B} & }
    \end{array}
    &
    \begin{array}{c}
      \xymatrix@-.5pc{
        \ar[r]|-@{|}^-{U_A} \ar@{=}[d] \ar@{}[dr]|{ \Downarrow \eta_{\hat{f}}}&
        \ar[r]|-@{|}^{\fhat}\ar[d]|f \ar@{}[dr]|{\Downarrow  \epsilon_{\hat{f}} }
        & \ar@{=}[d]\\
        \ar[r]|-@{|}_-{\fhat} &
        \ar[r]|-@{|}_-{U_B} &}
%       \xymatrix@-.5pc{
%         \ar[rr]|-@{|}^-{\fhat} \ar@{}[drr]|\iso \ar@{=}[d] &&
%         \ar@{=}[d] \\
%         \ar[r]|-@{|}^-{U_A} \ar@{=}[d] \ar@{}[dr]|\Downarrow &
%         \ar[r]|-@{|}^-{\fhat} \ar[d]_f \ar@{}[dr]|\Downarrow
%         & \ar@{=}[d]\\
%         \ar[r]|-@{|}_-{\fhat} &
%         \ar[r]|-@{|}_-{U_B} &\\
%         \ar[rr]|-@{|}_-{\fhat} \ar@{}[urr]|\iso \ar@{=}[u] &&
%         \ar@{=}[u]}
    \end{array} &=
    \begin{array}{c}
      \xymatrix@-.5pc{
        \ar[r]|-@{|}^-{\fhat} \ar@{=}[d] \ar@{}[dr]|{\Downarrow 1_{\fhat}}
        & \ar@{=}[d]\\
        \ar[r]|-@{|}_-{\fhat} & }
    \end{array}
  \end{align}
  A \textbf{conjoint} of $f$, denoted $\fchk\maps B\hto A$, is a
  companion of $f$ in the ``loose opposite'' double category $\lD\lop$.
\end{defn}

\begin{rmk}
  We momentarily suspend our convention of pretending that our double
  categories are strict to mention that the second
  equation in~\eqref{eq:compeqn} actually requires an insertion of unit
  isomorphisms to make sense.
\end{rmk}

The form of this definition is due
to~\cite{gp:double-adjoints,dpp:spans}, but the ideas date back
to~\cite{bs:dblgpd-xedmod}; see
also~\cite{bm:dbl-thin-conn,fiore:pscat}.  In the terminology of these
references, a \emph{connection} on a double category is equivalent to
a strictly functorial choice of a companion for each tight arrow.

% a loose 1-cell $\fchk\maps B\hto
%   A$ together with 2-morphisms
%   \[\begin{array}{c}
%     \xymatrix@-.5pc{
%       \ar[r]|-@{|}^-{\fchk} \ar@{=}[d] \ar@{}[dr]|\Downarrow
%       & \ar[d]^f\\
%       \ar[r]|-@{|}_-{U_B} & }
%   \end{array}\quad\text{and}\quad
%   \begin{array}{c}
%     \xymatrix@-.5pc{
%       \ar[r]|-@{|}^-{U_A} \ar[d]_f \ar@{}[dr]|\Downarrow
%       & \ar@{=}[d]\\
%       \ar[r]|-@{|}_-{\fchk} & }
%   \end{array}\]
%   such that the following equations hold.
%   \begin{align*}
%     \begin{array}{c}
%       \xymatrix@-.5pc{
%         \ar[r]|-@{|}^-{U_A} \ar[d]_f \ar@{}[dr]|\Downarrow
%         & \ar@{=}[d]\\
%         \ar[r]|-{\fchk} \ar@{=}[d] \ar@{}[dr]|\Downarrow
%         & \ar[d]^f\\
%         \ar[r]|-@{|}_-{U_B} & }
%     \end{array} &= 
%     \begin{array}{c}
%       \xymatrix@-.5pc{ \ar[r]|-@{|}^-{U_A} \ar[d]_f
%         \ar@{}[dr]|{\Downarrow U_f} &  \ar[d]^f\\
%         \ar[r]|-@{|}_-{U_B} & }
%     \end{array}
%     &
%     \begin{array}{c}
%       \xymatrix@-.5pc{
%         \ar[rr]|-@{|}^-{\fchk} \ar@{}[drr]|\iso \ar@{=}[d] &&
%         \ar@{=}[d] \\
%         \ar[r]|-@{|}^-{\fchk} \ar@{=}[d] \ar@{}[dr]|\Downarrow &
%         \ar[r]|-@{|}^-{U_A} \ar[d]_f \ar@{}[dr]|\Downarrow
%         & \ar@{=}[d]\\
%         \ar[r]|-@{|}_-{U_B} &
%         \ar[r]|-@{|}_-{\fchk} &\\
%         \ar[rr]|-@{|}_-{\fchk} \ar@{}[urr]|\iso \ar@{=}[u] &&
%         \ar@{=}[u]}
%     \end{array} &=
%     \begin{array}{c}
%       \xymatrix@-.5pc{
%         \ar[r]|-@{|}^-{\fchk} \ar@{=}[d]
%         & \ar@{=}[d]\\
%         \ar[r]|-@{|}_-{\fchk} & }
%     \end{array}
%   \end{align*}


\begin{egs}
  \lMod, \lnCob, and \lProf\ have companions and conjoints for all tight morphisms.  In \lMod, the companion
  of a ring homomorphism $f\maps A\to B$ is $B$ regarded as an
  $A$-$B$-bimodule via $f$ on the left, and dually for its conjoint.
  In \lnCob, companions and conjoints are obtained by regarding a
  diffeomorphism as a cobordism.  And in \lProf, companions and
  conjoints are obtained by regarding a functor $f\maps A\to B$ as a
  `representable' profunctor $B(f-,-)$ or $B(-,f-)$.
\end{egs}

% \begin{rmk}
%   For an $(n\times 1)$-category (recall \autoref{rmk:monglob}), the
%   lifting condition we should require is simply that each double
%   category $\lD_{i+1} \toto \lD_i$, for $0\le i < n$, is fibrant.
% \end{rmk}

The existence of companions and conjoints gives us a way to `lift'
tight 1-cells to loose ones.  What is even more crucial
for our application, however, is that these liftings are unique up to
isomorphism, and that these isomorphisms are canonical and coherent.
This is the content of the following lemmas.  We state most of them
only for companions, but all have dual versions for conjoints.

%Later we will use them to prove the existence of structure isomorphisms and the commutativity of diagrams needed for $\cH$ to preserve monoidal structures.

\begin{lem}\label{thm:theta}
  Let $\fhat\maps A\hto B$ and $\fhat'\maps A\hto B$ be companions of
  $f$ (that is, each comes \emph{equipped with} 2-morphisms as in
  \autoref{def:companion}).  Then there is a unique globular isomorphism
  $\theta_{\fhat,\fhat'}\maps \fhat\too[\iso]\fhat'$ such that
  \begin{equation}\label{eq:comp-iso}
    \vcenter{\xymatrix@R=1.5pc{
        \ar[r]|-@{|}^-{U_A} \ar@{=}[d] \ar@{}[dr]|{\Downarrow \eta_{\hat{f}}} &  \ar[d]^f\\
        \ar[r]|-{\fhat} \ar@{=}[d] \ar@{}[dr]|{\Downarrow \theta_{\fhat,\fhat'}} &  \ar@{=}[d]\\
        \ar[r]|-{\fhat'} \ar[d]_f \ar@{}[dr]|{\Downarrow \epsilon_{\hat{f}'}} & \ar@{=}[d]\\
        \ar[r]|-@{|}_-{U_B} & }} \quad = \quad
    \vcenter{\xymatrix@-.5pc{ \ar[r]|-@{|}^-{U_A} \ar[d]_f
        \ar@{}[dr]|{\Downarrow U_f} &  \ar[d]^f\\
        \ar[r]|-@{|}_-{U_B} & .}}
  \end{equation}
\end{lem}
\begin{proof}
  Composing~\eqref{eq:comp-iso} on the left with
  $\vcenter{\xymatrix@-.5pc{ \ar[r]|-@{|}^-{U_A} \ar@{=}[d]
      \ar@{}[dr]|{\Downarrow \eta_{\hat{f}}} & \ar[d]^f\\ \ar[r]|-@{|}_-{\fhat'} & }}$
  and on the right with $\vcenter{\xymatrix@-.5pc{
      \ar[r]|-@{|}^-{\fhat} \ar[d]_f \ar@{}[dr]|{\Downarrow \epsilon_{\hat{f}}}&
      \ar@{=}[d]\\ \ar[r]|-@{|}_-{U_B} & }}$, and using the second
  equation~\eqref{eq:compeqn}, we see that if~\eqref{eq:comp-iso} is
  satisfied then $\theta_{\fhat,\fhat'}$ must be the composite
  \begin{equation}
    \vcenter{\xymatrix@-.5pc{
        \ar[r]|-@{|}^-{U_A} \ar@{=}[d] \ar@{}[dr]|{\Downarrow \eta_{\hat{f}'}}&
        \ar[r]|-@{|}^-{\fhat} \ar[d]|f \ar@{}[dr]|{\Downarrow \epsilon_{\hat{f}}}
        & \ar@{=}[d]\\
        \ar[r]|-@{|}_-{\fhat'} &
        \ar[r]|-@{|}_-{U_B} &}}\label{eq:theta}
%     \vcenter{\xymatrix@-.5pc{
%         \ar[rr]|-@{|}^-{\fhat} \ar@{}[drr]|\iso \ar@{=}[d] &&
%         \ar@{=}[d] \\
%         \ar[r]|-@{|}^-{U_A} \ar@{=}[d] \ar@{}[dr]|\Downarrow &
%         \ar[r]|-@{|}^-{\fhat} \ar[d]_f \ar@{}[dr]|\Downarrow
%         & \ar@{=}[d]\\
%         \ar[r]|-@{|}_-{\fhat'} &
%         \ar[r]|-@{|}_-{U_B} &\\
%         \ar[rr]|-@{|}_-{\fhat'} \ar@{}[urr]|\iso \ar@{=}[u] &&
%         \ar@{=}[u]}}\label{eq:theta}
  \end{equation}
  Two applications of the first equation~\eqref{eq:compeqn} shows that
  this indeed satisfies~\eqref{eq:comp-iso}.  As for its being an
  isomorphism, we have the dual composite $\theta_{\fhat',\fhat}$:
  \[\vcenter{\xymatrix@-.5pc{
      \ar[r]|-@{|}^-{U_A} \ar@{=}[d] \ar@{}[dr]|{\Downarrow \eta_{\hat{f}}} &
      \ar[r]|-@{|}^{\fhat'}\ar[d]|f \ar@{}[dr]|
{\Downarrow \epsilon_{\hat{f}'}}
      & \ar@{=}[d]\\
      \ar[r]|-@{|}_-{\fhat} &
      \ar[r]|-@{|}_-{U_B} &}}\]
  which we verify is an inverse using~\eqref{eq:compeqn}:
  \[\vcenter{\xymatrix@-.5pc{
      \ar[r]|-@{|}^{U_A}\ar@{=}[d] \ar@{}[dr]|{=} &
      \ar[r]|-@{|}^{U_A}\ar@{=}[d] \ar@{}[dr]|{\Downarrow \eta_{\hat{f}'}} &
      \ar[r]|-@{|}^{\fhat}\ar[d]|f \ar@{}[dr]|{\Downarrow \epsilon_{\hat{f}}} &
      \ar@{=}[d]\\
      \ar[r]|{U_A}\ar@{=}[d] \ar@{}[dr]|{\Downarrow \eta_{\hat{f}}} &
      \ar[r]|{\fhat'}\ar[d]|f \ar@{}[dr]|{\Downarrow \epsilon_{\hat{f}'}} &
      \ar[r]|{U_B}\ar@{=}[d] \ar@{}[dr]|{=} &
      \ar@{=}[d]\\
      \ar[r]|-@{|}_{\fhat} &
      \ar[r]|-@{|}_{U_B} &
      \ar[r]|-@{|}_{U_B} &
    }} \;=\;
  \vcenter{\xymatrix@-.5pc{
      \ar[r]|-@{|}^-{U_A} \ar@{=}[d] \ar@{}[dr]|{\Downarrow  \eta_{\hat{f}}}&
      \ar[r]|-@{|}^{\fhat}\ar[d]|f \ar@{}[dr]|{\Downarrow \epsilon_{\hat{f}}}
      & \ar@{=}[d]\\
      \ar[r]|-@{|}_-{\fhat} &
      \ar[r]|-@{|}_-{U_B} &}} \;=\;
  \vcenter{\xymatrix@-.5pc{
      \ar[r]|-@{|}^-{\fhat} \ar@{=}[d] \ar@{}[dr]|{\Downarrow 1_{\fhat}}
      & \ar@{=}[d]\\
      \ar[r]|-@{|}_-{\fhat} & }}\]
  (and dually).
\end{proof}

\begin{lem}\label{thm:theta-id}
  For any companion \fhat\ of $f$ we have $\theta_{\fhat,\fhat}=1_{\fhat}$.
\end{lem}
\begin{proof}
  This is the second equation~\eqref{eq:compeqn}.
\end{proof}

\begin{lem}\label{thm:theta-compose-vert}
  Suppose that $f$ has three companions $\fhat$, $\fhat'$, and
  $\fhat''$.  Then $\theta_{\fhat,\fhat''} = \theta_{\fhat',\fhat''}
  \circ\theta_{\fhat,\fhat'}$.
\end{lem}
\begin{proof}
  By definition, we have
  \[\theta_{\fhat',\fhat''} \circ\theta_{\fhat,\fhat'} =\;
  \vcenter{\xymatrix@-.5pc{
      \ar[r]|-@{|}^{U_A}\ar@{=}[d] \ar@{}[dr]|{=} &
      \ar[r]|-@{|}^{U_A}\ar@{=}[d] \ar@{}[dr]|{\Downarrow \eta_{\hat{f}'}} &
      \ar[r]|-@{|}^{\fhat}\ar[d]|f \ar@{}[dr]|{\Downarrow \epsilon_{\hat{f}}} &
      \ar@{=}[d]\\
      \ar[r]|{U_A}\ar@{=}[d] \ar@{}[dr]|{\Downarrow \eta_{\hat{f}''}} &
      \ar[r]|{\fhat'}\ar[d]|f \ar@{}[dr]|{\Downarrow \epsilon_{\hat{f}'}} &
      \ar[r]|{U_B}\ar@{=}[d] \ar@{}[dr]|{=} &
      \ar@{=}[d]\\
      \ar[r]|-@{|}_{\fhat''} &
      \ar[r]|-@{|}_{U_B} &
      \ar[r]|-@{|}_{U_B} &
    }} \;=\;
  \vcenter{\xymatrix@-.5pc{
      \ar[r]|-@{|}^-{U_A} \ar@{=}[d] \ar@{}[dr]|{\Downarrow \eta_{\hat{f}''}}&
      \ar[r]|-@{|}^{\fhat}\ar[d]|f \ar@{}[dr]|{\Downarrow \epsilon_{\hat{f}}}
      & \ar@{=}[d]\\
      \ar[r]|-@{|}_-{\fhat''} &
      \ar[r]|-@{|}_-{U_B} &}} \;=
  \theta_{\fhat,\fhat''}\]
  as desired.
\end{proof}

\begin{lem}\label{thm:comp-unit}
  $U_A\maps A\hto A$ is always a companion of $1_A\maps A\to A$ in a
  canonical way.
\end{lem}
\begin{proof}
  We take both defining 2-morphisms to be
  $1_{U_A}$; the truth of~\eqref{eq:compeqn} is evident.
\end{proof}

\begin{lem}\label{thm:comp-compose}
  Suppose that $f\maps A\to B$ has a companion \fhat\ and $g\maps B\to
  C$ has a companion \ghat.  Then $\ghat\odot\fhat$ is a companion of
  $gf$.
\end{lem}
\begin{proof}
  We take the defining 2-morphisms to be the composites
  \[\vcenter{\xymatrix@-.5pc{
      \ar[r]|-@{|}^-{\fhat} \ar[d]_f \ar@{}[dr]|{\Downarrow \epsilon_{\hat{f}}}&
      \ar[r]|-@{|}^-{\ghat} \ar@{=}[d] \ar@{}[dr]|{1_{\ghat}} &
      \ar@{=}[d]\\
      \ar[r]|-{U_B} \ar[d]_g \ar@{}[dr]|{U_g} &
      \ar[r]|-{\ghat} \ar[d]|g \ar@{}[dr]|{\Downarrow \epsilon_{\hat{g}}}&
      \ar@{=}[d]\\
      \ar[r]|-@{|}_-{U_C} &
      \ar[r]|-@{|}_-{U_C} &
    }}\quad\text{and}\quad
  \vcenter{\xymatrix@-.5pc{
      \ar[r]|-@{|}^-{U_A} \ar@{=}[d] \ar@{}[dr]|{\Downarrow \eta_{\hat{f}}} &
      \ar[r]|-@{|}^-{U_A} \ar[d]|f \ar@{}[dr]|{U_f} &
      \ar[d]^f\\
      \ar[r]|-{\fhat} \ar@{=}[d] \ar@{}[dr]|{1_{\fhat}} &
      \ar[r]|-{U_B} \ar@{=}[d] \ar@{}[dr]|{\Downarrow \eta_{\hat{g}}}&
      \ar[d]^g\\
      \ar[r]|-@{|}_-{\fhat} &
      \ar[r]|-@{|}_-{\ghat} &
    }}
  \]
  It is easy to verify that these satisfy~\eqref{eq:compeqn}, using
  the interchange law for $\odot$ and $\circ$ in a double category.
\end{proof}

\begin{lem}\label{thm:theta-compose-horiz}
  Suppose that $f\maps A\to B$ has companions $\fhat$ and $\fhat'$,
  and that $g\maps B\to C$ has companions $\ghat$ and $\ghat'$.  Then
  $\theta_{\ghat,\ghat'}\odot \theta_{\fhat,\fhat'}  =
    \theta_{\ghat\odot\fhat, \ghat'\odot\fhat'}$.
\end{lem}
\begin{proof}
  Using the interchange law for $\odot$ and $\circ$, we have:
  \begin{align}
    \theta_{\ghat\odot\fhat, \ghat'\odot\fhat'} &=\;
    \vcenter{\xymatrix@-.5pc{
        \ar[r]|-@{|}^-{U_A} \ar@{=}[d] \ar@{}[dr]|{\Downarrow \eta_{\hat{f}'}}&
        \ar[r]|-@{|}^-{U_A} \ar[d]|f \ar@{}[dr]|{U_f} &
        \ar[r]|-@{|}^-{\fhat} \ar[d]|f \ar@{}[dr]|{\Downarrow \epsilon_{\hat{f}}}&
        \ar[r]|-@{|}^-{\ghat} \ar@{=}[d] \ar@{}[dr]|{1_{\fhat}} &
        \ar@{=}[d]\\
        \ar[r]|-{\fhat'} \ar@{=}[d] \ar@{}[dr]|{1_{\ghat}} &
        \ar[r]|-{U_B} \ar@{=}[d] \ar@{}[dr]|{\Downarrow \eta_{\hat{g}'}} &
        \ar[r]|-{U_B} \ar[d]|g \ar@{}[dr]|{U_g} &
        \ar[r]|-{\ghat} \ar[d]|g \ar@{}[dr]|{\Downarrow \epsilon_{\hat{g}}} &
        \ar@{=}[d]\\
        \ar[r]|-@{|}_-{\fhat'} &
        \ar[r]|-@{|}_-{\ghat'} &
        \ar[r]|-@{|}_-{U_C} &
        \ar[r]|-@{|}_-{U_C} &
      }}
    \;=\;
    \vcenter{\xymatrix@-.5pc{
        \ar[r]|-@{|}^-{U_A} \ar@{=}[d] \ar@{}[dr]|{\Downarrow \eta_{\hat{f}'}}&
        \ar[r]|-@{|}^-{\fhat} \ar[d]|f \ar@{}[dr]|{\Downarrow \epsilon_{\hat{f}}}&
        \ar[r]|-@{|}^-{\ghat} \ar@{=}[d] \ar@{}[dr]|{1_{\fhat}} &
        \ar@{=}[d]\\
        \ar[r]|-{\fhat'} \ar@{=}[d] \ar@{}[dr]|{1_{\ghat}} &
        \ar[r]|-{U_B} \ar@{=}[d] \ar@{}[dr]|{\Downarrow \eta_{\hat{g}'}} &
        \ar[r]|-{\ghat} \ar[d]|g \ar@{}[dr]|{\Downarrow \epsilon_{\hat{g}}} &
        \ar@{=}[d]\\
        \ar[r]|-@{|}_-{\fhat'} &
        \ar[r]|-@{|}_-{\ghat'} &
        \ar[r]|-@{|}_-{U_C} &
      }}\\
    &=\;
    \vcenter{\xymatrix@-.5pc{
        \ar[r]|-@{|}^-{U_A} \ar@{=}[d] \ar@{}[dr]|{\Downarrow \eta_{\hat{f}'}}&
        \ar[r]|-@{|}^-{\fhat} \ar[d]|f \ar@{}[dr]|{\Downarrow \epsilon_{\hat{f}}}&
        \ar[r]|-@{|}^-{U_B} \ar@{=}[d] \ar@{}[dr]|{1_{U_B}} &
        \ar[r]|-@{|}^-{\ghat} \ar@{=}[d] \ar@{}[dr]|{1_{\fhat}} &
        \ar@{=}[d]\\
        \ar[r]|-{\fhat'} \ar@{=}[d] \ar@{}[dr]|{1_{\ghat}} &
        \ar[r]|-{U_B} \ar@{=}[d] \ar@{}[dr]|{1_{U_B}} &
        \ar[r]|-{U_B} \ar@{=}[d] \ar@{}[dr]|{\Downarrow \eta_{\hat{g}'}} &
        \ar[r]|-{\ghat} \ar[d]|g \ar@{}[dr]|{\Downarrow \epsilon_{\hat{g}}} &
        \ar@{=}[d]\\
        \ar[r]|-@{|}_-{\fhat'} &
        \ar[r]|-@{|}_-{U_B} &
        \ar[r]|-@{|}_-{\ghat'} &
        \ar[r]|-@{|}_-{U_C} &
      }}\;=\;
    \vcenter{\xymatrix@-.5pc{
        \ar[r]|-@{|}^-{U_A} \ar@{=}[d] \ar@{}[dr]|{\Downarrow \eta_{\hat{f}'}}&
        \ar[r]|-@{|}^-{\fhat} \ar[d]|f \ar@{}[dr]|{\Downarrow \epsilon_{\hat{f}}}&
        \ar[r]|-@{|}^-{U_B} \ar@{=}[d] \ar@{}[dr]|{\Downarrow \eta_{\hat{g}'}}&
        \ar[r]|-@{|}^-{\ghat} \ar[d]|g \ar@{}[dr]|{\Downarrow \epsilon_{\hat{g}}}& \ar@{=}[d]\\
        \ar[r]|-@{|}_-{\fhat'} &
        \ar[r]|-@{|}_-{U_B} &
        \ar[r]|-@{|}_-{\ghat'} &
        \ar[r]|-@{|}_-{U_C} &
      }}\\
    &=\;
    \theta_{\ghat,\ghat'}\odot \theta_{\fhat,\fhat'} 
  \end{align}
  as desired.
\end{proof}

\begin{lem}\label{thm:theta-unit}
  If $f\maps A\to B$ has a companion \fhat, then
  $\theta_{\fhat,\fhat\odot U_A}$ and $\theta_{\fhat,U_B\odot \fhat}$
  are equal to the unit constraints $\fhat \iso \fhat\odot U_A$ and
  $\fhat\iso U_B\odot \fhat$.
\end{lem}
\begin{proof}
  By definition, we have
  \[\theta_{\fhat,\fhat\odot U_A} =\;
  \vcenter{\xymatrix@-.5pc{
      \ar[r]|-@{|}^-{U_A} \ar@{=}[d] \ar@{}[dr]|{\Downarrow 1_{U_A}} &
      \ar[r]|-@{|}^-{U_A} \ar@{=}[d] \ar@{}[dr]|{1_{U_A}} &
      \ar@{=}[d] \ar[rr]|-@{|}^-{\fhat} \ar@{}[ddrr]|{\Downarrow \epsilon_{\hat{f}}}&& \ar@{=}[dd]\\
      \ar[r]|-{U_A} \ar@{=}[d] \ar@{}[dr]|{1_{U_A}} &
      \ar[r]|-{U_A} \ar@{=}[d] \ar@{}[dr]|{\Downarrow \eta_{\hat{f}}}&
      \ar[d]^f\\
      \ar[r]|-@{|}_-{U_A} &
      \ar[r]|-@{|}_-{\fhat} & \ar[rr]|-@{|}^-{U_B} &&
    }}\;=\;
  \vcenter{\xymatrix{ \ar[r]|-@{|}^-{U_A} \ar@{=}[d]
      \ar@{}[dr]|{\Downarrow 1_{U_A}} &  \ar@{=}[d]\\
      \ar[r]|-@{|}_-{U_A} & }}
  \]
  which, bearing in mind our suppression of unit and associativity
  constraints, means that in actuality it is the unit constraint
  $\fhat \iso \fhat\odot U_A$.  The other case is dual.
\end{proof}

\begin{lem}\label{thm:comp-func}
  Let $F\maps \lD\to\lE$ be a functor between double categories and
  let $f\maps A\to B$ have a companion \fhat\ in \lD.  Then $F(\fhat)$
  is a companion of $F(f)$ in \lE.
\end{lem}
\begin{proof}
  We take the defining 2-morphisms to be
  \[\vcenter{\xymatrix@R=1.5pc@C=3pc{
      \ar[r]|-@{|}^-{F(\fhat)} \ar[d]_{F(f)}
      \ar@{}[dr]|{F(\Downarrow \epsilon_{\hat{f}})} &  \ar@{=}[d]\\
      \ar[r]|-{F(U_B)} \ar@{=}[d] \ar@{}[dr]|\iso &  \ar@{=}[d]\\
      \ar[r]|-@{|}_-{U_{F(B)}} & }}
  \quad\text{and}\quad
  \vcenter{\xymatrix@R=1.5pc@C=3pc{
      \ar[r]|-@{|}^-{U_{FA}} \ar@{=}[d] \ar@{}[dr]|\iso & \ar@{=}[d]\\
      \ar[r]|-{F(U_{A})} \ar@{=}[d] \ar@{}[dr]|{F(\Downarrow \eta_{\hat{f}})} & 
      \ar[d]^{F(f)}\\
      \ar[r]|-@{|}_-{F(\fhat)} & .}}\]
  The axioms~\eqref{eq:compeqn} follow directly from those for \fhat.
\end{proof}

% \begin{lem}\label{thm:comp-ten}
%   Suppose that \lD\ is a monoidal double category and that $f\maps
%   A\to B$ and $g\maps C\to D$ have companions \fhat\ and \ghat\
%   respectively.  Then $\fhat\ten\ghat$ is a companion of $f\ten g$.
% \end{lem}
% \begin{proof}
%   This follows from \autoref{thm:comp-func}, since $\ten\maps
%   \lD\times\lD\to\lD$ is a functor, and a companion in $\lD\times\lD$
%   is simply a pair of companions in \lD.
% %   We take the defining 2-morphisms to be
% %   \[\vcenter{\xymatrix@R=1.5pc@C=3pc{
% %       \ar[r]|-@{|}^-{\fhat\ten\ghat} \ar[d]_{f\ten g}
% %       \ar@{}[dr]|{\Downarrow\ten\Downarrow} &  \ar@{=}[d]\\
% %       \ar[r]|-{U_B\ten U_D} \ar@{=}[d] \ar@{}[dr]|\iso &  \ar@{=}[d]\\
% %       \ar[r]|-@{|}_-{U_{B\ten D}} & }}
% %   \quad\text{and}\quad
% %   \vcenter{\xymatrix@R=1.5pc@C=3pc{
% %       \ar[r]|-@{|}^-{U_{A\ten C}} \ar@{=}[d] \ar@{}[dr]|\iso & \ar@{=}[d]\\
% %       \ar[r]|-{U_{A}\ten U_C} \ar@{=}[d] \ar@{}[dr]|{\Downarrow\ten\Downarrow} & 
% %       \ar[d]^{f\ten g}\\
% %       \ar[r]|-@{|}_-{\fhat\ten\ghat} & .}}\]
% \end{proof}

\begin{lem}\label{thm:theta-func}
  Suppose that $F\maps \lD\to\lE$ is a functor and that $f\maps A\to
  B$ has companions \fhat\ and $\fhat'$ in \lD.  Then
  $\theta_{F(\fhat),F(\fhat')} = F(\theta_{\fhat,\fhat'})$.
\end{lem}
\begin{proof}
  Using the axioms of a pseudo double functor and the definition of
  the 2-morphisms in \autoref{thm:comp-func}, we have
  \begin{equation}
    F(\theta_{\fhat,\fhat'})
    =\;
    \vcenter{\xymatrix@C=3pc{
        \ar[r]|-@{|}^-{F(\fhat)}
        \ar[d] \ar@{}[dr]|{\Downarrow F(\eta_{\hat{f}'} \odot\epsilon_{\hat{f}})} &  \ar[d]\\
        \ar[r]|-@{|}_-{F(\fhat')} &}}
    \;=\;
    \vcenter{\xymatrix@C=3pc{
        \ar[rr]|-@{|}^-{F(\fhat)}
        \ar@{=}[d] \ar@{}[drr]|\iso &&  \ar@{=}[d]\\
        \ar[r]|-@{|}^-{F(U_{A})} \ar@{=}[d]
        \ar@{}[dr]|{\Downarrow F(\eta_{\hat{f}'})} &
        \ar[r]|-@{|}^-{F(\fhat)} \ar[d]|{F(f)}
        \ar@{}[dr]|{\Downarrow F(\epsilon_{\hat{f}})}
        & \ar@{=}[d]\\
        \ar[r]|-@{|}_-{F(\fhat')} \ar@{}[drr]|\iso\ar@{=}[d] &
        \ar[r]|-@{|}_-{U_{F(B)}} & \ar@{=}[d]\\
        \ar[rr]|-@{|}_-{F(\fhat')} && }}
    \;=\;
    \vcenter{\xymatrix@R=1.5pc@C=3pc{
        \ar[r]|-@{|}^-{U_{F(A)}} \ar@{=}[d] \ar@{}[dr]|\iso &
        \ar[r]|-@{|}^-{F(\fhat)} \ar@{=}[d] \ar@{}[dr]|=
        & \ar@{=}[d]\\
        \ar[r]|-{F(U_{A})} \ar@{=}[d] \ar@{}[dr]|{\Downarrow F(\eta_{\hat{f}'})} &
        \ar[r]|-{F(\fhat)} \ar[d]|{F(f)} \ar@{}[dr]|{\Downarrow F(\epsilon_{\hat{f}})}
        & \ar@{=}[d]\\
        \ar[r]|-{F(\fhat')}  \ar@{=}[d] \ar@{}[dr]|= &
        \ar[r]|-{F(U_{B})} \ar@{}[dr]|\iso  \ar@{=}[d] & \ar@{=}[d]\\
        \ar[r]|-@{|}_-{F(\fhat')} &
        \ar[r]|-@{|}_-{U_{F(B)}} &}}
    \;=
    \theta_{F(\fhat),\,F(\fhat')}
  \end{equation}
  as desired.
\end{proof}

% \begin{lem}\label{thm:theta-ten}
%   Suppose that \lD\ is a monoidal double category, that $f\maps A\to
%   B$ has companions \fhat\ and $\fhat'$, and that $g\maps C\to D$ has
%   companions \ghat\ and $\ghat'$.  Then $\theta_{\fhat,\fhat'} \ten
%   \theta_{\ghat,\ghat'} = \theta_{\fhat\ten \ghat, \fhat'\ten\ghat'}.$
% \end{lem}
% \begin{proof}
%   This follows from \autoref{thm:theta-func} in the same way that
%   \autoref{thm:comp-ten} follows from \autoref{thm:comp-func}.
% %   Using the naturality and functoriality axioms spelled out in
% %   \S\ref{sec:symm-mono-double}, we have
% %   \begin{equation}
% %     \theta_{\fhat,\fhat'} \ten \theta_{\ghat,\ghat'}
% %     =\;
% %     \vcenter{\xymatrix@C=4.5pc{
% %         \ar[r]|-@{|}^-{\fhat\ten\ghat} \ar@{=}[d] \ar@{}[dr]|\iso &  \ar@{=}[d]\\
% %         \ar[r]|-@{|}^-{(U_A\odot \fhat)\ten(U_C\odot \ghat)}
% %         \ar[d] \ar@{}[dr]|{(\Downarrow\odot\Downarrow)\ten(\Downarrow\odot\Downarrow)} &  \ar[d]\\
% %         \ar[r]|-@{|}_-{(\fhat'\odot U_C) \ten (\ghat'\odot U_D)} \ar@{=}[d] \ar@{}[dr]|\iso &  \ar@{=}[d]\\
% %         \ar[r]|-@{|}_-{\fhat'\ten\ghat'} & }}
% %     \;=\;
% %     \vcenter{\xymatrix@C=2pc{
% %         \ar[rr]|-@{|}^-{\fhat\ten\ghat} \ar@{=}[d] \ar@{}[drr]|\iso &&  \ar@{=}[d]\\
% %         \ar[rr]|-@{|}^-{(U_A\odot \fhat)\ten(U_C\odot \ghat)}
% %         \ar@{=}[d] \ar@{}[drr]|\iso &&  \ar@{=}[d]\\
% %         \ar[r]|-@{|}^-{U_{A}\ten U_C} \ar@{=}[d]
% %         \ar@{}[dr]|{\Downarrow\ten \Downarrow} &
% %         \ar[r]|-@{|}^-{\fhat\ten\ghat} \ar[d]|{f\ten g}
% %         \ar@{}[dr]|{\Downarrow\ten\Downarrow}
% %         & \ar@{=}[d]\\
% %         \ar[r]|-@{|}_-{\fhat'\ten\ghat'} \ar@{}[drr]|\iso\ar@{=}[d] &
% %         \ar[r]|-@{|}_-{U_{B\ten D}} & \ar@{=}[d]\\
% %         \ar[rr]|-@{|}_-{(\fhat'\odot U_C) \ten (\ghat'\odot U_D)}
% %         \ar@{=}[d] \ar@{}[drr]|\iso &&  \ar@{=}[d]\\
% %         \ar[rr]|-@{|}_-{\fhat'\ten\ghat'} && }}
% %     \;=\;
% %     \vcenter{\xymatrix@R=1.5pc@C=2.5pc{
% %         \ar[rr]|-@{|}^-{\fhat\ten\ghat} \ar@{}[drr]|\iso \ar@{=}[d] &&
% %         \ar@{=}[d] \\
% %         \ar[r]|-@{|}^-{U_{A\ten C}} \ar@{=}[d] \ar@{}[dr]|\iso &
% %         \ar[r]|-@{|}^-{\fhat\ten\ghat} \ar@{=}[d] \ar@{}[dr]|=
% %         & \ar@{=}[d]\\
% %         \ar[r]|-@{|}^-{U_{A}\ten U_C} \ar@{=}[d] \ar@{}[dr]|{\Downarrow\ten\Downarrow} &
% %         \ar[r]|-@{|}^-{\fhat\ten\ghat} \ar[d]|{f\ten g} \ar@{}[dr]|{\Downarrow\ten\Downarrow}
% %         & \ar@{=}[d]\\
% %         \ar[r]|-@{|}_-{\fhat'\ten\ghat'}  \ar@{=}[d] \ar@{}[dr]|= &
% %         \ar[r]|-@{|}_-{U_{B}\ten U_D} \ar@{}[dr]|\iso  \ar@{=}[d] & \ar@{=}[d]\\
% %         \ar[r]|-@{|}_-{\fhat'\ten\ghat'} &
% %         \ar[r]|-@{|}_-{U_{B\ten D}} &\\
% %         \ar[rr]|-@{|}_-{\fhat'\ten\ghat'} \ar@{}[urr]|\iso \ar@{=}[u] &&
% %         \ar@{=}[u]}}
% %     \;=
% %     \theta_{\fhat\ten\ghat}
% %   \end{equation}
% %   as desired.
% \end{proof}


\begin{lem}\label{thm:comp-iso}
  If $f\maps A\to B$ is a tight isomorphism with a companion \fhat,
  then \fhat\ is a conjoint of its inverse $f\inv$.
\end{lem}
\begin{proof}
  The composites
  \[\vcenter{\xymatrix@-.5pc{
      \ar[r]|-@{|}^{\fhat}\ar[d]_f \ar@{}[dr]|{\Downarrow} &
      \ar@{=}[d]\\
      \ar[r]|{U_B}\ar[d]_{f\inv} \ar@{}[dr]|{\Downarrow U_{f\inv}} &
      \ar[d]^{f\inv}\\
      \ar[r]|-@{|}_{U_A} &
    }}\quad\text{and}\quad
  \vcenter{\xymatrix@-.5pc{
      \ar[r]|-@{|}^{U_B}\ar[d]_{f\inv} \ar@{}[dr]|{\Downarrow U_{f\inv}} &
      \ar[d]^{f\inv}\\
      \ar[r]|{U_A}\ar@{=}[d] \ar@{}[dr]|{\Downarrow} &
      \ar[d]^f\\
      \ar[r]|-@{|}_{\fhat} &
    }}
  \]
  exhibit \fhat\ as a conjoint of $f\inv$.
\end{proof}

\begin{lem}\label{thm:compconj-adj}
  If $f\maps A\to B$ has both a companion \fhat\ and a conjoint \fchk,
  then we have an adjunction $\fhat\adj\fchk$ in $\cH\lD$.  If $f$ is
  an isomorphism, then this is an adjoint equivalence.
\end{lem}
\begin{proof}
  The unit and counit of the adjunction $\fhat\adj\fchk$ are the
  composites
  \[\vcenter{\xymatrix@-.5pc{
      \ar[r]|-@{|}^{U_A}\ar@{=}[d] \ar@{}[dr]|{\Downarrow \eta_{\hat{f}}} &
      \ar[r]|-@{|}^{U_A}\ar[d]|{f} \ar@{}[dr]|{\Downarrow \eta_{\check{f}}} &
      \ar@{=}[d]\\
      \ar[r]|-@{|}_{\fhat} &
      \ar[r]|-@{|}_{\fchk} &
    }}\quad\text{and}\quad
  \vcenter{\xymatrix@-.5pc{
      \ar[r]|-@{|}^{\fchk}\ar@{=}[d] \ar@{}[dr]|{\Downarrow \epsilon_{\check{f}}} &
      \ar[r]|-@{|}^{\fhat}\ar[d]|{f} \ar@{}[dr]|{\Downarrow \epsilon_{\hat{f}}} &
      \ar@{=}[d]\\
      \ar[r]|-@{|}_{U_B} &
      \ar[r]|-@{|}_{U_B} &
    }}
  \]
%   \[\vcenter{\xymatrix@-.5pc{
%       \ar[rr]|-@{|}^{U_A}\ar@{=}[d] \ar@{}[drr]|{\iso} &&
%       \ar@{=}[d]\\
%       \ar[r]|-@{|}^{U_A}\ar@{=}[d] \ar@{}[dr]|{\Downarrow} &
%       \ar[r]|-@{|}^{U_A}\ar[d]|{f} \ar@{}[dr]|{\Downarrow} &
%       \ar@{=}[d]\\
%       \ar[r]|-@{|}_{\fhat} &
%       \ar[r]|-@{|}_{\fchk} &
%     }}\quad\text{and}\quad
%   \vcenter{\xymatrix@-.5pc{
%       \ar[r]|-@{|}^{\fchk}\ar@{=}[d] \ar@{}[dr]|{\Downarrow} &
%       \ar[r]|-@{|}^{\fhat}\ar[d]|{f} \ar@{}[dr]|{\Downarrow} &
%       \ar@{=}[d]\\
%       \ar[r]|-@{|}_{U_B}\ar@{=}[d] \ar@{}[drr]|{\iso} &
%       \ar[r]|-@{|}_{U_B} &
%       \ar@{=}[d]\\
%       \ar[rr]|-@{|}_{U_B} &&.
%     }}
%   \]
  The triangle identities follow from~\eqref{eq:compeqn}.  If $f$ is
  an isomorphism, then by the dual of \autoref{thm:comp-iso}, \fchk\
  is a companion of $f\inv$.  But then by \autoref{thm:comp-compose}
  $\fchk\odot \fhat$ is a companion of $1_A=f\inv \circ f$ and
  $\fhat\odot\fchk$ is a companion of $1_B = f\circ f\inv$, and hence
  \fhat\ and \fchk\ are equivalences.  We can then check that in this
  case the above unit and counit actually are the isomorphisms
  $\theta$, or appeal to the general fact that any adjunction
  involving an equivalence is an adjoint equivalence.
\end{proof}

% To conclude this section, we combine some of the Lemmas above to derive a more general statement which will play a central role in section~\ref{sec:constr-symm-mono}.
% \begin{lem}\label{lem:equal}  % FALSE as stated
% Any two composites of $\theta$-isomorphisms that have the same source and target loose 1-cells are equal.
% \end{lem}
% \begin{proof}
% By Lemmas~\ref{thm:theta-compose-vert} and~\ref{thm:theta-compose-horiz}, $\theta$-isomorphisms are closed under composition. By uniqueness of $\theta$-isomorphisms, any two compositions of $\theta$-isomorphisms that have the same source and target loose 1-cells must be equal.
% \end{proof}

\begin{lem}\label{lem:FUtheta}
Suppose $F:\lD \rightarrow \lE$ is a functor of double categories. The 2-cell $F_U$ is equal to $\theta_{\id_{FA}, F\id_A}$.
\end{lem}

\begin{proof}
We show that equation~\eqref{eq:comp-iso} holds when we substitute $\theta_{\id_{FA}, F\id_A}$ by $\hat{F}_U$.  Unfolding the definitions of $\eta_{U_{FA}}$, $\epsilon_{FU_A}$, and $\hat{F}_U$, and applying functoriality of $F$, we obtain an expression that can be rewritten to $U_{\id_{FA}}$. It follows that $F_U$ is a $\theta$-isomorphism, by the uniqueness of $\theta s$ in this expression.
\end{proof}

\begin{rmk}
  It is tempting to want to state a general coherence theorem along the lines of ``any two composites of $\theta$-isomorphisms having the same source and target are equal.''
  However, like statements such as ``any two composites of constraints in a monoidal category are equal'', this statement is actually literally false, because to determine a $\theta$-isomorphism requires not only a source and target but also the choice of companion data.
  If a given 1-cell is a companion of the same 1-cell in more than one way (which is the case as soon as it has any nontrivial automorphisms), then there will be different $\theta$-isomorphisms with the same source and target.
  This is analogous to how in a particular monoidal category there can be ``accidental'' composites of constraints that are not covered by the coherence theorem.
  It is probably possible to state a general coherence theorem for $\theta$-isomorphisms that is sufficiently careful to be true, but we will not need this.
\end{rmk}


\begin{rmk}
  Since all the tight constraints of a monoidal double category are invertible, to construct its underlying monoidal bicategory we only need it to have companions (and hence, by \cref{thm:comp-iso}, conjoints) for all tight \emph{isomorphisms}.
  In~\cite{gg:ldstr-tricat} double categories of this sort were called ``fibrant'', but we prefer to call them \textbf{isofibrant} to emphasize the restriction to isomorphisms; note that this condition is equivalent to asking that the (source, target) functor $\lD_1 \to \lD_0\times\lD_0$ is an ``isofibration''.
  
  To lift lax or colax monoidal \emph{functors}, and noninvertible transformations between monoidal functors, to the bicategorical level, we require our double categories to have companions (or conjoints, depending on the directions) for noninvertible tight morphisms as well.
  In~\cite{shulman:frbi} double categories with companions and conjoints for \emph{all} tight morphisms were called \emph{framed bicategories}; we might also call them \textbf{bifibrant} since in this case the (source, target) functor $\lD_1 \to \lD_0\times\lD_0$ is a bifibration (i.e.\ both a fibration and an opfibration; though in fact, it is sufficient to assume it to be one \emph{or} the other).
  In \cref{sec:1x1-to-bicat} we will see that a further condition is also required to ensure that the ``componentwise'' companion of a tight transformation is pseudo, rather than lax or colax, natural.
\end{rmk}


% Local Variables:
% TeX-master: "smbicat"
% End:


\section{From double categories to bicategories}
\label{sec:1x1-to-bicat}

We are now equipped to lift structures on fibrant double categories to
their horizontal bicategories.  In this section we show that passage
from fibrant double categories to bicategories is given by a functor of a suitable sort.

As a point of notation, we write $\odot$ for the composition of
1-cells in a bicategory, since our bicategories are generally of the
form $\cH(\lD)$.  As advocated by Max Kelly, we say \textbf{functor}
to mean a morphism between bicategories that preserves composition up
to isomorphism; equivalent terms include \emph{weak 2-functor},
\emph{pseudofunctor}, and \emph{homomorphism}.

Recall that the assignment $\cH$ sends each double category $\lC$ to the horizontal bicategory  $\cH(\lD)$ of objects, 1-cells, and globular 2-morphisms of $\lD$.  Note that functors of double categories and functors of bicategories compose strictly associatively; hence, we can talk about the 1-category of double categories and bicategories, which we denote as ${\bf Dbl}$.  

\begin{thm}\label{thm:1-func}
 If \lD\ is a double category, then $\cH(\lD)$ is a bicategory, and
  any functor $F\maps \lD\to\lE$ induces a functor $\cH(F)\maps
  \cH(\lD)\to\cH(\lE)$.  In this way $\cH$ defines a functor of
  1-categories $\mathbf{Dbl}\to \mathbf{Bicat}$.
\end{thm}
\begin{proof}
 The constraints of $F$ are all globular, hence give constraints for
  $\cH(F)$.  Functoriality is evident.
\end{proof}

Note that this is a stronger condition than we need for our main result. The action of \cH\ on transformations is less obvious. It
requires the presence of companions or conjoints, to lift the part of the data given by vertical morphisms to horizontal 1-cells. Before we discuss how this works, we briefly recall some definitions regarding transformations between functors of bicategories.

If $F,G\maps \cA\to\cB$ are functors between bicategories, then an
\textbf{oplax transformation} $\al\maps F\to G$ consists of 1-cells
$\al_A\maps FA\to GA$ and 2-cells
\[\vcenter{\xymatrix{ \ar[r]^{Ff}\ar[d]_{\al_A} \drtwocell\omit{\al_f} &  \ar[d]^{\al_B}\\
  \ar[r]_{Gf} & }}\]
such that for any 2-cell $\xymatrix{A \rtwocell^f_g{x} & B}$ in \cA,
\begin{equation}
  \label{eq:laxtransf-nat}
  \vcenter{\xymatrix@R=1pc@C=3pc{
      \rtwocell^{Ff}_{Fg}{Fx}\ar[dd]_{\al_A} 
      &  \ar[dd]^{\al_B}\\
      \drtwocell\omit{\al_g} & \\
      \ar[r]_{Gg} & }}\;=\;
  \vcenter{\xymatrix@R=1pc@C=3pc{
      \ar[r]^{Ff}\ar[dd]_{\al_A} \drtwocell\omit{\al_f} &
      \ar[dd]^{\al_B}\\ & \\
      \rtwocell^{Gf}_{Gg}{Gx} & }}
\end{equation}
and moreover for any $A$ and any $f,g$ in \cA,
\begin{equation}
  \vcenter{\xymatrix@R=5pc{
      \rtwocell^{1_{FA}}_{F(1_A)}{\iso} \ar[d]_{\al_A} \drtwocell\omit{\al_{1_A}} &  \ar[d]^{\al_A}\\
      \rtwocell^{G(1_A)}_{1_{GA}}{\iso} & }} \;=\;
  \vcenter{\xymatrix{ \ar[r]^{1_{FA}}\ar[d]_{\al_A} \drtwocell\omit{\iso}&  \ar[d]^{\al_A}\\
      \ar[r]_{1_{GA}} &
    }}
  \quad\text{and}\quad
  \vcenter{\xymatrix{
      \ar[r]|{Ff}\ar[d]_{\al_A} \drtwocell\omit{\al_f}
      \rruppertwocell^{F(gf)}{\iso}
      &
      \ar[r]|{Fg}\ar[d]|{\al_B} \drtwocell\omit{\al_g} &
      \ar[d]^{\al_C}\\
      \ar[r]|{Gf} \rrlowertwocell_{G(gf)}{\iso} & \ar[r]|{Gg} & }}
  \;=\;
  \vcenter{\xymatrix{ \ar[r]^{F(gf)}\ar[d]_{\al_A} \drtwocell\omit{\al_{gf}} &  \ar[d]^{\al_C}\\
      \ar[r]_{G(gf)} & }}\label{eq:laxtransf-ax}
\end{equation}
It is a \textbf{lax transformation} if the 2-cells $\al_f$ go the
other direction, and a \textbf{pseudo transformation} if they are
isomorphisms.

It was argued by Nick Gurski in ~\cite{nick:tricats}, that often one would like to have a stronger notion of equivalence tranformations, than simply pseudo transformations. For practical purposes, it is useful to have a notion that provides canonical pseudoinverses for all of the appropriate structure constraints, as well as the necessary modifications to exhibit this pseudoinvertibility explicitly. It turns out that pseudo natural adjoint equivalences, which we discuss below, do the job.

By doctrinal adjunction~\cite{kelly:doc-adjn}, given collections of
1-cells $\al_A\maps FA\to GA$ and $\be_A\maps GA\to FA$ and
adjunctions $\al_A\adj \be_A$ in \cB, there is a bijection between
\begin{inparaenum}
\item collections of 2-cells $\al_f$ making $\al$ an oplax
  transformation and
\item collections of 2-cells $\be_f$ making $\be$ a lax
  transformation.
\end{inparaenum}
Two such transformations correspond under this bijection if and only if
\begin{equation}
  \vcenter{\xymatrix@-.5pc{F(f) \ar[r]^-{\eta \odot \mbox{\tiny id}_{F(f)}}
      \ar[d]_{\mbox{\tiny id}_{F(f)}\odot \eta} &
      \be_B\odot \al_B \odot F(f) \ar[d]^{\mbox{\tiny id}_{\be_B} \odot \al_f}\\
      F(f) \odot \be_A\odot \al_A\ar[r]_-{\be_f \odot \mbox{\tiny id}_{\al_A}} &
      \be_B\odot G(f) \odot \al_A}}
  \quad\text{and}\quad
  \vcenter{\xymatrix@-.5pc{\al_B\odot F(f)\odot \be_A
      \ar[r]^-{\mbox{\tiny id}_{\al_B}\odot \be_f}\ar[d]_{\al_f \odot \mbox{\tiny id}_{\be_A}}&
      \al_B \odot \be_B \odot G(f)\ar[d]^{\ep \odot \mbox{\tiny id}_{G(f)}}\\
      G(f)\odot \al_A\odot \be_A \ar[r]_-{\mbox{\tiny id}_{G(f)} \odot \ep} & G(f)}}\label{eq:conjtrans}
\end{equation}
commute.  If we have a pointwise adjunction between an oplax and a lax
transformation, whose 2-cell structures correspond under this
bijection, we call it a \textbf{conjunctional transformation}
$(\al\conj \be)\maps F\to G$.  (These are the conjoint pairs in a
double category whose horizontal arrows are lax transformations and
whose vertical arrows are oplax transformations.)

Of particular importance is the case when both $\al$ and \be\ are
pseudo natural and each adjunction $\al_A\adj \be_A$ is an adjoint
equivalence.  In this case we call $\al\conj \be$ a \textbf{pseudo
  natural adjoint equivalence}.  A pseudo natural adjoint equivalence
can equivalently be defined as an internal equivalence in the
bicategory $\cBicat(\cA,\cB)$ of functors, pseudo natural
transformations, and modifications $\cA\to\cB$. 

Recall also that if $\al,\al'\maps F\to G$ are oplax transformations,
a \textbf{modification} $\mu\maps \al\to\al'$ consists of 2-cells
$\mu_A\maps \al_A\to\al'_A$ such that
\begin{equation}
  \vcenter{\xymatrix@C=1pc@R=2.5pc{ \ar[rr]^{Ff}\dtwocell_{\al'_A}^{\al_A}{\mu_A}  &
      \drtwocell\omit{\al_f} &  \ar[d]^{\al_B}\\
      \ar[rr]_{Gf} && }} \quad=\quad
  \vcenter{\xymatrix@C=1pc@R=2.5pc{ \ar[rr]^{Ff}\ar[d]_{\al'_A} \drtwocell\omit{\al'_f} && 
      \dtwocell^{\al_B}_{\al'_B}{\mu_B}\\
      \ar[rr]_{Gf} && }}\label{eq:modif-ax}
\end{equation}
There is an evident notion of modification between lax transformations
as well.  Finally, given conjunctional transformations $\al\conj\be$
and $\al'\conj \be'$, there is a bijection between modifications
$\al\to\al'$ and $\be'\to\be$, where $\mu\maps \al\to\al'$ corresponds
to $\bar{\mu}\maps \be'\to\be$ with components $\bar{\mu}_A$ defined
by:
\[\vcenter{\xymatrix@-.5pc{
    && FA \ar@{=}[drr] \ddtwocell<5>^{\al_A}_{\al'_A}{\mu_A}\\
    GA \ar[urr]^{\be'_A} \ar@{=}[drr] & \Swarrow_\ep && \Swarrow_\eta & FA\\
    &&GA\ar[urr]_{\be_A}
  }}\]
The modifications $\bar{\mu}$ and \mu\ are called \textbf{mates}, and
are compatible with composition (see \cite{ks:r2cats}).  Thus, given
$\cA,\cB$ we can define a bicategory $\Conj(\cA,\cB)$, whose objects
are functors $\cA\to\cB$, whose 1-cells are conjunctional
transformations considered as pointing in the direction of their left
adjoints, and whose 2-cells are mate-pairs of modifications.

% Note that a 2-cell $\al$ in \cDbl\ is an isomorphism just when each
%$\al_A$, \emph{and} each $\al_M$, is invertible.

\begin{thm}\label{thm:h-locfr}
  If \lD\ is a double category and \lE\ is a fibrant double category
  with chosen companions and conjoints, we have a functor of bicategories
  \begin{align}
    \cDbl(\lD,\lE) &\too \Conj(\cH(\lD),\cH(\lE))\\
    F &\mapsto \cH(F)\\
    \al &\mapsto (\alhat\conj\alchk).
  \end{align}
  Moreover, if \al\ is an isomorphism, then $\alhat\conj\alchk$ is a
  pseudo natural adjoint equivalence.
\end{thm}

Note that we are here regarding the 1-category $\cDbl(\lD,\lE)$ as a
bicategory with only identity 2-cells.

\begin{proof}
  We denote the chosen companion and conjoint of $f$ in \lE\ by \fhat\
  and \fchk, as usual.  We define $\alhat$ as follows: its 1-cell
  components are $\alhat_A = \widehat{\al_A}$, and its 2-cell
  component $\alhat_f$ is the composite
  \begin{equation}
    \vcenter{\xymatrix@R=1.5pc@C=2.5pc{
        \ar[r]|-@{|}^{U_{FA}}\ar@{=}[d] \ar@{}[dr]|{\Downarrow \eta_{\hat{\alpha}_A}} &
        \ar[r]^{Ff}\ar[d]|{\al_A} \ar@{}[dr]|{\Downarrow \al_f} &
        \ar[r]|-@{|}^{\alhat_B}\ar[d]|{\al_B} \ar@{}[dr]|{\Downarrow \epsilon_{\hat{\alpha}_B}} &
        \ar@{=}[d]\\
        \ar[r]|-@{|}_{\alhat_A} &
        \ar[r]_{Gf} &
        \ar[r]|-@{|}_{U_{GB}} & 
      }}\label{eq:oplax-2cell}
%     \vcenter{\xymatrix@R=1.5pc@C=2.5pc{
%         \ar[r]|-@{|}^{Ff}\ar@{=}[d] \ar@{}[drrr]|{\iso} &
%         \ar[rr]|-@{|}^{\alhat_B} &&
%         \ar@{=}[d]\\
%         \ar[r]|-@{|}^{U_{FA}}\ar@{=}[d] \ar@{}[dr]|{\Downarrow} &
%         \ar[r]|{Ff}\ar[d]|{\al_A} \ar@{}[dr]|{\Downarrow \al_f} &
%         \ar[r]|-@{|}^{\alhat_B}\ar[d]|{\al_B} \ar@{}[dr]|{\Downarrow} &
%         \ar@{=}[d]\\
%         \ar[r]|-@{|}_{\alhat_A} \ar@{=}[d] \ar@{}[drrr]|\iso &
%         \ar[r]|{Gf} &
%         \ar[r]|-@{|}_{U_{GB}} & \ar@{=}[d]\\
%         \ar[r]|-@{|}_{\alhat_A} & \ar[rr]|-@{|}_{Gf} &&
%       }}\label{eq:oplax-2cell}
  \end{equation}
  Equations~\eqref{eq:laxtransf-nat} and~\eqref{eq:laxtransf-ax}
  follow directly from \autoref{thm:dbl-transf}.  The construction of
  $\alchk$ is dual, using conjoints, and \autoref{thm:compconj-adj}
  shows that $\alhat_A\adj \alchk_A$.  For the first equation
  in~\eqref{eq:conjtrans}, we have
  \begin{equation}
    \vcenter{\xymatrix@-.5pc{
        \ar[r]|-@{|}^{U_{FA}}\ar@{=}[d] \ar@{}[dr]|{=} &
        \ar[r]|-@{|}^{Ff}\ar@{=}[d] \ar@{}[dr]|{=} &
        \ar[r]|-@{|}^{U_{FB}}\ar@{=}[d] \ar@{}[dr]|{\Downarrow \eta_{\hat{\alpha}_B}} &
        \ar[r]|-@{|}^{U_{FB}}\ar[d]|{\al_B} \ar@{}[dr]|{\Downarrow \eta_{\check{\alpha}_B}} &
        \ar@{=}[d]\\
        \ar[r]|{U_{FA}}\ar@{=}[d] \ar@{}[dr]|{\Downarrow \eta_{\hat{\alpha}_A}} &
        \ar[r]|{Ff}\ar[d]|{\al_A} \ar@{}[dr]|{\Downarrow\al_f} &
        \ar[r]|{\alhat_B}\ar[d]|{\al_B} \ar@{}[dr]|{\Downarrow \epsilon_{\hat{\alpha}_B}} &
        \ar[r]|{\alchk_B}\ar@{=}[d] \ar@{}[dr]|{=} &
        \ar@{=}[d]\\
        \ar[r]|-@{|}_{\alhat_A} &
        \ar[r]|-@{|}_{Gf} &
        \ar[r]|-@{|}_{U_{GB}} &
        \ar[r]|-@{|}_{\alchk_B} &
      }}\;=\;
    \vcenter{\xymatrix@-.5pc{
        \ar[r]|-@{|}^{U_{FA}}\ar@{=}[d] \ar@{}[dr]|{\Downarrow \eta_{\hat{\alpha}_A}} &
        \ar[r]|-@{|}^{Ff}\ar[d]|{\al_A} \ar@{}[dr]|{\Downarrow \al_f} &
        \ar[r]|-@{|}^{U_{FB}}\ar[d]|{\al_B} \ar@{}[dr]|{\Downarrow \eta_{\check{\alpha}_B}} &
        \ar@{=}[d]\\
        \ar[r]|-@{|}_{\alhat_A} &
        \ar[r]|-@{|}_{Gf} &
        \ar[r]|-@{|}_{\alchk_B} &
      }}\;=\;
    \vcenter{\xymatrix@-.5pc{
        \ar[r]|-@{|}^{U_{FA}}\ar@{=}[d] \ar@{}[dr]|{\Downarrow \eta_{\hat{\alpha}_A}} &
        \ar[r]|-@{|}^{U_{FA}}\ar[d]|{\al_A} \ar@{}[dr]|{\Downarrow \eta_{\check{\alpha}_A}} &
        \ar[r]|-@{|}^{Ff}\ar@{=}[d] \ar@{}[dr]|{=} &
        \ar[r]|-@{|}^{U_{FB}}\ar@{=}[d] \ar@{}[dr]|{=} &
        \ar@{=}[d]\\
        \ar[r]|{\alhat_A}\ar@{=}[d] \ar@{}[dr]|{=} &
        \ar[r]|{\alchk_A}\ar@{=}[d] \ar@{}[dr]|{\Downarrow \epsilon_{\check{\alpha}_A}} &
        \ar[r]|{Ff}\ar[d]|{\al_A} \ar@{}[dr]|{\Downarrow\al_f} &
        \ar[r]|{U_{FB}}\ar[d]|{\al_B} \ar@{}[dr]|{\Downarrow \eta_{\check{\alpha}_B}} &
        \ar@{=}[d]\\
        \ar[r]|-@{|}_{\alhat_A} &
        \ar[r]|-@{|}_{U_{GA}} &
        \ar[r]|-@{|}_{Gf} &
        \ar[r]|-@{|}_{\alchk_B} &
        }},
  \end{equation}
  and the second is dual.  Thus $(\alhat\conj\alchk)$ is a
  conjunctional transformation.

  It is left to check the axioms for functors of bicategories. Suppose we are given $\al\maps F\to G$ and $\be\maps G\to H$.  Then by
  \autoref{thm:comp-compose}, $\behat_A\odot\alhat_A$ is a companion
  of $\be_A\circ \al_A$, so we have a canonical isomorphism given by the icon
  \[\theta_{\widehat{\be\al}_A, \,\behat_A\odot\alhat_A}\maps
  \widehat{\be\al}_A \too[\iso] \behat_A\odot\alhat_A.
  \]
  Of course, we also have $\theta_{\widehat{1_A},U_A}\maps
  \widehat{1_A} \too[\iso] U_A$ by \autoref{thm:comp-unit}.  These
  constraints are automatically natural, since $\cDbl(\lD,\lE)$ has no
  nonidentity 2-cells.  The axiom for the composition constraint says
  that two constructed isomorphisms of the form
  \[\widehat{\gm\be\al}_A \too[\iso] (\gmhat_A \odot \behat_A)\odot \alhat_A\]
  are equal.  However, both $\widehat{\gm\be\al}_A$ and $(\gmhat_A
  \odot \behat_A)\odot \alhat_A$ are companions of $\gm_A\be_A\al_A$,
  and both of these isomorphisms are constructed from composites of $\theta$s; hence by
  Lemma \ref{lem:equal}, they are both equal to
  \[\theta_{\widehat{\gm\be\al}_A,\, (\gmhat_A \odot \behat_A)\odot
    \alhat_A}\] and thus equal to each other.  The same argument
  applies to the axioms for the unit constraint; thus we have a functor of bicategories.

  Finally, if $\al$ is an isomorphism, then in particular each $\al_A$
  is an isomorphism, so by \autoref{thm:compconj-adj} each
  $\alhat_A\adj \alchk_A$ is an adjoint equivalence.  But \al\ being
  an isomorphism also implies that each 2-cell
  \[\vcenter{\xymatrix@-.5pc{ \ar[r]|-@{|}^-{Ff} \ar[d]_{\al_A} \ar@{}[dr]|{\Downarrow\al_f} &  \ar[d]^{\al_B}\\
      \ar[r]|-@{|}_-{Gf} & }}\]
  is an isomorphism.  From its inverse we form the composite
  \[\vcenter{\xymatrix@R=1.5pc@C=3pc{
      \ar[r]|-@{|}^{\alhat_A}\ar@{=}[d] \ar@{}[dr]|{\Downarrow \eta_{\hat{\alpha}_A^{-1}}} &
      \ar[r]^{Gf}\ar[d]|{\al_A\inv} \ar@{}[dr]|{\Downarrow\al_f\inv} &
      \ar[r]|-@{|}^{U_{GB}}\ar[d]|{\al_B\inv} \ar@{}[dr]|{\Downarrow \epsilon_{\hat{\alpha}_B^{-1}}} &
      \ar@{=}[d]\\
      \ar[r]|-@{|}_{U_{FA}}&
      \ar[r]_{Ff} &
      \ar[r]|-@{|}_{\alhat_B} &
    }}\]
%   \[\vcenter{\xymatrix@R=1.5pc@C=3pc{
%       \ar[r]|-@{|}^{\alhat_A}\ar@{=}[d] \ar@{}[drrr]|{\iso} &
%       \ar[rr]|-@{|}^{Gf} &&
%       \ar@{=}[d]\\
%       \ar[r]|-@{|}^{\alhat_A}\ar@{=}[d] \ar@{}[dr]|{\Downarrow} &
%       \ar[r]|{Gf}\ar[d]|{\al_A\inv} \ar@{}[dr]|{\Downarrow\al_f\inv} &
%       \ar[r]|-@{|}^{U_{GB}}\ar[d]|{\al_B\inv} \ar@{}[dr]|{\Downarrow} &
%       \ar@{=}[d]\\
%       \ar[r]|-@{|}_{U_{FA}} \ar@{=}[d] \ar@{}[drrr]|\iso &
%       \ar[r]|{Ff} &
%       \ar[r]|-@{|}_{\alchk_A} & \ar@{=}[d]\\
%       \ar[r]|-@{|}_{Ff} &
%       \ar[rr]|-@{|}_{\alchk_A} &&
%     }}\]
  which we can then verify to be an inverse of~\eqref{eq:oplax-2cell}.
  Thus $\alhat$, and dually $\alchk$, is pseudo natural, and hence
  $\alhat\conj\alchk$ is a pseudo natural adjoint equivalence.
\end{proof}

We can also promote \autoref{thm:theta} to a functorial uniqueness.

\begin{lem}\label{thm:h-locfr-uniq}
  Let \lD\ be a double category and \lE\ a fibrant double category
  with two different sets of choices $\fhat,\fchk$ and $\fhat',\fchk'$
  of companions and conjoints for each vertical 1-morphism $f$, giving
  rise to two different functors
  \[\cH,\cH'\maps \cDbl(\lD,\lE)\too \Conj(\cH(\lD),\cH(\lE)).\]
  Then the isomorphisms $\theta$ from \autoref{thm:theta} fit together
  into an iconic pseudo natural adjoint equivalence $\cH\eqv \cH'$ which is the
  identity on objects.
\end{lem}
\begin{proof}
  We must first show that for a given transformation $\al\maps F\to
  G\maps \lD\to\lE$ in \cDbl, the isomorphisms \th\ correspond to 2-cells $\alhat \iso \alhat'$ of $\Conj(\cH(\lD),\cH(\lE))$; that is, they form invertible
  modifications.
  Substituting~\eqref{eq:oplax-2cell} and the definition of \th\
  into~\eqref{eq:modif-ax}, this becomes the assertion that
  \begin{equation}
    \vcenter{\xymatrix@R=1.5pc@C=2pc{
        &
        \ar[r]|-@{|}^{U_{FA}}\ar@{=}[d] \ar@{}[dr]|{\Downarrow \eta_{\hat{\alpha}_A}} &
        \ar[r]^{Ff}\ar[d]|{\al_A} \ar@{}[dr]|{\Downarrow \al_f} &
        \ar[r]|-@{|}^{\alhat_B}\ar[d]|{\al_B} \ar@{}[dr]|{\Downarrow \epsilon_{\hat{\alpha}_B}} &
        \ar@{=}[d]\\
        \ar[r]|-@{|}^{U_{FA}} \ar@{=}[d] \ar@{}[dr]|{\Downarrow \eta_{\hat{\alpha}_A'}} &
        \ar[r]|{\alhat_A} \ar[d]|{\al_A} \ar@{}[dr]|{\Downarrow \epsilon_{\hat{\alpha}_A}}&
        \ar[r]_{Gf}  \ar@{=}[d] &
        \ar[r]|-@{|}_{U_{GB}} & \\
        \ar[r]|-@{|}_{\alhat_A'} & \ar[r]|-@{|}_{U_{GB}}&&
      }} \;=\;
    \vcenter{\xymatrix@R=1.5pc@C=2pc{
        && \ar@{=}[d] \ar[r]|-@{|}^{U_{FA}} \ar@{}[dr]|{\Downarrow \eta_{\hat{\alpha}_B'}} &
        \ar[d]|{\al_B} \ar[r]|-@{|}^{\alhat_B} \ar@{}[dr]|{\Downarrow \epsilon_{\hat{\alpha}_B}}
        &
        \ar@{=}[d] &\\
        \ar[r]|-@{|}^{U_{FA}}\ar@{=}[d] \ar@{}[dr]|{\Downarrow \eta_{\hat{\alpha}_A'}} &
        \ar[r]^{Ff}\ar[d]|{\al_A} \ar@{}[dr]|{\Downarrow \al_f} &
        \ar[r]|{\alhat_B'}\ar[d]|{\al_B} \ar@{}[dr]|{\Downarrow \epsilon_{\hat{\alpha}_B'}} &
        \ar@{=}[d] \ar[r]|-@{|}_{U_{GB}}&\\
        \ar[r]|-@{|}_{\alhat_A'} &
        \ar[r]_{Gf} &
        \ar[r]|-@{|}_{U_{GB}} & .
      }}
  \end{equation}
  This follows from two applications of~\eqref{eq:compeqn}, one for
  $\alhat_A$ and one for $\alhat_B'$.  (The mate of \th\ is, of
  course, uniquely determined.)  Now, to show that these form a pseudo
  natural adjoint equivalence, it remains only to check that they do,
  in fact, form a pseudo natural transformation which is the identity
  on objects, i.e.\ that~\eqref{eq:laxtransf-nat}
  and~\eqref{eq:laxtransf-ax} are satisfied.
  But~\eqref{eq:laxtransf-nat} is vacuous since $\cDbl(\lD,\lE)$ has
  no nonidentity 2-cells, and~\eqref{eq:laxtransf-ax} follows from uniqueness of $\theta$s,
  Lemma \ref{lem:equal}, since all the constraints involved are
  also instances of \th.
\end{proof}



It seems that we should have a functor from fibrant double categories
to a tricategory of bicategories, functors, conjunctional
transformations, and modifications, but there is no tricategory
containing conjunctional transformations since the interchange law
only holds laxly.  Thus, we introduce the following definition.

\begin{defn}
  We will say that a transformation $\alpha$ in $\cDbl$ is \textbf{horizontally strong} if $\hat\alpha$ is pseudonatural, and \textbf{co-horizontally strong} if $\check \alpha$ is pseudonatural.
\end{defn}

Let $\cDbl_{\bf F}$
denote the sub-2-category of \cDbl\ containing the fibrant double
categories, all functors between them, and only horizontally-strong transformations; we regard it as a tricategory with only identity 3-cells.
Let \cBicat\ denote the tricategory of
bicategories, functors, pseudo natural transformations, and
modifications.  We aim to define a functor of tricategories $\cH\maps \cDbl^f_g\to \cBicat$.

However, to avoid having to work with the full complicated definition of tricategory, we observe that both $\cDbl_{\bf F}$ and \cBicat are \textbf{iconic tricategories}, meaning that all of its associativity and unit constraints are given by icons (i.e.\ their 1-cell component are identities).
As observed in~\cite{shulman:psalg}, an iconic tricategory can equivalently be viewed as a bicategory enriched over the bicategory \Icon\ of bicategories, functors, and icons (which has strict products).
That is, it is a structure \bS with objects, hom-bicategories $\bS(A,B)$, composition functors $\bS(B,C)\times \bS(A,B)$, identities $1_A \in \bS(A,A)$, and associativity and unitality \emph{icons} that satisfy the usual pentagon and triangle identities.
(There is no room for any higher coherence, since the data of an icon consists of 2-cells only.)
In particular, composition of 1-cells (objects of $\bS(A,B)$) is strictly associative, though the associativity for horizonatl composition of 2-cells is mediated by an icon.

Similarly, we have the notion of \textbf{iconic functor}, namely an \Icon-enriched functor of bicategories.
We can write this explicitly as follows.

\begin{defn}\label{def:Iconfunc}
Let ${\bf S,T}$ be \Icon-enriched bicategories. An \Icon-enriched functor $\cT: {\bf T} \rightarrow {\bf S}$ consists of the following data:
\begin{enumerate}
\item An assignment on objects that sends each object $A$ of ${\bf T}$ to an object $\cT A$ of ${\bf S}$.
\item For each two objects $A,B$, a functor (1-cell in \Icon) ${\bf T}(A,B) \rightarrow {\bf S}(\cT(A),\cT(B))$
\item For every triple of objects $A,B,C$ of ${\bf T}$, an invertible icon (2-cell in \Icon) 
\begin{align} 
\begin{tikzpicture}
\node(1) at (0,0) {${\bf T}(A,B) \times {\bf T}(B,C)$};
\node(2) at (5,0) {${\bf S}(\cT(A),\cT(C))$};
\draw[->] (1) to[in=155, out=25] node[above]{$\cT \circ \odot$} (2); 
\draw[->] (1) to[in=-155, out=-25] node[below]{$\odot \circ (\cT,\cT)$} (2); 
\node at (2.5,0) {$\Downarrow \phi \iso$};
\end{tikzpicture}
\end{align}
where $\odot$ denotes horizontal composition.
\item For every object $A$ of {\bf T} an invertible icon
\begin{align}
\begin{tikzpicture}[xscale=.5, yscale=.3]
\node(1) at (0,0) {$I$};
\node(2) at (5,0) {${\bf T}(A,A)$};
\node(3) at (5,-5) {${\bf S}(\cT(A),\cT(A))$};
\draw[->] (1) to node[above]{$i_{A}$} (2); 
\draw[->] (1) to node[below]{$\j_{\cT(B)}$} (3);
\draw[->] (2) to node[right]{$\cT$} (3); 
\node at (3.5,-1.5) {$\Downarrow \phi_u \iso$};
\end{tikzpicture}
\end{align}
where $i$ and $j$ are the unitors of ${\bf S}$ and ${\bf T}$.
\item The usual coherence diagrams, Definition 10 of ~\cite{nick:tricatsbook} commute.
\end{enumerate}
\end{defn}

\begin{thm}\label{thm:h-functor}
  There is an iconic functor $\cH\maps \cDbl^f_g\to \cBicat$.
\end{thm}
\begin{proof}
The first two requirements in \autoref{def:Iconfunc} are satisfied by Theorems \ref{thm:1-func} and \ref{thm:h-locfr}. The third requirement amounts to the existence of an invertible modification

\begin{equation}
\begin{tikzpicture}[xscale=0.75, yscale=1.3]
\node (tl) at (0,2) {${\cH}(H) \circ {\cH}(F)$};
\node (bl) at (0,1) {${\cH}(H \circ F)$};
\node (tr) at (4,2) {${\cH}(K) \circ {\cH}(G)$};
\node (br) at (4,1) {${\cH}(K \circ G)$};
\draw[->] (tl) to node [above] {$\hat{\beta} * \hat{\alpha}$} (tr);
\draw[->] (tr) to node [right] {$\phi_{K,G}$} (br);
\draw[->] (tl) to node [right] {$\phi_{H,F}$} (bl);
\draw[->] (bl) to node [below] {$\widehat{\be*\al}$} (br);
\node at (2,1.5) {$\Arrow{5} \phi_{\alpha,\beta} $};
\end{tikzpicture}
\end{equation}

for every pair of transformations between fibrant double categories
  \[\vcenter{\xymatrix{\lC \rtwocell^F_G{\al} & \lD \rtwocell^H_K{\be}
      & \lE}}\]
satisfying the naturality and composition constraints ~\ref{eq:laxtransf-ax} and ~\ref{thm:1-func}.

However, since composition of
  1-cells in $\cDbl_{\bf F}$ and \cBicat\ is strictly associative and
  unital, and \cH\ strictly preserves this composition (Theorem \ref{thm:1-func}),
  the 1-cell components are identities. Consequently, it is merely  a modificaton $\phi\maps \behat * \alhat \iso \widehat{\be*\al}$, such that 
 
 \begin{equation}
        \vcenter{\xymatrix@-.5pc{
        1_{{\cH}H \circ {\cH}F} \ar[r]\ar[d]_{=} &
        \hat{1}_{K}*\hat{1}_{H}\ar[d]^{\phi}\\
        1_{{\cH}(H \circ F)}\ar[r] &
        \widehat{1_{K} *1_{H}}}} \quad\text{and}\quad       
    \vcenter{\xymatrix@-.5pc{
        \widehat{\gm\al}*\widehat{\de\be} \ar[r]\ar[d]_\phi &
        (\gmhat*\dehat)\odot(\alhat*\behat)\ar[d]^{\phi \odot \phi}\\
        \widehat{\gm\al*\de\be}\ar[r] &
        (\widehat{\gm*\de})\odot(\widehat{\al*\be})}}
  \end{equation}
commute. (Here we are writing $*$ for the `Godement product' of 2-cells in $\cDbl$ and $\cBicat$.)    
Note that condition ~\ref{eq:laxtransf-nat} is satisfied because $\cDbl_{\bf F}$ has no nonidentity 3-cells.
  
  Now by Lemmas \ref{thm:comp-compose} and
  \ref{thm:comp-func}, $(\behat *\alhat)_A = \behat_{GA} \circ
  H(\alhat_A)$ is a companion of $(\be*\al)_A = \be_{GA} \circ
  H(\al_A)$.  Therefore, we take the component $\chi_A$ to be
  \[\theta_{\behat_{GA} \circ H(\alhat_A),\, \widehat{\be*\al}_A}.\]
  The equations above are satisfied by uniqueness of $\theta$-isomorphisms.
   Equation~\eqref{eq:modif-ax}, saying that these form a modification,
  becomes the equality of two large composites of 2-cells in \lD,
  which as usual follows from~\eqref{eq:compeqn}. The equations above are satisfied by uniqueness of $\theta$-isomorphisms, Lemma ~\ref{lem:equal}.

  For $\iota$, we require for every $F\maps \lD\to\lE$ an isomorphism  $\iota\maps \widehat{1_F} \iso 1_{\cH(F)}$ satisfying a couple of
  axioms which simply require it to be equal to the unit constraint of
  the local functor \cH\ from \autoref{thm:h-locfr}; these are the
  2-cell components of the unit constraint.  %Finally, the required
  %modifications merely amount to the \emph{assertions} that
  %\[\vcenter{\xymatrix@-.5pc{\gmhat*\behat*\alhat \ar[r]^\chi\ar[d]_\chi &
   %   \widehat{\gm*\be}*\alhat \ar[d]^\chi\\
   %   \gmhat*\widehat{\be*\al}\ar[r]_\chi & \widehat{\gm*\be*\al}}},\qquad
  %\vcenter{\xymatrix@-.5pc{ \alhat \ar[r]^-\iota \ar@{=}[dr] &
   %   \widehat{1_F}*\alhat \ar[d]^\chi \\ & \alhat }}, \;\text{and}\qquad
  %\vcenter{\xymatrix@-.5pc{ \alhat \ar[r]^-\iota \ar@{=}[dr] &
   %   \alhat*\widehat{1_F} \ar[d]^\chi \\ & \alhat }}\]
  %commute; again this follows from \autoref{thm:theta-compose-vert}.
\end{proof}

We observe that the enriched functor $\cH$ preserves products, in the following strict sense.
If $\mathcal{V}$ is a monoidal 2-category with strict 2-categorical finite products (such as \Icon), we say that a $\mathcal{V}$-enriched bicategory ${\bf B}$ has \textbf{finite products} when for each two objects $C,D \in {\bf B}$ composition with the projections of their categorical product defines an \emph{isomorphism} in $\mathcal{V}$ (not merely an equivalence), and similarly there is a strict terminal object $1$ such that $\bB(A,1)$ is strictly terminal in \cV.

\begin{align}
{\bf B}(A, C \times D) \xrightarrow{\cong} {\bf B}(A,C) \times {\bf B}(A,D)
\end{align}

This clearly holds for \cBicat\ and \cDbl.
There is an evident notion of when a \cV-enriched functor of bicategories \textbf{preserves products} (up to isomorphism).

\begin{thm}
The iconic functor $\cH: \cBicat \rightarrow \cDbl$ preserves products.
\end{thm}

\begin{proof}
We show that there is an isomorphism $\cH \circ (- \times -) \rightarrow (- \times -) \circ (\cH,\cH)$ for each pair of double categories $(\mathbb{D}, \mathbb{E})$.
Since $\cH$ merely forgets a part of the double categories and double functors, we have identities 
$\cH(\mathbb{D} \times \mathbb{E}) = \cH(\mathbb{D}) \times \cH(\mathbb{E})$ and  $\cH(F \times G) = \cH(F) \times \cH(G)$ for all $F,G: \mathbb{D} \rightarrow \mathbb{E}$. As a result, we only need an invertible modification $\widehat{\alpha \times \beta} \rightarrow \hat{\alpha} \times \hat{\beta}$ for all 2-cells $\alpha, \beta$ in $\mathcal{D}bl$. This modification is given by  an argument analogous to Lemma \ref{thm:h-locfr-uniq}. 
Composing with this isomorphism and the projections an isomorphism $\mathcal{B}icat({\bf A}, \cH(\mathbb{D} \times \mathbb{E})) \xrightarrow{\cong} \mathcal{B}icat({\bf A}, \cH(\mathbb{D})) \times \mathcal{B}icat({\bf A}, \cH(\mathbb{E}))$, since $\mathcal{B}icat$ has enriched products.
\end{proof}

We end this section with one final Lemma.

\begin{lem}\label{thm:theta-nat}
  Suppose $F,G\maps \lD\to\lE$ are functors of double categories, $\al\maps F\to G$ is a
  transformation, and that $f\maps A\to B$ has a companion \fhat\ in
  \lD.  Then the oplax comparison 2-cell for \alhat:
  \[\vcenter{\xymatrix{
      \ar[r]^{F(\fhat)}\ar[d]_{\alhat_A} \drtwocell\omit{\;\alhat_{\fhat}}&  \ar[d]^{\alhat_B}\\
      \ar[r]_{G(\fhat)} & }}\]
  is equal to $\theta_{\alhat_B\odot F(\fhat),\, G(\fhat) \odot
    \alhat_A}$ (and in particular is an isomorphism).
\end{lem}
\begin{proof}
  By definition $\alhat_A$ and $\alhat_B$ are companions of $\al_A$
  and $\al_B$, respectively, and by \autoref{thm:comp-func} $F(\fhat)$
  and $G(\fhat)$ are companions of $F(f)$ and $G(f)$, respectively.
  Thus, by \autoref{thm:comp-compose} the domain and codomain of
  $\alhat_{\fhat}$ are both companions of $G(f) \circ \al_A = \al_B
  \circ F(f)$, so at least the asserted $\theta$ isomorphism exists.
  Now, by taking the definition~\eqref{eq:oplax-2cell} of
  $\alhat_{\fhat}$ and substituting it for $\theta$
  in~\eqref{eq:comp-iso}, using the axioms for companions and the
  naturality of $\al$ on 2-morphisms, we see that $\alhat_{\fhat}$
  satisfies~\eqref{eq:comp-iso} and hence must be equal to $\theta$.
\end{proof}




\section{Monoidal objects in locally cubical bicategories}
\label{sec:mono-objects}

We now move on to define an appropriate abstract sort of ``monoidal object'' that will be preserved by the product-preserving functor $\cH$, and that specializes to monoidal double categories and to monoidal bicategories.
It would be nice if we could stay entirely in the world of iconic tricategories (that is, \Icon-enriched bicategories); but unfortunately the usual composition of monoidal functors between monoidal bicategories is not strictly associative, so they do not form an iconic tricategory.

However, they do form a more general structure, namely a bicategory enriched over \cDbl; in~\cite{gg:ldstr-tricat} this is called a \textbf{locally cubical bicategory}.
Since any bicategory can be regarded as a double category with only identity 1-morphisms, any iconic tricategory can be regarded as a locally cubical bicategory, but the latter are more general.
In particular, in a locally cubical bicategory the composition of 1-morphisms is associative only up to an invertible (vertical) 2-morphism.
And indeed, one of the results of~\cite{gg:ldstr-tricat} is that monoidal bicategories form a locally cubical bicategory; here we will generalize this to monoidal objects, perhaps braided and symmetric, in any iconic tricategory with finite products --- and indeed, in any locally cubical bicategory with finite products.

Since \cDbl\ is also a cartesian monoidal 2-category, we can define what it means for a locally cubical bicategory to have finite products, and this property is preserved when regarding an iconic tricategory as a locally cubical bicategory.
In particular, this applies to \cDblf\ and to \cBicat\ --- but actually, in place of the iconic tricategory \cBicat\ considered up until now we will focus instead on the locally cubical bicategory of bicategories constructed in~\cite{gg:ldstr-tricat}, whose ``locally horizontal part'' is \cBicat, but whose vertical 2-cells are \emph{icons}.
We denote this by \fBicat; it is easy to see that it also has products preserved by the inclusion $\cBicat\to \fBicat$, so that the composite functor $\cH : \cDblf \to\fBicat$ still preserves products.

We now define symmetric, braided and monoidal structures on objects, 1-cells, 2-cells, and 3-cells internal to a locally cubical bicategory with products, by taking the definitions of monoidal, braided, and symmetric structure for bicategories given in~\cite{nick:tricatsbook},~\cite{mccrudden:bal-coalgb}, and~\cite{gg:ldstr-tricat} and regarding the data of bicategories, functors, pseudonatural transformations, and modifications abstractly as objects, 1-cells, 2-cells, and 3-cells in a locally cubical bicategory.

Note that under this translation pseudonatural transformations become \emph{horizontal} 2-cells.
The horizontal 2-cells in \cDblf\ (which has no nonidentity vertical 2-morphisms) are the (vertical) transformations, while those in \fBicat\ are exactly the pseudonatural transformations (its vertical 2-morphisms are icons).
Let \fB\ be a locally cubical bicategory with products.

\begin{defn}
A {\bf monoidal object} in \fB\ is an object $A$, equipped with 1-cells $\otimes_A: A \times A \rightarrow A$ and $I_A: * \rightarrow A$, and 2-cells
\begin{itemize} 
\item $\alpha: \otimes \odot (\id \times \otimes) \Rightarrow \otimes \odot (\otimes \times \id)$
\item $l: \otimes \odot (I \times \id) \Rightarrow \id$ and $r:\otimes \odot (\id \times I) \Rightarrow \id$ 
\end{itemize}
Finally, it must be equipped with the invertible 3-cells $\pi, \mu, \lambda, \rho$, relating the two different ways around the Mac Lane pentagon and the three other coherence diagrams given in Definition 4.1 of~\cite{nick:tricatsbook}, which satisfy the appropriate three axioms.

A monoidal object is {\bf braided} if in addition there is a 2-cell $\sigma_A: \otimes \Rightarrow \otimes \circ \tau$, where $\tau: A \times A \rightarrow A \times A$ interchanges the two copies of $A$; and if there are invertible 3-cells 

\begin{equation}
  \begin{aligned}
\begin{tikzpicture}[xscale=0.9]
\node (t) at (2,3) {$\ten (\ten \times \id)$};
\node (tl) at (0,2) {$\ten(\ten \times \id)$};
\node (bl) at (0,1) {$\ten (\id \times \ten)$};
\node (b) at (2,0) {$\ten (\id \times \ten)$};
\node (tr) at (4,2) {$\ten(\id \times \ten)$};
\node (br) at (4,1) {$\ten (\ten \times \id)$};
\draw[->] (t) to node [above,xshift=10pt, yshift=-2] {$\alpha$} (tr);
\draw[->] (tr) to node [right] {$\sigma$} (br);
\draw[->] (br) to node [below,xshift=10pt, yshift=2pt] {$\alpha$} (b);
\draw[->] (t) to node [above,xshift=-10pt, yshift=-2pt] {$\sigma \otimes \mbox{id}$} (tl);
\draw[->] (tl) to node [left] {$\alpha$} (bl);
\draw[->] (bl) to node [below,xshift=-10pt,yshift=2pt] {$\mathid \otimes \sigma$} (b);
\node at (2,1.5) {$\Rightarrow_{R \iso}$};
\end{tikzpicture}
  \end{aligned}
\hspace{5pt}\mbox{and} \hspace{5pt}
\begin{aligned}
\begin{tikzpicture}[xscale=0.9]
\node (t) at (2,3) {$\ten(\id \times \ten)$};
\node (tl) at (0,2) {$\ten(\id \times \ten)$};
\node (bl) at (0,1) {$\ten(\ten \times \id)$};
\node (b) at (2,0) {$\ten(\ten \times \id)$};
\node (tr) at (4,2) {$\ten(\ten \times \id)$};
\node (br) at (4,1) {$\ten(\id \times \ten)$};
\draw[->] (tr) to node [above,xshift=10pt, yshift=-2] {$\alpha$} (t);
\draw[->] (tr) to node [right] {$\sigma$} (br);
\draw[->] (b) to node [below,xshift=10pt, yshift=2pt] {$\alpha$} (br);
\draw[->] (t) to node [above,xshift=-10pt, yshift=-2pt] {$\mbox{id} \otimes \sigma$} (tl);
\draw[->] (tl) to node [left] {${\alpha}^{-1}$} (bl);
\draw[->] (bl) to node [below,xshift=-10pt,yshift=2pt] {$\sigma \otimes \mathid$} (b);
\node at (2,1.5) {$\Rightarrow_{S \iso}$};
\end{tikzpicture}
\end{aligned}
\end{equation}
satisfying the axioms (BA1), (BA2), (BA3), and (BA4) given in~\cite[p136--139]{mccrudden:bal-coalgb} . 
It is {\bf sylleptic} when there exists an invertible 3-cell

 \[
 \begin{tikzpicture}
 \node (tl) at (-2,1) {$\ten$};
 \node (tr) at (2,1) {$\ten$};
 \node (b) at (0,-.25) {$\ten \circ \tau$};
 \draw[double] (tl)  -- (tr);
 \draw[->] (tl) to node[left, yshift=-5pt]{$\sigma$} (b);
 \draw[->] (b) to node[right, yshift=-5pt] {$\sigma$}(tr);
 \node at (0,0.5) {\footnotesize $\Downarrow \upsilon \iso$}; 
 \end{tikzpicture}
 \]
  satisfying the axioms (SA1), (SA2) on~\cite[p144--145]{mccrudden:bal-coalgb}. It is {\bf symmetric} if in addition, it satisfies the axiom given on~\cite[p91]{mccrudden:bal-coalgb}.
\end{defn}

By construction, these definitions give the expected results in \fBicat.
In \cDblf, where there are no nonidentity 3-cells, they reduce to the definitions from section~\ref{sec:symm-mono-double}; and in particular every syllepsis is a symmetry.

\begin{defn}
Let $A,B$ be monoidal objects in \fB. A 1-cell $f:A \rightarrow B$ is {\bf lax monoidal} when it is equipped with the following 2-cells:
\begin{itemize}
\item $\chi: \otimes_B \odot (f,f) \Rightarrow f \odot \otimes_A$
\item $\iota: I_B \Rightarrow f \odot I_A$
\end{itemize}
As well as invertible 3-cells $\omega, \gamma$, and $\delta$ given in Definition 4.10 of~\cite{nick:tricatsbook}, expressing the usual associativity and unitality conditions, which satisfy the two given commutativity axioms.
A monoidal 1-cell is called {\bf braided}, when $A$ and $B$ are braided and there is a 2-cell $u: \sigma_B \odot \chi  \Rightarrow \chi \odot f\sigma_A$, satisfying the braiding axioms analogous to (BHA1) and (BHA2) given in  \cite[p141-142]{mccrudden:bal-coalgb}. It is {\bf symmetric} when $A$ and $B$ are symmetric and the 3-cells defining the braided monoidal structure of $f$ satisfy the additional axiom analogous to  (SHA1) given in   \cite[p145]{mccrudden:bal-coalgb}.

When the natural transformations are in the opposite direction, the functor is {\bf oplax monoidal}, and when they are isomorphisms, the functor is {\bf strong monoidal}.
\end{defn}



\begin{defn}\label{Def:monverttrans}
Let $f, g:A \rightarrow B$ be monoidal 1-cells in \fB. A {\bf monoidal 2-cell} $\alpha: f \Rightarrow g$ is a (horizontal) 2-cell in \fB\ that is equipped with 3-cells
\begin{itemize}
\item $\Pi: \chi_f \odot \alpha \Rightarrow (\alpha \otimes \alpha) \odot \chi_g$
\item $M: \iota_f \odot \alpha \Rightarrow \iota_g$
\end{itemize}
such that the three coherence axioms in definition 3 of~\cite{gg:ldstr-tricat} hold.
A monoidal 2-cell is {\bf braided} or {\bf symmetric} when $f,g$ are braided or symmetric, and in addition the additional coherence axiom analogous to (BTA1) of~\cite[p143]{mccrudden:bal-coalgb} holds. Note that here the author writes $\rho$ for our braiding 1-cell $\sigma$.
\end{defn}

As remarked above, we will actually construct a locally cubical bicategory of monoidal objects.
The monoidal 2-cells will be the horizontal 2-cells therein; we now define the vertical 2-morphisms.

\begin{defn}
  Let $f, g:A \rightarrow B$ be monoidal 1-cells in \fB.
  A \textbf{monoidal icon} $\alpha: f \Rightarrow g$ is a (vertical) 2-morphism in \fB\ such that \dots [TODO].
\end{defn}

\begin{defn}
  A \textbf{monoidal 3-cell} is \dots [TODO].
\end{defn}

\begin{thm}
  Monoidal objects, lax monoidal 1-cells, monoidal 2-cells, monoidal icons, and monoidal 3-cells in \fB\ form a locally cubical bicategory, and similarly for colax and strong monoidal 1-cells and for braided, sylleptic, and symmetric objects and morphisms.
\end{thm}
\begin{proof}
  TODO.
\end{proof}


% Local Variables:
% TeX-master: "smbicat"
% End:


%\section{Symmetric Monoidal Bicategories}
\label{sec:constr-symm-mono}

In this section we will discuss how the functor $\mathcal{H}$ from the previous section extends to a functor from the iconic tricategories of monoidal double categories, braided monoidal double categories, and symmetric monoidal double categories, which we will denote with ${\mathcal{M}on\mathcal{D}bl^f_g}$, ${\mathcal{B}r\mathcal{M}on\mathcal{D}bl^f_g}$, and ${\mathcal{S}ym\mathcal{M}on\mathcal{D}bl^f_g}$, respectively. Their images are the iconic tricategories  of monoidal bicategories, braided monoidal bicategories, and symmetric monoidal bicategories, $\mathcal{M}on\mathcal{B}icat$, $\mathcal{B}r\mathcal{M}on\mathcal{B}icat$, and $\mathcal{S}ym\mathcal{M}on\mathcal{B}icat$, respectively.

Note that there are several related notions of each of these tricategories, depending on whether we want the functors within these tricategories to be lax monoidal, oplax monoidal, or strongly monoidal. We represent these versions with a subscript $ l,p,$ or $c$, respectively. 
As we have observed before, we restrict to transformations in $\mathcal{D}bl$ that are horizontally strong. Under this condition, we will prove the main theorem of this section, stated below. %We leave a further generalization to future work.    
%[{\it this seems out of context now..}]

\begin{thm}\label{thm:trifunctor2}
For $w=l,p,$ or $c$, the functor $\mathcal{H}: \mathcal{D}bl^f_g \rightarrow \mathcal{B}icat$ extends to functors 
\begin{align}
\mathcal{H}^M_w: \hspace{1cm} &{\mathcal{M}on\mathcal{D}bl^f_g}_w &\rightarrow \hspace{1cm} &{\mathcal{M}on\mathcal{B}icat}_w\\
\mathcal{H}^{B}_w: \hspace{1cm} &{\mathcal{B}r\mathcal{M}on \mathcal{D}bl^f_g}_w &\rightarrow \hspace{1cm} &{\mathcal{B}r\mathcal{M}on\mathcal{B}icat}_w\\ 
 \mathcal{H}^{S}_w: \hspace{1cm} &{\mathcal{S}ym\mathcal{M}on\mathcal{D}bl^f_g}_w &\rightarrow \hspace{1cm} &{\mathcal{S}ym \mathcal{M}on\mathcal{B}icat}_w
\end{align} 
\end{thm}


  
 \begin{lem}
  Let ${\bf S,T}$ be iconic tricategories with products. Any iconic trifunctor $\mathcal{T}: T \rightarrow S$ that preserves products, preserves  monoidal objects, 1-cells and 2-cell as well as any braided or symmetric structure of these objects, 1-cells and 2-cells.
 \end{lem}
 
 \begin{proof}
Let $A$ be a monoidal object. The functor $\mathcal{T}$ preserves products, so we have a product $\cT(A) \times_{\cT} \cT(A) = \cT(A \times A)$. As a consequence $\ten\maps
  A\times A\to A$ induces a 1-cell $\ten_{\mathcal{T}} \maps
 \mathcal{T}A\times_{\cT} \mathcal{T} A\to\mathcal{T}A$ with a unit 1-cell $I_{\mathcal{T}}$, where $\cT(f) \ten_{\cT} \cT(g) := \cT(f \ten g)$ and $I_{\cT}:= \mathcal{T}(I)$. 
 
 Note that conditions $(iii)$ and $(iv)$ of definition \ref{def:Iconfunc} imply that for all 1-cells $f,g$ we have the equality $\cT(f \odot g) = \cT(f) \odot \cT(g)$ and for the identity 1-cell $\id_A$, $\cT(\id_A) = \id_{\cT(A)}$. Hence, the associativity 2-cell of $A$ gives rise to a 2-cell
  \[\vcenter{\xymatrix@C=6pc{\cT(A)\times\cT(A)\times\cT(A) \rtwocell^{\ten_{\cT}
        (\Id\times\ten_{\cT})}_{\ten_{\cT}(\ten_{\cT}\times\Id)}{\hspace{.2cm}\fa_{\cT}\eqv} &\cT(A) }}\]
  which simply equals $\cT(\alpha)$.
  
  Likewise, the unit constraints $l, r$ as well as the constraints for (braided) monoidal 1-cells $\sigma$ induce 1-cells $l_{\cT}, r_{\cT}$, and $\sigma_{\cT}$, respectively.
  
 Furthermore, the invertible 3-cell filling the Mac Lane pentagon lifts to the invertible 3-cell of the Mac Lane Pentagon for $\cT(A)$. Which is simply its image under $\cT$, composed with the isomorphic $3$-cells given by (iii) of definition \ref{def:Iconfunc}, ensuring that it has the right type.
%   \[\xy
%  (-10,0)*{\ten_{\cT}(\ten_{\cT},\Id)(\ten_{\cT}, \Id, %\Id)}="A";
%  (20,10)*{\ten_{\cT}(\ten_{\cT},\Id)(\Id, \ten_{\cT},\Id)}="B";
%  (50,0)*{\ten_{\cT}(\Id,\ten_{\cT})(\Id, \ten_{\cT},\Id)}="C";
%  (0,-15)*{\ten_{\cT}(\Id, \ten_{\cT})(\ten, \Id, \Id)}="D";
%  (40,-15)*{\ten_{\cT}(\Id, \ten_{\cT})(\Id, \Id, \ten, _{\cT})}="E";
%  (20,-5)*{\scriptstyle\Downarrow \pi \iso};
%  \ar "B";"A";^{\fa_{\cT} \ten_{\cT} \id}
%  \ar "C";"B";^{\fa_{\cT}}
%  \ar "D";"A";_{\fa_{\cT}}
%  \ar "E";"D";_{\fa_{\cT}}
%  \ar "E";"C";^{\id\ten_{\cT} \fa_{\cT}}
%  \endxy
%  \]
  
\begin{tikzpicture}[yscale=1.5, xscale=3]
\node(tl) at (0,1) {$\ten (\ten \times \Id)(\ten \times \Id \times \Id)$};
\node(t) at (1.5,2) {$\ten (\ten \times \Id)(\Id \times \ten \times \Id)$};
\node(tr) at (3,1) {$\ten (\Id \times \ten )(\Id \times \ten \times \Id)$};
\node(br) at (3,0) {$\ten (\Id \times \ten )(\Id \times \Id \times \ten )$};
\node(b) at (1.5,-1) {$\ten (\ten \times \Id)(\Id \times \Id \times \ten )$};
\node(bl) at (0,0) {$\ten (\Id \times \ten )(\ten \times \Id \times \Id)$};
\draw[->] (tl) to node[left, yshift=1pt] {$\alpha \ten \id$} (t);
\draw[->] (t) to node[right, yshift=1pt] {$\alpha$} (tr);
\draw[->] (tr) to node[right] {$\id \ten \alpha$} (br);
\draw[->] (tl) to node[left] {$\alpha$} (bl);
\draw[->] (bl) to node[left,yshift=-1pt] {$\id$} (b);
\draw[->] (b) to node[right,yshift=-1pt] {$\alpha$} (br);
\draw[->] (tl) to [in=155, out=5] (br);
\draw[->] (tl) to [in=180, out=-10] (br);
\draw[->] (tl) to [in=180, out=10](tr);
\draw[->] (bl) to [in=185, out=0](br);
\node at (1.5,.6) {$\Downarrow \cT(\pi) \iso$};
\node at (2.5,.6) {$\Downarrow \phi^{-1} \iso$};
\node at (1.5,1.5) {$\Downarrow \phi^{-1} \iso$};
\node at (.5,.4) {$\Downarrow \phi \iso$};
\node at (1.5,-.5) {$\Downarrow \phi \iso $};
\end{tikzpicture}  

 Note that there may be several way to past these 3-cells, but by coherence of pseudofunctors, the result is the same. Likewise, the isomorphic 3-cells $\mu, \lambda,\rho$ and the isomorphic 3-cells $R,S$, and $v$ witnessing the braiding and syllepsis, lift to the appropriate 3-cell for $\cT(A)$. This is also the case for the isomorphic 3-cells $R,S$ that characterise the braiding. 
   Finally, we need to show that the three equations between pasting composites of $\pi_{\cT}, \mu_{\cT}, \lambda_{\cT}, \rho_{\cT}$ hold ???

The argument for 1-cells and 2-cells is similar. 
Let $f,g:A \rightarrow B$ be monoidal 1-cells and let $\alpha: f \Rightarrow g$ be a monoidal 2-cell.
The 2-cells $\chi: \otimes_B \odot (f,f) \Rightarrow f \odot \otimes_A$ and $\iota: I_B \Rightarrow f \odot I_A$ are mapped to $\cT(\chi): \otimes_{\cT(B)} \odot (\cT(f),\cT(f)) \Rightarrow \cT(f) \odot \otimes_{\cT(A)}$ and $\iota: I_{\cT(B)} \Rightarrow \cT(f) \odot I_{\cT(A)}$. Similarly, any 2-cell witnessing a braiding $u: \sigma_B \odot \chi  \Rightarrow \chi \odot f\sigma_A$  is mapped to $\cT(u): \sigma_{\cT(B)} \odot \cT(\chi)  \Rightarrow \cT(\chi) \odot \cT(f)\sigma_{\cT(A)}$.
Analogously to $\alpha$, the invertible 3-cells $\omega, \gamma$, and $\delta$, and $\Pi$ and $M$ are lifted to their images under $\cT$, augmented with instances of $\phi$ to ensure that the 3-cells have the right type.
 It is left to show that the equations pasting composites of these 3-cells together commute.
 \end{proof}
 

\subsection{Lifting Symmetric Monoidal Structures}
We are now ready to lift monoidal structures from double categories to
bicategories.  If we had a theory of symmetric monoidal tricategories,
we could do this by improving \autoref{thm:h-functor} to say that
$\cH$ is a symmetric monoidal functor, and then conclude that it
preserves pseudomonoids.  However, in the absence of such a theory, we
give a direct proof.

\begin{rmk}\label{rmk:sym}
  For monoidal bicategories, there is a notion in between braided and
  symmetric, called \emph{sylleptic}, in which the the braiding is
  self-inverse up to an isomorphism (the \emph{syllepsis}) but this
  isomorphism is not maximally coherent.  Since in our approach the
  syllepsis will be an isomorphism of the form
  $\theta_{\fhat,\fhat'}$, it is \emph{always} maximally coherent;
  thus our method cannot produce sylleptic monoidal bicategories that
  are not symmetric.
\end{rmk}

\begin{lem}\label{thm:mon11-monbi}
  If \lD\ is a {\bf fibrant monoidal double category}, then $\cH(\lD)$ is a
  monoidal bicategory.  
\end{lem}


\begin{proof}[Proof of \autoref{thm:mon11-monbi}]
  A monoidal bicategory is defined to be a tricategory with one
  object.  We use the definition of tricategory
  from~\cite{nick:tricats}, which is the same as that
  of~\cite{gps:tricats} except that the associativity and unit
  constraints are pseudo natural adjoint equivalences, rather than
  merely pseudo transformations whose components are equivalences.

  The functor \cH\ evidently preserves products, so $\ten\maps
  \lD\times\lD\to\lD$ induces a functor $\ten\maps
  \cH(\lD)\times\cH(\lD)\to \cH(\lD)$, and of course $I$ is still the
  unit.  The associativity constraint of \lD\ is a natural isomorphism
  \[\vcenter{\xymatrix@C=5pc{\lD\times\lD\times\lD \rtwocell^{\ten
        (\Id\times\ten)}_{\ten(\ten\times\Id)}{\fa\iso} &\lD }}\]
  so by \autoref{thm:h-locfr} it gives rise to a pseudo natural
  adjoint equivalence
  \[\vcenter{\xymatrix@C=6pc{\cH(\lD)\times\cH(\lD)\times\cH(\lD) \rtwocell^{\ten
        (\Id\times\ten)}_{\ten(\ten\times\Id)}{\fahat\eqv} &\cH(\lD) }}\]
  Likewise, the unit constraints of \lD\ induce pseudo natural adjoint
  equivalences.

  The final four pieces of data for a monoidal bicategory are
  invertible modifications relating various composites of the
  associativity and unit transformations.  The first is a
  ``pentagonator'' which relates the two ways to go around the Mac
  Lane pentagon:
  \[\xy
  (-10,0)*{((A\ten B)\ten C)\ten D}="A";
  (20,10)*{(A\ten (B\ten C))\ten D}="B";
  (50,0)*{A\ten ((B\ten C)\ten D)}="C";
  (0,-15)*{(A\ten B)\ten (C\ten D)}="D";
  (40,-15)*{A\ten (B\ten (C\ten D))}="E";
  (20,-5)*{\scriptstyle\pi\Downarrow\iso};
  \ar "B";"A";^{\fahat\ten U_D}
  \ar "C";"B";^{\fahat}
  \ar "D";"A";_{\fahat}
  \ar "E";"D";_{\fahat}
  \ar "E";"C";^{U_A\ten \fahat}
  \endxy
  \]
By Lemma's ~\ref{thm:comp-unit},  ~\ref{thm:comp-compose}, and ~\ref{thm:comp-ten}, the outside of this diagram commutes.  
 As a result, we can construct $\pi$ by Lemma \ref{lem:modification}. The other invertible modifications $\mu, \lambda$, and $\rho$ are constructed in the same way.

  Finally, we must show that three equations between pasting
  composites of 2-cells hold, relating composites of
  $\pi,\mu,\lambda,\rho$.  However, in each of these equations, both
  the domain and the codomain of the 2-cells involved are companions
  of the same isomorphism in $\lD_0$.  For the 5-associahedron, this
  isomorphism is the unique constraint
  \[(((A\ten B)\ten C)\ten D)\ten E \too[\iso] A\ten (B\ten (C\ten
  (D\ten E)));
  \]
  for the other two it is simply the associator $(A\ten B)\ten C
  \too[\iso] A\ten (B\ten C)$.  By Lemma \ref{lem:modification},
  every 2-cell in these diagrams is a $\theta$-isomorphism relating
  two companions of the same vertical isomorphism.  Therefore, Lemma ~\ref{lem:equal} implies that the three equations hold.
\end{proof}


  Now suppose that \lD\ is braided; to show that $\cH(\lD)$ is braided
  we seemingly must first have a definition of braided monoidal
  bicategory.  The interested reader may follow the tortuous path of
  the definition of braided monoidal 2-categories and bicategories
  through the literature, starting from~\cite{kv:2cat-zam,kv:bm2cat}
  and continuing, with occasional corrections,
  through~\cite{bn:hda-i,ds:monbi-hopfagbd,crans:centers,mccrudden:bal-coalgb},
  and~\cite{gurski:brmonbicat}.  However, the details of the
  definition are essentially unimportant for us; since our constraints
  and coherence are produced in a universal way, any reasonable data
  can be produced and any reasonable axioms will be satisfied.  For
  concreteness, we use the definition of~\cite{mccrudden:bal-coalgb}.

\begin{lem}\label{thm:br11-brbi}
  If \lD\ is a {\bf fibrant braided monoidal double category}, then $\cH(\lD)$ is a
 braided monoidal bicategory.  
\end{lem}

\begin{proof}
  The first piece of data we require to make $\cH(\lD)$ braided is a
  pseudo natural adjoint equivalence $\mathord{\otimes} \too[\eqv]
  \mathord{\otimes}\circ \tau$, where $\tau$ is the switch
  isomorphism.  This arises by \autoref{thm:h-locfr} from the braiding $\sigma$
  of \lD.  We also require two invertible modifications filling the
  usual hexagons for a braiding:
  \begin{equation}
  \begin{aligned}
\begin{tikzpicture}[xscale=0.9]
\node (t) at (2,3) {$(A \otimes B) \otimes C$};
\node (tl) at (0,2) {$(B \otimes A) \otimes C)$};
\node (bl) at (0,1) {$B \otimes  (A \otimes C)$};
\node (b) at (2,0) {$B \otimes (C \otimes A)$};
\node (tr) at (4,2) {$A \otimes (B \otimes C)$};
\node (br) at (4,1) {$(B \otimes C) \otimes A$};
\draw[->] (t) to node [above,xshift=10pt, yshift=-2] {$\hat{\alpha}$} (tr);
\draw[->] (tr) to node [right] {$\hat{\sigma}$} (br);
\draw[->] (br) to node [below,xshift=10pt, yshift=2pt] {$\hat{\alpha}$} (b);
\draw[->] (t) to node [above,xshift=-10pt, yshift=-2pt] {$\hat{\sigma} \otimes \mbox{id}$} (tl);
\draw[->] (tl) to node [left] {$\hat{\alpha}$} (bl);
\draw[->] (bl) to node [below,xshift=-10pt,yshift=2pt] {$\mathid \otimes \hat{\sigma}$} (b);
\node at (2,1.5) {$\Leftarrow_{\zeta \iso}$};
\end{tikzpicture}
  \end{aligned}
\hspace{5pt}\mbox{and} \hspace{5pt}
\begin{aligned}
\begin{tikzpicture}[xscale=0.9]
\node (t) at (2,3) {$A \otimes (B \otimes C)$};
\node (tl) at (0,2) {$A \otimes (C \otimes B)$};
\node (bl) at (0,1) {$(A \otimes  C) \otimes B$};
\node (b) at (2,0) {$(C \otimes A) \otimes B$};
\node (tr) at (4,2) {$(A \otimes B) \otimes C$};
\node (br) at (4,1) {$C \otimes (A \otimes B)$};
\draw[->] (tr) to node [above,xshift=10pt, yshift=-2] {$\hat{\alpha}$} (t);
\draw[->] (tr) to node [right] {$\hat{\sigma} $} (br);
\draw[->] (b) to node [below,xshift=10pt, yshift=2pt] {$\hat{\alpha}$} (br);
\draw[->] (t) to node [above,xshift=-10pt, yshift=-2pt] {$\mbox{id} \otimes \hat{\sigma}$} (tl);
\draw[->] (tl) to node [left] {$\hat{\alpha}^{-1}$} (bl);
\draw[->] (bl) to node [below,xshift=-10pt,yshift=2pt] {$\hat{\sigma} \otimes \mathid$} (b);
\node at (2,1.5) {$\Leftarrow_{\xi \iso}$};
\end{tikzpicture}
\end{aligned}
\end{equation}

  As before, since the corresponding hexagons commute in $\lD_0$, by Lemma ~\ref{lem:modification} we have $\theta$-isomorphisms that we can take as $\ze$
  and $\xi$.  Finally, we must verify that the four 2-cell diagrams
  in~\cite[p136--139]{mccrudden:bal-coalgb} involving \ze\ and \xi\
  commute.  As with the axioms for a monoidal bicategory, both sides
  of these equalities are made up of $\theta$s relating companions of
  a single morphism in $\lD_0$, and thus by uniqueness they must be
  equal.

%   for a Gray monoid these can be
%   found in~\cite[p216--217]{bn:hda-i}
%   and~\cite[p191--193]{crans:centers}

%   According to~\cite{crans:centers}, there is also a missing unit
%   axiom in the definition of braided Gray monoid
%   in~\cite{bn:hda-i,ds:monbi-hopfagbd}.  In a braided monoidal
%   bicategory, this axiom should become an isomorphism
%   \[\vcenter{\xymatrix{A\ten I \ar[rr] \ar[dr] &
%       \ar@{}[d]|-{\Downarrow\iso}& 
%       I\ten A \ar@{<-}[dl]\\ & A}}\]
%   We produce this, and its coherence, in the same way.
\end{proof}

\begin{lem}\label{thm:sym11-symbi}
  If \lD\ is a {\bf fibrant symmetric monoidal double category}, then $\cH(\lD)$ is a
 symmetric monoidal bicategory.  
\end{lem}

\begin{proof}
Suppose that \lD\ is symmetric.  To make $\cH(\lD)$ symmetric,
  we require first a \emph{syllepsis}, i.e.\ an invertible
  modification
  \[
 \begin{tikzpicture}
 \node (tl) at (-2,1) {$A \otimes B$};
 \node (tr) at (2,1) {$A \otimes B$};
 \node (b) at (0,-.25) {$B \otimes A$};
 \draw[double] (tl)  -- (tr);
 \draw[->] (tl) to node[left, yshift=-5pt]{$\hat{\sigma}$} (b);
 \draw[->] (b) to node[right, yshift=-5pt] {$\hat{\sigma}$}(tr);
 \node at (0,0.5) {\footnotesize $\Downarrow \nu\iso$}; 
 \end{tikzpicture}
 \]
  Since the braiding in $\lD_0$ is self-inverse, the top and bottom of
  this triangle are both companions of $1_{A\ten B}$; thus we have a
  $\theta$-isomorphism between them which we take as $\nu$.  For
  $\cH(\lD)$ to be sylleptic, the syllepsis must satisfy the two
  axioms on~\cite[p144--145]{mccrudden:bal-coalgb}.  As before, these
  diagrams of 2-cells are made up entirely of $\theta$s relating
  companions of a single morphism in $\lD_0$, so they commute by
  uniqueness of $\theta$.

% ; the first two relate
%   companions of associativities and the last two of units.

% .  For a sylleptic Gray monoid, these
%   axioms can be found in~\cite[p128]{ds:monbi-hopfagbd}
%   and~\cite[p207]{crans:centers}; for a sylleptic monoidal bicategory
%   they are in

  Finally, for $\cH(\lD)$ to be symmetric, the syllepsis must satisfy
  one additional axiom, given on~\cite[p91]{mccrudden:bal-coalgb}.
  This follows automatically for the same reasons as before.
%  in~\cite[p131]{ds:monbi-hopfagbd}
%   and~\cite[p208]{crans:centers}.
\end{proof}




\begin{rmk}
  Essentially the same proof as that of \autoref{thm:mon11-monbi}
  shows that any fibrant 2x1-category has an underlying tricategory.
  Note that unlike the construction of bicategories from
  1x1-categories (i.e.\ double categories), this case requires
  fibrancy even in the absence of monoidal structure, since the
  associativity and unit constraints of a 2x1-category are not 1-cells
  but rather morphisms of 0-cells.  There are many naturally occurring
  fibrant symmetric monoidal 2x1-categories, such as $\lD_0=$
  commutative rings, $\lD_1=$ algebras, and $\lD_2=$ modules, or the
  symmetric monoidal 2x1-category of \emph{conformal nets} defined
  in~\cite{bdh:confnets-i}.  All of these have underlying
  tricategories, which will be symmetric monoidal for any reasonable
  definition of symmetric monoidal tricategory.  More generally, as
  stated in \S\ref{sec:introduction}, we expect any fibrant $(n\times
  k)$-category to have an underlying $(n+k)$-category.
\end{rmk}

Now we will show that the functor $\cH$ lifts monoidal double functors to monoidal pseudo functors. We use definition 3.3.1 of ~\cite{nick:tricats} of a trifunctor, between tricategories with one object. From now on we will assume that all double categories are fibrant.

\begin{lem}\label{lem:monfun}
If $F$ is a {\bf strong monoidal double functor}, $\cH(F)$ is a monoidal pseudo functor, if $F$ is {\bf lax monoidal double functor}, $\cH(F)$ is a lax monoidal pseudo functor, and if $F$ is an {\bf oplax monoidal double functor}, $\cH(F)$ is an oplax monoidal pseudo functor. 
\end{lem}

\begin{proof}
Let $\D, \E$ be monoidal double categories, and let $F:\D \rightarrow \E$ be a strong monoidal double functor. 
By Theorem ~\ref{thm:h-locfr}, the vertical transformations $\phi: \otimes \circ (F,F) \rightarrow F \circ \otimes $ and $\phi_u: I_{\E} \rightarrow F(I_{\D})$, are mapped by $\cH$ to the pseudo natural adjoint equivalences $\hat{\phi}$ and $\hat{\phi_u}$.

We need to show that there exist invertible modifications relating  different compositions of $F$, $\hat{\phi}$, $\hat{\phi_u}$, the associator, and the unitors. 
The components of the first one of these modifications are given below.

\begin{equation}
\begin{tikzpicture}[scale=1.5]
\node (t) at (2,3) {$(FA \otimes FB) \otimes FC$};
\node (tl) at (0,2) {$FA \otimes (FB \otimes FC)$};
\node (bl) at (0,1) {$FA \otimes  F(B \otimes C)$};
\node (b) at (2,0) {$F((A \otimes (B \otimes C))$};
\node (tr) at (4,2) {$F(A \otimes B) \otimes FC$};
\node (br) at (4,1) {$F((A \otimes B) \otimes C)$};
\draw[->] (t) to node [above,xshift=20pt] {$\hat{\phi}_{A,B} \otimes U_{FC}$} (tr);
\draw[->] (tr) to node [right] {$\hat{\phi}_{A \otimes B,C} $} (br);
\draw[->] (br) to node [below,xshift=20pt] {$F\hat{\alpha}_{A,B,C}$} (b);
\draw[->] (t) to node [above,xshift=-10pt] {$\hat{\alpha}_{FA, FB, FC}$} (tl);
\draw[->] (tl) to node [right] {$U_{FA} \otimes \hat{\phi}_{B,C} $} (bl);
\draw[->] (bl) to node [below,xshift=-10pt] {$\hat{\phi}_{A, B \otimes C}$} (b);
\node at (2,1.5) {$\Downarrow \omega_{A,B,C}$};
\end{tikzpicture}
\end{equation} 

By Lemmas ~\ref{thm:comp-unit}
and ~\ref{thm:comp-ten}, the morphisms above are companions first coherence diagram for monoidal functors. As a result, the existence of $\omega$ follows from Lemma \ref{lem:modification}. The other modifications, $\delta$ and $\gamma$, are constructed in the same way.

Finally, we need to show that the equalities between compositions of structure modifications hold, as described in definition 3.3.1 of ~\cite{nick:tricats}. The source and target of these composites are $F(((A \otimes B) \otimes C) \otimes D) \rightarrow FA \otimes ( FB \otimes (FC \otimes FD)))$, and $F(A \otimes B) \rightarrow F(A \otimes (I \otimes B))$, respectively.  Each individual composed 2-cell is a $\theta$-isomorphism, either by definition, or by Lemma \ref{thm:theta-nat}. As a result, the equality follows from uniqueness of compositions of $\theta$-isomorphisms.

By the same argument, one can show that oplax/lax monoidal double functors are lifted by $\mathcal{H}$ to lax/oplax monoidal pseudofunctors. In the latter case, the arrows corresponding to all instances of $\phi$ are reversed. 
\end{proof}


\begin{lem}\label{lem:brfun}
Let $(F, \phi) :\D \rightarrow \E$ be a {\bf braided monoidal double functor}, then $\cH(F):\cH(\D) \rightarrow \cH(\E)$ is a braided monoidal pseudofunctor. 
\end{lem}

\begin{proof}
We follow the definition of a braided monoidal pseudo functor given in ~\cite{mccrudden:bal-coalgb}.
We need to show that there exists an invertible modification with the following components:

\begin{tikzpicture}[xscale=4,yscale=2]
\node (tl) at (0,1) {$FA \otimes FB$};
\node (tr) at (1,1) {$FB \otimes FA$};
\node (bl) at (0,0) {$F(A \otimes B)$};
\node (br) at (1,0) {$F(B \otimes A)$};
\node at (0.5,0.5){$\Downarrow \mu_{a,B}$};
\draw[->] (tl) to node [above]{$\hat{\sigma}_{FA,FB}$} (tr);
\draw[->] (bl) to node [above]{$F\hat{\sigma}_{A,B}$} (br);
\draw[->](tl) to node [left]{$\hat{\phi}_{A,B}$} (bl);
\draw[->] (tr) to node [right]{$\hat{\phi}_{B,A}$} (br);
\end{tikzpicture}
 
As before, this modification is induced by the characteristic diagram for braided monoidal functors, by Lemma ~\ref{lem:modification}.
It remains to show that the two diagrams of 2-cells given  in ~\cite{mccrudden:bal-coalgb} commute. The domain and codomain of these compositions are of the form $(FA \otimes FB) \otimes FC  \rightarrow F(B \otimes (C \otimes A))$  and $(FA \otimes FB) \otimes FC) \rightarrow F((C \otimes A) \otimes B)$ in $\E_0$.
Each individual composed 2-cell is a $\theta$-isomorphism, either by definition, or by Lemma ~\ref{thm:theta-nat}. As a result, the equality follows from uniqueness of $\theta$-isomorphisms. 
\end{proof}

\begin{lem}\label{lem:symfun}
Let $F:(\D, \phi) \rightarrow (\E, \phi)$ be a {\bf symmetric monoidal double functor}, then $\cH(F): \cH(\D) \rightarrow \cH(\E)$ is a symmetric monoidal pseudofunctor. 
\end{lem}

Recall that by remark ~\ref{rmk:sym}, any syllepsis obtained by our construction is a $\theta$-isomorphism, which makes it a symmetry. This means that it is sufficient to show that symmetric monoidal double functors are mapped by $\cH$ to sylleptic monoidal pseudo functors. 

Recall that a sylleptic bicategory is a braided bicategory with an additional invertible modification $\nu: \id \rightarrow \hat{\sigma} \odot \hat{\sigma}$. 

Following the definition given in \cite{mccrudden:bal-coalgb}, 
we need to show that the equality between the compositions of 2-cells below holds. 

\begin{equation}
\begin{aligned}
\begin{tikzpicture}[xscale=1.2, yscale=1.8]
\node (t) at (2,3) {$FB \otimes FA$};
\node (tl) at (0,2) {$FA \otimes FB$};
\node (bl) at (0,1) {$F(A\otimes B)$};
\node (tr) at (4,2) {$FA \otimes FB$};
\node (br) at (4,1) {$F(A\otimes B)$};
\draw[->] (t) to node [above] {$\hat{\sigma}$} (tr);
\draw[->] (tr) to node [right] {$\hat{\phi}_{A,B}$} (br);
\draw[->] (tl) to node [above] {$\hat{\sigma}$} (t);
\draw[->] (tl) to node [right] {$\hat{\phi}_{A,B}$} (bl);
\draw[->] (tl) to node [above] {$U_{FA \otimes FB}$} (tr);
\draw[->] (bl) to[in=-135, out=-45] node [below] {$FU_{A\otimes B}$} (br);
\draw[->] (bl) to node [below] {$U_{F(A \otimes B)}$} (br);
\node at (2,2.6) {$\nu \Uparrow$};
\node at (2,1.5) {$U_{\hat{\phi}_{A,B}} \Downarrow$};
\node at (2, 0.4) {$\hat{F}_U \Downarrow$};
\end{tikzpicture}
\end{aligned}
=
\begin{aligned}
\begin{tikzpicture}[xscale=1.2, yscale=1.5]
\node (t) at (2,3) {$FB \otimes FA$};
\node (tl) at (0,2) {$FA \otimes FB$};
\node (bl) at (0,1) {$F(A\otimes B)$};
\node (m) at (2,2) {$F(B\otimes A)$};
\node (tr) at (4,2) {$FA \otimes FB$};
\node (br) at (4,1) {$F(A\otimes B)$};
\draw[->] (t) to node [above] {$\hat{\sigma}$} (tr);
\draw[->] (t) to node [right] {$\hat{\phi}_{B,A}$} (m);
\draw[->] (tr) to node [right] {$\hat{\phi}_{A,B}$} (br);
\draw[->] (tl) to node [above] {$\hat{\sigma}$} (t);
\draw[->] (tl) to node [right] {$\hat{\phi}_{A,B}$} (bl);
\draw[->] (bl) to node [left,yshift=10pt, xshift=14pt] {$F\hat{\sigma}_{A,B}$} (m);
\draw[->] (m) to node [right,xshift=-7,yshift=6] {$F{\hat{\sigma}}_{B,A}$} (br);
\draw[->] (bl) to[in=-135, out=-45] node [below] {$FU_{A\otimes B}$} (br);
\draw[->] (bl) to node [below] {$F({\hat{\sigma}}_{B,A} \odot {\hat{\sigma}}_{A, B})$} (br);
\node at (1.1,2.2) {$\mu \Downarrow$};
\node at (2.9,2.2) {$\mu \Downarrow$};
\node at (2,1.3) {$ \hat{F}_{\odot} \Downarrow$};
\node at (2, 0.3) {$F\nu \Uparrow$};
\end{tikzpicture}
\end{aligned}
\end{equation} 

Each individual 2-cell is a $\theta$-isomorphism. For $U_{\hat{\phi}}$ and $\hat{F}_{\odot}$, 
this follows from Lemma ~\ref{thm:theta-nat}, for $\hat{F}_{U}$ this follows from Lemma ~\ref{lem:FUtheta}, since $\hat{F}_{U}$ equals $F_U$ and for $F \nu$ this follows from Lemma ~\ref{thm:theta-func}. The other 2-cells are $\theta$-isomorphisms by definition.
As a result, we can apply Lemma \ref{lem:equal} to obtain the equality.  

\begin{lem}\label{lem:montran}
Let $\beta: (F, \phi) \rightarrow (G,\psi)$ be a {\bf monoidal vertical transformation}. The image of $\beta$ under $\mathcal{H}$, $\hat{\beta}: (F, \phi) \rightarrow (G,\psi)$, is a monoidal pseudonatural adjoint equivalence.
\end{lem}

\begin{proof}
Following Definition 3 of ~\cite{gg:ldstr-tricat} of a pseudo icon, applied to tricategories with one object, we need to show that there exist invertible modifications $\kappa$ and $\lambda$ of the form:

\begin{tikzpicture}[xscale=4, yscale=2]
\node (tl) at (0,1) {$FA \otimes FB$};
\node (bl) at (0,0) {$F(A \otimes B)$};
\node (tr) at (1,1) {$GA \otimes GB$};
\node (br) at (1,0) {$G(A \otimes B)$};
\draw[->] (tl) to node [above] {$\hat{\beta}_A \otimes \hat{\beta}_B$} (tr);
\draw[->] (bl) to node [below] {$\hat{\beta}_{A \otimes B}$} (br);
\draw[->] (tl) to node [left] {$\hat{\phi}_{A,B}$} (bl);
\draw[->] (tr) to node [right] {$\hat{\psi}_{A,B}$} (br);
\node at (0.5,0.5) {$\Downarrow \kappa$};
\end{tikzpicture}
\hspace{1cm}
\begin{tikzpicture}[xscale=4,yscale=2]
\node (t) at (0.5,1) {$I_{\E}$};
\node (bl) at (0,0) {$FI_{\D}$};
\node (br) at (1,0) {$GI_{\D}$};
\draw[->] (t) to node [left] {$\hat{\phi}_U$} (bl);
\draw[->] (t) to node [right] {$\hat{\psi}_U$} (br);
\draw[->] (bl) to node [below] {$\hat{\beta}_{I}$} (br);
\node at (0.5,0.4) {$\Downarrow \iota$};
\end{tikzpicture}

This follows from Lemma \ref{lem:modification}. 
It is left to check that the three coherence axioms in definition 3 of ~\cite{gg:ldstr-tricat} hold. These correspond to equalities of compositions of modifications, whose domain and codomain correspond are the isomorphisms $I_{\D} \otimes FA \rightarrow GA$, $FA \otimes I_{\D} \rightarrow GA$, and $(FA \otimes FB) \otimes FC \rightarrow G(A \otimes (B \otimes C))$.
The first of the three equalities is given by the diagrams below.

\begin{equation}
\begin{aligned}
\begin{tikzpicture}[scale=1.7]
\node (llb) at (0,1) {$I \otimes GA$};
\node (llt) at (0,2) {$I \otimes FA$};
\node (bbl) at (1,0) {$GA$};
\node (bml)at (1,1) {$FA$};
\node (tl) at (0.8,3) {$FI \otimes FA$};
\node (bbr) at (2,0) {$GA$};
\node (bmr) at (2,1) {$FA$};
\node (tr) at (2.2,3) {$F(I \otimes A)$};
\node (brr) at (3,1) {$GA$};
\node (trr) at (3,2) {$FA$};
\draw[->] (llb) to node [below]{$\hat{l}$} (bbl);
\draw[->] (llt) to node [left]{$\id \otimes \hat{\beta}_A$} (llb);
\draw[->] (llt) to node [above,xshift=-12,yshift=-2]{${\hat{F}_U} \otimes \id$} (tl);
\draw[->] (llt) to node [above]{$\hat{l}$} (bml);
\draw[->] (tl) to node [above]{$\hat{\phi}_{I,A}$} (tr);
\draw[double] (bml) to (bmr);
\draw[double] (bbl) to (bbr);
\draw[double] (bbr) to (brr);
\draw[->] (tr) to node [above,xshift=2]{$F\hat{l}$} (trr);
\draw[double] (bmr) to (trr);
\draw[->] (trr) to node [right]{$\hat{\beta}_A$} (brr);
\draw[->] (bml) to node [right]{$\hat{\beta}_A$} (bbl);
\draw[->] (bmr) to node [right]{$\hat{\beta}_A$} (bbr);
\node at (1.5, 2) {$\Downarrow \gamma$};
\node at (0.5, 1) {$\hat{l}_{\hat{\beta}}$};
\node at (1.5, 0.5) {$=$};
\node at (2.5, 1) {$=$};
\end{tikzpicture}
\end{aligned}
=
\begin{aligned}
\begin{tikzpicture}[scale=1.7]
\node (llb) at (0,1) {$I \otimes GA$};
\node (llt) at (0,2) {$I \otimes FA$};
\node (bbl) at (1,0) {$GA$};
\node (bml) at (0.8,2) {$GI \otimes GA$};
\node (tl) at (0.8,3) {$FI \otimes FA$};
\node (bbr) at (2,0) {$GA$};
\node (bmr) at (2.2,2) {$G(I \otimes A)$};
\node (tr) at (2.2,3) {$F(I \otimes A)$};
\node (brr) at (3,1) {$GA$};
\node (trr) at (3,2) {$FA$};
\draw[->] (llb) to node [below]{$\hat{l}$} (bbl);
\draw[->] (llt) to node [left]{$\id \otimes \hat{\beta}_A$} (llb);
\draw[->] (llt) to node [above,xshift=-16pt,yshift=-6pt]{${\hat{F}_U} \otimes \id$} (tl);
\draw[->] (llb) to node [below, xshift=16,yshift=6]{$ \hat{G}_U \otimes \id$} (bml);
\draw[->] (tl) to node [above]{$\hat{\phi}_{I,A}$} (tr);
\draw[->] (bml) to node [below] {$\psi_{I,A}$} (bmr);
\draw[double] (bbl) to (bbr);
\draw[double] (bbr) to (brr);
\draw[->] (tr) to node [above,xshift=4, yshift=-2]{$F\hat{l}$} (trr);
\draw[->] (bmr) to node [below,xshift=-10,yshift=8] {$G \hat{l}$} (brr);
\draw[->] (trr) to node [right]{$\hat{\beta}_A$} (brr);
\draw[->] (tl) to node [right,xshift=-2,yshift=-2]{$\hat{\beta}_I \otimes \hat{\beta}_A$} (bml);
\draw[->] (tr) to node [left,xshift=2,yshift=-2]{$\hat{\beta}_{I \otimes A}$} (bmr);
\node at (1.5, 2.7) {$\Downarrow \kappa$};
\node at (0.3, 1.6) {$\Downarrow$};
\node at (0.3, 1.8) {$\iota \otimes \id$};
\node at (2.7,1.8) {$\Downarrow \hat{\beta}_{\hat{l}}$};
\node at (1.5, 1) {$\Downarrow \gamma$};
\end{tikzpicture}
\end{aligned}
\end{equation}

All individual 2-cells are $\theta$-isomorphisms, by Lemmas ~\ref{thm:comp-ten} and ~\ref{thm:theta-nat}. The equality follows by uniqueness of $\theta$-isomorphisms.
\end{proof}

\begin{lem}\label{lem:brtran}
Let $\beta: (F, \phi) \rightarrow (G,\psi)$ be a {\bf braided monoidal vertical transformation}. The image of $\beta$ under $\mathcal{H}$, $\hat{\beta}: (F, \phi) \rightarrow (G,\psi)$ is a braided monoidal pseudonatural adjoint equivalence.
\end{lem}

\begin{proof}
We need to verify that axiom (BTA1) of ~\cite{mccrudden:bal-coalgb} holds, which is given below.

\begin{equation}
\begin{aligned}
\begin{tikzpicture}[scale=1.3]
\node (tl) at (1,4) {$FB \otimes FA$};
\node (tr) at (3,4) {$F(B \otimes A)$};
\node (ml) at (0,2) {$FA \otimes FB$};
\node (mm) at (2,2) {$F(A \otimes B)$};
\node (mr) at (4,2) {$G(B \otimes A)$};
\node (bl) at (1,0) {$GA \otimes GB$};
\node (br) at (3,0) {$G(A \otimes B)$};
\draw[->] (tl) to node [above] {$\hat{\phi}_{B,A}$} (tr);
\draw[->] (ml) to node [left] {$\hat{\rho}_{FA,FB}$} (tl);
\draw[->] (ml) to node [above] {$\hat{\phi}_{A,B}$} (mm);
\draw[->] (mm) to node [left] {$F\hat{\rho}_{A,B}$} (tr);
\draw[->] (tr) to node [right] {$\hat{\beta}_{B\otimes A}$} (mr);
\draw[->] (mm) to node [left] {$\hat{\beta}_{A\otimes B}$} (br);
\draw[->] (br) to node [right] {$G\hat{\rho}_{A,B}$} (mr);
\draw[->] (bl) to node [below] {$\hat{\phi}_{A,B}$} (br);
\draw[->] (ml) to node [left] {$\hat{\beta}_A \otimes \hat{\beta}_B$} (bl);
\node at (1.2,3) {$\Rightarrow \mu$};
\node at (1.2,1) {$\Rightarrow \kappa$};
\node at (3,2.5) {$\hat{\beta}_{\hat{\beta}_{A,B}}\Uparrow$};
\end{tikzpicture}
\end{aligned}
\hspace{0.25cm}
=
\hspace{0.25cm}
\begin{aligned}
\begin{tikzpicture}[scale=1.3]
\node (tl) at (1,4) {$FB \otimes FA$};
\node (tr) at (3,4) {$F(B \otimes A)$};
\node (ml) at (0,2) {$FA \otimes FB$};
\node (mm) at (2,2) {$GB \otimes GA$};
\node (mr) at (4,2) {$G(B \otimes A)$};
\node (bl) at (1,0) {$GA \otimes GB$};
\node (br) at (3,0) {$G(A \otimes B)$};
\draw[->] (tl) to node [above] {$\hat{\phi}_{B,A}$} (tr);
\draw[->] (ml) to node [left] {$\hat{\rho}_{FA,FB}$} (tl);
\draw[->] (tl) to node [right] {$\hat{\beta}_B \otimes \hat{\beta}_A$} (mm);
\draw[->] (mm) to node [above] {$\hat{\phi}_{A,B}$} (mr);
\draw[->] (tr) to node [right] {$\hat{\beta}_{B\otimes A}$} (mr);
\draw[->] (bl) to node [right] {$\hat{\rho}_{GA,GB}$} (mm);
\draw[->] (br) to node [right] {$G\hat{\rho}_{A,B}$} (mr);
\draw[->] (bl) to node [below] {$\hat{\phi}_{A,B}$} (br);
\draw[->] (ml) to node [left] {$\hat{\beta}_A \otimes \hat{\beta}_B$} (bl);
\node at (1,2.5) {$\hat{\rho}_{\hat{\beta}_A,\hat{\beta}_B} \Uparrow $};
\node at (2.5,3.2) {$\Rightarrow \kappa$};
\node at (2.5,1.2) {$\Rightarrow \mu$};
\end{tikzpicture}
\end{aligned}
\end{equation} 

All individual 2-cells are $\theta$-isomorphisms, either by Lemma \ref{thm:theta-nat} or by definition. The equality follows from Lemma \ref{lem:equal}.
\end{proof}

\begin{cor}\label{cor:symtran}
Let $\beta: (F, \phi) \rightarrow (G,\psi)$ be a {\bf symmetric monoidal vertical transformation}. The image of $\beta$ under $\mathcal{H}$, $\hat{\beta}: (F, \phi) \rightarrow (G,\psi)$, is a symmetric monoidal pseudonatural adjoint equivalence.
\end{cor}

\begin{proof}
Symmetric monoidal adjoint equivalences are simply braided adjoint equivalences between symmetric monoidal pseudofunctors. 
\end{proof}

This brings us to the proof of our theorem.

\begin{proof}[Proof of Theorem \ref{thm:trifunctor2}]
Combining Lemmas \ref{lem:monfun}, \ref{lem:brfun},  \ref{lem:symfun}, \ref{lem:montran}, \ref{lem:brtran} and Corollary \ref{cor:symtran} gives the required result.
\end{proof}


% Local Variables:
% TeX-master: "smbicat"
% End:


%\section{Intercategories of monoidal objects}
\label{sec:intercat}

In the previous section we constructed, from a product-preserving functor of locally cubical bicategories, \emph{twelve} new functors between locally cubical bicategories of monoidal objects according to whether the objects are braided, sylleptic, symmetric, or none, and whether the morphisms are lax, colax, or strong.
We now show that this menagerie can be reduced to only \emph{four} functors (according to the choice of objects), by incorporating lax, colax, and strong monoidal morphisms into a single structure.

One dimension down, the relevant structure is a \emph{strict} double category, with lax and colax morphisms as its horizontal and vertical arrows; see e.g.~\cite{gp:double-adjoints,shulman:dblderived}.
In our categorified case, we will use the \emph{intercategories} of~\cite{gp:intercategories-i,gp:intercategories-ii}.


% Local Variables:
% TeX-master: "smbicat"
% End:


\section{An example}\label{sec:Alg} 

\subsection*{Monoids and bimodules in double categories}
\label{sec:mod}

As a demonstration of our method, we apply Theorem~\ref{thm:H} to prove that the family of bicategories $\mathcal{A}lg[{\mathbb{D}}]$ of monoids, bimodules and bimodule homomorphisms in a monoidal double category $\mathbb{D}$ is monoidal. The result of this section builds on work of the first author in~\cite[Theorem 11.5]{shulman:frbi} and some extentions of this theorem in the second author's PhD thesis~\cite[Chapter 5]{westerPhDthesis}.

The family $\mathcal{A}lg[{\mathbb{D}}]$ contains the well-known bicategories $\cat{2Vect}$, $\cat{2Hilb}$  introduced in~\cite{kapranov562},~\cite{baez2004higher}, the bicategory \cat{Prof}~\cite{benabou}. Additionally, it contains various subcategories that are relevant in the field of categorical quantum mechanics, such as the equivariant completion of a braided monoidal bicategory~\cite{carquevillerunkel}, which is a tool for finding topological quantum field theories, as well as the bicategory $2[CP^*[{\bf C}]]$ defined in~\cite{heunenvicarywester} as the mathematical foundation of a diagrammatic language for quantum protocols. %For full generality, we formulate the result for the categories of special, dagger, Frobenius, commutative, and symmetric monoids, as well as dagger bimodules.

A {\bf monoid} $(a, A, \mult, \unit)$ in a monoidal double category $\mathbb{D}$ consists of an object $a$, a loose 1-cell $A: a \mapsto a$ and globular 2-cells $\mult: A \odot A \rightarrow A$, $\unit: I_a \rightarrow A$, such that $\mult \circ (\mult \odot \id) = \mult \circ (\id \odot \mult)$ and $\mult \circ (\unit \odot \id) = \id = \mult \circ (\id \odot \mult)$.
In other words, it is a monoid in the usual sense in the monoidal category $\mathbb{D}(a,a)$. A {\bf monoid homomorphism} $(A,\tinymult[gray dot], \tinyunit[gray dot]) \mapsto (B,\tinymult[black dot], \tinyunit[black dot])$ is a pair $(f, {\bf f})$ of a 1-morphism $f:a\rightarrow b$ in $\mathbb{D}$ and a 2-cell ${\bf f}:A \rightarrow B$ in $\mathbb{D}$ that respects the multiplication, $f \circ \tinymult[gray dot] = \tinymult[black dot] \circ (f \odot f)$, as well as the unit, $\tinyunit[black dot] = f \circ \tinyunit[gray dot]$. $f: A \rightarrow B$. A {\bf bimodule} $(a,A,\tinymult[gray dot], \tinyunit[gray dot]) \mapsto (b, B,\tinymult[black dot], \tinyunit[black dot])$ is a pair (${\bf M}, M)$ of a $1-cell$ $M$ in $\doub{D}$ and a globular 2-cell ${\bf M}:A \times M \times B \rightarrow M$ in $\doub{D}$, with the structure of an $A$-$B$- bimodule. We will simply write $M$ for the bimodule $({\bf M}, m)$. This is also called an {\bf A-B-bimodule}. Note that $S(M) = T(A)$ and $T(M) = S(B)$. Let $\phi: (a,A,\tinymult[gray dot], \tinyunit[gray dot]) \rightarrow (c,C,\tinymult[black dot], \tinyunit[black dot])$ and $\psi: (b,B,\tinymult[gray dot], \tinyunit[gray dot]) \rightarrow (d,D,\tinymult[black dot], \tinyunit[black dot])$ be monoid homomorphisms and let $M$ and $N$ be an $A$-$B$-bimodule and a $C$-$D$-bimodule, respectively. A {\bf $(\phi, \psi)$-equivariant map} is a 2-morphism $f:M \rightarrow N$ in ${\doub{D}}$ such that ${\bf M} \circ (\phi \tens f \tens \psi) = f \circ {\bf N}$.

In other sources~\cite{westerthesis}~\cite{jamiesbook}~\cite{heunenvicarywester},~\cite{carquevillerunkel},  monoids are called algebras and equivariant maps correspond to extended bimodule homomorphisms. 


%We apply Theorem~\ref{thm:lcbcfunctor} to the double category below.

\begin{defn}
Let ${\mathbb{D}}$ be a double category. We can define a new double category $\mathbb{A}lg[{\mathbb{D}}]$ consisting of the elements listed below.

\begin{itemize}
\item 0-cells are {\bf monoids} $(a, A,\tinymult[gray dot], \tinyunit[gray dot])$, in the monoidal category ${\mathbb{D}_1}$. 
\item 1-morphisms $(A,\tinymult[gray dot], \tinyunit[gray dot]) \mapsto (B,\tinymult[black dot], \tinyunit[black dot])$ are {\bf monoid homomorphism}
\item horizontal 1-cells $(A,\tinymult[gray dot], \tinyunit[gray dot]) \mapsto (B,\tinymult[black dot], \tinyunit[black dot])$ are {\bf bimodules}
\item 2-cells $({\bf M},M) \rightarrow ({\bf N},N)$ from a $A$-$B$-bimodule to a $C$-$D$-bimodule are $(\phi, \psi)$-equivariant maps, where $\phi:  A\rightarrow C$, $\psi: B \rightarrow D$ are monoid homomorphisms.  
\end{itemize}
%\begin{itemize}
%\item 0-cells are {\bf monoids} $(A,\tinymult[gray dot], \tinyunit[gray dot])$, in the monoidal category ${\mathbb{D}_1}$. Note that this requires that $S(A)=T(A)$.
%\item 1-cells $(A,\tinymult[gray dot], \tinyunit[gray dot]) \mapsto (B,\tinymult[black dot], \tinyunit[black dot])$ are 1-cells ${\bf M}:A \rightarrow B$ with the structure of an $A-B-$bimodule.
%\item 2-cells are 2-morphisms in ${\cat B}$ that are bimodule homomorphisms.
%\end{itemize}
Structural data regarding this construction, such as horizontal composition, is described in~\cite{shulman:frbi}, where the double category is called $\mathbb{M}od(\mathbb{D})$. A more detailed description in the case that $\mathbb{D}$ is a monoidal category is given in~\cite{westerPhDthesis}.
\end{defn}

For the double category to be well-defined, we need certain coequalizers in $\mathbb{D}$ to exist. To this end we introduce the definition below, which is derived from~\cite[Definition 11.4]{shulman:frbi}.

\begin{defn}
We say that a monoidal double category $\mathbb{D}$ has {\bf local coequalizers} if for each object $a$, the hom-category $\mathbb{D}(a,a)$ has all coequalizers and the coequalizers are preserved by $\odot$. We write $\mathcal{D}bl^l$ for the 2-category of double categories with local coequalizers and $\cDblf^l$ for the 2-category of double categories with local coequalizers and loosely strong companions.
\end{defn}

\begin{prop}\label{thm:eqcomp}
Let $\mathbb{D}$ be a monoidal double category with local coequalizers and companions and conjoints, and such that the tensor product $\tens$ preserves the local coequalizers. The loose bicategory $\mathcal{A}lg[\mathbb{D}]$ of $\mathbb{A}lg[\mathbb{D}]$ is monoidal; it is braided or symmetric whenever $\mathbb{D}$ is braided or symmetric.
\end{prop}

\begin{proof}
It was shown in~\cite[Examples 9.2]{shulman:frbi} that the doube category above is symmetric monoidal and has loosely strong companions and conjoints. Therefore, we may apply Theorem~\ref{thm:lcbcfunctor}, which states that the horizontal bicategory $\mathcal{A}lg[\mathbb{D}]$ is symmetric monoidal.
\end{proof}


An explicit description of the monoidal structure for the special case when ${\mathbb{D}}$ is a monoidal category was given in~\cite{westerPhDthesis}. 

Similarly, monoidal functors $F,G:\mathbb{D} \rightarrow \mathbb{E}$ between double categories lift to monoidal functors of the form $\mathbb{A}lgF,\mathbb{A}lgG: \mathcal{A}lg[\mathbb{D}] \rightarrow \mathcal{A}lg[\mathbb{E}]$ and monoidal transformations $\alpha: F \Rightarrow G$ lift to monoidal transformations of the form $\mathbb{A}lg\alpha: \mathbb{A}lgF \Rightarrow \mathbb{A}lgG$. In fact, the $\mathbb{A}lg$ construction gives rise to a functor, as shown below. 

\begin{prop}\label{prop:funcAlg}
The $\bAlg$ construction gives rise to the following pseudo functors
\begin{align*}
\mathcal{M}on_s\mathcal{D}bl^l _f\rightarrow \mathcal{M}on\mathcal{B}icat\\
\mathcal{B}r\mathcal{M}on_s\mathcal{D}bl^l_f \rightarrow \mathcal{B}r\mathcal{M}on\mathcal{B}icat\\
\mathcal{S}ym\mathcal{M}on_s\mathcal{D}bl^l_f \rightarrow \mathcal{S}ym\mathcal{M}on\mathcal{B}icat\\
\end{align*}
\end{prop}

\begin{proof}
By~\cite[Proposition11.22]{shulman:frbi}, $\mathbb{A}lg$ gives rise to a functor $\mathcal{M}on_s\mathcal{D}bl_f^l \rightarrow \mathcal{M}on_s \mathcal{D}bl_f^l$, as well as the braided and symmetric versions. We compose this with the functor from Theorem~\ref{thm:H} to obtain the result
\end{proof}

\begin{cor}
The bicategory $\cat{2Vect}$ is symmetric monoidal.
\end{cor}

\begin{proof}
We obtain $\cat{2Vect}$ as $\cAlg[\cat{FVect}]$ from the braided monoidal category $\cat{FVect}$ of finite dimensional vector spaces and linear maps. The category $\cat{FVect}$ is symmetric monoidal, contains local coequalisers and has a tensor product that preserves coequalisers. The result followis from Proposition~\ref{prop:funcAlg}.
\end{proof}

\begin{cor}
The bicategory $\cat{Prof}$ is symmetric monoidal.
\end{cor}

\begin{proof}
The bicategory $\cat{Prof}$ can be constructed as the category $\cAlg[Span[\cat{C}]]$ from the monoidal double category of spans~\cite[Examples 4.2]{shulman:frbi}.
This category has local coequalisers and loosely strong companions, and if \cat{C} has limits, it is monoidal~\cite[Examples 4.15, 9.2]{shulman:frbi}. The result followis from Proposition~\ref{prop:funcAlg}.
\end{proof}

\subsection*{Applications in Quantum Theory}
Frobenius algebras and modules play an important role in quantum theory. 

\begin{defn}
A {\bf Frobenius algebra} in a monoidal double category ${\cat{C}}$ is a monoid $(A, \tinymult, \tinyunit)$ together with a comonoid 
$(a, A, \tinycomult, \tinycounit)$ that satisfies the equation $
(\id \tens \tinymult[gray dot]) \circ \alpha \circ (\tinycomult[gray dot] \tens \id) = (\tinymult[gray dot] \tens \id) \circ \alpha^{-1} \circ (\id \tens \tinycomult[gray dot] )$.
A monoid in a braided monoidal category {\cat C} is called {\bf commutative} when $
\tinymult[gray dot] \circ \sigma = \tinymult[gray dot]$;
it is {\bf symmetric} if the weaker condition 
$\tinycounit[gray dot] \circ \tinymult[gray dot] \circ \sigma = \tinycounit[gray dot] \circ \tinymult[gray dot]
$ holds.
A pair of a monoid  $\tinymult[gray dot]$ and a comonoid $\tinycomult[black dot]$  is called {\bf special} when the equation $ \tinymult[gray dot] \circ \tinycomult[black dot] = \id$ holds.
\end{defn}


Recently, the bicategory of Frobenius algebras, bimodules and bimodule homomorphisms in a monoidal bicategory $\cat{B}$ was introduced in~\cite{carquevillerunkel} as the {\it equivariant completion }of $\cat{B}$. This is a tool for generating topological quantum field theories, which is a non-trivial process in general.

    
% A {\bf dagger monoidal category} is a monoidal category that is equiped with a dagger $\dagger$, such that the equalities below hold.
 
% \begin{align*}
% (f \otimes g)^{\dagger} &= g^{\dagger} \otimes f^{\dagger}\\
% \alpha^{\dagger} &= \alpha^{-1} \\
%  \rho^{\dagger} &= \rho^{-1} \\
  % \lambda^{\dagger} &= \lambda^{-1} 
 %\end{align*}
 
% A {\bf dagger braided monoidal category} is a dagger monoidal category with a braiding, such that the equation below holds
 
% \begin{equation*}
 %   \sigma^{\dagger} = \sigma^{-1} \\
% \end{equation*}
 
 
% A {\bf monoidal dagger functor} $F:{\bf C} \rightarrow {\bf D}$ is a functor between monoidal dagger categories that preserves the dagger, which means that $F\circ \dagger = \dagger \circ F$. ?????
%\end{defn}

%We will write $\bAlg_{S}[{\bf C}]$ and $\cAlg_{S}[{\bf C}]$ for $S\subset\{spec, \mbox{Frob}, c, sym\}$ for the sub-double categories and sub-bicategories of $\bAlg[{\bf C}]$ and $\cAlg[{\bf C}]$, respectively, where the objects are restricted to the type of monoids specified by $S$. We denote special by $spec$, Frobenius by $\mbox{Frob}$, commutative by $c$ and symmetric by $sym$. 


Let $\doub{D}$ be a double category. The {\bf equivariant completion} of the horizontal bicategory $\cH(\doub{D})$ is the horizontal bicategory of the double category $\mathbb{E}q[{\mathbb{D}}]$ defined below.  

\begin{defn}
Let ${\mathbb{D}}$ be a double category with local coequalisers and loosely strong companions, where the tensor product preserves coequalisers. We can define a new double category $\mathbb{E}q[{\mathbb{D}}]$ consisting of the elements listed below.


\begin{itemize}
\item 0-cells are {\bf Frobenius algebras} $(a, A,\tinymult[gray dot], \tinyunit[gray dot])$, in the monoidal category ${\mathbb{D}_1}$. 
\item 1-morphisms $(A,\tinymult[gray dot], \tinyunit[gray dot]) \mapsto (B,\tinymult[black dot], \tinyunit[black dot])$ are {\bf monoid homomorphism}
\item horizontal 1-cells $(A,\tinymult[gray dot], \tinyunit[gray dot]) \mapsto (B,\tinymult[black dot], \tinyunit[black dot])$ are {\bf bimodules}
\item 2-cells $({\bf M},M) \rightarrow ({\bf N},N)$ from a $A$-$B$-bimodule to a $C$-$D$-bimodule are $(\phi, \psi)$-equivariant maps, where $\phi:  A\rightarrow C$, $\psi: B \rightarrow D$ are monoid homomorphisms.  
\end{itemize}

Structural data regarding this construction, such as horizontal composition, is as for $\mathbb{A}lg[{\mathbb{D}}]$.
\end{defn}

\begin{cor}
The equivariant completion of a monoidal bicategory $\cat{B}$ is monoidal. It is braided whenever $\cat{B}$ is braided and it is symmetric whenever $\cat{B}$ is symmetric.
\end{cor}

\begin{proof}
This follows from Proposition~\ref{thm:eqcomp} applied to the double category $\mathbb{E}q[\cat{B}]$, where \cat{B} is regarded as a double category with trivial 1-morphisms.
\end{proof}


Another related example in quantum theory is the bicategory of $\cAlg[\mathbb{D}]$ special dagger Frobenius algebras, bimodules and bimodule homomorphisms in a monoidal bicategory~\cite{heunenvicarywester}, which was introduced as a mathematical foundation for a diagrammatic language of quantum protocols. Examples are the well-known bicategory $\cat{2Hilb}$ and the bicategory $2[CP[{\bf C}]]$ of mixed quantum states. The monoidal structure is essential for such applications, as it enables the description of compound quantum systems, as well as parallel quantum protocols. 


\begin{defn}
Let ${\bf C}$ be a category. A dagger $\dagger: {\cat C} \rightarrow {\cat C}$, is a contravariant functor which is the identity on objects such that $\dagger(\dagger(f)) = f$. 
A {\bf \index{dagger monoidal category}dagger monoidal category} is a monoidal category that is equipped with a dagger $\dagger$, such that the equalities below hold.
 \begin{align*}
 (f \otimes g)^{\dagger} &= g^{\dagger} \otimes f^{\dagger}\\
 \alpha^{\dagger} &= \alpha^{-1} \\
  \rho^{\dagger} &= \rho^{-1} \\
   \lambda^{\dagger} &= \lambda^{-1} 
 \end{align*}
A {\bf \index{dagger braided monoidal category}dagger braided monoidal category} is a dagger monoidal category with a braiding that satisfies the equality below.
 \begin{equation*}
    \sigma^{\dagger} = \sigma^{-1} \\
 \end{equation*}
 A Frobenius algebra in a dagger braided monoidal category $\cat{C}$ is a {\bf dagger Frobenius algebra} when the comonoid is the dagger image of the monoid. 
A bimodule is called a {\bf dagger bimodule} when the equation below holds, where the comonoid is the dagger of the monoid and we denote $\dagger({\bf M})$ by ${\bf M}^{\dagger}$.

\begin{equation}
{\bf M}^{\dagger} = (\id \tens {\bf M }\tens \id) \circ (\tinycomult[white dot] \tens \id \tens \tinycomult[black dot]) \circ (\tinyunit[white dot] \tens \id \tens \tinyunit[black dot])
\end{equation}
\end{defn}

\begin{defn}
Let $f: A\rightarrow B$ be an algebra homomorphism between dagger Frobenius algebras in a dagger category. The {\bf conjugate} $f_*$ of $f$ is defined as $f_* := (\id \otimes \tinycounit[gray dot]) \circ (\id \otimes \tinymult[gray dot]) \circ (\id \otimes f^{\dagger} \otimes \id) \circ (\tinycomult[gray dot] \otimes \id) \circ (\tinyunit[gray dot] \otimes \id)$
An algebra homomorphism $f$ is {\bf self-conjugate} if $f=f_*$.
\end{defn}



To prove that the $2[-]$ construction preserves braided monoidal structure, one could define a suitable notion of dagger double category and prove that Proposition~\ref{thm:eqcomp} can be specialised to the case where monoids are special dagger Frobenius algebras and bimodules are dager bimodules. Instead, we make use of the direct proof of the fibrant and monoidal structure of de double category $\ltwo[\cat{C}]$ defined in~\cite{westerPhDthesis}, for the special case that $\cat{C}$ is a monoidal dagger category,~\cite[Prop 5.4.25]{westerPhDthesis}. This double category consists of special dagger Frobenius algebras, self-conjugate algebra homomorphisms, dagger bimodules and equivariant maps.

\begin{thm}
The assignment $2[-]$ gives rise to the functors of locally cubical bicategories below,
\begin{align*}
 \mathcal{B}r\mathcal{M}on\mathcal{C}at \rightarrow \mathcal{B}r\mathcal{M}on\mathcal{B}icat\\
 \mathcal{S}ym\mathcal{M}on\mathcal{C}at \rightarrow \mathcal{S}ym\mathcal{M}on\mathcal{B}icat
\end{align*}
Where $\mathcal{B}r\mathcal{M}on\mathcal{C}at$ and $\mathcal{S}ym\mathcal{M}on\mathcal{C}at$ are the locally cubical bicategories of braided monoidal bicategories and symmetric monoidal bicategories, respectively, where all coequalizers exist and the tensor product preserves coequalizers. these locally cubical bicategories have only trivial loose 2-cells and 3-cells
\end{thm}

\begin{proof}
This follows directly from~\cite[Proposition5.4.25]{westerPhDthesis} and Theorem~\ref{thm:eqcomp}.
\end{proof}

Similarly, one can show that the result holds for the bicategory of commutative or symmetric dagger Frobenius algebras, dagger bimodules and bimodule homomorphisms.


\begin{cor}
The bicategory $\cat{2Hilb}$ is symmetric monoidal.
\end{cor}

\begin{proof}
As shown in~\cite[Section 3.6.3]{westerthesis}, $\cat{2Hilb}$ is equivalent to $2[\cat{FHilb}]$, where $\cat{FHilb}$ is the symmetric monoidal category of finite Hilbert spaces and linear maps, which contains all coequalizers.
\end{proof}

\subsection*{Black-boxing of open Markov processes}
\label{sec:markov}

As another example, the authors of~\cite{bc:markov} construct a symmetric monoidal double category $\mathbb{M}\mathsf{ark}$ whose objects are finite sets and whose loose 1-cells are ``open Markov processes''.
The loose composition and the tensor product then give two related ways to put together smaller open Markov processes into larger ones (and eventually into closed ones).
In addition, they constructed a symmetric monoidal \emph{functor}, called \textbf{black-boxing}, from this double category to a double category $\mathbb{L}\mathsf{inRel}$ of linear relations.

The authors of~\cite{bc:markov} also showed that both of these double categories have companions for tight isomorphisms, and then used our results to conclude that their loose bicategories $\mathcal{M}\mathit{ark}$ and $\mathcal{L}\mathit{inRel}$ are symmetric monoidal bicategories.
At the time of writing~\cite{bc:markov}, only the earlier version~\cite{shulman:smbicat} of this paper was available, which constructed monoidal bicategories but not monoidal functors between them; thus, the authors of~\cite{bc:markov} were only able to conjecture that their black-boxing double functor induced a symmetric monoidal functor of bicategories.
However, with Theorem~\ref{thm:lcbcfunctor} now in hand, we can prove their conjecture:

\begin{thm}[{\cite[Conjecture 6.7]{bc:markov}}]
  There exists a symmetric monoidal functor of bicategories $\blacksquare : \mathcal{M}\mathit{ark} \to \mathcal{L}\mathit{inRel}$ that maps
  \begin{enumerate}
  \item any finite set $S$ to the vector space $\blacksquare(S) = \mathbb{R}^S \oplus \mathbb{R}^S$,
  \item any open Markov process $S \xto{i} (X,H) \xot{o} T$ to the linear relation
    \[ \blacksquare(S \xto{i} (X,H) \xot{o} T) \subseteq \mathbb{R}^S\oplus \mathbb{R}^S\oplus \mathbb{R}^T\oplus \mathbb{R}^T \]
    consisting of all 4-tuples $(i^*(v),I,o^*(v),O)$ where $v\in \mathbb{R}^X$ is some steady state with inflows $I$ and outflows $O$ (see~\cite[Definition 2.7]{bc:markov}), and
  \item any globular morphism of open Markov processes
    \[
      \begin{tikzcd}
        & {(X,H)} \arrow[dd, "p"] &  \\
        S \arrow[ru, "i_1"] \arrow[rd, "i_1'"'] &  & T \arrow[lu, "o_1"'] \arrow[ld, "o_1'"] \\
        & {(X',H')} & 
      \end{tikzcd}
    \]
    to the inclusion $\blacksquare(X,H) \subseteq \blacksquare(X',H')$.
  \end{enumerate}
\end{thm}
\begin{proof}
  Apply Theorem~\ref{thm:lcbcfunctor} to the symmetric monoidal double functor of~\cite[Theorem 5.5]{bc:markov}.
\end{proof}


% Local Variables:
% TeX-master: "smbicat"
% End:

 
\bibliographystyle{alpha}
\bibliography{smbicat}

\appendix
\section{Coherence Equations for Locally-Double Bicategories}
\label{ap:coherence}

We give the coherence equations for  monoidal objects, 1-cells, 2-cells and icons in a locally cubical bicategory, defined in Section~\ref{sec:mono-objects}. For readability, we omit certain cells, arising as coherence constraints for double categories. Where we do so, we write $\hat{\alpha}$ for the composition of a 3-cell ${\alpha}$ with coherence cells that ensure that the source and target 2-cells are of the right type.

\subsubsection*{Monoidal Object}

%
\documentclass[12pt]{ociamthesis}
\usepackage{tikz}
\usepackage{amsmath}
\usepackage{rotating}

\usepackage{amssymb,amsmath,stmaryrd,txfonts,mathrsfs,amsthm}
\usepackage[all,2cell]{xy}
\usepackage[neveradjust]{paralist}
\usepackage{hyperref}
\usepackage{mathtools}
\usepackage{tikz}
\usetikzlibrary{trees}
\usetikzlibrary{topaths}
\usetikzlibrary{decorations}
\usetikzlibrary{decorations.pathreplacing}
\usetikzlibrary{decorations.pathmorphing}
\usetikzlibrary{decorations.markings}
\usetikzlibrary{matrix,backgrounds,folding}
\usetikzlibrary{chains,scopes,positioning,fit}
\usetikzlibrary{arrows,shadows}
\usetikzlibrary{calc} 
\usetikzlibrary{chains}
\usetikzlibrary{shapes,shapes.geometric,shapes.misc}
\usepackage{smbicat}


\makeatletter
\let\ea\expandafter

%% Defining commands that are always in math mode.
\def\mdef#1#2{\ea\ea\ea\gdef\ea\ea\noexpand#1\ea{\ea\ensuremath\ea{#2}}}
\def\alwaysmath#1{\ea\ea\ea\global\ea\ea\ea\let\ea\ea\csname your@#1\endcsname\csname #1\endcsname
  \ea\def\csname #1\endcsname{\ensuremath{\csname your@#1\endcsname}}}

% Script letters
\newcommand{\sA}{\ensuremath{\mathscr{A}}}
\newcommand{\sB}{\ensuremath{\mathscr{B}}}
\newcommand{\sC}{\ensuremath{\mathscr{C}}}
\newcommand{\sD}{\ensuremath{\mathscr{D}}}
\newcommand{\sE}{\ensuremath{\mathscr{E}}}
\newcommand{\sF}{\ensuremath{\mathscr{F}}}
\newcommand{\sG}{\ensuremath{\mathscr{G}}}
\newcommand{\sH}{\ensuremath{\mathscr{H}}}
\newcommand{\sI}{\ensuremath{\mathscr{I}}}
\newcommand{\sJ}{\ensuremath{\mathscr{J}}}
\newcommand{\sK}{\ensuremath{\mathscr{K}}}
\newcommand{\sL}{\ensuremath{\mathscr{L}}}
\newcommand{\sM}{\ensuremath{\mathscr{M}}}
\newcommand{\sN}{\ensuremath{\mathscr{N}}}
\newcommand{\sO}{\ensuremath{\mathscr{O}}}
\newcommand{\sP}{\ensuremath{\mathscr{P}}}
\newcommand{\sQ}{\ensuremath{\mathscr{Q}}}
\newcommand{\sR}{\ensuremath{\mathscr{R}}}
\newcommand{\sS}{\ensuremath{\mathscr{S}}}
\newcommand{\sT}{\ensuremath{\mathscr{T}}}
\newcommand{\sU}{\ensuremath{\mathscr{U}}}
\newcommand{\sV}{\ensuremath{\mathscr{V}}}
\newcommand{\sW}{\ensuremath{\mathscr{W}}}
\newcommand{\sX}{\ensuremath{\mathscr{X}}}
\newcommand{\sY}{\ensuremath{\mathscr{Y}}}
\newcommand{\sZ}{\ensuremath{\mathscr{Z}}}

% Calligraphic letters
\newcommand{\cA}{\ensuremath{\mathcal{A}}}
\newcommand{\cB}{\ensuremath{\mathcal{B}}}
\newcommand{\cC}{\ensuremath{\mathcal{C}}}
\newcommand{\cD}{\ensuremath{\mathcal{D}}}
\newcommand{\cE}{\ensuremath{\mathcal{E}}}
\newcommand{\cF}{\ensuremath{\mathcal{F}}}
\newcommand{\cG}{\ensuremath{\mathcal{G}}}
\newcommand{\cH}{\ensuremath{\mathcal{H}}}
\newcommand{\cI}{\ensuremath{\mathcal{I}}}
\newcommand{\cJ}{\ensuremath{\mathcal{J}}}
\newcommand{\cK}{\ensuremath{\mathcal{K}}}
\newcommand{\cL}{\ensuremath{\mathcal{L}}}
\newcommand{\cM}{\ensuremath{\mathcal{M}}}
\newcommand{\cN}{\ensuremath{\mathcal{N}}}
\newcommand{\cO}{\ensuremath{\mathcal{O}}}
\newcommand{\cP}{\ensuremath{\mathcal{P}}}
\newcommand{\cQ}{\ensuremath{\mathcal{Q}}}
\newcommand{\cR}{\ensuremath{\mathcal{R}}}
\newcommand{\cS}{\ensuremath{\mathcal{S}}}
\newcommand{\cT}{\ensuremath{\mathcal{T}}}
\newcommand{\cU}{\ensuremath{\mathcal{U}}}
\newcommand{\cV}{\ensuremath{\mathcal{V}}}
\newcommand{\cW}{\ensuremath{\mathcal{W}}}
\newcommand{\cX}{\ensuremath{\mathcal{X}}}
\newcommand{\cY}{\ensuremath{\mathcal{Y}}}
\newcommand{\cZ}{\ensuremath{\mathcal{Z}}}

% blackboard bold letters
\newcommand{\lA}{\ensuremath{\mathbb{A}}}
\newcommand{\lB}{\ensuremath{\mathbb{B}}}
\newcommand{\lC}{\ensuremath{\mathbb{C}}}
\newcommand{\lD}{\ensuremath{\mathbb{D}}}
\newcommand{\lE}{\ensuremath{\mathbb{E}}}
\newcommand{\lF}{\ensuremath{\mathbb{F}}}
\newcommand{\lG}{\ensuremath{\mathbb{G}}}
\newcommand{\lH}{\ensuremath{\mathbb{H}}}
\newcommand{\lI}{\ensuremath{\mathbb{I}}}
\newcommand{\lJ}{\ensuremath{\mathbb{J}}}
\newcommand{\lK}{\ensuremath{\mathbb{K}}}
\newcommand{\lL}{\ensuremath{\mathbb{L}}}
\newcommand{\lM}{\ensuremath{\mathbb{M}}}
\newcommand{\lN}{\ensuremath{\mathbb{N}}}
\newcommand{\lO}{\ensuremath{\mathbb{O}}}
\newcommand{\lP}{\ensuremath{\mathbb{P}}}
\newcommand{\lQ}{\ensuremath{\mathbb{Q}}}
\newcommand{\lR}{\ensuremath{\mathbb{R}}}
\newcommand{\lS}{\ensuremath{\mathbb{S}}}
\newcommand{\lT}{\ensuremath{\mathbb{T}}}
\newcommand{\lU}{\ensuremath{\mathbb{U}}}
\newcommand{\lV}{\ensuremath{\mathbb{V}}}
\newcommand{\lW}{\ensuremath{\mathbb{W}}}
\newcommand{\lX}{\ensuremath{\mathbb{X}}}
\newcommand{\lY}{\ensuremath{\mathbb{Y}}}
\newcommand{\lZ}{\ensuremath{\mathbb{Z}}}

% bold letters
\newcommand{\bA}{\ensuremath{\mathbf{A}}}
\newcommand{\bB}{\ensuremath{\mathbf{B}}}
\newcommand{\bC}{\ensuremath{\mathbf{C}}}
\newcommand{\bD}{\ensuremath{\mathbf{D}}}
\newcommand{\bE}{\ensuremath{\mathbf{E}}}
\newcommand{\bF}{\ensuremath{\mathbf{F}}}
\newcommand{\bG}{\ensuremath{\mathbf{G}}}
\newcommand{\bH}{\ensuremath{\mathbf{H}}}
\newcommand{\bI}{\ensuremath{\mathbf{I}}}
\newcommand{\bJ}{\ensuremath{\mathbf{J}}}
\newcommand{\bK}{\ensuremath{\mathbf{K}}}
\newcommand{\bL}{\ensuremath{\mathbf{L}}}
\newcommand{\bM}{\ensuremath{\mathbf{M}}}
\newcommand{\bN}{\ensuremath{\mathbf{N}}}
\newcommand{\bO}{\ensuremath{\mathbf{O}}}
\newcommand{\bP}{\ensuremath{\mathbf{P}}}
\newcommand{\bQ}{\ensuremath{\mathbf{Q}}}
\newcommand{\bR}{\ensuremath{\mathbf{R}}}
\newcommand{\bS}{\ensuremath{\mathbf{S}}}
\newcommand{\bT}{\ensuremath{\mathbf{T}}}
\newcommand{\bU}{\ensuremath{\mathbf{U}}}
\newcommand{\bV}{\ensuremath{\mathbf{V}}}
\newcommand{\bW}{\ensuremath{\mathbf{W}}}
\newcommand{\bX}{\ensuremath{\mathbf{X}}}
\newcommand{\bY}{\ensuremath{\mathbf{Y}}}
\newcommand{\bZ}{\ensuremath{\mathbf{Z}}}

% fraktur letters
\newcommand{\fa}{\ensuremath{\mathfrak{a}}}
\newcommand{\fb}{\ensuremath{\mathfrak{b}}}
\newcommand{\fc}{\ensuremath{\mathfrak{c}}}
\newcommand{\fd}{\ensuremath{\mathfrak{d}}}
\newcommand{\fe}{\ensuremath{\mathfrak{e}}}
\newcommand{\ff}{\ensuremath{\mathfrak{f}}}
\newcommand{\fg}{\ensuremath{\mathfrak{g}}}
\newcommand{\fh}{\ensuremath{\mathfrak{h}}}
\newcommand{\fj}{\ensuremath{\mathfrak{j}}}
\newcommand{\fk}{\ensuremath{\mathfrak{k}}}
\newcommand{\fl}{\ensuremath{\mathfrak{l}}}
\newcommand{\fm}{\ensuremath{\mathfrak{m}}}
\newcommand{\fn}{\ensuremath{\mathfrak{n}}}
\newcommand{\fo}{\ensuremath{\mathfrak{o}}}
\newcommand{\fp}{\ensuremath{\mathfrak{p}}}
\newcommand{\fq}{\ensuremath{\mathfrak{q}}}
\newcommand{\fr}{\ensuremath{\mathfrak{r}}}
\newcommand{\fs}{\ensuremath{\mathfrak{s}}}
\newcommand{\ft}{\ensuremath{\mathfrak{t}}}
\newcommand{\fu}{\ensuremath{\mathfrak{u}}}
\newcommand{\fv}{\ensuremath{\mathfrak{v}}}
\newcommand{\fw}{\ensuremath{\mathfrak{w}}}
\newcommand{\fx}{\ensuremath{\mathfrak{x}}}
\newcommand{\fy}{\ensuremath{\mathfrak{y}}}
\newcommand{\fz}{\ensuremath{\mathfrak{z}}}

% fraktur letters
\newcommand{\fA}{\ensuremath{\mathfrak{A}}}
\newcommand{\fB}{\ensuremath{\mathfrak{B}}}
\newcommand{\fC}{\ensuremath{\mathfrak{C}}}

\mdef\fahat{\hat{\fa}}

% Underline letters
\newcommand{\uA}{\ensuremath{\underline{A}}}
\newcommand{\uB}{\ensuremath{\underline{B}}}
\newcommand{\uC}{\ensuremath{\underline{C}}}
\newcommand{\uD}{\ensuremath{\underline{D}}}
\newcommand{\uE}{\ensuremath{\underline{E}}}
\newcommand{\uF}{\ensuremath{\underline{F}}}
\newcommand{\uG}{\ensuremath{\underline{G}}}
\newcommand{\uH}{\ensuremath{\underline{H}}}
\newcommand{\uI}{\ensuremath{\underline{I}}}
\newcommand{\uJ}{\ensuremath{\underline{J}}}
\newcommand{\uK}{\ensuremath{\underline{K}}}
\newcommand{\uL}{\ensuremath{\underline{L}}}
\newcommand{\uM}{\ensuremath{\underline{M}}}
\newcommand{\uN}{\ensuremath{\underline{N}}}
\newcommand{\uO}{\ensuremath{\underline{O}}}
\newcommand{\uP}{\ensuremath{\underline{P}}}
\newcommand{\uQ}{\ensuremath{\underline{Q}}}
\newcommand{\uR}{\ensuremath{\underline{R}}}
\newcommand{\uS}{\ensuremath{\underline{S}}}
\newcommand{\uT}{\ensuremath{\underline{T}}}
\newcommand{\uU}{\ensuremath{\underline{U}}}
\newcommand{\uV}{\ensuremath{\underline{V}}}
\newcommand{\uW}{\ensuremath{\underline{W}}}
\newcommand{\uX}{\ensuremath{\underline{X}}}
\newcommand{\uY}{\ensuremath{\underline{Y}}}
\newcommand{\uZ}{\ensuremath{\underline{Z}}}

% bars
\newcommand{\Abar}{\ensuremath{\overline{A}}}
\newcommand{\Bbar}{\ensuremath{\overline{B}}}
\newcommand{\Cbar}{\ensuremath{\overline{C}}}
\newcommand{\Dbar}{\ensuremath{\overline{D}}}
\newcommand{\Ebar}{\ensuremath{\overline{E}}}
\newcommand{\Fbar}{\ensuremath{\overline{F}}}
\newcommand{\Gbar}{\ensuremath{\overline{G}}}
\newcommand{\Hbar}{\ensuremath{\overline{H}}}
\newcommand{\Ibar}{\ensuremath{\overline{I}}}
\newcommand{\Jbar}{\ensuremath{\overline{J}}}
\newcommand{\Kbar}{\ensuremath{\overline{K}}}
\newcommand{\Lbar}{\ensuremath{\overline{L}}}
\newcommand{\Mbar}{\ensuremath{\overline{M}}}
\newcommand{\Nbar}{\ensuremath{\overline{N}}}
\newcommand{\Obar}{\ensuremath{\overline{O}}}
\newcommand{\Pbar}{\ensuremath{\overline{P}}}
\newcommand{\Qbar}{\ensuremath{\overline{Q}}}
\newcommand{\Rbar}{\ensuremath{\overline{R}}}
\newcommand{\Sbar}{\ensuremath{\overline{S}}}
\newcommand{\Tbar}{\ensuremath{\overline{T}}}
\newcommand{\Ubar}{\ensuremath{\overline{U}}}
\newcommand{\Vbar}{\ensuremath{\overline{V}}}
\newcommand{\Wbar}{\ensuremath{\overline{W}}}
\newcommand{\Xbar}{\ensuremath{\overline{X}}}
\newcommand{\Ybar}{\ensuremath{\overline{Y}}}
\newcommand{\Zbar}{\ensuremath{\overline{Z}}}
\newcommand{\abar}{\ensuremath{\overline{a}}}
\newcommand{\bbar}{\ensuremath{\overline{b}}}
\newcommand{\cbar}{\ensuremath{\overline{c}}}
\newcommand{\dbar}{\ensuremath{\overline{d}}}
\newcommand{\ebar}{\ensuremath{\overline{e}}}
\newcommand{\fbar}{\ensuremath{\overline{f}}}
\newcommand{\gbar}{\ensuremath{\overline{g}}}
%\newcommand{\hbar}{\ensuremath{\overline{h}}} % whoops, \hbar means something else!
\newcommand{\ibar}{\ensuremath{\overline{\imath}}}
\newcommand{\jbar}{\ensuremath{\overline{\jmath}}}
\newcommand{\kbar}{\ensuremath{\overline{k}}}
\newcommand{\lbar}{\ensuremath{\overline{l}}}
\newcommand{\mbar}{\ensuremath{\overline{m}}}
\newcommand{\nbar}{\ensuremath{\overline{n}}}
%\newcommand{\obar}{\ensuremath{\overline{o}}}
\newcommand{\pbar}{\ensuremath{\overline{p}}}
\newcommand{\qbar}{\ensuremath{\overline{q}}}
\newcommand{\rbar}{\ensuremath{\overline{r}}}
\newcommand{\sbar}{\ensuremath{\overline{s}}}
\newcommand{\tbar}{\ensuremath{\overline{t}}}
\newcommand{\ubar}{\ensuremath{\overline{u}}}
\newcommand{\vbar}{\ensuremath{\overline{v}}}
\newcommand{\wbar}{\ensuremath{\overline{w}}}
\newcommand{\xbar}{\ensuremath{\overline{x}}}
\newcommand{\ybar}{\ensuremath{\overline{y}}}
\newcommand{\zbar}{\ensuremath{\overline{z}}}

% tildes
\newcommand{\Atil}{\ensuremath{\widetilde{A}}}
\newcommand{\Btil}{\ensuremath{\widetilde{B}}}
\newcommand{\Ctil}{\ensuremath{\widetilde{C}}}
\newcommand{\Dtil}{\ensuremath{\widetilde{D}}}
\newcommand{\Etil}{\ensuremath{\widetilde{E}}}
\newcommand{\Ftil}{\ensuremath{\widetilde{F}}}
\newcommand{\Gtil}{\ensuremath{\widetilde{G}}}
\newcommand{\Htil}{\ensuremath{\widetilde{H}}}
\newcommand{\Itil}{\ensuremath{\widetilde{I}}}
\newcommand{\Jtil}{\ensuremath{\widetilde{J}}}
\newcommand{\Ktil}{\ensuremath{\widetilde{K}}}
\newcommand{\Ltil}{\ensuremath{\widetilde{L}}}
\newcommand{\Mtil}{\ensuremath{\widetilde{M}}}
\newcommand{\Ntil}{\ensuremath{\widetilde{N}}}
\newcommand{\Otil}{\ensuremath{\widetilde{O}}}
\newcommand{\Ptil}{\ensuremath{\widetilde{P}}}
\newcommand{\Qtil}{\ensuremath{\widetilde{Q}}}
\newcommand{\Rtil}{\ensuremath{\widetilde{R}}}
\newcommand{\Stil}{\ensuremath{\widetilde{S}}}
\newcommand{\Ttil}{\ensuremath{\widetilde{T}}}
\newcommand{\Util}{\ensuremath{\widetilde{U}}}
\newcommand{\Vtil}{\ensuremath{\widetilde{V}}}
\newcommand{\Wtil}{\ensuremath{\widetilde{W}}}
\newcommand{\Xtil}{\ensuremath{\widetilde{X}}}
\newcommand{\Ytil}{\ensuremath{\widetilde{Y}}}
\newcommand{\Ztil}{\ensuremath{\widetilde{Z}}}
\newcommand{\atil}{\ensuremath{\widetilde{a}}}
\newcommand{\btil}{\ensuremath{\widetilde{b}}}
\newcommand{\ctil}{\ensuremath{\widetilde{c}}}
\newcommand{\dtil}{\ensuremath{\widetilde{d}}}
\newcommand{\etil}{\ensuremath{\widetilde{e}}}
\newcommand{\ftil}{\ensuremath{\widetilde{f}}}
\newcommand{\gtil}{\ensuremath{\widetilde{g}}}
\newcommand{\htil}{\ensuremath{\widetilde{h}}}
\newcommand{\itil}{\ensuremath{\widetilde{\imath}}}
\newcommand{\jtil}{\ensuremath{\widetilde{\jmath}}}
\newcommand{\ktil}{\ensuremath{\widetilde{k}}}
\newcommand{\ltil}{\ensuremath{\widetilde{l}}}
\newcommand{\mtil}{\ensuremath{\widetilde{m}}}
\newcommand{\ntil}{\ensuremath{\widetilde{n}}}
\newcommand{\otil}{\ensuremath{\widetilde{o}}}
\newcommand{\ptil}{\ensuremath{\widetilde{p}}}
\newcommand{\qtil}{\ensuremath{\widetilde{q}}}
\newcommand{\rtil}{\ensuremath{\widetilde{r}}}
\newcommand{\stil}{\ensuremath{\widetilde{s}}}
\newcommand{\ttil}{\ensuremath{\widetilde{t}}}
\newcommand{\util}{\ensuremath{\widetilde{u}}}
\newcommand{\vtil}{\ensuremath{\widetilde{v}}}
\newcommand{\wtil}{\ensuremath{\widetilde{w}}}
\newcommand{\xtil}{\ensuremath{\widetilde{x}}}
\newcommand{\ytil}{\ensuremath{\widetilde{y}}}
\newcommand{\ztil}{\ensuremath{\widetilde{z}}}

% Hats
\newcommand{\Ahat}{\ensuremath{\widehat{A}}}
\newcommand{\Bhat}{\ensuremath{\widehat{B}}}
\newcommand{\Chat}{\ensuremath{\widehat{C}}}
\newcommand{\Dhat}{\ensuremath{\widehat{D}}}
\newcommand{\Ehat}{\ensuremath{\widehat{E}}}
\newcommand{\Fhat}{\ensuremath{\widehat{F}}}
\newcommand{\Ghat}{\ensuremath{\widehat{G}}}
\newcommand{\Hhat}{\ensuremath{\widehat{H}}}
\newcommand{\Ihat}{\ensuremath{\widehat{I}}}
\newcommand{\Jhat}{\ensuremath{\widehat{J}}}
\newcommand{\Khat}{\ensuremath{\widehat{K}}}
\newcommand{\Lhat}{\ensuremath{\widehat{L}}}
\newcommand{\Mhat}{\ensuremath{\widehat{M}}}
\newcommand{\Nhat}{\ensuremath{\widehat{N}}}
\newcommand{\Ohat}{\ensuremath{\widehat{O}}}
\newcommand{\Phat}{\ensuremath{\widehat{P}}}
\newcommand{\Qhat}{\ensuremath{\widehat{Q}}}
\newcommand{\Rhat}{\ensuremath{\widehat{R}}}
\newcommand{\Shat}{\ensuremath{\widehat{S}}}
\newcommand{\That}{\ensuremath{\widehat{T}}}
\newcommand{\Uhat}{\ensuremath{\widehat{U}}}
\newcommand{\Vhat}{\ensuremath{\widehat{V}}}
\newcommand{\What}{\ensuremath{\widehat{W}}}
\newcommand{\Xhat}{\ensuremath{\widehat{X}}}
\newcommand{\Yhat}{\ensuremath{\widehat{Y}}}
\newcommand{\Zhat}{\ensuremath{\widehat{Z}}}
\newcommand{\ahat}{\ensuremath{\hat{a}}}
\newcommand{\bhat}{\ensuremath{\hat{b}}}
\newcommand{\chat}{\ensuremath{\hat{c}}}
\newcommand{\dhat}{\ensuremath{\hat{d}}}
\newcommand{\ehat}{\ensuremath{\hat{e}}}
\newcommand{\fhat}{\ensuremath{\hat{f}}}
\newcommand{\ghat}{\ensuremath{\hat{g}}}
\newcommand{\hhat}{\ensuremath{\hat{h}}}
\newcommand{\ihat}{\ensuremath{\hat{\imath}}}
\newcommand{\jhat}{\ensuremath{\hat{\jmath}}}
\newcommand{\khat}{\ensuremath{\hat{k}}}
\newcommand{\lhat}{\ensuremath{\hat{l}}}
\newcommand{\mhat}{\ensuremath{\hat{m}}}
\newcommand{\nhat}{\ensuremath{\hat{n}}}
\newcommand{\ohat}{\ensuremath{\hat{o}}}
\newcommand{\phat}{\ensuremath{\hat{p}}}
\newcommand{\qhat}{\ensuremath{\hat{q}}}
\newcommand{\rhat}{\ensuremath{\hat{r}}}
\newcommand{\shat}{\ensuremath{\hat{s}}}
\newcommand{\that}{\ensuremath{\hat{t}}}
\newcommand{\uhat}{\ensuremath{\hat{u}}}
\newcommand{\vhat}{\ensuremath{\hat{v}}}
\newcommand{\what}{\ensuremath{\hat{w}}}
\newcommand{\xhat}{\ensuremath{\hat{x}}}
\newcommand{\yhat}{\ensuremath{\hat{y}}}
\newcommand{\zhat}{\ensuremath{\hat{z}}}

%% FONTS AND DECORATION FOR GREEK LETTERS

%% the package `mathbbol' gives us blackboard bold greek and numbers,
%% but it does it by redefining \mathbb to use a different font, so that
%% all the other \mathbb letters look different too.  Here we import the
%% font with bb greek and numbers, but assign it a different name,
%% \mathbbb, so as not to replace the usual one.
\DeclareSymbolFont{bbold}{U}{bbold}{m}{n}
\DeclareSymbolFontAlphabet{\mathbbb}{bbold}
\newcommand{\bbDelta}{\ensuremath{\mathbbb{\Delta}}}
\newcommand{\bbone}{\ensuremath{\mathbbb{1}}}
\newcommand{\bbtwo}{\ensuremath{\mathbbb{2}}}
\newcommand{\bbthree}{\ensuremath{\mathbbb{3}}}

% greek with bars
\newcommand{\albar}{\ensuremath{\overline{\alpha}}}
\newcommand{\bebar}{\ensuremath{\overline{\beta}}}
\newcommand{\gmbar}{\ensuremath{\overline{\gamma}}}
\newcommand{\debar}{\ensuremath{\overline{\delta}}}
\newcommand{\phibar}{\ensuremath{\overline{\varphi}}}
\newcommand{\psibar}{\ensuremath{\overline{\psi}}}
\newcommand{\xibar}{\ensuremath{\overline{\xi}}}
\newcommand{\ombar}{\ensuremath{\overline{\omega}}}

% greek with hats
\newcommand{\alhat}{\ensuremath{\hat{\alpha}}}
\newcommand{\behat}{\ensuremath{\hat{\beta}}}
\newcommand{\gmhat}{\ensuremath{\hat{\gamma}}}
\newcommand{\dehat}{\ensuremath{\hat{\delta}}}

% greek with checks
\newcommand{\alchk}{\ensuremath{\check{\alpha}}}
\newcommand{\bechk}{\ensuremath{\check{\beta}}}
\newcommand{\gmchk}{\ensuremath{\check{\gamma}}}
\newcommand{\dechk}{\ensuremath{\check{\delta}}}

% greek with tildes
\newcommand{\altil}{\ensuremath{\widetilde{\alpha}}}
\newcommand{\betil}{\ensuremath{\widetilde{\beta}}}
\newcommand{\gmtil}{\ensuremath{\widetilde{\gamma}}}
\newcommand{\phitil}{\ensuremath{\widetilde{\varphi}}}
\newcommand{\psitil}{\ensuremath{\widetilde{\psi}}}
\newcommand{\xitil}{\ensuremath{\widetilde{\xi}}}
\newcommand{\omtil}{\ensuremath{\widetilde{\omega}}}

% MISCELLANEOUS SYMBOLS
\mdef\del{\partial}
\mdef\delbar{\overline{\partial}}
\let\sm\wedge
\newcommand{\dd}[1]{\ensuremath{\frac{\partial}{\partial {#1}}}}
\newcommand{\inv}{^{-1}}
\newcommand{\dual}{^{\vee}}
\mdef\hf{\textstyle\frac{1}{2}}
\mdef\thrd{\textstyle\frac{1}{3}}
\mdef\qtr{\textstyle\frac{1}{4}}
\let\meet\wedge
\let\join\vee
\let\dn\downarrow
\newcommand{\op}{^{\mathit{op}}}
\newcommand{\co}{^{\mathit{co}}}
\newcommand{\coop}{^{\mathit{coop}}}
\let\adj\dashv
\SelectTips{cm}{}
\newdir{ >}{{}*!/-10pt/@{>}}    % extra spacing for tail arrows in XYpic
\newcommand{\pushoutcorner}[1][dr]{\save*!/#1+1.2pc/#1:(1,-1)@^{|-}\restore}
\newcommand{\pullbackcorner}[1][dr]{\save*!/#1-1.2pc/#1:(-1,1)@^{|-}\restore}
\let\iso\cong
\let\eqv\simeq
\let\cng\equiv
\mdef\Id{\mathrm{Id}}
\mdef\id{\mathrm{id}}
\alwaysmath{ell}
\alwaysmath{infty}
\alwaysmath{odot}
\def\frc#1/#2.{\frac{#1}{#2}}   % \frc x^2+1 / x^2-1 .
\mdef\ten{\mathrel{\otimes}}
\mdef\bigten{\bigotimes}
\mdef\sqten{\mathrel{\boxtimes}}
\def\pow(#1,#2){\mathop{\pitchfork}(#1,#2)} % powers and
\def\cpw{\mathop{\odot}}                    % copowers
\newcommand{\mathid}{\mbox{id}}
\newcommand{\cat}[1]{\ensuremath{\mathbf{#1}}}


%% OPERATORS
\DeclareMathOperator\lan{Lan}
\DeclareMathOperator\ran{Ran}
\DeclareMathOperator\colim{colim}
\DeclareMathOperator\coeq{coeq}
\DeclareMathOperator\eq{eq}
\DeclareMathOperator\Tot{Tot}
\DeclareMathOperator\cosk{cosk}
\DeclareMathOperator\sk{sk}
\DeclareMathOperator\im{im}
\DeclareMathOperator\Spec{Spec}
\DeclareMathOperator\Ho{Ho}
\DeclareMathOperator\Aut{Aut}
\DeclareMathOperator\End{End}
\DeclareMathOperator\Hom{Hom}
\DeclareMathOperator\Map{Map}

%% TIKZ ARROWS AND HIGHER CELLS
\makeatletter
\def\slashedarrowfill@#1#2#3#4#5{%
  $\m@th\thickmuskip0mu\medmuskip\thickmuskip\thinmuskip\thickmuskip
   \relax#5#1\mkern-7mu%
   \cleaders\hbox{$#5\mkern-2mu#2\mkern-2mu$}\hfill
   \mathclap{#3}\mathclap{#2}%
   \cleaders\hbox{$#5\mkern-2mu#2\mkern-2mu$}\hfill
   \mkern-7mu#4$%
}

\def\Rightslashedarrowfill@{%
  \slashedarrowfill@\Relbar\Relbar\Mapstochar\Rightarrow}
\newcommand\xslashedRightarrow[2][]{%
  \ext@arrow 0055{\Rightslashedarrowfill@}{#1}{#2}}
\def\hTo{\xslashedRightarrow{}}
\def\hToo{\xslashedRightarrow{\quad}}
\let\xhTo\xslashedRightarrow

\pagestyle{empty}

\newcommand{\Rightthreecell}{\RRightarrow}
\newcommand{\Rtwocell}{\Rightarrow}

\tikzstyle{doubletick}=[-implies, double equal sign distance, postaction={decorate},decoration={markings,mark=at position .5 with {\draw[-] (0,-0.1) -- (0,0.1);}}]

\tikzstyle{darrow}=[-implies, double equal sign distance]

\tikzstyle{doubleeq}=[double equal sign distance]


%% ARROWS
% \to already exists
\newcommand{\too}[1][]{\ensuremath{\overset{#1}{\longrightarrow}}}
\newcommand{\ot}{\ensuremath{\leftarrow}}
\newcommand{\oot}[1][]{\ensuremath{\overset{#1}{\longleftarrow}}}
\let\toot\rightleftarrows
\let\otto\leftrightarrows
\let\Impl\Rightarrow
\let\imp\Rightarrow
\let\toto\rightrightarrows
\let\into\hookrightarrow
\let\xinto\xhookrightarrow
\mdef\we{\overset{\sim}{\longrightarrow}}
\mdef\leftwe{\overset{\sim}{\longleftarrow}}
\let\mono\rightarrowtail
\let\leftmono\leftarrowtail
\let\cof\rightarrowtail
\let\leftcof\leftarrowtail
\let\epi\twoheadrightarrow
\let\leftepi\twoheadleftarrow
\let\fib\twoheadrightarrow
\let\leftfib\twoheadleftarrow
\let\cohto\rightsquigarrow
\let\maps\colon
\newcommand{\spam}{\,:\!}       % \maps for left arrows

\newsavebox{\DDownarrowbox}
\savebox{\DDownarrowbox}{\tikz[scale=1.5]{\draw[-implies,double equal
sign distance] (0,.1) -- (0,-.1); \draw (0,.1) -- (0,-.1);}}
\newcommand{\DDownarrow}{\mathrel{\raisebox{-.2em}{\usebox{\DDownarrowbox}}}}

\newsavebox{\RRightarrowbox}
\savebox{\RRightarrowbox}{\tikz[scale=1.5]{\draw[-implies,double equal
sign distance] (-.1,0) -- (.1,0); \draw (-.1,0) -- (.1,0);}}
\newcommand{\RRightarrow}{\mathrel{\raisebox{.2em}{\usebox{\RRightarrowbox}}}}

%\newsavebox{\Rightslashedarrowbox}
%\savebox{\Rightslashedarrowbox}{\tikz[scale=1.5]{\draw[Rightslashedarrow{}] (-.1,0) -- (1,0);}}
%\newcommand{\Rightslashedarrow}{\mathrel{\raisebox{-.2em}%{\usebox{\Rightslashedarrowbox}}}}


%% EXTENSIBLE ARROWS
\let\xto\xrightarrow
\let\xot\xleftarrow
% See Voss' Mathmode.tex for instructions on how to create new
% extensible arrows.
\def\rightarrowtailfill@{\arrowfill@{\Yright\joinrel\relbar}\relbar\rightarrow}
\newcommand\xrightarrowtail[2][]{\ext@arrow 0055{\rightarrowtailfill@}{#1}{#2}}
\let\xmono\xrightarrowtail
\let\xcof\xrightarrowtail
\def\twoheadrightarrowfill@{\arrowfill@{\relbar\joinrel\relbar}\relbar\twoheadrightarrow}
\newcommand\xtwoheadrightarrow[2][]{\ext@arrow 0055{\twoheadrightarrowfill@}{#1}{#2}}
\let\xepi\xtwoheadrightarrow
\let\xfib\xtwoheadrightarrow
% Let's leave the left-going ones until I need them.

%% EXTENSIBLE SLASHED ARROWS
% Making extensible slashed arrows, by modifying the underlying AMS code.
% Arguments are:
% 1 = arrowhead on the left (\relbar or \Relbar if none)
% 2 = fill character (usually \relbar or \Relbar)
% 3 = slash character (such as \mapstochar or \Mapstochar)
% 4 = arrowhead on the left (\relbar or \Relbar if none)
% 5 = display mode (\displaystyle etc)
\def\slashedarrowfill@#1#2#3#4#5{%
  $\m@th\thickmuskip0mu\medmuskip\thickmuskip\thinmuskip\thickmuskip
   \relax#5#1\mkern-7mu%
   \cleaders\hbox{$#5\mkern-2mu#2\mkern-2mu$}\hfill
   \mathclap{#3}\mathclap{#2}%
   \cleaders\hbox{$#5\mkern-2mu#2\mkern-2mu$}\hfill
   \mkern-7mu#4$%
}
% Here's the idea: \<slashed>arrowfill@ should be a box containing
% some stretchable space that is the "middle of the arrow".  This
% space is created as a "leader" using \cleader<thing>\hfill, which
% fills an \hfill of space with copies of <thing>.  Here instead of
% just one \cleader, we use two, with the slash in between (and an
% extra copy of the filler, to avoid extra space around the slash).
\def\rightslashedarrowfill@{%
  \slashedarrowfill@\relbar\relbar\mapstochar\rightarrow}
\newcommand\xslashedrightarrow[2][]{%
  \ext@arrow 0055{\rightslashedarrowfill@}{#1}{#2}}
\mdef\hto{\xslashedrightarrow{}}
\mdef\htoo{\xslashedrightarrow{\quad}}
\let\xhto\xslashedrightarrow

%% To get a slashed arrow in XYpic, do
% \[\xymatrix{A \ar[r]|-@{|} & B}\]

% ISOMORPHISMS
\def\xiso#1{\mathrel{\mathrlap{\smash{\xto[\smash{\raisebox{1.3mm}{$\scriptstyle\sim$}}]{#1}}}\hphantom{\xto{#1}}}}
\def\toiso{\xto{\smash{\raisebox{-.5mm}{$\scriptstyle\sim$}}}}

% SHADOWS
\def\shvar#1#2{{\ensuremath{%
  \hspace{1mm}\makebox[-1mm]{$#1\langle$}\makebox[0mm]{$#1\langle$}\hspace{1mm}%
  {#2}%
  \makebox[1mm]{$#1\rangle$}\makebox[0mm]{$#1\rangle$}%
}}}
\def\sh{\shvar{}}
\def\scriptsh{\shvar{\scriptstyle}}
\def\bigsh{\shvar{\big}}
\def\Bigsh{\shvar{\Big}}
\def\biggsh{\shvar{\bigg}}
\def\Biggsh{\shvar{\Bigg}}

%HIGHER CELLS



% THEOREM-TYPE ENVIRONMENTS, hacked to
%% (a) number all with the same numbers, and
%% (b) have the right names for autoref
\def\defthm#1#2{%
  \newtheorem{#1}{#2}[section]%
  \expandafter\def\csname #1autorefname\endcsname{#2}%
  \expandafter\let\csname c@#1\endcsname\c@thm}
\newtheorem{thm}{Theorem}[section]
\newcommand{\thmautorefname}{Theorem}
\defthm{cor}{Corollary}
\defthm{prop}{Proposition}
\defthm{lem}{Lemma}
\defthm{sch}{Scholium}
\defthm{assume}{Assumption}
\defthm{claim}{Claim}
\defthm{conj}{Conjecture}
\defthm{hyp}{Hypothesis}
\defthm{fact}{Fact}
\theoremstyle{definition}
\defthm{defn}{Definition}
\defthm{notn}{Notation}
\theoremstyle{remark}
\defthm{rmk}{Remark}
\defthm{eg}{Example}
\defthm{egs}{Examples}
\defthm{ex}{Exercise}
\defthm{ceg}{Counterexample}

% How to get QED symbols inside equations at the end of the statements
% of theorems.  AMS LaTeX knows how to do this inside equations at the
% end of *proofs* with \qedhere, and at the end of the statement of a
% theorem with \qed (meaning no proof will be given), but it can't
% seem to combine the two.  Use this instead.
\def\thmqedhere{\expandafter\csname\csname @currenvir\endcsname @qed\endcsname}

% Number numbered lists as (i), (ii), ...
\renewcommand{\theenumi}{(\roman{enumi})}
\renewcommand{\labelenumi}{\theenumi}

%% Labeling that keeps track of theorem-type names.  Superseded by
%% autoref from hyperref, as above, but we keep this in case we are
%% using a journal style file that is incompatible with hyperref.
% 
% \ifx\SK@label\undefined\let\SK@label\label\fi
% \let\your@thm\@thm
% \def\@thm#1#2#3{\gdef\currthmtype{#3}\your@thm{#1}{#2}{#3}}
% \def\xlabel#1{{\let\your@currentlabel\@currentlabel\def\@currentlabel
% {\currthmtype~\your@currentlabel}
% \SK@label{#1@}}\label{#1}}
% \def\xref#1{\ref{#1@}}

% Also number formulas with the theorem counter
\let\c@equation\c@thm
\numberwithin{equation}{section}

% Only show numbers for equations that are actually referenced (or
% whose tags are specified manually) - uses mathtools.
\mathtoolsset{showonlyrefs,showmanualtags}

%% Fix enumerate spacing with paralist.  This has two parts:
%%   1. enable mixing of "old spacing" lists with those adjusted by paralist
%%   2. allow us to specify a number based on which to adjust the spacing
%% For the first, use \killspacingtrue when you want the spacing
%% adjusted, then \killspacingfalse to turn adjustment off.  For the
%% second, use \maxenum=14 to set the widest number you want the
%% spacing to be calculated with.
\newlength\oldleftmargini       % save old spacing
\newlength\oldleftmarginii
\newlength\oldleftmarginiii
\newlength\oldleftmarginiv
\newlength\oldleftmarginv
\newlength\oldleftmarginvi
\newcount\maxenum
\maxenum=7
\newif\ifkillspacing
\def\@adjust@enum@labelwidth{%
  \advance\@listdepth by 1\relax
  \ifkillspacing                % do the paralist thing
    \csname c@\@enumctr\endcsname\maxenum
    \settowidth{\@tempdima}{%
      \csname label\@enumctr\endcsname\hspace{\labelsep}}%
    \csname leftmargin\romannumeral\@listdepth\endcsname
      \@tempdima
  \else                         % otherwise, reset it
    \csname fixspacing\romannumeral\@listdepth\endcsname
  \fi
  \advance\@listdepth by -1\relax}
% these commands, one for each enum level (I couldn't get a generic
% one to work), test whether oldleftmargin has been set yet, and if
% not, set it from leftmargin; otherwise, they reset leftmargin to
% it.  Just setting oldleftmargin to leftmargin in the preamble
% doesn't seem to work.
\def\fixspacingi{\ifnum\oldleftmargini=0\setlength\oldleftmargini\leftmargini\else\setlength\leftmargini\oldleftmargini\fi}
\def\fixspacingii{\ifnum\oldleftmarginii=0\setlength\oldleftmarginii\leftmarginii\else\setlength\leftmarginii\oldleftmarginii\fi}
\def\fixspacingiii{\ifnum\oldleftmarginiii=0\setlength\oldleftmarginiii\leftmarginiii\else\setlength\leftmarginiii\oldleftmarginiii\fi}
\def\fixspacingiv{\ifnum\oldleftmarginiv=0\setlength\oldleftmarginiv\leftmarginiv\else\setlength\leftmarginiv\oldleftmarginiv\fi}
\def\fixspacingv{\ifnum\oldleftmarginv=0\setlength\oldleftmarginv\leftmarginv\else\setlength\leftmarginv\oldleftmarginv\fi}
\def\fixspacingvi{\ifnum\oldleftmarginvi=0\setlength\oldleftmarginvi\leftmarginvi\else\setlength\leftmarginvi\oldleftmarginvi\fi}

%% Fix paralist references, so that we can refer to (1) instead of
%% just 1.
\def\pl@label#1#2{%
  \edef\pl@the{\noexpand#1{\@enumctr}}%
  \pl@lab\expandafter{\the\pl@lab\csname yourthe\@enumctr\endcsname}%
  \advance\@tempcnta1
  \pl@loop}
\def\@enumlabel@#1[#2]{%
  \@plmylabeltrue
  \@tempcnta0
  \pl@lab{}%
  \let\pl@the\pl@qmark
  \expandafter\pl@loop\@gobble#2\@@@
  \ifnum\@tempcnta=1\else
    \PackageWarning{paralist}{Incorrect label; no or multiple
      counters.\MessageBreak The label is: \@gobble#2}%
  \fi
  \expandafter\edef\csname label\@enumctr\endcsname{\the\pl@lab}%
  \expandafter\edef\csname the\@enumctr\endcsname{\the\pl@lab}%
  \expandafter\let\csname yourthe\@enumctr\endcsname\pl@the
  #1}


% GREEK LETTERS, ETC.
\alwaysmath{alpha}
\alwaysmath{beta}
\alwaysmath{gamma}
\alwaysmath{Gamma}
\alwaysmath{delta}
\alwaysmath{Delta}
\alwaysmath{epsilon}
\mdef\ep{\varepsilon}
\alwaysmath{zeta}
\alwaysmath{eta}
\alwaysmath{theta}
\alwaysmath{Theta}
\alwaysmath{iota}
\alwaysmath{kappa}
\alwaysmath{lambda}
\alwaysmath{Lambda}
\alwaysmath{mu}
\alwaysmath{nu}
\alwaysmath{xi}
\alwaysmath{pi}
\alwaysmath{rho}
\alwaysmath{sigma}
\alwaysmath{Sigma}
\alwaysmath{tau}
\alwaysmath{upsilon}
\alwaysmath{Upsilon}
\alwaysmath{phi}
\alwaysmath{Pi}
\alwaysmath{Phi}
\mdef\ph{\varphi}
\alwaysmath{chi}
\alwaysmath{psi}
\alwaysmath{Psi}
\alwaysmath{omega}
\alwaysmath{Omega}
\let\al\alpha
\let\be\beta
\let\gm\gamma
\let\Gm\Gamma
\let\de\delta
\let\De\Delta
\let\si\sigma
\let\Si\Sigma
\let\om\omega
\let\ka\kappa
\let\la\lambda
\let\La\Lambda
\let\ze\zeta
\let\th\theta
\let\Th\Theta
\let\vth\vartheta

\makeatother

% Tikz styles
\tikzstyle{tickarrow}=[->,postaction={decorate},decoration={markings,mark=at position .5 with {\draw[-] (0,-0.1) -- (0,0.1);}},line width=0.50]

% Local Variables:
% mode: latex
% TeX-master: ""
% End:

\begin{document}

{\small
\begin{equation*}\hspace{-2cm}
\begin{tikzpicture}[xscale=2.5, yscale=3]
%%%%Row A
\node (A0) at (0,5) {$\substack{\tens(\tens \times \transid)\\(\tens \times \transid \times \transid)\\(\tens \times \transid \times \transid \times \transid)}$};
\node (A1) at (0,6.5) {$\substack{\tens(\tens \times \transid)\\(\tens \times \transid \times \transid)\\(\transid \times \tens \times \transid \times \transid)}$};
\node (A2) at (1.5,7) {$\substack{\tens(\tens \times \transid)\\(\transid \times \tens \times \transid)\\(\transid \times \tens \times \transid \times \transid)}$};
\node (A3) at (3,7.5) {$\substack{\tens(\tens \times \transid)\\(\transid \times \tens \times \transid)\\(\transid \times \transid \times \tens \times \transid)}$};
\node (A4) at (4.5,7) {$\substack{\tens(\transid \times \tens)\\(\transid \times \tens \times \transid)\\(\transid \times \transid \times \tens \times \transid)}$};
\node (A5) at (6,6.5) {$\substack{\tens(\transid \times \tens)\\(\transid  \times \transid \times \tens)\\(\transid \times \transid \times \tens \times \transid)}$};
\node (A6) at (6,5) {$\substack{\tens(\transid \times \tens)\\(\transid  \times \transid \times \tens)\\(\transid \times \transid \times \transid \times \tens )}$};
%%%%%%
\draw[doubleloose] (A0) to node[above, xshift=-20pt]{$\substack{\looseid \looseid \\(\alpha \times \looseid \times \looseid)}$} (A1);
\draw[doubleloose] (A1) to node[above, xshift=-16pt]{$\substack{\looseid (\alpha \times \looseid) \looseid}$}
(A2);
\draw[doubleloose] (A2) to node[above, xshift=-16pt]{$\substack{\looseid \looseid \\ (\looseid \times \alpha \times \looseid)}$} (A3);
\draw[doubleloose] (A3) to node[above, xshift=16pt]{$\substack{ \alpha \looseid \looseid}$} (A4);
\draw[doubleloose] (A4) to node[above, xshift=16pt]{$\substack{\looseid (\looseid \times \alpha) \looseid}$} (A5);
\draw[doubleloose] (A5) to node[above, xshift=16pt]{$\substack{ \looseid \looseid \\ (\looseid \times \looseid \times \alpha)}$} (A6);
%%%%row B
\node (B3) at (3,6.5) {$\substack{\tens(\transid \times \tens)\\(\transid \times \tens \times \transid)\\(\transid \times \tens  \times \transid \times \transid)}$};
%%%%%%
\draw[doubleloose] (A2) to node[above, xshift=16pt]{$\substack{ \alpha \looseid \looseid }$} (B3);
\draw[doubleloose] (B3) to node[above, xshift=-16pt]{$\substack{ \looseid \looseid \\ (\looseid \times \alpha \times \looseid)}$} (A4);
%%%%RowC
\node (C4) at (3,5.5) {$\substack{\tens(\transid \times \tens)\\(\transid  \times \transid \times \tens) \\ (\transid \times \tens \times \transid \times \transid)}$};
\node (C5) at (4,5) {$\substack{\tens(\transid \times \tens)\\(\transid  \times \tens\times \transid) \\ (\transid \times \tens \times \transid \times \transid)}$};
%%%%%% Extra cells
\draw[doubleloose] (A1) to[out=0, in=125] node[above]{$\substack{S(\pi)\looseid}$} (C4);
\draw[doubleloose] (A1) to[out=-35, in=180] node[above, xshift=5pt]{$\substack{T(\pi) \looseid}$}(C4);
%%%%%%%Extra cells
\draw[doubleloose] (B3) to[out=0, in=115] node[above, xshift=15pt]{$\substack{\looseid \\( \looseid \times S(\pi))}$} (A6);
\draw[doubleloose] (B3) to[out=-35, in=160] node[above, xshift=15pt]{$\substack{\looseid \\( \looseid \times T(\pi))}$}(A6);
%%%%%%%%%
\draw[doubleloose] (B3) to node[right]{$\substack{ \looseid \\ (\looseid \times \alpha) \\ \looseid}$} (C4);
\draw[doubletighteq] (C4) to (C5);
\draw[doubleloose] (C5) to node[above]{$\substack{\looseid (\looseid \times \alpha) \looseid}$} (A6);
%%%%RowD
\node (D2) at (1,5) {$\substack{\tens(\transid \times \tens)\\(\tens \times \transid \times \transid)\\(\transid \times \tens \times \transid \times \transid)}$};
\node (D3) at (2,5) {$\substack{\tens(\tens \times \transid)\\(\transid \times \transid \times \tens)\\(\transid \times \tens \times \transid \times \transid)}$};
%%%%%%
\draw[doubleloose] (A1) to node[above, xshift=10pt] {$\substack{\alpha \looseid \looseid}$}   (D2);
\draw[doubletighteq] (D2) to  (D3);
\draw[doubleloose] (D3) to node[above, xshift=-12pt, yshift=-3pt] {$\substack{ \alpha \looseid \looseid}$} (C4);
%%%%RowE
\node (E1) at (1 ,3.5) {$\substack{\tens( \transid \times \tens)\\(\tens \times \transid \times \transid)\\(\tens \times \transid \times  \transid \times \transid)}$};
\node (E3) at (2.5 ,4.5) {$\substack{\tens( \tens \times \transid)\\(\transid \times \tens \times \transid)\\(\transid \times  \transid \times \transid \times \tens )}$};
%%%%%%
\draw[doubleloose] (A0) to node[above,xshift=-23]{$\substack{\alpha \looseid \looseid}$} (E1);
\draw[doubleloose] (E1) to node[above,xshift=20] {$\substack{\looseid \looseid \\ (\alpha \times \looseid \times \looseid)}$}   (D2);
\draw[doubletighteq] (D3) to  (E3);
\draw[doubleloose] (E3) to node[above,xshift=-10pt, yshift=2pt] {$\substack{\alpha \looseid \looseid}$}   (C5);
%%%%Row G
\node (G3) at (2.5,3) {$\substack{\tens (\tens \times \transid) \\(\transid \times \transid \times \tens) \\ (\tens \times \transid \times \transid \times \transid)}$};
\node (G4) at (4,3) {$\substack{\tens (\transid \times \tens) \\ ( \transid \times \transid \times \tens) \\ (\tens \times \transid \times \transid \times \transid)}$};
\node (G5) at (5,3) {$\substack{\tens(\tens\times \transid)\\(\transid  \times \transid \times \tens)\\(\transid \times \transid  \times \transid \times \tens)}$};
%%%%%%
\draw[doubletighteq] (E1) to  (G3);
\draw[doubleloose] (G3) to  node[above]{$\substack{ \alpha \looseid \looseid}$}(G4);
\draw[doubletighteq] (G4) to  (G5);
\draw[doubleloose] (G5) to  node[above, xshift=20pt]{$\substack{ \alpha \looseid  \looseid}$}(A6);
\draw[doubleloose] (G3) to  node[right]{$\substack{ \looseid \\ (\alpha \times \looseid) \\ \looseid}$}(E3);
%%%%%Extra cells
\draw[doubleloose] (G3) to[out=60, in=200] node[above, xshift=-15pt]{$\substack{S(\pi) \looseid}$} (A6);
\draw[doubleloose] (G3) to[out=15, in=230] node[below, xshift=15pt]{$\substack{T(\pi) \looseid }$}(A6);
%%%%%3-cells
\node at (3,7) {$\substack{\DDownarrow \iso }$};
\node at (1.6,6.7) {$\substack{\DDownarrow \iso }$};
\node at (4.5,6.7) {$\substack{\DDownarrow \iso }$};
\node at (4.25,6) {$\substack{\DDownarrow \tightid (\tightid \times \pi)}$};
\node at (.5,5) {$\substack{\DDownarrow \iso}$};
\node at (1.5,6) {$\substack{\DDownarrow \pi \tightid}$};
\node at (1.5,5.3) {$\substack{\DDownarrow \iso }$};
\node at (4,5.5) {$\substack{\DDownarrow \iso }$};
\node at (3,5) {$\substack{=}$};
\node at (1.5,4) {$\substack{\DDownarrow \iso}$};
\node at (3.5,4.5) {$\substack{\DDownarrow \iso}$};
\node at (4,4) {$\substack{\DDownarrow \pi \tightid}$};
\node at (4.5,3.5) {$\substack{\DDownarrow \iso}$};
\end{tikzpicture} \hspace{-2cm}
\end{equation*}
\begin{equation}\label{eq:monobjeq1}
=
\end{equation}
\begin{equation*}\hspace{-2cm}
\begin{tikzpicture}[xscale=2.5, yscale=3]
%%%%Row A
\node (A0) at (0,6) {$\substack{\tens(\tens \times \transid)\\(\tens \times \transid \times \transid)\\(\tens \times \transid \times \transid \times \transid)}$};
\node (A1) at (0,7) {$\substack{\tens(\tens \times \transid)\\(\tens \times \transid \times \transid)\\(\transid \times \tens \times \transid \times \transid)}$};
\node (A2) at (1.5,7.5) {$\substack{\tens(\tens \times \transid)\\(\transid \times \tens \times \transid)\\(\transid \times \tens \times \transid \times \transid)}$};
\node (A3) at (3,7) {$\substack{\tens(\tens \times \transid)\\(\transid \times \tens \times \transid)\\(\transid \times \transid \times \tens \times \transid)}$};
\node (A4) at (4,7) {$\substack{\tens(\transid \times \tens)\\(\transid \times \tens \times \transid)\\(\transid \times \transid \times \tens \times \transid)}$};
\node (A5) at (5,6) {$\substack{\tens(\transid \times \tens)\\(\transid  \times \transid \times \tens)\\(\transid \times \transid \times \tens \times \transid)}$};
\node (A6) at (6,5) {$\substack{\tens(\transid \times \tens)\\(\transid  \times \transid \times \tens)\\(\transid \times \transid \times \transid \times \tens )}$};
%%%%%%
\draw[doubleloose] (A0) to node[left, xshift=-5pt]{$\substack{\looseid \looseid \\(\alpha \times \looseid \times \looseid)}$} (A1);
\draw[doubleloose] (A1) to node[above, xshift=-16pt]{$\substack{\looseid (\alpha \times \looseid) \looseid}$}
(A2);
\draw[doubleloose] (A2) to node[above, xshift=16pt]{$\substack{\looseid \looseid \\ (\looseid \times \alpha \times \looseid)}$} (A3);
\draw[doubleloose] (A3) to node[above]{$\substack{ \alpha \looseid \looseid}$} (A4);
\draw[doubleloose] (A4) to node[above, xshift=16pt]{$\substack{\looseid (\looseid \times \alpha) \looseid}$} (A5);
\draw[doubleloose] (A5) to node[above, xshift=16pt]{$\substack{ \looseid \looseid \\ (\looseid \times \looseid \times \alpha)}$} (A6);
%%%%row B
\node (B1) at (1.5,5.5) {$\substack{\tens (\tens \times \transid) \\ (\transid \times \tens \times \transid) \\ (\tens \times \transid \times \transid \times \transid)}$};
\node (B2) at (3,6) {$\substack{\tens (\tens \times \transid) \\ (\tens \times \transid \times \transid) \\ (\transid \times \transid \times \tens \times \transid)}$};
%%%%%%
\draw[doubleloose] (A0) to node[above, xshift=16pt]{$\substack{\looseid (\alpha \times \looseid)  \looseid}$} (B1);
\draw[doubletighteq] (B1) to  (B2);
\draw[doubleloose] (B2) to node[below, xshift=16pt]{$\substack{ \looseid (\alpha \times \looseid) }$} (A3);
%%%%row C
\node (C1) at (2.5,4.5) {$\substack{\tens( \tens \times \transid)\\( \transid \times \tens \times  \transid)\\(\tens \times \transid \times \transid \times  \transid)}$};
\node (C3) at (3.5,5) {$\substack{\tens( \transid \times \tens)\\( \tens \times \transid \times \transid)\\(\transid \times \transid \times \tens \times \transid)}$};
\node (C4) at (4.5,5) {$\substack{\tens(\tens \times \transid)\\(\transid  \times \transid \times \tens)\\(\transid \times \transid \times \tens \times \transid)}$};
%%%%%%
\draw[doubleloose] (B1) to node[right, xshift=5pt]{$\substack{ \alpha \looseid  \looseid}$} (C1);
\draw[doubletighteq] (C1) to  (C3);
\draw[doubleloose] (B2) to node[left, xshift=-3pt]{$\substack{ \alpha \looseid  \looseid}$} (C3);
\draw[doubletighteq] (C3) to (C4);
\draw[doubleloose] (C4) to node[right, xshift=2pt]{$\substack{ \alpha \looseid  \looseid}$} (A5);
%%%%row D
\node (D5) at (5,4) {$\substack{\tens(\tens \times \transid)\\(\transid  \times \transid \times \tens)\\(\transid \times \transid  \times \transid \times \tens)}$};
%%%%%%
\draw[doubleloose] (C4) to node[below, xshift=-16pt]{$\substack{  \looseid  \looseid \\ (\looseid \times \looseid \times \alpha)}$} (D5);
\draw[doubleloose] (D5) to  node[right, xshift=2pt]{$\substack{ \alpha \looseid  \looseid}$}(A6);
%%%%row E
\node (E2) at (0,4.5) {$\substack{\tens (\transid \times \tens)\\(\tens \times \transid \times \transid) \\ (\tens \times \transid \times \transid \times \transid)}$};
\node (E3) at (1,3.5) {$\substack{\tens (\tens \times \transid) \\(\transid \times \transid \times \tens) \\ (\tens \times \transid \times \transid \times \transid)}$};
\node (E4) at (2.5,3.5) {$\substack{\tens (\transid \times \tens) \\ ( \transid \times \transid \times \tens) \\ (\tens \times \transid \times \transid \times \transid)}$};
%%%%%%
\draw[doubleloose] (A0) to node[left, xshift=-2pt]{$\substack{ \alpha \looseid \looseid}$} (E2);
\draw[doubletighteq] (E2) to  (E3);
\draw[doubleloose] (E3) to  node[above]{$\substack{ \alpha \looseid \looseid}$}(E4);
\draw[doubletighteq] (E4) to  (D5);
\draw[doubleloose] (C1) to  node[right]{$\substack{ \looseid\\ (\looseid \times \alpha) \\ \looseid}$}(E4);
%%%%% 3cells
\node at (1.5,6.5) {$\substack{ \DDownarrow \tightid (\pi \times \tightid)}$};
\node at (4,6) {$\substack{\DDownarrow \pi \tightid} $};
\node at (2,4.75) {$\substack{\DDownarrow \iso} $};
\node at (1.25,4.75) {$\substack{\DDownarrow \pi \tightid} $};
\node at (.5,4.25) {$\substack{\DDownarrow \iso} $};
\node at (2.5,5.25) {$\substack{=}$};
\node at (5.25,5) {$\substack{\DDownarrow \iso}$};
\node at (3.5,4.5) {$\substack{\DDownarrow \iso} $};
%%%%%Extra cells
\draw[doubleloose] (A0) to[out=60, in=180] node[above, xshift=-15pt]{$\substack{\looseid (S(\pi) \times \looseid) }$} (A3);
\draw[doubleloose] (A0) to[out=0, in=230] node[below, xshift=15pt]{$\substack{\looseid (T(\pi) \times \looseid)  }$}(A3);
%%%%%Extra cells
\draw[doubleloose] (B2) to[out=35, in=155] node[above]{$\substack{S(\pi) \looseid }$} (A5);
\draw[doubleloose] (B2) to[out=-35, in=205] node[below]{$\substack{T(\pi)  \looseid  }$}(A5);
%%%%%Extra cells
\draw[doubleloose] (A0) to[out=-30, in=120] node[right]{$\substack{S(\pi) \looseid }$} (E4);
\draw[doubleloose] (A0) to[out=-70, in=160] node[left]{$\substack{T(\pi)  \looseid  }$}(E4);
\end{tikzpicture}\hspace{-2cm}
\end{equation*}}
\end{document}  \newpage
%
\documentclass[12pt]{ociamthesis}
\usepackage{tikz}
\usepackage{amsmath}
\usepackage{rotating}

\usepackage{amssymb,amsmath,stmaryrd,txfonts,mathrsfs,amsthm}
\usepackage[all,2cell]{xy}
\usepackage[neveradjust]{paralist}
\usepackage{hyperref}
\usepackage{mathtools}
\usepackage{tikz}
\usetikzlibrary{trees}
\usetikzlibrary{topaths}
\usetikzlibrary{decorations}
\usetikzlibrary{decorations.pathreplacing}
\usetikzlibrary{decorations.pathmorphing}
\usetikzlibrary{decorations.markings}
\usetikzlibrary{matrix,backgrounds,folding}
\usetikzlibrary{chains,scopes,positioning,fit}
\usetikzlibrary{arrows,shadows}
\usetikzlibrary{calc} 
\usetikzlibrary{chains}
\usetikzlibrary{shapes,shapes.geometric,shapes.misc}
\usepackage{smbicat}


\makeatletter
\let\ea\expandafter

%% Defining commands that are always in math mode.
\def\mdef#1#2{\ea\ea\ea\gdef\ea\ea\noexpand#1\ea{\ea\ensuremath\ea{#2}}}
\def\alwaysmath#1{\ea\ea\ea\global\ea\ea\ea\let\ea\ea\csname your@#1\endcsname\csname #1\endcsname
  \ea\def\csname #1\endcsname{\ensuremath{\csname your@#1\endcsname}}}

% Script letters
\newcommand{\sA}{\ensuremath{\mathscr{A}}}
\newcommand{\sB}{\ensuremath{\mathscr{B}}}
\newcommand{\sC}{\ensuremath{\mathscr{C}}}
\newcommand{\sD}{\ensuremath{\mathscr{D}}}
\newcommand{\sE}{\ensuremath{\mathscr{E}}}
\newcommand{\sF}{\ensuremath{\mathscr{F}}}
\newcommand{\sG}{\ensuremath{\mathscr{G}}}
\newcommand{\sH}{\ensuremath{\mathscr{H}}}
\newcommand{\sI}{\ensuremath{\mathscr{I}}}
\newcommand{\sJ}{\ensuremath{\mathscr{J}}}
\newcommand{\sK}{\ensuremath{\mathscr{K}}}
\newcommand{\sL}{\ensuremath{\mathscr{L}}}
\newcommand{\sM}{\ensuremath{\mathscr{M}}}
\newcommand{\sN}{\ensuremath{\mathscr{N}}}
\newcommand{\sO}{\ensuremath{\mathscr{O}}}
\newcommand{\sP}{\ensuremath{\mathscr{P}}}
\newcommand{\sQ}{\ensuremath{\mathscr{Q}}}
\newcommand{\sR}{\ensuremath{\mathscr{R}}}
\newcommand{\sS}{\ensuremath{\mathscr{S}}}
\newcommand{\sT}{\ensuremath{\mathscr{T}}}
\newcommand{\sU}{\ensuremath{\mathscr{U}}}
\newcommand{\sV}{\ensuremath{\mathscr{V}}}
\newcommand{\sW}{\ensuremath{\mathscr{W}}}
\newcommand{\sX}{\ensuremath{\mathscr{X}}}
\newcommand{\sY}{\ensuremath{\mathscr{Y}}}
\newcommand{\sZ}{\ensuremath{\mathscr{Z}}}

% Calligraphic letters
\newcommand{\cA}{\ensuremath{\mathcal{A}}}
\newcommand{\cB}{\ensuremath{\mathcal{B}}}
\newcommand{\cC}{\ensuremath{\mathcal{C}}}
\newcommand{\cD}{\ensuremath{\mathcal{D}}}
\newcommand{\cE}{\ensuremath{\mathcal{E}}}
\newcommand{\cF}{\ensuremath{\mathcal{F}}}
\newcommand{\cG}{\ensuremath{\mathcal{G}}}
\newcommand{\cH}{\ensuremath{\mathcal{H}}}
\newcommand{\cI}{\ensuremath{\mathcal{I}}}
\newcommand{\cJ}{\ensuremath{\mathcal{J}}}
\newcommand{\cK}{\ensuremath{\mathcal{K}}}
\newcommand{\cL}{\ensuremath{\mathcal{L}}}
\newcommand{\cM}{\ensuremath{\mathcal{M}}}
\newcommand{\cN}{\ensuremath{\mathcal{N}}}
\newcommand{\cO}{\ensuremath{\mathcal{O}}}
\newcommand{\cP}{\ensuremath{\mathcal{P}}}
\newcommand{\cQ}{\ensuremath{\mathcal{Q}}}
\newcommand{\cR}{\ensuremath{\mathcal{R}}}
\newcommand{\cS}{\ensuremath{\mathcal{S}}}
\newcommand{\cT}{\ensuremath{\mathcal{T}}}
\newcommand{\cU}{\ensuremath{\mathcal{U}}}
\newcommand{\cV}{\ensuremath{\mathcal{V}}}
\newcommand{\cW}{\ensuremath{\mathcal{W}}}
\newcommand{\cX}{\ensuremath{\mathcal{X}}}
\newcommand{\cY}{\ensuremath{\mathcal{Y}}}
\newcommand{\cZ}{\ensuremath{\mathcal{Z}}}

% blackboard bold letters
\newcommand{\lA}{\ensuremath{\mathbb{A}}}
\newcommand{\lB}{\ensuremath{\mathbb{B}}}
\newcommand{\lC}{\ensuremath{\mathbb{C}}}
\newcommand{\lD}{\ensuremath{\mathbb{D}}}
\newcommand{\lE}{\ensuremath{\mathbb{E}}}
\newcommand{\lF}{\ensuremath{\mathbb{F}}}
\newcommand{\lG}{\ensuremath{\mathbb{G}}}
\newcommand{\lH}{\ensuremath{\mathbb{H}}}
\newcommand{\lI}{\ensuremath{\mathbb{I}}}
\newcommand{\lJ}{\ensuremath{\mathbb{J}}}
\newcommand{\lK}{\ensuremath{\mathbb{K}}}
\newcommand{\lL}{\ensuremath{\mathbb{L}}}
\newcommand{\lM}{\ensuremath{\mathbb{M}}}
\newcommand{\lN}{\ensuremath{\mathbb{N}}}
\newcommand{\lO}{\ensuremath{\mathbb{O}}}
\newcommand{\lP}{\ensuremath{\mathbb{P}}}
\newcommand{\lQ}{\ensuremath{\mathbb{Q}}}
\newcommand{\lR}{\ensuremath{\mathbb{R}}}
\newcommand{\lS}{\ensuremath{\mathbb{S}}}
\newcommand{\lT}{\ensuremath{\mathbb{T}}}
\newcommand{\lU}{\ensuremath{\mathbb{U}}}
\newcommand{\lV}{\ensuremath{\mathbb{V}}}
\newcommand{\lW}{\ensuremath{\mathbb{W}}}
\newcommand{\lX}{\ensuremath{\mathbb{X}}}
\newcommand{\lY}{\ensuremath{\mathbb{Y}}}
\newcommand{\lZ}{\ensuremath{\mathbb{Z}}}

% bold letters
\newcommand{\bA}{\ensuremath{\mathbf{A}}}
\newcommand{\bB}{\ensuremath{\mathbf{B}}}
\newcommand{\bC}{\ensuremath{\mathbf{C}}}
\newcommand{\bD}{\ensuremath{\mathbf{D}}}
\newcommand{\bE}{\ensuremath{\mathbf{E}}}
\newcommand{\bF}{\ensuremath{\mathbf{F}}}
\newcommand{\bG}{\ensuremath{\mathbf{G}}}
\newcommand{\bH}{\ensuremath{\mathbf{H}}}
\newcommand{\bI}{\ensuremath{\mathbf{I}}}
\newcommand{\bJ}{\ensuremath{\mathbf{J}}}
\newcommand{\bK}{\ensuremath{\mathbf{K}}}
\newcommand{\bL}{\ensuremath{\mathbf{L}}}
\newcommand{\bM}{\ensuremath{\mathbf{M}}}
\newcommand{\bN}{\ensuremath{\mathbf{N}}}
\newcommand{\bO}{\ensuremath{\mathbf{O}}}
\newcommand{\bP}{\ensuremath{\mathbf{P}}}
\newcommand{\bQ}{\ensuremath{\mathbf{Q}}}
\newcommand{\bR}{\ensuremath{\mathbf{R}}}
\newcommand{\bS}{\ensuremath{\mathbf{S}}}
\newcommand{\bT}{\ensuremath{\mathbf{T}}}
\newcommand{\bU}{\ensuremath{\mathbf{U}}}
\newcommand{\bV}{\ensuremath{\mathbf{V}}}
\newcommand{\bW}{\ensuremath{\mathbf{W}}}
\newcommand{\bX}{\ensuremath{\mathbf{X}}}
\newcommand{\bY}{\ensuremath{\mathbf{Y}}}
\newcommand{\bZ}{\ensuremath{\mathbf{Z}}}

% fraktur letters
\newcommand{\fa}{\ensuremath{\mathfrak{a}}}
\newcommand{\fb}{\ensuremath{\mathfrak{b}}}
\newcommand{\fc}{\ensuremath{\mathfrak{c}}}
\newcommand{\fd}{\ensuremath{\mathfrak{d}}}
\newcommand{\fe}{\ensuremath{\mathfrak{e}}}
\newcommand{\ff}{\ensuremath{\mathfrak{f}}}
\newcommand{\fg}{\ensuremath{\mathfrak{g}}}
\newcommand{\fh}{\ensuremath{\mathfrak{h}}}
\newcommand{\fj}{\ensuremath{\mathfrak{j}}}
\newcommand{\fk}{\ensuremath{\mathfrak{k}}}
\newcommand{\fl}{\ensuremath{\mathfrak{l}}}
\newcommand{\fm}{\ensuremath{\mathfrak{m}}}
\newcommand{\fn}{\ensuremath{\mathfrak{n}}}
\newcommand{\fo}{\ensuremath{\mathfrak{o}}}
\newcommand{\fp}{\ensuremath{\mathfrak{p}}}
\newcommand{\fq}{\ensuremath{\mathfrak{q}}}
\newcommand{\fr}{\ensuremath{\mathfrak{r}}}
\newcommand{\fs}{\ensuremath{\mathfrak{s}}}
\newcommand{\ft}{\ensuremath{\mathfrak{t}}}
\newcommand{\fu}{\ensuremath{\mathfrak{u}}}
\newcommand{\fv}{\ensuremath{\mathfrak{v}}}
\newcommand{\fw}{\ensuremath{\mathfrak{w}}}
\newcommand{\fx}{\ensuremath{\mathfrak{x}}}
\newcommand{\fy}{\ensuremath{\mathfrak{y}}}
\newcommand{\fz}{\ensuremath{\mathfrak{z}}}

% fraktur letters
\newcommand{\fA}{\ensuremath{\mathfrak{A}}}
\newcommand{\fB}{\ensuremath{\mathfrak{B}}}
\newcommand{\fC}{\ensuremath{\mathfrak{C}}}

\mdef\fahat{\hat{\fa}}

% Underline letters
\newcommand{\uA}{\ensuremath{\underline{A}}}
\newcommand{\uB}{\ensuremath{\underline{B}}}
\newcommand{\uC}{\ensuremath{\underline{C}}}
\newcommand{\uD}{\ensuremath{\underline{D}}}
\newcommand{\uE}{\ensuremath{\underline{E}}}
\newcommand{\uF}{\ensuremath{\underline{F}}}
\newcommand{\uG}{\ensuremath{\underline{G}}}
\newcommand{\uH}{\ensuremath{\underline{H}}}
\newcommand{\uI}{\ensuremath{\underline{I}}}
\newcommand{\uJ}{\ensuremath{\underline{J}}}
\newcommand{\uK}{\ensuremath{\underline{K}}}
\newcommand{\uL}{\ensuremath{\underline{L}}}
\newcommand{\uM}{\ensuremath{\underline{M}}}
\newcommand{\uN}{\ensuremath{\underline{N}}}
\newcommand{\uO}{\ensuremath{\underline{O}}}
\newcommand{\uP}{\ensuremath{\underline{P}}}
\newcommand{\uQ}{\ensuremath{\underline{Q}}}
\newcommand{\uR}{\ensuremath{\underline{R}}}
\newcommand{\uS}{\ensuremath{\underline{S}}}
\newcommand{\uT}{\ensuremath{\underline{T}}}
\newcommand{\uU}{\ensuremath{\underline{U}}}
\newcommand{\uV}{\ensuremath{\underline{V}}}
\newcommand{\uW}{\ensuremath{\underline{W}}}
\newcommand{\uX}{\ensuremath{\underline{X}}}
\newcommand{\uY}{\ensuremath{\underline{Y}}}
\newcommand{\uZ}{\ensuremath{\underline{Z}}}

% bars
\newcommand{\Abar}{\ensuremath{\overline{A}}}
\newcommand{\Bbar}{\ensuremath{\overline{B}}}
\newcommand{\Cbar}{\ensuremath{\overline{C}}}
\newcommand{\Dbar}{\ensuremath{\overline{D}}}
\newcommand{\Ebar}{\ensuremath{\overline{E}}}
\newcommand{\Fbar}{\ensuremath{\overline{F}}}
\newcommand{\Gbar}{\ensuremath{\overline{G}}}
\newcommand{\Hbar}{\ensuremath{\overline{H}}}
\newcommand{\Ibar}{\ensuremath{\overline{I}}}
\newcommand{\Jbar}{\ensuremath{\overline{J}}}
\newcommand{\Kbar}{\ensuremath{\overline{K}}}
\newcommand{\Lbar}{\ensuremath{\overline{L}}}
\newcommand{\Mbar}{\ensuremath{\overline{M}}}
\newcommand{\Nbar}{\ensuremath{\overline{N}}}
\newcommand{\Obar}{\ensuremath{\overline{O}}}
\newcommand{\Pbar}{\ensuremath{\overline{P}}}
\newcommand{\Qbar}{\ensuremath{\overline{Q}}}
\newcommand{\Rbar}{\ensuremath{\overline{R}}}
\newcommand{\Sbar}{\ensuremath{\overline{S}}}
\newcommand{\Tbar}{\ensuremath{\overline{T}}}
\newcommand{\Ubar}{\ensuremath{\overline{U}}}
\newcommand{\Vbar}{\ensuremath{\overline{V}}}
\newcommand{\Wbar}{\ensuremath{\overline{W}}}
\newcommand{\Xbar}{\ensuremath{\overline{X}}}
\newcommand{\Ybar}{\ensuremath{\overline{Y}}}
\newcommand{\Zbar}{\ensuremath{\overline{Z}}}
\newcommand{\abar}{\ensuremath{\overline{a}}}
\newcommand{\bbar}{\ensuremath{\overline{b}}}
\newcommand{\cbar}{\ensuremath{\overline{c}}}
\newcommand{\dbar}{\ensuremath{\overline{d}}}
\newcommand{\ebar}{\ensuremath{\overline{e}}}
\newcommand{\fbar}{\ensuremath{\overline{f}}}
\newcommand{\gbar}{\ensuremath{\overline{g}}}
%\newcommand{\hbar}{\ensuremath{\overline{h}}} % whoops, \hbar means something else!
\newcommand{\ibar}{\ensuremath{\overline{\imath}}}
\newcommand{\jbar}{\ensuremath{\overline{\jmath}}}
\newcommand{\kbar}{\ensuremath{\overline{k}}}
\newcommand{\lbar}{\ensuremath{\overline{l}}}
\newcommand{\mbar}{\ensuremath{\overline{m}}}
\newcommand{\nbar}{\ensuremath{\overline{n}}}
%\newcommand{\obar}{\ensuremath{\overline{o}}}
\newcommand{\pbar}{\ensuremath{\overline{p}}}
\newcommand{\qbar}{\ensuremath{\overline{q}}}
\newcommand{\rbar}{\ensuremath{\overline{r}}}
\newcommand{\sbar}{\ensuremath{\overline{s}}}
\newcommand{\tbar}{\ensuremath{\overline{t}}}
\newcommand{\ubar}{\ensuremath{\overline{u}}}
\newcommand{\vbar}{\ensuremath{\overline{v}}}
\newcommand{\wbar}{\ensuremath{\overline{w}}}
\newcommand{\xbar}{\ensuremath{\overline{x}}}
\newcommand{\ybar}{\ensuremath{\overline{y}}}
\newcommand{\zbar}{\ensuremath{\overline{z}}}

% tildes
\newcommand{\Atil}{\ensuremath{\widetilde{A}}}
\newcommand{\Btil}{\ensuremath{\widetilde{B}}}
\newcommand{\Ctil}{\ensuremath{\widetilde{C}}}
\newcommand{\Dtil}{\ensuremath{\widetilde{D}}}
\newcommand{\Etil}{\ensuremath{\widetilde{E}}}
\newcommand{\Ftil}{\ensuremath{\widetilde{F}}}
\newcommand{\Gtil}{\ensuremath{\widetilde{G}}}
\newcommand{\Htil}{\ensuremath{\widetilde{H}}}
\newcommand{\Itil}{\ensuremath{\widetilde{I}}}
\newcommand{\Jtil}{\ensuremath{\widetilde{J}}}
\newcommand{\Ktil}{\ensuremath{\widetilde{K}}}
\newcommand{\Ltil}{\ensuremath{\widetilde{L}}}
\newcommand{\Mtil}{\ensuremath{\widetilde{M}}}
\newcommand{\Ntil}{\ensuremath{\widetilde{N}}}
\newcommand{\Otil}{\ensuremath{\widetilde{O}}}
\newcommand{\Ptil}{\ensuremath{\widetilde{P}}}
\newcommand{\Qtil}{\ensuremath{\widetilde{Q}}}
\newcommand{\Rtil}{\ensuremath{\widetilde{R}}}
\newcommand{\Stil}{\ensuremath{\widetilde{S}}}
\newcommand{\Ttil}{\ensuremath{\widetilde{T}}}
\newcommand{\Util}{\ensuremath{\widetilde{U}}}
\newcommand{\Vtil}{\ensuremath{\widetilde{V}}}
\newcommand{\Wtil}{\ensuremath{\widetilde{W}}}
\newcommand{\Xtil}{\ensuremath{\widetilde{X}}}
\newcommand{\Ytil}{\ensuremath{\widetilde{Y}}}
\newcommand{\Ztil}{\ensuremath{\widetilde{Z}}}
\newcommand{\atil}{\ensuremath{\widetilde{a}}}
\newcommand{\btil}{\ensuremath{\widetilde{b}}}
\newcommand{\ctil}{\ensuremath{\widetilde{c}}}
\newcommand{\dtil}{\ensuremath{\widetilde{d}}}
\newcommand{\etil}{\ensuremath{\widetilde{e}}}
\newcommand{\ftil}{\ensuremath{\widetilde{f}}}
\newcommand{\gtil}{\ensuremath{\widetilde{g}}}
\newcommand{\htil}{\ensuremath{\widetilde{h}}}
\newcommand{\itil}{\ensuremath{\widetilde{\imath}}}
\newcommand{\jtil}{\ensuremath{\widetilde{\jmath}}}
\newcommand{\ktil}{\ensuremath{\widetilde{k}}}
\newcommand{\ltil}{\ensuremath{\widetilde{l}}}
\newcommand{\mtil}{\ensuremath{\widetilde{m}}}
\newcommand{\ntil}{\ensuremath{\widetilde{n}}}
\newcommand{\otil}{\ensuremath{\widetilde{o}}}
\newcommand{\ptil}{\ensuremath{\widetilde{p}}}
\newcommand{\qtil}{\ensuremath{\widetilde{q}}}
\newcommand{\rtil}{\ensuremath{\widetilde{r}}}
\newcommand{\stil}{\ensuremath{\widetilde{s}}}
\newcommand{\ttil}{\ensuremath{\widetilde{t}}}
\newcommand{\util}{\ensuremath{\widetilde{u}}}
\newcommand{\vtil}{\ensuremath{\widetilde{v}}}
\newcommand{\wtil}{\ensuremath{\widetilde{w}}}
\newcommand{\xtil}{\ensuremath{\widetilde{x}}}
\newcommand{\ytil}{\ensuremath{\widetilde{y}}}
\newcommand{\ztil}{\ensuremath{\widetilde{z}}}

% Hats
\newcommand{\Ahat}{\ensuremath{\widehat{A}}}
\newcommand{\Bhat}{\ensuremath{\widehat{B}}}
\newcommand{\Chat}{\ensuremath{\widehat{C}}}
\newcommand{\Dhat}{\ensuremath{\widehat{D}}}
\newcommand{\Ehat}{\ensuremath{\widehat{E}}}
\newcommand{\Fhat}{\ensuremath{\widehat{F}}}
\newcommand{\Ghat}{\ensuremath{\widehat{G}}}
\newcommand{\Hhat}{\ensuremath{\widehat{H}}}
\newcommand{\Ihat}{\ensuremath{\widehat{I}}}
\newcommand{\Jhat}{\ensuremath{\widehat{J}}}
\newcommand{\Khat}{\ensuremath{\widehat{K}}}
\newcommand{\Lhat}{\ensuremath{\widehat{L}}}
\newcommand{\Mhat}{\ensuremath{\widehat{M}}}
\newcommand{\Nhat}{\ensuremath{\widehat{N}}}
\newcommand{\Ohat}{\ensuremath{\widehat{O}}}
\newcommand{\Phat}{\ensuremath{\widehat{P}}}
\newcommand{\Qhat}{\ensuremath{\widehat{Q}}}
\newcommand{\Rhat}{\ensuremath{\widehat{R}}}
\newcommand{\Shat}{\ensuremath{\widehat{S}}}
\newcommand{\That}{\ensuremath{\widehat{T}}}
\newcommand{\Uhat}{\ensuremath{\widehat{U}}}
\newcommand{\Vhat}{\ensuremath{\widehat{V}}}
\newcommand{\What}{\ensuremath{\widehat{W}}}
\newcommand{\Xhat}{\ensuremath{\widehat{X}}}
\newcommand{\Yhat}{\ensuremath{\widehat{Y}}}
\newcommand{\Zhat}{\ensuremath{\widehat{Z}}}
\newcommand{\ahat}{\ensuremath{\hat{a}}}
\newcommand{\bhat}{\ensuremath{\hat{b}}}
\newcommand{\chat}{\ensuremath{\hat{c}}}
\newcommand{\dhat}{\ensuremath{\hat{d}}}
\newcommand{\ehat}{\ensuremath{\hat{e}}}
\newcommand{\fhat}{\ensuremath{\hat{f}}}
\newcommand{\ghat}{\ensuremath{\hat{g}}}
\newcommand{\hhat}{\ensuremath{\hat{h}}}
\newcommand{\ihat}{\ensuremath{\hat{\imath}}}
\newcommand{\jhat}{\ensuremath{\hat{\jmath}}}
\newcommand{\khat}{\ensuremath{\hat{k}}}
\newcommand{\lhat}{\ensuremath{\hat{l}}}
\newcommand{\mhat}{\ensuremath{\hat{m}}}
\newcommand{\nhat}{\ensuremath{\hat{n}}}
\newcommand{\ohat}{\ensuremath{\hat{o}}}
\newcommand{\phat}{\ensuremath{\hat{p}}}
\newcommand{\qhat}{\ensuremath{\hat{q}}}
\newcommand{\rhat}{\ensuremath{\hat{r}}}
\newcommand{\shat}{\ensuremath{\hat{s}}}
\newcommand{\that}{\ensuremath{\hat{t}}}
\newcommand{\uhat}{\ensuremath{\hat{u}}}
\newcommand{\vhat}{\ensuremath{\hat{v}}}
\newcommand{\what}{\ensuremath{\hat{w}}}
\newcommand{\xhat}{\ensuremath{\hat{x}}}
\newcommand{\yhat}{\ensuremath{\hat{y}}}
\newcommand{\zhat}{\ensuremath{\hat{z}}}

%% FONTS AND DECORATION FOR GREEK LETTERS

%% the package `mathbbol' gives us blackboard bold greek and numbers,
%% but it does it by redefining \mathbb to use a different font, so that
%% all the other \mathbb letters look different too.  Here we import the
%% font with bb greek and numbers, but assign it a different name,
%% \mathbbb, so as not to replace the usual one.
\DeclareSymbolFont{bbold}{U}{bbold}{m}{n}
\DeclareSymbolFontAlphabet{\mathbbb}{bbold}
\newcommand{\bbDelta}{\ensuremath{\mathbbb{\Delta}}}
\newcommand{\bbone}{\ensuremath{\mathbbb{1}}}
\newcommand{\bbtwo}{\ensuremath{\mathbbb{2}}}
\newcommand{\bbthree}{\ensuremath{\mathbbb{3}}}

% greek with bars
\newcommand{\albar}{\ensuremath{\overline{\alpha}}}
\newcommand{\bebar}{\ensuremath{\overline{\beta}}}
\newcommand{\gmbar}{\ensuremath{\overline{\gamma}}}
\newcommand{\debar}{\ensuremath{\overline{\delta}}}
\newcommand{\phibar}{\ensuremath{\overline{\varphi}}}
\newcommand{\psibar}{\ensuremath{\overline{\psi}}}
\newcommand{\xibar}{\ensuremath{\overline{\xi}}}
\newcommand{\ombar}{\ensuremath{\overline{\omega}}}

% greek with hats
\newcommand{\alhat}{\ensuremath{\hat{\alpha}}}
\newcommand{\behat}{\ensuremath{\hat{\beta}}}
\newcommand{\gmhat}{\ensuremath{\hat{\gamma}}}
\newcommand{\dehat}{\ensuremath{\hat{\delta}}}

% greek with checks
\newcommand{\alchk}{\ensuremath{\check{\alpha}}}
\newcommand{\bechk}{\ensuremath{\check{\beta}}}
\newcommand{\gmchk}{\ensuremath{\check{\gamma}}}
\newcommand{\dechk}{\ensuremath{\check{\delta}}}

% greek with tildes
\newcommand{\altil}{\ensuremath{\widetilde{\alpha}}}
\newcommand{\betil}{\ensuremath{\widetilde{\beta}}}
\newcommand{\gmtil}{\ensuremath{\widetilde{\gamma}}}
\newcommand{\phitil}{\ensuremath{\widetilde{\varphi}}}
\newcommand{\psitil}{\ensuremath{\widetilde{\psi}}}
\newcommand{\xitil}{\ensuremath{\widetilde{\xi}}}
\newcommand{\omtil}{\ensuremath{\widetilde{\omega}}}

% MISCELLANEOUS SYMBOLS
\mdef\del{\partial}
\mdef\delbar{\overline{\partial}}
\let\sm\wedge
\newcommand{\dd}[1]{\ensuremath{\frac{\partial}{\partial {#1}}}}
\newcommand{\inv}{^{-1}}
\newcommand{\dual}{^{\vee}}
\mdef\hf{\textstyle\frac{1}{2}}
\mdef\thrd{\textstyle\frac{1}{3}}
\mdef\qtr{\textstyle\frac{1}{4}}
\let\meet\wedge
\let\join\vee
\let\dn\downarrow
\newcommand{\op}{^{\mathit{op}}}
\newcommand{\co}{^{\mathit{co}}}
\newcommand{\coop}{^{\mathit{coop}}}
\let\adj\dashv
\SelectTips{cm}{}
\newdir{ >}{{}*!/-10pt/@{>}}    % extra spacing for tail arrows in XYpic
\newcommand{\pushoutcorner}[1][dr]{\save*!/#1+1.2pc/#1:(1,-1)@^{|-}\restore}
\newcommand{\pullbackcorner}[1][dr]{\save*!/#1-1.2pc/#1:(-1,1)@^{|-}\restore}
\let\iso\cong
\let\eqv\simeq
\let\cng\equiv
\mdef\Id{\mathrm{Id}}
\mdef\id{\mathrm{id}}
\alwaysmath{ell}
\alwaysmath{infty}
\alwaysmath{odot}
\def\frc#1/#2.{\frac{#1}{#2}}   % \frc x^2+1 / x^2-1 .
\mdef\ten{\mathrel{\otimes}}
\mdef\bigten{\bigotimes}
\mdef\sqten{\mathrel{\boxtimes}}
\def\pow(#1,#2){\mathop{\pitchfork}(#1,#2)} % powers and
\def\cpw{\mathop{\odot}}                    % copowers
\newcommand{\mathid}{\mbox{id}}
\newcommand{\cat}[1]{\ensuremath{\mathbf{#1}}}


%% OPERATORS
\DeclareMathOperator\lan{Lan}
\DeclareMathOperator\ran{Ran}
\DeclareMathOperator\colim{colim}
\DeclareMathOperator\coeq{coeq}
\DeclareMathOperator\eq{eq}
\DeclareMathOperator\Tot{Tot}
\DeclareMathOperator\cosk{cosk}
\DeclareMathOperator\sk{sk}
\DeclareMathOperator\im{im}
\DeclareMathOperator\Spec{Spec}
\DeclareMathOperator\Ho{Ho}
\DeclareMathOperator\Aut{Aut}
\DeclareMathOperator\End{End}
\DeclareMathOperator\Hom{Hom}
\DeclareMathOperator\Map{Map}

%% TIKZ ARROWS AND HIGHER CELLS
\makeatletter
\def\slashedarrowfill@#1#2#3#4#5{%
  $\m@th\thickmuskip0mu\medmuskip\thickmuskip\thinmuskip\thickmuskip
   \relax#5#1\mkern-7mu%
   \cleaders\hbox{$#5\mkern-2mu#2\mkern-2mu$}\hfill
   \mathclap{#3}\mathclap{#2}%
   \cleaders\hbox{$#5\mkern-2mu#2\mkern-2mu$}\hfill
   \mkern-7mu#4$%
}

\def\Rightslashedarrowfill@{%
  \slashedarrowfill@\Relbar\Relbar\Mapstochar\Rightarrow}
\newcommand\xslashedRightarrow[2][]{%
  \ext@arrow 0055{\Rightslashedarrowfill@}{#1}{#2}}
\def\hTo{\xslashedRightarrow{}}
\def\hToo{\xslashedRightarrow{\quad}}
\let\xhTo\xslashedRightarrow

\pagestyle{empty}

\newcommand{\Rightthreecell}{\RRightarrow}
\newcommand{\Rtwocell}{\Rightarrow}

\tikzstyle{doubletick}=[-implies, double equal sign distance, postaction={decorate},decoration={markings,mark=at position .5 with {\draw[-] (0,-0.1) -- (0,0.1);}}]

\tikzstyle{darrow}=[-implies, double equal sign distance]

\tikzstyle{doubleeq}=[double equal sign distance]


%% ARROWS
% \to already exists
\newcommand{\too}[1][]{\ensuremath{\overset{#1}{\longrightarrow}}}
\newcommand{\ot}{\ensuremath{\leftarrow}}
\newcommand{\oot}[1][]{\ensuremath{\overset{#1}{\longleftarrow}}}
\let\toot\rightleftarrows
\let\otto\leftrightarrows
\let\Impl\Rightarrow
\let\imp\Rightarrow
\let\toto\rightrightarrows
\let\into\hookrightarrow
\let\xinto\xhookrightarrow
\mdef\we{\overset{\sim}{\longrightarrow}}
\mdef\leftwe{\overset{\sim}{\longleftarrow}}
\let\mono\rightarrowtail
\let\leftmono\leftarrowtail
\let\cof\rightarrowtail
\let\leftcof\leftarrowtail
\let\epi\twoheadrightarrow
\let\leftepi\twoheadleftarrow
\let\fib\twoheadrightarrow
\let\leftfib\twoheadleftarrow
\let\cohto\rightsquigarrow
\let\maps\colon
\newcommand{\spam}{\,:\!}       % \maps for left arrows

\newsavebox{\DDownarrowbox}
\savebox{\DDownarrowbox}{\tikz[scale=1.5]{\draw[-implies,double equal
sign distance] (0,.1) -- (0,-.1); \draw (0,.1) -- (0,-.1);}}
\newcommand{\DDownarrow}{\mathrel{\raisebox{-.2em}{\usebox{\DDownarrowbox}}}}

\newsavebox{\RRightarrowbox}
\savebox{\RRightarrowbox}{\tikz[scale=1.5]{\draw[-implies,double equal
sign distance] (-.1,0) -- (.1,0); \draw (-.1,0) -- (.1,0);}}
\newcommand{\RRightarrow}{\mathrel{\raisebox{.2em}{\usebox{\RRightarrowbox}}}}

%\newsavebox{\Rightslashedarrowbox}
%\savebox{\Rightslashedarrowbox}{\tikz[scale=1.5]{\draw[Rightslashedarrow{}] (-.1,0) -- (1,0);}}
%\newcommand{\Rightslashedarrow}{\mathrel{\raisebox{-.2em}%{\usebox{\Rightslashedarrowbox}}}}


%% EXTENSIBLE ARROWS
\let\xto\xrightarrow
\let\xot\xleftarrow
% See Voss' Mathmode.tex for instructions on how to create new
% extensible arrows.
\def\rightarrowtailfill@{\arrowfill@{\Yright\joinrel\relbar}\relbar\rightarrow}
\newcommand\xrightarrowtail[2][]{\ext@arrow 0055{\rightarrowtailfill@}{#1}{#2}}
\let\xmono\xrightarrowtail
\let\xcof\xrightarrowtail
\def\twoheadrightarrowfill@{\arrowfill@{\relbar\joinrel\relbar}\relbar\twoheadrightarrow}
\newcommand\xtwoheadrightarrow[2][]{\ext@arrow 0055{\twoheadrightarrowfill@}{#1}{#2}}
\let\xepi\xtwoheadrightarrow
\let\xfib\xtwoheadrightarrow
% Let's leave the left-going ones until I need them.

%% EXTENSIBLE SLASHED ARROWS
% Making extensible slashed arrows, by modifying the underlying AMS code.
% Arguments are:
% 1 = arrowhead on the left (\relbar or \Relbar if none)
% 2 = fill character (usually \relbar or \Relbar)
% 3 = slash character (such as \mapstochar or \Mapstochar)
% 4 = arrowhead on the left (\relbar or \Relbar if none)
% 5 = display mode (\displaystyle etc)
\def\slashedarrowfill@#1#2#3#4#5{%
  $\m@th\thickmuskip0mu\medmuskip\thickmuskip\thinmuskip\thickmuskip
   \relax#5#1\mkern-7mu%
   \cleaders\hbox{$#5\mkern-2mu#2\mkern-2mu$}\hfill
   \mathclap{#3}\mathclap{#2}%
   \cleaders\hbox{$#5\mkern-2mu#2\mkern-2mu$}\hfill
   \mkern-7mu#4$%
}
% Here's the idea: \<slashed>arrowfill@ should be a box containing
% some stretchable space that is the "middle of the arrow".  This
% space is created as a "leader" using \cleader<thing>\hfill, which
% fills an \hfill of space with copies of <thing>.  Here instead of
% just one \cleader, we use two, with the slash in between (and an
% extra copy of the filler, to avoid extra space around the slash).
\def\rightslashedarrowfill@{%
  \slashedarrowfill@\relbar\relbar\mapstochar\rightarrow}
\newcommand\xslashedrightarrow[2][]{%
  \ext@arrow 0055{\rightslashedarrowfill@}{#1}{#2}}
\mdef\hto{\xslashedrightarrow{}}
\mdef\htoo{\xslashedrightarrow{\quad}}
\let\xhto\xslashedrightarrow

%% To get a slashed arrow in XYpic, do
% \[\xymatrix{A \ar[r]|-@{|} & B}\]

% ISOMORPHISMS
\def\xiso#1{\mathrel{\mathrlap{\smash{\xto[\smash{\raisebox{1.3mm}{$\scriptstyle\sim$}}]{#1}}}\hphantom{\xto{#1}}}}
\def\toiso{\xto{\smash{\raisebox{-.5mm}{$\scriptstyle\sim$}}}}

% SHADOWS
\def\shvar#1#2{{\ensuremath{%
  \hspace{1mm}\makebox[-1mm]{$#1\langle$}\makebox[0mm]{$#1\langle$}\hspace{1mm}%
  {#2}%
  \makebox[1mm]{$#1\rangle$}\makebox[0mm]{$#1\rangle$}%
}}}
\def\sh{\shvar{}}
\def\scriptsh{\shvar{\scriptstyle}}
\def\bigsh{\shvar{\big}}
\def\Bigsh{\shvar{\Big}}
\def\biggsh{\shvar{\bigg}}
\def\Biggsh{\shvar{\Bigg}}

%HIGHER CELLS



% THEOREM-TYPE ENVIRONMENTS, hacked to
%% (a) number all with the same numbers, and
%% (b) have the right names for autoref
\def\defthm#1#2{%
  \newtheorem{#1}{#2}[section]%
  \expandafter\def\csname #1autorefname\endcsname{#2}%
  \expandafter\let\csname c@#1\endcsname\c@thm}
\newtheorem{thm}{Theorem}[section]
\newcommand{\thmautorefname}{Theorem}
\defthm{cor}{Corollary}
\defthm{prop}{Proposition}
\defthm{lem}{Lemma}
\defthm{sch}{Scholium}
\defthm{assume}{Assumption}
\defthm{claim}{Claim}
\defthm{conj}{Conjecture}
\defthm{hyp}{Hypothesis}
\defthm{fact}{Fact}
\theoremstyle{definition}
\defthm{defn}{Definition}
\defthm{notn}{Notation}
\theoremstyle{remark}
\defthm{rmk}{Remark}
\defthm{eg}{Example}
\defthm{egs}{Examples}
\defthm{ex}{Exercise}
\defthm{ceg}{Counterexample}

% How to get QED symbols inside equations at the end of the statements
% of theorems.  AMS LaTeX knows how to do this inside equations at the
% end of *proofs* with \qedhere, and at the end of the statement of a
% theorem with \qed (meaning no proof will be given), but it can't
% seem to combine the two.  Use this instead.
\def\thmqedhere{\expandafter\csname\csname @currenvir\endcsname @qed\endcsname}

% Number numbered lists as (i), (ii), ...
\renewcommand{\theenumi}{(\roman{enumi})}
\renewcommand{\labelenumi}{\theenumi}

%% Labeling that keeps track of theorem-type names.  Superseded by
%% autoref from hyperref, as above, but we keep this in case we are
%% using a journal style file that is incompatible with hyperref.
% 
% \ifx\SK@label\undefined\let\SK@label\label\fi
% \let\your@thm\@thm
% \def\@thm#1#2#3{\gdef\currthmtype{#3}\your@thm{#1}{#2}{#3}}
% \def\xlabel#1{{\let\your@currentlabel\@currentlabel\def\@currentlabel
% {\currthmtype~\your@currentlabel}
% \SK@label{#1@}}\label{#1}}
% \def\xref#1{\ref{#1@}}

% Also number formulas with the theorem counter
\let\c@equation\c@thm
\numberwithin{equation}{section}

% Only show numbers for equations that are actually referenced (or
% whose tags are specified manually) - uses mathtools.
\mathtoolsset{showonlyrefs,showmanualtags}

%% Fix enumerate spacing with paralist.  This has two parts:
%%   1. enable mixing of "old spacing" lists with those adjusted by paralist
%%   2. allow us to specify a number based on which to adjust the spacing
%% For the first, use \killspacingtrue when you want the spacing
%% adjusted, then \killspacingfalse to turn adjustment off.  For the
%% second, use \maxenum=14 to set the widest number you want the
%% spacing to be calculated with.
\newlength\oldleftmargini       % save old spacing
\newlength\oldleftmarginii
\newlength\oldleftmarginiii
\newlength\oldleftmarginiv
\newlength\oldleftmarginv
\newlength\oldleftmarginvi
\newcount\maxenum
\maxenum=7
\newif\ifkillspacing
\def\@adjust@enum@labelwidth{%
  \advance\@listdepth by 1\relax
  \ifkillspacing                % do the paralist thing
    \csname c@\@enumctr\endcsname\maxenum
    \settowidth{\@tempdima}{%
      \csname label\@enumctr\endcsname\hspace{\labelsep}}%
    \csname leftmargin\romannumeral\@listdepth\endcsname
      \@tempdima
  \else                         % otherwise, reset it
    \csname fixspacing\romannumeral\@listdepth\endcsname
  \fi
  \advance\@listdepth by -1\relax}
% these commands, one for each enum level (I couldn't get a generic
% one to work), test whether oldleftmargin has been set yet, and if
% not, set it from leftmargin; otherwise, they reset leftmargin to
% it.  Just setting oldleftmargin to leftmargin in the preamble
% doesn't seem to work.
\def\fixspacingi{\ifnum\oldleftmargini=0\setlength\oldleftmargini\leftmargini\else\setlength\leftmargini\oldleftmargini\fi}
\def\fixspacingii{\ifnum\oldleftmarginii=0\setlength\oldleftmarginii\leftmarginii\else\setlength\leftmarginii\oldleftmarginii\fi}
\def\fixspacingiii{\ifnum\oldleftmarginiii=0\setlength\oldleftmarginiii\leftmarginiii\else\setlength\leftmarginiii\oldleftmarginiii\fi}
\def\fixspacingiv{\ifnum\oldleftmarginiv=0\setlength\oldleftmarginiv\leftmarginiv\else\setlength\leftmarginiv\oldleftmarginiv\fi}
\def\fixspacingv{\ifnum\oldleftmarginv=0\setlength\oldleftmarginv\leftmarginv\else\setlength\leftmarginv\oldleftmarginv\fi}
\def\fixspacingvi{\ifnum\oldleftmarginvi=0\setlength\oldleftmarginvi\leftmarginvi\else\setlength\leftmarginvi\oldleftmarginvi\fi}

%% Fix paralist references, so that we can refer to (1) instead of
%% just 1.
\def\pl@label#1#2{%
  \edef\pl@the{\noexpand#1{\@enumctr}}%
  \pl@lab\expandafter{\the\pl@lab\csname yourthe\@enumctr\endcsname}%
  \advance\@tempcnta1
  \pl@loop}
\def\@enumlabel@#1[#2]{%
  \@plmylabeltrue
  \@tempcnta0
  \pl@lab{}%
  \let\pl@the\pl@qmark
  \expandafter\pl@loop\@gobble#2\@@@
  \ifnum\@tempcnta=1\else
    \PackageWarning{paralist}{Incorrect label; no or multiple
      counters.\MessageBreak The label is: \@gobble#2}%
  \fi
  \expandafter\edef\csname label\@enumctr\endcsname{\the\pl@lab}%
  \expandafter\edef\csname the\@enumctr\endcsname{\the\pl@lab}%
  \expandafter\let\csname yourthe\@enumctr\endcsname\pl@the
  #1}


% GREEK LETTERS, ETC.
\alwaysmath{alpha}
\alwaysmath{beta}
\alwaysmath{gamma}
\alwaysmath{Gamma}
\alwaysmath{delta}
\alwaysmath{Delta}
\alwaysmath{epsilon}
\mdef\ep{\varepsilon}
\alwaysmath{zeta}
\alwaysmath{eta}
\alwaysmath{theta}
\alwaysmath{Theta}
\alwaysmath{iota}
\alwaysmath{kappa}
\alwaysmath{lambda}
\alwaysmath{Lambda}
\alwaysmath{mu}
\alwaysmath{nu}
\alwaysmath{xi}
\alwaysmath{pi}
\alwaysmath{rho}
\alwaysmath{sigma}
\alwaysmath{Sigma}
\alwaysmath{tau}
\alwaysmath{upsilon}
\alwaysmath{Upsilon}
\alwaysmath{phi}
\alwaysmath{Pi}
\alwaysmath{Phi}
\mdef\ph{\varphi}
\alwaysmath{chi}
\alwaysmath{psi}
\alwaysmath{Psi}
\alwaysmath{omega}
\alwaysmath{Omega}
\let\al\alpha
\let\be\beta
\let\gm\gamma
\let\Gm\Gamma
\let\de\delta
\let\De\Delta
\let\si\sigma
\let\Si\Sigma
\let\om\omega
\let\ka\kappa
\let\la\lambda
\let\La\Lambda
\let\ze\zeta
\let\th\theta
\let\Th\Theta
\let\vth\vartheta

\makeatother

% Tikz styles
\tikzstyle{tickarrow}=[->,postaction={decorate},decoration={markings,mark=at position .5 with {\draw[-] (0,-0.1) -- (0,0.1);}},line width=0.50]

% Local Variables:
% mode: latex
% TeX-master: ""
% End:

\begin{document}

{\small
\begin{equation*}
\begin{tikzpicture}[xscale=2.25, yscale=1.5]
%%%% Row A
\node (A1) at (-.5,7){$\substack{\tens (\tens \times \transid)\\ (\transid \times \transid \times \transid)}$};
\node (A2) at (0,9){$\substack{\tens (\tens \times \transid)\\ ([\tens(\transid \times I] \times \transid \times \transid )}$};
\node (A3) at (1,9){$\substack{\tens (\tens \times \transid) \\  (\tens \times \transid \times \transid) \\ (\transid \times I \times \transid \times \transid)}$};
\node (A5) at (2.5,10){$\substack{\tens (\tens \times \transid) \\ ( \transid \times \tens \times \transid) \\ (\transid \times I \times \transid \times \transid)}$};
\node (A6) at (3.5,10){$\substack{\tens (\tens \times \transid) \\  (\transid \times [\tens (I \times \transid)] \times \transid)}$};
\node (A7) at (5,9.5){$\substack{\tens (\tens \times \transid) \\ (\transid \times \transid \times \transid)}$};
\node (A75) at (6,8.5){$\substack{\tens (\transid \times \tens) \\ (\transid \times \transid \times \transid)}$};
\node (A8) at (6,5){$\substack{\tens (\transid \times \tens)}$};
%%%
\draw[doubleloose] (A1) to node[above, xshift=-20]{$\substack{\looseid \looseid \\ (r^{-1} \times \looseid \times \looseid) }$} (A2);
\draw[doubletighteq] (A2) to  (A3);
\draw[doubleloose] (A3) to node[above, yshift=5pt]{$\substack{\looseid (\alpha \times \looseid) \looseid }$} (A5);
\draw[doubletighteq] (A5) to (A6);
\draw[doubleloose] (A6) to node[above, xshift=8pt]{$\substack{\looseid \looseid \\ (\looseid \times l \times \looseid) }$} (A7);
\draw[doubleloose] (A7) to node[above]{$\substack{\alpha}$} (A75);
\draw[doubletighteq] (A75) to (A8);
%%%% Row B
\node (B6) at (3,8){$\substack{\tens (\transid \times \tens) \\ ( \transid \times \tens \times \transid) \\ (\transid \times I \times \transid \times \transid)}$};
\node (B7) at (4,8){$\substack{\tens (\transid \times \tens) \\ ( \transid \times [\tens (I \times \transid)] \times \transid)}$};
%%%
\draw[doubleloose] (A5) to node[right]{$\substack{\alpha \looseid \looseid  }$} (B6);
\draw[doubletighteq] (B6) to  (B7); 
\draw[doubleloose] (B7) to node[below, xshift=10pt, yshift=-5pt]{$\substack{\looseid \looseid (\looseid \times l \times \looseid) }$} (A75); 
%%%% Row C
\node (C7) at (3,6){$\substack{\tens (\transid \times \tens) \\ ( \transid \times \transid \times \tens) \\ (\transid \times I \times \transid  \times \transid)}$};
\node (C75) at (4,5){$\substack{\tens (\transid \times [\tens (I \times \transid)]) \\ (\transid \times \transid \times \tens) }$};
\node (C8) at (5,5){$\substack{\tens (\transid \times \transid) \\ (\transid \times \transid \times \tens) }$};
%%%
\draw[doubleloose] (B6) to node[right]{$\substack{\looseid (\looseid \times \alpha) \\ \looseid}$} (C7); 
\draw[doubletighteq] (C7) to (C75);
\draw[doubleloose] (C75) to node[above]{$\substack{ \looseid (\looseid \times l ) \\ \looseid}$} (C8); 
\draw[doubletighteq] (C8) to  (A8); 
%%%% Row D
\node (D4) at (0,4){$\substack{\tens ([\tens  (\transid \times I)] \times \tens) \\ ( \transid \times \transid \times \transid) }$};
\node (D5) at (1,5){$\substack{\tens (\transid \times \tens) \\ ( \tens \times \transid \times \transid) \\ (\transid \times I \times \transid \times \transid) }$};
\node (D6) at (2,5){$\substack{\tens (\tens \times \transid) \\ ( \transid \times \transid \times \tens) \\ (\transid \times I \times \transid \times \transid)}$};
%%%
\draw[doubleloose] (A3) to node[left]{$\substack{\alpha \looseid \looseid }$} (D5);
\draw[doubletighteq] (D4) to (D5);
\draw[doubletighteq] (D5) to (D6);
\draw[doubleloose] (D6) to node[above,xshift=-10pt]{$\substack{\alpha \looseid    \looseid}$} (C7); 
%%%% Row E
\node (E1) at (-.5,3){$\substack{\tens (\transid \times \tens)\\ (\transid \times \transid \times \transid)}$};
\node (E3) at (0,2){$\substack{\tens (\transid \times \tens)}$};
%%%
\draw[doubleloose] (A1) to node[left]{$\substack{\alpha}$} (E1);
\draw[doubleloose] (E1) to node[right]{$\substack{ \looseid ( r^{-1} \times \looseid ) \looseid}$}  (D4);
\draw[doubletighteq] (E1) to (E3);
\draw[doubletighteq] (E3) to[out= 0, in=270] (A8); 
%%%% 3-cells
\node at (0.5,6.5) {$\substack{\DDownarrow \iso } $};
\node at (5.5,7.5) {$\substack{\DDownarrow \iso } $};
\node at (4.5,6.7) {$\substack{\DDownarrow \tightid \lambda}$};
\node at (4,5.8) {$\substack{\DDownarrow \iso } $};
\node at (2.5,8.5){$\substack{\DDownarrow \iso } $};
\node at (2,7){$\substack{\DDownarrow \pi \tightid}$};
\node at (1.5,6){$\substack{\DDownarrow \iso } $};
\node at (4,8.5){$\substack{\DDownarrow \iso } $};
\node at (2.5,4.5){$\substack{\DDownarrow  \iso}$};
\node at (3.5,3){$\substack{\DDownarrow  \looseid (\looseid \times \mu)}$};
%%%%%
\draw[doubleloose] (A3) to[out= -35, in=115] node[right]{$\substack{S(\pi) \looseid}$} (C7); 
\draw[doubleloose] (A3) to[out= -75, in=180] node[left]{$\substack{T(\pi) \\ \looseid}$} (C7);
%%%%%
\draw[doubleloose] (B6) to[out= -25, in=115] node[above, xshift=10pt]{$\substack{\looseid S(\lambda)}$} (A8); 
\draw[doubleloose] (B6) to[out= -55, in=160] node[below, xshift=-10pt]{$\substack{\looseid T(\lambda) }$} (A8);
%%%%%
\draw[doubleloose] (E1) to[out= 25, in=225] node[above]{$\substack{\looseid (\looseid \times S(\mu))}$} (A8); 
\end{tikzpicture}
\end{equation*}
\begin{equation}\label{eq:monobjeq2}
=
\end{equation}
\begin{equation*}
\begin{tikzpicture}[xscale=3, yscale=1]
%%%% Row A
\node (A1) at (-.5,7){$\substack{\tens (\tens \times \transid)\\ (\transid \times \transid \times \transid)}$};
\node (A2) at (0,9){$\substack{\tens (\tens \times \transid)\\ ([\tens(\transid \times I] \times \transid \times \transid )}$};
\node (A3) at (1,10){$\substack{\tens (\tens \times \transid) \\  (\tens \times \transid \times \transid) \\ (\transid \times I \times \transid \times \transid)}$};
\node (A5) at (2.5,10){$\substack{\tens (\tens \times \transid) \\ ( \transid \times \tens \times \transid) \\ (\transid \times I \times \transid \times \transid)}$};
\node (A6) at (3.5,9){$\substack{\tens (\tens \times \transid) \\  (\transid \times [\tens (I \times \transid)] \times \transid)}$};
\node (A7) at (4,7){$\substack{\tens (\tens \times \transid) \\ (\transid \times \transid \times \transid)}$};
\node (A75) at (4,6){$\substack{\tens (\transid \times \tens) \\ (\transid \times \transid \times \transid)}$};
\node (A8) at (3.5,5){$\substack{\tens (\transid \times \tens)}$};
%%%
\draw[doubleloose] (A1) to node[above, xshift=-20]{$\substack{\looseid \looseid \\ (r^{-1} \times \looseid \times \looseid) }$} (A2);
\draw[doubletighteq] (A2) to  (A3);
\draw[doubleloose] (A3) to node[above, yshift=5pt]{$\substack{\looseid (\alpha \times \looseid) \looseid }$} (A5);
\draw[doubletighteq] (A5) to (A6);
\draw[doubleloose] (A6) to node[right, xshift=8pt]{$\substack{\looseid \looseid \\ (\looseid \times l \times \looseid) }$} (A7);
\draw[doubleloose] (A7) to node[right]{$\substack{\alpha}$} (A75);
\draw[doubletighteq] (A75) to (A8);
%%%% Row E
\node (E1) at (-.5,6){$\substack{\tens (\transid \times \tens)\\ (\transid \times \transid \times \transid)}$};
\node (E3) at (0,5){$\substack{\tens (\transid \times \tens)}$};
%%%
\draw[doubleloose] (A1) to node[left]{$\substack{\alpha}$} (E1);
\draw[doubletighteq] (E1) to (E3);
\draw[doubletighteq] (E3) to (A8); 
%%%% 3-cells
\node at (1.75,9){$\substack{\DDownarrow  \iso}$};
\node at (1.75,7.5){$\substack{\DDownarrow  \tightid (\mu \times \tightid)}$};
\node at (1.75,6){$\substack{\DDownarrow  \iso}$};
%%%%%
\draw[doubleloose] (A1) to node[above]{$\substack{\looseid (\looseid \times \looseid)}$} (A7); 
\draw[doubleloose] (A1) to[out= 55, in=125] node[above]{$\substack{\looseid (S(\mu) \times \looseid)}$} (A7); 
\end{tikzpicture}
\end{equation*}}
\end{document}  \newpage
%
\documentclass[12pt]{ociamthesis}
\usepackage{tikz}
\usepackage{amsmath}
\usepackage{rotating}

\usepackage{amssymb,amsmath,stmaryrd,txfonts,mathrsfs,amsthm}
\usepackage[all,2cell]{xy}
\usepackage[neveradjust]{paralist}
\usepackage{hyperref}
\usepackage{mathtools}
\usepackage{tikz}
\usetikzlibrary{trees}
\usetikzlibrary{topaths}
\usetikzlibrary{decorations}
\usetikzlibrary{decorations.pathreplacing}
\usetikzlibrary{decorations.pathmorphing}
\usetikzlibrary{decorations.markings}
\usetikzlibrary{matrix,backgrounds,folding}
\usetikzlibrary{chains,scopes,positioning,fit}
\usetikzlibrary{arrows,shadows}
\usetikzlibrary{calc} 
\usetikzlibrary{chains}
\usetikzlibrary{shapes,shapes.geometric,shapes.misc}
\usepackage{smbicat}


\makeatletter
\let\ea\expandafter

%% Defining commands that are always in math mode.
\def\mdef#1#2{\ea\ea\ea\gdef\ea\ea\noexpand#1\ea{\ea\ensuremath\ea{#2}}}
\def\alwaysmath#1{\ea\ea\ea\global\ea\ea\ea\let\ea\ea\csname your@#1\endcsname\csname #1\endcsname
  \ea\def\csname #1\endcsname{\ensuremath{\csname your@#1\endcsname}}}

% Script letters
\newcommand{\sA}{\ensuremath{\mathscr{A}}}
\newcommand{\sB}{\ensuremath{\mathscr{B}}}
\newcommand{\sC}{\ensuremath{\mathscr{C}}}
\newcommand{\sD}{\ensuremath{\mathscr{D}}}
\newcommand{\sE}{\ensuremath{\mathscr{E}}}
\newcommand{\sF}{\ensuremath{\mathscr{F}}}
\newcommand{\sG}{\ensuremath{\mathscr{G}}}
\newcommand{\sH}{\ensuremath{\mathscr{H}}}
\newcommand{\sI}{\ensuremath{\mathscr{I}}}
\newcommand{\sJ}{\ensuremath{\mathscr{J}}}
\newcommand{\sK}{\ensuremath{\mathscr{K}}}
\newcommand{\sL}{\ensuremath{\mathscr{L}}}
\newcommand{\sM}{\ensuremath{\mathscr{M}}}
\newcommand{\sN}{\ensuremath{\mathscr{N}}}
\newcommand{\sO}{\ensuremath{\mathscr{O}}}
\newcommand{\sP}{\ensuremath{\mathscr{P}}}
\newcommand{\sQ}{\ensuremath{\mathscr{Q}}}
\newcommand{\sR}{\ensuremath{\mathscr{R}}}
\newcommand{\sS}{\ensuremath{\mathscr{S}}}
\newcommand{\sT}{\ensuremath{\mathscr{T}}}
\newcommand{\sU}{\ensuremath{\mathscr{U}}}
\newcommand{\sV}{\ensuremath{\mathscr{V}}}
\newcommand{\sW}{\ensuremath{\mathscr{W}}}
\newcommand{\sX}{\ensuremath{\mathscr{X}}}
\newcommand{\sY}{\ensuremath{\mathscr{Y}}}
\newcommand{\sZ}{\ensuremath{\mathscr{Z}}}

% Calligraphic letters
\newcommand{\cA}{\ensuremath{\mathcal{A}}}
\newcommand{\cB}{\ensuremath{\mathcal{B}}}
\newcommand{\cC}{\ensuremath{\mathcal{C}}}
\newcommand{\cD}{\ensuremath{\mathcal{D}}}
\newcommand{\cE}{\ensuremath{\mathcal{E}}}
\newcommand{\cF}{\ensuremath{\mathcal{F}}}
\newcommand{\cG}{\ensuremath{\mathcal{G}}}
\newcommand{\cH}{\ensuremath{\mathcal{H}}}
\newcommand{\cI}{\ensuremath{\mathcal{I}}}
\newcommand{\cJ}{\ensuremath{\mathcal{J}}}
\newcommand{\cK}{\ensuremath{\mathcal{K}}}
\newcommand{\cL}{\ensuremath{\mathcal{L}}}
\newcommand{\cM}{\ensuremath{\mathcal{M}}}
\newcommand{\cN}{\ensuremath{\mathcal{N}}}
\newcommand{\cO}{\ensuremath{\mathcal{O}}}
\newcommand{\cP}{\ensuremath{\mathcal{P}}}
\newcommand{\cQ}{\ensuremath{\mathcal{Q}}}
\newcommand{\cR}{\ensuremath{\mathcal{R}}}
\newcommand{\cS}{\ensuremath{\mathcal{S}}}
\newcommand{\cT}{\ensuremath{\mathcal{T}}}
\newcommand{\cU}{\ensuremath{\mathcal{U}}}
\newcommand{\cV}{\ensuremath{\mathcal{V}}}
\newcommand{\cW}{\ensuremath{\mathcal{W}}}
\newcommand{\cX}{\ensuremath{\mathcal{X}}}
\newcommand{\cY}{\ensuremath{\mathcal{Y}}}
\newcommand{\cZ}{\ensuremath{\mathcal{Z}}}

% blackboard bold letters
\newcommand{\lA}{\ensuremath{\mathbb{A}}}
\newcommand{\lB}{\ensuremath{\mathbb{B}}}
\newcommand{\lC}{\ensuremath{\mathbb{C}}}
\newcommand{\lD}{\ensuremath{\mathbb{D}}}
\newcommand{\lE}{\ensuremath{\mathbb{E}}}
\newcommand{\lF}{\ensuremath{\mathbb{F}}}
\newcommand{\lG}{\ensuremath{\mathbb{G}}}
\newcommand{\lH}{\ensuremath{\mathbb{H}}}
\newcommand{\lI}{\ensuremath{\mathbb{I}}}
\newcommand{\lJ}{\ensuremath{\mathbb{J}}}
\newcommand{\lK}{\ensuremath{\mathbb{K}}}
\newcommand{\lL}{\ensuremath{\mathbb{L}}}
\newcommand{\lM}{\ensuremath{\mathbb{M}}}
\newcommand{\lN}{\ensuremath{\mathbb{N}}}
\newcommand{\lO}{\ensuremath{\mathbb{O}}}
\newcommand{\lP}{\ensuremath{\mathbb{P}}}
\newcommand{\lQ}{\ensuremath{\mathbb{Q}}}
\newcommand{\lR}{\ensuremath{\mathbb{R}}}
\newcommand{\lS}{\ensuremath{\mathbb{S}}}
\newcommand{\lT}{\ensuremath{\mathbb{T}}}
\newcommand{\lU}{\ensuremath{\mathbb{U}}}
\newcommand{\lV}{\ensuremath{\mathbb{V}}}
\newcommand{\lW}{\ensuremath{\mathbb{W}}}
\newcommand{\lX}{\ensuremath{\mathbb{X}}}
\newcommand{\lY}{\ensuremath{\mathbb{Y}}}
\newcommand{\lZ}{\ensuremath{\mathbb{Z}}}

% bold letters
\newcommand{\bA}{\ensuremath{\mathbf{A}}}
\newcommand{\bB}{\ensuremath{\mathbf{B}}}
\newcommand{\bC}{\ensuremath{\mathbf{C}}}
\newcommand{\bD}{\ensuremath{\mathbf{D}}}
\newcommand{\bE}{\ensuremath{\mathbf{E}}}
\newcommand{\bF}{\ensuremath{\mathbf{F}}}
\newcommand{\bG}{\ensuremath{\mathbf{G}}}
\newcommand{\bH}{\ensuremath{\mathbf{H}}}
\newcommand{\bI}{\ensuremath{\mathbf{I}}}
\newcommand{\bJ}{\ensuremath{\mathbf{J}}}
\newcommand{\bK}{\ensuremath{\mathbf{K}}}
\newcommand{\bL}{\ensuremath{\mathbf{L}}}
\newcommand{\bM}{\ensuremath{\mathbf{M}}}
\newcommand{\bN}{\ensuremath{\mathbf{N}}}
\newcommand{\bO}{\ensuremath{\mathbf{O}}}
\newcommand{\bP}{\ensuremath{\mathbf{P}}}
\newcommand{\bQ}{\ensuremath{\mathbf{Q}}}
\newcommand{\bR}{\ensuremath{\mathbf{R}}}
\newcommand{\bS}{\ensuremath{\mathbf{S}}}
\newcommand{\bT}{\ensuremath{\mathbf{T}}}
\newcommand{\bU}{\ensuremath{\mathbf{U}}}
\newcommand{\bV}{\ensuremath{\mathbf{V}}}
\newcommand{\bW}{\ensuremath{\mathbf{W}}}
\newcommand{\bX}{\ensuremath{\mathbf{X}}}
\newcommand{\bY}{\ensuremath{\mathbf{Y}}}
\newcommand{\bZ}{\ensuremath{\mathbf{Z}}}

% fraktur letters
\newcommand{\fa}{\ensuremath{\mathfrak{a}}}
\newcommand{\fb}{\ensuremath{\mathfrak{b}}}
\newcommand{\fc}{\ensuremath{\mathfrak{c}}}
\newcommand{\fd}{\ensuremath{\mathfrak{d}}}
\newcommand{\fe}{\ensuremath{\mathfrak{e}}}
\newcommand{\ff}{\ensuremath{\mathfrak{f}}}
\newcommand{\fg}{\ensuremath{\mathfrak{g}}}
\newcommand{\fh}{\ensuremath{\mathfrak{h}}}
\newcommand{\fj}{\ensuremath{\mathfrak{j}}}
\newcommand{\fk}{\ensuremath{\mathfrak{k}}}
\newcommand{\fl}{\ensuremath{\mathfrak{l}}}
\newcommand{\fm}{\ensuremath{\mathfrak{m}}}
\newcommand{\fn}{\ensuremath{\mathfrak{n}}}
\newcommand{\fo}{\ensuremath{\mathfrak{o}}}
\newcommand{\fp}{\ensuremath{\mathfrak{p}}}
\newcommand{\fq}{\ensuremath{\mathfrak{q}}}
\newcommand{\fr}{\ensuremath{\mathfrak{r}}}
\newcommand{\fs}{\ensuremath{\mathfrak{s}}}
\newcommand{\ft}{\ensuremath{\mathfrak{t}}}
\newcommand{\fu}{\ensuremath{\mathfrak{u}}}
\newcommand{\fv}{\ensuremath{\mathfrak{v}}}
\newcommand{\fw}{\ensuremath{\mathfrak{w}}}
\newcommand{\fx}{\ensuremath{\mathfrak{x}}}
\newcommand{\fy}{\ensuremath{\mathfrak{y}}}
\newcommand{\fz}{\ensuremath{\mathfrak{z}}}

% fraktur letters
\newcommand{\fA}{\ensuremath{\mathfrak{A}}}
\newcommand{\fB}{\ensuremath{\mathfrak{B}}}
\newcommand{\fC}{\ensuremath{\mathfrak{C}}}

\mdef\fahat{\hat{\fa}}

% Underline letters
\newcommand{\uA}{\ensuremath{\underline{A}}}
\newcommand{\uB}{\ensuremath{\underline{B}}}
\newcommand{\uC}{\ensuremath{\underline{C}}}
\newcommand{\uD}{\ensuremath{\underline{D}}}
\newcommand{\uE}{\ensuremath{\underline{E}}}
\newcommand{\uF}{\ensuremath{\underline{F}}}
\newcommand{\uG}{\ensuremath{\underline{G}}}
\newcommand{\uH}{\ensuremath{\underline{H}}}
\newcommand{\uI}{\ensuremath{\underline{I}}}
\newcommand{\uJ}{\ensuremath{\underline{J}}}
\newcommand{\uK}{\ensuremath{\underline{K}}}
\newcommand{\uL}{\ensuremath{\underline{L}}}
\newcommand{\uM}{\ensuremath{\underline{M}}}
\newcommand{\uN}{\ensuremath{\underline{N}}}
\newcommand{\uO}{\ensuremath{\underline{O}}}
\newcommand{\uP}{\ensuremath{\underline{P}}}
\newcommand{\uQ}{\ensuremath{\underline{Q}}}
\newcommand{\uR}{\ensuremath{\underline{R}}}
\newcommand{\uS}{\ensuremath{\underline{S}}}
\newcommand{\uT}{\ensuremath{\underline{T}}}
\newcommand{\uU}{\ensuremath{\underline{U}}}
\newcommand{\uV}{\ensuremath{\underline{V}}}
\newcommand{\uW}{\ensuremath{\underline{W}}}
\newcommand{\uX}{\ensuremath{\underline{X}}}
\newcommand{\uY}{\ensuremath{\underline{Y}}}
\newcommand{\uZ}{\ensuremath{\underline{Z}}}

% bars
\newcommand{\Abar}{\ensuremath{\overline{A}}}
\newcommand{\Bbar}{\ensuremath{\overline{B}}}
\newcommand{\Cbar}{\ensuremath{\overline{C}}}
\newcommand{\Dbar}{\ensuremath{\overline{D}}}
\newcommand{\Ebar}{\ensuremath{\overline{E}}}
\newcommand{\Fbar}{\ensuremath{\overline{F}}}
\newcommand{\Gbar}{\ensuremath{\overline{G}}}
\newcommand{\Hbar}{\ensuremath{\overline{H}}}
\newcommand{\Ibar}{\ensuremath{\overline{I}}}
\newcommand{\Jbar}{\ensuremath{\overline{J}}}
\newcommand{\Kbar}{\ensuremath{\overline{K}}}
\newcommand{\Lbar}{\ensuremath{\overline{L}}}
\newcommand{\Mbar}{\ensuremath{\overline{M}}}
\newcommand{\Nbar}{\ensuremath{\overline{N}}}
\newcommand{\Obar}{\ensuremath{\overline{O}}}
\newcommand{\Pbar}{\ensuremath{\overline{P}}}
\newcommand{\Qbar}{\ensuremath{\overline{Q}}}
\newcommand{\Rbar}{\ensuremath{\overline{R}}}
\newcommand{\Sbar}{\ensuremath{\overline{S}}}
\newcommand{\Tbar}{\ensuremath{\overline{T}}}
\newcommand{\Ubar}{\ensuremath{\overline{U}}}
\newcommand{\Vbar}{\ensuremath{\overline{V}}}
\newcommand{\Wbar}{\ensuremath{\overline{W}}}
\newcommand{\Xbar}{\ensuremath{\overline{X}}}
\newcommand{\Ybar}{\ensuremath{\overline{Y}}}
\newcommand{\Zbar}{\ensuremath{\overline{Z}}}
\newcommand{\abar}{\ensuremath{\overline{a}}}
\newcommand{\bbar}{\ensuremath{\overline{b}}}
\newcommand{\cbar}{\ensuremath{\overline{c}}}
\newcommand{\dbar}{\ensuremath{\overline{d}}}
\newcommand{\ebar}{\ensuremath{\overline{e}}}
\newcommand{\fbar}{\ensuremath{\overline{f}}}
\newcommand{\gbar}{\ensuremath{\overline{g}}}
%\newcommand{\hbar}{\ensuremath{\overline{h}}} % whoops, \hbar means something else!
\newcommand{\ibar}{\ensuremath{\overline{\imath}}}
\newcommand{\jbar}{\ensuremath{\overline{\jmath}}}
\newcommand{\kbar}{\ensuremath{\overline{k}}}
\newcommand{\lbar}{\ensuremath{\overline{l}}}
\newcommand{\mbar}{\ensuremath{\overline{m}}}
\newcommand{\nbar}{\ensuremath{\overline{n}}}
%\newcommand{\obar}{\ensuremath{\overline{o}}}
\newcommand{\pbar}{\ensuremath{\overline{p}}}
\newcommand{\qbar}{\ensuremath{\overline{q}}}
\newcommand{\rbar}{\ensuremath{\overline{r}}}
\newcommand{\sbar}{\ensuremath{\overline{s}}}
\newcommand{\tbar}{\ensuremath{\overline{t}}}
\newcommand{\ubar}{\ensuremath{\overline{u}}}
\newcommand{\vbar}{\ensuremath{\overline{v}}}
\newcommand{\wbar}{\ensuremath{\overline{w}}}
\newcommand{\xbar}{\ensuremath{\overline{x}}}
\newcommand{\ybar}{\ensuremath{\overline{y}}}
\newcommand{\zbar}{\ensuremath{\overline{z}}}

% tildes
\newcommand{\Atil}{\ensuremath{\widetilde{A}}}
\newcommand{\Btil}{\ensuremath{\widetilde{B}}}
\newcommand{\Ctil}{\ensuremath{\widetilde{C}}}
\newcommand{\Dtil}{\ensuremath{\widetilde{D}}}
\newcommand{\Etil}{\ensuremath{\widetilde{E}}}
\newcommand{\Ftil}{\ensuremath{\widetilde{F}}}
\newcommand{\Gtil}{\ensuremath{\widetilde{G}}}
\newcommand{\Htil}{\ensuremath{\widetilde{H}}}
\newcommand{\Itil}{\ensuremath{\widetilde{I}}}
\newcommand{\Jtil}{\ensuremath{\widetilde{J}}}
\newcommand{\Ktil}{\ensuremath{\widetilde{K}}}
\newcommand{\Ltil}{\ensuremath{\widetilde{L}}}
\newcommand{\Mtil}{\ensuremath{\widetilde{M}}}
\newcommand{\Ntil}{\ensuremath{\widetilde{N}}}
\newcommand{\Otil}{\ensuremath{\widetilde{O}}}
\newcommand{\Ptil}{\ensuremath{\widetilde{P}}}
\newcommand{\Qtil}{\ensuremath{\widetilde{Q}}}
\newcommand{\Rtil}{\ensuremath{\widetilde{R}}}
\newcommand{\Stil}{\ensuremath{\widetilde{S}}}
\newcommand{\Ttil}{\ensuremath{\widetilde{T}}}
\newcommand{\Util}{\ensuremath{\widetilde{U}}}
\newcommand{\Vtil}{\ensuremath{\widetilde{V}}}
\newcommand{\Wtil}{\ensuremath{\widetilde{W}}}
\newcommand{\Xtil}{\ensuremath{\widetilde{X}}}
\newcommand{\Ytil}{\ensuremath{\widetilde{Y}}}
\newcommand{\Ztil}{\ensuremath{\widetilde{Z}}}
\newcommand{\atil}{\ensuremath{\widetilde{a}}}
\newcommand{\btil}{\ensuremath{\widetilde{b}}}
\newcommand{\ctil}{\ensuremath{\widetilde{c}}}
\newcommand{\dtil}{\ensuremath{\widetilde{d}}}
\newcommand{\etil}{\ensuremath{\widetilde{e}}}
\newcommand{\ftil}{\ensuremath{\widetilde{f}}}
\newcommand{\gtil}{\ensuremath{\widetilde{g}}}
\newcommand{\htil}{\ensuremath{\widetilde{h}}}
\newcommand{\itil}{\ensuremath{\widetilde{\imath}}}
\newcommand{\jtil}{\ensuremath{\widetilde{\jmath}}}
\newcommand{\ktil}{\ensuremath{\widetilde{k}}}
\newcommand{\ltil}{\ensuremath{\widetilde{l}}}
\newcommand{\mtil}{\ensuremath{\widetilde{m}}}
\newcommand{\ntil}{\ensuremath{\widetilde{n}}}
\newcommand{\otil}{\ensuremath{\widetilde{o}}}
\newcommand{\ptil}{\ensuremath{\widetilde{p}}}
\newcommand{\qtil}{\ensuremath{\widetilde{q}}}
\newcommand{\rtil}{\ensuremath{\widetilde{r}}}
\newcommand{\stil}{\ensuremath{\widetilde{s}}}
\newcommand{\ttil}{\ensuremath{\widetilde{t}}}
\newcommand{\util}{\ensuremath{\widetilde{u}}}
\newcommand{\vtil}{\ensuremath{\widetilde{v}}}
\newcommand{\wtil}{\ensuremath{\widetilde{w}}}
\newcommand{\xtil}{\ensuremath{\widetilde{x}}}
\newcommand{\ytil}{\ensuremath{\widetilde{y}}}
\newcommand{\ztil}{\ensuremath{\widetilde{z}}}

% Hats
\newcommand{\Ahat}{\ensuremath{\widehat{A}}}
\newcommand{\Bhat}{\ensuremath{\widehat{B}}}
\newcommand{\Chat}{\ensuremath{\widehat{C}}}
\newcommand{\Dhat}{\ensuremath{\widehat{D}}}
\newcommand{\Ehat}{\ensuremath{\widehat{E}}}
\newcommand{\Fhat}{\ensuremath{\widehat{F}}}
\newcommand{\Ghat}{\ensuremath{\widehat{G}}}
\newcommand{\Hhat}{\ensuremath{\widehat{H}}}
\newcommand{\Ihat}{\ensuremath{\widehat{I}}}
\newcommand{\Jhat}{\ensuremath{\widehat{J}}}
\newcommand{\Khat}{\ensuremath{\widehat{K}}}
\newcommand{\Lhat}{\ensuremath{\widehat{L}}}
\newcommand{\Mhat}{\ensuremath{\widehat{M}}}
\newcommand{\Nhat}{\ensuremath{\widehat{N}}}
\newcommand{\Ohat}{\ensuremath{\widehat{O}}}
\newcommand{\Phat}{\ensuremath{\widehat{P}}}
\newcommand{\Qhat}{\ensuremath{\widehat{Q}}}
\newcommand{\Rhat}{\ensuremath{\widehat{R}}}
\newcommand{\Shat}{\ensuremath{\widehat{S}}}
\newcommand{\That}{\ensuremath{\widehat{T}}}
\newcommand{\Uhat}{\ensuremath{\widehat{U}}}
\newcommand{\Vhat}{\ensuremath{\widehat{V}}}
\newcommand{\What}{\ensuremath{\widehat{W}}}
\newcommand{\Xhat}{\ensuremath{\widehat{X}}}
\newcommand{\Yhat}{\ensuremath{\widehat{Y}}}
\newcommand{\Zhat}{\ensuremath{\widehat{Z}}}
\newcommand{\ahat}{\ensuremath{\hat{a}}}
\newcommand{\bhat}{\ensuremath{\hat{b}}}
\newcommand{\chat}{\ensuremath{\hat{c}}}
\newcommand{\dhat}{\ensuremath{\hat{d}}}
\newcommand{\ehat}{\ensuremath{\hat{e}}}
\newcommand{\fhat}{\ensuremath{\hat{f}}}
\newcommand{\ghat}{\ensuremath{\hat{g}}}
\newcommand{\hhat}{\ensuremath{\hat{h}}}
\newcommand{\ihat}{\ensuremath{\hat{\imath}}}
\newcommand{\jhat}{\ensuremath{\hat{\jmath}}}
\newcommand{\khat}{\ensuremath{\hat{k}}}
\newcommand{\lhat}{\ensuremath{\hat{l}}}
\newcommand{\mhat}{\ensuremath{\hat{m}}}
\newcommand{\nhat}{\ensuremath{\hat{n}}}
\newcommand{\ohat}{\ensuremath{\hat{o}}}
\newcommand{\phat}{\ensuremath{\hat{p}}}
\newcommand{\qhat}{\ensuremath{\hat{q}}}
\newcommand{\rhat}{\ensuremath{\hat{r}}}
\newcommand{\shat}{\ensuremath{\hat{s}}}
\newcommand{\that}{\ensuremath{\hat{t}}}
\newcommand{\uhat}{\ensuremath{\hat{u}}}
\newcommand{\vhat}{\ensuremath{\hat{v}}}
\newcommand{\what}{\ensuremath{\hat{w}}}
\newcommand{\xhat}{\ensuremath{\hat{x}}}
\newcommand{\yhat}{\ensuremath{\hat{y}}}
\newcommand{\zhat}{\ensuremath{\hat{z}}}

%% FONTS AND DECORATION FOR GREEK LETTERS

%% the package `mathbbol' gives us blackboard bold greek and numbers,
%% but it does it by redefining \mathbb to use a different font, so that
%% all the other \mathbb letters look different too.  Here we import the
%% font with bb greek and numbers, but assign it a different name,
%% \mathbbb, so as not to replace the usual one.
\DeclareSymbolFont{bbold}{U}{bbold}{m}{n}
\DeclareSymbolFontAlphabet{\mathbbb}{bbold}
\newcommand{\bbDelta}{\ensuremath{\mathbbb{\Delta}}}
\newcommand{\bbone}{\ensuremath{\mathbbb{1}}}
\newcommand{\bbtwo}{\ensuremath{\mathbbb{2}}}
\newcommand{\bbthree}{\ensuremath{\mathbbb{3}}}

% greek with bars
\newcommand{\albar}{\ensuremath{\overline{\alpha}}}
\newcommand{\bebar}{\ensuremath{\overline{\beta}}}
\newcommand{\gmbar}{\ensuremath{\overline{\gamma}}}
\newcommand{\debar}{\ensuremath{\overline{\delta}}}
\newcommand{\phibar}{\ensuremath{\overline{\varphi}}}
\newcommand{\psibar}{\ensuremath{\overline{\psi}}}
\newcommand{\xibar}{\ensuremath{\overline{\xi}}}
\newcommand{\ombar}{\ensuremath{\overline{\omega}}}

% greek with hats
\newcommand{\alhat}{\ensuremath{\hat{\alpha}}}
\newcommand{\behat}{\ensuremath{\hat{\beta}}}
\newcommand{\gmhat}{\ensuremath{\hat{\gamma}}}
\newcommand{\dehat}{\ensuremath{\hat{\delta}}}

% greek with checks
\newcommand{\alchk}{\ensuremath{\check{\alpha}}}
\newcommand{\bechk}{\ensuremath{\check{\beta}}}
\newcommand{\gmchk}{\ensuremath{\check{\gamma}}}
\newcommand{\dechk}{\ensuremath{\check{\delta}}}

% greek with tildes
\newcommand{\altil}{\ensuremath{\widetilde{\alpha}}}
\newcommand{\betil}{\ensuremath{\widetilde{\beta}}}
\newcommand{\gmtil}{\ensuremath{\widetilde{\gamma}}}
\newcommand{\phitil}{\ensuremath{\widetilde{\varphi}}}
\newcommand{\psitil}{\ensuremath{\widetilde{\psi}}}
\newcommand{\xitil}{\ensuremath{\widetilde{\xi}}}
\newcommand{\omtil}{\ensuremath{\widetilde{\omega}}}

% MISCELLANEOUS SYMBOLS
\mdef\del{\partial}
\mdef\delbar{\overline{\partial}}
\let\sm\wedge
\newcommand{\dd}[1]{\ensuremath{\frac{\partial}{\partial {#1}}}}
\newcommand{\inv}{^{-1}}
\newcommand{\dual}{^{\vee}}
\mdef\hf{\textstyle\frac{1}{2}}
\mdef\thrd{\textstyle\frac{1}{3}}
\mdef\qtr{\textstyle\frac{1}{4}}
\let\meet\wedge
\let\join\vee
\let\dn\downarrow
\newcommand{\op}{^{\mathit{op}}}
\newcommand{\co}{^{\mathit{co}}}
\newcommand{\coop}{^{\mathit{coop}}}
\let\adj\dashv
\SelectTips{cm}{}
\newdir{ >}{{}*!/-10pt/@{>}}    % extra spacing for tail arrows in XYpic
\newcommand{\pushoutcorner}[1][dr]{\save*!/#1+1.2pc/#1:(1,-1)@^{|-}\restore}
\newcommand{\pullbackcorner}[1][dr]{\save*!/#1-1.2pc/#1:(-1,1)@^{|-}\restore}
\let\iso\cong
\let\eqv\simeq
\let\cng\equiv
\mdef\Id{\mathrm{Id}}
\mdef\id{\mathrm{id}}
\alwaysmath{ell}
\alwaysmath{infty}
\alwaysmath{odot}
\def\frc#1/#2.{\frac{#1}{#2}}   % \frc x^2+1 / x^2-1 .
\mdef\ten{\mathrel{\otimes}}
\mdef\bigten{\bigotimes}
\mdef\sqten{\mathrel{\boxtimes}}
\def\pow(#1,#2){\mathop{\pitchfork}(#1,#2)} % powers and
\def\cpw{\mathop{\odot}}                    % copowers
\newcommand{\mathid}{\mbox{id}}
\newcommand{\cat}[1]{\ensuremath{\mathbf{#1}}}


%% OPERATORS
\DeclareMathOperator\lan{Lan}
\DeclareMathOperator\ran{Ran}
\DeclareMathOperator\colim{colim}
\DeclareMathOperator\coeq{coeq}
\DeclareMathOperator\eq{eq}
\DeclareMathOperator\Tot{Tot}
\DeclareMathOperator\cosk{cosk}
\DeclareMathOperator\sk{sk}
\DeclareMathOperator\im{im}
\DeclareMathOperator\Spec{Spec}
\DeclareMathOperator\Ho{Ho}
\DeclareMathOperator\Aut{Aut}
\DeclareMathOperator\End{End}
\DeclareMathOperator\Hom{Hom}
\DeclareMathOperator\Map{Map}

%% TIKZ ARROWS AND HIGHER CELLS
\makeatletter
\def\slashedarrowfill@#1#2#3#4#5{%
  $\m@th\thickmuskip0mu\medmuskip\thickmuskip\thinmuskip\thickmuskip
   \relax#5#1\mkern-7mu%
   \cleaders\hbox{$#5\mkern-2mu#2\mkern-2mu$}\hfill
   \mathclap{#3}\mathclap{#2}%
   \cleaders\hbox{$#5\mkern-2mu#2\mkern-2mu$}\hfill
   \mkern-7mu#4$%
}

\def\Rightslashedarrowfill@{%
  \slashedarrowfill@\Relbar\Relbar\Mapstochar\Rightarrow}
\newcommand\xslashedRightarrow[2][]{%
  \ext@arrow 0055{\Rightslashedarrowfill@}{#1}{#2}}
\def\hTo{\xslashedRightarrow{}}
\def\hToo{\xslashedRightarrow{\quad}}
\let\xhTo\xslashedRightarrow

\pagestyle{empty}

\newcommand{\Rightthreecell}{\RRightarrow}
\newcommand{\Rtwocell}{\Rightarrow}

\tikzstyle{doubletick}=[-implies, double equal sign distance, postaction={decorate},decoration={markings,mark=at position .5 with {\draw[-] (0,-0.1) -- (0,0.1);}}]

\tikzstyle{darrow}=[-implies, double equal sign distance]

\tikzstyle{doubleeq}=[double equal sign distance]


%% ARROWS
% \to already exists
\newcommand{\too}[1][]{\ensuremath{\overset{#1}{\longrightarrow}}}
\newcommand{\ot}{\ensuremath{\leftarrow}}
\newcommand{\oot}[1][]{\ensuremath{\overset{#1}{\longleftarrow}}}
\let\toot\rightleftarrows
\let\otto\leftrightarrows
\let\Impl\Rightarrow
\let\imp\Rightarrow
\let\toto\rightrightarrows
\let\into\hookrightarrow
\let\xinto\xhookrightarrow
\mdef\we{\overset{\sim}{\longrightarrow}}
\mdef\leftwe{\overset{\sim}{\longleftarrow}}
\let\mono\rightarrowtail
\let\leftmono\leftarrowtail
\let\cof\rightarrowtail
\let\leftcof\leftarrowtail
\let\epi\twoheadrightarrow
\let\leftepi\twoheadleftarrow
\let\fib\twoheadrightarrow
\let\leftfib\twoheadleftarrow
\let\cohto\rightsquigarrow
\let\maps\colon
\newcommand{\spam}{\,:\!}       % \maps for left arrows

\newsavebox{\DDownarrowbox}
\savebox{\DDownarrowbox}{\tikz[scale=1.5]{\draw[-implies,double equal
sign distance] (0,.1) -- (0,-.1); \draw (0,.1) -- (0,-.1);}}
\newcommand{\DDownarrow}{\mathrel{\raisebox{-.2em}{\usebox{\DDownarrowbox}}}}

\newsavebox{\RRightarrowbox}
\savebox{\RRightarrowbox}{\tikz[scale=1.5]{\draw[-implies,double equal
sign distance] (-.1,0) -- (.1,0); \draw (-.1,0) -- (.1,0);}}
\newcommand{\RRightarrow}{\mathrel{\raisebox{.2em}{\usebox{\RRightarrowbox}}}}

%\newsavebox{\Rightslashedarrowbox}
%\savebox{\Rightslashedarrowbox}{\tikz[scale=1.5]{\draw[Rightslashedarrow{}] (-.1,0) -- (1,0);}}
%\newcommand{\Rightslashedarrow}{\mathrel{\raisebox{-.2em}%{\usebox{\Rightslashedarrowbox}}}}


%% EXTENSIBLE ARROWS
\let\xto\xrightarrow
\let\xot\xleftarrow
% See Voss' Mathmode.tex for instructions on how to create new
% extensible arrows.
\def\rightarrowtailfill@{\arrowfill@{\Yright\joinrel\relbar}\relbar\rightarrow}
\newcommand\xrightarrowtail[2][]{\ext@arrow 0055{\rightarrowtailfill@}{#1}{#2}}
\let\xmono\xrightarrowtail
\let\xcof\xrightarrowtail
\def\twoheadrightarrowfill@{\arrowfill@{\relbar\joinrel\relbar}\relbar\twoheadrightarrow}
\newcommand\xtwoheadrightarrow[2][]{\ext@arrow 0055{\twoheadrightarrowfill@}{#1}{#2}}
\let\xepi\xtwoheadrightarrow
\let\xfib\xtwoheadrightarrow
% Let's leave the left-going ones until I need them.

%% EXTENSIBLE SLASHED ARROWS
% Making extensible slashed arrows, by modifying the underlying AMS code.
% Arguments are:
% 1 = arrowhead on the left (\relbar or \Relbar if none)
% 2 = fill character (usually \relbar or \Relbar)
% 3 = slash character (such as \mapstochar or \Mapstochar)
% 4 = arrowhead on the left (\relbar or \Relbar if none)
% 5 = display mode (\displaystyle etc)
\def\slashedarrowfill@#1#2#3#4#5{%
  $\m@th\thickmuskip0mu\medmuskip\thickmuskip\thinmuskip\thickmuskip
   \relax#5#1\mkern-7mu%
   \cleaders\hbox{$#5\mkern-2mu#2\mkern-2mu$}\hfill
   \mathclap{#3}\mathclap{#2}%
   \cleaders\hbox{$#5\mkern-2mu#2\mkern-2mu$}\hfill
   \mkern-7mu#4$%
}
% Here's the idea: \<slashed>arrowfill@ should be a box containing
% some stretchable space that is the "middle of the arrow".  This
% space is created as a "leader" using \cleader<thing>\hfill, which
% fills an \hfill of space with copies of <thing>.  Here instead of
% just one \cleader, we use two, with the slash in between (and an
% extra copy of the filler, to avoid extra space around the slash).
\def\rightslashedarrowfill@{%
  \slashedarrowfill@\relbar\relbar\mapstochar\rightarrow}
\newcommand\xslashedrightarrow[2][]{%
  \ext@arrow 0055{\rightslashedarrowfill@}{#1}{#2}}
\mdef\hto{\xslashedrightarrow{}}
\mdef\htoo{\xslashedrightarrow{\quad}}
\let\xhto\xslashedrightarrow

%% To get a slashed arrow in XYpic, do
% \[\xymatrix{A \ar[r]|-@{|} & B}\]

% ISOMORPHISMS
\def\xiso#1{\mathrel{\mathrlap{\smash{\xto[\smash{\raisebox{1.3mm}{$\scriptstyle\sim$}}]{#1}}}\hphantom{\xto{#1}}}}
\def\toiso{\xto{\smash{\raisebox{-.5mm}{$\scriptstyle\sim$}}}}

% SHADOWS
\def\shvar#1#2{{\ensuremath{%
  \hspace{1mm}\makebox[-1mm]{$#1\langle$}\makebox[0mm]{$#1\langle$}\hspace{1mm}%
  {#2}%
  \makebox[1mm]{$#1\rangle$}\makebox[0mm]{$#1\rangle$}%
}}}
\def\sh{\shvar{}}
\def\scriptsh{\shvar{\scriptstyle}}
\def\bigsh{\shvar{\big}}
\def\Bigsh{\shvar{\Big}}
\def\biggsh{\shvar{\bigg}}
\def\Biggsh{\shvar{\Bigg}}

%HIGHER CELLS



% THEOREM-TYPE ENVIRONMENTS, hacked to
%% (a) number all with the same numbers, and
%% (b) have the right names for autoref
\def\defthm#1#2{%
  \newtheorem{#1}{#2}[section]%
  \expandafter\def\csname #1autorefname\endcsname{#2}%
  \expandafter\let\csname c@#1\endcsname\c@thm}
\newtheorem{thm}{Theorem}[section]
\newcommand{\thmautorefname}{Theorem}
\defthm{cor}{Corollary}
\defthm{prop}{Proposition}
\defthm{lem}{Lemma}
\defthm{sch}{Scholium}
\defthm{assume}{Assumption}
\defthm{claim}{Claim}
\defthm{conj}{Conjecture}
\defthm{hyp}{Hypothesis}
\defthm{fact}{Fact}
\theoremstyle{definition}
\defthm{defn}{Definition}
\defthm{notn}{Notation}
\theoremstyle{remark}
\defthm{rmk}{Remark}
\defthm{eg}{Example}
\defthm{egs}{Examples}
\defthm{ex}{Exercise}
\defthm{ceg}{Counterexample}

% How to get QED symbols inside equations at the end of the statements
% of theorems.  AMS LaTeX knows how to do this inside equations at the
% end of *proofs* with \qedhere, and at the end of the statement of a
% theorem with \qed (meaning no proof will be given), but it can't
% seem to combine the two.  Use this instead.
\def\thmqedhere{\expandafter\csname\csname @currenvir\endcsname @qed\endcsname}

% Number numbered lists as (i), (ii), ...
\renewcommand{\theenumi}{(\roman{enumi})}
\renewcommand{\labelenumi}{\theenumi}

%% Labeling that keeps track of theorem-type names.  Superseded by
%% autoref from hyperref, as above, but we keep this in case we are
%% using a journal style file that is incompatible with hyperref.
% 
% \ifx\SK@label\undefined\let\SK@label\label\fi
% \let\your@thm\@thm
% \def\@thm#1#2#3{\gdef\currthmtype{#3}\your@thm{#1}{#2}{#3}}
% \def\xlabel#1{{\let\your@currentlabel\@currentlabel\def\@currentlabel
% {\currthmtype~\your@currentlabel}
% \SK@label{#1@}}\label{#1}}
% \def\xref#1{\ref{#1@}}

% Also number formulas with the theorem counter
\let\c@equation\c@thm
\numberwithin{equation}{section}

% Only show numbers for equations that are actually referenced (or
% whose tags are specified manually) - uses mathtools.
\mathtoolsset{showonlyrefs,showmanualtags}

%% Fix enumerate spacing with paralist.  This has two parts:
%%   1. enable mixing of "old spacing" lists with those adjusted by paralist
%%   2. allow us to specify a number based on which to adjust the spacing
%% For the first, use \killspacingtrue when you want the spacing
%% adjusted, then \killspacingfalse to turn adjustment off.  For the
%% second, use \maxenum=14 to set the widest number you want the
%% spacing to be calculated with.
\newlength\oldleftmargini       % save old spacing
\newlength\oldleftmarginii
\newlength\oldleftmarginiii
\newlength\oldleftmarginiv
\newlength\oldleftmarginv
\newlength\oldleftmarginvi
\newcount\maxenum
\maxenum=7
\newif\ifkillspacing
\def\@adjust@enum@labelwidth{%
  \advance\@listdepth by 1\relax
  \ifkillspacing                % do the paralist thing
    \csname c@\@enumctr\endcsname\maxenum
    \settowidth{\@tempdima}{%
      \csname label\@enumctr\endcsname\hspace{\labelsep}}%
    \csname leftmargin\romannumeral\@listdepth\endcsname
      \@tempdima
  \else                         % otherwise, reset it
    \csname fixspacing\romannumeral\@listdepth\endcsname
  \fi
  \advance\@listdepth by -1\relax}
% these commands, one for each enum level (I couldn't get a generic
% one to work), test whether oldleftmargin has been set yet, and if
% not, set it from leftmargin; otherwise, they reset leftmargin to
% it.  Just setting oldleftmargin to leftmargin in the preamble
% doesn't seem to work.
\def\fixspacingi{\ifnum\oldleftmargini=0\setlength\oldleftmargini\leftmargini\else\setlength\leftmargini\oldleftmargini\fi}
\def\fixspacingii{\ifnum\oldleftmarginii=0\setlength\oldleftmarginii\leftmarginii\else\setlength\leftmarginii\oldleftmarginii\fi}
\def\fixspacingiii{\ifnum\oldleftmarginiii=0\setlength\oldleftmarginiii\leftmarginiii\else\setlength\leftmarginiii\oldleftmarginiii\fi}
\def\fixspacingiv{\ifnum\oldleftmarginiv=0\setlength\oldleftmarginiv\leftmarginiv\else\setlength\leftmarginiv\oldleftmarginiv\fi}
\def\fixspacingv{\ifnum\oldleftmarginv=0\setlength\oldleftmarginv\leftmarginv\else\setlength\leftmarginv\oldleftmarginv\fi}
\def\fixspacingvi{\ifnum\oldleftmarginvi=0\setlength\oldleftmarginvi\leftmarginvi\else\setlength\leftmarginvi\oldleftmarginvi\fi}

%% Fix paralist references, so that we can refer to (1) instead of
%% just 1.
\def\pl@label#1#2{%
  \edef\pl@the{\noexpand#1{\@enumctr}}%
  \pl@lab\expandafter{\the\pl@lab\csname yourthe\@enumctr\endcsname}%
  \advance\@tempcnta1
  \pl@loop}
\def\@enumlabel@#1[#2]{%
  \@plmylabeltrue
  \@tempcnta0
  \pl@lab{}%
  \let\pl@the\pl@qmark
  \expandafter\pl@loop\@gobble#2\@@@
  \ifnum\@tempcnta=1\else
    \PackageWarning{paralist}{Incorrect label; no or multiple
      counters.\MessageBreak The label is: \@gobble#2}%
  \fi
  \expandafter\edef\csname label\@enumctr\endcsname{\the\pl@lab}%
  \expandafter\edef\csname the\@enumctr\endcsname{\the\pl@lab}%
  \expandafter\let\csname yourthe\@enumctr\endcsname\pl@the
  #1}


% GREEK LETTERS, ETC.
\alwaysmath{alpha}
\alwaysmath{beta}
\alwaysmath{gamma}
\alwaysmath{Gamma}
\alwaysmath{delta}
\alwaysmath{Delta}
\alwaysmath{epsilon}
\mdef\ep{\varepsilon}
\alwaysmath{zeta}
\alwaysmath{eta}
\alwaysmath{theta}
\alwaysmath{Theta}
\alwaysmath{iota}
\alwaysmath{kappa}
\alwaysmath{lambda}
\alwaysmath{Lambda}
\alwaysmath{mu}
\alwaysmath{nu}
\alwaysmath{xi}
\alwaysmath{pi}
\alwaysmath{rho}
\alwaysmath{sigma}
\alwaysmath{Sigma}
\alwaysmath{tau}
\alwaysmath{upsilon}
\alwaysmath{Upsilon}
\alwaysmath{phi}
\alwaysmath{Pi}
\alwaysmath{Phi}
\mdef\ph{\varphi}
\alwaysmath{chi}
\alwaysmath{psi}
\alwaysmath{Psi}
\alwaysmath{omega}
\alwaysmath{Omega}
\let\al\alpha
\let\be\beta
\let\gm\gamma
\let\Gm\Gamma
\let\de\delta
\let\De\Delta
\let\si\sigma
\let\Si\Sigma
\let\om\omega
\let\ka\kappa
\let\la\lambda
\let\La\Lambda
\let\ze\zeta
\let\th\theta
\let\Th\Theta
\let\vth\vartheta

\makeatother

% Tikz styles
\tikzstyle{tickarrow}=[->,postaction={decorate},decoration={markings,mark=at position .5 with {\draw[-] (0,-0.1) -- (0,0.1);}},line width=0.50]

% Local Variables:
% mode: latex
% TeX-master: ""
% End:

\begin{document}

{\small
\begin{equation*}
\begin{aligned}
\begin{tikzpicture}[xscale=2.25, yscale=1.5]
%%%% Row A
\node (A1) at (0,7){$\substack{\tens (\tens \times \transid)}$};
\node (A3) at (3,7){$\substack{\tens (\transid \times \tens) }$};
\node (A5) at (4.5,7){$\substack{\tens (\transid \times \tens) \\ ( \transid \times \tens \times \transid) \\ (\transid \times \transid  \times I \times \transid)}$};
\node (A7) at (6,7){$\substack{\tens (\transid \times \tens)\\(\transid \times \transid \times \tens)\\ (\transid \times \transid  \times I \times \transid)}$};
\node (A8) at (7,7){$\substack{\tens (\transid \times \tens)}$};
%%%
\draw[doubleloose] (A1) to node[above]{$\substack{\alpha }$} (A3);
\draw[doubleloose] (A3) to node[above]{$\substack{\looseid \looseid \\(\looseid \times r^{-1} \times  \looseid) }$} (A5);
\draw[doubleloose] (A5) to node[above]{$\substack{\looseid (\looseid \times \alpha) \looseid }$} (A7);
\draw[doubleloose] (A7) to node[above]{$\substack{\looseid \looseid \\ (\looseid \times \looseid \times l)}$} (A8);
%%%% Row B
\node (B1) at (0,6){$\substack{\tens (\tens \times \transid)}$};
\node (B3) at (3,6){$\substack{\tens (\tens \times \transid) \\ (\transid \times \tens \times \transid) \\ (\transid \times \transid \times I \times \transid)}$};
\node (B5) at (4.5,6){$\substack{\tens (\transid \times \tens) \\ ( \transid \times \tens \times \transid) \\ (\transid \times \transid  \times I \times \transid)}$};
\node (B7) at (6,6){$\substack{\tens (\transid \times \tens)\\(\transid \times \transid \times \tens)\\ (\transid \times \transid  \times I \times \transid)}$};
\node (B8) at (7,6){$\substack{\tens (\transid \times \tens)}$};
%%%
\draw[doubleloose] (B1) to node[above]{$\substack{\looseid \looseid \\(\looseid \times r^{-1} \times  \looseid)}$} (B3);
\draw[doubleloose] (B3) to node[above]{$\substack{\alpha \looseid \looseid  }$} (B5);
\draw[doubleloose] (B5) to node[above]{$\substack{\looseid (\looseid \times \alpha) \looseid }$} (B7);
\draw[doubleloose] (B7) to node[above]{$\substack{\looseid \looseid \\ (\looseid \times \looseid \times l)}$} (B8);
%%% AB
\draw[doubletighteq] (A1) to (B1);
\draw[doubletighteq] (A5) to (B5);
\draw[doubletighteq] (A8) to (B8);
%%%% Row C
\node (C1) at (0,5){$\substack{\tens (\tens \times \transid)}$};
\node (C15) at (.9,5){$\substack{\tens (\tens \times \transid) \\(\transid \times I \times \transid) \\(\tens \times \transid \times \transid)}$};
\node (C2) at (1.9,5){$\substack{\tens (\tens \times \transid) \\(\tens \times \transid \times \transid) \\(\transid \times \transid \times I \times \transid) \\}$};
\node (C3) at (3,5){$\substack{\tens (\tens \times \transid) \\ (\transid \times \tens \times \transid) \\ (\transid \times \transid \times I \times \transid)}$};
\node (C5) at (4.5,5){$\substack{\tens (\transid \times \tens) \\ ( \transid \times \tens \times \transid) \\ (\transid \times \transid  \times I \times \transid)}$};
\node (C7) at (6,5){$\substack{\tens (\transid \times \tens)\\(\transid \times \transid \times \tens)\\ (\transid \times \transid  \times I \times \transid)}$};
\node (C8) at (7,5){$\substack{\tens (\transid \times \tens)}$};
%%%
\draw[doubleloose] (C1) to node[above]{$\substack{\looseid (r^{-1} \times \looseid)\\ \looseid}$} (C15);
\draw[doubletighteq] (C15) to  (C2);
\draw[doubleloose] (C2) to node[above]{$\substack{\looseid (\alpha \times \looseid) \\ \looseid }$} (C3);
\draw[doubleloose] (C3) to node[above]{$\substack{\substack{\alpha } \looseid \looseid  }$} (C5);
\draw[doubleloose] (C5) to node[above]{$\substack{\looseid (\looseid \times \alpha) \looseid }$} (C7);
\draw[doubleloose] (C7) to node[above]{$\substack{\looseid \looseid \\ (\looseid \times \looseid \times l)}$} (C8);
%%% BC
\draw[doubletighteq] (B1) to (C1);
\draw[doubletighteq] (B3) to (C3);
\draw[doubletighteq] (B8) to (C8);
%%%% Row D
\node (D1) at (0,4){$\substack{\tens (\tens \times \transid)}$};
\node (D15) at (.9,4){$\substack{\tens (\tens \times \transid) \\(\transid \times I \times \transid) \\(\tens \times \transid \times \transid)}$};
\node (D2) at (1.9,4){$\substack{\tens (\tens \times \transid) \\(\tens \times \transid \times \transid) \\(\transid \times \transid \times I \times \transid) \\}$};
\node (D3) at (3,4){$\substack{\tens (\transid \times \tens) \\ (\tens \times \transid \times  \transid) \\ (\transid \times \transid \times I \times \transid)}$};
\node (D5) at (4.5,4){$\substack{\tens (\tens \times \transid) \\ (\transid \times \transid \times \tens) \\ (\transid \times \transid \times I \times \transid)}$};
\node (D7) at (6,4){$\substack{\tens (\transid \times \tens) \\ (\transid \times \transid \times \tens) \\ (\transid \times \transid \times I \times \transid)}$};
\node (D8) at (7,4){$\substack{\tens (\transid \times \tens)}$};
%%%
\draw[doubleloose] (D1) to node[above]{$\substack{\looseid (r^{-1} \times \looseid)\\ \looseid}$} (D15);
\draw[doubletighteq] (D15) to  (D2);
\draw[doubleloose] (D2) to node[above]{$\substack{\alpha \looseid \looseid }$} (D3);
\draw[doubletighteq] (D3) to (D5);
\draw[doubleloose] (D5) to node[above]{$\substack{\alpha  \looseid \looseid  }$} (D7);
\draw[doubleloose] (D7) to node[above]{$\substack{\looseid \looseid \\ (\looseid \times \looseid \times l)}$} (D8);
%%% CD
\draw[doubletighteq] (C1) to (D1);
\draw[doubletighteq] (C2) to (D2);
\draw[doubletighteq] (C7) to (D7);
\draw[doubletighteq] (C8) to (D8);
%%%% Row E
\node (E1) at (0,3){$\substack{\tens (\tens \times \transid)}$};
\node (E15) at (.9,3){$\substack{\tens (\tens \times \transid) \\(\transid \times I \times \transid) \\(\tens \times \transid \times \transid)}$};
\node (E2) at (1.9,3){$\substack{\tens (\tens \times \transid) \\(\tens \times \transid \times \transid) \\(\transid \times \transid \times I \times \transid) \\}$};
\node (E3) at (3,3){$\substack{\tens (\transid \times \tens) \\ (\tens \times \transid \times  \transid) \\ (\transid \times \transid \times I \times \transid)}$};
\node (E5) at (4.5,3){$\substack{\tens (\transid \times \tens) \\ (\transid \times \transid \times \tens) \\ (\transid \times \transid \times I \times \transid)}$};
\node (E7) at (6,3){$\substack{\tens (\tens \times \transid)}$};
\node (E8) at (7,3){$\substack{\tens (\transid \times \tens)}$};
%%%
\draw[doubleloose] (E1) to node[above]{$\substack{\looseid (r^{-1} \times \looseid)\\ \looseid}$} (E15);
\draw[doubletighteq] (E15) to  (E2);
\draw[doubleloose] (E2) to node[above]{$\substack{\alpha \looseid \looseid }$} (E3);
\draw[doubletighteq] (E3) to (E5);
\draw[doubleloose] (E5) to node[above]{$\substack{\looseid \looseid (\looseid \times \looseid \times l)  }$} (E7);
\draw[doubleloose] (E7) to node[above]{$\substack{\alpha }$} (E8);
%%% DE
\draw[doubletighteq] (D1) to (E1);
\draw[doubletighteq] (D5) to (E5);
\draw[doubletighteq] (D8) to (E8);
%%%% Row F
\node (F1) at (0,2){$\substack{\tens (\tens \times \transid)}$};
\node (F7) at (6,2){$\substack{\tens (\tens \times \transid)}$};
\node (F8) at (7,2){$\substack{\tens (\transid \times \tens)}$};
%%%
\draw[doubleloose] (F1) to node[above]{$\substack{\looseid}$} (F7);
\draw[doubleloose] (F7) to node[above]{$\substack{\alpha }$} (F8);
%%% EF
\draw[doubletighteq] (E1) to (F1);
\draw[doubletighteq] (E7) to (F7);
\draw[doubletighteq] (E8) to (F8);
%%% 3-cells
\node at (2.5,6.5) {$\substack{\iso \horl \verc {\horr}^{-1} }$};
\node at (6,6.5) {$\substack{= }$};
\node at (1.5,5.5) {$\substack{\Downarrow \tightid \tightid (\tightid \times \rho \times \tightid) }$};
\node at (5,5.5) {$\substack{= }$};
\node at (1,4.5) {$\substack{=}$};
\node at (4,4.5) {$\substack{\DDownarrow \pi \tightid }$};
\node at (6.5,4.5) {$\substack{=}$};
\node at (2.5,3.5) {$\substack{=}$};
\node at (6,3.5) {$\substack{\iso \horl \verc {\horr}^{-1} }$};
\node at (3,2.5) {$\substack{\DDownarrow \tightid (\mu \times \tightid) }$};
\node at (6.5,2.5) {$\substack{=}$};
\end{tikzpicture} 
\end{aligned}
\end{equation*}
\begin{equation}\label{eq:monobjeq3}
=
\end{equation}
\begin{equation*}
\begin{tikzpicture}[xscale=2.25, yscale=1.5]
%%%% Row A
\node (A1) at (0,7){$\substack{\tens (\tens \times \transid)}$};
\node (A3) at (2,7){$\substack{\tens (\transid \times \tens) }$};
\node (A5) at (4,7){$\substack{\tens (\transid \times \tens) \\ ( \transid \times \tens \times \transid) \\ (\transid \times \transid  \times I \times \transid)}$};
\node (A7) at (6,7){$\substack{\tens (\transid \times \tens)\\(\transid \times \transid \times \tens)\\ (\transid \times \transid  \times I \times \transid)}$};
\node (A8) at (7,7){$\substack{\tens (\transid \times \tens)}$};
%%%
\draw[doubleloose] (A1) to node[above]{$\substack{\alpha }$} (A3);
\draw[doubleloose] (A3) to node[above]{$\substack{\looseid \looseid \\(\looseid \times r^{-1} \times  \looseid) }$} (A5);
\draw[doubleloose] (A5) to node[above]{$\substack{\looseid (\looseid \times \alpha) \looseid }$} (A7);
\draw[doubleloose] (A7) to node[above]{$\substack{\looseid \looseid \\ (\looseid \times l \times \looseid)}$} (A8);
%%%% Row B
\node (B1) at (0,6){$\substack{\tens (\tens \times \transid)}$};
\node (B3) at (2,6){$\substack{\tens ( \transid \times \tens)}$};
\node (B8) at (7,6){$\substack{\tens (\transid \times \tens)}$};
%%%
\draw[doubleloose] (B1) to node[above]{$\substack{\alpha }$} (B3);
\draw[doubleloose] (B3) to node[above]{$\substack{\looseid}$}(B8);
%%% AB
\draw[doubletighteq] (A1) to (B1);
\draw[doubletighteq] (A3) to (B3);
\draw[doubletighteq] (A8) to (B8);
%%%% Row C
\node (C1) at (0,5){$\substack{\tens (\tens \times \transid)}$};
\node (C7) at (6,5){$\substack{\tens (\tens \times \transid)}$};\node (C8) at (7,5){$\substack{\tens (\transid \times \tens)}$};
%%%
\draw[doubleloose] (C1) to node[above]{$\looseid$} (C7);
\draw[doubleloose] (C7) to node[above]{$\substack{\alpha }$}(C8);
%%% BC
\draw[doubletighteq] (B1) to  (C1);
\draw[doubletighteq] (B8) to  (C8);
%%%
\node at (5,6.5) {$\substack{ \DDownarrow \tightid (\tightid \times \mu)}$};
\node at (6.5,6.5) {$\substack{=}$};
\node at (3.5,5.5) {$\substack{\iso \horl \verc {\horr}^{-1} }$};
\end{tikzpicture} 
\end{equation*}}
\end{document}  \newpage


\subsubsection*{Lax Monoidal 1-cell}

%
\documentclass[12pt]{ociamthesis}
\usepackage{tikz}
\usepackage{amsmath}
\usepackage{rotating}

\usepackage{amssymb,amsmath,stmaryrd,txfonts,mathrsfs,amsthm}
\usepackage[all,2cell]{xy}
\usepackage[neveradjust]{paralist}
\usepackage{hyperref}
\usepackage{mathtools}
\usepackage{tikz}
\usetikzlibrary{trees}
\usetikzlibrary{topaths}
\usetikzlibrary{decorations}
\usetikzlibrary{decorations.pathreplacing}
\usetikzlibrary{decorations.pathmorphing}
\usetikzlibrary{decorations.markings}
\usetikzlibrary{matrix,backgrounds,folding}
\usetikzlibrary{chains,scopes,positioning,fit}
\usetikzlibrary{arrows,shadows}
\usetikzlibrary{calc} 
\usetikzlibrary{chains}
\usetikzlibrary{shapes,shapes.geometric,shapes.misc}
\usepackage{smbicat}


\makeatletter
\let\ea\expandafter

%% Defining commands that are always in math mode.
\def\mdef#1#2{\ea\ea\ea\gdef\ea\ea\noexpand#1\ea{\ea\ensuremath\ea{#2}}}
\def\alwaysmath#1{\ea\ea\ea\global\ea\ea\ea\let\ea\ea\csname your@#1\endcsname\csname #1\endcsname
  \ea\def\csname #1\endcsname{\ensuremath{\csname your@#1\endcsname}}}

% Script letters
\newcommand{\sA}{\ensuremath{\mathscr{A}}}
\newcommand{\sB}{\ensuremath{\mathscr{B}}}
\newcommand{\sC}{\ensuremath{\mathscr{C}}}
\newcommand{\sD}{\ensuremath{\mathscr{D}}}
\newcommand{\sE}{\ensuremath{\mathscr{E}}}
\newcommand{\sF}{\ensuremath{\mathscr{F}}}
\newcommand{\sG}{\ensuremath{\mathscr{G}}}
\newcommand{\sH}{\ensuremath{\mathscr{H}}}
\newcommand{\sI}{\ensuremath{\mathscr{I}}}
\newcommand{\sJ}{\ensuremath{\mathscr{J}}}
\newcommand{\sK}{\ensuremath{\mathscr{K}}}
\newcommand{\sL}{\ensuremath{\mathscr{L}}}
\newcommand{\sM}{\ensuremath{\mathscr{M}}}
\newcommand{\sN}{\ensuremath{\mathscr{N}}}
\newcommand{\sO}{\ensuremath{\mathscr{O}}}
\newcommand{\sP}{\ensuremath{\mathscr{P}}}
\newcommand{\sQ}{\ensuremath{\mathscr{Q}}}
\newcommand{\sR}{\ensuremath{\mathscr{R}}}
\newcommand{\sS}{\ensuremath{\mathscr{S}}}
\newcommand{\sT}{\ensuremath{\mathscr{T}}}
\newcommand{\sU}{\ensuremath{\mathscr{U}}}
\newcommand{\sV}{\ensuremath{\mathscr{V}}}
\newcommand{\sW}{\ensuremath{\mathscr{W}}}
\newcommand{\sX}{\ensuremath{\mathscr{X}}}
\newcommand{\sY}{\ensuremath{\mathscr{Y}}}
\newcommand{\sZ}{\ensuremath{\mathscr{Z}}}

% Calligraphic letters
\newcommand{\cA}{\ensuremath{\mathcal{A}}}
\newcommand{\cB}{\ensuremath{\mathcal{B}}}
\newcommand{\cC}{\ensuremath{\mathcal{C}}}
\newcommand{\cD}{\ensuremath{\mathcal{D}}}
\newcommand{\cE}{\ensuremath{\mathcal{E}}}
\newcommand{\cF}{\ensuremath{\mathcal{F}}}
\newcommand{\cG}{\ensuremath{\mathcal{G}}}
\newcommand{\cH}{\ensuremath{\mathcal{H}}}
\newcommand{\cI}{\ensuremath{\mathcal{I}}}
\newcommand{\cJ}{\ensuremath{\mathcal{J}}}
\newcommand{\cK}{\ensuremath{\mathcal{K}}}
\newcommand{\cL}{\ensuremath{\mathcal{L}}}
\newcommand{\cM}{\ensuremath{\mathcal{M}}}
\newcommand{\cN}{\ensuremath{\mathcal{N}}}
\newcommand{\cO}{\ensuremath{\mathcal{O}}}
\newcommand{\cP}{\ensuremath{\mathcal{P}}}
\newcommand{\cQ}{\ensuremath{\mathcal{Q}}}
\newcommand{\cR}{\ensuremath{\mathcal{R}}}
\newcommand{\cS}{\ensuremath{\mathcal{S}}}
\newcommand{\cT}{\ensuremath{\mathcal{T}}}
\newcommand{\cU}{\ensuremath{\mathcal{U}}}
\newcommand{\cV}{\ensuremath{\mathcal{V}}}
\newcommand{\cW}{\ensuremath{\mathcal{W}}}
\newcommand{\cX}{\ensuremath{\mathcal{X}}}
\newcommand{\cY}{\ensuremath{\mathcal{Y}}}
\newcommand{\cZ}{\ensuremath{\mathcal{Z}}}

% blackboard bold letters
\newcommand{\lA}{\ensuremath{\mathbb{A}}}
\newcommand{\lB}{\ensuremath{\mathbb{B}}}
\newcommand{\lC}{\ensuremath{\mathbb{C}}}
\newcommand{\lD}{\ensuremath{\mathbb{D}}}
\newcommand{\lE}{\ensuremath{\mathbb{E}}}
\newcommand{\lF}{\ensuremath{\mathbb{F}}}
\newcommand{\lG}{\ensuremath{\mathbb{G}}}
\newcommand{\lH}{\ensuremath{\mathbb{H}}}
\newcommand{\lI}{\ensuremath{\mathbb{I}}}
\newcommand{\lJ}{\ensuremath{\mathbb{J}}}
\newcommand{\lK}{\ensuremath{\mathbb{K}}}
\newcommand{\lL}{\ensuremath{\mathbb{L}}}
\newcommand{\lM}{\ensuremath{\mathbb{M}}}
\newcommand{\lN}{\ensuremath{\mathbb{N}}}
\newcommand{\lO}{\ensuremath{\mathbb{O}}}
\newcommand{\lP}{\ensuremath{\mathbb{P}}}
\newcommand{\lQ}{\ensuremath{\mathbb{Q}}}
\newcommand{\lR}{\ensuremath{\mathbb{R}}}
\newcommand{\lS}{\ensuremath{\mathbb{S}}}
\newcommand{\lT}{\ensuremath{\mathbb{T}}}
\newcommand{\lU}{\ensuremath{\mathbb{U}}}
\newcommand{\lV}{\ensuremath{\mathbb{V}}}
\newcommand{\lW}{\ensuremath{\mathbb{W}}}
\newcommand{\lX}{\ensuremath{\mathbb{X}}}
\newcommand{\lY}{\ensuremath{\mathbb{Y}}}
\newcommand{\lZ}{\ensuremath{\mathbb{Z}}}

% bold letters
\newcommand{\bA}{\ensuremath{\mathbf{A}}}
\newcommand{\bB}{\ensuremath{\mathbf{B}}}
\newcommand{\bC}{\ensuremath{\mathbf{C}}}
\newcommand{\bD}{\ensuremath{\mathbf{D}}}
\newcommand{\bE}{\ensuremath{\mathbf{E}}}
\newcommand{\bF}{\ensuremath{\mathbf{F}}}
\newcommand{\bG}{\ensuremath{\mathbf{G}}}
\newcommand{\bH}{\ensuremath{\mathbf{H}}}
\newcommand{\bI}{\ensuremath{\mathbf{I}}}
\newcommand{\bJ}{\ensuremath{\mathbf{J}}}
\newcommand{\bK}{\ensuremath{\mathbf{K}}}
\newcommand{\bL}{\ensuremath{\mathbf{L}}}
\newcommand{\bM}{\ensuremath{\mathbf{M}}}
\newcommand{\bN}{\ensuremath{\mathbf{N}}}
\newcommand{\bO}{\ensuremath{\mathbf{O}}}
\newcommand{\bP}{\ensuremath{\mathbf{P}}}
\newcommand{\bQ}{\ensuremath{\mathbf{Q}}}
\newcommand{\bR}{\ensuremath{\mathbf{R}}}
\newcommand{\bS}{\ensuremath{\mathbf{S}}}
\newcommand{\bT}{\ensuremath{\mathbf{T}}}
\newcommand{\bU}{\ensuremath{\mathbf{U}}}
\newcommand{\bV}{\ensuremath{\mathbf{V}}}
\newcommand{\bW}{\ensuremath{\mathbf{W}}}
\newcommand{\bX}{\ensuremath{\mathbf{X}}}
\newcommand{\bY}{\ensuremath{\mathbf{Y}}}
\newcommand{\bZ}{\ensuremath{\mathbf{Z}}}

% fraktur letters
\newcommand{\fa}{\ensuremath{\mathfrak{a}}}
\newcommand{\fb}{\ensuremath{\mathfrak{b}}}
\newcommand{\fc}{\ensuremath{\mathfrak{c}}}
\newcommand{\fd}{\ensuremath{\mathfrak{d}}}
\newcommand{\fe}{\ensuremath{\mathfrak{e}}}
\newcommand{\ff}{\ensuremath{\mathfrak{f}}}
\newcommand{\fg}{\ensuremath{\mathfrak{g}}}
\newcommand{\fh}{\ensuremath{\mathfrak{h}}}
\newcommand{\fj}{\ensuremath{\mathfrak{j}}}
\newcommand{\fk}{\ensuremath{\mathfrak{k}}}
\newcommand{\fl}{\ensuremath{\mathfrak{l}}}
\newcommand{\fm}{\ensuremath{\mathfrak{m}}}
\newcommand{\fn}{\ensuremath{\mathfrak{n}}}
\newcommand{\fo}{\ensuremath{\mathfrak{o}}}
\newcommand{\fp}{\ensuremath{\mathfrak{p}}}
\newcommand{\fq}{\ensuremath{\mathfrak{q}}}
\newcommand{\fr}{\ensuremath{\mathfrak{r}}}
\newcommand{\fs}{\ensuremath{\mathfrak{s}}}
\newcommand{\ft}{\ensuremath{\mathfrak{t}}}
\newcommand{\fu}{\ensuremath{\mathfrak{u}}}
\newcommand{\fv}{\ensuremath{\mathfrak{v}}}
\newcommand{\fw}{\ensuremath{\mathfrak{w}}}
\newcommand{\fx}{\ensuremath{\mathfrak{x}}}
\newcommand{\fy}{\ensuremath{\mathfrak{y}}}
\newcommand{\fz}{\ensuremath{\mathfrak{z}}}

% fraktur letters
\newcommand{\fA}{\ensuremath{\mathfrak{A}}}
\newcommand{\fB}{\ensuremath{\mathfrak{B}}}
\newcommand{\fC}{\ensuremath{\mathfrak{C}}}

\mdef\fahat{\hat{\fa}}

% Underline letters
\newcommand{\uA}{\ensuremath{\underline{A}}}
\newcommand{\uB}{\ensuremath{\underline{B}}}
\newcommand{\uC}{\ensuremath{\underline{C}}}
\newcommand{\uD}{\ensuremath{\underline{D}}}
\newcommand{\uE}{\ensuremath{\underline{E}}}
\newcommand{\uF}{\ensuremath{\underline{F}}}
\newcommand{\uG}{\ensuremath{\underline{G}}}
\newcommand{\uH}{\ensuremath{\underline{H}}}
\newcommand{\uI}{\ensuremath{\underline{I}}}
\newcommand{\uJ}{\ensuremath{\underline{J}}}
\newcommand{\uK}{\ensuremath{\underline{K}}}
\newcommand{\uL}{\ensuremath{\underline{L}}}
\newcommand{\uM}{\ensuremath{\underline{M}}}
\newcommand{\uN}{\ensuremath{\underline{N}}}
\newcommand{\uO}{\ensuremath{\underline{O}}}
\newcommand{\uP}{\ensuremath{\underline{P}}}
\newcommand{\uQ}{\ensuremath{\underline{Q}}}
\newcommand{\uR}{\ensuremath{\underline{R}}}
\newcommand{\uS}{\ensuremath{\underline{S}}}
\newcommand{\uT}{\ensuremath{\underline{T}}}
\newcommand{\uU}{\ensuremath{\underline{U}}}
\newcommand{\uV}{\ensuremath{\underline{V}}}
\newcommand{\uW}{\ensuremath{\underline{W}}}
\newcommand{\uX}{\ensuremath{\underline{X}}}
\newcommand{\uY}{\ensuremath{\underline{Y}}}
\newcommand{\uZ}{\ensuremath{\underline{Z}}}

% bars
\newcommand{\Abar}{\ensuremath{\overline{A}}}
\newcommand{\Bbar}{\ensuremath{\overline{B}}}
\newcommand{\Cbar}{\ensuremath{\overline{C}}}
\newcommand{\Dbar}{\ensuremath{\overline{D}}}
\newcommand{\Ebar}{\ensuremath{\overline{E}}}
\newcommand{\Fbar}{\ensuremath{\overline{F}}}
\newcommand{\Gbar}{\ensuremath{\overline{G}}}
\newcommand{\Hbar}{\ensuremath{\overline{H}}}
\newcommand{\Ibar}{\ensuremath{\overline{I}}}
\newcommand{\Jbar}{\ensuremath{\overline{J}}}
\newcommand{\Kbar}{\ensuremath{\overline{K}}}
\newcommand{\Lbar}{\ensuremath{\overline{L}}}
\newcommand{\Mbar}{\ensuremath{\overline{M}}}
\newcommand{\Nbar}{\ensuremath{\overline{N}}}
\newcommand{\Obar}{\ensuremath{\overline{O}}}
\newcommand{\Pbar}{\ensuremath{\overline{P}}}
\newcommand{\Qbar}{\ensuremath{\overline{Q}}}
\newcommand{\Rbar}{\ensuremath{\overline{R}}}
\newcommand{\Sbar}{\ensuremath{\overline{S}}}
\newcommand{\Tbar}{\ensuremath{\overline{T}}}
\newcommand{\Ubar}{\ensuremath{\overline{U}}}
\newcommand{\Vbar}{\ensuremath{\overline{V}}}
\newcommand{\Wbar}{\ensuremath{\overline{W}}}
\newcommand{\Xbar}{\ensuremath{\overline{X}}}
\newcommand{\Ybar}{\ensuremath{\overline{Y}}}
\newcommand{\Zbar}{\ensuremath{\overline{Z}}}
\newcommand{\abar}{\ensuremath{\overline{a}}}
\newcommand{\bbar}{\ensuremath{\overline{b}}}
\newcommand{\cbar}{\ensuremath{\overline{c}}}
\newcommand{\dbar}{\ensuremath{\overline{d}}}
\newcommand{\ebar}{\ensuremath{\overline{e}}}
\newcommand{\fbar}{\ensuremath{\overline{f}}}
\newcommand{\gbar}{\ensuremath{\overline{g}}}
%\newcommand{\hbar}{\ensuremath{\overline{h}}} % whoops, \hbar means something else!
\newcommand{\ibar}{\ensuremath{\overline{\imath}}}
\newcommand{\jbar}{\ensuremath{\overline{\jmath}}}
\newcommand{\kbar}{\ensuremath{\overline{k}}}
\newcommand{\lbar}{\ensuremath{\overline{l}}}
\newcommand{\mbar}{\ensuremath{\overline{m}}}
\newcommand{\nbar}{\ensuremath{\overline{n}}}
%\newcommand{\obar}{\ensuremath{\overline{o}}}
\newcommand{\pbar}{\ensuremath{\overline{p}}}
\newcommand{\qbar}{\ensuremath{\overline{q}}}
\newcommand{\rbar}{\ensuremath{\overline{r}}}
\newcommand{\sbar}{\ensuremath{\overline{s}}}
\newcommand{\tbar}{\ensuremath{\overline{t}}}
\newcommand{\ubar}{\ensuremath{\overline{u}}}
\newcommand{\vbar}{\ensuremath{\overline{v}}}
\newcommand{\wbar}{\ensuremath{\overline{w}}}
\newcommand{\xbar}{\ensuremath{\overline{x}}}
\newcommand{\ybar}{\ensuremath{\overline{y}}}
\newcommand{\zbar}{\ensuremath{\overline{z}}}

% tildes
\newcommand{\Atil}{\ensuremath{\widetilde{A}}}
\newcommand{\Btil}{\ensuremath{\widetilde{B}}}
\newcommand{\Ctil}{\ensuremath{\widetilde{C}}}
\newcommand{\Dtil}{\ensuremath{\widetilde{D}}}
\newcommand{\Etil}{\ensuremath{\widetilde{E}}}
\newcommand{\Ftil}{\ensuremath{\widetilde{F}}}
\newcommand{\Gtil}{\ensuremath{\widetilde{G}}}
\newcommand{\Htil}{\ensuremath{\widetilde{H}}}
\newcommand{\Itil}{\ensuremath{\widetilde{I}}}
\newcommand{\Jtil}{\ensuremath{\widetilde{J}}}
\newcommand{\Ktil}{\ensuremath{\widetilde{K}}}
\newcommand{\Ltil}{\ensuremath{\widetilde{L}}}
\newcommand{\Mtil}{\ensuremath{\widetilde{M}}}
\newcommand{\Ntil}{\ensuremath{\widetilde{N}}}
\newcommand{\Otil}{\ensuremath{\widetilde{O}}}
\newcommand{\Ptil}{\ensuremath{\widetilde{P}}}
\newcommand{\Qtil}{\ensuremath{\widetilde{Q}}}
\newcommand{\Rtil}{\ensuremath{\widetilde{R}}}
\newcommand{\Stil}{\ensuremath{\widetilde{S}}}
\newcommand{\Ttil}{\ensuremath{\widetilde{T}}}
\newcommand{\Util}{\ensuremath{\widetilde{U}}}
\newcommand{\Vtil}{\ensuremath{\widetilde{V}}}
\newcommand{\Wtil}{\ensuremath{\widetilde{W}}}
\newcommand{\Xtil}{\ensuremath{\widetilde{X}}}
\newcommand{\Ytil}{\ensuremath{\widetilde{Y}}}
\newcommand{\Ztil}{\ensuremath{\widetilde{Z}}}
\newcommand{\atil}{\ensuremath{\widetilde{a}}}
\newcommand{\btil}{\ensuremath{\widetilde{b}}}
\newcommand{\ctil}{\ensuremath{\widetilde{c}}}
\newcommand{\dtil}{\ensuremath{\widetilde{d}}}
\newcommand{\etil}{\ensuremath{\widetilde{e}}}
\newcommand{\ftil}{\ensuremath{\widetilde{f}}}
\newcommand{\gtil}{\ensuremath{\widetilde{g}}}
\newcommand{\htil}{\ensuremath{\widetilde{h}}}
\newcommand{\itil}{\ensuremath{\widetilde{\imath}}}
\newcommand{\jtil}{\ensuremath{\widetilde{\jmath}}}
\newcommand{\ktil}{\ensuremath{\widetilde{k}}}
\newcommand{\ltil}{\ensuremath{\widetilde{l}}}
\newcommand{\mtil}{\ensuremath{\widetilde{m}}}
\newcommand{\ntil}{\ensuremath{\widetilde{n}}}
\newcommand{\otil}{\ensuremath{\widetilde{o}}}
\newcommand{\ptil}{\ensuremath{\widetilde{p}}}
\newcommand{\qtil}{\ensuremath{\widetilde{q}}}
\newcommand{\rtil}{\ensuremath{\widetilde{r}}}
\newcommand{\stil}{\ensuremath{\widetilde{s}}}
\newcommand{\ttil}{\ensuremath{\widetilde{t}}}
\newcommand{\util}{\ensuremath{\widetilde{u}}}
\newcommand{\vtil}{\ensuremath{\widetilde{v}}}
\newcommand{\wtil}{\ensuremath{\widetilde{w}}}
\newcommand{\xtil}{\ensuremath{\widetilde{x}}}
\newcommand{\ytil}{\ensuremath{\widetilde{y}}}
\newcommand{\ztil}{\ensuremath{\widetilde{z}}}

% Hats
\newcommand{\Ahat}{\ensuremath{\widehat{A}}}
\newcommand{\Bhat}{\ensuremath{\widehat{B}}}
\newcommand{\Chat}{\ensuremath{\widehat{C}}}
\newcommand{\Dhat}{\ensuremath{\widehat{D}}}
\newcommand{\Ehat}{\ensuremath{\widehat{E}}}
\newcommand{\Fhat}{\ensuremath{\widehat{F}}}
\newcommand{\Ghat}{\ensuremath{\widehat{G}}}
\newcommand{\Hhat}{\ensuremath{\widehat{H}}}
\newcommand{\Ihat}{\ensuremath{\widehat{I}}}
\newcommand{\Jhat}{\ensuremath{\widehat{J}}}
\newcommand{\Khat}{\ensuremath{\widehat{K}}}
\newcommand{\Lhat}{\ensuremath{\widehat{L}}}
\newcommand{\Mhat}{\ensuremath{\widehat{M}}}
\newcommand{\Nhat}{\ensuremath{\widehat{N}}}
\newcommand{\Ohat}{\ensuremath{\widehat{O}}}
\newcommand{\Phat}{\ensuremath{\widehat{P}}}
\newcommand{\Qhat}{\ensuremath{\widehat{Q}}}
\newcommand{\Rhat}{\ensuremath{\widehat{R}}}
\newcommand{\Shat}{\ensuremath{\widehat{S}}}
\newcommand{\That}{\ensuremath{\widehat{T}}}
\newcommand{\Uhat}{\ensuremath{\widehat{U}}}
\newcommand{\Vhat}{\ensuremath{\widehat{V}}}
\newcommand{\What}{\ensuremath{\widehat{W}}}
\newcommand{\Xhat}{\ensuremath{\widehat{X}}}
\newcommand{\Yhat}{\ensuremath{\widehat{Y}}}
\newcommand{\Zhat}{\ensuremath{\widehat{Z}}}
\newcommand{\ahat}{\ensuremath{\hat{a}}}
\newcommand{\bhat}{\ensuremath{\hat{b}}}
\newcommand{\chat}{\ensuremath{\hat{c}}}
\newcommand{\dhat}{\ensuremath{\hat{d}}}
\newcommand{\ehat}{\ensuremath{\hat{e}}}
\newcommand{\fhat}{\ensuremath{\hat{f}}}
\newcommand{\ghat}{\ensuremath{\hat{g}}}
\newcommand{\hhat}{\ensuremath{\hat{h}}}
\newcommand{\ihat}{\ensuremath{\hat{\imath}}}
\newcommand{\jhat}{\ensuremath{\hat{\jmath}}}
\newcommand{\khat}{\ensuremath{\hat{k}}}
\newcommand{\lhat}{\ensuremath{\hat{l}}}
\newcommand{\mhat}{\ensuremath{\hat{m}}}
\newcommand{\nhat}{\ensuremath{\hat{n}}}
\newcommand{\ohat}{\ensuremath{\hat{o}}}
\newcommand{\phat}{\ensuremath{\hat{p}}}
\newcommand{\qhat}{\ensuremath{\hat{q}}}
\newcommand{\rhat}{\ensuremath{\hat{r}}}
\newcommand{\shat}{\ensuremath{\hat{s}}}
\newcommand{\that}{\ensuremath{\hat{t}}}
\newcommand{\uhat}{\ensuremath{\hat{u}}}
\newcommand{\vhat}{\ensuremath{\hat{v}}}
\newcommand{\what}{\ensuremath{\hat{w}}}
\newcommand{\xhat}{\ensuremath{\hat{x}}}
\newcommand{\yhat}{\ensuremath{\hat{y}}}
\newcommand{\zhat}{\ensuremath{\hat{z}}}

%% FONTS AND DECORATION FOR GREEK LETTERS

%% the package `mathbbol' gives us blackboard bold greek and numbers,
%% but it does it by redefining \mathbb to use a different font, so that
%% all the other \mathbb letters look different too.  Here we import the
%% font with bb greek and numbers, but assign it a different name,
%% \mathbbb, so as not to replace the usual one.
\DeclareSymbolFont{bbold}{U}{bbold}{m}{n}
\DeclareSymbolFontAlphabet{\mathbbb}{bbold}
\newcommand{\bbDelta}{\ensuremath{\mathbbb{\Delta}}}
\newcommand{\bbone}{\ensuremath{\mathbbb{1}}}
\newcommand{\bbtwo}{\ensuremath{\mathbbb{2}}}
\newcommand{\bbthree}{\ensuremath{\mathbbb{3}}}

% greek with bars
\newcommand{\albar}{\ensuremath{\overline{\alpha}}}
\newcommand{\bebar}{\ensuremath{\overline{\beta}}}
\newcommand{\gmbar}{\ensuremath{\overline{\gamma}}}
\newcommand{\debar}{\ensuremath{\overline{\delta}}}
\newcommand{\phibar}{\ensuremath{\overline{\varphi}}}
\newcommand{\psibar}{\ensuremath{\overline{\psi}}}
\newcommand{\xibar}{\ensuremath{\overline{\xi}}}
\newcommand{\ombar}{\ensuremath{\overline{\omega}}}

% greek with hats
\newcommand{\alhat}{\ensuremath{\hat{\alpha}}}
\newcommand{\behat}{\ensuremath{\hat{\beta}}}
\newcommand{\gmhat}{\ensuremath{\hat{\gamma}}}
\newcommand{\dehat}{\ensuremath{\hat{\delta}}}

% greek with checks
\newcommand{\alchk}{\ensuremath{\check{\alpha}}}
\newcommand{\bechk}{\ensuremath{\check{\beta}}}
\newcommand{\gmchk}{\ensuremath{\check{\gamma}}}
\newcommand{\dechk}{\ensuremath{\check{\delta}}}

% greek with tildes
\newcommand{\altil}{\ensuremath{\widetilde{\alpha}}}
\newcommand{\betil}{\ensuremath{\widetilde{\beta}}}
\newcommand{\gmtil}{\ensuremath{\widetilde{\gamma}}}
\newcommand{\phitil}{\ensuremath{\widetilde{\varphi}}}
\newcommand{\psitil}{\ensuremath{\widetilde{\psi}}}
\newcommand{\xitil}{\ensuremath{\widetilde{\xi}}}
\newcommand{\omtil}{\ensuremath{\widetilde{\omega}}}

% MISCELLANEOUS SYMBOLS
\mdef\del{\partial}
\mdef\delbar{\overline{\partial}}
\let\sm\wedge
\newcommand{\dd}[1]{\ensuremath{\frac{\partial}{\partial {#1}}}}
\newcommand{\inv}{^{-1}}
\newcommand{\dual}{^{\vee}}
\mdef\hf{\textstyle\frac{1}{2}}
\mdef\thrd{\textstyle\frac{1}{3}}
\mdef\qtr{\textstyle\frac{1}{4}}
\let\meet\wedge
\let\join\vee
\let\dn\downarrow
\newcommand{\op}{^{\mathit{op}}}
\newcommand{\co}{^{\mathit{co}}}
\newcommand{\coop}{^{\mathit{coop}}}
\let\adj\dashv
\SelectTips{cm}{}
\newdir{ >}{{}*!/-10pt/@{>}}    % extra spacing for tail arrows in XYpic
\newcommand{\pushoutcorner}[1][dr]{\save*!/#1+1.2pc/#1:(1,-1)@^{|-}\restore}
\newcommand{\pullbackcorner}[1][dr]{\save*!/#1-1.2pc/#1:(-1,1)@^{|-}\restore}
\let\iso\cong
\let\eqv\simeq
\let\cng\equiv
\mdef\Id{\mathrm{Id}}
\mdef\id{\mathrm{id}}
\alwaysmath{ell}
\alwaysmath{infty}
\alwaysmath{odot}
\def\frc#1/#2.{\frac{#1}{#2}}   % \frc x^2+1 / x^2-1 .
\mdef\ten{\mathrel{\otimes}}
\mdef\bigten{\bigotimes}
\mdef\sqten{\mathrel{\boxtimes}}
\def\pow(#1,#2){\mathop{\pitchfork}(#1,#2)} % powers and
\def\cpw{\mathop{\odot}}                    % copowers
\newcommand{\mathid}{\mbox{id}}
\newcommand{\cat}[1]{\ensuremath{\mathbf{#1}}}


%% OPERATORS
\DeclareMathOperator\lan{Lan}
\DeclareMathOperator\ran{Ran}
\DeclareMathOperator\colim{colim}
\DeclareMathOperator\coeq{coeq}
\DeclareMathOperator\eq{eq}
\DeclareMathOperator\Tot{Tot}
\DeclareMathOperator\cosk{cosk}
\DeclareMathOperator\sk{sk}
\DeclareMathOperator\im{im}
\DeclareMathOperator\Spec{Spec}
\DeclareMathOperator\Ho{Ho}
\DeclareMathOperator\Aut{Aut}
\DeclareMathOperator\End{End}
\DeclareMathOperator\Hom{Hom}
\DeclareMathOperator\Map{Map}

%% TIKZ ARROWS AND HIGHER CELLS
\makeatletter
\def\slashedarrowfill@#1#2#3#4#5{%
  $\m@th\thickmuskip0mu\medmuskip\thickmuskip\thinmuskip\thickmuskip
   \relax#5#1\mkern-7mu%
   \cleaders\hbox{$#5\mkern-2mu#2\mkern-2mu$}\hfill
   \mathclap{#3}\mathclap{#2}%
   \cleaders\hbox{$#5\mkern-2mu#2\mkern-2mu$}\hfill
   \mkern-7mu#4$%
}

\def\Rightslashedarrowfill@{%
  \slashedarrowfill@\Relbar\Relbar\Mapstochar\Rightarrow}
\newcommand\xslashedRightarrow[2][]{%
  \ext@arrow 0055{\Rightslashedarrowfill@}{#1}{#2}}
\def\hTo{\xslashedRightarrow{}}
\def\hToo{\xslashedRightarrow{\quad}}
\let\xhTo\xslashedRightarrow

\pagestyle{empty}

\newcommand{\Rightthreecell}{\RRightarrow}
\newcommand{\Rtwocell}{\Rightarrow}

\tikzstyle{doubletick}=[-implies, double equal sign distance, postaction={decorate},decoration={markings,mark=at position .5 with {\draw[-] (0,-0.1) -- (0,0.1);}}]

\tikzstyle{darrow}=[-implies, double equal sign distance]

\tikzstyle{doubleeq}=[double equal sign distance]


%% ARROWS
% \to already exists
\newcommand{\too}[1][]{\ensuremath{\overset{#1}{\longrightarrow}}}
\newcommand{\ot}{\ensuremath{\leftarrow}}
\newcommand{\oot}[1][]{\ensuremath{\overset{#1}{\longleftarrow}}}
\let\toot\rightleftarrows
\let\otto\leftrightarrows
\let\Impl\Rightarrow
\let\imp\Rightarrow
\let\toto\rightrightarrows
\let\into\hookrightarrow
\let\xinto\xhookrightarrow
\mdef\we{\overset{\sim}{\longrightarrow}}
\mdef\leftwe{\overset{\sim}{\longleftarrow}}
\let\mono\rightarrowtail
\let\leftmono\leftarrowtail
\let\cof\rightarrowtail
\let\leftcof\leftarrowtail
\let\epi\twoheadrightarrow
\let\leftepi\twoheadleftarrow
\let\fib\twoheadrightarrow
\let\leftfib\twoheadleftarrow
\let\cohto\rightsquigarrow
\let\maps\colon
\newcommand{\spam}{\,:\!}       % \maps for left arrows

\newsavebox{\DDownarrowbox}
\savebox{\DDownarrowbox}{\tikz[scale=1.5]{\draw[-implies,double equal
sign distance] (0,.1) -- (0,-.1); \draw (0,.1) -- (0,-.1);}}
\newcommand{\DDownarrow}{\mathrel{\raisebox{-.2em}{\usebox{\DDownarrowbox}}}}

\newsavebox{\RRightarrowbox}
\savebox{\RRightarrowbox}{\tikz[scale=1.5]{\draw[-implies,double equal
sign distance] (-.1,0) -- (.1,0); \draw (-.1,0) -- (.1,0);}}
\newcommand{\RRightarrow}{\mathrel{\raisebox{.2em}{\usebox{\RRightarrowbox}}}}

%\newsavebox{\Rightslashedarrowbox}
%\savebox{\Rightslashedarrowbox}{\tikz[scale=1.5]{\draw[Rightslashedarrow{}] (-.1,0) -- (1,0);}}
%\newcommand{\Rightslashedarrow}{\mathrel{\raisebox{-.2em}%{\usebox{\Rightslashedarrowbox}}}}


%% EXTENSIBLE ARROWS
\let\xto\xrightarrow
\let\xot\xleftarrow
% See Voss' Mathmode.tex for instructions on how to create new
% extensible arrows.
\def\rightarrowtailfill@{\arrowfill@{\Yright\joinrel\relbar}\relbar\rightarrow}
\newcommand\xrightarrowtail[2][]{\ext@arrow 0055{\rightarrowtailfill@}{#1}{#2}}
\let\xmono\xrightarrowtail
\let\xcof\xrightarrowtail
\def\twoheadrightarrowfill@{\arrowfill@{\relbar\joinrel\relbar}\relbar\twoheadrightarrow}
\newcommand\xtwoheadrightarrow[2][]{\ext@arrow 0055{\twoheadrightarrowfill@}{#1}{#2}}
\let\xepi\xtwoheadrightarrow
\let\xfib\xtwoheadrightarrow
% Let's leave the left-going ones until I need them.

%% EXTENSIBLE SLASHED ARROWS
% Making extensible slashed arrows, by modifying the underlying AMS code.
% Arguments are:
% 1 = arrowhead on the left (\relbar or \Relbar if none)
% 2 = fill character (usually \relbar or \Relbar)
% 3 = slash character (such as \mapstochar or \Mapstochar)
% 4 = arrowhead on the left (\relbar or \Relbar if none)
% 5 = display mode (\displaystyle etc)
\def\slashedarrowfill@#1#2#3#4#5{%
  $\m@th\thickmuskip0mu\medmuskip\thickmuskip\thinmuskip\thickmuskip
   \relax#5#1\mkern-7mu%
   \cleaders\hbox{$#5\mkern-2mu#2\mkern-2mu$}\hfill
   \mathclap{#3}\mathclap{#2}%
   \cleaders\hbox{$#5\mkern-2mu#2\mkern-2mu$}\hfill
   \mkern-7mu#4$%
}
% Here's the idea: \<slashed>arrowfill@ should be a box containing
% some stretchable space that is the "middle of the arrow".  This
% space is created as a "leader" using \cleader<thing>\hfill, which
% fills an \hfill of space with copies of <thing>.  Here instead of
% just one \cleader, we use two, with the slash in between (and an
% extra copy of the filler, to avoid extra space around the slash).
\def\rightslashedarrowfill@{%
  \slashedarrowfill@\relbar\relbar\mapstochar\rightarrow}
\newcommand\xslashedrightarrow[2][]{%
  \ext@arrow 0055{\rightslashedarrowfill@}{#1}{#2}}
\mdef\hto{\xslashedrightarrow{}}
\mdef\htoo{\xslashedrightarrow{\quad}}
\let\xhto\xslashedrightarrow

%% To get a slashed arrow in XYpic, do
% \[\xymatrix{A \ar[r]|-@{|} & B}\]

% ISOMORPHISMS
\def\xiso#1{\mathrel{\mathrlap{\smash{\xto[\smash{\raisebox{1.3mm}{$\scriptstyle\sim$}}]{#1}}}\hphantom{\xto{#1}}}}
\def\toiso{\xto{\smash{\raisebox{-.5mm}{$\scriptstyle\sim$}}}}

% SHADOWS
\def\shvar#1#2{{\ensuremath{%
  \hspace{1mm}\makebox[-1mm]{$#1\langle$}\makebox[0mm]{$#1\langle$}\hspace{1mm}%
  {#2}%
  \makebox[1mm]{$#1\rangle$}\makebox[0mm]{$#1\rangle$}%
}}}
\def\sh{\shvar{}}
\def\scriptsh{\shvar{\scriptstyle}}
\def\bigsh{\shvar{\big}}
\def\Bigsh{\shvar{\Big}}
\def\biggsh{\shvar{\bigg}}
\def\Biggsh{\shvar{\Bigg}}

%HIGHER CELLS



% THEOREM-TYPE ENVIRONMENTS, hacked to
%% (a) number all with the same numbers, and
%% (b) have the right names for autoref
\def\defthm#1#2{%
  \newtheorem{#1}{#2}[section]%
  \expandafter\def\csname #1autorefname\endcsname{#2}%
  \expandafter\let\csname c@#1\endcsname\c@thm}
\newtheorem{thm}{Theorem}[section]
\newcommand{\thmautorefname}{Theorem}
\defthm{cor}{Corollary}
\defthm{prop}{Proposition}
\defthm{lem}{Lemma}
\defthm{sch}{Scholium}
\defthm{assume}{Assumption}
\defthm{claim}{Claim}
\defthm{conj}{Conjecture}
\defthm{hyp}{Hypothesis}
\defthm{fact}{Fact}
\theoremstyle{definition}
\defthm{defn}{Definition}
\defthm{notn}{Notation}
\theoremstyle{remark}
\defthm{rmk}{Remark}
\defthm{eg}{Example}
\defthm{egs}{Examples}
\defthm{ex}{Exercise}
\defthm{ceg}{Counterexample}

% How to get QED symbols inside equations at the end of the statements
% of theorems.  AMS LaTeX knows how to do this inside equations at the
% end of *proofs* with \qedhere, and at the end of the statement of a
% theorem with \qed (meaning no proof will be given), but it can't
% seem to combine the two.  Use this instead.
\def\thmqedhere{\expandafter\csname\csname @currenvir\endcsname @qed\endcsname}

% Number numbered lists as (i), (ii), ...
\renewcommand{\theenumi}{(\roman{enumi})}
\renewcommand{\labelenumi}{\theenumi}

%% Labeling that keeps track of theorem-type names.  Superseded by
%% autoref from hyperref, as above, but we keep this in case we are
%% using a journal style file that is incompatible with hyperref.
% 
% \ifx\SK@label\undefined\let\SK@label\label\fi
% \let\your@thm\@thm
% \def\@thm#1#2#3{\gdef\currthmtype{#3}\your@thm{#1}{#2}{#3}}
% \def\xlabel#1{{\let\your@currentlabel\@currentlabel\def\@currentlabel
% {\currthmtype~\your@currentlabel}
% \SK@label{#1@}}\label{#1}}
% \def\xref#1{\ref{#1@}}

% Also number formulas with the theorem counter
\let\c@equation\c@thm
\numberwithin{equation}{section}

% Only show numbers for equations that are actually referenced (or
% whose tags are specified manually) - uses mathtools.
\mathtoolsset{showonlyrefs,showmanualtags}

%% Fix enumerate spacing with paralist.  This has two parts:
%%   1. enable mixing of "old spacing" lists with those adjusted by paralist
%%   2. allow us to specify a number based on which to adjust the spacing
%% For the first, use \killspacingtrue when you want the spacing
%% adjusted, then \killspacingfalse to turn adjustment off.  For the
%% second, use \maxenum=14 to set the widest number you want the
%% spacing to be calculated with.
\newlength\oldleftmargini       % save old spacing
\newlength\oldleftmarginii
\newlength\oldleftmarginiii
\newlength\oldleftmarginiv
\newlength\oldleftmarginv
\newlength\oldleftmarginvi
\newcount\maxenum
\maxenum=7
\newif\ifkillspacing
\def\@adjust@enum@labelwidth{%
  \advance\@listdepth by 1\relax
  \ifkillspacing                % do the paralist thing
    \csname c@\@enumctr\endcsname\maxenum
    \settowidth{\@tempdima}{%
      \csname label\@enumctr\endcsname\hspace{\labelsep}}%
    \csname leftmargin\romannumeral\@listdepth\endcsname
      \@tempdima
  \else                         % otherwise, reset it
    \csname fixspacing\romannumeral\@listdepth\endcsname
  \fi
  \advance\@listdepth by -1\relax}
% these commands, one for each enum level (I couldn't get a generic
% one to work), test whether oldleftmargin has been set yet, and if
% not, set it from leftmargin; otherwise, they reset leftmargin to
% it.  Just setting oldleftmargin to leftmargin in the preamble
% doesn't seem to work.
\def\fixspacingi{\ifnum\oldleftmargini=0\setlength\oldleftmargini\leftmargini\else\setlength\leftmargini\oldleftmargini\fi}
\def\fixspacingii{\ifnum\oldleftmarginii=0\setlength\oldleftmarginii\leftmarginii\else\setlength\leftmarginii\oldleftmarginii\fi}
\def\fixspacingiii{\ifnum\oldleftmarginiii=0\setlength\oldleftmarginiii\leftmarginiii\else\setlength\leftmarginiii\oldleftmarginiii\fi}
\def\fixspacingiv{\ifnum\oldleftmarginiv=0\setlength\oldleftmarginiv\leftmarginiv\else\setlength\leftmarginiv\oldleftmarginiv\fi}
\def\fixspacingv{\ifnum\oldleftmarginv=0\setlength\oldleftmarginv\leftmarginv\else\setlength\leftmarginv\oldleftmarginv\fi}
\def\fixspacingvi{\ifnum\oldleftmarginvi=0\setlength\oldleftmarginvi\leftmarginvi\else\setlength\leftmarginvi\oldleftmarginvi\fi}

%% Fix paralist references, so that we can refer to (1) instead of
%% just 1.
\def\pl@label#1#2{%
  \edef\pl@the{\noexpand#1{\@enumctr}}%
  \pl@lab\expandafter{\the\pl@lab\csname yourthe\@enumctr\endcsname}%
  \advance\@tempcnta1
  \pl@loop}
\def\@enumlabel@#1[#2]{%
  \@plmylabeltrue
  \@tempcnta0
  \pl@lab{}%
  \let\pl@the\pl@qmark
  \expandafter\pl@loop\@gobble#2\@@@
  \ifnum\@tempcnta=1\else
    \PackageWarning{paralist}{Incorrect label; no or multiple
      counters.\MessageBreak The label is: \@gobble#2}%
  \fi
  \expandafter\edef\csname label\@enumctr\endcsname{\the\pl@lab}%
  \expandafter\edef\csname the\@enumctr\endcsname{\the\pl@lab}%
  \expandafter\let\csname yourthe\@enumctr\endcsname\pl@the
  #1}


% GREEK LETTERS, ETC.
\alwaysmath{alpha}
\alwaysmath{beta}
\alwaysmath{gamma}
\alwaysmath{Gamma}
\alwaysmath{delta}
\alwaysmath{Delta}
\alwaysmath{epsilon}
\mdef\ep{\varepsilon}
\alwaysmath{zeta}
\alwaysmath{eta}
\alwaysmath{theta}
\alwaysmath{Theta}
\alwaysmath{iota}
\alwaysmath{kappa}
\alwaysmath{lambda}
\alwaysmath{Lambda}
\alwaysmath{mu}
\alwaysmath{nu}
\alwaysmath{xi}
\alwaysmath{pi}
\alwaysmath{rho}
\alwaysmath{sigma}
\alwaysmath{Sigma}
\alwaysmath{tau}
\alwaysmath{upsilon}
\alwaysmath{Upsilon}
\alwaysmath{phi}
\alwaysmath{Pi}
\alwaysmath{Phi}
\mdef\ph{\varphi}
\alwaysmath{chi}
\alwaysmath{psi}
\alwaysmath{Psi}
\alwaysmath{omega}
\alwaysmath{Omega}
\let\al\alpha
\let\be\beta
\let\gm\gamma
\let\Gm\Gamma
\let\de\delta
\let\De\Delta
\let\si\sigma
\let\Si\Sigma
\let\om\omega
\let\ka\kappa
\let\la\lambda
\let\La\Lambda
\let\ze\zeta
\let\th\theta
\let\Th\Theta
\let\vth\vartheta

\makeatother

% Tikz styles
\tikzstyle{tickarrow}=[->,postaction={decorate},decoration={markings,mark=at position .5 with {\draw[-] (0,-0.1) -- (0,0.1);}},line width=0.50]

% Local Variables:
% mode: latex
% TeX-master: ""
% End:

\begin{document}

{\small
\begin{equation*}\hspace{-2cm}
\begin{tikzpicture}[xscale=1.75, yscale=1.5]
%%%%Row A
\node (A0) at (-1,4) {$\substack{\tens(\tens \times \transid)\\(\tens \times \transid \times \transid)\\(f \times f \times f \times f)}$};
\node (A1) at (-.5,6.5) {$\substack{\tens(\tens \times \transid)\\(f \tens \times f \times f)}$};
\node (A2) at (1.5,7.5) {$\substack{\tens(f\tens \times f)\\(\tens \times \transid \times \transid)}$};
\node (A3) at (3,8) {$\substack{f\tens(\tens \times \transid)\\( \tens \times \transid \times \transid)}$};
\node (A4) at (4.5,7.5) {$\substack{f\tens(\tens \times \transid)\\(\transid \times \tens \times \transid)}$};
\node (A5) at (6,6.5) {$\substack{f\tens(\transid \times \tens)\\(\transid  \times \tens \times \transid)}$};
\node (A6) at (7,4.5) {$\substack{f\tens(\transid \times \tens)\\(\transid  \times \transid \times \tens)}$};
%%%%%%
\draw[doubleloose] (A0) to node[above, xshift=-16pt]{$\substack{\looseid \looseid \\(\chi \times \looseid \times \looseid)}$} (A1);
\draw[doubleloose] (A1) to node[above, xshift=-16pt]{$\substack{\looseid (\chi \times \looseid) \looseid}$}
(A2);
\draw[doubleloose] (A2) to node[above, xshift=-16pt]{$\substack{\chi \looseid \looseid }$} (A3);
\draw[doubleloose] (A3) to node[above, xshift=16pt]{$\substack{  \looseid \looseid \\ (\alpha \times \looseid )}$} (A4);
\draw[doubleloose] (A4) to node[above, xshift=16pt]{$\substack{\looseid \alpha \looseid }$} (A5);
\draw[doubleloose] (A5) to node[above, xshift=16pt]{$\substack{ \looseid \looseid \\ (\looseid \times \alpha)}$} (A6);
%%%%Row B
\node (B3) at (3,6) {$\substack{\tens(f\tens \times f)\\( \transid \times \tens \times \transid)}$};
%%%%%%
\draw[doubleloose] (A2) to node[below, xshift=-16pt]{$\substack{\looseid \looseid (\alpha \times \looseid) }$} (B3);
\draw[doubleloose] (B3) to node[below, xshift=16pt]{$\substack{  \chi \looseid \looseid}$} (A4);
%%%%Row C
\node (C1) at (1.5,3) {$\substack{\tens(\tens \times \transid)\\(\transid \times \tens \times \transid) \\(f \times f \times f \times f)}$};
\node (C2) at (3,4.5) {$\substack{\tens(\tens \times \transid)\\(f\times f\tens \times f)}$};
%%%%%%
\draw[doubleloose] (A0) to node[above, xshift=10pt]{$\substack{\looseid (\alpha \times \looseid)\\ \looseid}$} (C1);
\draw[doubleloose] (C1) to node[above, xshift=-12]{$\substack{\looseid \looseid \\(\chi \times \looseid \times \looseid)}$} (C2);
\draw[doubleloose] (C2) to node[left]{$\substack{\looseid \\ (\chi \times \looseid) \\ \looseid}$} (B3);
%%%%Row D
\node (D3) at (3.5,3.5) {$\substack{\tens(\transid \times \tens)\\( f \times f\tens \times f)}$};
\node (D4) at (5.5,3.5) {$\substack{\tens(f \times f\tens)\\(\transid \times \tens \times \transid)}$};
%%%%%%
\draw[doubleloose] (C2) to node[right]{$\substack{\alpha \looseid }$} (D3);
\draw[doubleloose] (D3) to node[above]{$\substack{  \looseid (\looseid \times \chi)\\ \looseid }$} (D4);
\draw[doubleloose] (D4) to node[right]{$\substack{\chi \looseid \looseid }$} (A5);
%%%%Row E
\node (E2) at (2,2) {$\substack{\tens( \transid \times \tens)\\(\transid \times \tens \times \transid) \\ (f\times f \times f \times f)}$};
\node (E5) at (6.5,1) {$\substack{\tens(f \times f\tens)\\(\transid   \times \transid \times \tens)}$};
%%%%%%
\draw[doubleloose] (C1) to node[left]{$\substack{\alpha \looseid \looseid}$} (E2);
\draw[doubleloose] (E2) to node[below, xshift=14pt]{$\substack{\looseid \looseid \\ (\looseid \times \chi \times \looseid)}$} (D3);
\draw[doubleloose] (D4) to node[right]{$\substack{\looseid \looseid \\ (\looseid \times \alpha ) }$} (E5);
\draw[doubleloose] (E5) to node[right, xshift=2pt]{$\substack{ \chi \looseid \looseid}$} (A6);
%%%%Row F
\node (F3) at (2,-1) {$\substack{\tens(\transid \times \tens)\\( \transid \times \transid \times \tens) \\ ( f \times f \times f \times f)}$};
\node (F4) at (4,-1) {$\substack{\tens(\transid \times \tens) \\(f \times f \times f\tens)}$};
%%%%%%
\draw[doubleloose] (E2) to node[right]{$\substack{\looseid \\ (\looseid \times \alpha)\\ \looseid}$} (F3);
\draw[doubleloose] (F3) to node[above]{$\substack{  \looseid \looseid \\ (\looseid \times \looseid \times \chi) }$} (F4);
\draw[doubleloose] (F4) to node[above, xshift=-16pt]{$\substack{\looseid (\looseid \times \chi) }$} (E5);
%%%%Row G
\node (G1) at (-1,1) {$\substack{\tens(\transid \times \tens)\\(\tens \times \transid \times \transid) \\(f \times f \times f \times f)}$};
\node (G2) at (0,-.5) {$\substack{\tens( \tens \times \transid)\\(\transid \times \transid \times \tens) \\ (f\times f \times f \times f)}$};
%%%%%%
\draw[doubleloose] (A0) to node[below, xshift=-16pt]{$\substack{\alpha  \looseid  \looseid}$} (G1);
\draw[doubleeq] (G1) to node[above]{} (G2);
\draw[doubleloose] (G2) to node[above]{$\substack{\alpha \looseid \looseid}$} (F3);
%%%%%3-cells
\node at (3,7) {$\DDownarrow \iso $};
\node at (1,6.5) {$\DDownarrow \iso $};
\node at (1,5) {$\DDownarrow \tightid (\omega \times \tightid)$};
\node at (1,4) {$\DDownarrow \iso $};
\node at (4.5,6.75) {$\DDownarrow \iso$};
\node at (4.5,5.5) {$\DDownarrow \omega \tightid$};
\node at (4.5,4.25) {$\DDownarrow \iso$};
\node at (2.5,3) {$\DDownarrow \iso$};
\node at (6.25,3.75) {$\DDownarrow \iso$};
\node at (4.25,3) {$\DDownarrow \iso$};
\node at (4,1.5) {$\DDownarrow  \tightid \omega$};
\node at (4.25,0) {$\DDownarrow \iso$};
\node at (1.75,1.25) {$\DDownarrow \iso$};
\node at (.5,1.55) {$\DDownarrow \pi \tightid $};
\node at (-.25,.5) {$\DDownarrow \iso$};
%%%%%Extra cells
\draw[doubleloose] (A0) to[out=55, in=180] node[above, yshift=5pt]{$\substack{\looseid (S(\omega) \times \looseid) }$} (B3);
\draw[doubleloose] (A0) to[out=15, in=225] node[below,yshift=-5pt]{$\substack{\looseid (T(\omega) \times \looseid)  }$}(B3);
%%%%%Extra cells
\draw[doubleloose] (C2) to[out=55, in=180] node[above, yshift=5pt]{$\substack{S(\omega) \looseid }$} (A5);
\draw[doubleloose] (C2) to[out=15, in=225] node[below,yshift=-5pt]{$\substack{T(\omega) \looseid  }$}(A5);
%%%%%Extra cells
\draw[doubleloose] (A0) to[out=-35, in=105] node[right, yshift=5pt]{$\substack{S(\pi) \\ \looseid }$} (F3);
\draw[doubleloose] (A0) to[out=-75, in=145] node[left,yshift=-5pt]{$\substack{T(\pi) \\ \looseid  }$}(F3);
%%%%%Extra cells
\draw[doubleloose] (E2) to[out=10, in=140] node[above]{$\substack{\looseid (\looseid \times S(\omega) )}$} (E5);
\draw[doubleloose] (E2) to[out=-45, in=190] node[below]{$\substack{\looseid (\looseid \times t(\omega) )}$}(E5);
\end{tikzpicture} \hspace{-2cm}
\end{equation*}}
{\small
\begin{equation}\label{eq:laxfunc1}
=
\end{equation}
\begin{equation*}\hspace{-2cm}
\begin{tikzpicture}[xscale=2.25, yscale=1.5]
%%%%Row A
\node (A0) at (0,5) {$\substack{\tens(\tens \times \transid)\\(\tens \times \transid \times \transid)\\(f \times f \times f \times f)}$};
\node (A1) at (0,6.5) {$\substack{\tens(\tens \times \transid)\\(f \tens \times f \times f)}$};
\node (A2) at (1.5,8) {$\substack{\tens(f\tens \times f)\\(\tens \times \transid \times \transid)}$};
\node (A3) at (3,8) {$\substack{f\tens(\tens \times \transid)\\( \tens \times \transid \times \transid)}$};
\node (A4) at (4.5,8) {$\substack{f\tens(\tens \times \transid)\\(\transid \times \tens \times \transid)}$};
\node (A5) at (6,6.5) {$\substack{f\tens(\transid \times \tens)\\(\transid  \times \tens \times \transid)}$};
\node (A6) at (6,5) {$\substack{f\tens(\transid \times \tens)\\(\transid  \times \transid \times \tens)}$};
%%%%%%
\draw[doubleloose] (A0) to node[above, xshift=-16pt]{$\substack{\looseid \looseid \\(\chi \times \looseid \times \looseid)}$} (A1);
\draw[doubleloose] (A1) to node[above, xshift=-16pt]{$\substack{\looseid (\chi \times \looseid) \looseid}$}
(A2);
\draw[doubleloose] (A2) to node[above, xshift=-16pt]{$\substack{\chi \looseid \looseid }$} (A3);
\draw[doubleloose] (A3) to node[above, xshift=16pt]{$\substack{  \looseid \looseid \\ (\alpha \times \looseid )}$} (A4);
\draw[doubleloose] (A4) to node[above, xshift=16pt]{$\substack{\looseid \alpha \looseid }$} (A5);
\draw[doubleloose] (A5) to node[above, xshift=16pt]{$\substack{ \looseid \looseid \\ (\looseid \times \alpha)}$} (A6);
%%%%Row B
\node (B4) at (3,6) {$\substack{f\tens(\transid \times \tens)\\(\tens \times \transid \times  \transid)}$};
\node (B5) at (4,5) {$\substack{f\tens(\tens \times \transid)\\(\transid  \times \transid \times \tens )}$};
%%%%%%
\draw[doubleloose] (A3) to node[above, xshift=16pt]{$\substack{  \looseid \alpha \looseid}$} (B4);
\draw[doubleeq] (B4) to  (B5);
\draw[doubleloose] (B5) to node[above, xshift=16pt]{$\substack{\looseid \alpha \looseid}$} (A6);
%%%%Row C
\node (C2) at (1.5,4.5) {$\substack{\tens(\transid \times \tens)\\(f\tens \times f \times f)}$};
\node (C3) at (2.5,5) {$\substack{\tens(f \times f\tens)\\(\tens \times \transid \times  \transid)}$};
%%%%%%
\draw[doubleloose] (A1) to node[above]{$\substack{\alpha \looseid}$}
(C2);
\draw[doubleloose] (C2) to node[above,  xshift=-16pt]{$\substack{\looseid (\looseid \times \chi)\\ \looseid}$} (C3);
\draw[doubleloose] (C3) to node[above]{$\substack{  \chi \looseid}$} (B4);
%%%%Row G
\node (G1) at (1,2) {$\substack{\tens(\transid \times \tens)\\( \tens \times \transid \times \transid)\\ (f \times f \times f \times f)}$};
%%%%%%
\draw[doubleloose] (A0) to node[above]{$\substack{\alpha \looseid \looseid}$} (G1);
\draw[doubleloose] (G1) to node[above, xshift=-16pt, yshift=-4pt]{$\substack{ \looseid \looseid \\(\chi \times \looseid \times \looseid)}$}
(C2);
%%%%Row D
\node (D2) at (2,1) {$\substack{\tens (\transid \times f \tens ) \\ (\tens \times \transid \times \transid ) \\ (  f \times f \times \transid \times \transid)}$};
%%%%%%%%%
\draw[doubleloose] (G1) to node[above]{$\substack{  \looseid \\(\looseid \times \chi) \looseid}$}
(D2);
\draw[doubleloose] (D2) to node[below,  xshift=16pt]{$\substack{\looseid \looseid \\(\chi \times \looseid \times \looseid)}$} (C3);
%%%%Row E
\node (E3) at (3,1) {$\substack{\tens (\tens \times \transid) \\ (f \times f \times f\tens)}$};
\node (E4) at (4,1) {$\substack{\tens (\tens \times \transid) \\ (f \times f \times f\tens)}$};
\node (E5) at (5,1) {$\substack{\tens(f \times f\tens)\\(\transid  \times \transid \times \tens)}$};
%%%%%%
\draw[doubleeq] (D2) to (E3);
\draw[doubleloose] (E3) to node[above]{$\substack{\alpha \looseid \looseid}$} (E4);
\draw[doubleloose] (E4) to node[above]{$\substack{\looseid (\looseid \times \chi)\\ \looseid}$}  (E5);
\draw[doubleloose] (E5) to node[above]{$\substack{\chi \looseid \looseid}$} (A6);
%%%%Row F
\node (F2) at (2,0) {$\substack{\tens(\tens \times \transid)\\(\transid \times \transid \times \tens) \\(f\times f \times f \times f) }$};
\node (F3) at (3,0) {$\substack{\tens(\transid \times \tens)\\(\transid \times \transid \times \tens) \\ (f\times f \times f \times f)  }$};
%%%%%%
\draw[doubleeq] (G1) to (F2);
\draw[doubleloose] (F2) to node[above]{$\substack{\alpha \looseid \looseid}$} (F3);
\draw[doubleloose] (F3) to node[above]{$\substack{\looseid \looseid \\(  \looseid \times \looseid \times \chi)}$} (E4);
%%%% 3-cells
\node at (4.5,6.5) {$\DDownarrow \tightid \pi $};
\node at (1.5,6) {$\DDownarrow \omega \tightid $};
\node at (.5, 5) {$\DDownarrow \iso$};
\node at (2, 3) {$\DDownarrow \iso$};
\node at (4, 3) {$\DDownarrow \omega \tightid$};
\node at (2.5, .5) {$\DDownarrow \iso $};
%%%%%Extra arrows
\draw[doubleloose] (A1) to[out=25, in=145 ] node[above] {$\substack{S(\omega) \looseid}$} (B4);
\draw[doubleloose] (A1) to[out=-35, in=205 ] node[below] {$\substack{T(\omega) \looseid}$} (B4);
%%%%%Extra arrows
\draw[doubleloose] (A3) to[out=-15, in=125 ] node[above, xshift=10] {$\substack{\looseid S(\pi)}$} (A6);
\draw[doubleloose] (A3) to[out=-60, in=160 ] node[below, xshift=-10] {$\substack{\looseid T(\pi) }$} (A6);
%%%%%Extra arrows
\draw[doubleloose] (D2) to[out=60, in=200 ] node[above, xshift=-10] {$\substack{ S(\omega) \looseid}$} (A6);
\draw[doubleloose] (D2) to[out=15, in=240 ] node[below, xshift=10] {$\substack{T(\omega) \looseid }$} (A6);
\end{tikzpicture} \hspace{-2cm}
\end{equation*}}
\end{document}  \newpage
%
\documentclass[12pt]{ociamthesis}
\usepackage{tikz}
\usepackage{amsmath}
\usepackage{rotating}

\usepackage{amssymb,amsmath,stmaryrd,txfonts,mathrsfs,amsthm}
\usepackage[all,2cell]{xy}
\usepackage[neveradjust]{paralist}
\usepackage{hyperref}
\usepackage{mathtools}
\usepackage{tikz}
\usetikzlibrary{trees}
\usetikzlibrary{topaths}
\usetikzlibrary{decorations}
\usetikzlibrary{decorations.pathreplacing}
\usetikzlibrary{decorations.pathmorphing}
\usetikzlibrary{decorations.markings}
\usetikzlibrary{matrix,backgrounds,folding}
\usetikzlibrary{chains,scopes,positioning,fit}
\usetikzlibrary{arrows,shadows}
\usetikzlibrary{calc} 
\usetikzlibrary{chains}
\usetikzlibrary{shapes,shapes.geometric,shapes.misc}
\usepackage{smbicat}


\makeatletter
\let\ea\expandafter

%% Defining commands that are always in math mode.
\def\mdef#1#2{\ea\ea\ea\gdef\ea\ea\noexpand#1\ea{\ea\ensuremath\ea{#2}}}
\def\alwaysmath#1{\ea\ea\ea\global\ea\ea\ea\let\ea\ea\csname your@#1\endcsname\csname #1\endcsname
  \ea\def\csname #1\endcsname{\ensuremath{\csname your@#1\endcsname}}}

% Script letters
\newcommand{\sA}{\ensuremath{\mathscr{A}}}
\newcommand{\sB}{\ensuremath{\mathscr{B}}}
\newcommand{\sC}{\ensuremath{\mathscr{C}}}
\newcommand{\sD}{\ensuremath{\mathscr{D}}}
\newcommand{\sE}{\ensuremath{\mathscr{E}}}
\newcommand{\sF}{\ensuremath{\mathscr{F}}}
\newcommand{\sG}{\ensuremath{\mathscr{G}}}
\newcommand{\sH}{\ensuremath{\mathscr{H}}}
\newcommand{\sI}{\ensuremath{\mathscr{I}}}
\newcommand{\sJ}{\ensuremath{\mathscr{J}}}
\newcommand{\sK}{\ensuremath{\mathscr{K}}}
\newcommand{\sL}{\ensuremath{\mathscr{L}}}
\newcommand{\sM}{\ensuremath{\mathscr{M}}}
\newcommand{\sN}{\ensuremath{\mathscr{N}}}
\newcommand{\sO}{\ensuremath{\mathscr{O}}}
\newcommand{\sP}{\ensuremath{\mathscr{P}}}
\newcommand{\sQ}{\ensuremath{\mathscr{Q}}}
\newcommand{\sR}{\ensuremath{\mathscr{R}}}
\newcommand{\sS}{\ensuremath{\mathscr{S}}}
\newcommand{\sT}{\ensuremath{\mathscr{T}}}
\newcommand{\sU}{\ensuremath{\mathscr{U}}}
\newcommand{\sV}{\ensuremath{\mathscr{V}}}
\newcommand{\sW}{\ensuremath{\mathscr{W}}}
\newcommand{\sX}{\ensuremath{\mathscr{X}}}
\newcommand{\sY}{\ensuremath{\mathscr{Y}}}
\newcommand{\sZ}{\ensuremath{\mathscr{Z}}}

% Calligraphic letters
\newcommand{\cA}{\ensuremath{\mathcal{A}}}
\newcommand{\cB}{\ensuremath{\mathcal{B}}}
\newcommand{\cC}{\ensuremath{\mathcal{C}}}
\newcommand{\cD}{\ensuremath{\mathcal{D}}}
\newcommand{\cE}{\ensuremath{\mathcal{E}}}
\newcommand{\cF}{\ensuremath{\mathcal{F}}}
\newcommand{\cG}{\ensuremath{\mathcal{G}}}
\newcommand{\cH}{\ensuremath{\mathcal{H}}}
\newcommand{\cI}{\ensuremath{\mathcal{I}}}
\newcommand{\cJ}{\ensuremath{\mathcal{J}}}
\newcommand{\cK}{\ensuremath{\mathcal{K}}}
\newcommand{\cL}{\ensuremath{\mathcal{L}}}
\newcommand{\cM}{\ensuremath{\mathcal{M}}}
\newcommand{\cN}{\ensuremath{\mathcal{N}}}
\newcommand{\cO}{\ensuremath{\mathcal{O}}}
\newcommand{\cP}{\ensuremath{\mathcal{P}}}
\newcommand{\cQ}{\ensuremath{\mathcal{Q}}}
\newcommand{\cR}{\ensuremath{\mathcal{R}}}
\newcommand{\cS}{\ensuremath{\mathcal{S}}}
\newcommand{\cT}{\ensuremath{\mathcal{T}}}
\newcommand{\cU}{\ensuremath{\mathcal{U}}}
\newcommand{\cV}{\ensuremath{\mathcal{V}}}
\newcommand{\cW}{\ensuremath{\mathcal{W}}}
\newcommand{\cX}{\ensuremath{\mathcal{X}}}
\newcommand{\cY}{\ensuremath{\mathcal{Y}}}
\newcommand{\cZ}{\ensuremath{\mathcal{Z}}}

% blackboard bold letters
\newcommand{\lA}{\ensuremath{\mathbb{A}}}
\newcommand{\lB}{\ensuremath{\mathbb{B}}}
\newcommand{\lC}{\ensuremath{\mathbb{C}}}
\newcommand{\lD}{\ensuremath{\mathbb{D}}}
\newcommand{\lE}{\ensuremath{\mathbb{E}}}
\newcommand{\lF}{\ensuremath{\mathbb{F}}}
\newcommand{\lG}{\ensuremath{\mathbb{G}}}
\newcommand{\lH}{\ensuremath{\mathbb{H}}}
\newcommand{\lI}{\ensuremath{\mathbb{I}}}
\newcommand{\lJ}{\ensuremath{\mathbb{J}}}
\newcommand{\lK}{\ensuremath{\mathbb{K}}}
\newcommand{\lL}{\ensuremath{\mathbb{L}}}
\newcommand{\lM}{\ensuremath{\mathbb{M}}}
\newcommand{\lN}{\ensuremath{\mathbb{N}}}
\newcommand{\lO}{\ensuremath{\mathbb{O}}}
\newcommand{\lP}{\ensuremath{\mathbb{P}}}
\newcommand{\lQ}{\ensuremath{\mathbb{Q}}}
\newcommand{\lR}{\ensuremath{\mathbb{R}}}
\newcommand{\lS}{\ensuremath{\mathbb{S}}}
\newcommand{\lT}{\ensuremath{\mathbb{T}}}
\newcommand{\lU}{\ensuremath{\mathbb{U}}}
\newcommand{\lV}{\ensuremath{\mathbb{V}}}
\newcommand{\lW}{\ensuremath{\mathbb{W}}}
\newcommand{\lX}{\ensuremath{\mathbb{X}}}
\newcommand{\lY}{\ensuremath{\mathbb{Y}}}
\newcommand{\lZ}{\ensuremath{\mathbb{Z}}}

% bold letters
\newcommand{\bA}{\ensuremath{\mathbf{A}}}
\newcommand{\bB}{\ensuremath{\mathbf{B}}}
\newcommand{\bC}{\ensuremath{\mathbf{C}}}
\newcommand{\bD}{\ensuremath{\mathbf{D}}}
\newcommand{\bE}{\ensuremath{\mathbf{E}}}
\newcommand{\bF}{\ensuremath{\mathbf{F}}}
\newcommand{\bG}{\ensuremath{\mathbf{G}}}
\newcommand{\bH}{\ensuremath{\mathbf{H}}}
\newcommand{\bI}{\ensuremath{\mathbf{I}}}
\newcommand{\bJ}{\ensuremath{\mathbf{J}}}
\newcommand{\bK}{\ensuremath{\mathbf{K}}}
\newcommand{\bL}{\ensuremath{\mathbf{L}}}
\newcommand{\bM}{\ensuremath{\mathbf{M}}}
\newcommand{\bN}{\ensuremath{\mathbf{N}}}
\newcommand{\bO}{\ensuremath{\mathbf{O}}}
\newcommand{\bP}{\ensuremath{\mathbf{P}}}
\newcommand{\bQ}{\ensuremath{\mathbf{Q}}}
\newcommand{\bR}{\ensuremath{\mathbf{R}}}
\newcommand{\bS}{\ensuremath{\mathbf{S}}}
\newcommand{\bT}{\ensuremath{\mathbf{T}}}
\newcommand{\bU}{\ensuremath{\mathbf{U}}}
\newcommand{\bV}{\ensuremath{\mathbf{V}}}
\newcommand{\bW}{\ensuremath{\mathbf{W}}}
\newcommand{\bX}{\ensuremath{\mathbf{X}}}
\newcommand{\bY}{\ensuremath{\mathbf{Y}}}
\newcommand{\bZ}{\ensuremath{\mathbf{Z}}}

% fraktur letters
\newcommand{\fa}{\ensuremath{\mathfrak{a}}}
\newcommand{\fb}{\ensuremath{\mathfrak{b}}}
\newcommand{\fc}{\ensuremath{\mathfrak{c}}}
\newcommand{\fd}{\ensuremath{\mathfrak{d}}}
\newcommand{\fe}{\ensuremath{\mathfrak{e}}}
\newcommand{\ff}{\ensuremath{\mathfrak{f}}}
\newcommand{\fg}{\ensuremath{\mathfrak{g}}}
\newcommand{\fh}{\ensuremath{\mathfrak{h}}}
\newcommand{\fj}{\ensuremath{\mathfrak{j}}}
\newcommand{\fk}{\ensuremath{\mathfrak{k}}}
\newcommand{\fl}{\ensuremath{\mathfrak{l}}}
\newcommand{\fm}{\ensuremath{\mathfrak{m}}}
\newcommand{\fn}{\ensuremath{\mathfrak{n}}}
\newcommand{\fo}{\ensuremath{\mathfrak{o}}}
\newcommand{\fp}{\ensuremath{\mathfrak{p}}}
\newcommand{\fq}{\ensuremath{\mathfrak{q}}}
\newcommand{\fr}{\ensuremath{\mathfrak{r}}}
\newcommand{\fs}{\ensuremath{\mathfrak{s}}}
\newcommand{\ft}{\ensuremath{\mathfrak{t}}}
\newcommand{\fu}{\ensuremath{\mathfrak{u}}}
\newcommand{\fv}{\ensuremath{\mathfrak{v}}}
\newcommand{\fw}{\ensuremath{\mathfrak{w}}}
\newcommand{\fx}{\ensuremath{\mathfrak{x}}}
\newcommand{\fy}{\ensuremath{\mathfrak{y}}}
\newcommand{\fz}{\ensuremath{\mathfrak{z}}}

% fraktur letters
\newcommand{\fA}{\ensuremath{\mathfrak{A}}}
\newcommand{\fB}{\ensuremath{\mathfrak{B}}}
\newcommand{\fC}{\ensuremath{\mathfrak{C}}}

\mdef\fahat{\hat{\fa}}

% Underline letters
\newcommand{\uA}{\ensuremath{\underline{A}}}
\newcommand{\uB}{\ensuremath{\underline{B}}}
\newcommand{\uC}{\ensuremath{\underline{C}}}
\newcommand{\uD}{\ensuremath{\underline{D}}}
\newcommand{\uE}{\ensuremath{\underline{E}}}
\newcommand{\uF}{\ensuremath{\underline{F}}}
\newcommand{\uG}{\ensuremath{\underline{G}}}
\newcommand{\uH}{\ensuremath{\underline{H}}}
\newcommand{\uI}{\ensuremath{\underline{I}}}
\newcommand{\uJ}{\ensuremath{\underline{J}}}
\newcommand{\uK}{\ensuremath{\underline{K}}}
\newcommand{\uL}{\ensuremath{\underline{L}}}
\newcommand{\uM}{\ensuremath{\underline{M}}}
\newcommand{\uN}{\ensuremath{\underline{N}}}
\newcommand{\uO}{\ensuremath{\underline{O}}}
\newcommand{\uP}{\ensuremath{\underline{P}}}
\newcommand{\uQ}{\ensuremath{\underline{Q}}}
\newcommand{\uR}{\ensuremath{\underline{R}}}
\newcommand{\uS}{\ensuremath{\underline{S}}}
\newcommand{\uT}{\ensuremath{\underline{T}}}
\newcommand{\uU}{\ensuremath{\underline{U}}}
\newcommand{\uV}{\ensuremath{\underline{V}}}
\newcommand{\uW}{\ensuremath{\underline{W}}}
\newcommand{\uX}{\ensuremath{\underline{X}}}
\newcommand{\uY}{\ensuremath{\underline{Y}}}
\newcommand{\uZ}{\ensuremath{\underline{Z}}}

% bars
\newcommand{\Abar}{\ensuremath{\overline{A}}}
\newcommand{\Bbar}{\ensuremath{\overline{B}}}
\newcommand{\Cbar}{\ensuremath{\overline{C}}}
\newcommand{\Dbar}{\ensuremath{\overline{D}}}
\newcommand{\Ebar}{\ensuremath{\overline{E}}}
\newcommand{\Fbar}{\ensuremath{\overline{F}}}
\newcommand{\Gbar}{\ensuremath{\overline{G}}}
\newcommand{\Hbar}{\ensuremath{\overline{H}}}
\newcommand{\Ibar}{\ensuremath{\overline{I}}}
\newcommand{\Jbar}{\ensuremath{\overline{J}}}
\newcommand{\Kbar}{\ensuremath{\overline{K}}}
\newcommand{\Lbar}{\ensuremath{\overline{L}}}
\newcommand{\Mbar}{\ensuremath{\overline{M}}}
\newcommand{\Nbar}{\ensuremath{\overline{N}}}
\newcommand{\Obar}{\ensuremath{\overline{O}}}
\newcommand{\Pbar}{\ensuremath{\overline{P}}}
\newcommand{\Qbar}{\ensuremath{\overline{Q}}}
\newcommand{\Rbar}{\ensuremath{\overline{R}}}
\newcommand{\Sbar}{\ensuremath{\overline{S}}}
\newcommand{\Tbar}{\ensuremath{\overline{T}}}
\newcommand{\Ubar}{\ensuremath{\overline{U}}}
\newcommand{\Vbar}{\ensuremath{\overline{V}}}
\newcommand{\Wbar}{\ensuremath{\overline{W}}}
\newcommand{\Xbar}{\ensuremath{\overline{X}}}
\newcommand{\Ybar}{\ensuremath{\overline{Y}}}
\newcommand{\Zbar}{\ensuremath{\overline{Z}}}
\newcommand{\abar}{\ensuremath{\overline{a}}}
\newcommand{\bbar}{\ensuremath{\overline{b}}}
\newcommand{\cbar}{\ensuremath{\overline{c}}}
\newcommand{\dbar}{\ensuremath{\overline{d}}}
\newcommand{\ebar}{\ensuremath{\overline{e}}}
\newcommand{\fbar}{\ensuremath{\overline{f}}}
\newcommand{\gbar}{\ensuremath{\overline{g}}}
%\newcommand{\hbar}{\ensuremath{\overline{h}}} % whoops, \hbar means something else!
\newcommand{\ibar}{\ensuremath{\overline{\imath}}}
\newcommand{\jbar}{\ensuremath{\overline{\jmath}}}
\newcommand{\kbar}{\ensuremath{\overline{k}}}
\newcommand{\lbar}{\ensuremath{\overline{l}}}
\newcommand{\mbar}{\ensuremath{\overline{m}}}
\newcommand{\nbar}{\ensuremath{\overline{n}}}
%\newcommand{\obar}{\ensuremath{\overline{o}}}
\newcommand{\pbar}{\ensuremath{\overline{p}}}
\newcommand{\qbar}{\ensuremath{\overline{q}}}
\newcommand{\rbar}{\ensuremath{\overline{r}}}
\newcommand{\sbar}{\ensuremath{\overline{s}}}
\newcommand{\tbar}{\ensuremath{\overline{t}}}
\newcommand{\ubar}{\ensuremath{\overline{u}}}
\newcommand{\vbar}{\ensuremath{\overline{v}}}
\newcommand{\wbar}{\ensuremath{\overline{w}}}
\newcommand{\xbar}{\ensuremath{\overline{x}}}
\newcommand{\ybar}{\ensuremath{\overline{y}}}
\newcommand{\zbar}{\ensuremath{\overline{z}}}

% tildes
\newcommand{\Atil}{\ensuremath{\widetilde{A}}}
\newcommand{\Btil}{\ensuremath{\widetilde{B}}}
\newcommand{\Ctil}{\ensuremath{\widetilde{C}}}
\newcommand{\Dtil}{\ensuremath{\widetilde{D}}}
\newcommand{\Etil}{\ensuremath{\widetilde{E}}}
\newcommand{\Ftil}{\ensuremath{\widetilde{F}}}
\newcommand{\Gtil}{\ensuremath{\widetilde{G}}}
\newcommand{\Htil}{\ensuremath{\widetilde{H}}}
\newcommand{\Itil}{\ensuremath{\widetilde{I}}}
\newcommand{\Jtil}{\ensuremath{\widetilde{J}}}
\newcommand{\Ktil}{\ensuremath{\widetilde{K}}}
\newcommand{\Ltil}{\ensuremath{\widetilde{L}}}
\newcommand{\Mtil}{\ensuremath{\widetilde{M}}}
\newcommand{\Ntil}{\ensuremath{\widetilde{N}}}
\newcommand{\Otil}{\ensuremath{\widetilde{O}}}
\newcommand{\Ptil}{\ensuremath{\widetilde{P}}}
\newcommand{\Qtil}{\ensuremath{\widetilde{Q}}}
\newcommand{\Rtil}{\ensuremath{\widetilde{R}}}
\newcommand{\Stil}{\ensuremath{\widetilde{S}}}
\newcommand{\Ttil}{\ensuremath{\widetilde{T}}}
\newcommand{\Util}{\ensuremath{\widetilde{U}}}
\newcommand{\Vtil}{\ensuremath{\widetilde{V}}}
\newcommand{\Wtil}{\ensuremath{\widetilde{W}}}
\newcommand{\Xtil}{\ensuremath{\widetilde{X}}}
\newcommand{\Ytil}{\ensuremath{\widetilde{Y}}}
\newcommand{\Ztil}{\ensuremath{\widetilde{Z}}}
\newcommand{\atil}{\ensuremath{\widetilde{a}}}
\newcommand{\btil}{\ensuremath{\widetilde{b}}}
\newcommand{\ctil}{\ensuremath{\widetilde{c}}}
\newcommand{\dtil}{\ensuremath{\widetilde{d}}}
\newcommand{\etil}{\ensuremath{\widetilde{e}}}
\newcommand{\ftil}{\ensuremath{\widetilde{f}}}
\newcommand{\gtil}{\ensuremath{\widetilde{g}}}
\newcommand{\htil}{\ensuremath{\widetilde{h}}}
\newcommand{\itil}{\ensuremath{\widetilde{\imath}}}
\newcommand{\jtil}{\ensuremath{\widetilde{\jmath}}}
\newcommand{\ktil}{\ensuremath{\widetilde{k}}}
\newcommand{\ltil}{\ensuremath{\widetilde{l}}}
\newcommand{\mtil}{\ensuremath{\widetilde{m}}}
\newcommand{\ntil}{\ensuremath{\widetilde{n}}}
\newcommand{\otil}{\ensuremath{\widetilde{o}}}
\newcommand{\ptil}{\ensuremath{\widetilde{p}}}
\newcommand{\qtil}{\ensuremath{\widetilde{q}}}
\newcommand{\rtil}{\ensuremath{\widetilde{r}}}
\newcommand{\stil}{\ensuremath{\widetilde{s}}}
\newcommand{\ttil}{\ensuremath{\widetilde{t}}}
\newcommand{\util}{\ensuremath{\widetilde{u}}}
\newcommand{\vtil}{\ensuremath{\widetilde{v}}}
\newcommand{\wtil}{\ensuremath{\widetilde{w}}}
\newcommand{\xtil}{\ensuremath{\widetilde{x}}}
\newcommand{\ytil}{\ensuremath{\widetilde{y}}}
\newcommand{\ztil}{\ensuremath{\widetilde{z}}}

% Hats
\newcommand{\Ahat}{\ensuremath{\widehat{A}}}
\newcommand{\Bhat}{\ensuremath{\widehat{B}}}
\newcommand{\Chat}{\ensuremath{\widehat{C}}}
\newcommand{\Dhat}{\ensuremath{\widehat{D}}}
\newcommand{\Ehat}{\ensuremath{\widehat{E}}}
\newcommand{\Fhat}{\ensuremath{\widehat{F}}}
\newcommand{\Ghat}{\ensuremath{\widehat{G}}}
\newcommand{\Hhat}{\ensuremath{\widehat{H}}}
\newcommand{\Ihat}{\ensuremath{\widehat{I}}}
\newcommand{\Jhat}{\ensuremath{\widehat{J}}}
\newcommand{\Khat}{\ensuremath{\widehat{K}}}
\newcommand{\Lhat}{\ensuremath{\widehat{L}}}
\newcommand{\Mhat}{\ensuremath{\widehat{M}}}
\newcommand{\Nhat}{\ensuremath{\widehat{N}}}
\newcommand{\Ohat}{\ensuremath{\widehat{O}}}
\newcommand{\Phat}{\ensuremath{\widehat{P}}}
\newcommand{\Qhat}{\ensuremath{\widehat{Q}}}
\newcommand{\Rhat}{\ensuremath{\widehat{R}}}
\newcommand{\Shat}{\ensuremath{\widehat{S}}}
\newcommand{\That}{\ensuremath{\widehat{T}}}
\newcommand{\Uhat}{\ensuremath{\widehat{U}}}
\newcommand{\Vhat}{\ensuremath{\widehat{V}}}
\newcommand{\What}{\ensuremath{\widehat{W}}}
\newcommand{\Xhat}{\ensuremath{\widehat{X}}}
\newcommand{\Yhat}{\ensuremath{\widehat{Y}}}
\newcommand{\Zhat}{\ensuremath{\widehat{Z}}}
\newcommand{\ahat}{\ensuremath{\hat{a}}}
\newcommand{\bhat}{\ensuremath{\hat{b}}}
\newcommand{\chat}{\ensuremath{\hat{c}}}
\newcommand{\dhat}{\ensuremath{\hat{d}}}
\newcommand{\ehat}{\ensuremath{\hat{e}}}
\newcommand{\fhat}{\ensuremath{\hat{f}}}
\newcommand{\ghat}{\ensuremath{\hat{g}}}
\newcommand{\hhat}{\ensuremath{\hat{h}}}
\newcommand{\ihat}{\ensuremath{\hat{\imath}}}
\newcommand{\jhat}{\ensuremath{\hat{\jmath}}}
\newcommand{\khat}{\ensuremath{\hat{k}}}
\newcommand{\lhat}{\ensuremath{\hat{l}}}
\newcommand{\mhat}{\ensuremath{\hat{m}}}
\newcommand{\nhat}{\ensuremath{\hat{n}}}
\newcommand{\ohat}{\ensuremath{\hat{o}}}
\newcommand{\phat}{\ensuremath{\hat{p}}}
\newcommand{\qhat}{\ensuremath{\hat{q}}}
\newcommand{\rhat}{\ensuremath{\hat{r}}}
\newcommand{\shat}{\ensuremath{\hat{s}}}
\newcommand{\that}{\ensuremath{\hat{t}}}
\newcommand{\uhat}{\ensuremath{\hat{u}}}
\newcommand{\vhat}{\ensuremath{\hat{v}}}
\newcommand{\what}{\ensuremath{\hat{w}}}
\newcommand{\xhat}{\ensuremath{\hat{x}}}
\newcommand{\yhat}{\ensuremath{\hat{y}}}
\newcommand{\zhat}{\ensuremath{\hat{z}}}

%% FONTS AND DECORATION FOR GREEK LETTERS

%% the package `mathbbol' gives us blackboard bold greek and numbers,
%% but it does it by redefining \mathbb to use a different font, so that
%% all the other \mathbb letters look different too.  Here we import the
%% font with bb greek and numbers, but assign it a different name,
%% \mathbbb, so as not to replace the usual one.
\DeclareSymbolFont{bbold}{U}{bbold}{m}{n}
\DeclareSymbolFontAlphabet{\mathbbb}{bbold}
\newcommand{\bbDelta}{\ensuremath{\mathbbb{\Delta}}}
\newcommand{\bbone}{\ensuremath{\mathbbb{1}}}
\newcommand{\bbtwo}{\ensuremath{\mathbbb{2}}}
\newcommand{\bbthree}{\ensuremath{\mathbbb{3}}}

% greek with bars
\newcommand{\albar}{\ensuremath{\overline{\alpha}}}
\newcommand{\bebar}{\ensuremath{\overline{\beta}}}
\newcommand{\gmbar}{\ensuremath{\overline{\gamma}}}
\newcommand{\debar}{\ensuremath{\overline{\delta}}}
\newcommand{\phibar}{\ensuremath{\overline{\varphi}}}
\newcommand{\psibar}{\ensuremath{\overline{\psi}}}
\newcommand{\xibar}{\ensuremath{\overline{\xi}}}
\newcommand{\ombar}{\ensuremath{\overline{\omega}}}

% greek with hats
\newcommand{\alhat}{\ensuremath{\hat{\alpha}}}
\newcommand{\behat}{\ensuremath{\hat{\beta}}}
\newcommand{\gmhat}{\ensuremath{\hat{\gamma}}}
\newcommand{\dehat}{\ensuremath{\hat{\delta}}}

% greek with checks
\newcommand{\alchk}{\ensuremath{\check{\alpha}}}
\newcommand{\bechk}{\ensuremath{\check{\beta}}}
\newcommand{\gmchk}{\ensuremath{\check{\gamma}}}
\newcommand{\dechk}{\ensuremath{\check{\delta}}}

% greek with tildes
\newcommand{\altil}{\ensuremath{\widetilde{\alpha}}}
\newcommand{\betil}{\ensuremath{\widetilde{\beta}}}
\newcommand{\gmtil}{\ensuremath{\widetilde{\gamma}}}
\newcommand{\phitil}{\ensuremath{\widetilde{\varphi}}}
\newcommand{\psitil}{\ensuremath{\widetilde{\psi}}}
\newcommand{\xitil}{\ensuremath{\widetilde{\xi}}}
\newcommand{\omtil}{\ensuremath{\widetilde{\omega}}}

% MISCELLANEOUS SYMBOLS
\mdef\del{\partial}
\mdef\delbar{\overline{\partial}}
\let\sm\wedge
\newcommand{\dd}[1]{\ensuremath{\frac{\partial}{\partial {#1}}}}
\newcommand{\inv}{^{-1}}
\newcommand{\dual}{^{\vee}}
\mdef\hf{\textstyle\frac{1}{2}}
\mdef\thrd{\textstyle\frac{1}{3}}
\mdef\qtr{\textstyle\frac{1}{4}}
\let\meet\wedge
\let\join\vee
\let\dn\downarrow
\newcommand{\op}{^{\mathit{op}}}
\newcommand{\co}{^{\mathit{co}}}
\newcommand{\coop}{^{\mathit{coop}}}
\let\adj\dashv
\SelectTips{cm}{}
\newdir{ >}{{}*!/-10pt/@{>}}    % extra spacing for tail arrows in XYpic
\newcommand{\pushoutcorner}[1][dr]{\save*!/#1+1.2pc/#1:(1,-1)@^{|-}\restore}
\newcommand{\pullbackcorner}[1][dr]{\save*!/#1-1.2pc/#1:(-1,1)@^{|-}\restore}
\let\iso\cong
\let\eqv\simeq
\let\cng\equiv
\mdef\Id{\mathrm{Id}}
\mdef\id{\mathrm{id}}
\alwaysmath{ell}
\alwaysmath{infty}
\alwaysmath{odot}
\def\frc#1/#2.{\frac{#1}{#2}}   % \frc x^2+1 / x^2-1 .
\mdef\ten{\mathrel{\otimes}}
\mdef\bigten{\bigotimes}
\mdef\sqten{\mathrel{\boxtimes}}
\def\pow(#1,#2){\mathop{\pitchfork}(#1,#2)} % powers and
\def\cpw{\mathop{\odot}}                    % copowers
\newcommand{\mathid}{\mbox{id}}
\newcommand{\cat}[1]{\ensuremath{\mathbf{#1}}}


%% OPERATORS
\DeclareMathOperator\lan{Lan}
\DeclareMathOperator\ran{Ran}
\DeclareMathOperator\colim{colim}
\DeclareMathOperator\coeq{coeq}
\DeclareMathOperator\eq{eq}
\DeclareMathOperator\Tot{Tot}
\DeclareMathOperator\cosk{cosk}
\DeclareMathOperator\sk{sk}
\DeclareMathOperator\im{im}
\DeclareMathOperator\Spec{Spec}
\DeclareMathOperator\Ho{Ho}
\DeclareMathOperator\Aut{Aut}
\DeclareMathOperator\End{End}
\DeclareMathOperator\Hom{Hom}
\DeclareMathOperator\Map{Map}

%% TIKZ ARROWS AND HIGHER CELLS
\makeatletter
\def\slashedarrowfill@#1#2#3#4#5{%
  $\m@th\thickmuskip0mu\medmuskip\thickmuskip\thinmuskip\thickmuskip
   \relax#5#1\mkern-7mu%
   \cleaders\hbox{$#5\mkern-2mu#2\mkern-2mu$}\hfill
   \mathclap{#3}\mathclap{#2}%
   \cleaders\hbox{$#5\mkern-2mu#2\mkern-2mu$}\hfill
   \mkern-7mu#4$%
}

\def\Rightslashedarrowfill@{%
  \slashedarrowfill@\Relbar\Relbar\Mapstochar\Rightarrow}
\newcommand\xslashedRightarrow[2][]{%
  \ext@arrow 0055{\Rightslashedarrowfill@}{#1}{#2}}
\def\hTo{\xslashedRightarrow{}}
\def\hToo{\xslashedRightarrow{\quad}}
\let\xhTo\xslashedRightarrow

\pagestyle{empty}

\newcommand{\Rightthreecell}{\RRightarrow}
\newcommand{\Rtwocell}{\Rightarrow}

\tikzstyle{doubletick}=[-implies, double equal sign distance, postaction={decorate},decoration={markings,mark=at position .5 with {\draw[-] (0,-0.1) -- (0,0.1);}}]

\tikzstyle{darrow}=[-implies, double equal sign distance]

\tikzstyle{doubleeq}=[double equal sign distance]


%% ARROWS
% \to already exists
\newcommand{\too}[1][]{\ensuremath{\overset{#1}{\longrightarrow}}}
\newcommand{\ot}{\ensuremath{\leftarrow}}
\newcommand{\oot}[1][]{\ensuremath{\overset{#1}{\longleftarrow}}}
\let\toot\rightleftarrows
\let\otto\leftrightarrows
\let\Impl\Rightarrow
\let\imp\Rightarrow
\let\toto\rightrightarrows
\let\into\hookrightarrow
\let\xinto\xhookrightarrow
\mdef\we{\overset{\sim}{\longrightarrow}}
\mdef\leftwe{\overset{\sim}{\longleftarrow}}
\let\mono\rightarrowtail
\let\leftmono\leftarrowtail
\let\cof\rightarrowtail
\let\leftcof\leftarrowtail
\let\epi\twoheadrightarrow
\let\leftepi\twoheadleftarrow
\let\fib\twoheadrightarrow
\let\leftfib\twoheadleftarrow
\let\cohto\rightsquigarrow
\let\maps\colon
\newcommand{\spam}{\,:\!}       % \maps for left arrows

\newsavebox{\DDownarrowbox}
\savebox{\DDownarrowbox}{\tikz[scale=1.5]{\draw[-implies,double equal
sign distance] (0,.1) -- (0,-.1); \draw (0,.1) -- (0,-.1);}}
\newcommand{\DDownarrow}{\mathrel{\raisebox{-.2em}{\usebox{\DDownarrowbox}}}}

\newsavebox{\RRightarrowbox}
\savebox{\RRightarrowbox}{\tikz[scale=1.5]{\draw[-implies,double equal
sign distance] (-.1,0) -- (.1,0); \draw (-.1,0) -- (.1,0);}}
\newcommand{\RRightarrow}{\mathrel{\raisebox{.2em}{\usebox{\RRightarrowbox}}}}

%\newsavebox{\Rightslashedarrowbox}
%\savebox{\Rightslashedarrowbox}{\tikz[scale=1.5]{\draw[Rightslashedarrow{}] (-.1,0) -- (1,0);}}
%\newcommand{\Rightslashedarrow}{\mathrel{\raisebox{-.2em}%{\usebox{\Rightslashedarrowbox}}}}


%% EXTENSIBLE ARROWS
\let\xto\xrightarrow
\let\xot\xleftarrow
% See Voss' Mathmode.tex for instructions on how to create new
% extensible arrows.
\def\rightarrowtailfill@{\arrowfill@{\Yright\joinrel\relbar}\relbar\rightarrow}
\newcommand\xrightarrowtail[2][]{\ext@arrow 0055{\rightarrowtailfill@}{#1}{#2}}
\let\xmono\xrightarrowtail
\let\xcof\xrightarrowtail
\def\twoheadrightarrowfill@{\arrowfill@{\relbar\joinrel\relbar}\relbar\twoheadrightarrow}
\newcommand\xtwoheadrightarrow[2][]{\ext@arrow 0055{\twoheadrightarrowfill@}{#1}{#2}}
\let\xepi\xtwoheadrightarrow
\let\xfib\xtwoheadrightarrow
% Let's leave the left-going ones until I need them.

%% EXTENSIBLE SLASHED ARROWS
% Making extensible slashed arrows, by modifying the underlying AMS code.
% Arguments are:
% 1 = arrowhead on the left (\relbar or \Relbar if none)
% 2 = fill character (usually \relbar or \Relbar)
% 3 = slash character (such as \mapstochar or \Mapstochar)
% 4 = arrowhead on the left (\relbar or \Relbar if none)
% 5 = display mode (\displaystyle etc)
\def\slashedarrowfill@#1#2#3#4#5{%
  $\m@th\thickmuskip0mu\medmuskip\thickmuskip\thinmuskip\thickmuskip
   \relax#5#1\mkern-7mu%
   \cleaders\hbox{$#5\mkern-2mu#2\mkern-2mu$}\hfill
   \mathclap{#3}\mathclap{#2}%
   \cleaders\hbox{$#5\mkern-2mu#2\mkern-2mu$}\hfill
   \mkern-7mu#4$%
}
% Here's the idea: \<slashed>arrowfill@ should be a box containing
% some stretchable space that is the "middle of the arrow".  This
% space is created as a "leader" using \cleader<thing>\hfill, which
% fills an \hfill of space with copies of <thing>.  Here instead of
% just one \cleader, we use two, with the slash in between (and an
% extra copy of the filler, to avoid extra space around the slash).
\def\rightslashedarrowfill@{%
  \slashedarrowfill@\relbar\relbar\mapstochar\rightarrow}
\newcommand\xslashedrightarrow[2][]{%
  \ext@arrow 0055{\rightslashedarrowfill@}{#1}{#2}}
\mdef\hto{\xslashedrightarrow{}}
\mdef\htoo{\xslashedrightarrow{\quad}}
\let\xhto\xslashedrightarrow

%% To get a slashed arrow in XYpic, do
% \[\xymatrix{A \ar[r]|-@{|} & B}\]

% ISOMORPHISMS
\def\xiso#1{\mathrel{\mathrlap{\smash{\xto[\smash{\raisebox{1.3mm}{$\scriptstyle\sim$}}]{#1}}}\hphantom{\xto{#1}}}}
\def\toiso{\xto{\smash{\raisebox{-.5mm}{$\scriptstyle\sim$}}}}

% SHADOWS
\def\shvar#1#2{{\ensuremath{%
  \hspace{1mm}\makebox[-1mm]{$#1\langle$}\makebox[0mm]{$#1\langle$}\hspace{1mm}%
  {#2}%
  \makebox[1mm]{$#1\rangle$}\makebox[0mm]{$#1\rangle$}%
}}}
\def\sh{\shvar{}}
\def\scriptsh{\shvar{\scriptstyle}}
\def\bigsh{\shvar{\big}}
\def\Bigsh{\shvar{\Big}}
\def\biggsh{\shvar{\bigg}}
\def\Biggsh{\shvar{\Bigg}}

%HIGHER CELLS



% THEOREM-TYPE ENVIRONMENTS, hacked to
%% (a) number all with the same numbers, and
%% (b) have the right names for autoref
\def\defthm#1#2{%
  \newtheorem{#1}{#2}[section]%
  \expandafter\def\csname #1autorefname\endcsname{#2}%
  \expandafter\let\csname c@#1\endcsname\c@thm}
\newtheorem{thm}{Theorem}[section]
\newcommand{\thmautorefname}{Theorem}
\defthm{cor}{Corollary}
\defthm{prop}{Proposition}
\defthm{lem}{Lemma}
\defthm{sch}{Scholium}
\defthm{assume}{Assumption}
\defthm{claim}{Claim}
\defthm{conj}{Conjecture}
\defthm{hyp}{Hypothesis}
\defthm{fact}{Fact}
\theoremstyle{definition}
\defthm{defn}{Definition}
\defthm{notn}{Notation}
\theoremstyle{remark}
\defthm{rmk}{Remark}
\defthm{eg}{Example}
\defthm{egs}{Examples}
\defthm{ex}{Exercise}
\defthm{ceg}{Counterexample}

% How to get QED symbols inside equations at the end of the statements
% of theorems.  AMS LaTeX knows how to do this inside equations at the
% end of *proofs* with \qedhere, and at the end of the statement of a
% theorem with \qed (meaning no proof will be given), but it can't
% seem to combine the two.  Use this instead.
\def\thmqedhere{\expandafter\csname\csname @currenvir\endcsname @qed\endcsname}

% Number numbered lists as (i), (ii), ...
\renewcommand{\theenumi}{(\roman{enumi})}
\renewcommand{\labelenumi}{\theenumi}

%% Labeling that keeps track of theorem-type names.  Superseded by
%% autoref from hyperref, as above, but we keep this in case we are
%% using a journal style file that is incompatible with hyperref.
% 
% \ifx\SK@label\undefined\let\SK@label\label\fi
% \let\your@thm\@thm
% \def\@thm#1#2#3{\gdef\currthmtype{#3}\your@thm{#1}{#2}{#3}}
% \def\xlabel#1{{\let\your@currentlabel\@currentlabel\def\@currentlabel
% {\currthmtype~\your@currentlabel}
% \SK@label{#1@}}\label{#1}}
% \def\xref#1{\ref{#1@}}

% Also number formulas with the theorem counter
\let\c@equation\c@thm
\numberwithin{equation}{section}

% Only show numbers for equations that are actually referenced (or
% whose tags are specified manually) - uses mathtools.
\mathtoolsset{showonlyrefs,showmanualtags}

%% Fix enumerate spacing with paralist.  This has two parts:
%%   1. enable mixing of "old spacing" lists with those adjusted by paralist
%%   2. allow us to specify a number based on which to adjust the spacing
%% For the first, use \killspacingtrue when you want the spacing
%% adjusted, then \killspacingfalse to turn adjustment off.  For the
%% second, use \maxenum=14 to set the widest number you want the
%% spacing to be calculated with.
\newlength\oldleftmargini       % save old spacing
\newlength\oldleftmarginii
\newlength\oldleftmarginiii
\newlength\oldleftmarginiv
\newlength\oldleftmarginv
\newlength\oldleftmarginvi
\newcount\maxenum
\maxenum=7
\newif\ifkillspacing
\def\@adjust@enum@labelwidth{%
  \advance\@listdepth by 1\relax
  \ifkillspacing                % do the paralist thing
    \csname c@\@enumctr\endcsname\maxenum
    \settowidth{\@tempdima}{%
      \csname label\@enumctr\endcsname\hspace{\labelsep}}%
    \csname leftmargin\romannumeral\@listdepth\endcsname
      \@tempdima
  \else                         % otherwise, reset it
    \csname fixspacing\romannumeral\@listdepth\endcsname
  \fi
  \advance\@listdepth by -1\relax}
% these commands, one for each enum level (I couldn't get a generic
% one to work), test whether oldleftmargin has been set yet, and if
% not, set it from leftmargin; otherwise, they reset leftmargin to
% it.  Just setting oldleftmargin to leftmargin in the preamble
% doesn't seem to work.
\def\fixspacingi{\ifnum\oldleftmargini=0\setlength\oldleftmargini\leftmargini\else\setlength\leftmargini\oldleftmargini\fi}
\def\fixspacingii{\ifnum\oldleftmarginii=0\setlength\oldleftmarginii\leftmarginii\else\setlength\leftmarginii\oldleftmarginii\fi}
\def\fixspacingiii{\ifnum\oldleftmarginiii=0\setlength\oldleftmarginiii\leftmarginiii\else\setlength\leftmarginiii\oldleftmarginiii\fi}
\def\fixspacingiv{\ifnum\oldleftmarginiv=0\setlength\oldleftmarginiv\leftmarginiv\else\setlength\leftmarginiv\oldleftmarginiv\fi}
\def\fixspacingv{\ifnum\oldleftmarginv=0\setlength\oldleftmarginv\leftmarginv\else\setlength\leftmarginv\oldleftmarginv\fi}
\def\fixspacingvi{\ifnum\oldleftmarginvi=0\setlength\oldleftmarginvi\leftmarginvi\else\setlength\leftmarginvi\oldleftmarginvi\fi}

%% Fix paralist references, so that we can refer to (1) instead of
%% just 1.
\def\pl@label#1#2{%
  \edef\pl@the{\noexpand#1{\@enumctr}}%
  \pl@lab\expandafter{\the\pl@lab\csname yourthe\@enumctr\endcsname}%
  \advance\@tempcnta1
  \pl@loop}
\def\@enumlabel@#1[#2]{%
  \@plmylabeltrue
  \@tempcnta0
  \pl@lab{}%
  \let\pl@the\pl@qmark
  \expandafter\pl@loop\@gobble#2\@@@
  \ifnum\@tempcnta=1\else
    \PackageWarning{paralist}{Incorrect label; no or multiple
      counters.\MessageBreak The label is: \@gobble#2}%
  \fi
  \expandafter\edef\csname label\@enumctr\endcsname{\the\pl@lab}%
  \expandafter\edef\csname the\@enumctr\endcsname{\the\pl@lab}%
  \expandafter\let\csname yourthe\@enumctr\endcsname\pl@the
  #1}


% GREEK LETTERS, ETC.
\alwaysmath{alpha}
\alwaysmath{beta}
\alwaysmath{gamma}
\alwaysmath{Gamma}
\alwaysmath{delta}
\alwaysmath{Delta}
\alwaysmath{epsilon}
\mdef\ep{\varepsilon}
\alwaysmath{zeta}
\alwaysmath{eta}
\alwaysmath{theta}
\alwaysmath{Theta}
\alwaysmath{iota}
\alwaysmath{kappa}
\alwaysmath{lambda}
\alwaysmath{Lambda}
\alwaysmath{mu}
\alwaysmath{nu}
\alwaysmath{xi}
\alwaysmath{pi}
\alwaysmath{rho}
\alwaysmath{sigma}
\alwaysmath{Sigma}
\alwaysmath{tau}
\alwaysmath{upsilon}
\alwaysmath{Upsilon}
\alwaysmath{phi}
\alwaysmath{Pi}
\alwaysmath{Phi}
\mdef\ph{\varphi}
\alwaysmath{chi}
\alwaysmath{psi}
\alwaysmath{Psi}
\alwaysmath{omega}
\alwaysmath{Omega}
\let\al\alpha
\let\be\beta
\let\gm\gamma
\let\Gm\Gamma
\let\de\delta
\let\De\Delta
\let\si\sigma
\let\Si\Sigma
\let\om\omega
\let\ka\kappa
\let\la\lambda
\let\La\Lambda
\let\ze\zeta
\let\th\theta
\let\Th\Theta
\let\vth\vartheta

\makeatother

% Tikz styles
\tikzstyle{tickarrow}=[->,postaction={decorate},decoration={markings,mark=at position .5 with {\draw[-] (0,-0.1) -- (0,0.1);}},line width=0.50]

% Local Variables:
% mode: latex
% TeX-master: ""
% End:

\begin{document}

{\small
\begin{equation}\label{eq:laxfunc2}\hspace{-2cm}
\begin{tikzpicture}[xscale=2.25, yscale=1.5]
%%%%Row A
\node (A0) at (0,2) {$\substack{\tens (f\times f)}$};
\node (A3) at (0,6) {$\substack{f \tens }$};
\node (A31) at (.5,7.5) {$\substack{f \tens \\ (\transid \times \transid)}$};
\node (A32) at (2,8.5) {$\substack{f \tens \\ ([\tens(\transid \times I)] \times \transid)}$};
\node (A4) at (3,8.5) {$\substack{f \tens (\tens \times \transid) \\ (\transid \times I\times \transid)}$};
\node (A5) at (4.5,8) {$\substack{f \tens (\transid \times \tens) \\ (\transid \times I \times \transid)}$};
\node (A51) at (5.5,7) {$\substack{f \tens (\transid \times [\tens (I \times \transid)]) }$};
\node (A52) at (6,6) {$\substack{f \tens (\transid \times \transid) }$};
\node (A6) at (6,5) {$\substack{ f \tens }$};
%%%%
\draw[doubleloose] (A0) to node[left] {$\substack{\chi}$} (A3);
\draw[doubleeq] (A3) to  (A31);
\draw[doubleloose] (A31) to node[above, yshift=10pt] {$\substack{\looseid \looseid (r^{-1} \times \transid)}$} (A32);
\draw[doubleeq] (A32) to (A4);
\draw[doubleloose] (A4) to node[above] {$\substack{\looseid \alpha \looseid}$} (A5);
\draw[doubleeq] (A5) to  (A51);
\draw[doubleloose] (A51) to node[right] {$\substack{\looseid  \looseid (\looseid \times l)}$} (A52);
\draw[doubleeq] (A52) to  (A6);
%%%%Row B
\node (B2) at (1,5.5) {$\substack{\tens (f \times f)\\(\transid \times \transid) }$};
\node (B3) at (2,7) {$\substack{\tens (f \times f)\\([\tens(\transid \times I)] \times \transid)}$};
\node (B35) at (2.25,7.75) {$\substack{\tens (f \times f)\\(\tens \times \transid) \\ (\transid \times I \times \transid)}$};
%%%%
\draw[doubleeq] (A0) to (B2);
\draw[doubleloose] (B2) to node[left] {$\substack{\looseid \looseid (r^{-1} \times \looseid)}$} (B3);
\draw[doubleeq] (B3) to (B35);
\draw[doubleloose] (B35) to node[right, xshift=3pt] {$\substack{\chi \looseid \looseid}$} (A4);
%%%%Row C
\node (C1) at (.75,2) {$\substack{\tens (\transid \times \transid) \\ (f \times f)}$};
\node (C2) at (1.75,2.5) {$\substack{\tens ([\tens ( \transid \times I)]\times \transid) \\ (f \times f)}$};
\node (C3) at (2.5,3) {$\substack{\tens (\tens \times \transid) \\ (f \times I \times f)}$};
\node (C4) at (2.5,5) {$\substack{\tens \\(\tens \times \transid) \\ (f \times fI \times f)}$};
%%%%
\draw[doubleeq] (A0) to (C1);
\draw[doubleloose] (C1) to node[above] {$\substack{\looseid \\ (r^{-1} \times \looseid)\looseid }$} (C2);
\draw[doubleeq] (C2) to (C3);
\draw[doubleloose] (C3) to node[left] {$\substack{\looseid \looseid \\ (\looseid \times \iota \times \looseid)}$} (C4);
\draw[doubleloose] (B3) to node[right] {$\substack{ \looseid (\chi \times \looseid) \looseid}$} (C4);
%%%%Row D
\node (D2) at (3.5,5.5) {$\substack{\tens (\transid \times \tens )\\ (f \times fI \times f) }$};
\node (D3) at (4,6.5) {$\substack{\tens \\(f \times [\tens (f\times f)] )\\ (\transid \times I \times \transid) }$};
\node (D4) at (5,5.5) {$\substack{ \tens (f \times f\tens) \\ (\transid \times I \times \transid)}$};
\node (D5) at (5,6.5) {$\substack{ \tens (f \times f) \\ (\transid \times \tens) \\ (\transid \times I \times \transid)}$};
\node (D6) at (5,4.5) {$\substack{ \tens (f \times f) \\ (\transid \times [\tens(I \times \transid)]) }$};
\node (D7) at (5.5,3.5) {$\substack{ \tens (f \times f) \\ (\transid \times  \transid) }$};
%%%%
\draw[doubleloose] (C4) to node[above] {$\substack{ \alpha \looseid}$} (D2);
\draw[doubleloose] (D3) to node[below, xshift=-3pt] {$\substack{ \looseid \\ (\looseid \times \chi) \looseid}$} (D4);
\draw[doubleeq] (D2) to  (D3);
\draw[doubleeq] (D4) to  (D5);
\draw[doubleeq] (D4) to  (D6);
\draw[doubleloose] (D5) to node[right] {$\substack{\chi \looseid \looseid}$} (A5);
%%%%Row E
\node (E2) at (3.5,2.5) {$\substack{\tens ( \transid \times \tens) \\ (f \times I \times f)}$};
\node (E3) at (4.25,2) {$\substack{\tens ( \transid \times [\tens(I \times \transid)]) \\ (f \times f)}$};
\node (E4) at (5.25,1.5) {$\substack{\tens ( \transid \times  \transid) \\ (f  \times f)}$};
\node (E5) at (6,2) {$\substack{ \tens (f \times f)}$};
%%%%
\draw[doubleloose] (C3) to node[below] {$\substack{\alpha \looseid }$} (E2);
\draw[doubleloose] (E2) to node[right] {$\substack{ \looseid \looseid \\ (\looseid  \times \iota \times \looseid)}$} (D2);
\draw[doubleloose] (D6) to node[right] {$\substack{\looseid  \looseid (\looseid \times l)}$} (D7);
\draw[doubleeq] (D7) to  (E5);
\draw[doubleloose] (E5) to node[right] {$\chi$} (A6);
%%%%
\draw[doubleeq] (E2) to  (E3);
\draw[doubleloose] (E3) to node[above] {$\substack{ \looseid  (\looseid  \times l)}$} (E4);
\draw[doubleeq] (E4) to  (E5);
%%%%
\draw[doubleloose] (A0) to[out=270, in=285] node[above] {$\substack{\looseid}$} (E5);
%%%% 3-cells
\node at (.5,6) {$\DDownarrow \iso$};
\node at (2,6) {$\DDownarrow \iso$};
\node at (1.5,4.5) {$\DDownarrow \tightid \delta$};
\node at (2,3.5) {$\DDownarrow \iso$};
\node at (3.4,8.1) {$\DDownarrow \iso$};
\node at (3.5,7.5
) {$\DDownarrow \omega \tightid$};
\node at (5.5,5) {$\DDownarrow \iso$};
\node at (4.25,4.5) {$\DDownarrow \iso$};
\node at (4.25,2.6) {$\DDownarrow \tightid \gamma$};
\node at (5.2,1.85) {$\DDownarrow \iso$};
\node at (3,3.5) {$\DDownarrow \iso$};
\node at (3,2) {$\DDownarrow \iso$};
\node at (3,1) {$\DDownarrow \mu$};
\node at (3,.25) {$\DDownarrow \iso$};
%%%%%%
\draw[doubleloose] (A0) to[in=135, out=65] node[above,xshift=44pt, yshift=36pt]{$\substack{\looseid S(\delta)}$} (C4);
\draw[doubleloose] (A0) to[in=225, out=20] node[left]{$\substack{\looseid T(\delta)}$} (C4);
%%%%%%%%
\draw[doubleloose] (B35) to node[below]{$\substack{S(\omega) \looseid }$} (A5);
\draw[doubleloose] (B35) to[in=245, out=-45] node[below]{$\substack{ T(\omega) \looseid}$} (A5);
%%%%%%%%
\draw[doubleloose] (E2) to node[above,xshift=5pt]{$\substack{S(\gamma) \looseid }$} (E5);
\draw[doubleloose] (E2) to[in=155, out=60] node[above]{$\substack{ T(\gamma) \looseid}$} (E5);
%%%%%%%%
\draw[doubleloose] (C1) to node[below]{$\substack{S(\mu) \looseid }$} (E4);
\draw[doubleloose] (C1) to[in=215, out=-45] node[below]{$\substack{T(\mu) \looseid }$} (E4);
\end{tikzpicture}\hspace{-2cm}
\end{equation}
\begin{equation*}
=
\end{equation*}
\begin{equation*}\hspace{-2cm}
\begin{tikzpicture}[xscale=2.25, yscale=1.5]
%%%%Row A
\node (A0) at (0,5) {$\substack{\tens (f\times f)}$};
\node (A3) at (0,6) {$\substack{f \tens }$};
\node (A31) at (.5,7.5) {$\substack{f \tens \\ (\transid \times \transid)}$};
\node (A32) at (2,8.5) {$\substack{f \tens \\ ([\tens(\transid \times I)] \times \transid)}$};
\node (A4) at (3,8.5) {$\substack{f \tens (\tens \times \transid) \\ (\transid \times I\times \transid)}$};
\node (A5) at (4.5,8.5) {$\substack{f \tens (\transid \times \tens) \\ (\transid \times I \times \transid)}$};
\node (A51) at (5.5,7.5) {$\substack{f \tens (\transid \times [\tens (I \times \transid)]) }$};
\node (A52) at (6,6.5) {$\substack{f \tens (\transid \times \transid) }$};
\node (A6) at (6,6) {$\substack{ f \tens }$};
%%%%
\draw[doubleloose] (A0) to node[left] {$\substack{\chi}$} (A3);
\draw[doubleeq] (A3) to  (A31);
\draw[doubleloose] (A31) to node[above, yshift=10pt] {$\substack{\looseid \looseid (r^{-1} \times \transid)}$} (A32);
\draw[doubleeq] (A32) to (A4);
\draw[doubleloose] (A4) to node[above] {$\substack{\looseid \alpha \looseid}$} (A5);
\draw[doubleeq] (A5) to  (A51);
\draw[doubleloose] (A51) to node[right] {$\substack{\looseid  \looseid (\looseid \times l)}$} (A52);
\draw[doubleeq] (A52) to  (A6);
\node (E5) at (6,5) {$\substack{ \tens (f \times f)}$};
\draw[doubleloose] (E5) to node[right] {$\chi$} (A6);
%%%%
\draw[doubleloose] (A0) to node[above] {$\substack{\looseid}$} (E5);
%%%% 3-cells
\node at (4,8) {$\DDownarrow \iso$};
\node at (3,7) {$\DDownarrow \mu$};
\node at (1.5,6.5) {$\DDownarrow \iso$};
\node at (3,5) {$\DDownarrow \iso$};
\draw[doubleloose] (A3) to node[below]{$\substack{ \looseid }$} (A6);
\draw[doubleloose] (A31) to[in=190, out=-35] node[below]{$\substack{T(\mu) \looseid }$} (A52);
\draw[doubleloose] (A31) to[in=145, out=20] node[below]{$\substack{S(\mu) \looseid }$} (A52);
\end{tikzpicture}\hspace{-2cm}
\end{equation*}}
\end{document}  \newpage

\subsubsection*{Strong Monoidal 1-cell}

%The structure cells $\chi$ and $\bar{\chi}$ form an adjoint equivalence, if  there exist globular 3-cells $\eta$ and $\epsilon$, such that the equations below hold. 

%%
\documentclass[12pt]{ociamthesis}
\usepackage{tikz}
\newcommand{\id}{\mathrm{id}}
\begin{document}

\begin{equation}\label{eq:strong2mates}
\begin{pic}[scale=1.75]
\draw[fill=blue, opacity = 0.5, draw=blue] (-1,0) -- (-1,-2) -- (0.6,-2) -- (0.6, -.6) -- (0,-.6) -- (0,-1.4) --  (-.6,-1.4) -- (-.6, 0) -- (-1,0);
\draw[fill=red, opacity = 0.5, draw=red] (-.6,0) -- (-.6,-1.4) -- (0,-1.4) -- (0,-.6) -- (.6,-.6) -- (.6,-2) -- (1,-2) -- (1,0) -- (-.6,0);   
   %  \draw (.3,-.6) to (.3,0);
     \draw (0,-1.4) to node[left]{$\bar{\chi}$} (0,-.6);
     \draw (-.6,-1.4) to (-.6,0);
       \node[morphism, minimum width=20mm] (l) at (-.3,-1.4) {$\eta$};
       \node[morphism, minimum width=20mm] (r) at (.3,-.6) {$\epsilon$};
%\draw (-.3, -2) to (l.south);
\node at (-.6,.2) {$\chi$};
\node at (-.3,-1.8) {$\otimes(f\times f)$};
\node at (.3,-.2) {$f\otimes$};
\node at (.6,-2.2) {$\chi$};
    \end{pic}
    =
    \begin{pic}[scale=1.75]
\draw[fill=red, opacity = 0.5, draw=red] (-1,0) -- (-1,-2) -- (-0.6,-2) -- (-.6, -.6) -- (0,-.6) -- (0,-1.4) --  (.6,-1.4) -- (.6, 0) -- (-1,0);
\draw[fill=blue, opacity = 0.5, draw=blue] (-.6,-2) -- (-.6,-.6) -- (0,-.6) -- (0,-1.4) -- (.6,-1.4) -- (.6,0) -- (1,0) -- (1,-2) -- (-.6,-2);   
 %    \draw (.3,-1.4) to (.3,-2);
     \draw (0,-1.4) to node[left]{$\chi$} (0,-.6);
     \draw (-.6,-.6) to (-.6,-2);
       \node[morphism, minimum width=20mm] (l) at (-.3,-.6) {$\epsilon$};
       \node[morphism, minimum width=20mm] (r) at (.3,-1.4) {$\eta $};
%\draw (-.3, 0) to (l.north);
\node at (-.6,-2.2) {$\bar{\chi}$};
\node at (-.3,-.2) {$f\otimes $};
\node at (.3,-1.8) {$\otimes(f\times f)$};
\node at (.6,.2) {$\bar{\chi}$};
    \end{pic}
\end{equation}
\end{document} 


The 3-cells $\omega$ and $\bar{\omega}$ correspond to each other as mates, if the equation below holds. Here, one needs to note that for two adjoint equivalences $\alpha \dashv \bar{\alpha}$ and $\beta \dashv \bar{\beta}$, where $\alpha: f \rightarrow g$ and $\beta: h \rightarrow k$, there is an adjoint equivalence $\alpha \beta \dashv \bar{\alpha}\bar{\beta}$ witnessed by $\eta_{\alpha \beta} := \eta_{\alpha} \eta_{\beta}$ and $\epsilon_{\alpha \beta} :=  \epsilon_{\alpha} \epsilon_{\beta}$.  

%
\documentclass[12pt]{ociamthesis}
\usepackage{tikz}
\newcommand{\id}{\mathrm{id}}
\begin{document}

\begin{align}\label{eq:strong2mates2}
\begin{pic}[xscale=-.8]
\draw[fill=blue, opacity = 0.5, draw=black] (0,8) -- (0,0) -- (7,0) -- (7,5) -- (6,5) -- (6,1) -- (1,1) -- (1,8) -- (0,8);
\draw[fill=purple, opacity = 0.5, draw=black] (1,8) -- (1,1) -- (6,1) -- (6,4) -- (5,4) -- (5,2) -- (2,2) -- (2,8) -- (1,8); 
\draw[fill=red, opacity = 0.5, draw=black] (2,8) -- (2,2) -- (5,2) -- (5,4) -- (4,4) -- (4,3) -- (3,3) -- (3,8) -- (2,8); 
\draw[fill=orange, opacity = 0.5, draw=black] (3,8) -- (3,3) -- (4,3) -- (4,7) -- (9,7) -- (9,0) -- (10,0) -- (10,8) -- (3,8); 
\draw[fill=yellow, opacity = 0.5, draw=black] (9,0) -- (8,0) -- (8,6) -- (5,6) -- (5,4) -- (4,4) -- (4,7) -- (9,7) -- (9,0);
\draw[fill=green, opacity = 0.5, draw=black] (7,0) -- (8,0) -- (8,6) -- (5,6) -- (5,4) -- (6,4) -- (6,5) -- (7,5) -- (7,0) ;
\node[morphism, minimum width=20mm] (l) at (5,4) {$\bar{\omega}$};
\node[morphism, minimum width=10mm] (l) at (3.5,3) {$\eta_{\chi \looseid}$};
\node[morphism, minimum width=28mm] (l) at (3.5,2) {$\eta_{\looseid (\looseid\times\chi)}$};
\node[morphism, minimum width=42mm] (l) at (3.5,1) {$\eta_{\alpha\looseid}$};
\node[morphism, minimum width=10mm] (l) at (6.5,5) {$\epsilon_{\looseid(\chi \times \looseid)}$};
\node[morphism, minimum width=28mm] (l) at (6.5,6) {$\epsilon_{\chi\looseid}$};
\node[morphism, minimum width=42mm] (l) at (6.5,7) {$\epsilon_{\looseid\alpha}$};
    \end{pic}
    =
    \begin{pic}[xscale=-0.8]
\draw[fill=blue, opacity = 0.5, draw=black] (0,8) -- (0,0) -- (1,0) -- (1,8) -- (0,8);
\draw[fill=purple, opacity = 0.5, draw=black] (1,8) -- (1,4) -- (2,4) -- (2,8) -- (1,8); 
\draw[fill=red, opacity = 0.5, draw=black] (2,8) -- (2,4) -- (3,4) -- (3,8) --  (2,8); 
\draw[fill=orange, opacity = 0.5, draw=black] (3,8) -- (3,0) -- (4,0) -- (4,8) -- (3,8); 
\draw[fill=green, opacity = 0.5, draw=black] (1,4) -- (2,4) -- (2,0) -- (1,0) -- (1,4);
\draw[fill=yellow, opacity = 0.5, draw=black] (2,4) -- (3,4) -- (3,0) -- (2,0) -- (2,4);
\node[morphism, minimum width=20mm] (l) at (2,4) {$\omega$};
    \end{pic}
\end{align}
\end{document} 


%
\documentclass[12pt]{ociamthesis}
\usepackage{tikz}
\newcommand{\id}{\mathrm{id}}
\begin{document}

\begin{align}\label{eq:strong2mates3}
 \begin{pic}[yscale=.8, xscale=-.5]
\draw[fill=blue, opacity = 0.5, draw=black] (0,8) -- (0,0) -- (7,0) -- (7,5) -- (6,5) -- (6,4) -- (5,4) -- (5,2) -- (2,2) -- (2,8) -- (0,8);
\draw[fill=orange, opacity = 0.5, draw=black] (2,8) -- (2,2) -- (5,2) -- (5,4) -- (4,4) -- (4,7) -- (9,7) -- (9,0) -- (10,0) -- (10,8) -- (2,8); 
\draw[fill=yellow, opacity = 0.5, draw=black] (9,0) -- (8,0) -- (8,6) -- (5,6) -- (5,4) -- (4,4) -- (4,7) -- (9,7) -- (9,0);
\draw[fill=green, opacity = 0.5, draw=black] (7,0) -- (8,0) -- (8,6) -- (5,6) -- (5,4) -- (6,4) -- (6,5) -- (7,5) -- (7,0) ;
\node[morphism, minimum width=15mm] (l) at (5,4) {$\bar{\gamma}$};
\node[morphism, minimum width=20mm] (l) at (3.5,2) {$\eta_{l \looseid}$};
\node[morphism, minimum width=10mm] (l) at (6.5,5) {$\epsilon_{\looseid(\iota \times \looseid)\looseid}$};
\node[morphism, minimum width=20mm] (l) at (6.5,6) {$\epsilon_{\chi\looseid \looseid}$};
\node[morphism, minimum width=32mm] (l) at (6.5,7) {$\epsilon_{\looseid l}$};
    \end{pic}
    =
    \begin{pic}[yscale=0.8, xscale=-.5]
\draw[fill=blue, opacity = 0.5, draw=black] (0,8) -- (0,0) -- (1,0) -- (1,4) -- (2,4) -- (2,8) -- (0,8);
\draw[fill=orange, opacity = 0.5, draw=black] (2,8) -- (2,4) -- (3,4) -- (3,0) -- (4,0) -- (4,8) -- (2,8); 
\draw[fill=green, opacity = 0.5, draw=black] (1,4) -- (2,4) -- (2,0) -- (1,0) -- (1,4);
\draw[fill=yellow, opacity = 0.5, draw=black] (2,4) -- (3,4) -- (3,0) -- (2,0) -- (2,4);
\node[morphism, minimum width=15mm] (l) at (2,4) {$\gamma$};
    \end{pic}
    \end{align}
\end{document} 


%
\documentclass[12pt]{ociamthesis}
\usepackage{tikz}
\newcommand{\id}{\mathrm{id}}
\begin{document}

\begin{align}\label{eq:strong2mates4}
\begin{pic}[yscale=.8,xscale=-.5]
\draw[fill=blue, opacity = 0.5, draw=black] (0,8) -- (0,0) -- (8,0) -- (8,6) -- (5,6) -- (5,4) -- (6,4) -- (6,1) -- (1,1) -- (1,8) -- (0,8);
\draw[fill=purple, opacity = 0.5, draw=black] (1,8) -- (1,1) -- (6,1) -- (6,4) -- (5,4) -- (5,2) -- (2,2) -- (2,8) -- (1,8); 
\draw[fill=red, opacity = 0.5, draw=black] (2,8) -- (2,2) -- (5,2) -- (5,4) -- (4,4) -- (4,3) -- (3,3) -- (3,8) -- (2,8); 
\draw[fill=orange, opacity = 0.5, draw=black] (3,8) -- (3,3) -- (4,3) -- (4,4) -- (5,4) -- (5,6) -- (8,6) -- (8,0) -- (10,0) -- (10,8) -- (3,8); 
\node[morphism, minimum width=15mm] (l) at (5,4) {$\bar{\delta}$};
\node[morphism, minimum width=10mm] (l) at (3.5,3) {$\eta_{\chi \looseid}$};
\node[morphism, minimum width=20mm] (l) at (3.5,2) {$\eta_{\looseid (\looseid\times\iota)\looseid}$};
\node[morphism, minimum width=32mm] (l) at (3.5,1) {$\eta_{r \looseid}$};
\node[morphism, minimum width=20mm] (l) at (6.5,6) {$\epsilon_{\looseid r}$};
    \end{pic}
    =
   \begin{pic}[yscale=0.8, xscale=-.5]
\draw[fill=blue, opacity = 0.5, draw=black] (0,8) -- (0,0) -- (2,0) --(2,4) -- (1,4) -- (1,8) -- (0,8);
\draw[fill=purple, opacity = 0.5, draw=black] (1,8) -- (1,4) -- (2,4) -- (2,8) -- (1,8); 
\draw[fill=red, opacity = 0.5, draw=black] (2,8) -- (2,4) -- (3,4) -- (3,8) --  (2,8); 
\draw[fill=orange, opacity = 0.5, draw=black] (3,8) -- (3,4) -- (2,4) -- (2,0) -- (4,0) -- (4,8) -- (3,8); 
\node[morphism, minimum width=15mm] (l) at (2,4) {$\delta$};
    \end{pic}
\end{align}

\end{document} 
\newpage

\subsubsection*{Lax Monoidal 2-cell}

 %
\documentclass[12pt]{ociamthesis}
\usepackage{tikz}
\newcommand{\id}{\mathrm{id}}
\begin{document}


\begin{equation*}
\begin{aligned}
\begin{tikzpicture}[xscale=3, yscale=1.5]
\node (t0) at (0,2) {\scriptsize $\tens(I_B \times f)$};
\node (t1) at (1,2) {\scriptsize $\tens(f I_A \times f)$};
\node (t2) at (2,2) {\scriptsize $f \tens(I_A \times \transid)$};
\node (t3) at (3,2) {\scriptsize $f $};
\node (t4) at (4,2) {\scriptsize $g $};
\node (m0) at (0,1) {\scriptsize $\tens(I_B \times \transid)f$};
\node (m3) at (3,1) {\scriptsize $f $};
\node (m4) at (4,1) {\scriptsize $g $};
\node (b0) at (0,0) {\scriptsize $\tens(I_B \times \transid)f$};
\node (b3) at (3,0) {\scriptsize $\tens (I_B \times \transid)g$};
\node (b4) at (4,0) {\scriptsize $g $};
\draw[doubleloose] (t0) to node[above]{\scriptsize $\looseid_{\tens}(\iota_f \times \looseid_f)$} (t1);
\draw[doubleloose] (t1) to node[above]{\scriptsize $\chi (\looseid_{I \times \transid})$} (t2);
\draw[doubleloose] (t2) to node[above]{\scriptsize $\looseid_f l$} (t3);
\draw[doubleloose] (t3) to node[above]{\scriptsize $\beta$} (t4);
\draw[doubleloose] (m0) to node[above]{\scriptsize $l \looseid_f$} (m3);
\draw[doubleloose] (m3) to node[above]{\scriptsize $\beta$} (m4);
\draw[doubleloose] (b0) to node[above]{\scriptsize $\looseid_{\tens}(\beta \times \looseid_I)$} (b3);
\draw[doubleloose] (b3) to node[above]{\scriptsize $l \looseid_g$} (b4);
\draw[doubletighteq] (t0) to (m0);
\draw[doubletighteq] (m0) to (b0);
\draw[doubletighteq] (t3) to (m3);
\draw[doubletighteq] (t4) to (m4);
\draw[doubletighteq] (m4) to (b4);
\node at (1.5,1.5) {\scriptsize $\DDownarrow \gamma^f$};
\node at (3.5,1.5) {\scriptsize $\DDownarrow \tightid_{\beta}$};
\node at (2,0.5) {\scriptsize $\iso$};
\end{tikzpicture}
\end{aligned}
\end{equation*}
\begin{equation}\label{eq:mon2cell1}
=
\end{equation}
\begin{equation*}
\begin{aligned}
\begin{tikzpicture}[xscale=3, yscale=1.5]
\node (04) at (0,4) {\scriptsize $\tens(I_B \times f)$};
\node (14) at (1,4) {\scriptsize $\tens(f I_A\times f)$};
\node (24) at (2,4) {\scriptsize $f \tens(I_A \times \transid_A)$};
\node (34) at (3,4) {\scriptsize $f $};
\node (44) at (4,4) {\scriptsize $g $};
%%%%%%
\node (03) at (0,3) {\scriptsize $\tens(I_B \times f)$};
\node (13) at (1,3) {\scriptsize $\tens(f I_A\times f)$};
\node (23) at (2,3) {\scriptsize $f \tens(I_A \times \transid_A)$};
\node (33) at (3,3) {\scriptsize $g \tens(I_A \times \transid_A)$};
\node (43) at (4,3) {\scriptsize $g $};
%%%%%%
\node (02) at (0,2) {\scriptsize $\tens(I_B \times f)$};
\node (12) at (1,2) {\scriptsize $\tens(f I_A \times f)$};
\node (22) at (2,2) {\scriptsize $\tens(g I_A \times g)$};
\node (32) at (3,2) {\scriptsize $g \tens (I_A\times \transid_A)$};
\node (42) at (4,2) {\scriptsize $g $};
%%%%%% 
\node (01) at (0,1) {\scriptsize $\tens(I_B \times f)$};
\node (11) at (1,1) {\scriptsize $\tens(I_B \times g)$};
\node (21) at (2,1) {\scriptsize $\tens(g I_A \times g)$};
\node (31) at (3,1) {\scriptsize $g \tens (I_A \times \transid_A)$};
\node (41) at (4,1) {\scriptsize $g $};
%%%%%%%
\node (00) at (0,0) {\scriptsize $\tens(I_B \times \transid) f$};
\node (10) at (1,0) {\scriptsize $\tens(I_B \times \transid) g$};
\node (40) at (4,0) {\scriptsize $g $};
%%%%%%%
\draw[doubleloose] (04) to node[above]{\scriptsize $\looseid_{\tens}(\looseid_f \times \iota_f) $} (14);
\draw[doubleloose] (14) to node[above]{\scriptsize $\chi (\looseid_{I \times \transid})$} (24);
\draw[doubleloose] (24) to node[above]{\scriptsize $\looseid_f l$} (34);
\draw[doubleloose] (34) to node[above]{\scriptsize$\beta$} (44);
%%%%%%%
\draw[doubleloose] (03) to node[above]{\scriptsize $\looseid_{\tens}(\looseid_f \times \iota_f) $} (13);
\draw[doubleloose] (13) to node[above]{\scriptsize $\chi (\looseid_{I \times \transid})$} (23);
\draw[doubleloose] (23) to node[above]{\scriptsize $\beta \looseid_{\tens}(\looseid_{I\times \transid})$} (33);
\draw[doubleloose] (33) to node[above]{\scriptsize$\looseid_g l$} (43);
%%%%
\draw[doubleloose] (02) to node[above]{\scriptsize $\looseid_{\tens} (\iota_f  \times \looseid_f)$} (12);
\draw[doubleloose] (12) to node[above]{\scriptsize $\looseid_{\tens} (\beta \looseid_I \times \beta) $} (22);
\draw[doubleloose] (22) to node[above]{\scriptsize $\chi_g \looseid_{I \times \id}$} (32);
\draw[doubleloose] (32) to node[above]{\scriptsize $\looseid_g l$} (42);
%%%%%%
\draw[doubleloose] (01) to node[above]{\scriptsize $\looseid_{\tens} (\looseid_I \times \beta)$} (11);
\draw[doubleloose] (11) to node[above]{\scriptsize $\looseid_{\tens} (\iota_g \times \looseid_g) $} (21);
\draw[doubleloose] (21) to node[above]{\scriptsize $\chi \looseid_{I_A \times \transid}$} (31);
\draw[doubleloose] (31) to node[above]{\scriptsize $\looseid_g l$} (41);
%%%%%%
\draw[doubleloose] (00) to node[above]{\scriptsize $\looseid_{\tens} (\looseid_I  \times \looseid) \beta$} (10);
\draw[doubleloose] (10) to node[above]{\scriptsize $l \looseid_g $} (40);
%%%%%%
\draw[doubletighteq] (04) to (03);
\draw[=] (24) to (23);
\draw[doubletighteq] (44) to (43);
%%%%%%
\draw[doubletighteq] (03) to (02);
\draw[doubletighteq] (13) to (12);
\draw[doubletighteq] (33) to (32);
\draw[doubletighteq] (43) to (42);
%%%%%%
\draw[doubletighteq] (02) to (01);
\draw[doubletighteq] (22) to (21);
\draw[doubletighteq] (42) to (41);
%%%%%%
\draw[doubletighteq] (01) to (00);
\draw[doubletighteq] (11) to (10);
\draw[doubletighteq] (41) to (40);
%%%%%%%%
\node at (1,3.5) {\scriptsize $=$};
\node at (3,3.5) {\scriptsize $\DDownarrow \cong $};
\node at (.5,2.5) {\scriptsize $=$};
\node at (2,2.5) {\scriptsize $\DDownarrow \overline{\Pi^{\beta}\tightid_{\looseid}}$};
\node at (3.5,2.5) {\scriptsize $=$};
\node at (1,1.5) {\scriptsize $\DDownarrow \overline{\tightid_{\looseid} (M^{\beta} \times (\horl \verc {\horr}^{-1}) }$};
\node at (3,1.5) {\scriptsize $=$};
\node at (.5,.5) {\scriptsize $=$};
\node at (2.5,0.5) {\scriptsize $\DDownarrow \gamma^g$};
\end{tikzpicture}
\end{aligned}
\end{equation*}


\end{document} 
 \newpage

%
\documentclass[12pt]{ociamthesis}
\usepackage{tikz}
\usepackage{amsmath}
\usepackage{rotating}

\usepackage{amssymb,amsmath,stmaryrd,txfonts,mathrsfs,amsthm}
\usepackage[all,2cell]{xy}
\usepackage[neveradjust]{paralist}
\usepackage{hyperref}
\usepackage{mathtools}
\usepackage{tikz}
\usetikzlibrary{trees}
\usetikzlibrary{topaths}
\usetikzlibrary{decorations}
\usetikzlibrary{decorations.pathreplacing}
\usetikzlibrary{decorations.pathmorphing}
\usetikzlibrary{decorations.markings}
\usetikzlibrary{matrix,backgrounds,folding}
\usetikzlibrary{chains,scopes,positioning,fit}
\usetikzlibrary{arrows,shadows}
\usetikzlibrary{calc} 
\usetikzlibrary{chains}
\usetikzlibrary{shapes,shapes.geometric,shapes.misc}
\usepackage{smbicat}


\makeatletter
\let\ea\expandafter

%% Defining commands that are always in math mode.
\def\mdef#1#2{\ea\ea\ea\gdef\ea\ea\noexpand#1\ea{\ea\ensuremath\ea{#2}}}
\def\alwaysmath#1{\ea\ea\ea\global\ea\ea\ea\let\ea\ea\csname your@#1\endcsname\csname #1\endcsname
  \ea\def\csname #1\endcsname{\ensuremath{\csname your@#1\endcsname}}}

% Script letters
\newcommand{\sA}{\ensuremath{\mathscr{A}}}
\newcommand{\sB}{\ensuremath{\mathscr{B}}}
\newcommand{\sC}{\ensuremath{\mathscr{C}}}
\newcommand{\sD}{\ensuremath{\mathscr{D}}}
\newcommand{\sE}{\ensuremath{\mathscr{E}}}
\newcommand{\sF}{\ensuremath{\mathscr{F}}}
\newcommand{\sG}{\ensuremath{\mathscr{G}}}
\newcommand{\sH}{\ensuremath{\mathscr{H}}}
\newcommand{\sI}{\ensuremath{\mathscr{I}}}
\newcommand{\sJ}{\ensuremath{\mathscr{J}}}
\newcommand{\sK}{\ensuremath{\mathscr{K}}}
\newcommand{\sL}{\ensuremath{\mathscr{L}}}
\newcommand{\sM}{\ensuremath{\mathscr{M}}}
\newcommand{\sN}{\ensuremath{\mathscr{N}}}
\newcommand{\sO}{\ensuremath{\mathscr{O}}}
\newcommand{\sP}{\ensuremath{\mathscr{P}}}
\newcommand{\sQ}{\ensuremath{\mathscr{Q}}}
\newcommand{\sR}{\ensuremath{\mathscr{R}}}
\newcommand{\sS}{\ensuremath{\mathscr{S}}}
\newcommand{\sT}{\ensuremath{\mathscr{T}}}
\newcommand{\sU}{\ensuremath{\mathscr{U}}}
\newcommand{\sV}{\ensuremath{\mathscr{V}}}
\newcommand{\sW}{\ensuremath{\mathscr{W}}}
\newcommand{\sX}{\ensuremath{\mathscr{X}}}
\newcommand{\sY}{\ensuremath{\mathscr{Y}}}
\newcommand{\sZ}{\ensuremath{\mathscr{Z}}}

% Calligraphic letters
\newcommand{\cA}{\ensuremath{\mathcal{A}}}
\newcommand{\cB}{\ensuremath{\mathcal{B}}}
\newcommand{\cC}{\ensuremath{\mathcal{C}}}
\newcommand{\cD}{\ensuremath{\mathcal{D}}}
\newcommand{\cE}{\ensuremath{\mathcal{E}}}
\newcommand{\cF}{\ensuremath{\mathcal{F}}}
\newcommand{\cG}{\ensuremath{\mathcal{G}}}
\newcommand{\cH}{\ensuremath{\mathcal{H}}}
\newcommand{\cI}{\ensuremath{\mathcal{I}}}
\newcommand{\cJ}{\ensuremath{\mathcal{J}}}
\newcommand{\cK}{\ensuremath{\mathcal{K}}}
\newcommand{\cL}{\ensuremath{\mathcal{L}}}
\newcommand{\cM}{\ensuremath{\mathcal{M}}}
\newcommand{\cN}{\ensuremath{\mathcal{N}}}
\newcommand{\cO}{\ensuremath{\mathcal{O}}}
\newcommand{\cP}{\ensuremath{\mathcal{P}}}
\newcommand{\cQ}{\ensuremath{\mathcal{Q}}}
\newcommand{\cR}{\ensuremath{\mathcal{R}}}
\newcommand{\cS}{\ensuremath{\mathcal{S}}}
\newcommand{\cT}{\ensuremath{\mathcal{T}}}
\newcommand{\cU}{\ensuremath{\mathcal{U}}}
\newcommand{\cV}{\ensuremath{\mathcal{V}}}
\newcommand{\cW}{\ensuremath{\mathcal{W}}}
\newcommand{\cX}{\ensuremath{\mathcal{X}}}
\newcommand{\cY}{\ensuremath{\mathcal{Y}}}
\newcommand{\cZ}{\ensuremath{\mathcal{Z}}}

% blackboard bold letters
\newcommand{\lA}{\ensuremath{\mathbb{A}}}
\newcommand{\lB}{\ensuremath{\mathbb{B}}}
\newcommand{\lC}{\ensuremath{\mathbb{C}}}
\newcommand{\lD}{\ensuremath{\mathbb{D}}}
\newcommand{\lE}{\ensuremath{\mathbb{E}}}
\newcommand{\lF}{\ensuremath{\mathbb{F}}}
\newcommand{\lG}{\ensuremath{\mathbb{G}}}
\newcommand{\lH}{\ensuremath{\mathbb{H}}}
\newcommand{\lI}{\ensuremath{\mathbb{I}}}
\newcommand{\lJ}{\ensuremath{\mathbb{J}}}
\newcommand{\lK}{\ensuremath{\mathbb{K}}}
\newcommand{\lL}{\ensuremath{\mathbb{L}}}
\newcommand{\lM}{\ensuremath{\mathbb{M}}}
\newcommand{\lN}{\ensuremath{\mathbb{N}}}
\newcommand{\lO}{\ensuremath{\mathbb{O}}}
\newcommand{\lP}{\ensuremath{\mathbb{P}}}
\newcommand{\lQ}{\ensuremath{\mathbb{Q}}}
\newcommand{\lR}{\ensuremath{\mathbb{R}}}
\newcommand{\lS}{\ensuremath{\mathbb{S}}}
\newcommand{\lT}{\ensuremath{\mathbb{T}}}
\newcommand{\lU}{\ensuremath{\mathbb{U}}}
\newcommand{\lV}{\ensuremath{\mathbb{V}}}
\newcommand{\lW}{\ensuremath{\mathbb{W}}}
\newcommand{\lX}{\ensuremath{\mathbb{X}}}
\newcommand{\lY}{\ensuremath{\mathbb{Y}}}
\newcommand{\lZ}{\ensuremath{\mathbb{Z}}}

% bold letters
\newcommand{\bA}{\ensuremath{\mathbf{A}}}
\newcommand{\bB}{\ensuremath{\mathbf{B}}}
\newcommand{\bC}{\ensuremath{\mathbf{C}}}
\newcommand{\bD}{\ensuremath{\mathbf{D}}}
\newcommand{\bE}{\ensuremath{\mathbf{E}}}
\newcommand{\bF}{\ensuremath{\mathbf{F}}}
\newcommand{\bG}{\ensuremath{\mathbf{G}}}
\newcommand{\bH}{\ensuremath{\mathbf{H}}}
\newcommand{\bI}{\ensuremath{\mathbf{I}}}
\newcommand{\bJ}{\ensuremath{\mathbf{J}}}
\newcommand{\bK}{\ensuremath{\mathbf{K}}}
\newcommand{\bL}{\ensuremath{\mathbf{L}}}
\newcommand{\bM}{\ensuremath{\mathbf{M}}}
\newcommand{\bN}{\ensuremath{\mathbf{N}}}
\newcommand{\bO}{\ensuremath{\mathbf{O}}}
\newcommand{\bP}{\ensuremath{\mathbf{P}}}
\newcommand{\bQ}{\ensuremath{\mathbf{Q}}}
\newcommand{\bR}{\ensuremath{\mathbf{R}}}
\newcommand{\bS}{\ensuremath{\mathbf{S}}}
\newcommand{\bT}{\ensuremath{\mathbf{T}}}
\newcommand{\bU}{\ensuremath{\mathbf{U}}}
\newcommand{\bV}{\ensuremath{\mathbf{V}}}
\newcommand{\bW}{\ensuremath{\mathbf{W}}}
\newcommand{\bX}{\ensuremath{\mathbf{X}}}
\newcommand{\bY}{\ensuremath{\mathbf{Y}}}
\newcommand{\bZ}{\ensuremath{\mathbf{Z}}}

% fraktur letters
\newcommand{\fa}{\ensuremath{\mathfrak{a}}}
\newcommand{\fb}{\ensuremath{\mathfrak{b}}}
\newcommand{\fc}{\ensuremath{\mathfrak{c}}}
\newcommand{\fd}{\ensuremath{\mathfrak{d}}}
\newcommand{\fe}{\ensuremath{\mathfrak{e}}}
\newcommand{\ff}{\ensuremath{\mathfrak{f}}}
\newcommand{\fg}{\ensuremath{\mathfrak{g}}}
\newcommand{\fh}{\ensuremath{\mathfrak{h}}}
\newcommand{\fj}{\ensuremath{\mathfrak{j}}}
\newcommand{\fk}{\ensuremath{\mathfrak{k}}}
\newcommand{\fl}{\ensuremath{\mathfrak{l}}}
\newcommand{\fm}{\ensuremath{\mathfrak{m}}}
\newcommand{\fn}{\ensuremath{\mathfrak{n}}}
\newcommand{\fo}{\ensuremath{\mathfrak{o}}}
\newcommand{\fp}{\ensuremath{\mathfrak{p}}}
\newcommand{\fq}{\ensuremath{\mathfrak{q}}}
\newcommand{\fr}{\ensuremath{\mathfrak{r}}}
\newcommand{\fs}{\ensuremath{\mathfrak{s}}}
\newcommand{\ft}{\ensuremath{\mathfrak{t}}}
\newcommand{\fu}{\ensuremath{\mathfrak{u}}}
\newcommand{\fv}{\ensuremath{\mathfrak{v}}}
\newcommand{\fw}{\ensuremath{\mathfrak{w}}}
\newcommand{\fx}{\ensuremath{\mathfrak{x}}}
\newcommand{\fy}{\ensuremath{\mathfrak{y}}}
\newcommand{\fz}{\ensuremath{\mathfrak{z}}}

% fraktur letters
\newcommand{\fA}{\ensuremath{\mathfrak{A}}}
\newcommand{\fB}{\ensuremath{\mathfrak{B}}}
\newcommand{\fC}{\ensuremath{\mathfrak{C}}}

\mdef\fahat{\hat{\fa}}

% Underline letters
\newcommand{\uA}{\ensuremath{\underline{A}}}
\newcommand{\uB}{\ensuremath{\underline{B}}}
\newcommand{\uC}{\ensuremath{\underline{C}}}
\newcommand{\uD}{\ensuremath{\underline{D}}}
\newcommand{\uE}{\ensuremath{\underline{E}}}
\newcommand{\uF}{\ensuremath{\underline{F}}}
\newcommand{\uG}{\ensuremath{\underline{G}}}
\newcommand{\uH}{\ensuremath{\underline{H}}}
\newcommand{\uI}{\ensuremath{\underline{I}}}
\newcommand{\uJ}{\ensuremath{\underline{J}}}
\newcommand{\uK}{\ensuremath{\underline{K}}}
\newcommand{\uL}{\ensuremath{\underline{L}}}
\newcommand{\uM}{\ensuremath{\underline{M}}}
\newcommand{\uN}{\ensuremath{\underline{N}}}
\newcommand{\uO}{\ensuremath{\underline{O}}}
\newcommand{\uP}{\ensuremath{\underline{P}}}
\newcommand{\uQ}{\ensuremath{\underline{Q}}}
\newcommand{\uR}{\ensuremath{\underline{R}}}
\newcommand{\uS}{\ensuremath{\underline{S}}}
\newcommand{\uT}{\ensuremath{\underline{T}}}
\newcommand{\uU}{\ensuremath{\underline{U}}}
\newcommand{\uV}{\ensuremath{\underline{V}}}
\newcommand{\uW}{\ensuremath{\underline{W}}}
\newcommand{\uX}{\ensuremath{\underline{X}}}
\newcommand{\uY}{\ensuremath{\underline{Y}}}
\newcommand{\uZ}{\ensuremath{\underline{Z}}}

% bars
\newcommand{\Abar}{\ensuremath{\overline{A}}}
\newcommand{\Bbar}{\ensuremath{\overline{B}}}
\newcommand{\Cbar}{\ensuremath{\overline{C}}}
\newcommand{\Dbar}{\ensuremath{\overline{D}}}
\newcommand{\Ebar}{\ensuremath{\overline{E}}}
\newcommand{\Fbar}{\ensuremath{\overline{F}}}
\newcommand{\Gbar}{\ensuremath{\overline{G}}}
\newcommand{\Hbar}{\ensuremath{\overline{H}}}
\newcommand{\Ibar}{\ensuremath{\overline{I}}}
\newcommand{\Jbar}{\ensuremath{\overline{J}}}
\newcommand{\Kbar}{\ensuremath{\overline{K}}}
\newcommand{\Lbar}{\ensuremath{\overline{L}}}
\newcommand{\Mbar}{\ensuremath{\overline{M}}}
\newcommand{\Nbar}{\ensuremath{\overline{N}}}
\newcommand{\Obar}{\ensuremath{\overline{O}}}
\newcommand{\Pbar}{\ensuremath{\overline{P}}}
\newcommand{\Qbar}{\ensuremath{\overline{Q}}}
\newcommand{\Rbar}{\ensuremath{\overline{R}}}
\newcommand{\Sbar}{\ensuremath{\overline{S}}}
\newcommand{\Tbar}{\ensuremath{\overline{T}}}
\newcommand{\Ubar}{\ensuremath{\overline{U}}}
\newcommand{\Vbar}{\ensuremath{\overline{V}}}
\newcommand{\Wbar}{\ensuremath{\overline{W}}}
\newcommand{\Xbar}{\ensuremath{\overline{X}}}
\newcommand{\Ybar}{\ensuremath{\overline{Y}}}
\newcommand{\Zbar}{\ensuremath{\overline{Z}}}
\newcommand{\abar}{\ensuremath{\overline{a}}}
\newcommand{\bbar}{\ensuremath{\overline{b}}}
\newcommand{\cbar}{\ensuremath{\overline{c}}}
\newcommand{\dbar}{\ensuremath{\overline{d}}}
\newcommand{\ebar}{\ensuremath{\overline{e}}}
\newcommand{\fbar}{\ensuremath{\overline{f}}}
\newcommand{\gbar}{\ensuremath{\overline{g}}}
%\newcommand{\hbar}{\ensuremath{\overline{h}}} % whoops, \hbar means something else!
\newcommand{\ibar}{\ensuremath{\overline{\imath}}}
\newcommand{\jbar}{\ensuremath{\overline{\jmath}}}
\newcommand{\kbar}{\ensuremath{\overline{k}}}
\newcommand{\lbar}{\ensuremath{\overline{l}}}
\newcommand{\mbar}{\ensuremath{\overline{m}}}
\newcommand{\nbar}{\ensuremath{\overline{n}}}
%\newcommand{\obar}{\ensuremath{\overline{o}}}
\newcommand{\pbar}{\ensuremath{\overline{p}}}
\newcommand{\qbar}{\ensuremath{\overline{q}}}
\newcommand{\rbar}{\ensuremath{\overline{r}}}
\newcommand{\sbar}{\ensuremath{\overline{s}}}
\newcommand{\tbar}{\ensuremath{\overline{t}}}
\newcommand{\ubar}{\ensuremath{\overline{u}}}
\newcommand{\vbar}{\ensuremath{\overline{v}}}
\newcommand{\wbar}{\ensuremath{\overline{w}}}
\newcommand{\xbar}{\ensuremath{\overline{x}}}
\newcommand{\ybar}{\ensuremath{\overline{y}}}
\newcommand{\zbar}{\ensuremath{\overline{z}}}

% tildes
\newcommand{\Atil}{\ensuremath{\widetilde{A}}}
\newcommand{\Btil}{\ensuremath{\widetilde{B}}}
\newcommand{\Ctil}{\ensuremath{\widetilde{C}}}
\newcommand{\Dtil}{\ensuremath{\widetilde{D}}}
\newcommand{\Etil}{\ensuremath{\widetilde{E}}}
\newcommand{\Ftil}{\ensuremath{\widetilde{F}}}
\newcommand{\Gtil}{\ensuremath{\widetilde{G}}}
\newcommand{\Htil}{\ensuremath{\widetilde{H}}}
\newcommand{\Itil}{\ensuremath{\widetilde{I}}}
\newcommand{\Jtil}{\ensuremath{\widetilde{J}}}
\newcommand{\Ktil}{\ensuremath{\widetilde{K}}}
\newcommand{\Ltil}{\ensuremath{\widetilde{L}}}
\newcommand{\Mtil}{\ensuremath{\widetilde{M}}}
\newcommand{\Ntil}{\ensuremath{\widetilde{N}}}
\newcommand{\Otil}{\ensuremath{\widetilde{O}}}
\newcommand{\Ptil}{\ensuremath{\widetilde{P}}}
\newcommand{\Qtil}{\ensuremath{\widetilde{Q}}}
\newcommand{\Rtil}{\ensuremath{\widetilde{R}}}
\newcommand{\Stil}{\ensuremath{\widetilde{S}}}
\newcommand{\Ttil}{\ensuremath{\widetilde{T}}}
\newcommand{\Util}{\ensuremath{\widetilde{U}}}
\newcommand{\Vtil}{\ensuremath{\widetilde{V}}}
\newcommand{\Wtil}{\ensuremath{\widetilde{W}}}
\newcommand{\Xtil}{\ensuremath{\widetilde{X}}}
\newcommand{\Ytil}{\ensuremath{\widetilde{Y}}}
\newcommand{\Ztil}{\ensuremath{\widetilde{Z}}}
\newcommand{\atil}{\ensuremath{\widetilde{a}}}
\newcommand{\btil}{\ensuremath{\widetilde{b}}}
\newcommand{\ctil}{\ensuremath{\widetilde{c}}}
\newcommand{\dtil}{\ensuremath{\widetilde{d}}}
\newcommand{\etil}{\ensuremath{\widetilde{e}}}
\newcommand{\ftil}{\ensuremath{\widetilde{f}}}
\newcommand{\gtil}{\ensuremath{\widetilde{g}}}
\newcommand{\htil}{\ensuremath{\widetilde{h}}}
\newcommand{\itil}{\ensuremath{\widetilde{\imath}}}
\newcommand{\jtil}{\ensuremath{\widetilde{\jmath}}}
\newcommand{\ktil}{\ensuremath{\widetilde{k}}}
\newcommand{\ltil}{\ensuremath{\widetilde{l}}}
\newcommand{\mtil}{\ensuremath{\widetilde{m}}}
\newcommand{\ntil}{\ensuremath{\widetilde{n}}}
\newcommand{\otil}{\ensuremath{\widetilde{o}}}
\newcommand{\ptil}{\ensuremath{\widetilde{p}}}
\newcommand{\qtil}{\ensuremath{\widetilde{q}}}
\newcommand{\rtil}{\ensuremath{\widetilde{r}}}
\newcommand{\stil}{\ensuremath{\widetilde{s}}}
\newcommand{\ttil}{\ensuremath{\widetilde{t}}}
\newcommand{\util}{\ensuremath{\widetilde{u}}}
\newcommand{\vtil}{\ensuremath{\widetilde{v}}}
\newcommand{\wtil}{\ensuremath{\widetilde{w}}}
\newcommand{\xtil}{\ensuremath{\widetilde{x}}}
\newcommand{\ytil}{\ensuremath{\widetilde{y}}}
\newcommand{\ztil}{\ensuremath{\widetilde{z}}}

% Hats
\newcommand{\Ahat}{\ensuremath{\widehat{A}}}
\newcommand{\Bhat}{\ensuremath{\widehat{B}}}
\newcommand{\Chat}{\ensuremath{\widehat{C}}}
\newcommand{\Dhat}{\ensuremath{\widehat{D}}}
\newcommand{\Ehat}{\ensuremath{\widehat{E}}}
\newcommand{\Fhat}{\ensuremath{\widehat{F}}}
\newcommand{\Ghat}{\ensuremath{\widehat{G}}}
\newcommand{\Hhat}{\ensuremath{\widehat{H}}}
\newcommand{\Ihat}{\ensuremath{\widehat{I}}}
\newcommand{\Jhat}{\ensuremath{\widehat{J}}}
\newcommand{\Khat}{\ensuremath{\widehat{K}}}
\newcommand{\Lhat}{\ensuremath{\widehat{L}}}
\newcommand{\Mhat}{\ensuremath{\widehat{M}}}
\newcommand{\Nhat}{\ensuremath{\widehat{N}}}
\newcommand{\Ohat}{\ensuremath{\widehat{O}}}
\newcommand{\Phat}{\ensuremath{\widehat{P}}}
\newcommand{\Qhat}{\ensuremath{\widehat{Q}}}
\newcommand{\Rhat}{\ensuremath{\widehat{R}}}
\newcommand{\Shat}{\ensuremath{\widehat{S}}}
\newcommand{\That}{\ensuremath{\widehat{T}}}
\newcommand{\Uhat}{\ensuremath{\widehat{U}}}
\newcommand{\Vhat}{\ensuremath{\widehat{V}}}
\newcommand{\What}{\ensuremath{\widehat{W}}}
\newcommand{\Xhat}{\ensuremath{\widehat{X}}}
\newcommand{\Yhat}{\ensuremath{\widehat{Y}}}
\newcommand{\Zhat}{\ensuremath{\widehat{Z}}}
\newcommand{\ahat}{\ensuremath{\hat{a}}}
\newcommand{\bhat}{\ensuremath{\hat{b}}}
\newcommand{\chat}{\ensuremath{\hat{c}}}
\newcommand{\dhat}{\ensuremath{\hat{d}}}
\newcommand{\ehat}{\ensuremath{\hat{e}}}
\newcommand{\fhat}{\ensuremath{\hat{f}}}
\newcommand{\ghat}{\ensuremath{\hat{g}}}
\newcommand{\hhat}{\ensuremath{\hat{h}}}
\newcommand{\ihat}{\ensuremath{\hat{\imath}}}
\newcommand{\jhat}{\ensuremath{\hat{\jmath}}}
\newcommand{\khat}{\ensuremath{\hat{k}}}
\newcommand{\lhat}{\ensuremath{\hat{l}}}
\newcommand{\mhat}{\ensuremath{\hat{m}}}
\newcommand{\nhat}{\ensuremath{\hat{n}}}
\newcommand{\ohat}{\ensuremath{\hat{o}}}
\newcommand{\phat}{\ensuremath{\hat{p}}}
\newcommand{\qhat}{\ensuremath{\hat{q}}}
\newcommand{\rhat}{\ensuremath{\hat{r}}}
\newcommand{\shat}{\ensuremath{\hat{s}}}
\newcommand{\that}{\ensuremath{\hat{t}}}
\newcommand{\uhat}{\ensuremath{\hat{u}}}
\newcommand{\vhat}{\ensuremath{\hat{v}}}
\newcommand{\what}{\ensuremath{\hat{w}}}
\newcommand{\xhat}{\ensuremath{\hat{x}}}
\newcommand{\yhat}{\ensuremath{\hat{y}}}
\newcommand{\zhat}{\ensuremath{\hat{z}}}

%% FONTS AND DECORATION FOR GREEK LETTERS

%% the package `mathbbol' gives us blackboard bold greek and numbers,
%% but it does it by redefining \mathbb to use a different font, so that
%% all the other \mathbb letters look different too.  Here we import the
%% font with bb greek and numbers, but assign it a different name,
%% \mathbbb, so as not to replace the usual one.
\DeclareSymbolFont{bbold}{U}{bbold}{m}{n}
\DeclareSymbolFontAlphabet{\mathbbb}{bbold}
\newcommand{\bbDelta}{\ensuremath{\mathbbb{\Delta}}}
\newcommand{\bbone}{\ensuremath{\mathbbb{1}}}
\newcommand{\bbtwo}{\ensuremath{\mathbbb{2}}}
\newcommand{\bbthree}{\ensuremath{\mathbbb{3}}}

% greek with bars
\newcommand{\albar}{\ensuremath{\overline{\alpha}}}
\newcommand{\bebar}{\ensuremath{\overline{\beta}}}
\newcommand{\gmbar}{\ensuremath{\overline{\gamma}}}
\newcommand{\debar}{\ensuremath{\overline{\delta}}}
\newcommand{\phibar}{\ensuremath{\overline{\varphi}}}
\newcommand{\psibar}{\ensuremath{\overline{\psi}}}
\newcommand{\xibar}{\ensuremath{\overline{\xi}}}
\newcommand{\ombar}{\ensuremath{\overline{\omega}}}

% greek with hats
\newcommand{\alhat}{\ensuremath{\hat{\alpha}}}
\newcommand{\behat}{\ensuremath{\hat{\beta}}}
\newcommand{\gmhat}{\ensuremath{\hat{\gamma}}}
\newcommand{\dehat}{\ensuremath{\hat{\delta}}}

% greek with checks
\newcommand{\alchk}{\ensuremath{\check{\alpha}}}
\newcommand{\bechk}{\ensuremath{\check{\beta}}}
\newcommand{\gmchk}{\ensuremath{\check{\gamma}}}
\newcommand{\dechk}{\ensuremath{\check{\delta}}}

% greek with tildes
\newcommand{\altil}{\ensuremath{\widetilde{\alpha}}}
\newcommand{\betil}{\ensuremath{\widetilde{\beta}}}
\newcommand{\gmtil}{\ensuremath{\widetilde{\gamma}}}
\newcommand{\phitil}{\ensuremath{\widetilde{\varphi}}}
\newcommand{\psitil}{\ensuremath{\widetilde{\psi}}}
\newcommand{\xitil}{\ensuremath{\widetilde{\xi}}}
\newcommand{\omtil}{\ensuremath{\widetilde{\omega}}}

% MISCELLANEOUS SYMBOLS
\mdef\del{\partial}
\mdef\delbar{\overline{\partial}}
\let\sm\wedge
\newcommand{\dd}[1]{\ensuremath{\frac{\partial}{\partial {#1}}}}
\newcommand{\inv}{^{-1}}
\newcommand{\dual}{^{\vee}}
\mdef\hf{\textstyle\frac{1}{2}}
\mdef\thrd{\textstyle\frac{1}{3}}
\mdef\qtr{\textstyle\frac{1}{4}}
\let\meet\wedge
\let\join\vee
\let\dn\downarrow
\newcommand{\op}{^{\mathit{op}}}
\newcommand{\co}{^{\mathit{co}}}
\newcommand{\coop}{^{\mathit{coop}}}
\let\adj\dashv
\SelectTips{cm}{}
\newdir{ >}{{}*!/-10pt/@{>}}    % extra spacing for tail arrows in XYpic
\newcommand{\pushoutcorner}[1][dr]{\save*!/#1+1.2pc/#1:(1,-1)@^{|-}\restore}
\newcommand{\pullbackcorner}[1][dr]{\save*!/#1-1.2pc/#1:(-1,1)@^{|-}\restore}
\let\iso\cong
\let\eqv\simeq
\let\cng\equiv
\mdef\Id{\mathrm{Id}}
\mdef\id{\mathrm{id}}
\alwaysmath{ell}
\alwaysmath{infty}
\alwaysmath{odot}
\def\frc#1/#2.{\frac{#1}{#2}}   % \frc x^2+1 / x^2-1 .
\mdef\ten{\mathrel{\otimes}}
\mdef\bigten{\bigotimes}
\mdef\sqten{\mathrel{\boxtimes}}
\def\pow(#1,#2){\mathop{\pitchfork}(#1,#2)} % powers and
\def\cpw{\mathop{\odot}}                    % copowers
\newcommand{\mathid}{\mbox{id}}
\newcommand{\cat}[1]{\ensuremath{\mathbf{#1}}}


%% OPERATORS
\DeclareMathOperator\lan{Lan}
\DeclareMathOperator\ran{Ran}
\DeclareMathOperator\colim{colim}
\DeclareMathOperator\coeq{coeq}
\DeclareMathOperator\eq{eq}
\DeclareMathOperator\Tot{Tot}
\DeclareMathOperator\cosk{cosk}
\DeclareMathOperator\sk{sk}
\DeclareMathOperator\im{im}
\DeclareMathOperator\Spec{Spec}
\DeclareMathOperator\Ho{Ho}
\DeclareMathOperator\Aut{Aut}
\DeclareMathOperator\End{End}
\DeclareMathOperator\Hom{Hom}
\DeclareMathOperator\Map{Map}

%% TIKZ ARROWS AND HIGHER CELLS
\makeatletter
\def\slashedarrowfill@#1#2#3#4#5{%
  $\m@th\thickmuskip0mu\medmuskip\thickmuskip\thinmuskip\thickmuskip
   \relax#5#1\mkern-7mu%
   \cleaders\hbox{$#5\mkern-2mu#2\mkern-2mu$}\hfill
   \mathclap{#3}\mathclap{#2}%
   \cleaders\hbox{$#5\mkern-2mu#2\mkern-2mu$}\hfill
   \mkern-7mu#4$%
}

\def\Rightslashedarrowfill@{%
  \slashedarrowfill@\Relbar\Relbar\Mapstochar\Rightarrow}
\newcommand\xslashedRightarrow[2][]{%
  \ext@arrow 0055{\Rightslashedarrowfill@}{#1}{#2}}
\def\hTo{\xslashedRightarrow{}}
\def\hToo{\xslashedRightarrow{\quad}}
\let\xhTo\xslashedRightarrow

\pagestyle{empty}

\newcommand{\Rightthreecell}{\RRightarrow}
\newcommand{\Rtwocell}{\Rightarrow}

\tikzstyle{doubletick}=[-implies, double equal sign distance, postaction={decorate},decoration={markings,mark=at position .5 with {\draw[-] (0,-0.1) -- (0,0.1);}}]

\tikzstyle{darrow}=[-implies, double equal sign distance]

\tikzstyle{doubleeq}=[double equal sign distance]


%% ARROWS
% \to already exists
\newcommand{\too}[1][]{\ensuremath{\overset{#1}{\longrightarrow}}}
\newcommand{\ot}{\ensuremath{\leftarrow}}
\newcommand{\oot}[1][]{\ensuremath{\overset{#1}{\longleftarrow}}}
\let\toot\rightleftarrows
\let\otto\leftrightarrows
\let\Impl\Rightarrow
\let\imp\Rightarrow
\let\toto\rightrightarrows
\let\into\hookrightarrow
\let\xinto\xhookrightarrow
\mdef\we{\overset{\sim}{\longrightarrow}}
\mdef\leftwe{\overset{\sim}{\longleftarrow}}
\let\mono\rightarrowtail
\let\leftmono\leftarrowtail
\let\cof\rightarrowtail
\let\leftcof\leftarrowtail
\let\epi\twoheadrightarrow
\let\leftepi\twoheadleftarrow
\let\fib\twoheadrightarrow
\let\leftfib\twoheadleftarrow
\let\cohto\rightsquigarrow
\let\maps\colon
\newcommand{\spam}{\,:\!}       % \maps for left arrows

\newsavebox{\DDownarrowbox}
\savebox{\DDownarrowbox}{\tikz[scale=1.5]{\draw[-implies,double equal
sign distance] (0,.1) -- (0,-.1); \draw (0,.1) -- (0,-.1);}}
\newcommand{\DDownarrow}{\mathrel{\raisebox{-.2em}{\usebox{\DDownarrowbox}}}}

\newsavebox{\RRightarrowbox}
\savebox{\RRightarrowbox}{\tikz[scale=1.5]{\draw[-implies,double equal
sign distance] (-.1,0) -- (.1,0); \draw (-.1,0) -- (.1,0);}}
\newcommand{\RRightarrow}{\mathrel{\raisebox{.2em}{\usebox{\RRightarrowbox}}}}

%\newsavebox{\Rightslashedarrowbox}
%\savebox{\Rightslashedarrowbox}{\tikz[scale=1.5]{\draw[Rightslashedarrow{}] (-.1,0) -- (1,0);}}
%\newcommand{\Rightslashedarrow}{\mathrel{\raisebox{-.2em}%{\usebox{\Rightslashedarrowbox}}}}


%% EXTENSIBLE ARROWS
\let\xto\xrightarrow
\let\xot\xleftarrow
% See Voss' Mathmode.tex for instructions on how to create new
% extensible arrows.
\def\rightarrowtailfill@{\arrowfill@{\Yright\joinrel\relbar}\relbar\rightarrow}
\newcommand\xrightarrowtail[2][]{\ext@arrow 0055{\rightarrowtailfill@}{#1}{#2}}
\let\xmono\xrightarrowtail
\let\xcof\xrightarrowtail
\def\twoheadrightarrowfill@{\arrowfill@{\relbar\joinrel\relbar}\relbar\twoheadrightarrow}
\newcommand\xtwoheadrightarrow[2][]{\ext@arrow 0055{\twoheadrightarrowfill@}{#1}{#2}}
\let\xepi\xtwoheadrightarrow
\let\xfib\xtwoheadrightarrow
% Let's leave the left-going ones until I need them.

%% EXTENSIBLE SLASHED ARROWS
% Making extensible slashed arrows, by modifying the underlying AMS code.
% Arguments are:
% 1 = arrowhead on the left (\relbar or \Relbar if none)
% 2 = fill character (usually \relbar or \Relbar)
% 3 = slash character (such as \mapstochar or \Mapstochar)
% 4 = arrowhead on the left (\relbar or \Relbar if none)
% 5 = display mode (\displaystyle etc)
\def\slashedarrowfill@#1#2#3#4#5{%
  $\m@th\thickmuskip0mu\medmuskip\thickmuskip\thinmuskip\thickmuskip
   \relax#5#1\mkern-7mu%
   \cleaders\hbox{$#5\mkern-2mu#2\mkern-2mu$}\hfill
   \mathclap{#3}\mathclap{#2}%
   \cleaders\hbox{$#5\mkern-2mu#2\mkern-2mu$}\hfill
   \mkern-7mu#4$%
}
% Here's the idea: \<slashed>arrowfill@ should be a box containing
% some stretchable space that is the "middle of the arrow".  This
% space is created as a "leader" using \cleader<thing>\hfill, which
% fills an \hfill of space with copies of <thing>.  Here instead of
% just one \cleader, we use two, with the slash in between (and an
% extra copy of the filler, to avoid extra space around the slash).
\def\rightslashedarrowfill@{%
  \slashedarrowfill@\relbar\relbar\mapstochar\rightarrow}
\newcommand\xslashedrightarrow[2][]{%
  \ext@arrow 0055{\rightslashedarrowfill@}{#1}{#2}}
\mdef\hto{\xslashedrightarrow{}}
\mdef\htoo{\xslashedrightarrow{\quad}}
\let\xhto\xslashedrightarrow

%% To get a slashed arrow in XYpic, do
% \[\xymatrix{A \ar[r]|-@{|} & B}\]

% ISOMORPHISMS
\def\xiso#1{\mathrel{\mathrlap{\smash{\xto[\smash{\raisebox{1.3mm}{$\scriptstyle\sim$}}]{#1}}}\hphantom{\xto{#1}}}}
\def\toiso{\xto{\smash{\raisebox{-.5mm}{$\scriptstyle\sim$}}}}

% SHADOWS
\def\shvar#1#2{{\ensuremath{%
  \hspace{1mm}\makebox[-1mm]{$#1\langle$}\makebox[0mm]{$#1\langle$}\hspace{1mm}%
  {#2}%
  \makebox[1mm]{$#1\rangle$}\makebox[0mm]{$#1\rangle$}%
}}}
\def\sh{\shvar{}}
\def\scriptsh{\shvar{\scriptstyle}}
\def\bigsh{\shvar{\big}}
\def\Bigsh{\shvar{\Big}}
\def\biggsh{\shvar{\bigg}}
\def\Biggsh{\shvar{\Bigg}}

%HIGHER CELLS



% THEOREM-TYPE ENVIRONMENTS, hacked to
%% (a) number all with the same numbers, and
%% (b) have the right names for autoref
\def\defthm#1#2{%
  \newtheorem{#1}{#2}[section]%
  \expandafter\def\csname #1autorefname\endcsname{#2}%
  \expandafter\let\csname c@#1\endcsname\c@thm}
\newtheorem{thm}{Theorem}[section]
\newcommand{\thmautorefname}{Theorem}
\defthm{cor}{Corollary}
\defthm{prop}{Proposition}
\defthm{lem}{Lemma}
\defthm{sch}{Scholium}
\defthm{assume}{Assumption}
\defthm{claim}{Claim}
\defthm{conj}{Conjecture}
\defthm{hyp}{Hypothesis}
\defthm{fact}{Fact}
\theoremstyle{definition}
\defthm{defn}{Definition}
\defthm{notn}{Notation}
\theoremstyle{remark}
\defthm{rmk}{Remark}
\defthm{eg}{Example}
\defthm{egs}{Examples}
\defthm{ex}{Exercise}
\defthm{ceg}{Counterexample}

% How to get QED symbols inside equations at the end of the statements
% of theorems.  AMS LaTeX knows how to do this inside equations at the
% end of *proofs* with \qedhere, and at the end of the statement of a
% theorem with \qed (meaning no proof will be given), but it can't
% seem to combine the two.  Use this instead.
\def\thmqedhere{\expandafter\csname\csname @currenvir\endcsname @qed\endcsname}

% Number numbered lists as (i), (ii), ...
\renewcommand{\theenumi}{(\roman{enumi})}
\renewcommand{\labelenumi}{\theenumi}

%% Labeling that keeps track of theorem-type names.  Superseded by
%% autoref from hyperref, as above, but we keep this in case we are
%% using a journal style file that is incompatible with hyperref.
% 
% \ifx\SK@label\undefined\let\SK@label\label\fi
% \let\your@thm\@thm
% \def\@thm#1#2#3{\gdef\currthmtype{#3}\your@thm{#1}{#2}{#3}}
% \def\xlabel#1{{\let\your@currentlabel\@currentlabel\def\@currentlabel
% {\currthmtype~\your@currentlabel}
% \SK@label{#1@}}\label{#1}}
% \def\xref#1{\ref{#1@}}

% Also number formulas with the theorem counter
\let\c@equation\c@thm
\numberwithin{equation}{section}

% Only show numbers for equations that are actually referenced (or
% whose tags are specified manually) - uses mathtools.
\mathtoolsset{showonlyrefs,showmanualtags}

%% Fix enumerate spacing with paralist.  This has two parts:
%%   1. enable mixing of "old spacing" lists with those adjusted by paralist
%%   2. allow us to specify a number based on which to adjust the spacing
%% For the first, use \killspacingtrue when you want the spacing
%% adjusted, then \killspacingfalse to turn adjustment off.  For the
%% second, use \maxenum=14 to set the widest number you want the
%% spacing to be calculated with.
\newlength\oldleftmargini       % save old spacing
\newlength\oldleftmarginii
\newlength\oldleftmarginiii
\newlength\oldleftmarginiv
\newlength\oldleftmarginv
\newlength\oldleftmarginvi
\newcount\maxenum
\maxenum=7
\newif\ifkillspacing
\def\@adjust@enum@labelwidth{%
  \advance\@listdepth by 1\relax
  \ifkillspacing                % do the paralist thing
    \csname c@\@enumctr\endcsname\maxenum
    \settowidth{\@tempdima}{%
      \csname label\@enumctr\endcsname\hspace{\labelsep}}%
    \csname leftmargin\romannumeral\@listdepth\endcsname
      \@tempdima
  \else                         % otherwise, reset it
    \csname fixspacing\romannumeral\@listdepth\endcsname
  \fi
  \advance\@listdepth by -1\relax}
% these commands, one for each enum level (I couldn't get a generic
% one to work), test whether oldleftmargin has been set yet, and if
% not, set it from leftmargin; otherwise, they reset leftmargin to
% it.  Just setting oldleftmargin to leftmargin in the preamble
% doesn't seem to work.
\def\fixspacingi{\ifnum\oldleftmargini=0\setlength\oldleftmargini\leftmargini\else\setlength\leftmargini\oldleftmargini\fi}
\def\fixspacingii{\ifnum\oldleftmarginii=0\setlength\oldleftmarginii\leftmarginii\else\setlength\leftmarginii\oldleftmarginii\fi}
\def\fixspacingiii{\ifnum\oldleftmarginiii=0\setlength\oldleftmarginiii\leftmarginiii\else\setlength\leftmarginiii\oldleftmarginiii\fi}
\def\fixspacingiv{\ifnum\oldleftmarginiv=0\setlength\oldleftmarginiv\leftmarginiv\else\setlength\leftmarginiv\oldleftmarginiv\fi}
\def\fixspacingv{\ifnum\oldleftmarginv=0\setlength\oldleftmarginv\leftmarginv\else\setlength\leftmarginv\oldleftmarginv\fi}
\def\fixspacingvi{\ifnum\oldleftmarginvi=0\setlength\oldleftmarginvi\leftmarginvi\else\setlength\leftmarginvi\oldleftmarginvi\fi}

%% Fix paralist references, so that we can refer to (1) instead of
%% just 1.
\def\pl@label#1#2{%
  \edef\pl@the{\noexpand#1{\@enumctr}}%
  \pl@lab\expandafter{\the\pl@lab\csname yourthe\@enumctr\endcsname}%
  \advance\@tempcnta1
  \pl@loop}
\def\@enumlabel@#1[#2]{%
  \@plmylabeltrue
  \@tempcnta0
  \pl@lab{}%
  \let\pl@the\pl@qmark
  \expandafter\pl@loop\@gobble#2\@@@
  \ifnum\@tempcnta=1\else
    \PackageWarning{paralist}{Incorrect label; no or multiple
      counters.\MessageBreak The label is: \@gobble#2}%
  \fi
  \expandafter\edef\csname label\@enumctr\endcsname{\the\pl@lab}%
  \expandafter\edef\csname the\@enumctr\endcsname{\the\pl@lab}%
  \expandafter\let\csname yourthe\@enumctr\endcsname\pl@the
  #1}


% GREEK LETTERS, ETC.
\alwaysmath{alpha}
\alwaysmath{beta}
\alwaysmath{gamma}
\alwaysmath{Gamma}
\alwaysmath{delta}
\alwaysmath{Delta}
\alwaysmath{epsilon}
\mdef\ep{\varepsilon}
\alwaysmath{zeta}
\alwaysmath{eta}
\alwaysmath{theta}
\alwaysmath{Theta}
\alwaysmath{iota}
\alwaysmath{kappa}
\alwaysmath{lambda}
\alwaysmath{Lambda}
\alwaysmath{mu}
\alwaysmath{nu}
\alwaysmath{xi}
\alwaysmath{pi}
\alwaysmath{rho}
\alwaysmath{sigma}
\alwaysmath{Sigma}
\alwaysmath{tau}
\alwaysmath{upsilon}
\alwaysmath{Upsilon}
\alwaysmath{phi}
\alwaysmath{Pi}
\alwaysmath{Phi}
\mdef\ph{\varphi}
\alwaysmath{chi}
\alwaysmath{psi}
\alwaysmath{Psi}
\alwaysmath{omega}
\alwaysmath{Omega}
\let\al\alpha
\let\be\beta
\let\gm\gamma
\let\Gm\Gamma
\let\de\delta
\let\De\Delta
\let\si\sigma
\let\Si\Sigma
\let\om\omega
\let\ka\kappa
\let\la\lambda
\let\La\Lambda
\let\ze\zeta
\let\th\theta
\let\Th\Theta
\let\vth\vartheta

\makeatother

% Tikz styles
\tikzstyle{tickarrow}=[->,postaction={decorate},decoration={markings,mark=at position .5 with {\draw[-] (0,-0.1) -- (0,0.1);}},line width=0.50]

% Local Variables:
% mode: latex
% TeX-master: ""
% End:

\begin{document}


\begin{equation*}\hspace{-2cm}
\begin{aligned}
\begin{tikzpicture}[xscale=3, yscale=1.5]
\node (t0) at (0,2) {\small $\tens(f\times I_B)$};
\node (t1) at (1,2) {\small $\tens(f\times fI_A)$};
\node (t15) at (2,2) {\small $\tens (f \times f) (\transid \times I)$};
\node (t2) at (3,2) {\small $f\tens(\id \times I_A)$};
\node (t3) at (4,2) {\small $f $};
\node (t4) at (5,2) {\small $g $};
\node (m0) at (0,1) {\small $\tens(\transid \times I_B)f$};
\node (b3) at (5,1) {\small $\tens (\transid \times I_B)g$};
%%%%%%%%%%%%%
\draw[doubleloose] (t0) to node[above]{$\substack{\looseid(\looseid \times \iota)}$} (t1);
\draw[doubleeq] (t1) to  (t15);
\draw[doubleloose] (t15) to node[above]{$\substack{\chi (\looseid \times \looseid)}$} (t2);
\draw[doubleloose] (t2) to node[above]{$\substack{\looseid r}$} (t3);
\draw[doubleloose] (t3) to node[above]{$\substack{\beta}$} (t4);
\draw[doubleloose] (m0) to node[above]{$\substack{r \looseid}$} (t3);
\draw[doubleloose] (m0) to node[above]{$\substack{\looseid_{\tens}(\beta \times \looseid_I)}$} (b3);
\draw[doubleloose] (b3) to node[right]{$\substack{r \looseid}$} (t4);
\draw[doubletighteq] (t0) to (m0);
\node at (1,1.5) {$\substack{\DDownarrow \delta^f}$};
\node at (4,1.5) {$\substack{\iso}$};
\end{tikzpicture}
\end{aligned}\hspace{-2cm}
\end{equation*}
\begin{equation}\label{eq:mon2cell2}
=
\end{equation}
\begin{equation*}\hspace{-2cm}
\begin{aligned}
\begin{tikzpicture}[xscale=3, yscale=1.5]
\node (04) at (0,4) {\small $\tens(f\times I_B)$};
\node (14) at (1,5.5) {\small $\tens(f\times f I_A)$};
\node (154) at (1.5,6) {\small $\tens(f \times f) (\transid \times I_A)$};
\node (24) at (3.5,6.5) {\small $f \tens(\transid \times I_A)$};
\node (34) at (4.5,5.5) {\small $f $};
\node (44) at (5,4) {\small $g $};
%%%%%%
\node (11) at (.5,2) {\small $\tens(g\times  I_B)$};
\node (22) at (2,2.5) {\small $\tens(g\times g I_A)$};
\node (32) at (3,3) {\small $\tens(g\times g) (\transid \times  I_A)$};
\node (33) at (4,4) {\small $g \tens(\transid \times I_A)$};
%%%%%%%
\node (00) at (0,1) {\small $\tens(\transid \times I_B)f$};
\node (10) at (5,1) {\small $\tens(\transid \times  I_B)g$};
%%%%%%%
\draw[doubleloose] (04) to node[above, xshift=-10pt]{$\substack{\looseid (\looseid \times \iota_f)}$} (14);
\draw[doubleeq] (14) to (154);
\draw[doubleloose] (154) to node[above]{$\substack{\chi \looseid}$} (24);
\draw[doubleloose] (24) to node[above]{$\substack{\looseid r}$} (34);
\draw[doubleloose] (34) to node[above, xshift=3pt]{$\substack{\beta}$} (44);
%%%%%%%
\draw[doubleloose] (24) to node[right]{$\substack{\beta \looseid \looseid}$} (33);
\draw[doubleloose] (33) to node[above]{$\substack{\looseid r}$} (44);
%%%%
\draw[doubleloose] (154) to node[below, xshift=-20pt]{$\substack{\looseid (\beta \times \beta)\looseid }$} (32);
\draw[doubleeq] (22) to (32);
\draw[doubleloose] (32) to node[above, xshift=-10pt]{$\substack{\chi \looseid}$} (33);
%%%%%%
\draw[doubleloose] (04) to node[right]{$\substack{\looseid (\beta \times \looseid)}$} (11);
\draw[doubleloose] (11) to node[above, xshift=-10pt]{$\substack{\looseid (\looseid \times \iota_g) }$} (22);
%%%%%%
\draw[doubleloose] (00) to node[above]{$\substack{\looseid  \looseid \beta}$} (10);
\draw[doubleloose] (10) to node[left]{$\substack{r \looseid }$} (44);
\draw[doubleloose] (14) to node[right]{$\substack{ \looseid (\beta \times \beta \looseid)}$} (22);
%%%%%%
\draw[doubletighteq] (04) to (00);
\draw[doubletighteq] (11) to (10);
%%%%%%%%
\node at (4.25,5) {$\substack{\DDownarrow \iso }$};
\node at (3.25,5.75) {$\substack{\DDownarrow\iso }$};
\node at (3,3.75) {$\substack{\DDownarrow\iso }$};
\node at (2,3.75) {$\substack{\DDownarrow\iso }$};
\node at (3,5) {$\substack{\DDownarrow \Pi^{\beta}\tightid}$};
\node at (1,4.75) {$\substack {\DDownarrow \iso }$};
\node at (0.8,3.75) {$\substack {\DDownarrow \tightid_{\looseid} (\tightid \times M^{\beta} ) }$};
\node at (.75,2.75) {$\substack {\DDownarrow \iso }$};
\node at (1,1.5) {$\substack{\DDownarrow \iso}$};
\node at (4,2.5) {$\substack{\DDownarrow \delta^g}$};
%%%%%
%\draw[doubleloose] (11) to node[below, xshift=3pt]{$\substack{S(\delta)\tightid}$} (44);
%\draw[doubleloose] (11) to[in=240, out=-20] node[below, xshift=-3pt]{$\substack{T(\delta)\tightid}$} (44);
%%%%%
\draw[doubleloose] (154) to[in=115, out=0]  node[below, xshift=3pt]{$\substack{S(\Pi)\tightid}$} (33);
\draw[doubleloose] (154) to[in=180
, out=-45] node[below, xshift=-3pt]{$\substack{T(\Pi)\tightid}$} (33);
%%%%%%
\draw[doubleloose] (04) to[in=135, out=20]  node[above, xshift=3pt, yshift=5pt]{$\substack{\looseid (\beta \times S(M)) }$} (22);
\draw[doubleloose] (04) to[in=180
, out=-45] node[above, xshift=3pt,yshift=5pt]{$\substack{\looseid (\beta \times T(M))}$} (22);
\end{tikzpicture}
\end{aligned}\hspace{-2cm}
\end{equation*}

\end{document} 
 \newpage

%
\documentclass[12pt]{ociamthesis}
\usepackage{tikz}
\newcommand{\id}{\mathrm{id}}
\begin{document}

\begin{equation*}
\begin{aligned}
\begin{tikzpicture}[xscale=3.5, yscale=1.5]
\node (04) at (0,4) {\scriptsize$ \tens( \tens \times \transid)(f \times f \times f)$};
\node (14) at (1,4) {\scriptsize $ \tens(f \tens \times f)$};
\node (24) at (2,4) {\scriptsize $f \tens(\tens \times \transid)$};
\node (34) at (3,4) {\scriptsize $f\tens (\transid \times \tens)$};
\node (44) at (4,4) {\scriptsize $g \tens (\transid \times \tens)$};
\node (03) at (0,3) {\scriptsize $\tens( \tens \times \transid)(f \times f \times f)$};
\node (13) at (1,3) {\scriptsize $\tens( \transid \times \tens)(f \times f \times f)$};
\node (23) at (2,3) {\scriptsize $\tens (f \times f \tens)$};
\node (33) at (3,3) {\scriptsize $f \tens (\transid \times  \tens)$};
\node (43) at (4,3) {\scriptsize $g \tens (\transid \times  \tens)$};
\node (02) at (0,2) {\scriptsize $\tens( \tens \times \transid)(f \times f \times f)$};
\node (12) at (1,2) {\scriptsize $\tens( \transid \times \tens)(f \times f \times f)$};
\node (22) at (2,2) {\scriptsize $\tens (f \times f \tens)$};
\node (32) at (3,2) {\scriptsize $\tens (g \times g \tens)$};
\node (42) at (4,2) {\scriptsize $g \tens (\transid \times  \tens)$};
%%%%%%%
\node (01) at (0,1) {\scriptsize $\tens( \tens \times \transid)(f \times f \times f)$};
\node (11) at (1,1) {\scriptsize $\tens( \transid \times \tens)(f \times f \times f)$};
\node (21) at (2,1) {\scriptsize $\tens (\transid \times \tens) (g \times g \times g)$};
\node (31) at (3,1) {\scriptsize $\tens (g \times g \tens)$};
\node (41) at (4,1) {\scriptsize $g \tens (\transid \times  \tens)$};
%%%%%%%
\node (00) at (0,0) {\scriptsize $\tens( \tens \times \transid)(f \times f \times f)$};
\node (10) at (1,0) {\scriptsize $\tens( \transid \times \tens)(g \times g \times g)$};
\node (20) at (2,0) {\scriptsize $\tens (\transid \times \tens) (g \times g \times g)$};
\node (30) at (3,0) {\scriptsize $\tens (g \times g \tens)$};
\node (40) at (4,0) {\scriptsize $g \tens (\transid \times  \tens)$};
%%%%%%%
\draw[doubleloose] (04) to node[above]{\scriptsize $\looseid_{\tens}(\chi_f \times \looseid_f)$} (14);
\draw[doubleloose] (14) to node[above]{\scriptsize $\chi_f \looseid_{\tens \times \transid}$} (24);
\draw[doubleloose] (24) to node[above]{\scriptsize $\looseid_{f}\alpha$} (34);
\draw[doubleloose] (34) to node[above]{\scriptsize $\beta \looseid_{\tens} \looseid_{\transid \times \tens}$} (44);
%%%%%%%%
\draw[doubleloose] (03) to node[above]{\scriptsize $\alpha \looseid_{f \times f \times f}$} (13);
\draw[doubleloose] (13) to node[above]{\scriptsize $\looseid_{\tens} (\looseid_{\transid} \times \chi_f)$} (23);
\draw[doubleloose] (23) to node[above]{\scriptsize $\chi_f \looseid_{\transid \times \tens}$} (33);
\draw[doubleloose] (33) to node[above]{\scriptsize $\beta \looseid_{\tens} \looseid_{\transid \times \tens}$} (43);
%%%%%%%%
\draw[doubleloose] (12) to node[above]{\scriptsize $\looseid_{\tens} (\looseid_{\transid} \times \chi_f)$} (22);
\draw[doubleloose] (22) to node[above]{\scriptsize $\looseid_{\tens} (\beta \times \beta) \looseid_{\transid \times \tens}$} (32);
\draw[doubleloose] (32) to node[above]{\scriptsize $\chi_g \looseid_{\transid \times \tens}$} (42);
%%%%%%%%
\draw[doubleloose] (01) to node[above]{\scriptsize $\alpha \looseid_{f \times f \times f}$} (11);
\draw[doubleloose] (11) to node[above]{\scriptsize $\looseid_{\tens} \looseid_{\transid \times \tens} (\beta \times \beta \times \beta)$} (21);
\draw[doubleloose] (21) to node[above]{\scriptsize $\looseid_{\tens} (\looseid_{\transid} \times \chi_g)$} (31);
%%%%%%%%
\draw[doubleloose] (00) to node[above]{\scriptsize $\looseid_{\tens} (\looseid_{\tens} \times \looseid_{\transid})(\beta \times \beta \times \beta)$} (10);
\draw[doubleloose] (10) to node[above]{\scriptsize $\alpha \looseid_{g \times g \times g}$} (20);
\draw[doubleloose] (20) to node[above]{\scriptsize $\looseid_{\tens} (\looseid_{\transid} \times \chi_g)$} (30);
\draw[doubleloose] (30) to node[above]{\scriptsize $\chi_g \looseid_{\transid \times \tens}$} (40);
%%%%%%%%
\draw[doubletighteq] (04) to (03);
\draw[doubletighteq] (34) to (33);
\draw[doubletighteq] (44) to (43);
%%%%%%%%%
\draw[doubletighteq] (03) to (02);
\draw[doubletighteq] (13) to (12);
\draw[doubletighteq] (23) to (22);
\draw[doubletighteq] (43) to (42);
%%%%%%%%%
\draw[doubletighteq] (02) to (01);
\draw[doubletighteq] (12) to (11);
\draw[doubletighteq] (32) to (31);
\draw[doubletighteq] (42) to (41);
%%%%%%%%%
\draw[doubletighteq] (01) to (00);
\draw[doubletighteq] (21) to (20);
\draw[doubletighteq] (31) to (30);
\draw[doubletighteq] (41) to (40);
%%%%%%%%%
\node at (1.5,3.5) {\scriptsize $\DDownarrow \omega^f$};
\node at (3.5,3.5) {\scriptsize $=$};
\node at (0.5,2) {\scriptsize $=$};
\node at (1.5,2.5) {\scriptsize $=$};
\node at (3,2.5) {\scriptsize $\DDownarrow \overline{\Pi^{\beta} \tightid_{\looseid_{\transid \times \tens}}}$};
\node at (2,1.5) {\scriptsize $\DDownarrow \overline{\tightid_{\looseid} ({\horl}^{-1} \horr \times \Pi^{\beta})}$};
\node at (3.5,1) {\scriptsize $=$};
\node at (1,0.5) {\scriptsize $\DDownarrow ({\horl}^{-1} \verc \horr) ({\horr}^{-1} \verc \horl)$};
\node at (2.5,0.5) {\scriptsize $=$};
\end{tikzpicture}
\end{aligned}
\end{equation*}
\begin{equation}\label{eq:mon2cell3}
=
\end{equation}
\begin{equation*}
\begin{aligned}
\begin{tikzpicture}[xscale=3.5, yscale=1.5]
\node (04) at (0,4) {\scriptsize $\tens( \tens \times \transid)(f \times f \times f)$};
\node (14) at (1,4) {\scriptsize $\tens(f \tens \times f)$};
\node (24) at (2,4) {\scriptsize $f \tens(\tens \times \transid)$};
\node (34) at (3,4) {\scriptsize $f\tens (\transid \times \tens)$};
\node (44) at (4,4) {\scriptsize $g \tens (\transid \times \tens)$};
%%%%%%%%%
\node (03) at (0,3) {\scriptsize $\tens( \tens \times \transid)(f \times f \times f)$};
\node (13) at (1,3) {\scriptsize $\tens( f \tens \times f)$};
\node (23) at (2,3) {\scriptsize $f \tens ( \tens \times \transid)$};
\node (33) at (3,3) {\scriptsize $g \tens (\tens \times  \transid)$};
\node (43) at (4,3) {\scriptsize $g \tens (\transid \times \tens)$};
\node (02) at (0,2) {\scriptsize $\tens( \tens \times \transid)(f \times f \times f)$};
\node (12) at (1,2) {\scriptsize $\tens(f \tens \times f )$};
\node (22) at (2,2) {\scriptsize $\tens (g \tens \times g)$};
\node (32) at (3,2) {\scriptsize $g \tens (\tens \times \transid)$};
\node (42) at (4,2) {\scriptsize $g \tens (\transid \times  \tens)$};
%%%%%%%
\node (01) at (0,1) {\scriptsize $\tens( \tens \times \transid)(f \times f \times f)$};
\node (11) at (1,1) {\scriptsize $\tens( \tens \times \transid)(g \times g \times g)$};
\node (21) at (2,1) {\scriptsize $\tens (g \tens \times g)$};
\node (31) at (3,1) {\scriptsize $g \tens ( \tens \times \transid )$};
\node (41) at (4,1) {\scriptsize $g \tens (\transid \times  \tens)$};
%%%%%%%
\node (00) at (0,0) {\scriptsize $\tens( \tens \times \transid)(f \times f \times f)$};
\node (10) at (1,0) {\scriptsize $\tens( \transid \times \tens)(g \times g \times g)$};
\node (20) at (2,0) {\scriptsize $\tens (\transid \times \tens) (g \times g \times g)$};
\node (30) at (3,0) {\scriptsize $\tens (g \times g \tens)$};
\node (40) at (4,0) {\scriptsize $g \tens (\transid \times  \tens)$};
%%%%%%%
\draw[doubleloose] (04) to node[above]{\scriptsize $\looseid_{\tens}(\chi_f \times \looseid_f)$} (14);
\draw[doubleloose] (14) to node[above]{\scriptsize $\chi_f (\looseid_{\tens \times \transid})$} (24);
\draw[doubleloose] (24) to node[above]{\scriptsize $\looseid_{f}\alpha$} (34);
\draw[doubleloose] (34) to node[above]{\scriptsize $\beta \looseid_{\tens} \looseid_{\transid \times \tens}$} (44);
%%%%%%%%
\draw[doubleloose] (13) to node[above]{\scriptsize $\chi_f \looseid_{\tens \times \transid}$} (23);
\draw[doubleloose] (23) to node[above]{\scriptsize $\beta \looseid_{\tens} \looseid_{\tens \times \transid}$} (33);
\draw[doubleloose] (33) to node[above]{\scriptsize $ \looseid_{g} \alpha$} (43);
%%%%%%%%
\draw[doubleloose] (02) to node[above]{\scriptsize $\looseid_{\tens}(\chi_f \times \looseid_f)$} (12);
\draw[doubleloose] (12) to node[above]{\scriptsize $\looseid_{\tens} (\beta \times \beta) \looseid_{\tens \times \transid}$} (22);
\draw[doubleloose] (22) to node[above]{\scriptsize $\chi_g \looseid_{\tens \times \transid}$} (32);
%%%%%%%%
\draw[doubleloose] (01) to node[above]{\scriptsize $\looseid_{\tens} \looseid_{\tens \times \transid}(\beta \times \beta \times \beta)$} (11);
\draw[doubleloose] (11) to node[above]{\scriptsize $\looseid_{\tens} (\chi_g \times \looseid_g)$} (21);
\draw[doubleloose] (21) to node[above]{\scriptsize $\chi_g \looseid_{\tens \times \transid}$} (31);
\draw[doubleloose] (31) to node[above]{\scriptsize $\looseid_g \alpha$} (41);
%%%%%%%%
\draw[doubleloose] (00) to node[above]{\scriptsize $\looseid_{\tens} (\looseid_{\tens \times \transid})(\beta \times \beta \times \beta)$} (10);
\draw[doubleloose] (10) to node[above]{\scriptsize $\alpha \looseid_{g \times g \times g}$} (20);
\draw[doubleloose] (20) to node[above]{\scriptsize $\looseid_{\tens} (\looseid_{\transid} \times \chi_g)$} (30);
\draw[doubleloose] (30) to node[above]{\scriptsize $\chi_g \looseid_{\transid \times \tens}$} (40);
%%%%%%%%
\draw[doubletighteq] (04) to (03);
\draw[doubletighteq] (14) to (13);
\draw[doubletighteq] (24) to (23);
\draw[doubletighteq] (44) to (43);
%%%%%%%%%
\draw[doubletighteq] (03) to (02);
\draw[doubletighteq] (13) to (12);
\draw[doubletighteq] (33) to (32);
\draw[doubletighteq] (43) to (42);
%%%%%%%%%
\draw[doubletighteq] (02) to (01);
\draw[doubletighteq] (22) to (21);
\draw[doubletighteq] (32) to (31);
\draw[doubletighteq] (42) to (41);
%%%%%%%%%
\draw[doubletighteq] (01) to (00);
\draw[doubletighteq] (11) to (10);
\draw[doubletighteq] (41) to (40);
%%%%%%%%%
\node at (.5,3) {\scriptsize $=$};
\node at (1.5,3.5) {\scriptsize $=$};
\node at (3,3.5) {\scriptsize $\DDownarrow ({\horl}^{-1} \verc \horr) ({\horr}^{-1} \verc \horl)$};
\node at (2,2.5) {\scriptsize $\DDownarrow \overline{\Pi^{\beta} \tightid_{\looseid_{\tens \times \transid}}}$};
\node at (2.5,1.5) {\scriptsize $=$};
\node at (.5,.5) {\scriptsize $=$};
\node at (2.5,0.5) {\scriptsize $\DDownarrow \omega^g$};
\node at (1,1.5) {\scriptsize $\DDownarrow \overline{\tightid_{\looseid_{\tens}} (\Pi^{\beta} \times  {\horr}^{-1} \verc \horl)}$};
\node at (3.5,2) {\scriptsize $=$};
\end{tikzpicture}
\end{aligned}
\end{equation*}

\end{document} 
 \newpage


%%
\documentclass[12pt]{ociamthesis}
\usepackage{tikz}
\newcommand{\id}{\mathrm{id}}
\begin{document}


\begin{equation}\label{eq:br2cell}
\begin{aligned}
\begin{tikzpicture}[xscale=3, yscale=1.5]
\node (03) at (0,3) {\small $\tens(f \times f)$};
\node (13) at (1,3) {\small $\tens \tau (f \times f)$};
\node (23) at (2,3) {\small $\tens(f \times f) \tau$};
\node (33) at (3,3) {\small $f \tens \tau$};
\node (43) at (4,3) {\small $g \tens \tau$};
%%%%%%
\node (02) at (0,2) {\small $\tens(f \times f)$};
\node (12) at (1,2) {\small $f \tens$};
\node (32) at (3,2) {\small $f \tens \tau$};
\node (42) at (4,2) {\small $g \tens \tau$};
%%%%%% 
\node (01) at (0,1) {\small $\tens(f \times f)$};
\node (11) at (1,1) {\small $f \tens$};
\node (31) at (3,1) {\small $g \tens $};
\node (41) at (4,1) {\small $g \tens \tau$};
%%%%%%%
\node (00) at (0,0) {\small $\tens(f \times f)$};
\node (10) at (1,0) {\small $\tens(g \times g)$};
\node (30) at (3,0) {\small $g \tens$};
\node (40) at (4,0) {\small $g \tens \tau$};
%%%%%%%
\draw[doubleloose] (03) to node[above]{\small $\sigma \looseid_{f \times f}$} (13);
\draw[double] (13) to (23);
\draw[doubleloose] (23) to node[above]{\small $\chi \looseid_{\tau}$} (33);
\draw[doubleloose] (33) to node[above]{\small $\beta \looseid_{\tens \tau}$} (43);
%%%%
\draw[doubleloose] (02) to node[above]{\small $\chi$} (12);
\draw[doubleloose] (12) to node[above]{\small $\looseid_f \sigma$} (32);
\draw[doubleloose] (32) to node[above]{\small $\beta \looseid_{\tens \tau}$} (42);
%%%%%%
\draw[doubleloose] (01) to node[above]{\small $\chi$} (11);
\draw[doubleloose] (11) to node[above]{\small $\beta \looseid_{\tens}$} (31);
\draw[doubleloose] (31) to node[above]{\small $ \looseid_g \sigma$} (41);
%%%%%%
\draw[doubleloose] (00) to node[above]{\small $\looseid_{\tens} (\beta \times \beta)$} (10);
\draw[doubleloose] (10) to node[above]{\small $\chi $} (30);
\draw[doubleloose] (30) to node[above]{\small $\looseid_g \sigma $} (40);
%%%%%%
\draw[doubletighteq] (03) to (02);
\draw[doubletighteq] (33) to (32);
\draw[doubletighteq] (43) to (42);
%%%%%%
\draw[doubletighteq] (02) to (01);
\draw[doubletighteq] (12) to (11);
\draw[doubletighteq] (42) to (41);
%%%%%%
\draw[doubletighteq] (01) to (00);
\draw[doubletighteq] (31) to (30);
\draw[doubletighteq] (41) to (40);
%%%%%%%%
\node at (1.5,2.5) {\small $\DDownarrow u$};
\node at (3.5,2.5) {\small $=$};
\node at (.5,1.5) {\small $=$};
\node at (2.5,1.5) {\small $\iso$};
\node at (1.5,.5) {\small $\DDownarrow \Pi^{\beta}$};
\node at (3.5,.5) {\small $=$};
\end{tikzpicture}
\end{aligned}
\end{equation}
\[=\]
\begin{equation*}
\begin{aligned}
\begin{tikzpicture}[xscale=3, yscale=1.5]
\node (03) at (0,3) {\small $\tens(f \times f)$};
\node (13) at (1,3) {\small $\tens \tau (f \times f)$};
\node (23) at (2,3) {\small $\tens(f \times f) \tau$};
\node (33) at (3,3) {\small $f \tens \tau$};
\node (43) at (4,3) {\small $g \tens \tau$};
%%%%%%
\node (02) at (0,2) {\small $\tens(f \times f)$};
\node (12) at (1,2) {\small $\tens \tau (f \times f)$};
\node (22) at (2,2) {\small $\tens (f \times f) \tau $};
\node (32) at (3,2) {\small $\tens (g \times g) \tau$};
\node (42) at (4,2) {\small $g \tens \tau$};
%%%%%% 
\node (01) at (0,1) {\small $\tens(f \times f)$};
\node (11) at (1,1) {\small $\tens (g \times g)$};
\node (21) at (2,1) {\small $\tens \tau (g \times g)$};
\node (31) at (3,1) {\small $\tens (g \times g) \tau $};
\node (41) at (4,1) {\small $g \tens \tau$};
%%%%%%%
\node (00) at (0,0) {\small $\tens(f \times f)$};
\node (10) at (1,0) {\small $\tens(g \times g)$};
\node (30) at (3,0) {\small $g \tens$};
\node (40) at (4,0) {\small $g \tens \tau$};
%%%%%%%
\draw[doubleloose] (03) to node[above]{\small $\sigma \looseid_{f \times f}$} (13);
\draw[double] (13) to (23);
\draw[doubleloose] (23) to node[above]{\small $\chi \looseid_{\tau}$} (33);
\draw[doubleloose] (33) to node[above]{\small $\beta \looseid_{\tens \tau}$} (43);
%%%%
\draw[doubleloose] (02) to node[above]{\small $\sigma \looseid_{f \times f}$} (12);
\draw[double] (12) to  (22);
\draw[doubleloose] (22) to node[above]{\small $\looseid_{\tens} (\beta \times \beta) \looseid_{\tau}$} (32);
\draw[doubleloose] (32) to node[above]{\small $\chi \looseid_{\tau}$} (42);
%%%%%%
\draw[doubleloose] (01) to node[above]{\small $\looseid_{\tens} (\beta \times \beta)$} (11);
\draw[doubleloose] (11) to node[above]{\small $\sigma \looseid_{g \times g}$} (21);
\draw[double] (21) to (31);
\draw[doubleloose] (31) to node[above]{\small $ \chi \looseid_{\tau}$} (41);
%%%%%%
\draw[doubleloose] (00) to node[above]{\small $\looseid_{\tens} (\beta \times \beta)$} (10);
\draw[doubleloose] (10) to node[above]{\small $\chi $} (30);
\draw[doubleloose] (30) to node[above]{\small $\looseid_g \sigma $} (40);
%%%%%%
\draw[doubletighteq] (03) to (02);
\draw[doubletighteq] (13) to (12);
\draw[doubletighteq] (43) to (42);
%%%%%%
\draw[doubletighteq] (02) to (01);
\draw[doubletighteq] (32) to (31);
\draw[doubletighteq] (42) to (41);
%%%%%%
\draw[doubletighteq] (01) to (00);
\draw[doubletighteq] (11) to (10);
\draw[doubletighteq] (41) to (40);
%%%%%%%%
\node at (.5,2.5) {\small $=$};
\node at (2.5,2.5) {\small $\DDownarrow \overline{\Pi^{\beta} \looseid_{\tau}}$};
\node at (1.5,1.5) {\small $\iso$};
\node at (3.5,1.5) {\small $=$};
\node at (.5,.5) {\small $=$};
\node at (2.5,.5) {\small $\DDownarrow u$};
\end{tikzpicture}
\end{aligned}
\end{equation*}

\end{document} 
 \newpage


%\subsubsection*{Strong Monoidal 2-cells}

%%
\documentclass[12pt]{ociamthesis}
\usepackage{tikz}
\newcommand{\id}{\mathrm{id}}
\begin{document}

\begin{equation}\label{eq:StrongMon2cell1}
    \begin{pic}[yscale=0.8, xscale=.5]
\draw[fill=blue, opacity = 0.5, draw=black] (0,4) -- (0,0) -- (3,0) -- (3,2) -- (2,2) -- (2,1) -- (1,1) -- (1,4) -- (0,4);
\draw[fill=orange, opacity = 0.5, draw=black] (1,4) -- (1,1) -- (2,1) -- (2,4) -- (1,4);
\draw[fill=green, opacity = 0.5, draw=black] (2,4) -- (2,2) -- (3,2) -- (3,3) -- (4,3) -- (4,0) -- (5,0) -- (5,4) -- (0,4);
\draw[fill=yellow, opacity = 0.5, draw=black] (3,0) -- (4,0) -- (4,3) -- (3,3) -- (3,0);
\node[morphism, minimum width=10mm] at (2.5,2) {$\Pi_{oplax}$};
\node[morphism, minimum width=10mm] at (1.5,1) {$\eta_{\chi}$};
\node[morphism, minimum width=10mm] at (3.5,3) {$\epsilon_{\chi}$};
\node at (2.6,-.2) {$\looseid(\beta \times \beta)  $};
\node at (4.4,-.2) {$\bar{\chi}$};
\node at (0.8,4.2) {$\bar{\chi} $};
\node at (2.2,4.2) {$ \beta \looseid$};
    \end{pic}
=
    \begin{pic}[yscale=0.8, xscale=.5]
\draw[fill=blue, opacity = 0.5, draw=black] (0,4) -- (0,0) -- (2,0) -- (2,4) -- (0,4);
\draw[fill=orange, opacity = 0.5, draw=black] (2,4) -- (2,2) -- (3,2) -- (3,4) -- (2,4);
\draw[fill=green, opacity = 0.5, draw=black] (3,4) -- (3,0) -- (5,0) -- (5,4) -- (3,4);
\draw[fill=yellow, opacity = 0.5, draw=black] (2,0) -- (3,0) -- (3,2) -- (2,2) -- (2,0);
\node[morphism, minimum width=10mm] at (2.5,2) {$\overline{\Pi_{lax}}$};
\node at (1.6,-.2) {$\looseid(\beta \times \beta)$};
\node at (3.4,-.2) {$\bar{\chi}$};
\node at (1.8,4.2) {$\beta \looseid$};
\node at (3.2,4.2) {$\bar{\chi}$};
    \end{pic}
    \end{equation}
\end{document} 

%%
\documentclass[12pt]{ociamthesis}
\usepackage{tikz}
\newcommand{\id}{\mathrm{id}}
\begin{document}

\begin{equation}\label{eq:StrongMon2cell2} 
        \begin{pic}[yscale=0.8, xscale=.5]
\draw[fill=yellow, opacity = 0.5, draw=black] (0,0) -- (1,0) -- (1,3) -- (2,3) -- (2,2) -- (3,2) -- (3,4) -- (0,4) -- (0,0);
\draw[fill=blue, opacity = 0.5, draw=black] (1,0) -- (2,0) -- (2,3) -- (1,3) -- (1,0);
\draw[fill=orange, opacity = 0.5, draw=black] (2,0) -- (5,0) -- (5,4) -- (4,4) -- (4,1) -- (3,1) -- (3,2) -- (2,2) -- (2,0);
\draw[fill=green, opacity = 0.5, draw=black] (3,4) -- (4,4) -- (4,1) -- (3,1) -- (3,4);
\node[morphism, minimum width=10mm] at (2.5,2) {$\overline{\Pi_{oplax}}$};
\node[morphism, minimum width=10mm] at (3.5,1) {$\eta_{\chi}$};
\node[morphism, minimum width=10mm] at (1.5,3) {$\epsilon_{\chi}$};
\node at (0.8,-.2) {$\chi$};
\node at (2.2,-.2) {$\beta \looseid $};
\node at (2.6,4.2) {$\chi$};
\node at (4.4,4.2) {$\looseid(\beta \times \beta)$};
    \end{pic}
=
    \begin{pic}[yscale=0.8, xscale=.5]
\draw[fill=yellow, opacity = 0.5, draw=black] (0,4) -- (0,0) -- (2,0) -- (2,4) -- (0,4);
\draw[fill=green, opacity = 0.5, draw=black] (2,4) -- (2,2) -- (3,2) -- (3,4) -- (2,4);
\draw[fill=orange, opacity = 0.5, draw=black] (3,4) -- (3,0) -- (5,0) -- (5,4) -- (3,4);
\draw[fill=blue, opacity = 0.5, draw=black] (2,0) -- (3,0) -- (3,2) -- (2,2) -- (2,0);
\node[morphism, minimum width=10mm] at (2.5,2) {\bf $\Pi_{lax}$};
\node at (1.8,-.2) {$\chi$};
\node at (3.2,-.2) {$\beta \looseid $};
\node at (1.6,4.2) {$\looseid(\beta \times \beta)$};
\node at (3.4,4.2) {$\chi$};
    \end{pic}
    \end{equation}
\end{document} 


\newpage
\subsubsection*{Monoidal Icon}

%
\documentclass[12pt]{ociamthesis}
\usepackage{tikz}
\newcommand{\id}{\mathrm{id}}
\begin{document}

\begin{equation}\label{eq:monicon1}
\begin{aligned}
\begin{tikzpicture}[xscale=4,yscale=2]
\node (02) at (0,2){$\tens(I \times f )i_2 $};
\node (12) at (1,2){$\tens(fI_A \times f)i_2 $};
\node (22) at (2,2){$f\tens(I_A \times \transid)i_2 $};
\node (32) at (3,2){$f $};
\node (01) at (0,1){$\tens(I_B \times \transid )i_2 f$};
\node (31) at (3,1){$f $};
\node (00) at (0,0){$\tens(I_B \times \transid )i_2 g$};
\node (30) at (3,0){$g$};
\draw[doubleloose] (02) to node[above]{$\looseid_{\tens} (\iota_f \times \looseid_f) \looseid_{i_2}$} (12);
\draw[doubleloose] (12) to node[above]{$\chi \looseid_{(I_A \times \transid)i_2}$} (22);
\draw[doubleloose] (22) to node[above]{$\looseid_f l$} (32);
\draw[doubleloose] (01) to node[above]{$l \looseid_f $} (31);
\draw[doubleloose] (00) to node[above]{$l \looseid_g $} (30);
\draw[=] (02) to node[left]{} (01);
\draw[=] (32) to node[left]{} (31);
\draw[doubletight] (01) to node[left]{$\tightid_{\tens} (\tightid_I\times \beta)\tightid_{i_2}$} (00);
\draw[doubletight] (31) to node[left]{$\beta$} (30);
\node at (1.5,1.5){$\DDownarrow \gamma^f$};
\node at (1.5,0.5){$\DDownarrow \looseid_{l}\tightid_{\beta}$};
\end{tikzpicture}
\end{aligned}
\end{equation}
\[=\]
\begin{equation*}
\begin{aligned}
\begin{tikzpicture}[xscale=4,yscale=2]
\node (02) at (0,2){$\tens(I_B \times f )i_2 $};
\node (12) at (1,2){$\tens(fI_A \times f)i_2 $};
\node (22) at (2,2){$f\tens(I_A \times \transid)i_2 $};
\node (32) at (3,2){$f $};
\node (01) at (0,1){$\tens(I_B \times g )i_2 $};
\node (11) at (1,1){$\tens(gI_A \times g)i_2 $};
\node (21) at (2,1){$g\tens(I_A \times \transid)i_2 $};
\node (31) at (3,1){$g $};
\node (00) at (0,0){$\tens(I_B \times \transid )i_2 g$};
\node (30) at (3,0){$g $};
\draw[doubleloose] (02) to node[above]{$\looseid_{\tens} (\iota_f \times \looseid_f) \looseid_{i_2}$} (12);
\draw[doubleloose] (12) to node[above]{$\chi \looseid_{(I_A \times \transid)i_2}$} (22);
\draw[doubleloose] (22) to node[above]{$\looseid_f l$} (32);
\draw[doubleloose] (01) to node[above]{$\looseid_{\tens} (\iota_g \times \looseid_g \looseid_{i_2}$} (11);
\draw[doubleloose] (11) to node[above]{$\chi \looseid_{(I_A \times \transid)i_2}$} (21);
\draw[doubleloose] (21) to node[above]{$\looseid_g l$} (31);
\draw[doubleloose] (00) to node[above]{$l \looseid_g $} (30);
\draw[doubletight] (02) to node[left]{$\tightid_{\tens(I\times \transid)}\beta$} (01);
\draw[doubletight] (12) to node[right]{$\tightid (\beta \times \beta) \tightid$} (11);
\draw[doubletight] (22) to node[left]{$\beta \tightid$} (21);
\draw[doubletight] (32) to node[left]{$\beta$} (31);
\draw[doubletight] (01) to node[left]{$\tightid_{\tens} (\tightid_I\times \beta)\tightid_{i_2}$} (00);
\draw[doubletight] (31) to node[left]{$\beta$} (30);
\node at (0.5,1.5){$\DDownarrow \tightid (N^{\beta} \times \tightid_{\looseid}) \tightid$};
\node at (1.5,1.5){$\DDownarrow \Sigma^{\beta} \tightid$};
\node at (2.5,1.5){$\DDownarrow \looseid_{\beta} \tightid_{l}$};
\node at (1.5,0.5){$\DDownarrow \gamma^g$};
\end{tikzpicture}
\end{aligned}
\end{equation*}


\end{document} 
\newpage

%
\documentclass[12pt]{ociamthesis}
\usepackage{tikz}
\newcommand{\id}{\mathrm{id}}
\begin{document}

\begin{equation*}\hspace{-2cm}
\begin{aligned}
\begin{tikzpicture}[xscale=4,yscale=2]
\node (02) at (0,2){$\tens(f \times I_B ) $};
\node (12) at (1,2){$\tens(f \times fI_A) $};
\node (22) at (2,2){$f\tens(\transid \times I_A) $};
\node (32) at (3,2){$f $};
\node (01) at (0,1){$\tens(\transid \times I_B) f$};
\node (31) at (3,1){$f $};
\node (00) at (0,0){$\tens(\transid \times I_B ) g$};
\node (30) at (3,0){$g $};
\draw[doubleloose] (02) to node[above]{$\looseid_{\tens} (\looseid_f \times \iota_f) $} (12);
\draw[doubleloose] (12) to node[above]{$\chi \looseid_{(\transid \times I_A)}$} (22);
\draw[doubleloose] (22) to node[above]{$\looseid_f r$} (32);
\draw[doubleloose] (01) to node[above]{$r \looseid_f $} (31);
\draw[doubleloose] (00) to node[above]{$r \looseid_g $} (30);
\draw[=] (02) to node[left]{} (01);
\draw[=] (32) to node[left]{} (31);
\draw[doubletight] (01) to node[left]{$\tightid_{\tens (\transid \times I)} \beta$} (00);
\draw[doubletight] (31) to node[left]{$\beta$} (30);
\node at (1.5,1.5){$\DDownarrow \delta^f$};
\node at (1.5,0.5){$\DDownarrow \looseid_{r}\tightid_{\beta}$};
\end{tikzpicture}
\end{aligned}\hspace{-2cm}
\end{equation*}
\begin{equation}\label{eq:monicon2}
  =
\end{equation}
\begin{equation*}\hspace{-2cm}
\begin{aligned}
\begin{tikzpicture}[xscale=4,yscale=2]
\node (02) at (0,2){$\tens(f \times I_B ) $};
\node (12) at (1,2){$\tens(f \times fI_A) $};
\node (22) at (2,2){$f\tens(\transid \times I_A) $};
\node (32) at (3,2){$f$};
\node (01) at (0,1){$\tens(g \times I_B) $};
\node (11) at (1,1){$\tens(g\times gI_A ) $};
\node (21) at (2,1){$g\tens(\transid \times I_A)  $};
\node (31) at (3,1){$g $};
\node (00) at (0,0){$\tens(\transid \times I_B ) g$};
\node (30) at (3,0){$g $};
\draw[doubleloose] (02) to node[above]{$\looseid_{\tens} (\looseid \times\iota_f) $} (12);
\draw[doubleloose] (12) to node[above]{$\chi \looseid_{(\transid \times I_A)}$} (22);
\draw[doubleloose] (22) to node[above]{$\looseid_f r$} (32);
\draw[doubleloose] (01) to node[above]{$\looseid_{\tens} (\looseid_g \times \iota_g) $} (11);
\draw[doubleloose] (11) to node[above]{$\chi \looseid_{(\transid \times I_A)}$} (21);
\draw[doubleloose] (21) to node[above]{$\looseid_g r$} (31);
\draw[doubleloose] (00) to node[above]{$r \looseid_g $} (30);
\draw[doubletight] (02) to node[left]{$\tightid_{\tens(\transid\times I)}\beta$} (01);
\draw[doubletight] (12) to node[right]{$\tightid (\beta \times \beta) \tightid$} (11);
\draw[doubletight] (22) to node[left]{$\beta \tightid$} (21);
\draw[doubletight] (32) to node[left]{$\beta$} (31);
\draw[doubletight] (01) to node[left]{$\tightid_{\tens} (\beta\times \tightid_I)$} (00);
\draw[doubletight] (31) to node[left]{$\beta$} (30);
\node at (0.5,1.5){$\DDownarrow \tightid (\tightid \times N^{\beta}) \tightid$};
\node at (1.5,1.5){$\DDownarrow \Sigma^{\beta} \tightid$};
\node at (2.5,1.5){$\DDownarrow \looseid_{\beta} \tightid_{r}$};
\node at (1.5,0.5){$\DDownarrow \delta^g$};
\end{tikzpicture}
\end{aligned}\hspace{-2cm}
\end{equation*}

\end{document} 
\newpage

%
\documentclass[12pt]{ociamthesis}
\usepackage{tikz}
\newcommand{\id}{\mathrm{id}}
\begin{document}
{\small
\begin{equation*}
\begin{aligned}
\begin{tikzpicture}[xscale=4,yscale=2]
\node (02) at (0,2){\scriptsize$\tens(\tens \times \transid)(f \times f\times f)$};
\node (12) at (1,2){$\scriptsize\tens(f\tens \times f) $};
\node (22) at (2,2){\scriptsize$f\tens(\tens \times \transid)$};
\node (32) at (3,2){\scriptsize$f\tens(\transid \times \tens)$};
\node (01) at (0,1){\scriptsize$\tens(\tens \times \transid)(f \times f\times f)$};
\node (11) at (1,1){\scriptsize$\tens(\transid \times \tens)(f \times f\times f)$};
\node (21) at (2,1){\scriptsize$\tens(f \times f\tens) $};
\node (31) at (3,1){\scriptsize$f \tens ( \transid \times \tens)$};
\node (00) at (0,0){\scriptsize$\tens(\tens \times \transid)(g \times g \times g \times g$};
\node (10) at (1,0){\scriptsize$\tens(\transid \times \tens)(g \times g\times g)$};
\node (20) at (2,0){\scriptsize$\tens(g \times g\tens) $};
\node (30) at (3,0){\scriptsize$g \tens ( \transid \times \tens)$};
\draw[doubleloose] (02) to node[above]{\scriptsize$\looseid_{\tens} (\chi \times \looseid_f)$} (12);
\draw[doubleloose] (12) to node[above]{\scriptsize$\chi \looseid_{(\tens \times \transid)}$} (22);
\draw[doubleloose] (22) to node[above]{\scriptsize$\looseid_{f} \alpha$} (32);
\draw[doubleloose] (01) to node[above]{\scriptsize$\alpha \looseid_{f\times f \times f}$} (11);
\draw[doubleloose] (11) to node[above]{\scriptsize$\looseid_{\tens} (\transid_f \times \chi)$} (21);
\draw[doubleloose] (21) to node[above]{\scriptsize$\chi \looseid_{\transid \times \tens}$} (31);
\draw[doubleloose] (00) to node[above]{\scriptsize$\alpha \looseid_{g\times g \times g} $} (10);
\draw[doubleloose] (10) to node[above]{\scriptsize$\looseid_{\tens} (\transid_g \times \chi)$} (20);
\draw[doubleloose] (20) to node[above]{\scriptsize$\chi \looseid_{\transid \times \tens}$} (30);
\draw[=] (02) to node[left]{} (01);
\draw[=] (32) to node[left]{} (31);
\draw[doubletight] (01) to node[left]{\scriptsize$\tightid (\beta\times \beta \times \beta)$} (00);
\draw[doubletight] (11) to node {\scriptsize$\tightid (\beta\times \beta \times \beta)$} (10);
\draw[doubletight] (21) to node {\scriptsize$\tightid (\beta\times \beta \tightid)$} (20);
\draw[doubletight] (31) to node[left]{\scriptsize$\beta \tightid$} (30);
\node at (1.5,1.5){\scriptsize$\DDownarrow \omega^f$};
\node at (0.5,0.5){\scriptsize$\DDownarrow \tightid_{\alpha} \looseid_{\beta \times \beta \times \beta}$};
\node at (1.5,0.5){\scriptsize$\DDownarrow \tightid (\tightid \times \Sigma^{\beta})$};
\node at (2.5,0.5){\scriptsize$\DDownarrow \Sigma^{\beta} \tightid$};
\end{tikzpicture}
\end{aligned}
\end{equation*}}
\begin{equation}\label{eq:monicon3}
=
\end{equation}
{\small
\begin{equation*}
\begin{aligned}
\begin{tikzpicture}[xscale=4,yscale=2]
\node (02) at (0,2){\scriptsize$\tens(\tens \times \transid)(f \times f\times f)$};
\node (12) at (1,2){\scriptsize$\tens(f\tens \times f) $};
\node (22) at (2,2){\scriptsize$f\tens(\tens \times \transid)$};
\node (32) at (3,2){\scriptsize$f\tens(\transid \times \tens)$};
\node (01) at (0,1){\scriptsize$\tens(\tens \times \transid)(g \times g\times g)$};
\node (11) at (1,1){\scriptsize$\tens(g\tens \times g)$};
\node (21) at (2,1){\scriptsize$g\tens(\tens \times \transid) $};
\node (31) at (3,1){\scriptsize$g \tens ( \transid \times \tens)$};
\node (00) at (0,0){\scriptsize$\tens(\tens \times \transid)(g \times g \times g \times g)$};
\node (10) at (1,0){\scriptsize$\tens(\transid \times \tens)(g \times g\times g)$};
\node (20) at (2,0){\scriptsize$\tens(g \times g\tens) $};
\node (30) at (3,0){\scriptsize$g \tens ( \transid \times \tens)$};
\draw[doubleloose] (02) to node[above]{\scriptsize$\looseid_{\tens} (\chi \times \looseid_f)$} (12);
\draw[doubleloose] (12) to node[above]{\scriptsize$\chi \looseid_{(\tens \times \transid)}$} (22);
\draw[doubleloose] (22) to node[above]{\scriptsize$\looseid_{f} \alpha$} (32);
\draw[doubleloose] (01) to node[above]{\scriptsize$\looseid_{\tens} (\chi \times \looseid_g)$} (11);
\draw[doubleloose] (11) to node[above]{\scriptsize$\chi \looseid_{(\tens \times \transid)}$} (21);
\draw[doubleloose] (21) to node[above]{\scriptsize$\looseid_{g} \alpha$} (31);
\draw[doubleloose] (00) to node[above]{\scriptsize$\alpha \looseid_{g\times g \times g} $} (10);
\draw[doubleloose] (10) to node[above]{\scriptsize$\looseid_{\tens} (\transid_g \times \chi)$} (20);
\draw[doubleloose] (20) to node[above]{\scriptsize$\chi \looseid_{\transid \times \tens}$} (30);
\draw[=] (01) to node[left]{} (00);
\draw[doubletight] (12) to node {\scriptsize$\tightid (\beta \tightid \times \beta)$} (11);
\draw[doubletight] (22) to node[left] {\scriptsize$\beta \tightid $} (21);
\draw[=] (31) to node[left]{} (30);
\draw[doubletight] (02) to node[left]{\scriptsize$\tightid (\beta\times \beta \times \beta)$} (01);
\draw[doubletight] (32) to node[left]{\scriptsize$\beta \tightid$} (31);
\node at (1.5,0.5){\scriptsize$\DDownarrow \omega^g$};
\node at (0.5,1.5){\scriptsize$\DDownarrow \tightid (\Sigma^{\beta} \times \tightid)$};
\node at (1.5,1.5){\scriptsize$\DDownarrow \Sigma^{\beta} \tightid$};
\node at (2.5,1.5){\scriptsize$\DDownarrow \looseid_{\beta} \tightid_{\alpha}$};
\end{tikzpicture}
\end{aligned}
\end{equation*}}

\end{document} 
\newpage


\end{document}
