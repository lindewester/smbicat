%
\documentclass{amsart}
\usepackage{amssymb,amsmath,stmaryrd,txfonts,mathrsfs,amsthm}
\usepackage[all,2cell]{xy}
\usepackage[neveradjust]{paralist}
\usepackage{hyperref}
\usepackage{mathtools}

\makeatletter
\let\ea\expandafter

%% Defining commands that are always in math mode.
\def\mdef#1#2{\ea\ea\ea\gdef\ea\ea\noexpand#1\ea{\ea\ensuremath\ea{#2}}}
\def\alwaysmath#1{\ea\ea\ea\global\ea\ea\ea\let\ea\ea\csname your@#1\endcsname\csname #1\endcsname
  \ea\def\csname #1\endcsname{\ensuremath{\csname your@#1\endcsname}}}

% Script letters
\newcommand{\sA}{\ensuremath{\mathscr{A}}}
\newcommand{\sB}{\ensuremath{\mathscr{B}}}
\newcommand{\sC}{\ensuremath{\mathscr{C}}}
\newcommand{\sD}{\ensuremath{\mathscr{D}}}
\newcommand{\sE}{\ensuremath{\mathscr{E}}}
\newcommand{\sF}{\ensuremath{\mathscr{F}}}
\newcommand{\sG}{\ensuremath{\mathscr{G}}}
\newcommand{\sH}{\ensuremath{\mathscr{H}}}
\newcommand{\sI}{\ensuremath{\mathscr{I}}}
\newcommand{\sJ}{\ensuremath{\mathscr{J}}}
\newcommand{\sK}{\ensuremath{\mathscr{K}}}
\newcommand{\sL}{\ensuremath{\mathscr{L}}}
\newcommand{\sM}{\ensuremath{\mathscr{M}}}
\newcommand{\sN}{\ensuremath{\mathscr{N}}}
\newcommand{\sO}{\ensuremath{\mathscr{O}}}
\newcommand{\sP}{\ensuremath{\mathscr{P}}}
\newcommand{\sQ}{\ensuremath{\mathscr{Q}}}
\newcommand{\sR}{\ensuremath{\mathscr{R}}}
\newcommand{\sS}{\ensuremath{\mathscr{S}}}
\newcommand{\sT}{\ensuremath{\mathscr{T}}}
\newcommand{\sU}{\ensuremath{\mathscr{U}}}
\newcommand{\sV}{\ensuremath{\mathscr{V}}}
\newcommand{\sW}{\ensuremath{\mathscr{W}}}
\newcommand{\sX}{\ensuremath{\mathscr{X}}}
\newcommand{\sY}{\ensuremath{\mathscr{Y}}}
\newcommand{\sZ}{\ensuremath{\mathscr{Z}}}

% Calligraphic letters
\newcommand{\cA}{\ensuremath{\mathcal{A}}}
\newcommand{\cB}{\ensuremath{\mathcal{B}}}
\newcommand{\cC}{\ensuremath{\mathcal{C}}}
\newcommand{\cD}{\ensuremath{\mathcal{D}}}
\newcommand{\cE}{\ensuremath{\mathcal{E}}}
\newcommand{\cF}{\ensuremath{\mathcal{F}}}
\newcommand{\cG}{\ensuremath{\mathcal{G}}}
\newcommand{\cH}{\ensuremath{\mathcal{H}}}
\newcommand{\cI}{\ensuremath{\mathcal{I}}}
\newcommand{\cJ}{\ensuremath{\mathcal{J}}}
\newcommand{\cK}{\ensuremath{\mathcal{K}}}
\newcommand{\cL}{\ensuremath{\mathcal{L}}}
\newcommand{\cM}{\ensuremath{\mathcal{M}}}
\newcommand{\cN}{\ensuremath{\mathcal{N}}}
\newcommand{\cO}{\ensuremath{\mathcal{O}}}
\newcommand{\cP}{\ensuremath{\mathcal{P}}}
\newcommand{\cQ}{\ensuremath{\mathcal{Q}}}
\newcommand{\cR}{\ensuremath{\mathcal{R}}}
\newcommand{\cS}{\ensuremath{\mathcal{S}}}
\newcommand{\cT}{\ensuremath{\mathcal{T}}}
\newcommand{\cU}{\ensuremath{\mathcal{U}}}
\newcommand{\cV}{\ensuremath{\mathcal{V}}}
\newcommand{\cW}{\ensuremath{\mathcal{W}}}
\newcommand{\cX}{\ensuremath{\mathcal{X}}}
\newcommand{\cY}{\ensuremath{\mathcal{Y}}}
\newcommand{\cZ}{\ensuremath{\mathcal{Z}}}

% blackboard bold letters
\newcommand{\lA}{\ensuremath{\mathbb{A}}}
\newcommand{\lB}{\ensuremath{\mathbb{B}}}
\newcommand{\lC}{\ensuremath{\mathbb{C}}}
\newcommand{\lD}{\ensuremath{\mathbb{D}}}
\newcommand{\lE}{\ensuremath{\mathbb{E}}}
\newcommand{\lF}{\ensuremath{\mathbb{F}}}
\newcommand{\lG}{\ensuremath{\mathbb{G}}}
\newcommand{\lH}{\ensuremath{\mathbb{H}}}
\newcommand{\lI}{\ensuremath{\mathbb{I}}}
\newcommand{\lJ}{\ensuremath{\mathbb{J}}}
\newcommand{\lK}{\ensuremath{\mathbb{K}}}
\newcommand{\lL}{\ensuremath{\mathbb{L}}}
\newcommand{\lM}{\ensuremath{\mathbb{M}}}
\newcommand{\lN}{\ensuremath{\mathbb{N}}}
\newcommand{\lO}{\ensuremath{\mathbb{O}}}
\newcommand{\lP}{\ensuremath{\mathbb{P}}}
\newcommand{\lQ}{\ensuremath{\mathbb{Q}}}
\newcommand{\lR}{\ensuremath{\mathbb{R}}}
\newcommand{\lS}{\ensuremath{\mathbb{S}}}
\newcommand{\lT}{\ensuremath{\mathbb{T}}}
\newcommand{\lU}{\ensuremath{\mathbb{U}}}
\newcommand{\lV}{\ensuremath{\mathbb{V}}}
\newcommand{\lW}{\ensuremath{\mathbb{W}}}
\newcommand{\lX}{\ensuremath{\mathbb{X}}}
\newcommand{\lY}{\ensuremath{\mathbb{Y}}}
\newcommand{\lZ}{\ensuremath{\mathbb{Z}}}

% bold letters
\newcommand{\bA}{\ensuremath{\mathbf{A}}}
\newcommand{\bB}{\ensuremath{\mathbf{B}}}
\newcommand{\bC}{\ensuremath{\mathbf{C}}}
\newcommand{\bD}{\ensuremath{\mathbf{D}}}
\newcommand{\bE}{\ensuremath{\mathbf{E}}}
\newcommand{\bF}{\ensuremath{\mathbf{F}}}
\newcommand{\bG}{\ensuremath{\mathbf{G}}}
\newcommand{\bH}{\ensuremath{\mathbf{H}}}
\newcommand{\bI}{\ensuremath{\mathbf{I}}}
\newcommand{\bJ}{\ensuremath{\mathbf{J}}}
\newcommand{\bK}{\ensuremath{\mathbf{K}}}
\newcommand{\bL}{\ensuremath{\mathbf{L}}}
\newcommand{\bM}{\ensuremath{\mathbf{M}}}
\newcommand{\bN}{\ensuremath{\mathbf{N}}}
\newcommand{\bO}{\ensuremath{\mathbf{O}}}
\newcommand{\bP}{\ensuremath{\mathbf{P}}}
\newcommand{\bQ}{\ensuremath{\mathbf{Q}}}
\newcommand{\bR}{\ensuremath{\mathbf{R}}}
\newcommand{\bS}{\ensuremath{\mathbf{S}}}
\newcommand{\bT}{\ensuremath{\mathbf{T}}}
\newcommand{\bU}{\ensuremath{\mathbf{U}}}
\newcommand{\bV}{\ensuremath{\mathbf{V}}}
\newcommand{\bW}{\ensuremath{\mathbf{W}}}
\newcommand{\bX}{\ensuremath{\mathbf{X}}}
\newcommand{\bY}{\ensuremath{\mathbf{Y}}}
\newcommand{\bZ}{\ensuremath{\mathbf{Z}}}

% fraktur letters
\newcommand{\fa}{\ensuremath{\mathfrak{a}}}
\newcommand{\fb}{\ensuremath{\mathfrak{b}}}
\newcommand{\fc}{\ensuremath{\mathfrak{c}}}
\newcommand{\fd}{\ensuremath{\mathfrak{d}}}
\newcommand{\fe}{\ensuremath{\mathfrak{e}}}
\newcommand{\ff}{\ensuremath{\mathfrak{f}}}
\newcommand{\fg}{\ensuremath{\mathfrak{g}}}
\newcommand{\fh}{\ensuremath{\mathfrak{h}}}
\newcommand{\fj}{\ensuremath{\mathfrak{j}}}
\newcommand{\fk}{\ensuremath{\mathfrak{k}}}
\newcommand{\fl}{\ensuremath{\mathfrak{l}}}
\newcommand{\fm}{\ensuremath{\mathfrak{m}}}
\newcommand{\fn}{\ensuremath{\mathfrak{n}}}
\newcommand{\fo}{\ensuremath{\mathfrak{o}}}
\newcommand{\fp}{\ensuremath{\mathfrak{p}}}
\newcommand{\fq}{\ensuremath{\mathfrak{q}}}
\newcommand{\fr}{\ensuremath{\mathfrak{r}}}
\newcommand{\fs}{\ensuremath{\mathfrak{s}}}
\newcommand{\ft}{\ensuremath{\mathfrak{t}}}
\newcommand{\fu}{\ensuremath{\mathfrak{u}}}
\newcommand{\fv}{\ensuremath{\mathfrak{v}}}
\newcommand{\fw}{\ensuremath{\mathfrak{w}}}
\newcommand{\fx}{\ensuremath{\mathfrak{x}}}
\newcommand{\fy}{\ensuremath{\mathfrak{y}}}
\newcommand{\fz}{\ensuremath{\mathfrak{z}}}

\mdef\fahat{\hat{\fa}}

% Underline letters
\newcommand{\uA}{\ensuremath{\underline{A}}}
\newcommand{\uB}{\ensuremath{\underline{B}}}
\newcommand{\uC}{\ensuremath{\underline{C}}}
\newcommand{\uD}{\ensuremath{\underline{D}}}
\newcommand{\uE}{\ensuremath{\underline{E}}}
\newcommand{\uF}{\ensuremath{\underline{F}}}
\newcommand{\uG}{\ensuremath{\underline{G}}}
\newcommand{\uH}{\ensuremath{\underline{H}}}
\newcommand{\uI}{\ensuremath{\underline{I}}}
\newcommand{\uJ}{\ensuremath{\underline{J}}}
\newcommand{\uK}{\ensuremath{\underline{K}}}
\newcommand{\uL}{\ensuremath{\underline{L}}}
\newcommand{\uM}{\ensuremath{\underline{M}}}
\newcommand{\uN}{\ensuremath{\underline{N}}}
\newcommand{\uO}{\ensuremath{\underline{O}}}
\newcommand{\uP}{\ensuremath{\underline{P}}}
\newcommand{\uQ}{\ensuremath{\underline{Q}}}
\newcommand{\uR}{\ensuremath{\underline{R}}}
\newcommand{\uS}{\ensuremath{\underline{S}}}
\newcommand{\uT}{\ensuremath{\underline{T}}}
\newcommand{\uU}{\ensuremath{\underline{U}}}
\newcommand{\uV}{\ensuremath{\underline{V}}}
\newcommand{\uW}{\ensuremath{\underline{W}}}
\newcommand{\uX}{\ensuremath{\underline{X}}}
\newcommand{\uY}{\ensuremath{\underline{Y}}}
\newcommand{\uZ}{\ensuremath{\underline{Z}}}

% bars
\newcommand{\Abar}{\ensuremath{\overline{A}}}
\newcommand{\Bbar}{\ensuremath{\overline{B}}}
\newcommand{\Cbar}{\ensuremath{\overline{C}}}
\newcommand{\Dbar}{\ensuremath{\overline{D}}}
\newcommand{\Ebar}{\ensuremath{\overline{E}}}
\newcommand{\Fbar}{\ensuremath{\overline{F}}}
\newcommand{\Gbar}{\ensuremath{\overline{G}}}
\newcommand{\Hbar}{\ensuremath{\overline{H}}}
\newcommand{\Ibar}{\ensuremath{\overline{I}}}
\newcommand{\Jbar}{\ensuremath{\overline{J}}}
\newcommand{\Kbar}{\ensuremath{\overline{K}}}
\newcommand{\Lbar}{\ensuremath{\overline{L}}}
\newcommand{\Mbar}{\ensuremath{\overline{M}}}
\newcommand{\Nbar}{\ensuremath{\overline{N}}}
\newcommand{\Obar}{\ensuremath{\overline{O}}}
\newcommand{\Pbar}{\ensuremath{\overline{P}}}
\newcommand{\Qbar}{\ensuremath{\overline{Q}}}
\newcommand{\Rbar}{\ensuremath{\overline{R}}}
\newcommand{\Sbar}{\ensuremath{\overline{S}}}
\newcommand{\Tbar}{\ensuremath{\overline{T}}}
\newcommand{\Ubar}{\ensuremath{\overline{U}}}
\newcommand{\Vbar}{\ensuremath{\overline{V}}}
\newcommand{\Wbar}{\ensuremath{\overline{W}}}
\newcommand{\Xbar}{\ensuremath{\overline{X}}}
\newcommand{\Ybar}{\ensuremath{\overline{Y}}}
\newcommand{\Zbar}{\ensuremath{\overline{Z}}}
\newcommand{\abar}{\ensuremath{\overline{a}}}
\newcommand{\bbar}{\ensuremath{\overline{b}}}
\newcommand{\cbar}{\ensuremath{\overline{c}}}
\newcommand{\dbar}{\ensuremath{\overline{d}}}
\newcommand{\ebar}{\ensuremath{\overline{e}}}
\newcommand{\fbar}{\ensuremath{\overline{f}}}
\newcommand{\gbar}{\ensuremath{\overline{g}}}
%\newcommand{\hbar}{\ensuremath{\overline{h}}} % whoops, \hbar means something else!
\newcommand{\ibar}{\ensuremath{\overline{\imath}}}
\newcommand{\jbar}{\ensuremath{\overline{\jmath}}}
\newcommand{\kbar}{\ensuremath{\overline{k}}}
\newcommand{\lbar}{\ensuremath{\overline{l}}}
\newcommand{\mbar}{\ensuremath{\overline{m}}}
\newcommand{\nbar}{\ensuremath{\overline{n}}}
%\newcommand{\obar}{\ensuremath{\overline{o}}}
\newcommand{\pbar}{\ensuremath{\overline{p}}}
\newcommand{\qbar}{\ensuremath{\overline{q}}}
\newcommand{\rbar}{\ensuremath{\overline{r}}}
\newcommand{\sbar}{\ensuremath{\overline{s}}}
\newcommand{\tbar}{\ensuremath{\overline{t}}}
\newcommand{\ubar}{\ensuremath{\overline{u}}}
\newcommand{\vbar}{\ensuremath{\overline{v}}}
\newcommand{\wbar}{\ensuremath{\overline{w}}}
\newcommand{\xbar}{\ensuremath{\overline{x}}}
\newcommand{\ybar}{\ensuremath{\overline{y}}}
\newcommand{\zbar}{\ensuremath{\overline{z}}}

% tildes
\newcommand{\Atil}{\ensuremath{\widetilde{A}}}
\newcommand{\Btil}{\ensuremath{\widetilde{B}}}
\newcommand{\Ctil}{\ensuremath{\widetilde{C}}}
\newcommand{\Dtil}{\ensuremath{\widetilde{D}}}
\newcommand{\Etil}{\ensuremath{\widetilde{E}}}
\newcommand{\Ftil}{\ensuremath{\widetilde{F}}}
\newcommand{\Gtil}{\ensuremath{\widetilde{G}}}
\newcommand{\Htil}{\ensuremath{\widetilde{H}}}
\newcommand{\Itil}{\ensuremath{\widetilde{I}}}
\newcommand{\Jtil}{\ensuremath{\widetilde{J}}}
\newcommand{\Ktil}{\ensuremath{\widetilde{K}}}
\newcommand{\Ltil}{\ensuremath{\widetilde{L}}}
\newcommand{\Mtil}{\ensuremath{\widetilde{M}}}
\newcommand{\Ntil}{\ensuremath{\widetilde{N}}}
\newcommand{\Otil}{\ensuremath{\widetilde{O}}}
\newcommand{\Ptil}{\ensuremath{\widetilde{P}}}
\newcommand{\Qtil}{\ensuremath{\widetilde{Q}}}
\newcommand{\Rtil}{\ensuremath{\widetilde{R}}}
\newcommand{\Stil}{\ensuremath{\widetilde{S}}}
\newcommand{\Ttil}{\ensuremath{\widetilde{T}}}
\newcommand{\Util}{\ensuremath{\widetilde{U}}}
\newcommand{\Vtil}{\ensuremath{\widetilde{V}}}
\newcommand{\Wtil}{\ensuremath{\widetilde{W}}}
\newcommand{\Xtil}{\ensuremath{\widetilde{X}}}
\newcommand{\Ytil}{\ensuremath{\widetilde{Y}}}
\newcommand{\Ztil}{\ensuremath{\widetilde{Z}}}
\newcommand{\atil}{\ensuremath{\widetilde{a}}}
\newcommand{\btil}{\ensuremath{\widetilde{b}}}
\newcommand{\ctil}{\ensuremath{\widetilde{c}}}
\newcommand{\dtil}{\ensuremath{\widetilde{d}}}
\newcommand{\etil}{\ensuremath{\widetilde{e}}}
\newcommand{\ftil}{\ensuremath{\widetilde{f}}}
\newcommand{\gtil}{\ensuremath{\widetilde{g}}}
\newcommand{\htil}{\ensuremath{\widetilde{h}}}
\newcommand{\itil}{\ensuremath{\widetilde{\imath}}}
\newcommand{\jtil}{\ensuremath{\widetilde{\jmath}}}
\newcommand{\ktil}{\ensuremath{\widetilde{k}}}
\newcommand{\ltil}{\ensuremath{\widetilde{l}}}
\newcommand{\mtil}{\ensuremath{\widetilde{m}}}
\newcommand{\ntil}{\ensuremath{\widetilde{n}}}
\newcommand{\otil}{\ensuremath{\widetilde{o}}}
\newcommand{\ptil}{\ensuremath{\widetilde{p}}}
\newcommand{\qtil}{\ensuremath{\widetilde{q}}}
\newcommand{\rtil}{\ensuremath{\widetilde{r}}}
\newcommand{\stil}{\ensuremath{\widetilde{s}}}
\newcommand{\ttil}{\ensuremath{\widetilde{t}}}
\newcommand{\util}{\ensuremath{\widetilde{u}}}
\newcommand{\vtil}{\ensuremath{\widetilde{v}}}
\newcommand{\wtil}{\ensuremath{\widetilde{w}}}
\newcommand{\xtil}{\ensuremath{\widetilde{x}}}
\newcommand{\ytil}{\ensuremath{\widetilde{y}}}
\newcommand{\ztil}{\ensuremath{\widetilde{z}}}

% Hats
\newcommand{\Ahat}{\ensuremath{\widehat{A}}}
\newcommand{\Bhat}{\ensuremath{\widehat{B}}}
\newcommand{\Chat}{\ensuremath{\widehat{C}}}
\newcommand{\Dhat}{\ensuremath{\widehat{D}}}
\newcommand{\Ehat}{\ensuremath{\widehat{E}}}
\newcommand{\Fhat}{\ensuremath{\widehat{F}}}
\newcommand{\Ghat}{\ensuremath{\widehat{G}}}
\newcommand{\Hhat}{\ensuremath{\widehat{H}}}
\newcommand{\Ihat}{\ensuremath{\widehat{I}}}
\newcommand{\Jhat}{\ensuremath{\widehat{J}}}
\newcommand{\Khat}{\ensuremath{\widehat{K}}}
\newcommand{\Lhat}{\ensuremath{\widehat{L}}}
\newcommand{\Mhat}{\ensuremath{\widehat{M}}}
\newcommand{\Nhat}{\ensuremath{\widehat{N}}}
\newcommand{\Ohat}{\ensuremath{\widehat{O}}}
\newcommand{\Phat}{\ensuremath{\widehat{P}}}
\newcommand{\Qhat}{\ensuremath{\widehat{Q}}}
\newcommand{\Rhat}{\ensuremath{\widehat{R}}}
\newcommand{\Shat}{\ensuremath{\widehat{S}}}
\newcommand{\That}{\ensuremath{\widehat{T}}}
\newcommand{\Uhat}{\ensuremath{\widehat{U}}}
\newcommand{\Vhat}{\ensuremath{\widehat{V}}}
\newcommand{\What}{\ensuremath{\widehat{W}}}
\newcommand{\Xhat}{\ensuremath{\widehat{X}}}
\newcommand{\Yhat}{\ensuremath{\widehat{Y}}}
\newcommand{\Zhat}{\ensuremath{\widehat{Z}}}
\newcommand{\ahat}{\ensuremath{\hat{a}}}
\newcommand{\bhat}{\ensuremath{\hat{b}}}
\newcommand{\chat}{\ensuremath{\hat{c}}}
\newcommand{\dhat}{\ensuremath{\hat{d}}}
\newcommand{\ehat}{\ensuremath{\hat{e}}}
\newcommand{\fhat}{\ensuremath{\hat{f}}}
\newcommand{\ghat}{\ensuremath{\hat{g}}}
\newcommand{\hhat}{\ensuremath{\hat{h}}}
\newcommand{\ihat}{\ensuremath{\hat{\imath}}}
\newcommand{\jhat}{\ensuremath{\hat{\jmath}}}
\newcommand{\khat}{\ensuremath{\hat{k}}}
\newcommand{\lhat}{\ensuremath{\hat{l}}}
\newcommand{\mhat}{\ensuremath{\hat{m}}}
\newcommand{\nhat}{\ensuremath{\hat{n}}}
\newcommand{\ohat}{\ensuremath{\hat{o}}}
\newcommand{\phat}{\ensuremath{\hat{p}}}
\newcommand{\qhat}{\ensuremath{\hat{q}}}
\newcommand{\rhat}{\ensuremath{\hat{r}}}
\newcommand{\shat}{\ensuremath{\hat{s}}}
\newcommand{\that}{\ensuremath{\hat{t}}}
\newcommand{\uhat}{\ensuremath{\hat{u}}}
\newcommand{\vhat}{\ensuremath{\hat{v}}}
\newcommand{\what}{\ensuremath{\hat{w}}}
\newcommand{\xhat}{\ensuremath{\hat{x}}}
\newcommand{\yhat}{\ensuremath{\hat{y}}}
\newcommand{\zhat}{\ensuremath{\hat{z}}}

%% FONTS AND DECORATION FOR GREEK LETTERS

%% the package `mathbbol' gives us blackboard bold greek and numbers,
%% but it does it by redefining \mathbb to use a different font, so that
%% all the other \mathbb letters look different too.  Here we import the
%% font with bb greek and numbers, but assign it a different name,
%% \mathbbb, so as not to replace the usual one.
\DeclareSymbolFont{bbold}{U}{bbold}{m}{n}
\DeclareSymbolFontAlphabet{\mathbbb}{bbold}
\newcommand{\bbDelta}{\ensuremath{\mathbbb{\Delta}}}
\newcommand{\bbone}{\ensuremath{\mathbbb{1}}}
\newcommand{\bbtwo}{\ensuremath{\mathbbb{2}}}
\newcommand{\bbthree}{\ensuremath{\mathbbb{3}}}

% greek with bars
\newcommand{\albar}{\ensuremath{\overline{\alpha}}}
\newcommand{\bebar}{\ensuremath{\overline{\beta}}}
\newcommand{\gmbar}{\ensuremath{\overline{\gamma}}}
\newcommand{\debar}{\ensuremath{\overline{\delta}}}
\newcommand{\phibar}{\ensuremath{\overline{\varphi}}}
\newcommand{\psibar}{\ensuremath{\overline{\psi}}}
\newcommand{\xibar}{\ensuremath{\overline{\xi}}}
\newcommand{\ombar}{\ensuremath{\overline{\omega}}}

% greek with hats
\newcommand{\alhat}{\ensuremath{\hat{\alpha}}}
\newcommand{\behat}{\ensuremath{\hat{\beta}}}
\newcommand{\gmhat}{\ensuremath{\hat{\gamma}}}
\newcommand{\dehat}{\ensuremath{\hat{\delta}}}

% greek with checks
\newcommand{\alchk}{\ensuremath{\check{\alpha}}}
\newcommand{\bechk}{\ensuremath{\check{\beta}}}
\newcommand{\gmchk}{\ensuremath{\check{\gamma}}}
\newcommand{\dechk}{\ensuremath{\check{\delta}}}

% greek with tildes
\newcommand{\altil}{\ensuremath{\widetilde{\alpha}}}
\newcommand{\betil}{\ensuremath{\widetilde{\beta}}}
\newcommand{\gmtil}{\ensuremath{\widetilde{\gamma}}}
\newcommand{\phitil}{\ensuremath{\widetilde{\varphi}}}
\newcommand{\psitil}{\ensuremath{\widetilde{\psi}}}
\newcommand{\xitil}{\ensuremath{\widetilde{\xi}}}
\newcommand{\omtil}{\ensuremath{\widetilde{\omega}}}

% MISCELLANEOUS SYMBOLS
\mdef\del{\partial}
\mdef\delbar{\overline{\partial}}
\let\sm\wedge
\newcommand{\dd}[1]{\ensuremath{\frac{\partial}{\partial {#1}}}}
\newcommand{\inv}{^{-1}}
\newcommand{\dual}{^{\vee}}
\mdef\hf{\textstyle\frac{1}{2}}
\mdef\thrd{\textstyle\frac{1}{3}}
\mdef\qtr{\textstyle\frac{1}{4}}
\let\meet\wedge
\let\join\vee
\let\dn\downarrow
\newcommand{\op}{^{\mathit{op}}}
\newcommand{\co}{^{\mathit{co}}}
\newcommand{\coop}{^{\mathit{coop}}}
\let\adj\dashv
\SelectTips{cm}{}
\newdir{ >}{{}*!/-10pt/@{>}}    % extra spacing for tail arrows in XYpic
\newcommand{\pushoutcorner}[1][dr]{\save*!/#1+1.2pc/#1:(1,-1)@^{|-}\restore}
\newcommand{\pullbackcorner}[1][dr]{\save*!/#1-1.2pc/#1:(-1,1)@^{|-}\restore}
\let\iso\cong
\let\eqv\simeq
\let\cng\equiv
\mdef\Id{\mathrm{Id}}
\mdef\id{\mathrm{id}}
\alwaysmath{ell}
\alwaysmath{infty}
\alwaysmath{odot}
\def\frc#1/#2.{\frac{#1}{#2}}   % \frc x^2+1 / x^2-1 .
\mdef\ten{\mathrel{\otimes}}
\mdef\bigten{\bigotimes}
\mdef\sqten{\mathrel{\boxtimes}}
\def\pow(#1,#2){\mathop{\pitchfork}(#1,#2)} % powers and
\def\cpw{\mathop{\odot}}                    % copowers

%% OPERATORS
\DeclareMathOperator\lan{Lan}
\DeclareMathOperator\ran{Ran}
\DeclareMathOperator\colim{colim}
\DeclareMathOperator\coeq{coeq}
\DeclareMathOperator\eq{eq}
\DeclareMathOperator\Tot{Tot}
\DeclareMathOperator\cosk{cosk}
\DeclareMathOperator\sk{sk}
\DeclareMathOperator\im{im}
\DeclareMathOperator\Spec{Spec}
\DeclareMathOperator\Ho{Ho}
\DeclareMathOperator\Aut{Aut}
\DeclareMathOperator\End{End}
\DeclareMathOperator\Hom{Hom}
\DeclareMathOperator\Map{Map}

%% ARROWS
% \to already exists
\newcommand{\too}[1][]{\ensuremath{\overset{#1}{\longrightarrow}}}
\newcommand{\ot}{\ensuremath{\leftarrow}}
\newcommand{\oot}[1][]{\ensuremath{\overset{#1}{\longleftarrow}}}
\let\toot\rightleftarrows
\let\otto\leftrightarrows
\let\Impl\Rightarrow
\let\imp\Rightarrow
\let\toto\rightrightarrows
\let\into\hookrightarrow
\let\xinto\xhookrightarrow
\mdef\we{\overset{\sim}{\longrightarrow}}
\mdef\leftwe{\overset{\sim}{\longleftarrow}}
\let\mono\rightarrowtail
\let\leftmono\leftarrowtail
\let\cof\rightarrowtail
\let\leftcof\leftarrowtail
\let\epi\twoheadrightarrow
\let\leftepi\twoheadleftarrow
\let\fib\twoheadrightarrow
\let\leftfib\twoheadleftarrow
\let\cohto\rightsquigarrow
\let\maps\colon
\newcommand{\spam}{\,:\!}       % \maps for left arrows

%% EXTENSIBLE ARROWS
\let\xto\xrightarrow
\let\xot\xleftarrow
% See Voss' Mathmode.tex for instructions on how to create new
% extensible arrows.
\def\rightarrowtailfill@{\arrowfill@{\Yright\joinrel\relbar}\relbar\rightarrow}
\newcommand\xrightarrowtail[2][]{\ext@arrow 0055{\rightarrowtailfill@}{#1}{#2}}
\let\xmono\xrightarrowtail
\let\xcof\xrightarrowtail
\def\twoheadrightarrowfill@{\arrowfill@{\relbar\joinrel\relbar}\relbar\twoheadrightarrow}
\newcommand\xtwoheadrightarrow[2][]{\ext@arrow 0055{\twoheadrightarrowfill@}{#1}{#2}}
\let\xepi\xtwoheadrightarrow
\let\xfib\xtwoheadrightarrow
% Let's leave the left-going ones until I need them.

%% EXTENSIBLE SLASHED ARROWS
% Making extensible slashed arrows, by modifying the underlying AMS code.
% Arguments are:
% 1 = arrowhead on the left (\relbar or \Relbar if none)
% 2 = fill character (usually \relbar or \Relbar)
% 3 = slash character (such as \mapstochar or \Mapstochar)
% 4 = arrowhead on the left (\relbar or \Relbar if none)
% 5 = display mode (\displaystyle etc)
\def\slashedarrowfill@#1#2#3#4#5{%
  $\m@th\thickmuskip0mu\medmuskip\thickmuskip\thinmuskip\thickmuskip
   \relax#5#1\mkern-7mu%
   \cleaders\hbox{$#5\mkern-2mu#2\mkern-2mu$}\hfill
   \mathclap{#3}\mathclap{#2}%
   \cleaders\hbox{$#5\mkern-2mu#2\mkern-2mu$}\hfill
   \mkern-7mu#4$%
}
% Here's the idea: \<slashed>arrowfill@ should be a box containing
% some stretchable space that is the "middle of the arrow".  This
% space is created as a "leader" using \cleader<thing>\hfill, which
% fills an \hfill of space with copies of <thing>.  Here instead of
% just one \cleader, we use two, with the slash in between (and an
% extra copy of the filler, to avoid extra space around the slash).
\def\rightslashedarrowfill@{%
  \slashedarrowfill@\relbar\relbar\mapstochar\rightarrow}
\newcommand\xslashedrightarrow[2][]{%
  \ext@arrow 0055{\rightslashedarrowfill@}{#1}{#2}}
\mdef\hto{\xslashedrightarrow{}}
\mdef\htoo{\xslashedrightarrow{\quad}}
\let\xhto\xslashedrightarrow

%% To get a slashed arrow in XYpic, do
% \[\xymatrix{A \ar[r]|-@{|} & B}\]

% ISOMORPHISMS
\def\xiso#1{\mathrel{\mathrlap{\smash{\xto[\smash{\raisebox{1.3mm}{$\scriptstyle\sim$}}]{#1}}}\hphantom{\xto{#1}}}}
\def\toiso{\xto{\smash{\raisebox{-.5mm}{$\scriptstyle\sim$}}}}

% SHADOWS
\def\shvar#1#2{{\ensuremath{%
  \hspace{1mm}\makebox[-1mm]{$#1\langle$}\makebox[0mm]{$#1\langle$}\hspace{1mm}%
  {#2}%
  \makebox[1mm]{$#1\rangle$}\makebox[0mm]{$#1\rangle$}%
}}}
\def\sh{\shvar{}}
\def\scriptsh{\shvar{\scriptstyle}}
\def\bigsh{\shvar{\big}}
\def\Bigsh{\shvar{\Big}}
\def\biggsh{\shvar{\bigg}}
\def\Biggsh{\shvar{\Bigg}}

% THEOREM-TYPE ENVIRONMENTS, hacked to
%% (a) number all with the same numbers, and
%% (b) have the right names for autoref
\def\defthm#1#2{%
  \newtheorem{#1}{#2}[section]%
  \expandafter\def\csname #1autorefname\endcsname{#2}%
  \expandafter\let\csname c@#1\endcsname\c@thm}
\newtheorem{thm}{Theorem}[section]
\newcommand{\thmautorefname}{Theorem}
\defthm{cor}{Corollary}
\defthm{prop}{Proposition}
\defthm{lem}{Lemma}
\defthm{sch}{Scholium}
\defthm{assume}{Assumption}
\defthm{claim}{Claim}
\defthm{conj}{Conjecture}
\defthm{hyp}{Hypothesis}
\defthm{fact}{Fact}
\theoremstyle{definition}
\defthm{defn}{Definition}
\defthm{notn}{Notation}
\theoremstyle{remark}
\defthm{rmk}{Remark}
\defthm{eg}{Example}
\defthm{egs}{Examples}
\defthm{ex}{Exercise}
\defthm{ceg}{Counterexample}

% How to get QED symbols inside equations at the end of the statements
% of theorems.  AMS LaTeX knows how to do this inside equations at the
% end of *proofs* with \qedhere, and at the end of the statement of a
% theorem with \qed (meaning no proof will be given), but it can't
% seem to combine the two.  Use this instead.
\def\thmqedhere{\expandafter\csname\csname @currenvir\endcsname @qed\endcsname}

% Number numbered lists as (i), (ii), ...
\renewcommand{\theenumi}{(\roman{enumi})}
\renewcommand{\labelenumi}{\theenumi}

%% Labeling that keeps track of theorem-type names.  Superseded by
%% autoref from hyperref, as above, but we keep this in case we are
%% using a journal style file that is incompatible with hyperref.
% 
% \ifx\SK@label\undefined\let\SK@label\label\fi
% \let\your@thm\@thm
% \def\@thm#1#2#3{\gdef\currthmtype{#3}\your@thm{#1}{#2}{#3}}
% \def\xlabel#1{{\let\your@currentlabel\@currentlabel\def\@currentlabel
% {\currthmtype~\your@currentlabel}
% \SK@label{#1@}}\label{#1}}
% \def\xref#1{\ref{#1@}}

% Also number formulas with the theorem counter
\let\c@equation\c@thm
\numberwithin{equation}{section}

% Only show numbers for equations that are actually referenced (or
% whose tags are specified manually) - uses mathtools.
\mathtoolsset{showonlyrefs,showmanualtags}

%% Fix enumerate spacing with paralist.  This has two parts:
%%   1. enable mixing of "old spacing" lists with those adjusted by paralist
%%   2. allow us to specify a number based on which to adjust the spacing
%% For the first, use \killspacingtrue when you want the spacing
%% adjusted, then \killspacingfalse to turn adjustment off.  For the
%% second, use \maxenum=14 to set the widest number you want the
%% spacing to be calculated with.
\newlength\oldleftmargini       % save old spacing
\newlength\oldleftmarginii
\newlength\oldleftmarginiii
\newlength\oldleftmarginiv
\newlength\oldleftmarginv
\newlength\oldleftmarginvi
\newcount\maxenum
\maxenum=7
\newif\ifkillspacing
\def\@adjust@enum@labelwidth{%
  \advance\@listdepth by 1\relax
  \ifkillspacing                % do the paralist thing
    \csname c@\@enumctr\endcsname\maxenum
    \settowidth{\@tempdima}{%
      \csname label\@enumctr\endcsname\hspace{\labelsep}}%
    \csname leftmargin\romannumeral\@listdepth\endcsname
      \@tempdima
  \else                         % otherwise, reset it
    \csname fixspacing\romannumeral\@listdepth\endcsname
  \fi
  \advance\@listdepth by -1\relax}
% these commands, one for each enum level (I couldn't get a generic
% one to work), test whether oldleftmargin has been set yet, and if
% not, set it from leftmargin; otherwise, they reset leftmargin to
% it.  Just setting oldleftmargin to leftmargin in the preamble
% doesn't seem to work.
\def\fixspacingi{\ifnum\oldleftmargini=0\setlength\oldleftmargini\leftmargini\else\setlength\leftmargini\oldleftmargini\fi}
\def\fixspacingii{\ifnum\oldleftmarginii=0\setlength\oldleftmarginii\leftmarginii\else\setlength\leftmarginii\oldleftmarginii\fi}
\def\fixspacingiii{\ifnum\oldleftmarginiii=0\setlength\oldleftmarginiii\leftmarginiii\else\setlength\leftmarginiii\oldleftmarginiii\fi}
\def\fixspacingiv{\ifnum\oldleftmarginiv=0\setlength\oldleftmarginiv\leftmarginiv\else\setlength\leftmarginiv\oldleftmarginiv\fi}
\def\fixspacingv{\ifnum\oldleftmarginv=0\setlength\oldleftmarginv\leftmarginv\else\setlength\leftmarginv\oldleftmarginv\fi}
\def\fixspacingvi{\ifnum\oldleftmarginvi=0\setlength\oldleftmarginvi\leftmarginvi\else\setlength\leftmarginvi\oldleftmarginvi\fi}

%% Fix paralist references, so that we can refer to (1) instead of
%% just 1.
\def\pl@label#1#2{%
  \edef\pl@the{\noexpand#1{\@enumctr}}%
  \pl@lab\expandafter{\the\pl@lab\csname yourthe\@enumctr\endcsname}%
  \advance\@tempcnta1
  \pl@loop}
\def\@enumlabel@#1[#2]{%
  \@plmylabeltrue
  \@tempcnta0
  \pl@lab{}%
  \let\pl@the\pl@qmark
  \expandafter\pl@loop\@gobble#2\@@@
  \ifnum\@tempcnta=1\else
    \PackageWarning{paralist}{Incorrect label; no or multiple
      counters.\MessageBreak The label is: \@gobble#2}%
  \fi
  \expandafter\edef\csname label\@enumctr\endcsname{\the\pl@lab}%
  \expandafter\edef\csname the\@enumctr\endcsname{\the\pl@lab}%
  \expandafter\let\csname yourthe\@enumctr\endcsname\pl@the
  #1}


% GREEK LETTERS, ETC.
\alwaysmath{alpha}
\alwaysmath{beta}
\alwaysmath{gamma}
\alwaysmath{Gamma}
\alwaysmath{delta}
\alwaysmath{Delta}
\alwaysmath{epsilon}
\mdef\ep{\varepsilon}
\alwaysmath{zeta}
\alwaysmath{eta}
\alwaysmath{theta}
\alwaysmath{Theta}
\alwaysmath{iota}
\alwaysmath{kappa}
\alwaysmath{lambda}
\alwaysmath{Lambda}
\alwaysmath{mu}
\alwaysmath{nu}
\alwaysmath{xi}
\alwaysmath{pi}
\alwaysmath{rho}
\alwaysmath{sigma}
\alwaysmath{Sigma}
\alwaysmath{tau}
\alwaysmath{upsilon}
\alwaysmath{Upsilon}
\alwaysmath{phi}
\alwaysmath{Pi}
\alwaysmath{Phi}
\mdef\ph{\varphi}
\alwaysmath{chi}
\alwaysmath{psi}
\alwaysmath{Psi}
\alwaysmath{omega}
\alwaysmath{Omega}
\let\al\alpha
\let\be\beta
\let\gm\gamma
\let\Gm\Gamma
\let\de\delta
\let\De\Delta
\let\si\sigma
\let\Si\Sigma
\let\om\omega
\let\ka\kappa
\let\la\lambda
\let\La\Lambda
\let\ze\zeta
\let\th\theta
\let\Th\Theta
\let\vth\vartheta

\makeatother

% Local Variables:
% mode: latex
% TeX-master: ""
% End:


\UseAllTwocells
\title{Constructing symmetric monoidal bicategories functorially}
\author{Michael A.\ Shulman}
\author{Linde Wester}
\thanks{This material is based on research sponsored by The United States Air Force Research Laboratory under agreement number FA9550-15-1-0053.  The U.S.~Government is authorized to reproduce and distribute reprints for Governmental purposes notwithstanding any copyright notation thereon.  The views and conclusions contained herein are those of the authors and should not be interpreted as necessarily representing the official policies or endorsements, either expressed or implied, of the United States Air Force Research Laboratory, the U.S.~Government, or Carnegie Mellon University.}
\mdef\cMod{\mathcal{M}\mathit{od}}
\mdef\cCat{\mathcal{C}\mathit{at}}
\mdef\cTwocat{2\text{-}\mathcal{C}\mathit{at}}
\mdef\cBicat{\mathcal{B}\mathit{icat}}
\mdef\lMod{\mathbb{M}\mathsf{od}}
\mdef\lnCob{n\mathbb{C}\mathsf{ob}}
\mdef\lProf{\mathbb{P}\mathsf{rof}}
\mdef\cDbl{\mathcal{D}\mathit{bl}}
\mdef\cDblf{\cDbl^{\mathbf{f}}}
\mdef\fchk{\check{f}}
\mdef\conj{\Yleft}
\mdef\Conj{\mathcal{C}\mathit{onj}}
\mdef\Icon{\mathcal{I}\mathit{con}}
\mdef\id{\mathsf{id}}
\mdef\D{\mathbb{D}}
\mdef\E{\mathbb{E}}
\newcommand{\Alg}{\mathbb{A}\mathit{lg}}
\newcommand{\tens}{\boxtimes}
\newcommand{\ver}{\cdot}
\newcommand{\hor}{\circ}

\newcounter{mondbl}             % For restarting enums manually

%\includeonly{}

\begin{document}

\begin{abstract}
  We present a method of constructing symmetric monoidal bicategories
  from symmetric monoidal double categories that satisfy a lifting
  condition. Such symmetric monoidal double categories frequently
  occur in nature, so the method is widely applicable, though not
  universally so. Our main example is the bicategory ${\cA}lg$ of algebras, bimodules and bimodule homomorphisms. The same method can be used to construct symmetric monoidal functors and symmetric monoidal transformations, and is functorial.   
\end{abstract}

\maketitle

\section{Introduction}
\label{sec:introduction}

Symmetric monoidal bicategories are important in many contexts.
However, the definition of even a monoidal bicategory
(see~\cite{gps:tricats,nick:tricats}), let alone a symmetric monoidal
one
(see~\cite{kv:2cat-zam,kv:bm2cat,bn:hda-i,ds:monbi-hopfagbd,crans:centers,mccrudden:bal-coalgb,gurski:brmonbicat}) 
or a monoidal functor between such~\cite{nick:tricatsbook,mccrudden:bal-coalgb}
% TODO: update references
is quite imposing, and time-consuming to verify in any example.

In this paper we describe a method for constructing (symmetric) monoidal
bicategories which is hardly more difficult than constructing a pair
of ordinary (symmetric) monoidal categories.
While not universally applicable, this method applies in many cases of interest.
The underlying idea has often been implicitly used in particular cases, such as
bicategories of enriched profunctors, but to our knowledge the first
general statement was claimed in~\cite[Appendix B]{shulman:frbi}.
In the unpublished~\cite{shulman:smbicat}, the first author worked out the details for the construction of monoidal bicategories themselves.
Here we include that work and build on it further to construct monoidal functors, transformations, and so on between monoidal bicategories as well, making the entire construction into a functor.
\footnote{These details have also been worked out
in~\cite[\S5]{gg:ldstr-tricat} from a different perspective, using
``locally-double bicategories''.  Our goal is to be more
comprehensive, considering also the case of symmetric monoidal
structure and the construction of monoidal functors and
transformations.}

The method relies on the fact that in many bicategories, the 1-cells
are not the most fundamental notion of `morphism' between the objects.
For instance, in the bicategory \cMod\ of rings, bimodules, and
bimodule maps, the more fundamental notion of morphism between objects
is a ring homomorphism. The addition of these extra morphisms promotes
a bicategory to a \emph{double category}, or a category internal to
\cCat.  The extra morphisms are usually stricter than the 1-cells in
the bicategory and easier to deal with for coherence questions; in
many cases it is quite easy to show that we have a \emph{symmetric
  monoidal double category}.  The central observation is that in most
cases (when the natural transformations have `loosely strong companions and conjoints' for the tight morphisms) we can then `lift' this
symmetric monoidal structure to the original bicategory.  That is, we
prove the following theorem:

\begin{thm}\label{thm:mondbl-monbi-intro}
  If \lD\ is a monoidal double category, of which the monoidal constraints have loosely strong companions and conjoints, then its underlying bicategory $\cH(\lD)$ is a monoidal bicategory.  If \lD\ is braided
  or symmetric, so is $\cH(\lD)$. Furthermore, the assignment $\cH$
  extends to a trifunctor, so we can use it to construct functors and
  transformations between monoidal bicategories as well.
\end{thm}

We expect similar theorems to be true in higher dimensions.  For
instance, Chris Douglas~\cite{douglas:tfttalk} has suggested that many
apparent tricategories are more naturally bicategories internal to
\cCat\ or categories internal to \cTwocat; and in most such cases
arising in practice, we can again `lift' the coherence to give a
tricategory after discarding the additional structure.

We propose the generic term \textbf{$(n\times k)$-category}
(pronounced ``$n$-by-$k$-category'') for an $n$-category internal to
$k$-categories, a structure which has $(n+1)(k+1)$ different types of
cells or morphisms arranged in an $(n+1)$ by $(k+1)$ grid.  Thus
double categories may be called \textbf{1x1-categories}, while in
place of tricategories we may consider 2x1-categories and
1x2-categories --- or even 1x1x1-categories, i.e.\ triple categories,
as in~\cite{gp:intercategories-i,gp:intercategories-ii}.  Any
$(n\times k)$-category which satisfies a suitable lifting property
should have an underlying $(n+k)$-category, but clearly as $n$ and $k$
grow an increasing amount of structure is discarded in this process.

% , such as duality and
% trace~\cite{ps:traces} or autonomous
% structure~\cite{ds:monbi-hopfagbd,street:funct-calc,dms:antipodes},

There is a good case to be made that often the extra morphisms should
\emph{not} be discarded; indeed this is the perspective of much work
on double categories.  However, even from this perspective, sometimes
it really is the underlying $(n+k)$-category that one cares about; for
instance, the Baez-Dolan cobordism hypothesis asserts a universal
property of the $(n+1)$-category of cobordisms which is not shared by
the $(n\times 1)$-category from which it is naturally constructed
(see~\cite{lurie:tft}).  Thus, regardless of one's philosophical bent,
results such as \autoref{thm:mondbl-monbi-intro} are of interest.

Proceeding to the contents of this paper, in
\S\ref{sec:symm-mono-double} we review the definition of symmetric
monoidal double categories, and in \S\ref{sec:comp-conj} we recall the
notions of `companion' and `conjoint' whose presence supplies the
necessary lifting property, which we call being \emph{fibrant}.  
(Fibrant double categories have also been called ``framed bicategories''~\cite{shulman:frbi}, and are roughly equivalent to ``proarrow equipments''~\cite{wood:proarrows-i}).

Then in \S\ref{sec:1x1-to-bicat} we describe a functor from double
categories with loosely strong companions and conjoints to bicategories.  This functor evidently preserves products, which intuitively means that it should preserve ``monoidal objects''; but since monoidal bicategories are such weak objects, to deduce our main theorem we need to enhance this functor to act on some higher-categorical structure that contains higher data such as pseudonatural transformations and modifications.
The obvious choice is a tricategory~\cite{gps:tricats}, but we can make do with a simpler structure, namely a \emph{locally cubical bicategory}~\cite{gg:ldstr-tricat}, i.e.\ a bicategory enriched over double categories.
In \S\ref{sec:mono-objects} we show that any product-preserving functor between locally cubical bicategories preserves monoidal objects of all sorts, and indeed induces another functor of locally cubical bicategories.
There is a lot to check here, but the hardest part is writing down all the definitions in the appropriate generality!
Then in \S\ref{sec:constr-symm-mono} we specialize this to the functor from double categories with loosely strong companions and conjoints for all transformations to bicategories, yielding functorial constructions of monoidal, braided, and symmetric monoidal bicategories.
Finally, in \S\ref{sec:Alg} we illustrate our method by constructing the symmetric monoidal bicategory ${\cA}lg(\bC)$ of algebras, bimodules and bimodule homomorphisms, as well as that of dagger algebras, dagger bimodules, and dagger bimodule homomorphisms, in a monoidal category \bC, varying functorially in \bC.

We would like to thank Peter May, Tom Fiore, Stephan Stolz, Chris
Douglas, Nick Gurski, and Jamie Vicary for helpful discussions and comments.

% Local Variables:
% TeX-master: "smbicat"
% End:


\section{Symmetric monoidal double categories}
\label{sec:symm-mono-double}

In this section, we recall basic notions of double categories to fix our terminology and notation, and define monoidal double categories and functors between them in an explicit way.
Double categories go back originally to Ehresmann
in~\cite{ehresmann:cat-str}; a brief introduction can be found
in~\cite{ks:r2cats}.  Other references
include~\cite{multi_funct_i,gp:double-limits,gp:double-adjoints,aleiferi2018cartesian}.


\begin{defn}\label{def:dblcat}
  A \textbf{(pseudo) double category} \lD\ consists of a `category of
  objects' $\lD_0$ and a `category of arrows' $\lD_1$, with structure
  functors
  \begin{align*}
    U&\maps \lD_0\to \lD_1\\
    S,T&\maps \lD_1\rightrightarrows \lD_0\\
    \odot&\maps \lD_1\times_{\lD_0}\lD_1\to \lD_1
  \end{align*}
  (where the pullback is over
  $\lD_1\too[T]\lD_0\overset{S}{\longleftarrow} \lD_1$) such that
  \begin{alignat*}{2}
    S(U_A) &= A &\qquad
    S(M\odot N) &= SN\\
    T(U_A) &= A &\qquad
    T(M\odot N) &= TM
  \end{alignat*}
  naturally, and equipped with natural isomorphisms
  \begin{align*}
    \fa &: (M\odot N) \odot P \too[\iso] M \odot (N \odot P)\\
    \fl &: U_B \odot M \too[\iso] M\\
    \fr &: M \odot U_A \too[\iso] M
  \end{align*}
  such that $S(\fa)$, $T(\fa)$, $S(\fl)$, $T(\fl)$, $S(\fr)$, and
  $T(\fr)$ are all identities, and such that the standard coherence
  axioms for a monoidal category or bicategory (such as Mac Lane's
  pentagon; see~\cite{maclane}) are satisfied.
\end{defn}

Just as a bicategory can be thought of as a category weakly
\emph{enriched} over \cCat, a pseudo double category can be thought of
as a category weakly \emph{internal} to \cCat.  Since these are the
kind of double categories of most interest to us, we will usually drop
the adjective ``pseudo.''

We call the objects of $\lD_0$ \textbf{objects} or \textbf{0-cells},
and we call the morphisms of $\lD_0$ \textbf{tight 1-cells}
and write them as $f\maps A\to B$.  We call the objects of $\lD_1$
\textbf{loose 1-cells}; if $M$ is a 1-cell with $S(M)=A$ and
$T(M)=B$, we write $M\maps A\hto B$.  We call a morphism $\alpha\maps
M\to N$ of $\lD_1$ with $S(\alpha)=f$ and $T(\alpha)=g$ a
\textbf{2-cell} and draw it as follows:
\begin{equation}\label{eq:square}
  \xymatrix@-.5pc{
    A \ar[r]|{|}^{M}  \ar[d]_f \ar@{}[dr]|{\Downarrow\alpha}&
    B\ar[d]^g\\
    C \ar[r]|{|}_N & D
  }.
\end{equation}
We will sometimes say ``$k$-morphism'' as a synonym for ``$k$-cell'' (tight or loose).
Note that composition of tight 1-cells is strictly associative and unital, while that of loose 1-cells is only weakly so.
This is the case in the majority of examples, and is part of what makes monoidal double categories so much easier to construct than monoidal bicategories.
(It is, however, possible to define double categories that are weak in both directions~\cite{verity:base-change}.)

The words ``tight'' and ``loose'', borrowed from~\cite{ls:limlax}, are intended to suggest one kind of morphism that is ``stricter'' than another.
Traditionally the two kinds of 1-arrow in a double category are called ``vertical'' and ``horizontal'' with reference to how they are drawn, but this creates confusion because some authors draw the tight morphisms (those with strictly associative composition) vertically and others draw them horizontally.
We usually draw the tight 1-cells vertically and the loose ones horizontally, but the terminology ``tight'' and ``loose'' unambiguously identifies which arrows we are talking about, independently of our conventions about how to draw pictures.%
\footnote{In~\cite{shulman:smbicat} a similar effect was intended by simply distinguishing between ``1-morphisms'' (the tight ones) and ``1-cells'' (the loose ones), but since ``morphism'' and ``cell'' (and ``arrow'') are generally used interchangeably in higher category theory this was not very successful.}

We write the composition of tight 1-morphisms $A\too[f] B\too[g] C$
and the tight composition of 2-morphisms $M\too[\alpha]
N\too[\beta] P$ as $g\circ f$ and $\beta\circ\alpha$, or sometimes
just $gf$ and $\beta\alpha$.  We write the loose composition of
1-cells $A\xhto{M} B \xhto{N} C$ as $A\xhto{N\odot M} C$ and that of
2-morphisms
\[\vcenter{\xymatrix{ \ar[r]|-@{|}^-{} \ar[d] \ar@{}[dr]|{\Downarrow\alpha} &
     \ar[r]|-@{|}^-{} \ar[d] \ar@{}[dr]|{\Downarrow\beta} &\ar[d]\\
  \ar[r]|-@{|}_-{} & \ar[r]|-@{|}_-{} & }}\]
as
\[\vcenter{\xymatrix@C=4pc{ \ar[r]|-@{|}^-{} \ar[d] \ar@{}[dr]|{\Downarrow\;\be\odot\al} &  \ar[d]\\
  \ar[r]|-@{|}_-{} & }}\]


The two different compositions of 2-morphisms obey an interchange law,
by the functoriality of $\odot$:
\[(M_1\odot M_2) \circ (N_1\odot N_2) = (M_1\circ N_1)\odot (M_2\circ N_2).
\]
Every object $A$ has a tight identity $1_A$ and a loose unit
$U_A$, every loose 1-cell $M$ has an identity 2-morphism $1_M$, every
tight 1-morphism $f$ has a loose unit 2-morphism $U_f$, and we
have $1_{U_A} = U_{1_A}$ (by the functoriality of $U$).

% \begin{rmk}\label{rmk:monglob}
%   In general, an $(n\times 1)$-category consists of 1-categories
%   $\lD_i$ for $0\le i\le n$, together with source, target, unit, and
%   composition functors and coherence isomorphisms.  We refer to the
%   objects of $\lD_i$ as \textbf{$i$-cells} and to the morphisms of
%   $\lD_i$ as \textbf{morphisms of $i$-cells} or \textbf{(tight)
%     $(i+1)$-morphisms}.  A formal definition can be found
%   in~\cite{batanin:monglob} under the name \emph{monoidal $n$-globular
%     category}.
% \end{rmk}

% We call $\lD_0$ the \textbf{vertical category} of \lD.  We say that two
% objects are isomorphic if they are isomorphic in $\lD_0$, and that two
% horizontal 1-cells are isomorphic if they are isomorphic in $\lD_1$.
% We will never refer to a horizontal 1-cell as an isomorphism.

A 2-morphism~\eqref{eq:square} where $f$ and $g$ are identities (such
as the constraint isomorphisms $\fa,\fl,\fr$) is called
\textbf{globular}.  Every double category \lD\ has a
\textbf{loose bicategory} $\cH(\lD)$ consisting of the objects,
loose 1-cells, and globular 2-morphisms.  In the literature, this is often called the ``horizontal'' or ``vertical'' bicategory of a double category (depending on conventions). Conversely, many naturally
occurring bicategories are actually the loose bicategory of some
naturally ocurring double category.  Here are just a few examples.

\begin{eg}
  The double category \lnCob\ has as objects closed $n$-manifolds, as
  tight 1-cells diffeomorphisms, as loose 1-cells cobordisms, and as
  2-cells diffeomorphisms between cobordisms.  Again $\cH(\lnCob)$
  is the usual bicategory of cobordisms.
\end{eg}

\begin{eg}
  The double category \lMod\ has as objects rings, as tight 1-cells ring
  homomorphisms, as loose 1-cells bimodules, and as 2-cells equivariant
  bimodule maps.  Its loose bicategory $\cMod = \cH(\lMod)$ is
  the usual bicategory of rings and bimodules. 
\end{eg}


\begin{eg}
  The double category \lProf\ has as objects categories, as
  tight 1-cells functors, as loose 1-cells \emph{profunctors} (a profunctor
  $A\hto B$ is a functor $B\op\times A\to \mathbf{Set}$), and as
  2-cells natural transformations.  Bicategories such as
  $\cH(\lProf)$ are commonly encountered in category theory,
  especially the enriched versions.
\end{eg}

Further examples will be found in Section~\ref{sec:Alg}.

% If $A$ and $B$ are objects of \lD, we write $\lD(A,B)$ for the
% \emph{set} of vertical arrows from $A$ to $B$ and $\calD(A,B)$ for the
% \emph{category} of horizontal 1-cells and globular 2-cells from $A$ to
% $B$.  It is standard in bicategory theory to say that something holds
% \textbf{locally} when it is true of all hom-categories $\calD(A,B)$, and
% we will extend this usage to double categories.

As opposed to bicategories, which naturally form a tricategory, double
categories naturally form a \emph{2-category}, a much simpler object.

\begin{defn}
  Let \lD\ and \lE\ be double categories.  A \textbf{(pseudo double)
    functor} $F \maps \lD\to \lE$ consists of the following.
  \begin{itemize}
  \item Functors $F_0\maps \lD_0 \to \lE_0$ and $F_1\maps \lD_1 \to
    \lE_1$ such that $S\circ F_1 = F_0\circ S$ and $T\circ F_1 =
    F_0\circ T$.
  \item Natural transformations $F_\odot\maps F_1M \odot F_1N \to
    F_1(M\odot N)$ and $F_U\maps U_{F_0 A} \to F_1(U_A)$, whose
    components are globular isomorphisms, and which satisfy the usual
    coherence axioms for a monoidal functor or pseudofunctor
    (see~\cite[\S{}XI.2]{maclane}).
  \end{itemize}
\end{defn}

\begin{defn}\label{thm:dbl-transf}
  A \textbf{(tight) transformation} between two functors $\alpha:
  F\to G:\lD\to\lE$ consists of natural transformations $\alpha_0\maps
  F_0\to G_0$ and $\alpha_1\maps F_1\to G_1$ (both usually written as
  $\alpha$), such that $S(\alpha_{M}) = \alpha_{SM}$ and
  $T(\alpha_{M}) = \alpha_{TM}$, and such that
  \[\vcenter{\xymatrix@-.5pc{
      FA \ar@{=}[d] \ar[r]|{|}^{FM}
      \ar@{}[drr]|{\Downarrow F_\odot} &
      FB \ar[r]|{|}^{FN} &
      FC \ar@{=}[d]\\
      FA \ar[rr]|{F(N\odot M)} \ar[d]_{\alpha_A}
      \ar@{}[drr]|{\Downarrow \alpha_{N\odot M}} &&
      FC \ar[d]^{\alpha_C}\\
      GA \ar[rr]|{|}_{G(N\odot M)} && GC
    }} =
  \vcenter{\xymatrix@-.5pc{
      FA \ar[d]_{\alpha_A} \ar@{}[dr]|{\Downarrow \alpha_M} \ar[r]|{|}^{FM} &
      FB \ar[d]|{\alpha_B} \ar@{}[dr]|{\Downarrow \alpha_N} \ar[r]|{|}^{FN} &
      FC \ar[d]^{\alpha_C}\\
      GA \ar@{=}[d] \ar[r]|{|}_{GM} \ar@{}[drr]|{\Downarrow G_\odot} &
      GB \ar[r]|{|}_{GN} &
      GC \ar@{=}[d]\\
      GA \ar[rr]|{|}_{G(N\odot M)} && GC
    }}\]
  for all 1-cells $M\colon A\hto B$ and $N\colon B\hto C$, and
  \[\vcenter{\xymatrix@-.5pc{
      FA \ar[rr]|{|}^{U_{FA}} \ar@{=}[d]
      \ar@{}[drr]|{\Downarrow F_U} &&
      FA \ar@{=}[d]\\
      FA \ar[rr]|{F(U_A)} \ar[d]_{\alpha_A}
      \ar@{}[drr]|{\Downarrow \alpha_{U_A}} &&
      FA \ar[d]^{\alpha_A}\\
      GA \ar[rr]|{|}_{G(U_A)} && GA
    }} =
  \vcenter{\xymatrix@-.5pc{
      FA \ar[rr]|{|}^{U_{FA}} \ar[d]_{\alpha_A}
      \ar@{}[drr]|{\Downarrow U_{\alpha_A}} &&
      FA \ar[d]^{\alpha_A}\\
      GA \ar[rr]|{U_{GA}} \ar@{=}[d]
      \ar@{}[drr]|{\Downarrow G_U} &&
      GA \ar@{=}[d]\\
      GA \ar[rr]|{|}_{G(U_A)} && GA.
    }}\]
  for all objects $A$.
\end{defn}

We write \cDbl\ for the strict 2-category of double categories, functors, and transformations, and $\mathbf{Dbl}$ for its underlying 1-category. As pseudofunctors compose strictly associatively, this is well-defined. Note that a 2-cell $\al$ in \cDbl\ is an isomorphism just when each
$\al_A$, \emph{and} each $\al_M$, is invertible.

\begin{rmk}
  A double category has three different opposites:
  in the \emph{loose opposite} $\lD\lop$ we reverse the loose 1-cells (and reverse the 2-cells in the loose direction) but not the tight ones, in the \emph{tight opposite} $\lD\ttop$ we reverse the tight ones but not the loose ones, and in the \emph{double opposite} $\lD\tlop$ we reverse both.
  These define 2-functors $(-)\lop : \cDbl\to\cDbl$, $(-)\ttop : \cDbl\co \to \cDbl$, and $(-)\tlop : \cDbl\co \to \cDbl$, where $(-)\co$ denotes reversal of 2-cells but not 1-cells in a 2-category.
\end{rmk}



The 2-category \cDbl\ gives us an easy way to define what we mean by a
\emph{symmetric monoidal double category}.
We say that a strict 2-category $\cK$ has \emph{finite products} if it has a strictly terminal object, i.e.\ an object $1$ such that each category $\cK(D,1)$ is \emph{isomorphic} to the terminal category, and any two objects $D,E$ have a strict product, i.e.\ a span $D \ot D\times E \to E$ such that the induced functors $\cK(X,D\times E) \to \cK(X,D) \times \cK(X,E)$ is an \emph{isomorphism} of categories.
In any such 2-category with finite products there is a notion of a \emph{pseudomonoid} (perhaps braided or symmetric), which generalizes the notion of monoidal category in \cCat.
We omit the general definition, since we will give a more general one in \cref{sec:mono-objects};
for now we just specialize it to \cDbl\ and obtain the following.

\begin{defn}\label{def:symmondoub}
  A \textbf{monoidal double category} is a double category equipped
  with functors $\ten\maps \lD\times\lD\to\lD$ and $I\maps * \to\lD$,
  and invertible transformations
  \begin{align*}
    \mathord{\otimes} \circ (\Id\times \mathord{\otimes})
    &\iso \mathord{\otimes} \circ (\mathord{\otimes} \times \Id)\\
    \mathord{\otimes} \circ (\Id\times I) &\iso \Id\\
    \mathord{\otimes} \circ (I\times \Id) &\iso \Id
  \end{align*}
  satisfying the usual axioms.  If it additionally has a braiding
  isomorphism
  \begin{align*}
    \mathord{\otimes} &\iso \mathord{\otimes} \circ \tau
  \end{align*}
  (where $\tau\maps \lD\times\lD\iso \lD\times\lD$ is the twist)
  satisfying the usual axioms, then it is \textbf{braided} or
  \textbf{symmetric}, according to whether or not the braiding is
  self-inverse.
\end{defn}

Unpacking this definition more explicitly, we see that a monoidal
double category is a double category together with the following
structure.
\begin{enumerate}
\item $\lD_0$ and $\lD_1$ are both monoidal categories.
\item If $I$ is the monoidal unit of $\lD_0$, then $U_I$ is the
  monoidal unit of $\lD_1$.\footnote{Actually, all the above
    definition requires is that $U_I$ is coherently \emph{isomorphic
      to} the monoidal unit of $\lD_1$, but we can always choose them
    to be equal without changing the rest of the structure.
    More precisely, we may choose the unit isomorphism
    $I_U : U_{I_{\lD_0}} \to I_{\lD_1}$ of the functor $I:* \to \lD$
    to be an identity.}
\item The functors $S$ and $T$ are strict monoidal, i.e.\ $S(M\ten N)
  = SM\ten SN$ and $T(M\ten N)=TM\ten TN$ and $S$ and $T$ also
  preserve the associativity and unit constraints.
\item \label{eq:mon1} We have globular isomorphisms derived from $\otimes_{\odot}$ and $\otimes_U$
  \[\fx\maps (M_1\ten N_1)\odot (M_2\ten N_2)\too[\iso] (M_1\odot M_2)\ten (N_1\odot N_2)\]
  and
  \[\fu\maps U_{A\ten B} \too[\iso] (U_A \ten U_B)\]
  such that the following diagrams commute, expressing that $\ten$ is a pseudo double functor:
\begin{equation}\label{eq:mondoub1}
\begin{aligned}
\begin{tikzpicture}[xscale=1.8, yscale=1.5]
\node (tl) at (0,2) {$((M_1 \tens N_1)\odot (M_2 \tens N_2)) \odot (M_3 \tens N_3)$};
\node (tr) at (4,2) {$((M_1 \odot M_2) \tens (N_1 \odot N_2)) \odot (M_3 \tens N_3)$};
\node (ml) at (0,1) {$(M_1 \tens N_1) \odot ((M_2 \tens N_2) \odot (M_3 \tens N_3))$};
\node (mr) at (4,1) {$((M_1 \odot M_2) \odot M_3) \tens ((N_1 \odot N_2) \odot N_3)$};
\node (bl) at (0,0) {$(M_1 \tens N_1) \odot ((M_2 \odot M_3) \tens (N_2 \odot N_3))$};
\node (br) at (4,0) {$(M_1 \odot (M_2 \odot M_3)) \tens (N_1 \odot (N_2 \odot N_3))$};
\draw[->] (tl) to node[above] {$\fx \odot \id$} (tr);
\draw[->] (tl) to node[left]{$\fa$} (ml);
\draw[->] (ml) to node[left]{$\id\odot \fx$} (bl);
\draw[->] (tr) to node[left]{$\fx$} (mr);
\draw[->] (mr) to node[left]{$\fa \odot \fa$} (br);
\draw[->] (bl) to node[above] {$\fx$} (br);
\end{tikzpicture}
    \end{aligned}
\end{equation}

%\begin{equation}\label{eq:mondoub1}
%\begin{aligned}
%  \xymatrix{
%    ((M_1\ten N_1)\odot (M_2\ten N_2)) \odot (M_3\ten N_3) \ar[r]\ar[d]
%    & ((M_1\odot M_2)\ten (N_1\odot N_2)) \odot (M_3\ten N_3) \ar[d]\\
%    (M_1\ten N_1)\odot ((M_2\ten N_2) \odot (M_3\ten N_3)) \ar[d] &
%    ((M_1\odot M_2)\odot M_3) \ten ((N_1\odot N_2)\odot N_3) \ar[d]\\
%    (M_1\ten N_1) \odot ((M_2\odot M_3) \ten (N_2\odot N_3))\ar[r] &
%    (M_1\odot (M_2\odot M_3)) \ten (N_1\odot (N_2\odot N_3))}
%    \end{aligned}
%\end{equation}

\begin{equation}\label{eq:mondoub2}
\begin{aligned}
\begin{tikzpicture}[xscale=2, yscale=1.5]
\node (tl) at (0,2) {$(M \tens N) \odot U_{C \tens D}$};
\node (tr) at (4,2) {$(M \tens N) \odot ( U_C \tens U_D)$};
\node (ml) at (0,1) {$M \tens N$};
\node (mr) at (4,1) {$(M \odot U_C) \tens (N \odot U_D)$};
\draw[->] (tl) to node[above] {$\id \odot \fu$} (tr);
\draw[->] (tl) to node[left]{$\fr$} (ml);
\draw[->] (tr) to node[left]{$\fx$} (mr);
\draw[->] (mr) to node[above] {$\fr \tens \fr$} (ml);
\end{tikzpicture}
    \end{aligned}
\end{equation}

%\begin{equation}\label{eq:mondoub2}
%\begin{aligned}    
%  \xymatrix{(M\ten N) \odot U_{C\ten D} \ar[r]\ar[d] &
%    (M\ten N)\odot (U_C\ten U_D) \ar[d]\\
%    M\ten N\ar@{<-}[r] & (M\odot U_C) \ten (N\odot U_D)}
%        \end{aligned}
%    \end{equation}
    
    \begin{equation}\label{eq:mondoub3}
\begin{aligned}
\begin{tikzpicture}[xscale=2, yscale=1.5]
\node (tl) at (0,2) {$U_{A \tens B} \odot (M \tens N) $};
\node (tr) at (4,2) {$(U_A \tens U_B) \odot (M \tens N) $};
\node (ml) at (0,1) {$M \tens N$};
\node (mr) at (4,1) {$(U_A \odot M ) \tens (U_B \odot N)$};
\draw[->] (tl) to node[above] {$\fu \odot \id$} (tr);
\draw[->] (tl) to node[left]{$\fl$} (ml);
\draw[->] (tr) to node[left]{$\fx$} (mr);
\draw[->] (mr) to node[above] {$\fl \tens \fl$} (ml);
\end{tikzpicture}
    \end{aligned}
\end{equation}

%    \begin{equation}\label{eq:mondoub3}
%    \begin{aligned}
%  \xymatrix{U_{A\ten B}\odot (M\ten N)  \ar[r]\ar[d] &
 %   (U_A\ten U_B)\odot (M\ten N) \ar[d]\\
 %   M\ten N\ar@{<-}[r] & (U_A \odot M) \ten (U_B\odot N)}
 %       \end{aligned}
%    \end{equation}
\item \label{eq:mon2}The following diagrams commute, expressing that the
  associativity isomorphism for $\ten$ is a transformation of double
  categories.
\begin{equation}\label{eq:mondoub4}
\begin{aligned}
\begin{tikzpicture}[xscale=1.8, yscale=1.5]
\node (tl) at (0,2) {$((M_1 \tens N_1) \tens P_1) \odot ((M_2 \tens N_2) \tens P_2) $};
\node (tr) at (4,2) {$(M_1 \tens (N_1 \tens P_1)) \odot (M_2 \tens (N_2 \tens P_2))$};
\node (ml) at (0,1) {$((M_1 \tens N_1) \odot (M_2 \tens N_2)) \tens (P_1 \odot P_2)$};
\node (mr) at (4,1) {$(M_1 \odot M_2)  \tens ((N_1 \tens P_1) \odot (N_2 \tens P_2))$};
\node (bl) at (0,0) {$((M_1 \odot M_2) \tens (N_1 \odot N_2)) \tens (P_1 \odot P_2)$};
\node (br) at (4,0) {$(M_1 \odot M_2) \tens ((N_1 \odot N_2) \tens (P_1 \odot P_2))$};
\draw[->] (tl) to node[above] {$\fa \odot \fa$} (tr);
\draw[->] (tl) to node[left]{$\fx$} (ml);
\draw[->] (ml) to node[left]{$ \fx \tens \id$} (bl);
\draw[->] (tr) to node[left]{$\fx$} (mr);
\draw[->] (mr) to node[left]{$\id \tens \fx$} (br);
\draw[->] (bl) to node[above] {$\fa$} (br);
\end{tikzpicture}
    \end{aligned}
\end{equation}
  
%     {\small 
%\begin{equation}\label{eq:mondoub4}
%\begin{aligned}
%  \xymatrix{
%    ((M_1\ten N_1)\ten P_1) \odot ((M_2\ten N_2)\ten P_2) \ar[r]\ar[d] &
 %   (M_1\ten (N_1\ten P_1)) \odot (M_2\ten (N_2\ten P_2)) \ar[d]\\
    %((M_1\ten N_1) \odot (M_2\ten N_2)) \ten (P_1\odot P_2) \ar[d] &
%    (M_1\odot M_2) \ten ((N_1\ten P_1)\odot (N_2\ten P_2))\ar[d] \\
   % ((M_1\odot M_2) \ten(N_1\odot N_2)) \ten (P_1\odot P_2) \ar[r] &
%    (M_1\odot M_2) \ten ((N_1\odot N_2)\ten (P_1\odot P_2))}
   %     \end{aligned}
      %  \end{equation}
        
        \begin{equation}\label{eq:mondoub5}
\begin{aligned}
\begin{tikzpicture}[xscale=1.8, yscale=1.5]
\node (tl) at (0,2) {$U_{(A \tens B) \tens C}$};
\node (tr) at (4,2) {$U_{A \tens (B \tens C)}$};
\node (ml) at (0,1) {$U_{A \tens B} \tens U_C$};
\node (mr) at (4,1) {$U_A \tens U_{B \tens C}$};
\node (bl) at (0,0) {$(U_A \tens U_B) \tens U_C$};
\node (br) at (4,0) {$U_A \tens (U_B \tens U_C)$};
\draw[->] (tl) to node[above] {$U_{\alpha_{A,B,C}}$} (tr);
\draw[->] (tl) to node[left]{$\fu$} (ml);
\draw[->] (ml) to node[left]{$\fu \tens \id$} (bl);
\draw[->] (tr) to node[left]{$\fu$} (mr);
\draw[->] (mr) to node[left]{$\id \tens \fu$} (br);
\draw[->] (bl) to node[above] {$\alpha_{U_A, U_B, U_C}$} (br);
\end{tikzpicture}
    \end{aligned}
\end{equation}
%    \begin{equation}\label{eq:mondoub5}
  %  \begin{aligned}
  %\xymatrix{
%    U_{(A\ten B)\ten C} \ar[r] \ar[d] & U_{A\ten (B\ten C)} \ar[d]\\
   % U_{A\ten B} \ten U_C \ar[d] & U_A\ten U_{B\ten C}\ar[d]\\
%    (U_A\ten U_B)\ten U_C \ar[r] & U_A\ten (U_B\ten U_C) }
   %     \end{aligned}
   % \end{equation}
    %}

\item \label{eq:mon3}The following diagrams commute, expressing that the unit
  isomorphisms for $\ten$ are transformations of double categories.
 \begin{equation}\label{eq:mondoub6}
\begin{aligned}
\begin{tikzpicture}[xscale=2, yscale=1.5]
\node (tl) at (0,2) {$(M \tens U_I) \odot (N \tens U_I) $};
\node (tr) at (4,2) {$(M \odot N) \tens ( U_I \odot U_I)$};
\node (ml) at (0,1) {$M \odot N$};
\node (mr) at (4,1) {$(M \odot N) \tens U_I$};
\draw[->] (tl) to node[above] {$\fx$} (tr);
\draw[->] (tl) to node[left]{$\rho_M \odot \rho_M$} (ml);
\draw[->] (tr) to node[left]{$\id \tens \rho_{U_I}$} (mr);
\draw[->] (mr) to node[above] {$\rho_{M \odot N}$} (ml);
\end{tikzpicture}
    \end{aligned}
\end{equation}
      
      \begin{equation}\label{eq:mondoub7}
\begin{aligned}
\begin{tikzpicture}[yscale=1.5]
\node (tl) at (0,2) {$U_{A\tens I}$};
\node (tr) at (4,2) {$U_A \tens  U_I$};
\node (mr) at (4,1) {$U_A$};
\draw[->] (tl) to node[above]{$\fu$} (tr);
\draw[->] (tl) to node[left]{$U_{\rho_A}$} (mr);
\draw[->] (tr) to node[right]{$\rho_{U_A}$} (mr);
\end{tikzpicture}
    \end{aligned}
\end{equation}

 \begin{equation}\label{eq:mondoub8}
\begin{aligned}
\begin{tikzpicture}[xscale=2, yscale=1.5]
\node (tl) at (0,2) {$(U_I \tens M) \odot ( U_I \tens N) $};
\node (tr) at (4,2) {$( U_I \odot U_I) \tens (M \odot N) $};
\node (ml) at (0,1) {$M \odot N$};
\node (mr) at (4,1) {$U_I \tens (M \odot N) $};
\draw[->] (tl) to node[above] {$\fx$} (tr);
\draw[->] (tl) to node[left]{$\lambda_M \odot \lambda_N$} (ml);
\draw[->] (tr) to node[left]{$\lambda_{U_I} \tens \id$} (mr);
\draw[->] (mr) to node[above] {$\lambda_{M \odot N}$} (ml);
\end{tikzpicture}
    \end{aligned}
\end{equation}
      
      \begin{equation}\label{eq:mondoub9}
\begin{aligned}
\begin{tikzpicture}[yscale=1.5]
\node (tl) at (0,2) {$U_{I\tens A}$};
\node (tr) at (4,2) {$U_I \tens  U_A$};
\node (mr) at (4,1) {$U_A$};
\draw[->] (tl) to node[above]{$\fu$} (tr);
\draw[->] (tl) to node[left]{$U_{\lambda_A}$} (mr);
\draw[->] (tr) to node[right]{$\lambda_{U_A}$} (mr);
\end{tikzpicture}
    \end{aligned}
\end{equation}

  \setcounter{mondbl}{\value{enumi}}
\end{enumerate}
Similarly, a \textbf{braided monoidal double category} is a monoidal double
category with the following additional structure.
\begin{enumerate}\setcounter{enumi}{\value{mondbl}}
\item $\lD_0$ and $\lD_1$ are braided monoidal categories.
\item The functors $S$ and $T$ are strict braided monoidal (i.e.\ they
  preserve the braidings).
\item \label{eq:braid1} The following diagrams commute, expressing that the braiding is
  a transformation of double categories. 
  \begin{equation}\label{eq:brmondoub1}
\begin{aligned}
\begin{tikzpicture}[xscale=2, yscale=1.5]
\node (tl) at (0,2) {$(M_1 \odot M_2) \tens (N_1 \odot N_2)$};
\node (tr) at (4,2) {$(N_1 \odot N_2) \tens (M_1 \odot M_2)$};
\node (ml) at (0,1) {$(M_1 \tens N_1) \odot (M_2 \tens N_2)$};
\node (mr) at (4,1) {$(N_1 \tens M_1) \odot (N_2 \tens M_2)$};
\draw[->] (tl) to node[above] {$\fs$} (tr);
\draw[->] (tl) to node[left]{$\fx$} (ml);
\draw[->] (tr) to node[left]{$\fx$} (mr);
\draw[->] (ml) to node[above] {$\fs \odot \fs$} (mr);
\end{tikzpicture}
    \end{aligned}
\end{equation}
  %\[\xymatrix{(M_1\odot M_2)\ten (N_1\odot N_2) \ar[r]^\fs\ar[d]_\fx &
  %  (N_1\odot N_2)\ten (M_1 \odot M_2)\ar[d]^\fx\\
   % (M_1\ten N_1)\odot (M_2\ten N_2) \ar[r]_{\fs\odot \fs} &
  %  (N_1\ten M_1) \odot (N_2 \ten M_2)}
 % \]
 \begin{equation}\label{eq:brmondoub2}
\begin{aligned}
\begin{tikzpicture}[xscale=2, yscale=1.5]
\node (tl) at (0,2) {$U_A \tens U_B$};
\node (tr) at (4,2) {$U_{A \tens B}$};
\node (ml) at (0,1) {$U_B \tens U_A$};
\node (mr) at (4,1) {$U_{B \tens A}$};
\draw[->] (tl) to node[above] {$\fu$} (tr);
\draw[->] (tl) to node[left]{$\fs$} (ml);
\draw[->] (tr) to node[left]{$U_{\fs}$} (mr);
\draw[->] (ml) to node[above] {$\fu$} (mr);
\end{tikzpicture}
    \end{aligned}
\end{equation}

 % \[\xymatrix{U_A \ten U_B \ar[r]^(0.55)\fu \ar[d]_\fs &
  %  U_{A\ten B} \ar[d]^{U_\fs}\\
  %  U_B\ten U_A \ar[r]_(0.55)\fu &
  %  U_{B\ten A}}.
  %\]
  \setcounter{mondbl}{\value{enumi}}
\end{enumerate}
Finally, a \textbf{symmetric monoidal double category} is a braided one such that
\begin{enumerate}\setcounter{enumi}{\value{mondbl}}
\item $\lD_0$ and $\lD_1$ are in fact symmetric monoidal.
\end{enumerate}
While there are a fair number of coherence diagrams in this definition, most of
them are fairly small, and in any given case most or all of them are
fairly obvious.  Thus, verifying that a given double category is
(braided or symmetric) monoidal is not a great deal of work.

%\fxnote[author=LW]{If we merge the examples with the final section, this should be included.}
\begin{eg}
  The examples \lMod, \lnCob, and \lProf\ are all easily seen to be
  symmetric monoidal under the tensor product of rings, disjoint union
  of manifolds, and cartesian product of categories, respectively.
\end{eg}

\begin{rmk}
  In a 2-category with finite products there is additionally the
  notion of a \emph{cartesian object}: one such that the diagonal
  $D\to D\times D$ and projection $D\to 1$ have right adjoints.  Any
  cartesian object is a symmetric pseudomonoid in a canonical way,
  just as any category with finite products is a monoidal category
  with its cartesian product.  Many of the ``cartesian bicategories''
  considered in~\cite{cw:cart-bicats-i,ckww:cartbicats-ii} are in
  fact the loose bicategory of some cartesian object in \cDbl,
  and inherit their monoidal structure in this way.
  Cartesian double categories have recently been further studied by~\cite{aleiferi2018cartesian}.
\end{rmk}

Two further general methods for constructing symmetric monoidal double
categories can be found in~\cite{shulman:frbi}.

\begin{rmk}
  The general yoga of internalization says that an $X$ internal to
  $Y$s internal to $Z$s is equivalent to a $Y$ internal to $X$s
  internal to $Z$s, but this is only strictly true when the
  internalizations are all strict.  We have defined a symmetric
  monoidal double category to be a (pseudo) symmetric monoid internal
  to (pseudo) categories internal to categories, but one could also
  consider a (pseudo) category internal to (pseudo) symmetric monoids
  internal to categories, i.e.\ a pseudo internal category in the
  2-category
  $\mathcal{S}\mathit{ym}\mathcal{M}\mathit{on}\mathcal{C}\mathit{at}$
  of symmetric monoidal categories and strong symmetric monoidal
  functors.  This would give \emph{almost} the same definition, except
  that $S$ and $T$ would only be strong monoidal (preserving $\ten$ up
  to isomorphism) rather than strict monoidal.  We prefer our
  definition, since $S$ and $T$ are strict monoidal in almost all
  examples, and keeping track of their constraints would be tedious.
\end{rmk}

Just as every bicategory is equivalent to a strict 2-category, it is
proven in~\cite{gp:double-limits} that every pseudo double category is
equivalent to a strict double category (one in which the associativity
and unit constraints for $\odot$ are identities).  Thus, from now on
we will usually omit to write these constraint isomorphisms (or
equivalently, implicitly strictify our double categories).  We
\emph{will} continue to write the constraint isomorphisms for the
monoidal structure $\ten$, since these are where the whole question
lies.

We now move on to define functors and transformations of monoidal double categories.
Like monoidal double categories themselves, these are also special cases of a notion that makes sense internal to any 2-category with products.

\begin{defn}\label{def:monfunc}
  Let $\D$, $\E$ be (braided/symmetric) monoidal double categories.  A {\bf (braided or symmetric) lax monoidal double functor} $F: \D \rightarrow \E$ is a pseudo double functor $F$, together with transformations $\phi : \otimes \circ (F,F) \rightarrow F \circ \otimes$ and $\phi_u:I_{\E}\rightarrow F \circ I_{\D}$ satisfying the usual axioms for (braided/symmetric) monoidal functors with respect to $\otimes$.
\end{defn}

Unfolding the definitions gives us:

\begin{enumerate}
\item $F_0$ and $F_1$ are (braided/symmetric) monoidal functors.
\item The equalities $F_0 \circ S_\D = S_\E \circ F_1$ and $F_0 \circ T_\D = T_\E \circ F_1$ are strict equalities of monoidal functors.
\item The following diagrams commute, expressing that $\phi$ is a transformation of double categories:

\begin{tikzpicture}[xscale=2, yscale=2]
\node (tl) at (0,2) {$(FN \otimes FL) \odot (FM \otimes FK)$};
\node (tr) at (4,2) {$F(N\otimes L) \odot F(M \otimes K)$};
\draw[->] (tl) to node[above] {$\phi \odot \phi$} (tr);
\node (ml) at (0,1) {$(FN \odot FM) \otimes (FL \odot FK)$};
\node (mr) at (4,1) {$F((N \otimes L) \odot (M \otimes K))$};
\node (bl) at (0,0) {$F(N \odot M) \otimes F(L \odot K)$};
\node (br) at (4,0) {$F((N \odot M)\otimes(L\odot K))$};
\draw[->] (tl) to node[left]{$\xi$} (ml);
\draw[->] (ml) to node[left]{$F_{\odot} \otimes F_{\odot}$} (bl);
\draw[->] (tr) to node[left]{$F_{\odot}$} (mr);
\draw[->] (mr) to node[left]{$F(\xi)$} (br);
\draw[->] (bl) to node[above] {$\phi$} (br);
\end{tikzpicture}

\begin{tikzpicture}[yscale=2]
\node (tl) at (0,2) {$U_{FA \otimes FB}$};
\node (tr) at (4,2) {$U_{F(A \otimes B)} $};
\draw[->] (tl) to node[above] {$U_{\phi}$} (tr);
\node (ml) at (0,1) {$U_{FA} \otimes U_{FB}$};
\node (mr) at (4,1) {$F(U_{A\otimes B}) $};
\node (bl) at (0,0) {$F(U_A) \otimes F(U_B)$};
\node (br) at (4,0) {$F(U_A \otimes U_B)$};
\draw[->] (tl) to node[left]{$u$} (ml);
\draw[->] (ml) to node[left]{$F_u \otimes F_u$} (bl);
\draw[->] (tr) to node[left]{$F_U$} (mr);
\draw[->] (mr) to node[left]{$F \circ u$} (br);
\draw[->] (bl) to node[above] {$\phi$} (br);
\end{tikzpicture}

\end{enumerate}

When the natural transformations are in the opposite direction, the functor is {\bf colax monoidal}, and when they are isomorphisms, the functor is {\bf strong monoidal}.

\begin{defn}\label{Def:monverttrans}
  Let $\D$, $\E$ be monoidal double categories and let $(F, \phi) ,(G,\psi): \D \rightarrow \E$ be monoidal double functors. A \textbf{monoidal transformation} $\alpha: F \rightarrow G$ is a tight transformation such that $\alpha_0$ and $\alpha_1$ are monoidal natural transformations.
  Explicitly, this means (in the lax case) that the following equalities hold:

\begin{equation}
\begin{aligned}
\begin{tikzpicture}
\node (tl) at (0,4) {$FA \otimes FB$};
\node (tr) at (4,4) {$FC \otimes FD$};
\node (ml) at (0,2) {$F(A\otimes B)$};
\node (mr) at (4,2) {$F(C \otimes D)$};
\node (bl) at (0,0) {$G(A \otimes B)$};
\node (br) at (4,0) {$G(C \otimes D)$};
\draw[style=tickarrow] (tl) to node [above] {$FM \otimes FN$} (tr);
\draw[style=tickarrow] (ml) to node [above] {$F(M\otimes N)$} (mr);
\draw[->] (tl) to node [left] {$\phi_{A,B}$} (ml);
\draw[->] (tr) to node [right] {$\phi_{C,D}$}(mr);
\draw[->] (ml) to node [left] {$\alpha_{A\otimes B}$} (bl);
\draw[->] (mr) to node [right] {$\alpha_{C \otimes D}$} (br);
\draw[style=tickarrow] (bl) to node [above] {$G(M \otimes N)$} (br);
\node at (2,3) {$\Downarrow \phi_{M,N}$};
\node at (2,1) {$\Downarrow \alpha_{M \otimes N}$};
\end{tikzpicture}
\end{aligned}
=
\begin{aligned}
\begin{tikzpicture}
\node (tl) at (0,4) {$FA \otimes FB$};
\node (tr) at (4,4) {$FC \otimes FD$};
\node (ml) at (0,2) {$GA \otimes GB$};
\node (mr) at (4,2) {$GC \otimes GD$};
\node (bl) at (0,0) {$G(A \otimes B)$};
\node (br) at (4,0) {$G(C \otimes D)$};
\draw[style=tickarrow] (tl) to node [above] {$FM \otimes FN$} (tr);
\draw[style=tickarrow] (ml) to node [above] {$GM \otimes GN$} (mr);
\draw[->] (tl) to node [left] {$\alpha_A \otimes \alpha_B$} (ml);
\draw[->] (tr) to node [right] {$\alpha_C \otimes \alpha_D$} (mr);
\draw[->] (ml) to node [left] {$\psi_{A,B}$} (bl);
\draw[->] (mr) to node [right] {$\psi_{C,D}$} (br);
\draw[style=tickarrow] (bl) to node [above] {$G(M \otimes N)$} (br);
\node at (2,3) {$\Downarrow \alpha_M \otimes \alpha_N$};
\node at (2,1) {$\Downarrow \psi_{M,N}$};
\end{tikzpicture}
\end{aligned}
\end{equation}

\begin{equation}
\begin{aligned}
\begin{tikzpicture}
\node (tl) at (0,4) {$I_{\mathbb{E}}$};
\node (tr) at (4,4) {$I_{\mathbb{E}}$};
\node (ml) at (0,2) {$F(I_{\mathbb{D}})$};
\node (mr) at (4,2) {$F(I_{\mathbb{D}})$};
\node (bl) at (0,0) {$G(I_{\mathbb{D}})$};
\node (br) at (4,0) {$G(I_{\mathbb{D}})$};
\draw[style=tickarrow] (tl) to node [above] {$U_{I_{\mathbb{E}}}$} (tr);
\draw[style=tickarrow] (ml) to node [above] {$F(U_{I_{\mathbb{D}}})$} (mr);
\draw[->] (tl) to node [left] {$\phi_{u_0}$} (ml);
\draw[->] (tr) to node [right] {$\phi_{u_1}$}(mr);
\draw[->] (ml) to node [left] {$\alpha_{I_{\mathbb{D}}}$} (bl);
\draw[->] (mr) to node [right] {$\alpha_{I_{\mathbb{D}}}$} (br);
\draw[style=tickarrow] (bl) to node [below] {$G(U_{I_{\mathbb{D}}})$} (br);
\node at (2,3) {$\Downarrow \phi_{u_1}$};
\node at (2,1) {$\Downarrow \alpha_{U_{I_{\mathbb{D}}}}$};
\end{tikzpicture}
\end{aligned}
=
\begin{aligned}
\begin{tikzpicture}
\node (tl) at (0,4) {$I_{\mathbb{E}}$};
\node (tr) at (4,4) {$I_{\mathbb{E}}$};
\node (ml) at (0,2) {$G(I_{\mathbb{D}})$};
\node (mr) at (4,2) {$G(I_{\mathbb{D}})$};
\draw[style=tickarrow] (tl) to node [above] {$U_{I_{\mathbb{E}}}$} (tr);
\draw[style=tickarrow] (ml) to node [below] {$G(U_{I_{\mathbb{D}}})$} (mr);
\draw[->] (tl) to node [left] {$\psi_{u}$} (ml);
\draw[->] (tr) to node [right] {$\psi_{u}$}(mr);
\node at (2,3) {$\Downarrow \psi_{u_1}$};
\end{tikzpicture}
\end{aligned}
\end{equation}


A {\bf braided or symmetric monoidal tight transformation} is a monoidal transformation between braided/symmetric monoidal functors.
\end{defn}

We have three strict 2-categories $\cMonDbll, \cMonDblc,\cMonDblp$ of monoidal double categories and lax, colax, or pseudo monoidal functors, respectively.
(More generally, we have three such 2-categories of pseudomonoids internal to any 2-category with finite products.)


% Local Variables:
% TeX-master: "smbicat"
% End:


\section{Companions and conjoints}
\label{sec:comp-conj}

Suppose that \lD\ is a monoidal double category; when does
$\cH(\lD)$ become a monoidal bicategory?  It clearly has a
unit object $I$, and the pseudo double functor $\ten\maps
\lD\times\lD\to\lD$ clearly induces a functor $\ten\maps
\cH(\lD)\times\cH(\lD)\to\cH(\lD)$.  However, the problem is that the
constraint isomorphisms such as $A\ten (B\ten C)\iso (A\ten B)\ten C$
are \emph{tight} 1-cells, which get discarded when we pass to
$\cH(\lD)$.  Thus, in order for $\cH(\lD)$ to inherit a symmetric
monoidal structure, we must have a way to make tight 1-cells
into loose ones.  Thus is the purpose of the following
definition.


\begin{defn}\label{def:companion}
  Let \lD\ be a double category and $f\maps A\to B$ a tight
  1-cell.  A \textbf{companion} of $f$ is a loose 1-cell
  $\fhat\maps A\hto B$ together with 2-morphisms
  \begin{equation*}
    \begin{array}{c}
      \xymatrix@-.5pc{
        \ar[r]|-@{|}^-{\fhat} \ar[d]_f \ar@{}[dr]|{\Downarrow \epsilon_{\hat{f}} }
        & \ar@{=}[d]\\
        \ar[r]|-@{|}_-{U_B} & }
    \end{array}\quad\text{and}\quad
    \begin{array}{c}
      \xymatrix@-.5pc{
        \ar[r]|-@{|}^-{U_A} \ar@{=}[d] \ar@{}[dr]|{\Downarrow \eta_{\hat{f}}}
        & \ar[d]^f\\
        \ar[r]|-@{|}_-{\fhat} & }
    \end{array}
  \end{equation*}
  such that the following equations hold.
  \begin{align}\label{eq:compeqn}
    \begin{array}{c}
      \xymatrix@-.5pc{
        \ar[r]|-@{|}^-{U_A} \ar@{=}[d] \ar@{}[dr]|{\Downarrow \eta_{\hat{f}}}
        & \ar[d]^f\\
        \ar[r]|-{\fhat} \ar[d]_f \ar@{}[dr]|
        {\Downarrow  \epsilon_{\hat{f}} }
        & \ar@{=}[d]\\
        \ar[r]|-@{|}_-{U_B} & }
    \end{array} &= 
    \begin{array}{c}
      \xymatrix@-.5pc{ \ar[r]|-@{|}^-{U_A} \ar[d]_f
        \ar@{}[dr]|{\Downarrow U_f} &  \ar[d]^f\\
        \ar[r]|-@{|}_-{U_B} & }
    \end{array}
    &
    \begin{array}{c}
      \xymatrix@-.5pc{
        \ar[r]|-@{|}^-{U_A} \ar@{=}[d] \ar@{}[dr]|{ \Downarrow \eta_{\hat{f}}}&
        \ar[r]|-@{|}^{\fhat}\ar[d]|f \ar@{}[dr]|{\Downarrow  \epsilon_{\hat{f}} }
        & \ar@{=}[d]\\
        \ar[r]|-@{|}_-{\fhat} &
        \ar[r]|-@{|}_-{U_B} &}
%       \xymatrix@-.5pc{
%         \ar[rr]|-@{|}^-{\fhat} \ar@{}[drr]|\iso \ar@{=}[d] &&
%         \ar@{=}[d] \\
%         \ar[r]|-@{|}^-{U_A} \ar@{=}[d] \ar@{}[dr]|\Downarrow &
%         \ar[r]|-@{|}^-{\fhat} \ar[d]_f \ar@{}[dr]|\Downarrow
%         & \ar@{=}[d]\\
%         \ar[r]|-@{|}_-{\fhat} &
%         \ar[r]|-@{|}_-{U_B} &\\
%         \ar[rr]|-@{|}_-{\fhat} \ar@{}[urr]|\iso \ar@{=}[u] &&
%         \ar@{=}[u]}
    \end{array} &=
    \begin{array}{c}
      \xymatrix@-.5pc{
        \ar[r]|-@{|}^-{\fhat} \ar@{=}[d] \ar@{}[dr]|{\Downarrow 1_{\fhat}}
        & \ar@{=}[d]\\
        \ar[r]|-@{|}_-{\fhat} & }
    \end{array}
  \end{align}
  A \textbf{conjoint} of $f$, denoted $\fchk\maps B\hto A$, is a
  companion of $f$ in the ``loose opposite'' double category $\lD\lop$.
\end{defn}

\begin{rmk}
  We momentarily suspend our convention of pretending that our double
  categories are strict to mention that the second
  equation in~\eqref{eq:compeqn} actually requires an insertion of unit
  isomorphisms to make sense.
\end{rmk}

The form of this definition is due
to~\cite{gp:double-adjoints,dpp:spans}, but the ideas date back
to~\cite{bs:dblgpd-xedmod}; see
also~\cite{bm:dbl-thin-conn,fiore:pscat}.  In the terminology of these
references, a \emph{connection} on a double category is equivalent to
a strictly functorial choice of a companion for each tight arrow.

% a loose 1-cell $\fchk\maps B\hto
%   A$ together with 2-morphisms
%   \[\begin{array}{c}
%     \xymatrix@-.5pc{
%       \ar[r]|-@{|}^-{\fchk} \ar@{=}[d] \ar@{}[dr]|\Downarrow
%       & \ar[d]^f\\
%       \ar[r]|-@{|}_-{U_B} & }
%   \end{array}\quad\text{and}\quad
%   \begin{array}{c}
%     \xymatrix@-.5pc{
%       \ar[r]|-@{|}^-{U_A} \ar[d]_f \ar@{}[dr]|\Downarrow
%       & \ar@{=}[d]\\
%       \ar[r]|-@{|}_-{\fchk} & }
%   \end{array}\]
%   such that the following equations hold.
%   \begin{align*}
%     \begin{array}{c}
%       \xymatrix@-.5pc{
%         \ar[r]|-@{|}^-{U_A} \ar[d]_f \ar@{}[dr]|\Downarrow
%         & \ar@{=}[d]\\
%         \ar[r]|-{\fchk} \ar@{=}[d] \ar@{}[dr]|\Downarrow
%         & \ar[d]^f\\
%         \ar[r]|-@{|}_-{U_B} & }
%     \end{array} &= 
%     \begin{array}{c}
%       \xymatrix@-.5pc{ \ar[r]|-@{|}^-{U_A} \ar[d]_f
%         \ar@{}[dr]|{\Downarrow U_f} &  \ar[d]^f\\
%         \ar[r]|-@{|}_-{U_B} & }
%     \end{array}
%     &
%     \begin{array}{c}
%       \xymatrix@-.5pc{
%         \ar[rr]|-@{|}^-{\fchk} \ar@{}[drr]|\iso \ar@{=}[d] &&
%         \ar@{=}[d] \\
%         \ar[r]|-@{|}^-{\fchk} \ar@{=}[d] \ar@{}[dr]|\Downarrow &
%         \ar[r]|-@{|}^-{U_A} \ar[d]_f \ar@{}[dr]|\Downarrow
%         & \ar@{=}[d]\\
%         \ar[r]|-@{|}_-{U_B} &
%         \ar[r]|-@{|}_-{\fchk} &\\
%         \ar[rr]|-@{|}_-{\fchk} \ar@{}[urr]|\iso \ar@{=}[u] &&
%         \ar@{=}[u]}
%     \end{array} &=
%     \begin{array}{c}
%       \xymatrix@-.5pc{
%         \ar[r]|-@{|}^-{\fchk} \ar@{=}[d]
%         & \ar@{=}[d]\\
%         \ar[r]|-@{|}_-{\fchk} & }
%     \end{array}
%   \end{align*}

\fxnote[author=LW]{If we merge the examples with the final section, this should be included.}
\begin{egs}
  \lMod, \lnCob, and \lProf\ have companions and conjoints for all tight morphisms.  In \lMod, the companion
  of a ring homomorphism $f\maps A\to B$ is $B$ regarded as an
  $A$-$B$-bimodule via $f$ on the left, and dually for its conjoint.
  In \lnCob, companions and conjoints are obtained by regarding a
  diffeomorphism as a cobordism.  And in \lProf, companions and
  conjoints are obtained by regarding a functor $f\maps A\to B$ as a
  `representable' profunctor $B(f-,-)$ or $B(-,f-)$.
\end{egs}

% \begin{rmk}
%   For an $(n\times 1)$-category (recall \autoref{rmk:monglob}), the
%   lifting condition we should require is simply that each double
%   category $\lD_{i+1} \toto \lD_i$, for $0\le i < n$, is fibrant.
% \end{rmk}

The existence of companions and conjoints gives us a way to `lift'
tight 1-cells to loose ones.  What is even more crucial
for our application, however, is that these liftings are unique up to
isomorphism, and that these isomorphisms are canonical and coherent.
This is the content of the following lemmas.  We state most of them
only for companions, but all have dual versions for conjoints.

%Later we will use them to prove the existence of structure isomorphisms and the commutativity of diagrams needed for $\cH$ to preserve monoidal structures.

\begin{lem}\label{thm:theta}
  Let $\fhat\maps A\hto B$ and $\fhat'\maps A\hto B$ be companions of
  $f$ (that is, each comes \emph{equipped with} 2-morphisms as in
  \autoref{def:companion}).  Then there is a unique globular isomorphism
  $\theta_{\fhat,\fhat'}\maps \fhat\too[\iso]\fhat'$ such that
  \begin{equation}\label{eq:comp-iso}
    \vcenter{\xymatrix@R=1.5pc{
        \ar[r]|-@{|}^-{U_A} \ar@{=}[d] \ar@{}[dr]|{\Downarrow \eta_{\hat{f}}} &  \ar[d]^f\\
        \ar[r]|-{\fhat} \ar@{=}[d] \ar@{}[dr]|{\Downarrow \theta_{\fhat,\fhat'}} &  \ar@{=}[d]\\
        \ar[r]|-{\fhat'} \ar[d]_f \ar@{}[dr]|{\Downarrow \epsilon_{\hat{f}'}} & \ar@{=}[d]\\
        \ar[r]|-@{|}_-{U_B} & }} \quad = \quad
    \vcenter{\xymatrix@-.5pc{ \ar[r]|-@{|}^-{U_A} \ar[d]_f
        \ar@{}[dr]|{\Downarrow U_f} &  \ar[d]^f\\
        \ar[r]|-@{|}_-{U_B} & .}}
  \end{equation}
\end{lem}
\begin{proof}
  Composing~\eqref{eq:comp-iso} on the left with
  $\vcenter{\xymatrix@-.5pc{ \ar[r]|-@{|}^-{U_A} \ar@{=}[d]
      \ar@{}[dr]|{\Downarrow \eta_{\hat{f}}} & \ar[d]^f\\ \ar[r]|-@{|}_-{\fhat'} & }}$
  and on the right with $\vcenter{\xymatrix@-.5pc{
      \ar[r]|-@{|}^-{\fhat} \ar[d]_f \ar@{}[dr]|{\Downarrow \epsilon_{\hat{f}}}&
      \ar@{=}[d]\\ \ar[r]|-@{|}_-{U_B} & }}$, and using the second
  equation~\eqref{eq:compeqn}, we see that if~\eqref{eq:comp-iso} is
  satisfied then $\theta_{\fhat,\fhat'}$ must be the composite
  \begin{equation}
    \vcenter{\xymatrix@-.5pc{
        \ar[r]|-@{|}^-{U_A} \ar@{=}[d] \ar@{}[dr]|{\Downarrow \eta_{\hat{f}'}}&
        \ar[r]|-@{|}^-{\fhat} \ar[d]|f \ar@{}[dr]|{\Downarrow \epsilon_{\hat{f}}}
        & \ar@{=}[d]\\
        \ar[r]|-@{|}_-{\fhat'} &
        \ar[r]|-@{|}_-{U_B} &}}\label{eq:theta}
%     \vcenter{\xymatrix@-.5pc{
%         \ar[rr]|-@{|}^-{\fhat} \ar@{}[drr]|\iso \ar@{=}[d] &&
%         \ar@{=}[d] \\
%         \ar[r]|-@{|}^-{U_A} \ar@{=}[d] \ar@{}[dr]|\Downarrow &
%         \ar[r]|-@{|}^-{\fhat} \ar[d]_f \ar@{}[dr]|\Downarrow
%         & \ar@{=}[d]\\
%         \ar[r]|-@{|}_-{\fhat'} &
%         \ar[r]|-@{|}_-{U_B} &\\
%         \ar[rr]|-@{|}_-{\fhat'} \ar@{}[urr]|\iso \ar@{=}[u] &&
%         \ar@{=}[u]}}\label{eq:theta}
  \end{equation}
  Two applications of the first equation~\eqref{eq:compeqn} shows that
  this indeed satisfies~\eqref{eq:comp-iso}.  As for its being an
  isomorphism, we have the dual composite $\theta_{\fhat',\fhat}$:
  \[\vcenter{\xymatrix@-.5pc{
      \ar[r]|-@{|}^-{U_A} \ar@{=}[d] \ar@{}[dr]|{\Downarrow \eta_{\hat{f}}} &
      \ar[r]|-@{|}^{\fhat'}\ar[d]|f \ar@{}[dr]|
{\Downarrow \epsilon_{\hat{f}'}}
      & \ar@{=}[d]\\
      \ar[r]|-@{|}_-{\fhat} &
      \ar[r]|-@{|}_-{U_B} &}}\]
  which we verify is an inverse using~\eqref{eq:compeqn}:
  \[\vcenter{\xymatrix@-.5pc{
      \ar[r]|-@{|}^{U_A}\ar@{=}[d] \ar@{}[dr]|{=} &
      \ar[r]|-@{|}^{U_A}\ar@{=}[d] \ar@{}[dr]|{\Downarrow \eta_{\hat{f}'}} &
      \ar[r]|-@{|}^{\fhat}\ar[d]|f \ar@{}[dr]|{\Downarrow \epsilon_{\hat{f}}} &
      \ar@{=}[d]\\
      \ar[r]|{U_A}\ar@{=}[d] \ar@{}[dr]|{\Downarrow \eta_{\hat{f}}} &
      \ar[r]|{\fhat'}\ar[d]|f \ar@{}[dr]|{\Downarrow \epsilon_{\hat{f}'}} &
      \ar[r]|{U_B}\ar@{=}[d] \ar@{}[dr]|{=} &
      \ar@{=}[d]\\
      \ar[r]|-@{|}_{\fhat} &
      \ar[r]|-@{|}_{U_B} &
      \ar[r]|-@{|}_{U_B} &
    }} \;=\;
  \vcenter{\xymatrix@-.5pc{
      \ar[r]|-@{|}^-{U_A} \ar@{=}[d] \ar@{}[dr]|{\Downarrow  \eta_{\hat{f}}}&
      \ar[r]|-@{|}^{\fhat}\ar[d]|f \ar@{}[dr]|{\Downarrow \epsilon_{\hat{f}}}
      & \ar@{=}[d]\\
      \ar[r]|-@{|}_-{\fhat} &
      \ar[r]|-@{|}_-{U_B} &}} \;=\;
  \vcenter{\xymatrix@-.5pc{
      \ar[r]|-@{|}^-{\fhat} \ar@{=}[d] \ar@{}[dr]|{\Downarrow 1_{\fhat}}
      & \ar@{=}[d]\\
      \ar[r]|-@{|}_-{\fhat} & }}\]
  (and dually).
\end{proof}

\begin{lem}\label{thm:theta-id}
  For any companion \fhat\ of $f$ we have $\theta_{\fhat,\fhat}=1_{\fhat}$.
\end{lem}
\begin{proof}
  This is the second equation~\eqref{eq:compeqn}.
\end{proof}

\begin{lem}\label{thm:theta-compose-vert}
  Suppose that $f$ has three companions $\fhat$, $\fhat'$, and
  $\fhat''$.  Then $\theta_{\fhat,\fhat''} = \theta_{\fhat',\fhat''}
  \circ\theta_{\fhat,\fhat'}$.
\end{lem}
\begin{proof}
  By definition, we have
  \[\theta_{\fhat',\fhat''} \circ\theta_{\fhat,\fhat'} =\;
  \vcenter{\xymatrix@-.5pc{
      \ar[r]|-@{|}^{U_A}\ar@{=}[d] \ar@{}[dr]|{=} &
      \ar[r]|-@{|}^{U_A}\ar@{=}[d] \ar@{}[dr]|{\Downarrow \eta_{\hat{f}'}} &
      \ar[r]|-@{|}^{\fhat}\ar[d]|f \ar@{}[dr]|{\Downarrow \epsilon_{\hat{f}}} &
      \ar@{=}[d]\\
      \ar[r]|{U_A}\ar@{=}[d] \ar@{}[dr]|{\Downarrow \eta_{\hat{f}''}} &
      \ar[r]|{\fhat'}\ar[d]|f \ar@{}[dr]|{\Downarrow \epsilon_{\hat{f}'}} &
      \ar[r]|{U_B}\ar@{=}[d] \ar@{}[dr]|{=} &
      \ar@{=}[d]\\
      \ar[r]|-@{|}_{\fhat''} &
      \ar[r]|-@{|}_{U_B} &
      \ar[r]|-@{|}_{U_B} &
    }} \;=\;
  \vcenter{\xymatrix@-.5pc{
      \ar[r]|-@{|}^-{U_A} \ar@{=}[d] \ar@{}[dr]|{\Downarrow \eta_{\hat{f}''}}&
      \ar[r]|-@{|}^{\fhat}\ar[d]|f \ar@{}[dr]|{\Downarrow \epsilon_{\hat{f}}}
      & \ar@{=}[d]\\
      \ar[r]|-@{|}_-{\fhat''} &
      \ar[r]|-@{|}_-{U_B} &}} \;=
  \theta_{\fhat,\fhat''}\]
  as desired.
\end{proof}

\begin{lem}\label{thm:comp-unit}
  $U_A\maps A\hto A$ is always a companion of $1_A\maps A\to A$ in a
  canonical way.
\end{lem}
\begin{proof}
  We take both defining 2-morphisms to be
  $1_{U_A}$; the truth of~\eqref{eq:compeqn} is evident.
\end{proof}

\begin{lem}\label{thm:comp-compose}
  Suppose that $f\maps A\to B$ has a companion \fhat\ and $g\maps B\to
  C$ has a companion \ghat.  Then $\ghat\odot\fhat$ is a companion of
  $gf$.
\end{lem}
\begin{proof}
  We take the defining 2-morphisms to be the composites
  \[\vcenter{\xymatrix@-.5pc{
      \ar[r]|-@{|}^-{\fhat} \ar[d]_f \ar@{}[dr]|{\Downarrow \epsilon_{\hat{f}}}&
      \ar[r]|-@{|}^-{\ghat} \ar@{=}[d] \ar@{}[dr]|{1_{\ghat}} &
      \ar@{=}[d]\\
      \ar[r]|-{U_B} \ar[d]_g \ar@{}[dr]|{U_g} &
      \ar[r]|-{\ghat} \ar[d]|g \ar@{}[dr]|{\Downarrow \epsilon_{\hat{g}}}&
      \ar@{=}[d]\\
      \ar[r]|-@{|}_-{U_C} &
      \ar[r]|-@{|}_-{U_C} &
    }}\quad\text{and}\quad
  \vcenter{\xymatrix@-.5pc{
      \ar[r]|-@{|}^-{U_A} \ar@{=}[d] \ar@{}[dr]|{\Downarrow \eta_{\hat{f}}} &
      \ar[r]|-@{|}^-{U_A} \ar[d]|f \ar@{}[dr]|{U_f} &
      \ar[d]^f\\
      \ar[r]|-{\fhat} \ar@{=}[d] \ar@{}[dr]|{1_{\fhat}} &
      \ar[r]|-{U_B} \ar@{=}[d] \ar@{}[dr]|{\Downarrow \eta_{\hat{g}}}&
      \ar[d]^g\\
      \ar[r]|-@{|}_-{\fhat} &
      \ar[r]|-@{|}_-{\ghat} &
    }}
  \]
  It is easy to verify that these satisfy~\eqref{eq:compeqn}, using
  the interchange law for $\odot$ and $\circ$ in a double category.
\end{proof}

\begin{lem}\label{thm:theta-compose-horiz}
  Suppose that $f\maps A\to B$ has companions $\fhat$ and $\fhat'$,
  and that $g\maps B\to C$ has companions $\ghat$ and $\ghat'$.  Then
  $\theta_{\ghat,\ghat'}\odot \theta_{\fhat,\fhat'}  =
    \theta_{\ghat\odot\fhat, \ghat'\odot\fhat'}$.
\end{lem}
\begin{proof}
  Using the interchange law for $\odot$ and $\circ$, we have:
  \begin{align}
    \theta_{\ghat\odot\fhat, \ghat'\odot\fhat'} &=\;
    \vcenter{\xymatrix@-.5pc{
        \ar[r]|-@{|}^-{U_A} \ar@{=}[d] \ar@{}[dr]|{\Downarrow \eta_{\hat{f}'}}&
        \ar[r]|-@{|}^-{U_A} \ar[d]|f \ar@{}[dr]|{U_f} &
        \ar[r]|-@{|}^-{\fhat} \ar[d]|f \ar@{}[dr]|{\Downarrow \epsilon_{\hat{f}}}&
        \ar[r]|-@{|}^-{\ghat} \ar@{=}[d] \ar@{}[dr]|{1_{\fhat}} &
        \ar@{=}[d]\\
        \ar[r]|-{\fhat'} \ar@{=}[d] \ar@{}[dr]|{1_{\ghat}} &
        \ar[r]|-{U_B} \ar@{=}[d] \ar@{}[dr]|{\Downarrow \eta_{\hat{g}'}} &
        \ar[r]|-{U_B} \ar[d]|g \ar@{}[dr]|{U_g} &
        \ar[r]|-{\ghat} \ar[d]|g \ar@{}[dr]|{\Downarrow \epsilon_{\hat{g}}} &
        \ar@{=}[d]\\
        \ar[r]|-@{|}_-{\fhat'} &
        \ar[r]|-@{|}_-{\ghat'} &
        \ar[r]|-@{|}_-{U_C} &
        \ar[r]|-@{|}_-{U_C} &
      }}
    \;=\;
    \vcenter{\xymatrix@-.5pc{
        \ar[r]|-@{|}^-{U_A} \ar@{=}[d] \ar@{}[dr]|{\Downarrow \eta_{\hat{f}'}}&
        \ar[r]|-@{|}^-{\fhat} \ar[d]|f \ar@{}[dr]|{\Downarrow \epsilon_{\hat{f}}}&
        \ar[r]|-@{|}^-{\ghat} \ar@{=}[d] \ar@{}[dr]|{1_{\fhat}} &
        \ar@{=}[d]\\
        \ar[r]|-{\fhat'} \ar@{=}[d] \ar@{}[dr]|{1_{\ghat}} &
        \ar[r]|-{U_B} \ar@{=}[d] \ar@{}[dr]|{\Downarrow \eta_{\hat{g}'}} &
        \ar[r]|-{\ghat} \ar[d]|g \ar@{}[dr]|{\Downarrow \epsilon_{\hat{g}}} &
        \ar@{=}[d]\\
        \ar[r]|-@{|}_-{\fhat'} &
        \ar[r]|-@{|}_-{\ghat'} &
        \ar[r]|-@{|}_-{U_C} &
      }}\\
    &=\;
    \vcenter{\xymatrix@-.5pc{
        \ar[r]|-@{|}^-{U_A} \ar@{=}[d] \ar@{}[dr]|{\Downarrow \eta_{\hat{f}'}}&
        \ar[r]|-@{|}^-{\fhat} \ar[d]|f \ar@{}[dr]|{\Downarrow \epsilon_{\hat{f}}}&
        \ar[r]|-@{|}^-{U_B} \ar@{=}[d] \ar@{}[dr]|{1_{U_B}} &
        \ar[r]|-@{|}^-{\ghat} \ar@{=}[d] \ar@{}[dr]|{1_{\fhat}} &
        \ar@{=}[d]\\
        \ar[r]|-{\fhat'} \ar@{=}[d] \ar@{}[dr]|{1_{\ghat}} &
        \ar[r]|-{U_B} \ar@{=}[d] \ar@{}[dr]|{1_{U_B}} &
        \ar[r]|-{U_B} \ar@{=}[d] \ar@{}[dr]|{\Downarrow \eta_{\hat{g}'}} &
        \ar[r]|-{\ghat} \ar[d]|g \ar@{}[dr]|{\Downarrow \epsilon_{\hat{g}}} &
        \ar@{=}[d]\\
        \ar[r]|-@{|}_-{\fhat'} &
        \ar[r]|-@{|}_-{U_B} &
        \ar[r]|-@{|}_-{\ghat'} &
        \ar[r]|-@{|}_-{U_C} &
      }}\;=\;
    \vcenter{\xymatrix@-.5pc{
        \ar[r]|-@{|}^-{U_A} \ar@{=}[d] \ar@{}[dr]|{\Downarrow \eta_{\hat{f}'}}&
        \ar[r]|-@{|}^-{\fhat} \ar[d]|f \ar@{}[dr]|{\Downarrow \epsilon_{\hat{f}}}&
        \ar[r]|-@{|}^-{U_B} \ar@{=}[d] \ar@{}[dr]|{\Downarrow \eta_{\hat{g}'}}&
        \ar[r]|-@{|}^-{\ghat} \ar[d]|g \ar@{}[dr]|{\Downarrow \epsilon_{\hat{g}}}& \ar@{=}[d]\\
        \ar[r]|-@{|}_-{\fhat'} &
        \ar[r]|-@{|}_-{U_B} &
        \ar[r]|-@{|}_-{\ghat'} &
        \ar[r]|-@{|}_-{U_C} &
      }}\\
    &=\;
    \theta_{\ghat,\ghat'}\odot \theta_{\fhat,\fhat'} 
  \end{align}
  as desired.
\end{proof}

\begin{lem}\label{thm:theta-unit}
  If $f\maps A\to B$ has a companion \fhat, then
  $\theta_{\fhat,\fhat\odot U_A}$ and $\theta_{\fhat,U_B\odot \fhat}$
  are equal to the unit constraints $\fhat \iso \fhat\odot U_A$ and
  $\fhat\iso U_B\odot \fhat$.
\end{lem}
\begin{proof}
  By definition, we have
  \[\theta_{\fhat,\fhat\odot U_A} =\;
  \vcenter{\xymatrix@-.5pc{
      \ar[r]|-@{|}^-{U_A} \ar@{=}[d] \ar@{}[dr]|{\Downarrow 1_{U_A}} &
      \ar[r]|-@{|}^-{U_A} \ar@{=}[d] \ar@{}[dr]|{1_{U_A}} &
      \ar@{=}[d] \ar[rr]|-@{|}^-{\fhat} \ar@{}[ddrr]|{\Downarrow \epsilon_{\hat{f}}}&& \ar@{=}[dd]\\
      \ar[r]|-{U_A} \ar@{=}[d] \ar@{}[dr]|{1_{U_A}} &
      \ar[r]|-{U_A} \ar@{=}[d] \ar@{}[dr]|{\Downarrow \eta_{\hat{f}}}&
      \ar[d]^f\\
      \ar[r]|-@{|}_-{U_A} &
      \ar[r]|-@{|}_-{\fhat} & \ar[rr]|-@{|}^-{U_B} &&
    }}\;=\;
  \vcenter{\xymatrix{ \ar[r]|-@{|}^-{U_A} \ar@{=}[d]
      \ar@{}[dr]|{\Downarrow 1_{U_A}} &  \ar@{=}[d]\\
      \ar[r]|-@{|}_-{U_A} & }}
  \]
  which, bearing in mind our suppression of unit and associativity
  constraints, means that in actuality it is the unit constraint
  $\fhat \iso \fhat\odot U_A$.  The other case is dual.
\end{proof}

\begin{lem}\label{thm:comp-func}
  Let $F\maps \lD\to\lE$ be a functor between double categories and
  let $f\maps A\to B$ have a companion \fhat\ in \lD.  Then $F(\fhat)$
  is a companion of $F(f)$ in \lE.
\end{lem}
\begin{proof}
  We take the defining 2-morphisms to be
  \[\vcenter{\xymatrix@R=1.5pc@C=3pc{
      \ar[r]|-@{|}^-{F(\fhat)} \ar[d]_{F(f)}
      \ar@{}[dr]|{F(\Downarrow \epsilon_{\hat{f}})} &  \ar@{=}[d]\\
      \ar[r]|-{F(U_B)} \ar@{=}[d] \ar@{}[dr]|\iso &  \ar@{=}[d]\\
      \ar[r]|-@{|}_-{U_{F(B)}} & }}
  \quad\text{and}\quad
  \vcenter{\xymatrix@R=1.5pc@C=3pc{
      \ar[r]|-@{|}^-{U_{FA}} \ar@{=}[d] \ar@{}[dr]|\iso & \ar@{=}[d]\\
      \ar[r]|-{F(U_{A})} \ar@{=}[d] \ar@{}[dr]|{F(\Downarrow \eta_{\hat{f}})} & 
      \ar[d]^{F(f)}\\
      \ar[r]|-@{|}_-{F(\fhat)} & .}}\]
  The axioms~\eqref{eq:compeqn} follow directly from those for \fhat.
\end{proof}

% \begin{lem}\label{thm:comp-ten}
%   Suppose that \lD\ is a monoidal double category and that $f\maps
%   A\to B$ and $g\maps C\to D$ have companions \fhat\ and \ghat\
%   respectively.  Then $\fhat\ten\ghat$ is a companion of $f\ten g$.
% \end{lem}
% \begin{proof}
%   This follows from \autoref{thm:comp-func}, since $\ten\maps
%   \lD\times\lD\to\lD$ is a functor, and a companion in $\lD\times\lD$
%   is simply a pair of companions in \lD.
% %   We take the defining 2-morphisms to be
% %   \[\vcenter{\xymatrix@R=1.5pc@C=3pc{
% %       \ar[r]|-@{|}^-{\fhat\ten\ghat} \ar[d]_{f\ten g}
% %       \ar@{}[dr]|{\Downarrow\ten\Downarrow} &  \ar@{=}[d]\\
% %       \ar[r]|-{U_B\ten U_D} \ar@{=}[d] \ar@{}[dr]|\iso &  \ar@{=}[d]\\
% %       \ar[r]|-@{|}_-{U_{B\ten D}} & }}
% %   \quad\text{and}\quad
% %   \vcenter{\xymatrix@R=1.5pc@C=3pc{
% %       \ar[r]|-@{|}^-{U_{A\ten C}} \ar@{=}[d] \ar@{}[dr]|\iso & \ar@{=}[d]\\
% %       \ar[r]|-{U_{A}\ten U_C} \ar@{=}[d] \ar@{}[dr]|{\Downarrow\ten\Downarrow} & 
% %       \ar[d]^{f\ten g}\\
% %       \ar[r]|-@{|}_-{\fhat\ten\ghat} & .}}\]
% \end{proof}

\begin{lem}\label{thm:theta-func}
  Suppose that $F\maps \lD\to\lE$ is a functor and that $f\maps A\to
  B$ has companions \fhat\ and $\fhat'$ in \lD.  Then
  $\theta_{F(\fhat),F(\fhat')} = F(\theta_{\fhat,\fhat'})$.
\end{lem}
\begin{proof}
  Using the axioms of a pseudo double functor and the definition of
  the 2-morphisms in \autoref{thm:comp-func}, we have
  \begin{equation}
    F(\theta_{\fhat,\fhat'})
    =\;
    \vcenter{\xymatrix@C=3pc{
        \ar[r]|-@{|}^-{F(\fhat)}
        \ar[d] \ar@{}[dr]|{\Downarrow F(\eta_{\hat{f}'} \odot\epsilon_{\hat{f}})} &  \ar[d]\\
        \ar[r]|-@{|}_-{F(\fhat')} &}}
    \;=\;
    \vcenter{\xymatrix@C=3pc{
        \ar[rr]|-@{|}^-{F(\fhat)}
        \ar@{=}[d] \ar@{}[drr]|\iso &&  \ar@{=}[d]\\
        \ar[r]|-@{|}^-{F(U_{A})} \ar@{=}[d]
        \ar@{}[dr]|{\Downarrow F(\eta_{\hat{f}'})} &
        \ar[r]|-@{|}^-{F(\fhat)} \ar[d]|{F(f)}
        \ar@{}[dr]|{\Downarrow F(\epsilon_{\hat{f}})}
        & \ar@{=}[d]\\
        \ar[r]|-@{|}_-{F(\fhat')} \ar@{}[drr]|\iso\ar@{=}[d] &
        \ar[r]|-@{|}_-{U_{F(B)}} & \ar@{=}[d]\\
        \ar[rr]|-@{|}_-{F(\fhat')} && }}
    \;=\;
    \vcenter{\xymatrix@R=1.5pc@C=3pc{
        \ar[r]|-@{|}^-{U_{F(A)}} \ar@{=}[d] \ar@{}[dr]|\iso &
        \ar[r]|-@{|}^-{F(\fhat)} \ar@{=}[d] \ar@{}[dr]|=
        & \ar@{=}[d]\\
        \ar[r]|-{F(U_{A})} \ar@{=}[d] \ar@{}[dr]|{\Downarrow F(\eta_{\hat{f}'})} &
        \ar[r]|-{F(\fhat)} \ar[d]|{F(f)} \ar@{}[dr]|{\Downarrow F(\epsilon_{\hat{f}})}
        & \ar@{=}[d]\\
        \ar[r]|-{F(\fhat')}  \ar@{=}[d] \ar@{}[dr]|= &
        \ar[r]|-{F(U_{B})} \ar@{}[dr]|\iso  \ar@{=}[d] & \ar@{=}[d]\\
        \ar[r]|-@{|}_-{F(\fhat')} &
        \ar[r]|-@{|}_-{U_{F(B)}} &}}
    \;=
    \theta_{F(\fhat),\,F(\fhat')}
  \end{equation}
  as desired.
\end{proof}

% \begin{lem}\label{thm:theta-ten}
%   Suppose that \lD\ is a monoidal double category, that $f\maps A\to
%   B$ has companions \fhat\ and $\fhat'$, and that $g\maps C\to D$ has
%   companions \ghat\ and $\ghat'$.  Then $\theta_{\fhat,\fhat'} \ten
%   \theta_{\ghat,\ghat'} = \theta_{\fhat\ten \ghat, \fhat'\ten\ghat'}.$
% \end{lem}
% \begin{proof}
%   This follows from \autoref{thm:theta-func} in the same way that
%   \autoref{thm:comp-ten} follows from \autoref{thm:comp-func}.
% %   Using the naturality and functoriality axioms spelled out in
% %   \S\ref{sec:symm-mono-double}, we have
% %   \begin{equation}
% %     \theta_{\fhat,\fhat'} \ten \theta_{\ghat,\ghat'}
% %     =\;
% %     \vcenter{\xymatrix@C=4.5pc{
% %         \ar[r]|-@{|}^-{\fhat\ten\ghat} \ar@{=}[d] \ar@{}[dr]|\iso &  \ar@{=}[d]\\
% %         \ar[r]|-@{|}^-{(U_A\odot \fhat)\ten(U_C\odot \ghat)}
% %         \ar[d] \ar@{}[dr]|{(\Downarrow\odot\Downarrow)\ten(\Downarrow\odot\Downarrow)} &  \ar[d]\\
% %         \ar[r]|-@{|}_-{(\fhat'\odot U_C) \ten (\ghat'\odot U_D)} \ar@{=}[d] \ar@{}[dr]|\iso &  \ar@{=}[d]\\
% %         \ar[r]|-@{|}_-{\fhat'\ten\ghat'} & }}
% %     \;=\;
% %     \vcenter{\xymatrix@C=2pc{
% %         \ar[rr]|-@{|}^-{\fhat\ten\ghat} \ar@{=}[d] \ar@{}[drr]|\iso &&  \ar@{=}[d]\\
% %         \ar[rr]|-@{|}^-{(U_A\odot \fhat)\ten(U_C\odot \ghat)}
% %         \ar@{=}[d] \ar@{}[drr]|\iso &&  \ar@{=}[d]\\
% %         \ar[r]|-@{|}^-{U_{A}\ten U_C} \ar@{=}[d]
% %         \ar@{}[dr]|{\Downarrow\ten \Downarrow} &
% %         \ar[r]|-@{|}^-{\fhat\ten\ghat} \ar[d]|{f\ten g}
% %         \ar@{}[dr]|{\Downarrow\ten\Downarrow}
% %         & \ar@{=}[d]\\
% %         \ar[r]|-@{|}_-{\fhat'\ten\ghat'} \ar@{}[drr]|\iso\ar@{=}[d] &
% %         \ar[r]|-@{|}_-{U_{B\ten D}} & \ar@{=}[d]\\
% %         \ar[rr]|-@{|}_-{(\fhat'\odot U_C) \ten (\ghat'\odot U_D)}
% %         \ar@{=}[d] \ar@{}[drr]|\iso &&  \ar@{=}[d]\\
% %         \ar[rr]|-@{|}_-{\fhat'\ten\ghat'} && }}
% %     \;=\;
% %     \vcenter{\xymatrix@R=1.5pc@C=2.5pc{
% %         \ar[rr]|-@{|}^-{\fhat\ten\ghat} \ar@{}[drr]|\iso \ar@{=}[d] &&
% %         \ar@{=}[d] \\
% %         \ar[r]|-@{|}^-{U_{A\ten C}} \ar@{=}[d] \ar@{}[dr]|\iso &
% %         \ar[r]|-@{|}^-{\fhat\ten\ghat} \ar@{=}[d] \ar@{}[dr]|=
% %         & \ar@{=}[d]\\
% %         \ar[r]|-@{|}^-{U_{A}\ten U_C} \ar@{=}[d] \ar@{}[dr]|{\Downarrow\ten\Downarrow} &
% %         \ar[r]|-@{|}^-{\fhat\ten\ghat} \ar[d]|{f\ten g} \ar@{}[dr]|{\Downarrow\ten\Downarrow}
% %         & \ar@{=}[d]\\
% %         \ar[r]|-@{|}_-{\fhat'\ten\ghat'}  \ar@{=}[d] \ar@{}[dr]|= &
% %         \ar[r]|-@{|}_-{U_{B}\ten U_D} \ar@{}[dr]|\iso  \ar@{=}[d] & \ar@{=}[d]\\
% %         \ar[r]|-@{|}_-{\fhat'\ten\ghat'} &
% %         \ar[r]|-@{|}_-{U_{B\ten D}} &\\
% %         \ar[rr]|-@{|}_-{\fhat'\ten\ghat'} \ar@{}[urr]|\iso \ar@{=}[u] &&
% %         \ar@{=}[u]}}
% %     \;=
% %     \theta_{\fhat\ten\ghat}
% %   \end{equation}
% %   as desired.
% \end{proof}


\begin{lem}\label{thm:comp-iso}
  If $f\maps A\to B$ is a tight isomorphism with a companion \fhat,
  then \fhat\ is a conjoint of its inverse $f\inv$.
\end{lem}
\begin{proof}
  The composites
  \[\vcenter{\xymatrix@-.5pc{
      \ar[r]|-@{|}^{\fhat}\ar[d]_f \ar@{}[dr]|{\Downarrow} &
      \ar@{=}[d]\\
      \ar[r]|{U_B}\ar[d]_{f\inv} \ar@{}[dr]|{\Downarrow U_{f\inv}} &
      \ar[d]^{f\inv}\\
      \ar[r]|-@{|}_{U_A} &
    }}\quad\text{and}\quad
  \vcenter{\xymatrix@-.5pc{
      \ar[r]|-@{|}^{U_B}\ar[d]_{f\inv} \ar@{}[dr]|{\Downarrow U_{f\inv}} &
      \ar[d]^{f\inv}\\
      \ar[r]|{U_A}\ar@{=}[d] \ar@{}[dr]|{\Downarrow} &
      \ar[d]^f\\
      \ar[r]|-@{|}_{\fhat} &
    }}
  \]
  exhibit \fhat\ as a conjoint of $f\inv$.
\end{proof}

\begin{lem}\label{thm:compconj-adj}
  If $f\maps A\to B$ has both a companion \fhat\ and a conjoint \fchk,
  then we have an adjunction $\fhat\adj\fchk$ in $\cH\lD$.  If $f$ is
  an isomorphism, then this is an adjoint equivalence.
\end{lem}
\begin{proof}
  The unit and counit of the adjunction $\fhat\adj\fchk$ are the
  composites
  \[\vcenter{\xymatrix@-.5pc{
      \ar[r]|-@{|}^{U_A}\ar@{=}[d] \ar@{}[dr]|{\Downarrow \eta_{\hat{f}}} &
      \ar[r]|-@{|}^{U_A}\ar[d]|{f} \ar@{}[dr]|{\Downarrow \eta_{\check{f}}} &
      \ar@{=}[d]\\
      \ar[r]|-@{|}_{\fhat} &
      \ar[r]|-@{|}_{\fchk} &
    }}\quad\text{and}\quad
  \vcenter{\xymatrix@-.5pc{
      \ar[r]|-@{|}^{\fchk}\ar@{=}[d] \ar@{}[dr]|{\Downarrow \epsilon_{\check{f}}} &
      \ar[r]|-@{|}^{\fhat}\ar[d]|{f} \ar@{}[dr]|{\Downarrow \epsilon_{\hat{f}}} &
      \ar@{=}[d]\\
      \ar[r]|-@{|}_{U_B} &
      \ar[r]|-@{|}_{U_B} &
    }}
  \]
%   \[\vcenter{\xymatrix@-.5pc{
%       \ar[rr]|-@{|}^{U_A}\ar@{=}[d] \ar@{}[drr]|{\iso} &&
%       \ar@{=}[d]\\
%       \ar[r]|-@{|}^{U_A}\ar@{=}[d] \ar@{}[dr]|{\Downarrow} &
%       \ar[r]|-@{|}^{U_A}\ar[d]|{f} \ar@{}[dr]|{\Downarrow} &
%       \ar@{=}[d]\\
%       \ar[r]|-@{|}_{\fhat} &
%       \ar[r]|-@{|}_{\fchk} &
%     }}\quad\text{and}\quad
%   \vcenter{\xymatrix@-.5pc{
%       \ar[r]|-@{|}^{\fchk}\ar@{=}[d] \ar@{}[dr]|{\Downarrow} &
%       \ar[r]|-@{|}^{\fhat}\ar[d]|{f} \ar@{}[dr]|{\Downarrow} &
%       \ar@{=}[d]\\
%       \ar[r]|-@{|}_{U_B}\ar@{=}[d] \ar@{}[drr]|{\iso} &
%       \ar[r]|-@{|}_{U_B} &
%       \ar@{=}[d]\\
%       \ar[rr]|-@{|}_{U_B} &&.
%     }}
%   \]
  The triangle identities follow from~\eqref{eq:compeqn}.  If $f$ is
  an isomorphism, then by the dual of \autoref{thm:comp-iso}, \fchk\
  is a companion of $f\inv$.  But then by \autoref{thm:comp-compose}
  $\fchk\odot \fhat$ is a companion of $1_A=f\inv \circ f$ and
  $\fhat\odot\fchk$ is a companion of $1_B = f\circ f\inv$, and hence
  \fhat\ and \fchk\ are equivalences.  We can then check that in this
  case the above unit and counit actually are the isomorphisms
  $\theta$, or appeal to the general fact that any adjunction
  involving an equivalence is an adjoint equivalence.
\end{proof}

% To conclude this section, we combine some of the Lemmas above to derive a more general statement which will play a central role in section~\ref{sec:constr-symm-mono}.
% \begin{lem}\label{lem:equal}  % FALSE as stated
% Any two composites of $\theta$-isomorphisms that have the same source and target loose 1-cells are equal.
% \end{lem}
% \begin{proof}
% By Lemmas~\ref{thm:theta-compose-vert} and~\ref{thm:theta-compose-horiz}, $\theta$-isomorphisms are closed under composition. By uniqueness of $\theta$-isomorphisms, any two compositions of $\theta$-isomorphisms that have the same source and target loose 1-cells must be equal.
% \end{proof}

\begin{lem}\label{lem:FUtheta}
Suppose $F:\lD \rightarrow \lE$ is a functor of double categories. The 2-cell $F_U$ is equal to $\theta_{\id_{FA}, F\id_A}$.
\end{lem}

\begin{proof}
We show that equation~\eqref{eq:comp-iso} holds when we substitute $\theta_{\id_{FA}, F\id_A}$ by $\hat{F}_U$.  Unfolding the definitions of $\eta_{U_{FA}}$, $\epsilon_{FU_A}$, and $\hat{F}_U$, and applying functoriality of $F$, we obtain an expression that can be rewritten to $U_{\id_{FA}}$. It follows that $F_U$ is a $\theta$-isomorphism, by the uniqueness of $\theta s$ in this expression.
\end{proof}

\begin{rmk}
  It is tempting to want to state a general coherence theorem along the lines of ``any two composites of $\theta$-isomorphisms having the same source and target are equal.''
  However, like statements such as ``any two composites of constraints in a monoidal category are equal'', this statement is actually literally false, because to determine a $\theta$-isomorphism requires not only a source and target but also the choice of companion data.
  If a given 1-cell is a companion of the same 1-cell in more than one way (which is the case as soon as it has any nontrivial automorphisms), then there will be different $\theta$-isomorphisms with the same source and target.
  This is analogous to how in a particular monoidal category there can be ``accidental'' composites of constraints that are not covered by the coherence theorem.
  It is probably possible to state a general coherence theorem for $\theta$-isomorphisms that is sufficiently careful to be true, but we will not need this.
\end{rmk}


\begin{rmk}
  Since all the tight constraints of a monoidal double category are invertible, to construct its underlying monoidal bicategory we only need it to have companions (and hence, by \cref{thm:comp-iso}, conjoints) for all tight \emph{isomorphisms}.
  In~\cite{gg:ldstr-tricat} double categories of this sort were called \textbf{fibrant}; one might also say \textbf{isofibrant} for emphasis.
  Note that this condition is equivalent to asking that the (source, target) functor $\lD_1 \to \lD_0\times\lD_0$ is an isofibration (i.e.\ has the isomorphism-lifting property).
  
  To lift lax or colax monoidal \emph{functors}, and noninvertible transformations between monoidal functors, to the bicategorical level, we require our double categories to have companions (or conjoints, depending on the directions) for noninvertible tight morphisms as well.
  In~\cite{shulman:frbi} double categories with companions and conjoints for \emph{all} tight morphisms were called \emph{framed bicategories}; this is equivalent to asking that the (source, target) functor $\lD_1 \to \lD_0\times\lD_0$ be a Grothendieck fibration or opfibration (either assumption implies the other).
  In \cref{sec:1x1-to-bicat} we will see that a further condition is also required to ensure that the ``componentwise'' companion of a tight transformation is pseudo, rather than lax or colax, natural.
\end{rmk}


% Local Variables:
% TeX-master: "smbicat"
% End:


\section{From double categories to bicategories}
\label{sec:1x1-to-bicat}

We are now equipped to lift structures on double categories to
their loose bicategories.  In this section we show that passage
from double categories to bicategories is given by a functor of locally cubical bicategories. In order to prove this, we first give an intermediate result that $\cH$ lifts to a functor of hom-bicategories
\begin{align}
    \cDbl(\lD,\lE) &\too \cBicat(\cH(\lD),\cH(\lE))
\end{align}

As a point of notation, we write $\odot$ for the composition of
1-cells in a bicategory, since our bicategories are generally of the
form $\cH(\lD)$.  As advocated by Max Kelly, we say \textbf{functor}
to mean a morphism between bicategories that preserves composition up
to coherent isomorphism; equivalent terms include \emph{weak 2-functor},
\emph{pseudofunctor}, and \emph{homomorphism}.
We will not discuss lax functors (a.k.a. ``morphisms'') between bicategories at all in this paper.

Recall that the assignment $\cH$ sends each double category $\lC$ to the loose bicategory  $\cH(\lD)$ of objects, 1-cells, and globular 2-morphisms of $\lD$.  Note that functors of double categories and bicategories compose strictly associatively; hence, we can talk about the 1-categories of double categories and bicategories, which we denote ${\bf Dbl}$ and ${\bf Bicat}$ respectively.

\begin{thm}\label{thm:1-func}
 If \lD\ is a double category, then $\cH(\lD)$ is a bicategory, and
  any functor $F\maps \lD\to\lE$ induces a functor $\cH(F)\maps
  \cH(\lD)\to\cH(\lE)$.  In this way $\cH$ defines a functor of
  1-categories $\mathbf{Dbl}\to \mathbf{Bicat}$.
\end{thm}
\begin{proof}
 The constraints of $F$ are all globular, hence give constraints for
  $\cH(F)$.  Functoriality is evident.
\end{proof}

% MS: This doesn't really make precise sense, and I'm not sure it's necessary.
%Note that this is a stronger condition than we need for our main result.
The action of \cH\ on transformations is less obvious. It
requires the presence of companions or conjoints to lift the part of the data given by vertical morphisms to loose 1-cells. Before we discuss how this works, we briefly recall some definitions regarding transformations between functors of bicategories.

If $F,G\maps \cA\to\cB$ are functors between bicategories, then an
\textbf{oplax transformation} $\al\maps F\to G$ consists of 1-cells
$\al_A\maps FA\to GA$ and 2-cells
\[\vcenter{\xymatrix{ \ar[r]^{Ff}\ar[d]_{\al_A} \drtwocell\omit{\al_f} &  \ar[d]^{\al_B}\\
  \ar[r]_{Gf} & }}\]
such that for any 2-cell $\xymatrix{A \rtwocell^f_g{x} & B}$ in \cA,
\begin{equation}
  \label{eq:laxtransf-nat}
  \vcenter{\xymatrix@R=1pc@C=3pc{
      \rtwocell^{Ff}_{Fg}{Fx}\ar[dd]_{\al_A} 
      &  \ar[dd]^{\al_B}\\
      \drtwocell\omit{\al_g} & \\
      \ar[r]_{Gg} & }}\;=\;
  \vcenter{\xymatrix@R=1pc@C=3pc{
      \ar[r]^{Ff}\ar[dd]_{\al_A} \drtwocell\omit{\al_f} &
      \ar[dd]^{\al_B}\\ & \\
      \rtwocell^{Gf}_{Gg}{Gx} & }}
\end{equation}
and moreover for any $A$ and any $f,g$ in \cA,
\begin{equation}
  \vcenter{\xymatrix@R=5pc{
      \rtwocell^{1_{FA}}_{F(1_A)}{\iso} \ar[d]_{\al_A} \drtwocell\omit{\al_{1_A}} &  \ar[d]^{\al_A}\\
      \rtwocell^{G(1_A)}_{1_{GA}}{\iso} & }} \;=\;
  \vcenter{\xymatrix{ \ar[r]^{1_{FA}}\ar[d]_{\al_A} \drtwocell\omit{\iso}&  \ar[d]^{\al_A}\\
      \ar[r]_{1_{GA}} &
    }}
  \quad\text{and}\quad
  \vcenter{\xymatrix{
      \ar[r]|{Ff}\ar[d]_{\al_A} \drtwocell\omit{\al_f}
      \rruppertwocell^{F(gf)}{\iso}
      &
      \ar[r]|{Fg}\ar[d]|{\al_B} \drtwocell\omit{\al_g} &
      \ar[d]^{\al_C}\\
      \ar[r]|{Gf} \rrlowertwocell_{G(gf)}{\iso} & \ar[r]|{Gg} & }}
  \;=\;
  \vcenter{\xymatrix{ \ar[r]^{F(gf)}\ar[d]_{\al_A} \drtwocell\omit{\al_{gf}} &  \ar[d]^{\al_C}\\
      \ar[r]_{G(gf)} & }}\label{eq:laxtransf-ax}
\end{equation}
It is a \textbf{lax transformation} if the 2-cells $\al_f$ go the
other direction, and a \textbf{pseudo transformation} if they are
isomorphisms.

When two functors of bicategories agree on objects, there is a simpler notion of transformation between them, called an \emph{icon}.
An icon is, morally speaking, an oplax transformation whose 1-cell components are all identities; but as noted by~\cite{lack:icons} this can be reexpressed without referring to these identity morphisms at all, yielding a definition that is easier to work with (because identity 1-cells in a bicategory are not strict).

\begin{defn}
Let $\cD, \cE$ be bicategories, and let $F,G: \cD \rightarrow \cE$ be functors that agree on objects. An \textbf{icon} $\alpha: F \Rightarrow G$ is given by a family of 2-cells $\alpha_f : Ff \Rightarrow Gf$ indexed by the 1-cells of $\cD$, which are natural in $f$ and such that for all objects $A, B, C$ and 1-cells $A \xrightarrow{f} B \xrightarrow{g} C$ the following equations hold:
%\[
%\begin{tikzpicture}[yscale=1,xscale=1.5]
%\node (tl) at (0,1) {$I_{FA}$};
%\node (tr) at (1,1) {$F I_A$};
%\node (bl) at (0,0) {$I_{GA}$};
%\node (br) at (1,0) {$GI_A$};
%\draw[->] (tl) to node[above]{$\phi$} (tr);
%\draw[->] (bl) to node[below]{$\phi$} (br);
%\draw[doubleeq] (tl) to (bl);
%\draw[->] (tr) to node[right]{$\alpha_{I_A}$} (br);
%\end{tikzpicture}
%\qquad
%\begin{tikzpicture}[yscale=1, xscale=1.5]
%\node (tl) at (0,1) {$F(g)  F(f)$};
%\node (tr) at (1,1) {$F(g  f)$};
%\node (bl) at (0,0) {$G(g)  G(f)$};
%\node (br) at (1,0) {$G(g  f)$};
%\draw[->] (tl) to node[above]{$\phi$} (tr);
%\draw[->] (bl) to node[below]{$\phi$} (br);
%\draw[->] (tl) to node[left]{$\alpha_g \tens \alpha_f$}(bl);
%\draw[->] (tr) to node[right]{$\alpha_{g \tens f}$} (br);
%\end{tikzpicture}
%\]

\begin{equation}\label{eq:iconeq}
\begin{aligned}
\begin{tikzpicture}[yscale=2,xscale=3]
\node (tl) at (0,1) {$I_{FA}$};
\node (tr) at (1,1) {$F I_A$};
\node (bl) at (0,0) {$I_{GA}$};
\node (br) at (1,0) {$GI_A$};
\draw[->] (tl) to node[above]{$\iso$} (tr);
\draw[->] (bl) to node[below]{$\iso$} (br);
\draw[->] (tl) to node[left]{$\iso$} (bl);
\draw[->] (tr) to node[right]{$\alpha_{I_A}$} (br);
\end{tikzpicture}
\end{aligned}
\qquad
\begin{aligned}\begin{tikzpicture}[yscale=2, xscale=3]
\node (tl) at (0,1) {$F(g) \odot F(f)$};
\node (tr) at (1,1) {$F(g \odot f)$};
\node (bl) at (0,0) {$G(g) \odot G(f)$};
\node (br) at (1,0) {$G(g \odot f)$};
\draw[->] (tl) to node[above]{$\iso$} (tr);
\draw[->] (bl) to node[below]{$\iso$} (br);
\draw[->] (tl) to node[left]{$\alpha_g \odot \alpha_f$}(bl);
\draw[->] (tr) to node[right]{$\alpha_{g \odot f}$} (br);
\end{tikzpicture}
\end{aligned}
\end{equation}
\end{defn}


Recall also that if $\al,\al'\maps F\to G$ are oplax transformations,
a \textbf{modification} $\mu\maps \al\to\al'$ consists of 2-cells
$\mu_A\maps \al_A\to\al'_A$ such that
\begin{equation}
  \vcenter{\xymatrix@C=1pc@R=2.5pc{ \ar[rr]^{Ff}\dtwocell_{\al'_A}^{\al_A}{\mu_A}  &
      \drtwocell\omit{\al_f} &  \ar[d]^{\al_B}\\
      \ar[rr]_{Gf} && }} \quad=\quad
  \vcenter{\xymatrix@C=1pc@R=2.5pc{ \ar[rr]^{Ff}\ar[d]_{\al'_A} \drtwocell\omit{\al'_f} && 
      \dtwocell^{\al_B}_{\al'_B}{\mu_B}\\
      \ar[rr]_{Gf} && }}\label{eq:modif-ax}
\end{equation}
There is an evident notion of modification between lax transformations
as well.
We have three bicategories
\[ \cBicat_c(\cA,\cB) \qquad \cBicat_l(\cA,\cB) \qquad \cBicat_p(\cA,\cB) \]
whose objects are the functors $\cA\to\cB$, whose morphisms are colax, lax, and pseudo transformations respectively, and whose 2-morphisms are modifications.

A \textbf{pseudo natural adjoint equivalence} is, by definition, an internal adjoint equivalence in $\cBicat_p(\cA,\cB)$.
However, by doctrinal adjunction~\cite{kelly:doc-adjn}, an internal adjoint equivalence in $\cBicat_c(\cA,\cB)$ or $\cBicat_l(\cA,\cB)$ is automatically pseudo natural as well.

Let $\cDblcf$ denote the sub-2-category of $\cDbl$ containing all double categories and all functors between them, but only the tight transformations $\al:F\to G: \lD\to\lE$ such that each tight component $\al_A$ has a loose companion $\widehat{\al_A}$.
Note that if \lE\ is isofibrant, every invertible $\al$ has this property, and if \lE\ has companions for all tight 1-morphisms then every transformation has this property.

\begin{thm}\label{thm:h-locfr}
  We have a functor of bicategories
  \begin{align}
    \cDblcf(\lD,\lE) &\too \cBicat_c(\cH(\lD),\cH(\lE))\\
    F &\mapsto \cH(F)\\
    \al &\mapsto \alhat.
  \end{align}
\end{thm}

Note that we are here regarding the 1-category $\cDblcf(\lD,\lE)$ as a bicategory with only identity 2-cells.
Since any functor of bicategories preserves adjoint equivalences, and an adjoint equivalence in a mere category is simply an isomorphism, it follows that if $\al$ is an isomorphism then $\alhat$ is (equipped with the structure of) a pseudo natural adjoint equivalence.

\begin{proof}
  We define the 1-cell components of $\alhat$ by choosing companions $\alhat_A = \widehat{\al_A}$ of each component of $\al$.
  The 2-cell component $\alhat_f$ is the composite
  \begin{equation}
    \vcenter{\xymatrix@R=1.5pc@C=2.5pc{
        \ar[r]|-@{|}^{U_{FA}}\ar@{=}[d] \ar@{}[dr]|{\Downarrow \eta_{\hat{\alpha}_A}} &
        \ar[r]^{Ff}\ar[d]|{\al_A} \ar@{}[dr]|{\Downarrow \al_f} &
        \ar[r]|-@{|}^{\alhat_B}\ar[d]|{\al_B} \ar@{}[dr]|{\Downarrow \epsilon_{\hat{\alpha}_B}} &
        \ar@{=}[d]\\
        \ar[r]|-@{|}_{\alhat_A} &
        \ar[r]_{Gf} &
        \ar[r]|-@{|}_{U_{GB}} & 
      }}\label{eq:oplax-2cell}
%     \vcenter{\xymatrix@R=1.5pc@C=2.5pc{
%         \ar[r]|-@{|}^{Ff}\ar@{=}[d] \ar@{}[drrr]|{\iso} &
%         \ar[rr]|-@{|}^{\alhat_B} &&
%         \ar@{=}[d]\\
%         \ar[r]|-@{|}^{U_{FA}}\ar@{=}[d] \ar@{}[dr]|{\Downarrow} &
%         \ar[r]|{Ff}\ar[d]|{\al_A} \ar@{}[dr]|{\Downarrow \al_f} &
%         \ar[r]|-@{|}^{\alhat_B}\ar[d]|{\al_B} \ar@{}[dr]|{\Downarrow} &
%         \ar@{=}[d]\\
%         \ar[r]|-@{|}_{\alhat_A} \ar@{=}[d] \ar@{}[drrr]|\iso &
%         \ar[r]|{Gf} &
%         \ar[r]|-@{|}_{U_{GB}} & \ar@{=}[d]\\
%         \ar[r]|-@{|}_{\alhat_A} & \ar[rr]|-@{|}_{Gf} &&
%       }}\label{eq:oplax-2cell}
  \end{equation}
  Equations~\eqref{eq:laxtransf-nat} and~\eqref{eq:laxtransf-ax}
  follow directly from \autoref{thm:dbl-transf}.

  It is left to construct the constraints and check the axioms for functors of bicategories. Suppose we are given $\al\maps F\to G$ and $\be\maps G\to H$.  Then by
  \autoref{thm:comp-compose}, $\behat_A\odot\alhat_A$ is a companion
  of $\be_A\circ \al_A$, so we have a canonical isomorphism given by the icon
  \[\theta_{\widehat{\be\al}_A, \,\behat_A\odot\alhat_A}\maps
  \widehat{\be\al}_A \too[\iso] \behat_A\odot\alhat_A.
  \]
  Of course, we also have $\theta_{\widehat{1_A},U_A}\maps
  \widehat{1_A} \too[\iso] U_A$ by \autoref{thm:comp-unit}.  These
  constraints are automatically natural, since $\cDbl(\lD,\lE)$ has no
  nonidentity 2-cells.  The axiom for the composition constraint says
  that two constructed isomorphisms of the form
  \[\widehat{\gm\be\al}_A \too[\iso] (\gmhat_A \odot \behat_A)\odot \alhat_A\]
  are equal.  However, both $\widehat{\gm\be\al}_A$ and $(\gmhat_A
  \odot \behat_A)\odot \alhat_A$ are companions of $\gm_A\be_A\al_A$,
  and both of these isomorphisms are constructed from composites of $\theta$s;
  hence they are both equal to
  \[\theta_{\widehat{\gm\be\al}_A,\, (\gmhat_A \odot \behat_A)\odot
    \alhat_A}\] and thus equal to each other.  The same argument
  applies to the axioms for the unit constraint; thus we have a functor of bicategories.
\end{proof}

Of course, the functor constructed in \cref{thm:h-locfr} depends on the choices of companions made in the proof.
However up to equivalence it does not depend on these choices:

\begin{lem}\label{thm:h-locfr-uniq}
  Suppose we make two different sets of choices of companions for each component of a tight transformation in $\cDblcf(\lD,\lE)$, yielding by the proof of \cref{thm:h-locfr} two different functors
  \[\cH,\cH'\maps \cDblcf(\lD,\lE)\too \cBicat_c(\cH(\lD),\cH(\lE)).\]
  Then the isomorphisms $\theta$ from \autoref{thm:theta} fit together
  into an invertible icon $\cH\cong \cH'$.
\end{lem}
\begin{proof}
  We must first show that for a given transformation $\al\maps F\to
  G\maps \lD\to\lE$ in \cDbl, the isomorphisms \th\ correspond to 2-cells $\alhat \iso \alhat'$ of $\cBicat_c(\cH(\lD),\cH(\lE))$; that is, they form invertible
  modifications.
  Substituting~\eqref{eq:oplax-2cell} and the definition of \th\
  into~\eqref{eq:modif-ax}, this becomes the assertion that
  \begin{equation}
    \vcenter{\xymatrix@R=1.5pc@C=2pc{
        &
        \ar[r]|-@{|}^{U_{FA}}\ar@{=}[d] \ar@{}[dr]|{\Downarrow \eta_{\hat{\alpha}_A}} &
        \ar[r]^{Ff}\ar[d]|{\al_A} \ar@{}[dr]|{\Downarrow \al_f} &
        \ar[r]|-@{|}^{\alhat_B}\ar[d]|{\al_B} \ar@{}[dr]|{\Downarrow \epsilon_{\hat{\alpha}_B}} &
        \ar@{=}[d]\\
        \ar[r]|-@{|}^{U_{FA}} \ar@{=}[d] \ar@{}[dr]|{\Downarrow \eta_{\hat{\alpha}_A'}} &
        \ar[r]|{\alhat_A} \ar[d]|{\al_A} \ar@{}[dr]|{\Downarrow \epsilon_{\hat{\alpha}_A}}&
        \ar[r]_{Gf}  \ar@{=}[d] &
        \ar[r]|-@{|}_{U_{GB}} & \\
        \ar[r]|-@{|}_{\alhat_A'} & \ar[r]|-@{|}_{U_{GB}}&&
      }} \;=\;
    \vcenter{\xymatrix@R=1.5pc@C=2pc{
        && \ar@{=}[d] \ar[r]|-@{|}^{U_{FA}} \ar@{}[dr]|{\Downarrow \eta_{\hat{\alpha}_B'}} &
        \ar[d]|{\al_B} \ar[r]|-@{|}^{\alhat_B} \ar@{}[dr]|{\Downarrow \epsilon_{\hat{\alpha}_B}}
        &
        \ar@{=}[d] &\\
        \ar[r]|-@{|}^{U_{FA}}\ar@{=}[d] \ar@{}[dr]|{\Downarrow \eta_{\hat{\alpha}_A'}} &
        \ar[r]^{Ff}\ar[d]|{\al_A} \ar@{}[dr]|{\Downarrow \al_f} &
        \ar[r]|{\alhat_B'}\ar[d]|{\al_B} \ar@{}[dr]|{\Downarrow \epsilon_{\hat{\alpha}_B'}} &
        \ar@{=}[d] \ar[r]|-@{|}_{U_{GB}}&\\
        \ar[r]|-@{|}_{\alhat_A'} &
        \ar[r]_{Gf} &
        \ar[r]|-@{|}_{U_{GB}} & .
      }}
  \end{equation}
  This follows from two applications of~\eqref{eq:compeqn}, one for
  $\alhat_A$ and one for $\alhat_B'$.
  Now we show that these form an invertible icon. The compatibility axiom with 2-cells is vacuous since $\cDblcf(\lD,\lE)$ has
  no nonidentity 2-cells, while the functoriality requirement follows from uniqueness of $\theta$s,
  since all the constraints involved are also instances of \th.
\end{proof}


\begin{rmk}
  There is a similar result for conjoints instead of companions, but hedged about with duality.
  Recall that a conjoint in $\lE$ is the same as a companion in the ``loose opposite'' $\lE\lop$, and more generally the 2-functor $(-)\lop:\cDbl \to \cDbl$ takes transformations with conjoints to transformations with companions.
  Thus if we denote by $\cDbllf(\lD,\lE)$ the category of functors and transformations having componentwise conjoints, then $(-)\lop$ defines a functor $\cDbllf(\lD,lE) \to \cDblcf(\lD\lop,\lE\lop)$.

  We also have $\cH(\lE\lop) = (\cH\lE)\op$, where $(-)\op$ denots the bicategory obtained by reversing 1-cells but not 2-cells.
  Since $(-)\op$ reverses the direction of transformations, and interchanges lax and colax transformations, we have a functor $\cBicat_c(\cA,\cB) \to \cBicat_l(\cA\op,\cB\op)\op$.
  Thus, composing all of this up, we obtain a functor
  \begin{multline*}
    \cDbllf(\lD,\lE) \to \cDblcf(\lD\lop,\lE\lop) \to \cBicat_c(\cH(\lD\lop),\cH(\lE\lop))\\ = \cBicat_c((\cH\lD)\op, (\cH\lE)\op) \to \cBicat_l(\cH\lD, \cH\lE)\op.
  \end{multline*}
  In other words, a transformation with componentwise conjoints induces a \emph{lax} transformation in the \emph{opposite} direction between loose bicategories.
  If a transformation has both componentwise companions and conjoints (such as if $\lE$ has all companions and conjoints), then the resulting colax and lax transformations are doctrinally adjoint; in~\cite{shulman:smbicat} we called such a pair a ``conjunctional transformation''.
\end{rmk}

So far we have been able to deal with arbitrary tight transformations and their resulting colax (or lax) transformations.
But when we come to talk about tricategories or locally cubical bicategories, we have to restrict to pseudo natural transformations on the side of bicategories, since there is no tricategory (or locally cubical bicategory) containing arbitrary colax transformations: the interchange law only holds laxly.
(One could write down a notion of ``colax tricategory'', but we will not attempt this.)
This restriction means we need a corresponding restriction on the double-categorical side.


\begin{defn}
  A transformation $\alpha$ in $\cDbl$ has \textbf{loosely strong companions} if each component $\alpha_A$ has a loose companion, and the resulting colax transformation $\hat\alpha$ is actually pseudo natural.
  We write $\cDblf$ for the 2-category of double categories, functors, and transformations with loosely strong companions.
\end{defn}

We are now ready to prove that $\cH$ lifts to a functor of locally cubical bicategories. The definition of a locally cubical bicategory can be found in~\cite{gg:ldstr-tricat}, where it is called a locally cubical bicategory. It can also be derived as a bicategory enriched in the monoidal 2-category $\cDbl$, in the following sense.

\begin{defn}\label{def:lcbc}
Let $\mathcal{V}$ be a monoidal 2-category. A $\mathcal{V}$-enriched bicategory $\fB$ consists of the following elements:
\begin{itemize}
\item A collection of objects $\fB=\{A,B,C,...\} $
\item For each two objects $A,B$ of $\fB$, an object $\fB(A,B)$ of $\mathcal{V}$. 
\item For each three objects $A, B, C$ of $\fB$, a 1-cell  $\comp: \fB(B,C) \otimes \fB(A,B) \rightarrow \fB(A,C)$ of $\mathcal{V}$. 
\item For each object $A$ of $\fB$, a 1-cell \newline $I_A: 1 \rightarrow \fB(A,A)$ of $\mathcal{V}$, where $1$ is the initial object of $\mathcal{V}$. 
\item For each four objects $A,B,C,D$ of $\fB$, a 2-cell of $\mathcal{V}$, depicted below. 
  \begin{center}
    %
\documentclass[12pt]{ociamthesis}
\usepackage{tikz}
\newcommand{\id}{\mathrm{id}}
\begin{document}

\begin{tikzpicture}[yscale=1.5, xscale=2.5]
\node (A) at (0,1){${\cat B}(C,D) \otimes {\cat B}(B, C) \otimes {\cat B}(A,B)$};
\node (C) at (2,1){${\cat B}(C,D) \otimes {\cat B}(A,C)$};
\node (D) at (0,0) {${\cat B}(B,D)  \otimes {\cat B}(A,B)$};
\node (E) at (2,0) {${\cat B}(A, D) $};
\node (B) at (1,.5) {$\Downarrow \alpha$};
\draw[->] (A) to node[above]{$\comp \otimes \id$} (C);
\draw[->] (A) to node[left]{$\id \otimes \comp$} (D);
\draw[->] (C) to node[right]{$\comp$} (E);
\draw[->] (D) to node[above]{$\comp$} (E);
\end{tikzpicture}
\end{document} 

    \end{center}
\item For each two objects $A,B$ of $\fB$, a natural isomorphisms 
  \begin{center}
    %
\documentclass[12pt]{ociamthesis}
\usepackage{tikz}
\newcommand{\id}{\mathrm{id}}
\begin{document}

\begin{equation*}
\begin{aligned}
\begin{tikzpicture}[yscale=2.5, xscale=3]
\node (A) at (0,1){${\cat B}(A,B) \otimes 1$};
\node (B) at (1,0) {${\cat B}(A,B)$};
\node (D) at (0,0) {${\cat B}(B,B) \otimes {\cat B}(A,B)$};
\node (C) at (.3,.3){$\Downarrow \lambda_{A}$};
\draw[->] (A) to node[above]{$\iso$} (B);
\draw[->] (D) to node[below]{$\comp$} (B);
\draw[->] (A) to node[left]{$  I_B \otimes \id$} (D);
\end{tikzpicture}
\hspace{1cm}
\begin{tikzpicture}[yscale=2.5, xscale=3]
\node (A) at (0,1){$1 \otimes {\cat B}(A,B)$};
\node (B) at (1,0) {${\cat B}(A,B)$};
\node (D) at (0,0) {${\cat B}(A,B) \otimes {\cat B}(A,A)$};
\node (C) at (.3,.3){$\Downarrow\rho$};
\draw[->] (A) to node[above]{$\iso$} (B);
\draw[->] (D) to node[below]{$\comp$} (B);
\draw[->] (A) to node[left]{$\id \otimes I_B$} (D);
\end{tikzpicture}
\end{aligned}
\end{equation*}
\end{document} 

    \end{center}
 \end{itemize}
This data needs to satisfy the usual axioms for bicategories~\cite{maclane}
\end{defn}

The 2-category $\cDbl$ has finite products, which gives the monoidal structure. 
%we can write down the definition of bicategory in terms of hom-categories, with for instance composition functors $\bT(A,B) \times \bT(B,C) \to \bT(A,C)$, and then replace all the hom-categories with hom-double-categories.
%(In other words, we pass through a general notion of bicategory enriched over a cartesian monoidal strict 2-category.)
When we apply the definition above to $\cDbl$, we see that a locally cubical bicategory has objects, hom-double categories {of 1-cells, tight 2-cells, loose 2-cells, and 3-cells which fit in a square of tight and loose 2-cells. We have double functors $\comp$, $I_A$ and tight isomorphisms $\alpha, \rho, \lambda$.

Note that a locally cubical bicategory with one object is precisely a monoidal double category in the sense of \cref{sec:symm-mono-double}, and thus the explicit description of the coherences in \cref{sec:symm-mono-double} can also be applied here.
We will primarily be concerned with the following two examples.

\begin{eg}
  Any category can be regarded as a double category in the \emph{loose} direction, with only identity tight morphisms and identity 2-cells.
  (Note that this is a \emph{strict} double category, not a pseudo one.)
  This operation preserves products, and thereby any strict 2-category can be regarded as a locally cubical bicategory.
  In particular, we will regard the 2-category $\cDblf$ as a locally cubical bicategory in this way; thus it has only identity tight 2-cells and identity 3-cells. We write $\fDblf$ for this locally cubical bicategory.
  % Let $\cDblf$ denote the sub-2-category of $\cDbl$ containing the double categories, all functors between them, and only transformations with loosely strong companions; we regard this as a locally cubical bicategory with only identity \fxnote*{The tight transformations between double categories are the \emph{loose} 2-morphisms in this locally double bicategory, right?}{tight} 2-morphisms and identity 3-cells.  The functor $\comp$ is defined on tight transformations as the Godement product. The pseudo double functor $\transid_A: * \rightarrow \cD bl(A,A)$ maps the cells in the trivial double category $*$ to the identity cells and morphisms in $\cD bl(A,A)$. 
\end{eg}

\begin{eg}
  By~\cite[Corollary 12]{gg:ldstr-tricat}, there is a locally cubical bicategory $\fBicat$ defined by the following data:
  \begin{itemize}
  \item Its objects are bicategories.
  \item Its morphisms are functors.
  \item Its loose 2-cells are pseudo natural transformations.
  \item Its tight 2-cells are icons~\cite{lack:icons}.
  \item Its 3-cells are \emph{cubical modifications} as defined in~\cite[Definition 13]{gg:ldstr-tricat}.
  \end{itemize}
\end{eg}

We recall the definition of a cubical modification.

\begin{defn}
Let $F,G,H,K: \cD \rightarrow \cE$ be pseudo functors; let $\alpha: F \Rightarrow G$, $\beta: H \Rightarrow K$ be pseudo transformations; let $\gamma: F \Rightarrow H$, $\delta: G \Rightarrow K$ be icons. A \textbf{cubical modification}
\[
\begin{tikzpicture}
\node (tl) at (0,1) {$F$};
\node (tr) at (1,1) {$G$};
\node (bl) at (0,0) {$H$};
\node (br) at (1,0) {$K$};
\draw[doubletight] (tl) to node[above]{$\alpha$} (tr);
\draw[doubletight] (bl) to node[below]{$\beta$} (br);
\draw[doubletight] (tl) to node[left]{$\gamma$} (bl);
\draw[doubletight] (tr) to node[right]{$\delta$} (br);
\node at (.5,.5) {$\DDownarrow \Gamma$};
\end{tikzpicture}
\]
is given by a family of 2-cells $\Gamma_A: \alpha_A \RRightarrow \beta_A$ such that for every 1-cell $f:A \rightarrow B$ of $\cD$, the following equality holds.

 \begin{equation}
 \begin{aligned}
 \begin{tikzpicture}[scale=1.5]
 \node (tl) at (-1,1) {$FA$};
 \node (tm) at (0,1) {$FB$};
 \node (tr) at (1,1) {$GB$};
 \node (bl) at (-1,0) {$FA$};
 \node (bm) at (0,0) {$GA$};
 \node (br) at (01,0) {$GB$};
 \node (bl1) at (-1,-.7){$HA$};  
 \node (bm1) at (0,-.7) {$KA$};
 \node (br1) at (1,-.7) {$KB$}; 
 \draw[doubletight] (tm)  to node[above]{$\alpha_B$} (tr);
 \draw[doubleeq] (bm) to (bm1);
 \draw[doubletight] (bm) to node[above] {$Gf$}(br);
 \draw[doubleeq] (tr) to (br);
 \draw[doubleeq] (tl)  to  (tm);
 \draw[doubleeq] (tl) to (bl);
 \draw[doubletight] (tl) to node[above]{$Ff$}(tm);
 \draw[doubletight] (bl) to node[above]{$\alpha_A$}(bm);
 \node at (0,.5) {\footnotesize $\Downarrow \alpha_f$}; 
 \node at (0.5,-.3) {\footnotesize $\Downarrow \delta_f$}; 
  \node at (-0.5,-.3) {\footnotesize $\Downarrow \Gamma_A$};
 \draw[doubletight] (bl1)  to node[above]{$\beta_A$} (bm1);
 \draw[doubletight] (bm1) to  node[above]{$Kf$}(br1);
 \draw[doubleeq] (bl)  to (bl1);
 \draw[doubleeq] (br)  to (br1);
 \end{tikzpicture}
 \end{aligned}
 =
\begin{aligned}
 \begin{tikzpicture}[scale=1.5]
 \node (tl) at (-1,1) {$FA$};
 \node (tm) at (0,1) {$FB$};
 \node (tr) at (1,1) {$GB$};
 \node (bl) at (-1,0) {$HA$};
 \node (bm) at (0,0) {$HB$};
 \node (br) at (01,0) {$KB$};
 \node (bl1) at (-1,-.7){$HA$};  
 \node (bm1) at (0,-.7) {$KA$};
 \node (br1) at (1,-.7) {$KB$}; 
 \draw[doubletight] (tm)  to node[above]{$\alpha_B$} (tr);
 \draw[doubleeq] (tm) to (bm);
 \draw[doubletight] (bm) to node[above] {$\beta_B$}(br);
 \draw[doubleeq] (tr) to (br);
 \draw[doubleeq] (tl)  to  (tm);
 \draw[doubleeq] (tl) to (bl);
 \draw[doubletight] (tl) to node[above]{$Ff$}(tm);
 \draw[doubletight] (bl) to node[above]{$Hf$}(bm);
 \node at (-0.5,.5) {\footnotesize $\Downarrow \gamma_f$}; 
 \node at (0.5,.5) {\footnotesize $\Downarrow \Gamma_B$}; 
 \draw[doubletight] (bl1)  to node[above]{$\beta_A$} (bm1);
 \draw[doubletight] (bm1) to  node[above]{$Kf$}(br1);
 \draw[doubleeq] (bl)  to (bl1);
 \draw[doubleeq] (br)  to (br1);
 \node at (0,-0.3) {\footnotesize $\DDownarrow \beta_f$}; 
 \end{tikzpicture}
 \end{aligned}
\end{equation}

\end{defn}

The double functor $\transid_A: * \rightarrow \fBicat(A,A)$ maps the cells in the trivial bicategory $*$ to the identity cells and morphisms of $\fBicat(A,A)$. 
The double functor $\comp$ is defined on functors of bicategories by composition. On pseudo transformations and icons it is given by the Godement product. On cubical modifications it is defined below:

\begin{equation*}
\begin{aligned}
 \begin{tikzpicture}[scale=2]
 \node (tl) at (-1,1) {$FF'A$};
 \node (tm) at (0,1) {$GF'A$};
 \node (tr) at (1,1) {$GG'A$};
 \node (bl) at (-1,0) {$HF'A$};
 \node (bm) at (0,0) {$KF'A$};
 \node (br) at (01,0) {$KG'A$};
 \node (bl1) at (-1,-1){$HH'A$};  
 \node (bm1) at (0,-1) {$KH'A$};
 \node (br1) at (1,-1) {$KK'A$}; 
 \draw[doubletight] (tm)  to node[above]{$G(\alpha'_A)$} (tr);
 \draw[doubleeq] (tm) to (bm);
 \draw[doubletight] (bm) to node[above] {$K(\alpha'_A)$}(br);
 \draw[doubleeq] (tr) to (br);
 \draw[doubleeq] (tl)  to  (tm);
 \draw[doubleeq] (tl) to (bl);
  \draw[doubleeq] (bm) to (bm1);
 \draw[doubletight] (tl) to node[above]{$\alpha_{F'A}$}(tm);
 \draw[doubletight] (bl) to node[above]{$\beta_{F'A}$}(bm);
 \node at (-0.5,.5) {\footnotesize $\Downarrow \Gamma_{F'A}$}; 
 \node at (0.5,.5) {\footnotesize $\Downarrow \delta_{\alpha'_A}$}; 
 \draw[doubletight] (bl1)  to node[above]{$\beta_{H'A}$} (bm1);
 \draw[doubletight] (bm1) to  node[above]{$K(\beta'A)$}(br1);
 \draw[doubleeq] (bl)  to (bl1);
 \draw[doubleeq] (br)  to (br1);
 \node at (-.5,-0.5) {\footnotesize $=$}; 
\node at (.5,-0.5) {\footnotesize $\DDownarrow K\Gamma'_A$}; 
\end{tikzpicture}
\end{aligned}
\end{equation*}
%%%%%%%%%

Functoriality follows from naturality of the icons. Note that there are several equivalent ways to define this composition on cubical modifications, by choosing different versions of the Godement product.  

Now, we can similarly obtain the notion of \emph{functor} between locally cubical bicategories from the definition of a $\mathcal{V}$-enriched functor between $\mathcal{V}$-enriched bicategories.

\begin{defn}\label{def:lcbcfunc}
Let $\mathcal{V}$ be a monoidal 2-category. Let ${\fB,\fC}$ be $\mathcal{V}$-enriched bicategories. A $\mathcal{V}$-enriched functor $F: {\fB} \rightarrow {\fC}$ consists of the following data:
\begin{enumerate}
\item An assignment on objects that sends each object $A$ of ${\fB}$ to an object $F A$ of ${\fC}$.
\item For each two objects $A,B$ of ${\fB}$, a 1-cell ${\fB}(A,B) \rightarrow {\fC}(F(A),F(B))$ of $\mathcal{V}$.
\item For every triple of objects $A,B,C$ of ${\fB}$, a 2-cell of $\mathcal{V}$ 
\begin{align} 
\begin{tikzpicture}
\node(1) at (0,0) {${\fB}(A,B) \otimes {\fB}(B,C)$};
\node(2) at (5,0) {${\fC}(F(A),F(C))$};
\draw[->] (1) to[in=155, out=25] node[above]{$\fB \comp $} (2); 
\draw[->] (1) to[in=-155, out=-25] node[below]{$ \comp (F,F)$} (2); 
\node at (2.5,0) {$\Downarrow \phi \iso$};
\end{tikzpicture}
\end{align}
\item For every object $A$ of ${\fB}$ a 2-cell of $\mathcal{V}$
\begin{align}
\begin{tikzpicture}[xscale=.5, yscale=.3]
\node(1) at (0,0) {$*$};
\node(2) at (5,0) {${\fC}(A,A)$};
\node(3) at (5,-5) {${\fB}(F(A),F(A))$};
\draw[->] (1) to node[above]{$\looseid_{A}$} (2); 
\draw[->] (1) to node[below]{$\looseid_{F(A)}$} (3);
\draw[->] (2) to node[right]{$F$} (3); 
\node at (3.5,-1.5) {$\Downarrow \phi_u \iso$};
\end{tikzpicture}
\end{align}
\item The usual coherence diagrams, Definition 10 of~\cite{nick:tricatsbook} commute.
\end{enumerate}
\end{defn}

When we apply this definition to the monoidal 2-category $\cDbl$, we see that a {\bf functor of locally cubical bicategories} $F$ consists of a map of objects $A \mapsto F A$; pseudo double functors ${\fB}(A,B) \rightarrow {\fC}(F(A),F(B))$ for each two objects $A,B$; and tight transformations $\phi$ for each two objects $A,B$ of ${\fB}$, and $\phi_u$ for every object $A$ of ${\fB}$, plus axioms.

\begin{thm}\label{thm:h-functor}
The map $\cL$ gives rise to a functor of locally cubical bicategories $\cL \maps \fDblf\to \fBicat$.
\end{thm}
\begin{proof}
Note that the 1-functor $\cH:\mathbf{Dbl}\to\mathbf{Bicat}$ has a left adjoint that regards a bicategory as a double category with only identity tight 1-cells.
Thus, the functor of bicategories $\cDblf(\D,\E) \rightarrow \cBicat(\cH(\D), \cH (\E)) = \cH( \fBicat(\cH(\D),\cH(\E)))$ has an adjunct pseudo double functor $\fDblf(\D,\E) \rightarrow \fBicat(\cH\D, \cH \E)$. Consequently, the first two requirements in \autoref{def:lcbcfunc} are satisfied by Theorems \ref{thm:1-func} and \ref{thm:h-locfr}.
Since $\cH$ strictly preserves composition of 1-cells,
the third requirement amounts to the existence of a tight transformation $\phi\maps \behat * \alhat \iso \widehat{\be*\al}$ for every pair of transformations with loosely strong companions 

  \[\vcenter{\xymatrix{\lC \rtwocell^F_G{\al} & \lD \rtwocell^H_K{\be}
      & \lE}}\]
      
      such that 
%
 \begin{equation}
        \vcenter{\xymatrix@-.5pc{
        1_{{\cH}H \odot {\cH}F} \ar[r]\ar[d]_{=} &
        \hat{1}_{H}* \hat{1}_{F}\ar[d]^{\phi}\\
        1_{{\cH}(H \odot F)}\ar[r] &
        \widehat{1_{H} * 1_{F}}}} \quad\text{and}\quad       
    \vcenter{\xymatrix@-.5pc{
        \widehat{\gm\al}* \widehat{\de\be} \ar[r]\ar[d]_\phi &
        (\gmhat* \dehat)\circ(\alhat* \behat)\ar[d]^{\phi * \phi}\\
        \widehat{\gm\al* \de\be}\ar[r] &
        (\widehat{\gm* \de})\circ(\widehat{\al* \be})}}
  \end{equation}
commute. 
Here, we use the 'Godement product' $*$ of 2-cells in $\cDbl$.  

  Now by Lemmas \ref{thm:comp-compose} and
  \ref{thm:comp-func}, $(\behat *\alhat)_A = \behat_{GA} \circ
  H(\alhat_A)$ is a companion of $(\be*\al)_A = \be_{GA} \circ
  H(\al_A)$.  Therefore, we take the component $(\phi_{\alpha,\beta})_A$ to be
  \[\theta_{\behat_{GA} \circ H(\alhat_A),\, \widehat{\be*\al}_A}.\]
 As the other morphisms in the diagrams above are also $\theta$-isomorphisms, the equations hold by Lemma~\ref{thm:theta-compose-vert}.
For the tight transformation  $\phi_u$ we can simply take the identity, since $\cH$ is strictly unital.
The coherence equations hold by Lemma~\ref{thm:theta-compose-vert}
\end{proof}


Our goal is to enhance this functor to act on ``monoidal objects''.
It is well-known that ``monoidal functors preserve monoid objects'', so our approach will be to categorify this: we will show that the functor $\cH$ is monoidal, in an appropriate sense, and that monoidal functors of this sort preserve monoidal objects of the appropriate sort.

In fact, the monoidality of $\cH$ is easy to describe, because the monoidal structures of $\fDblf$ and $\fBicat$ are cartesian and very strict.

In general, if $\mathcal{V}$ is a monoidal 2-category with strict 2-categorical finite products (such as \cDbl), we say that a $\mathcal{V}$-enriched bicategory ${\fB}$ has \textbf{finite products} when for each two objects $C,D \in {\fB}$ there is an object $C\times D$ with projections $C\times D\to C$ and $C\times D\to D$ (i.e.\ morphisms $I\to \fB(C\times D,C)$ and $I\to \fB(C\times D,D)$ in \cV) inducing an \emph{isomorphism} in $\mathcal{V}$ (not merely an equivalence):
%
\begin{align}
\fB(A, C \times D) \xrightarrow{\cong} \fB(A,C) \times \fB(A,D)
\end{align}
and similarly there is a strict terminal object $\ast$ such that $\fB(A,\ast)$ is strictly terminal in \cV\ for all $A$.
This holds for \fBicat\ and \fDblf, because cartesian products of bicategories and double categories are simply componentwise, and all the morphisms in \fBicat\ and \fDblf\ (no matter how weak) are defined in terms of data in their targets.

Similarly, we say that a functor $F$ of \cV-enriched bicategories \textbf{preserves products} if it takes the terminal object to a terminal object and pairs of product projections $A \leftarrow A\times B \to B$ to pairs of product projections (in the above strict sense).

\begin{thm}
The functor of locally cubical bicategories $\cH: \fDblf \rightarrow \fBicat$ preserves products.
\end{thm}
\begin{proof}
Since $\cH$ merely forgets a part of the double categories and double functors, we have simple equalities
$\cH(\mathbb{D} \times \mathbb{E}) = \cH(\mathbb{D}) \times \cH(\mathbb{E})$, and the product projections are likewise preserved.
The case of the terminal object is likewise easy.
\end{proof}

% Local Variables:
% TeX-master: "smbicat"
% End:


\section{Monoidal objects in locally cubical bicategories}
\label{sec:mono-objects}

We now move on to define an appropriate abstract sort of ``monoidal object'' that will be preserved by the product-preserving functor $\cH$, and that specializes to monoidal double categories and to monoidal bicategories.
Moreover, we want $\cH$ to act on monoidal morphisms between such monoidal objects, and indeed to extend to a functor between categories of monoidal objects.
It would be nice if this enhanced functor $\cH$ could stay entirely in the world of iconic tricategories (that is, \Icon-enriched bicategories); but unfortunately the usual composition of monoidal functors between monoidal bicategories is not strictly associative, so they do not form an iconic tricategory.

However, they do form a more general structure, namely a bicategory enriched over \cDbl; in~\cite{gg:ldstr-tricat} this is called a \textbf{locally cubical bicategory}.
Since any bicategory can be regarded as a double category with only identity tight 1-morphisms, any iconic tricategory can be regarded as a locally cubical bicategory, but the latter are more general.
In addition to the objects, 1-morphisms, 2-cells (now called ``loose 2-cells''), and 3-cells (now called ``globular 3-cells'') of an iconic tricategory, a locally cubical bicategory contains \emph{tight} 2-morphisms, and square-shaped 3-cells.
The composition of 1-morphisms is now associative only up to an invertible \emph{tight} 2-morphism, and one of the results of~\cite{gg:ldstr-tricat} is that monoidal bicategories form a locally cubical bicategory.
We will generalize this theorem to monoidal objects, perhaps braided and symmetric, in any iconic tricategory with finite products --- and indeed, in any locally cubical bicategory with finite products.

Since \cDbl, like \Icon, is a cartesian monoidal 2-category, we can define what it means for a locally cubical bicategory to have finite products, and this property is preserved when regarding an iconic tricategory as a locally cubical bicategory.
In particular, this applies to \cDblf\ and to \cBicat\ --- but actually, in place of the iconic tricategory \cBicat\ considered up until now we will focus instead on the locally cubical bicategory of bicategories constructed in~\cite{gg:ldstr-tricat}, whose ``locally loose part'' is \cBicat, but whose tight 2-cells are \emph{icons}.
We denote this by \fBicat; it is easy to see that it also has products preserved by the inclusion $\cBicat\to \fBicat$, so that the composite functor $\cH : \cDblf \to\fBicat$ still preserves products.

We now define symmetric, braided and monoidal structures on objects, 1-cells, 2-cells, and 3-cells internal to a locally cubical bicategory with products, by taking the definitions of monoidal, braided, and symmetric structure for bicategories given in~\cite{nick:tricatsbook},~\cite{mccrudden:bal-coalgb}, and~\cite{gg:ldstr-tricat} and regarding the data of bicategories, functors, pseudonatural transformations, and modifications abstractly as objects, 1-cells, 2-cells, and 3-cells in a locally cubical bicategory.

Note that under this translation pseudonatural transformations become \emph{loose} 2-cells and modifications become globular 3-cells.
The loose 2-cells in \cDblf\ (which has no nonidentity tight 2-morphisms) are the (tight) transformations, while those in \fBicat\ are exactly the pseudonatural transformations (its tight 2-morphisms are icons).

Before we give the definitions of monoidal cells, we recall the structure of a locally cubical bicategory and fix our notation. Locally cubical bicategories have three types of composition. As a locally cubical category is a specific kind of intercategory, we will adopt the notation introduced for intercategories in~\cite{gp:intercategories-i}. Firstly, we have loose composition ``$\horc$" within the hom- double categories. This gives us composition of loose 2-cells along a 1-cell boundary and of 3-cells along a tight 2-cell bounday. We write this composition in the order of diagrammatic composition: $\alpha \horc \beta$, meaning ``$\beta$ after $\alpha$". We write $\looseid_{f}$ and $\looseid_{\alpha}$ for the loose identity on a 1-cell $f$ and a tight 2-cell $\alpha$, respectively. Loose composition is weakly associative and loose identities hold up to isomorphism. 
Secondly, we have tight composition ``$\verc$" in the hom-double categories. This gives us composition of tight 2-cells along a 1-cell boundary and tight composition of 3-cells along a loose 2-cell boundary, written in the conventional order: $f \verc g$ denoting ``$f$ after $g$". We write $\tightid_f$ and $\tightid_{\alpha}$ for the tight identity on a 1-cell $f$ and a loose 2-cell $\alpha$, respectively. Tight composition is strictly associative and has strict identities. 
Thirdly, there is composition ``$\comp$" of 1-cells, 2-cells, and 3-cells along a 0-cell boundary, given by the enriched structure. We write this composition in the conventional order: $f \comp g$ meaning ``$f$ after $g$". When it is clear from the context, we omit the composition symbol ``$\comp$",  and write the juxtaposition of 1-cells instead. The identities for this composition are denoted by ``$\transid$" \fxnote*[author=MS]{Does this ever get applied to anything other than an object?  The other things in an intercategory that it would be applied to are all identities here.}{with the respective object, 1-cell, loose 2-cell or tight 2-cell in subscript}. This composition is weak with weak identities. We write $\compI$ for the unit double functor and $\compI_A$ for the image $\compI(A)$ of an object $A$, and likewise for 1-cells, 2-cells and 3-cells.  
%%%%%%rewrite this
By functoriality, the unit loose 2-cell $\compI_{\compI_A}$ is isomorphic to the loose identity $\looseid_{\compI_A}$, the unit tight 2-cells and the unit 3-cells equal the identities $\tightid_{\compI_A}$ and $\id_{\compI_{\compI_A}}$, respectively.
We write ``$\onecell$" to denote 1-cells, ``$\looseRightarrow$" to denote the loose 2-cells, ``$\Rightarrow$" to denote the tight 2-cells and ``$\RRightarrow$" to denote $3$-cells.

For readability, we will strictify the bicategory and the hom-double categories, as we did with the double categories in Section~\ref{sec:symm-mono-double}, except where we prove that this structure is preserved by monoidal cells. As a consequence, we omit the associativity and unit constraints for $\comp$ and $\verc$ in various places.
\begin{anfxnote}[author=MS]{Strictification}
  This sounds a little weird as written, because earlier on we made a point of moving from iconic tricategories to locally cubical bicategories \emph{because} the composition of monoidal functors of bicategories is \emph{not} strict.
  Also, it's not clear that the composition along objects can be strictified by any coherence theorem; isn't this one place where ``local fibrancy'' comes in?
  If it's just an abuse of notation, we should say that explicitly.
  Can we get by with only strictness of composing 1-morphisms along objects (rather than of the whole composition functor?), since that's true for $\fBicat$?
  We should emphasize in the definitions wherever the characteristic structure of an iconic tricategory or locally double bicategory is being used to substitute for ``obvious'' facts about bicategories in the definitions we are generalizing.
\end{anfxnote}

Let \fB\ be a locally cubical bicategory with products.

\begin{defn}
A {\bf monoidal object} in \fB\ is an object $A$, equipped with 1-cells $\otimes: A \times A \onecell A$ and $I_A: * \onecell A$, and adjoint equivalences
\begin{itemize} 
\item $\alpha: \ten  (\id \times \tens) \looseRightarrow{} \ten (\tens \times \id)$
\item $l: \ten (I \times \transid) i_2 \looseRightarrow{} \transid$ and $r:\ten (\transid \times I) i_1 \looseRightarrow{} \transid$ 
\end{itemize}
where $i_1$ and $i_2$ are the morphisms defining the product $A \times A$. Finally, it must be equipped with the invertible globular 3-cells $\pi, \mu, \lambda, \rho$, relating the two different ways around the Mac Lane pentagon and the three other coherence diagrams given in Definition 4.1 of~\cite{nick:tricatsbook}, which satisfy the appropriate three axioms.

A monoidal object is {\bf braided} if in addition it is equipped with a loose 2-cell $\sigma_A: \tens \looseRightarrow{} \mathord{\ten} \tau$, where $\tau: A \times A \rightarrow A \times A$ interchanges the two copies of $A$; and if there are invertible globular 3-cells 

\begin{equation}
  \begin{aligned}
\begin{tikzpicture}[xscale=0.9]
\node (t) at (2,3) {$\ten (\ten \times \transid)$};
\node (tl) at (0,2) {$\ten(\ten \times \transid)$};
\node (bl) at (0,1) {$\ten (\transid \times \ten)$};
\node (b) at (2,0) {$\ten (\transid \times \ten)$};
\node (tr) at (4,2) {$\ten(\transid \times \ten)$};
\node (br) at (4,1) {$\ten (\ten \times \transid)$};
\draw[doubleloose] (t) to node [above,xshift=10pt, yshift=-2] {$\alpha$} (tr);
\draw[doubleloose] (tr) to node [right] {$\sigma$} (br);
\draw[doubleloose] (br) to node [below,xshift=10pt, yshift=2pt] {$\alpha$} (b);
\draw[doubleloose] (t) to node [above, xshift=-10pt, yshift=-2pt] {$\sigma \ten \looseid$} (tl);
\draw[doubleloose] (tl) to node [left] {$\alpha$} (bl);
\draw[doubleloose] (bl) to node [below,xshift=-10pt,yshift=2pt] {$\looseid \ten \sigma$} (b);
\node at (2,1.5) {$\DDownarrow R \iso$};
\end{tikzpicture}
  \end{aligned}
\hspace{5pt}\mbox{and} \hspace{5pt}
\begin{aligned}
\begin{tikzpicture}[xscale=0.9]
\node (t) at (2,3) {$\ten(\transid \times \ten)$};
\node (tl) at (0,2) {$\ten(\transid \times \ten)$};
\node (bl) at (0,1) {$\ten(\ten \times \transid)$};
\node (b) at (2,0) {$\ten(\ten \times \transid)$};
\node (tr) at (4,2) {$\ten(\ten \times \transid)$};
\node (br) at (4,1) {$\ten(\transid \times \ten)$};
\draw[doubleloose] (tr) to node [above,xshift=10pt, yshift=-2] {$\alpha$} (t);
\draw[doubleloose] (tr) to node [right] {$\sigma$} (br);
\draw[doubleloose] (b) to node [below,xshift=10pt, yshift=2pt] {$\alpha$} (br);
\draw[doubleloose] (t) to node [above,xshift=-10pt, yshift=-2pt] {$\mbox{id} \ten \sigma$} (tl);
\draw[doubleloose] (tl) to node [left] {${\alpha}^{-1}$} (bl);
\draw[doubleloose] (bl) to node [below,xshift=-10pt,yshift=2pt] {$\sigma \ten \mathid$} (b);
\node at (2,1.5) {$\DDownarrow S \iso$};
\end{tikzpicture}
\end{aligned}
\end{equation}
satisfying the axioms (BA1), (BA2), (BA3), and (BA4) given in~\cite[p136--139]{mccrudden:bal-coalgb} . 
It is {\bf sylleptic} when it is additionally equipped with an invertible globular 3-cell

 \[
 \begin{tikzpicture}
 \node (tl) at (-2,1) {$\ten$};
 \node (tr) at (2,1) {$\ten$};
 \node (b) at (0,-.25) {$\tens \tau$};
 \draw[double] (tl)  -- (tr);
 \draw[doubleloose] (tl) to node[left, yshift=-5pt]{$\sigma$} (b);
 \draw[doubleloose] (b) to node[right, yshift=-5pt] {$\sigma$}(tr);
 \node at (0,0.5) {\footnotesize $\DDownarrow \upsilon \iso$}; 
 \end{tikzpicture}
 \]
  satisfying the axioms (SA1), (SA2) on~\cite[p144--145]{mccrudden:bal-coalgb}. It is {\bf symmetric} if in addition, it satisfies the axiom given on~\cite[p91]{mccrudden:bal-coalgb}.
\end{defn}

By construction, these definitions give the expected results in \fBicat.
In \cDblf, where there are no nonidentity 3-cells, they reduce to the definitions from section~\ref{sec:symm-mono-double}; and in particular every syllepsis is a symmetry.

\begin{defn}
Let $A,B$ be monoidal objects in \fB. A 1-cell $f:A \onecell B$ is {\bf lax monoidal} when it is equipped with the following loose 2-cells:
\begin{itemize}
\item $\chi: \mathord{\ten} (f \times f) \looseRightarrow{} f  \mathord{\otimes}  $
\item $\iota: I_B \looseRightarrow{} fI_A $
\end{itemize}
as well as globular invertible 3-cells 
%remember that we write the horizontal composition \horc in diagrammatic order!
\begin{align*}
& \omega:  \looseid_{\tens}(\chi \times \looseid_f)  \horc  \chi\looseid_{\tens \times \transid} \horc  \looseid_f \alpha \RRightarrow \alpha\looseid_{f \times f \times f}  \horc \looseid_{\tens}(\looseid \times \chi)  \horc \chi \looseid_{\transid \times \tens}  \\
 &\gamma: \looseid_{\tens}(\iota_f \times \looseid_f) \looseid_{i_2} \horc \chi \looseid_{I \times \transid} \looseid_{i_2} \horc \looseid_f l\RRightarrow l \looseid_f \\
 &\delta:  \looseid_f r^{-1} \RRightarrow r^{-1} \looseid_f \horc \looseid_{\tens} (\looseid \times \iota) \looseid_{i_1 f} \horc \chi \looseid_{(\transid \times I)i_1}
\end{align*}
as in Definition 4.10 of~\cite{nick:tricatsbook}, expressing the usual associativity and unitality conditions, which satisfy the three given commutativity axioms.
A monoidal 1-cell is called {\bf braided}, when $A$ and $B$ are braided and there is a globular 3-cell $u: \sigma_B \looseid_{f \times f} \verc \chi  \looseid_{\tau} \looseRightarrow \chi \horc (\looseid_f \sigma_A)$, satisfying the braiding axioms analogous to (BHA1) and (BHA2) given in  \cite[p141-142]{mccrudden:bal-coalgb}. It is {\bf sylleptic} when $A$ and $B$ are sylleptic and the 3-cells defining the braided monoidal structure of $f$ satisfy the additional axiom analogous to  (SHA1) given in   \cite[p145]{mccrudden:bal-coalgb}, and \textbf{symmetric} if it is sylleptic and $A$ and $B$ are symmetric.

The morphism $f$ is {\bf oplax monoidal} when it is equipped with 2-cells and 3-cells as shown below:
\begin{align*}
\bar{\chi} &: f  \otimes\looseRightarrow{} \ten (f \times f)\\
\bar{\iota} &: fI_A\looseRightarrow{} I_B \\
 \bar{\omega}&: \bar{\chi} \looseid_{\transid \times \tens}  \horc  \looseid_{\tens}(\looseid_f \times \bar{\chi})   \horc  \alpha^{-1}\looseid_{f \times f \times f} \RRightarrow  \looseid_f \alpha^{-1}  \horc  \bar{\chi} \looseid_{\tens \times \transid} \horc  \looseid_{\tens}(\bar{\chi} \times \looseid_f)\\ 
 \bar{\gamma}&: l^{-1} \looseid_f \RRightarrow  \looseid_f l^{-1}   \horc \bar{\chi} \looseid_{I \times \transid} \looseid_{i_2} \horc \looseid_{\tens}(\bar{\iota}_f \times \looseid_f) \looseid_{i_2}\\
 \bar{\delta}&:  \bar{\chi} \looseid_{\transid \times I} \horc \looseid_{\tens} (\looseid_{f}\times \bar{\iota}) \looseid_{i1} \horc r^{-1} \looseid_f \RRightarrow  \looseid_f r^{-1}
\end{align*}
satisfying analogous axioms.

If $f$ is both lax and oplax monoidal, the associated loose 2-cells $\chi$ and $\iota$ form adjoint equivalences with their oplax counterparts, and the 3-cells correspond to their oplax counterparts as mates under the adjoint equivalence structure, it is {\bf strong monoidal}.
\end{defn}



\begin{defn}\label{Def:monverttrans}
Let $f, g:A \onecell B$ be lax monoidal 1-cells in \fB. A {\bf lax monoidal 2-cell} $\beta: f \looseRightarrow g$ is a loose 2-cell in \fB\ that is equipped with globular 3-cells
\begin{itemize}
\item $\Pi: \chi_f \horc \beta  \looseid_{\ten} \RRightarrow{} \looseid_{\ten}(\beta \times \beta) \horc \chi_g$
\item $M: \iota_f \horc \beta  \looseid_{I_A} \RRightarrow{} \looseid_{I_B} \horc \iota_g$
\end{itemize}
such that coherence equations \eqref{eq:mon2cell1}, \eqref{eq:mon2cell2}, and \eqref{eq:mon2cell3} below hold. These axioms are analogous to (TA2), (TA3) and (TA4) of~\cite{gg:ldstr-tricat}.

\begin{anfxnote}[author=MS]{Lax/oplax}
  Is the matching of lax monoidal 1-cells and lax monoidal 2-cells between them necessary?
  I.e.\ could we have an oplax monoidal 2-cell (or I suppose ``monoidal oplax 2-cell'' might be clearer) between a pair of lax monoidal 1-cells?
\end{anfxnote}

Let $f, g:A \onecell B$ be oplax monoidal 1-cells in \fB. An {\bf oplax monoidal 2-cell} $\beta: f \looseRightarrow{} g$ is a loose 2-cell in \fB\ that is equipped with globular 3-cells
\begin{itemize}
\item $\bar{\Pi}: \beta \looseid_{\ten} \horc \bar{\chi}_g \RRightarrow{} \bar{\chi}_f \horc \looseid_{\ten}(\beta \times \beta)  $
\item $\bar{M}: \beta \looseid_{I_A} \horc \bar{\iota}_g  \RRightarrow{} \bar{\iota}_f \horc \looseid_{I} $
\end{itemize}
such that coherence equations analogous to \eqref{eq:mon2cell1}, \eqref{eq:mon2cell2}, and \eqref{eq:mon2cell3} hold.

If $g,f$ are strong monoidal, we call $\beta$ a {\bf strong monoidal 2-cell} when it is equipped with $M, \Pi, \bar{M}$ and $\bar{\Pi}$, which correspond to each other in pairs as mates under the adjoint equivalence structure on $\chi$ and $\iota$.

A monoidal 2-cell is {\bf braided}, {\bf sylleptic} or {\bf symmetric} when $f,g$ are braided, sylleptic or symmetric, and in addition the coherence axiom~\eqref{eq:br2cell} holds. This axiom is analogous to (BTA1) of~\cite[p143]{mccrudden:bal-coalgb}.
\end{defn}
\begin{anfxnote}[author=MS]{Stating axioms}
  Some of the axioms are stated explicitly even though they are ``analogous'' others in the literature; others are simply cited.
  Is there a difference?
  Also, it might be worth being more explicit about exactly how ``analogous'' these are; are there any changes at all that have to be made?

  It might also be better to put these axioms in floating \texttt{figure}s.
\end{anfxnote}

\begin{equation}\label{eq:mon2cell1}
\begin{aligned}
\begin{tikzpicture}[xscale=3, yscale=1.5]
\node (t0) at (0,2) {\small $\tens(I_B \times f)i_2$};
\node (t1) at (1,2) {\small $\tens(f I_A \times f)i_2$};
\node (t2) at (2,2) {\small $f \tens(I_A \times \transid)i_2$};
\node (t3) at (3,2) {\small $f $};
\node (t4) at (4,2) {\small $g $};
\node (m0) at (0,1) {\small $\tens(I_B \times \transid)i_2f$};
\node (m3) at (3,1) {\small $f $};
\node (m4) at (4,1) {\small $g $};
\node (b0) at (0,0) {\small $\tens(I_B \times \transid)i_2f$};
\node (b3) at (3,0) {\small $\tens (I_B \times \transid)i_2g$};
\node (b4) at (4,0) {\small $g $};
\draw[doubleloose] (t0) to node[above]{\small $\looseid_{\tens}(\iota_f \times \looseid_f)\looseid_{i_2}$} (t1);
\draw[doubleloose] (t1) to node[above]{\small $\chi (\looseid_{I \times \transid})\looseid_{i_2}$} (t2);
\draw[doubleloose] (t2) to node[above]{\small $\looseid_f l$} (t3);
\draw[doubleloose] (t3) to node[above]{\small $\beta$} (t4);
\draw[doubleloose] (m0) to node[above]{\small $l \looseid_f$} (m3);
\draw[doubleloose] (m3) to node[above]{\small $\beta$} (m4);
\draw[doubleloose] (b0) to node[above]{\small $\looseid_{\tens}(\beta \times \looseid_I)\looseid_{i_2}$} (b3);
\draw[doubleloose] (b3) to node[above]{\small $l \looseid_g$} (b4);
\draw[doubletighteq] (t0) to (m0);
\draw[doubletighteq] (m0) to (b0);
\draw[doubletighteq] (t3) to (m3);
\draw[doubletighteq] (t4) to (m4);
\draw[doubletighteq] (m4) to (b4);
\node at (1.5,1.5) {\small $\DDownarrow \gamma^f$};
\node at (3.5,1.5) {\small $\DDownarrow \tightid_{\beta}$};
\node at (2,0.5) {\small $\iso$};
\end{tikzpicture}
\end{aligned}
\end{equation}
\[
=
\]
\begin{equation*}
\begin{aligned}
\begin{tikzpicture}[xscale=3, yscale=1.5]
\node (04) at (0,4) {\small $\tens(I_B \times f)i_2$};
\node (14) at (1,4) {\small $\tens(f I_A\times f)i_2$};
\node (24) at (2,4) {\small $f \tens(I_A \times \transid_A)i_2$};
\node (34) at (3,4) {\small $f $};
\node (44) at (4,4) {\small $g $};
%%%%%%
\node (03) at (0,3) {\small $\tens(I_B \times f)i_2$};
\node (13) at (1,3) {\small $\tens(f I_A\times f)i_2$};
\node (23) at (2,3) {\small $f \tens(I_A \times \transid_A)i_2$};
\node (33) at (3,3) {\small $g \tens(I_A \times \transid_A)i_2$};
\node (43) at (4,3) {\small $g $};
%%%%%%
\node (02) at (0,2) {\small $\tens(I_B \times f)i_2$};
\node (12) at (1,2) {\small $\tens(f I_A \times f)i_2$};
\node (22) at (2,2) {\small $\tens(g I_A \times g)i_2$};
\node (32) at (3,2) {\small $g \tens (I_A\times \transid_A)i_2$};
\node (42) at (4,2) {\small $g $};
%%%%%% 
\node (01) at (0,1) {\small $\tens(I_B \times f)i_2$};
\node (11) at (1,1) {\small $\tens(I_B \times g)i_2$};
\node (21) at (2,1) {\small $\tens(g I_A \times g)i_2$};
\node (31) at (3,1) {\small $g \tens (I_A \times \transid_A)i_2$};
\node (41) at (4,1) {\small $g $};
%%%%%%%
\node (00) at (0,0) {\small $\tens(I_B \times \transid)i_2 f$};
\node (10) at (1,0) {\small $\tens(I_B \times \transid)i_2 g$};
\node (40) at (4,0) {\small $g $};
%%%%%%%
\draw[doubleloose] (04) to node[above]{\small $\looseid_{\tens}(\looseid_f \times \iota_f) \looseid_{i_2}$} (14);
\draw[doubleloose] (14) to node[above]{\small $\chi (\looseid_{I \times \transid})\looseid_{i_2}$} (24);
\draw[doubleloose] (24) to node[above]{\small $\looseid_f l$} (34);
\draw[doubleloose] (34) to node[above]{\small$\beta$} (44);
%%%%%%%
\draw[doubleloose] (03) to node[above]{\small $\looseid_{\tens}(\looseid_f \times \iota_f) \looseid_{i_2}$} (13);
\draw[doubleloose] (13) to node[above]{\small $\chi (\looseid_{I \times \transid})\looseid_{i_2}$} (23);
\draw[doubleloose] (23) to node[above]{\small $\beta \looseid_{\tens}(\looseid_{I\times \transid})\looseid_{i_2}$} (33);
\draw[doubleloose] (33) to node[above]{\small$\looseid_g l$} (43);
%%%%
\draw[doubleloose] (02) to node[above]{\small $\looseid_{\tens} (\iota_f  \times \looseid_f)\looseid_{i_2}$} (12);
\draw[doubleloose] (12) to node[above]{\small $\looseid_{\tens} (\beta \looseid_I \times \beta)\looseid_{i_2} $} (22);
\draw[doubleloose] (22) to node[above]{\small $\chi_g \looseid_{I \times \id}\looseid_{i_2}$} (32);
\draw[doubleloose] (32) to node[above]{\small $\looseid_g l$} (42);
%%%%%%
\draw[doubleloose] (01) to node[above]{\small $\looseid_{\tens} (\looseid_I \times \beta)\looseid_{i_2}$} (11);
\draw[doubleloose] (11) to node[above]{\small $\looseid_{\tens} (\iota_g \times \looseid_g) \looseid_{i_2}$} (21);
\draw[doubleloose] (21) to node[above]{\small $\chi \looseid_{I_A \times \transid}\looseid_{i_2}$} (31);
\draw[doubleloose] (31) to node[above]{\small $\looseid_g l$} (41);
%%%%%%
\draw[doubleloose] (00) to node[above]{\small $\looseid_{\tens} (\looseid_I  \times \looseid)\looseid_{i_2} \beta$} (10);
\draw[doubleloose] (10) to node[above]{\small $l \looseid_g $} (40);
%%%%%%
\draw[doubletighteq] (04) to (03);
\draw[=] (24) to (23);
\draw[doubletighteq] (44) to (43);
%%%%%%
\draw[doubletighteq] (03) to (02);
\draw[doubletighteq] (13) to (12);
\draw[doubletighteq] (33) to (32);
\draw[doubletighteq] (43) to (42);
%%%%%%
\draw[doubletighteq] (02) to (01);
\draw[doubletighteq] (22) to (21);
\draw[doubletighteq] (42) to (41);
%%%%%%
\draw[doubletighteq] (01) to (00);
\draw[doubletighteq] (11) to (10);
\draw[doubletighteq] (41) to (40);
%%%%%%%%
\node at (1,3.5) {\small $=$};
\node at (3,3.5) {\small $\DDownarrow \cong $};
\node at (.5,2.5) {\small $=$};
\node at (2,2.5) {\small $\DDownarrow \overline{\Pi^{\beta}\tightid_{\looseid}}$};
\node at (3.5,2.5) {\small $=$};
\node at (1,1.5) {\small $\DDownarrow \overline{\tightid_{\looseid} (M^{\beta} \times (\horl \verc {\horr}^{-1}) }$};
\node at (3,1.5) {\small $=$};
\node at (.5,.5) {\small $=$};
\node at (2.5,0.5) {\small $\DDownarrow \gamma^g$};
\end{tikzpicture}
\end{aligned}
\end{equation*}

%%%%%%%%%%%%%%%%%%%%%%%%%%%%%%%%%

\begin{equation}\label{eq:mon2cell2}
\begin{aligned}
\begin{tikzpicture}[xscale=3, yscale=1.5]
\node (t0) at (0,2) {\small $\tens(f\times I_B)i_1$};
\node (t1) at (1,2) {\small $\tens(f\times fI_A)i_1$};
\node (t2) at (2,2) {\small $f\tens(\id \times I_A)i_1$};
\node (t3) at (3,2) {\small $f $};
\node (t4) at (4,2) {\small $g $};
\node (m0) at (0,1) {\small $\tens(\transid \times I_B)i_1f$};
\node (m3) at (3,1) {\small $f $};
\node (m4) at (4,1) {\small $g $};
\node (b0) at (0,0) {\small $\tens(\transid \times I_B)i_1f$};
\node (b3) at (3,0) {\small $\tens (\transid \times I_B)i_1g$};
\node (b4) at (4,0) {\small $g $};
%%%%%%%%%%%%%
\draw[doubleloose] (t0) to node[above]{$\small \looseid_{\tens}(\looseid_f \times \iota_f)\looseid_{i_1}$} (t1);
\draw[doubleloose] (t1) to node[above]{\small $\chi (\looseid \times \looseid_I)\looseid_{i_1}$} (t2);
\draw[doubleloose] (t2) to node[above]{\small $\looseid_{f}r$} (t3);
\draw[doubleloose] (t3) to node[above]{\small $\beta$} (t4);
\draw[doubleloose] (m0) to node[above]{\small $r \looseid_f$} (m3);
\draw[doubleloose] (m3) to node[above]{\small $\beta$} (m4);
\draw[doubleloose] (b0) to node[above]{\small $\looseid_{\tens}(\beta \times \looseid_I)\looseid_{i_1}$} (b3);
\draw[doubleloose] (b3) to node[above]{\small $r \looseid_g$} (b4);
\draw[doubletighteq] (t0) to (m0);
\draw[doubletighteq] (m0) to (b0);
\draw[doubletighteq] (t3) to (m3);
\draw[doubletighteq] (t4) to (m4);
\draw[doubletighteq] (m4) to (b4);
\node at (1.5,1.5) {\small $\DDownarrow \delta^f$};
\node at (3.5,1.5) {\small $\DDownarrow \tightid_{\beta}$};
\node at (2,0.5) {\small $\iso$};
\end{tikzpicture}
\end{aligned}
\end{equation}
\[
=
\]
\begin{equation*}
\begin{aligned}
\begin{tikzpicture}[xscale=3, yscale=1.5]
\node (04) at (0,4) {\small $\tens(f\times I_B)i_1$};
\node (14) at (1,4) {\small $\tens(f\times f I_A)i_1$};
\node (24) at (2,4) {\small $f \tens(\transid \times I_A)i_1$};
\node (34) at (3,4) {\small $f $};
\node (44) at (4,4) {\small $g $};
%%%%%%
\node (03) at (0,3) {\small $\tens(f\times I_B)i_1$};
\node (13) at (1,3) {\small $\tens(f\times f I_A)i_1$};
\node (23) at (2,3) {\small $f \tens(\transid \times I_A)i_1$};
\node (33) at (3,3) {\small $g \tens(\transid \times I_A)i_1$};
\node (43) at (4,3) {\small $g $};
%%%%%%
\node (02) at (0,2) {\small $\tens(f\times I_B)i_1$};
\node (12) at (1,2) {\small $\tens(f\times f I_A)i_1$};
\node (22) at (2,2) {\small $\tens(g\times g I_A)i_1$};
\node (32) at (3,2) {\small $g\tens(\transid \times I_A)i_1$};
\node (42) at (4,2) {\small $g $};
%%%%%%
\node (01) at (0,1) {\small $\tens(f\times I_B)i_1$};
\node (11) at (1,1) {\small $\tens(g\times  I_B)i_1$};
\node (21) at (2,1) {\small $\tens(g\times g I_A)i_1$};
\node (31) at (3,1) {\small $g \tens (\transid \times I_A)i_1$};
\node (41) at (4,1) {\small $g $};
%%%%%%%
\node (00) at (0,0) {\small $\tens(\transid \times I_B)i_1f$};
\node (10) at (1,0) {\small $\tens(\transid \times  I_B)i_1g$};
\node (40) at (4,0) {\small $g $};
%%%%%%%
\draw[doubleloose] (04) to node[above]{\small $\looseid_{\tens}(\looseid_f \times \iota_f)\looseid_{i_1}$} (14);
\draw[doubleloose] (14) to node[above]{\small $\chi (\looseid_{\transid \times I})\looseid_{i_1}$} (24);
\draw[doubleloose] (24) to node[above]{\small $\looseid_f r$} (34);
\draw[doubleloose] (34) to node[above]{\small $\beta$} (44);
%%%%%%%
\draw[doubleloose] (03) to node[above]{\small $\looseid_{\tens}(\looseid_f \times \iota_f)\looseid_{i_1}$} (13);
\draw[doubleloose] (13) to node[above]{\small $\chi (\looseid_{\transid \times I})\looseid_{i_1}$} (23);
\draw[doubleloose] (23) to node[above]{\small $\beta \looseid_{\tens}(\looseid_{\transid \times I})\looseid_{i_1}$} (33);
\draw[doubleloose] (33) to node[above]{\small $\looseid_g r$} (43);
%%%%
\draw[doubleloose] (02) to node[above]{\small $\looseid_{\tens} (\looseid_f  \times \iota_f) \looseid_{i_1}$} (12);
\draw[doubleloose] (12) to node[above]{\small $\looseid_{\tens} (\beta \times \beta) \looseid_{\transid \times I}\looseid_{i_1}$} (22);
\draw[doubleloose] (22) to node[above]{\small $\chi \looseid_{\transid \times I}\looseid_{i_1}$} (32);
\draw[doubleloose] (32) to node[above]{\small $\looseid_g r$} (42);
%%%%%%
\draw[doubleloose] (01) to node[above]{\small $\looseid_{\tens} (\beta \times \looseid_I)\looseid_{i_1}$} (11);
\draw[doubleloose] (11) to node[above]{\small $\looseid_{\tens} (\looseid_g \times \iota_g) \looseid_{i_1}$} (21);
\draw[doubleloose] (21) to node[above]{\small $\chi \looseid_{\transid \times I}\looseid_{i_1}$} (31);
\draw[doubleloose] (31) to node[above]{\small $\looseid_g r$} (41);
%%%%%%
\draw[doubleloose] (00) to node[above]{\small $\looseid_{\tens} (\transid  \times \looseid_I)\looseid_{i_1} \beta$} (10);
\draw[doubleloose] (10) to node[above]{\small $r \looseid_g $} (40);
%%%%%%
\draw[doubletighteq] (04) to (03);
\draw[doubletighteq] (24) to (23);
\draw[doubletighteq] (44) to (43);
%%%%%%
\draw[doubletighteq] (03) to (02);
\draw[doubletighteq] (13) to (12);
\draw[doubletighteq] (33) to (32);
\draw[doubletighteq] (43) to (42);
%%%%%%
\draw[doubletighteq] (02) to (01);
\draw[doubletighteq] (22) to (21);
\draw[doubletighteq] (42) to (41);
%%%%%%
\draw[doubletighteq] (01) to (00);
\draw[doubletighteq] (11) to (10);
\draw[doubletighteq] (41) to (40);
%%%%%%%%
\node at (1,3.5) {\small $=$};
\node at (3,3.5) {\small $\iso $};
\node at (1,2.5) {\small $=$};
\node at (3,2.5) {\small $\iso r_{\beta}$};
\node at (.5,3) {\small $=$};
\node at (2,2.5) {\small $\DDownarrow \overline{\Pi^{\beta}\tightid_{\looseid}}$};
\node at (3.5,2.5) {\small $=$};
\node at (1,1.5) {\small $\DDownarrow \overline{\tightid_{\looseid_{\tens}} (\horl \verc {\horr}^{-1}) \times M^{\beta}}$};
\node at (3,1.5) {\small $=$};
\node at (.5,.5) {\small $=$};
\node at (2.5,0.5) {\small $\DDownarrow \delta^g$};
\end{tikzpicture}
\end{aligned}
\end{equation*}

%%%%%%%%%%%%%%%%%%%%%%%%%%%%%%%%%
\begin{equation}\label{eq:mon2cell3}
\begin{aligned}
\begin{tikzpicture}[xscale=3.5, yscale=1.5]
\node (04) at (0,4) {\small$ \tens( \tens \times \transid)(f \times f \times f)$};
\node (14) at (1,4) {\small $ \tens(f \tens \times f)$};
\node (24) at (2,4) {\small $f \tens(\tens \times \transid)$};
\node (34) at (3,4) {\small $f\tens (\transid \times \tens)$};
\node (44) at (4,4) {\small $g \tens (\transid \times \tens)$};
\node (03) at (0,3) {\small $\tens( \tens \times \transid)(f \times f \times f)$};
\node (13) at (1,3) {\small $\tens( \transid \times \tens)(f \times f \times f)$};
\node (23) at (2,3) {\small $\tens (f \times f \tens)$};
\node (33) at (3,3) {\small $f \tens (\transid \times  \tens)$};
\node (43) at (4,3) {\small $g \tens (\transid \times  \tens)$};
\node (02) at (0,2) {\small $\tens( \tens \times \transid)(f \times f \times f)$};
\node (12) at (1,2) {\small $\tens( \transid \times \tens)(f \times f \times f)$};
\node (22) at (2,2) {\small $\tens (f \times f \tens)$};
\node (32) at (3,2) {\small $\tens (g \times g \tens)$};
\node (42) at (4,2) {\small $g \tens (\transid \times  \tens)$};
%%%%%%%
\node (01) at (0,1) {\small $\tens( \tens \times \transid)(f \times f \times f)$};
\node (11) at (1,1) {\small $\tens( \transid \times \tens)(f \times f \times f)$};
\node (21) at (2,1) {\small $\tens (\transid \times \tens) (g \times g \times g)$};
\node (31) at (3,1) {\small $\tens (g \times g \tens)$};
\node (41) at (4,1) {\small $g \tens (\transid \times  \tens)$};
%%%%%%%
\node (00) at (0,0) {\small $\tens( \tens \times \transid)(f \times f \times f)$};
\node (10) at (1,0) {\small $\tens( \transid \times \tens)(g \times g \times g)$};
\node (20) at (2,0) {\small $\tens (\transid \times \tens) (g \times g \times g)$};
\node (30) at (3,0) {\small $\tens (g \times g \tens)$};
\node (40) at (4,0) {\small $g \tens (\transid \times  \tens)$};
%%%%%%%
\draw[doubleloose] (04) to node[above]{\small $\looseid_{\tens}(\chi_f \times \looseid_f)$} (14);
\draw[doubleloose] (14) to node[above]{\small $\chi_f \looseid_{\tens \times \transid}$} (24);
\draw[doubleloose] (24) to node[above]{\small $\looseid_{f}\alpha$} (34);
\draw[doubleloose] (34) to node[above]{\small $\beta \looseid_{\tens} \looseid_{\transid \times \tens}$} (44);
%%%%%%%%
\draw[doubleloose] (03) to node[above]{\small $\alpha \looseid_{f \times f \times f}$} (13);
\draw[doubleloose] (13) to node[above]{\small $\looseid_{\tens} (\looseid_{\transid} \times \chi_f)$} (23);
\draw[doubleloose] (23) to node[above]{\small $\chi_f \looseid_{\transid \times \tens}$} (33);
\draw[doubleloose] (33) to node[above]{\small $\beta \looseid_{\tens} \looseid_{\transid \times \tens}$} (43);
%%%%%%%%
\draw[doubleloose] (12) to node[above]{\small $\looseid_{\tens} (\looseid_{\transid} \times \chi_f)$} (22);
\draw[doubleloose] (22) to node[above]{\small $\looseid_{\tens} (\beta \times \beta) \looseid_{\transid \times \tens}$} (32);
\draw[doubleloose] (32) to node[above]{\small $\chi_g \looseid_{\transid \times \tens}$} (42);
%%%%%%%%
\draw[doubleloose] (01) to node[above]{\small $\alpha \looseid_{f \times f \times f}$} (11);
\draw[doubleloose] (11) to node[above]{\small $\looseid_{\tens} \looseid_{\transid \times \tens} (\beta \times \beta \times \beta)$} (21);
\draw[doubleloose] (21) to node[above]{\small $\looseid_{\tens} (\looseid_{\transid} \times \chi_g)$} (31);
%%%%%%%%
\draw[doubleloose] (00) to node[above]{\small $\looseid_{\tens} (\looseid_{\tens} \times \looseid_{\transid})(\beta \times \beta \times \beta)$} (10);
\draw[doubleloose] (10) to node[above]{\small $\alpha \looseid_{g \times g \times g}$} (20);
\draw[doubleloose] (20) to node[above]{\small $\looseid_{\tens} (\looseid_{\transid} \times \chi_g)$} (30);
\draw[doubleloose] (30) to node[above]{\small $\chi_g \looseid_{\transid \times \tens}$} (40);
%%%%%%%%
\draw[doubletighteq] (04) to (03);
\draw[doubletighteq] (34) to (33);
\draw[doubletighteq] (44) to (43);
%%%%%%%%%
\draw[doubletighteq] (03) to (02);
\draw[doubletighteq] (13) to (12);
\draw[doubletighteq] (23) to (22);
\draw[doubletighteq] (43) to (42);
%%%%%%%%%
\draw[doubletighteq] (02) to (01);
\draw[doubletighteq] (12) to (11);
\draw[doubletighteq] (32) to (31);
\draw[doubletighteq] (42) to (41);
%%%%%%%%%
\draw[doubletighteq] (01) to (00);
\draw[doubletighteq] (21) to (20);
\draw[doubletighteq] (31) to (30);
\draw[doubletighteq] (41) to (40);
%%%%%%%%%
\node at (1.5,3.5) {\small $\DDownarrow \omega^f$};
\node at (3.5,3.5) {\small $=$};
\node at (0.5,2) {\small $=$};
\node at (1.5,2.5) {\small $=$};
\node at (3,2.5) {\small $\DDownarrow \overline{\Pi^{\beta} \tightid_{\looseid_{\transid \times \tens}}}$};
\node at (2,1.5) {\small $\DDownarrow \overline{\tightid_{\looseid} ({\horl}^{-1} \horr \times \Pi^{\beta})}$};
\node at (3.5,1) {\small $=$};
\node at (1,0.5) {\small $\DDownarrow ({\horl}^{-1} \verc \horr) ({\horr}^{-1} \verc \horl)$};
\node at (2.5,0.5) {\small $=$};
\end{tikzpicture}
\end{aligned}
\end{equation}
\[
=
\]
\begin{equation*}
\begin{aligned}
\begin{tikzpicture}[xscale=3.5, yscale=1.5]
\node (04) at (0,4) {\small $\tens( \tens \times \transid)(f \times f \times f)$};
\node (14) at (1,4) {\small $\tens(f \tens \times f)$};
\node (24) at (2,4) {\small $f \tens(\tens \times \transid)$};
\node (34) at (3,4) {\small $f\tens (\transid \times \tens)$};
\node (44) at (4,4) {\small $g \tens (\transid \times \tens)$};
%%%%%%%%%
\node (03) at (0,3) {\small $\tens( \tens \times \transid)(f \times f \times f)$};
\node (13) at (1,3) {\small $\tens( f \tens \times f)$};
\node (23) at (2,3) {\small $f \tens ( \tens \times \transid)$};
\node (33) at (3,3) {\small $g \tens (\tens \times  \transid)$};
\node (43) at (4,3) {\small $g \tens (\transid \times \tens)$};
\node (02) at (0,2) {\small $\tens( \tens \times \transid)(f \times f \times f)$};
\node (12) at (1,2) {\small $\tens(f \tens \times f )$};
\node (22) at (2,2) {\small $\tens (g \tens \times g)$};
\node (32) at (3,2) {\small $g \tens (\tens \times \transid)$};
\node (42) at (4,2) {\small $g \tens (\transid \times  \tens)$};
%%%%%%%
\node (01) at (0,1) {\small $\tens( \tens \times \transid)(f \times f \times f)$};
\node (11) at (1,1) {\small $\tens( \tens \times \transid)(g \times g \times g)$};
\node (21) at (2,1) {\small $\tens (g \tens \times g)$};
\node (31) at (3,1) {\small $g \tens ( \tens \times \transid )$};
\node (41) at (4,1) {\small $g \tens (\transid \times  \tens)$};
%%%%%%%
\node (00) at (0,0) {\small $\tens( \tens \times \transid)(f \times f \times f)$};
\node (10) at (1,0) {\small $\tens( \transid \times \tens)(g \times g \times g)$};
\node (20) at (2,0) {\small $\tens (\transid \times \tens) (g \times g \times g)$};
\node (30) at (3,0) {\small $\tens (g \times g \tens)$};
\node (40) at (4,0) {\small $g \tens (\transid \times  \tens)$};
%%%%%%%
\draw[doubleloose] (04) to node[above]{\small $\looseid_{\tens}(\chi_f \times \looseid_f)$} (14);
\draw[doubleloose] (14) to node[above]{\small $\chi_f (\looseid_{\tens \times \transid})$} (24);
\draw[doubleloose] (24) to node[above]{\small $\looseid_{f}\alpha$} (34);
\draw[doubleloose] (34) to node[above]{\small $\beta \looseid_{\tens} \looseid_{\transid \times \tens}$} (44);
%%%%%%%%
\draw[doubleloose] (13) to node[above]{\small $\chi_f \looseid_{\tens \times \transid}$} (23);
\draw[doubleloose] (23) to node[above]{\small $\beta \looseid_{\tens} \looseid_{\tens \times \transid}$} (33);
\draw[doubleloose] (33) to node[above]{\small $ \looseid_{g} \alpha$} (43);
%%%%%%%%
\draw[doubleloose] (02) to node[above]{\small $\looseid_{\tens}(\chi_f \times \looseid_f)$} (12);
\draw[doubleloose] (12) to node[above]{\small $\looseid_{\tens} (\beta \times \beta) \looseid_{\tens \times \transid}$} (22);
\draw[doubleloose] (22) to node[above]{\small $\chi_g \looseid_{\tens \times \transid}$} (32);
%%%%%%%%
\draw[doubleloose] (01) to node[above]{\small $\looseid_{\tens} \looseid_{\tens \times \transid}(\beta \times \beta \times \beta)$} (11);
\draw[doubleloose] (11) to node[above]{\small $\looseid_{\tens} (\chi_g \times \looseid_g)$} (21);
\draw[doubleloose] (21) to node[above]{\small $\chi_g \looseid_{\tens \times \transid}$} (31);
\draw[doubleloose] (31) to node[above]{\small $\looseid_g \alpha$} (41);
%%%%%%%%
\draw[doubleloose] (00) to node[above]{\small $\looseid_{\tens} (\looseid_{\tens \times \transid})(\beta \times \beta \times \beta)$} (10);
\draw[doubleloose] (10) to node[above]{\small $\alpha \looseid_{g \times g \times g}$} (20);
\draw[doubleloose] (20) to node[above]{\small $\looseid_{\tens} (\looseid_{\transid} \times \chi_g)$} (30);
\draw[doubleloose] (30) to node[above]{\small $\chi_g \looseid_{\transid \times \tens}$} (40);
%%%%%%%%
\draw[doubletighteq] (04) to (03);
\draw[doubletighteq] (14) to (13);
\draw[doubletighteq] (24) to (23);
\draw[doubletighteq] (44) to (43);
%%%%%%%%%
\draw[doubletighteq] (03) to (02);
\draw[doubletighteq] (13) to (12);
\draw[doubletighteq] (33) to (32);
\draw[doubletighteq] (43) to (42);
%%%%%%%%%
\draw[doubletighteq] (02) to (01);
\draw[doubletighteq] (22) to (21);
\draw[doubletighteq] (32) to (31);
\draw[doubletighteq] (42) to (41);
%%%%%%%%%
\draw[doubletighteq] (01) to (00);
\draw[doubletighteq] (11) to (10);
\draw[doubletighteq] (41) to (40);
%%%%%%%%%
\node at (.5,3) {\small $=$};
\node at (1.5,3.5) {\small $=$};
\node at (3,3.5) {\small $\DDownarrow ({\horl}^{-1} \verc \horr) ({\horr}^{-1} \verc \horl)$};
\node at (2,2.5) {\small $\DDownarrow \overline{\Pi^{\beta} \tightid_{\looseid_{\tens \times \transid}}}$};
\node at (2.5,1.5) {\small $=$};
\node at (.5,.5) {\small $=$};
\node at (2.5,0.5) {\small $\DDownarrow \omega^g$};
\node at (1,1.5) {\small $\DDownarrow \overline{\tightid_{\looseid_{\tens}} (\Pi^{\beta} \times  {\horr}^{-1} \verc \horl)}$};
\node at (3.5,2) {\small $=$};
\end{tikzpicture}
\end{aligned}
\end{equation*}

\begin{equation}\label{eq:br2cell}
\begin{aligned}
\begin{tikzpicture}[xscale=3, yscale=1.5]
\node (03) at (0,3) {\small $\tens(f \times f)$};
\node (13) at (1,3) {\small $\tens \tau (f \times f)$};
\node (23) at (2,3) {\small $\tens(f \times f) \tau$};
\node (33) at (3,3) {\small $f \tens \tau$};
\node (43) at (4,3) {\small $g \tens \tau$};
%%%%%%
\node (02) at (0,2) {\small $\tens(f \times f)$};
\node (12) at (1,2) {\small $f \tens$};
\node (32) at (3,2) {\small $f \tens \tau$};
\node (42) at (4,2) {\small $g \tens \tau$};
%%%%%% 
\node (01) at (0,1) {\small $\tens(f \times f)$};
\node (11) at (1,1) {\small $f \tens$};
\node (31) at (3,1) {\small $g \tens $};
\node (41) at (4,1) {\small $g \tens \tau$};
%%%%%%%
\node (00) at (0,0) {\small $\tens(f \times f)$};
\node (10) at (1,0) {\small $\tens(g \times g)$};
\node (30) at (3,0) {\small $g \tens$};
\node (40) at (4,0) {\small $g \tens \tau$};
%%%%%%%
\draw[doubleloose] (03) to node[above]{\small $\sigma \looseid_{f \times f}$} (13);
\draw[double] (13) to (23);
\draw[doubleloose] (23) to node[above]{\small $\chi \looseid_{\tau}$} (33);
\draw[doubleloose] (33) to node[above]{\small $\beta \looseid_{\tens \tau}$} (43);
%%%%
\draw[doubleloose] (02) to node[above]{\small $\chi$} (12);
\draw[doubleloose] (12) to node[above]{\small $\looseid_f \sigma$} (32);
\draw[doubleloose] (32) to node[above]{\small $\beta \looseid_{\tens \tau}$} (42);
%%%%%%
\draw[doubleloose] (01) to node[above]{\small $\chi$} (11);
\draw[doubleloose] (11) to node[above]{\small $\beta \looseid_{\tens}$} (31);
\draw[doubleloose] (31) to node[above]{\small $ \looseid_g \sigma$} (41);
%%%%%%
\draw[doubleloose] (00) to node[above]{\small $\looseid_{\tens} (\beta \times \beta)$} (10);
\draw[doubleloose] (10) to node[above]{\small $\chi $} (30);
\draw[doubleloose] (30) to node[above]{\small $\looseid_g \sigma $} (40);
%%%%%%
\draw[doubletighteq] (03) to (02);
\draw[doubletighteq] (33) to (32);
\draw[doubletighteq] (43) to (42);
%%%%%%
\draw[doubletighteq] (02) to (01);
\draw[doubletighteq] (12) to (11);
\draw[doubletighteq] (42) to (41);
%%%%%%
\draw[doubletighteq] (01) to (00);
\draw[doubletighteq] (31) to (30);
\draw[doubletighteq] (41) to (40);
%%%%%%%%
\node at (1.5,2.5) {\small $\DDownarrow u$};
\node at (3.5,2.5) {\small $=$};
\node at (.5,1.5) {\small $=$};
\node at (2.5,1.5) {\small $\iso$};
\node at (1.5,.5) {\small $\DDownarrow \Pi^{\beta}$};
\node at (3.5,.5) {\small $=$};
\end{tikzpicture}
\end{aligned}
\end{equation}
\[=\]
\begin{equation*}
\begin{aligned}
\begin{tikzpicture}[xscale=3, yscale=1.5]
\node (03) at (0,3) {\small $\tens(f \times f)$};
\node (13) at (1,3) {\small $\tens \tau (f \times f)$};
\node (23) at (2,3) {\small $\tens(f \times f) \tau$};
\node (33) at (3,3) {\small $f \tens \tau$};
\node (43) at (4,3) {\small $g \tens \tau$};
%%%%%%
\node (02) at (0,2) {\small $\tens(f \times f)$};
\node (12) at (1,2) {\small $\tens \tau (f \times f)$};
\node (22) at (2,2) {\small $\tens (f \times f) \tau $};
\node (32) at (3,2) {\small $\tens (g \times g) \tau$};
\node (42) at (4,2) {\small $g \tens \tau$};
%%%%%% 
\node (01) at (0,1) {\small $\tens(f \times f)$};
\node (11) at (1,1) {\small $\tens (g \times g)$};
\node (21) at (2,1) {\small $\tens \tau (g \times g)$};
\node (31) at (3,1) {\small $\tens (g \times g) \tau $};
\node (41) at (4,1) {\small $g \tens \tau$};
%%%%%%%
\node (00) at (0,0) {\small $\tens(f \times f)$};
\node (10) at (1,0) {\small $\tens(g \times g)$};
\node (30) at (3,0) {\small $g \tens$};
\node (40) at (4,0) {\small $g \tens \tau$};
%%%%%%%
\draw[doubleloose] (03) to node[above]{\small $\sigma \looseid_{f \times f}$} (13);
\draw[double] (13) to (23);
\draw[doubleloose] (23) to node[above]{\small $\chi \looseid_{\tau}$} (33);
\draw[doubleloose] (33) to node[above]{\small $\beta \looseid_{\tens \tau}$} (43);
%%%%
\draw[doubleloose] (02) to node[above]{\small $\sigma \looseid_{f \times f}$} (12);
\draw[double] (12) to  (22);
\draw[doubleloose] (22) to node[above]{\small $\looseid_{\tens} (\beta \times \beta) \looseid_{\tau}$} (32);
\draw[doubleloose] (32) to node[above]{\small $\chi \looseid_{\tau}$} (42);
%%%%%%
\draw[doubleloose] (01) to node[above]{\small $\looseid_{\tens} (\beta \times \beta)$} (11);
\draw[doubleloose] (11) to node[above]{\small $\sigma \looseid_{g \times g}$} (21);
\draw[double] (21) to (31);
\draw[doubleloose] (31) to node[above]{\small $ \chi \looseid_{\tau}$} (41);
%%%%%%
\draw[doubleloose] (00) to node[above]{\small $\looseid_{\tens} (\beta \times \beta)$} (10);
\draw[doubleloose] (10) to node[above]{\small $\chi $} (30);
\draw[doubleloose] (30) to node[above]{\small $\looseid_g \sigma $} (40);
%%%%%%
\draw[doubletighteq] (03) to (02);
\draw[doubletighteq] (13) to (12);
\draw[doubletighteq] (43) to (42);
%%%%%%
\draw[doubletighteq] (02) to (01);
\draw[doubletighteq] (32) to (31);
\draw[doubletighteq] (42) to (41);
%%%%%%
\draw[doubletighteq] (01) to (00);
\draw[doubletighteq] (11) to (10);
\draw[doubletighteq] (41) to (40);
%%%%%%%%
\node at (.5,2.5) {\small $=$};
\node at (2.5,2.5) {\small $\DDownarrow \overline{\Pi^{\beta} \looseid_{\tau}}$};
\node at (1.5,1.5) {\small $\iso$};
\node at (3.5,1.5) {\small $=$};
\node at (.5,.5) {\small $=$};
\node at (2.5,.5) {\small $\DDownarrow u$};
\end{tikzpicture}
\end{aligned}
\end{equation*}


As remarked above, we will actually construct a locally cubical bicategory of monoidal objects.
The monoidal 2-cells will be the loose 2-cells therein; we now define the tight 2-morphisms.

\begin{defn}\label{Def:monicon}
  Let $f, g:A \rightarrow B$ be lax monoidal 1-cells in \fB.
  A \textbf{lax monoidal icon} $\beta: f \Rightarrow g$ is a (tight) 2-morphism in \fB\ that is equipped with 3-cells
\begin{equation}
\begin{aligned}
 \begin{tikzpicture}[scale=2]
 \node (tl) at (0,1) {$I_B$};
 \node (tr) at (1,1) {$f I_A$};
 \node (bl) at (0,0) {$I_B$};
 \node (br) at (01,0) {$g I_A$}; 
 \draw[doubleloose] (tl)  to node[above]{$\iota_f$} (tr);
 \draw[doubleeq] (tl) to (bl);
 \draw[doubleloose] (bl) to node[below] {$\iota_g$}(br);
  \draw[doubletight] (tr) to node[right] {$\beta \tightid_I$}(br);
 \node at (0.5,0.5) {\footnotesize $\DDownarrow N^{\beta}$}; 
 \end{tikzpicture}
 \end{aligned}
 \hspace{.5cm}
 \begin{aligned}
  \begin{tikzpicture}[scale=2]
 \node (tl) at (0,1) {$\ten (f \times f)$};
 \node (tr) at (1,1) {$f \ten$};
 \node (bl) at (0,0) {$\ten(g \times g)$};
 \node (br) at (01,0) {$g  \ten$}; 
 \draw[doubleloose] (tl)  to node[above]{$\chi_f$} (tr);
 \draw[doubletight] (tl) to node[left]{$\tightid_{\ten} (\beta \times \beta)$} (bl);
 \draw[doubleloose] (bl) to node[below] {$\chi_g$}(br);
  \draw[doubletight] (tr) to node[right] {$\beta \tightid_{\ten}$}(br);
 \node at (0.5,0.5) {\footnotesize $\DDownarrow \Sigma^{\beta}$}; 
 \end{tikzpicture}
\end{aligned}
\end{equation}

such that the coherence axioms~\ref{eq:monicon1},~\ref{eq:monicon2},~\ref{eq:monicon3} below hold. These axioms are analogous to (TA2), (TA3) and (TA4) of~\cite{gg:ldstr-tricat}.

If $f, g:A \rightarrow B$ are oplax monoidal 1-cells an \textbf{oplax monoidal icon} $\beta: f \Rightarrow g$ is a (tight) 2-morphism \fB\ equipped with the 3-cells, 

\begin{equation}
\begin{aligned}
 \begin{tikzpicture}[xscale=-2, yscale=2]
 \node (tl) at (0,1) {$I_B$};
 \node (tr) at (1,1) {$f I_A$};
 \node (bl) at (0,0) {$I_B$};
 \node (br) at (01,0) {$g I_A$}; 
 \draw[doubleloose] (tr)  to node[above]{$\bar{\iota}_f$} (tl);
 \draw[doubleeq] (tl) to (bl);
 \draw[doubleloose] (br) to node[below] {$\bar{\iota}_g$}(bl);
  \draw[doubletight] (tr) to node[left] {$\beta \tightid_I$}(br);
 \node at (0.5,0.5) {\footnotesize $\DDownarrow \bar{N}^{\beta}$}; 
 \end{tikzpicture}
 \end{aligned}
 \hspace{.5cm}
 \begin{aligned}
  \begin{tikzpicture}[xscale=-2, yscale=2]
 \node (tl) at (0,1) {$\ten (f \times f)$};
 \node (tr) at (1,1) {$f \ten$};
 \node (bl) at (0,0) {$\ten(g \times g)$};
 \node (br) at (01,0) {$g  \ten$}; 
 \draw[doubleloose] (tr)  to node[above]{$\bar{\chi}_f$} (tl);
 \draw[doubletight] (tl) to node[right]{$\tightid_{\ten} (\beta \times \beta)$} (bl);
 \draw[doubleloose] (br) to node[below] {$\bar{\chi}_g$}(bl);
  \draw[doubletight] (tr) to node[left] {$\beta \tightid_{\ten}$}(br);
 \node at (0.5,0.5) {\footnotesize $\DDownarrow \bar{\Sigma}^{\beta}$}; 
 \end{tikzpicture}
\end{aligned}
\end{equation}

such that coherence equations analogous to~\ref{eq:monicon1},~\ref{eq:monicon2}, and~\ref{eq:monicon3} hold.
If $g,f$ are strong monoidal, by a \textbf{strong monoidal icon} we mean a lax and an oplax one whose structure morphisms are inverse to each other in the \emph{loose} direction (modulated by the up-to-isomorphism invertibility of $\chi$ and $\iota$).

A monoidal icon is {\bf braided}, {\bf sylleptic} or {\bf symmetric} when $f,g$ are braided, sylleptic or symmetric, respectively, and in addition, the coherence axiom~\eqref{eq:bricon} below holds. This axiom is analogous to (BTA1) of~\cite[p143]{mccrudden:bal-coalgb}. 

\begin{equation}\label{eq:monicon1}
\begin{aligned}
\begin{tikzpicture}[xscale=4,yscale=2]
\node (02) at (0,2){$\tens(I \times f )i_2 $};
\node (12) at (1,2){$\tens(fI_A \times f)i_2 $};
\node (22) at (2,2){$f\tens(I_A \times \transid)i_2 $};
\node (32) at (3,2){$f $};
\node (01) at (0,1){$\tens(I_B \times \transid )i_2 f$};
\node (31) at (3,1){$f $};
\node (00) at (0,0){$\tens(I_B \times \transid )i_2 g$};
\node (30) at (3,0){$g$};
\draw[doubleloose] (02) to node[above]{$\looseid_{\tens} (\iota_f \times \looseid_f) \looseid_{i_2}$} (12);
\draw[doubleloose] (12) to node[above]{$\chi \looseid_{(I_A \times \transid)i_2}$} (22);
\draw[doubleloose] (22) to node[above]{$\looseid_f l$} (32);
\draw[doubleloose] (01) to node[above]{$l \looseid_f $} (31);
\draw[doubleloose] (00) to node[above]{$l \looseid_g $} (30);
\draw[=] (02) to node[left]{} (01);
\draw[=] (32) to node[left]{} (31);
\draw[doubletight] (01) to node[left]{$\tightid_{\tens} (\tightid_I\times \beta)\tightid_{i_2}$} (00);
\draw[doubletight] (31) to node[left]{$\beta$} (30);
\node at (1.5,1.5){$\DDownarrow \gamma^f$};
\node at (1.5,0.5){$\DDownarrow \looseid_{l}\tightid_{\beta}$};
\end{tikzpicture}
\end{aligned}
\end{equation}
\[=\]
\begin{equation*}
\begin{aligned}
\begin{tikzpicture}[xscale=4,yscale=2]
\node (02) at (0,2){$\tens(I_B \times f )i_2 $};
\node (12) at (1,2){$\tens(fI_A \times f)i_2 $};
\node (22) at (2,2){$f\tens(I_A \times \transid)i_2 $};
\node (32) at (3,2){$f $};
\node (01) at (0,1){$\tens(I_B \times g )i_2 $};
\node (11) at (1,1){$\tens(gI_A \times g)i_2 $};
\node (21) at (2,1){$g\tens(I_A \times \transid)i_2 $};
\node (31) at (3,1){$g $};
\node (00) at (0,0){$\tens(I_B \times \transid )i_2 g$};
\node (30) at (3,0){$g $};
\draw[doubleloose] (02) to node[above]{$\looseid_{\tens} (\iota_f \times \looseid_f) \looseid_{i_2}$} (12);
\draw[doubleloose] (12) to node[above]{$\chi \looseid_{(I_A \times \transid)i_2}$} (22);
\draw[doubleloose] (22) to node[above]{$\looseid_f l$} (32);
\draw[doubleloose] (01) to node[above]{$\looseid_{\tens} (\iota_g \times \looseid_g \looseid_{i_2}$} (11);
\draw[doubleloose] (11) to node[above]{$\chi \looseid_{(I_A \times \transid)i_2}$} (21);
\draw[doubleloose] (21) to node[above]{$\looseid_g l$} (31);
\draw[doubleloose] (00) to node[above]{$l \looseid_g $} (30);
\draw[doubletight] (02) to node[left]{$\tightid_{\tens(I\times \transid)}\beta$} (01);
\draw[doubletight] (12) to node[right]{$\tightid (\beta \times \beta) \tightid$} (11);
\draw[doubletight] (22) to node[left]{$\beta \tightid$} (21);
\draw[doubletight] (32) to node[left]{$\beta$} (31);
\draw[doubletight] (01) to node[left]{$\tightid_{\tens} (\tightid_I\times \beta)\tightid_{i_2}$} (00);
\draw[doubletight] (31) to node[left]{$\beta$} (30);
\node at (0.5,1.5){$\DDownarrow \tightid (N^{\beta} \times \tightid_{\looseid}) \tightid$};
\node at (1.5,1.5){$\DDownarrow \Sigma^{\beta} \tightid$};
\node at (2.5,1.5){$\DDownarrow \looseid_{\beta} \tightid_{l}$};
\node at (1.5,0.5){$\DDownarrow \gamma^g$};
\end{tikzpicture}
\end{aligned}
\end{equation*}

\begin{equation}\label{eq:monicon2}
\begin{aligned}
\begin{tikzpicture}[xscale=4,yscale=2]
\node (02) at (0,2){$\tens(f \times I_B )i_1 $};
\node (12) at (1,2){$\tens(f \times fI_A)i_1 $};
\node (22) at (2,2){$f\tens(\transid \times I_A)i_1 $};
\node (32) at (3,2){$f $};
\node (01) at (0,1){$\tens(\transid \times I_B)i_1 f$};
\node (31) at (3,1){$f $};
\node (00) at (0,0){$\tens(\transid \times I_B )i_1 g$};
\node (30) at (3,0){$g $};
\draw[doubleloose] (02) to node[above]{$\looseid_{\tens} (\looseid_f \times \iota_f) \looseid_{i_1}$} (12);
\draw[doubleloose] (12) to node[above]{$\chi \looseid_{(\transid \times I_A)i_1}$} (22);
\draw[doubleloose] (22) to node[above]{$\looseid_f r$} (32);
\draw[doubleloose] (01) to node[above]{$r \looseid_f $} (31);
\draw[doubleloose] (00) to node[above]{$r \looseid_g $} (30);
\draw[=] (02) to node[left]{} (01);
\draw[=] (32) to node[left]{} (31);
\draw[doubletight] (01) to node[left]{$\tightid_{\tens (\transid \times I)i_1} \beta$} (00);
\draw[doubletight] (31) to node[left]{$\beta$} (30);
\node at (1.5,1.5){$\DDownarrow \delta^f$};
\node at (1.5,0.5){$\DDownarrow \looseid_{r}\tightid_{\beta}$};
\end{tikzpicture}
\end{aligned}
\end{equation}
\[=\]
\begin{equation*}
\begin{aligned}
\begin{tikzpicture}[xscale=4,yscale=2]
\node (02) at (0,2){$\tens(f \times I_B )i_1 $};
\node (12) at (1,2){$\tens(f \times fI_A)i_1 $};
\node (22) at (2,2){$f\tens(\transid \times I_A)i_1 $};
\node (32) at (3,2){$f$};
\node (01) at (0,1){$\tens(g \times I_B)i_1 $};
\node (11) at (1,1){$\tens(g\times gI_A )i_1 $};
\node (21) at (2,1){$g\tens(\transid \times I_A) i_1 $};
\node (31) at (3,1){$g $};
\node (00) at (0,0){$\tens(\transid \times I_B )i_1 g$};
\node (30) at (3,0){$g $};
\draw[doubleloose] (02) to node[above]{$\looseid_{\tens} (\looseid \times\iota_f) \looseid_{i_1}$} (12);
\draw[doubleloose] (12) to node[above]{$\chi \looseid_{(\transid \times I_A)i_1}$} (22);
\draw[doubleloose] (22) to node[above]{$\looseid_f r$} (32);
\draw[doubleloose] (01) to node[above]{$\looseid_{\tens} (\looseid_g \times \iota_g) \looseid_{i_1}$} (11);
\draw[doubleloose] (11) to node[above]{$\chi \looseid_{(\transid \times I_A)i_1}$} (21);
\draw[doubleloose] (21) to node[above]{$\looseid_g r$} (31);
\draw[doubleloose] (00) to node[above]{$r \looseid_g $} (30);
\draw[doubletight] (02) to node[left]{$\tightid_{\tens(\transid\times I)}\beta$} (01);
\draw[doubletight] (12) to node[right]{$\tightid (\beta \times \beta) \tightid$} (11);
\draw[doubletight] (22) to node[left]{$\beta \tightid$} (21);
\draw[doubletight] (32) to node[left]{$\beta$} (31);
\draw[doubletight] (01) to node[left]{$\tightid_{\tens} (\beta\times \tightid_I)\tightid_{i_1}$} (00);
\draw[doubletight] (31) to node[left]{$\beta$} (30);
\node at (0.5,1.5){$\DDownarrow \tightid (\tightid \times N^{\beta}) \tightid$};
\node at (1.5,1.5){$\DDownarrow \Sigma^{\beta} \tightid$};
\node at (2.5,1.5){$\DDownarrow \looseid_{\beta} \tightid_{r}$};
\node at (1.5,0.5){$\DDownarrow \delta^g$};
\end{tikzpicture}
\end{aligned}
\end{equation*}

\begin{equation}\label{eq:monicon3}
\begin{aligned}
\begin{tikzpicture}[xscale=4,yscale=2]
\node (02) at (0,2){$\tens(\tens \times \transid)(f \times f\times f)$};
\node (12) at (1,2){$\tens(f\tens \times f) $};
\node (22) at (2,2){$f\tens(\tens \times \transid)$};
\node (32) at (3,2){$f\tens(\transid \times \tens)$};
\node (01) at (0,1){$\tens(\tens \times \transid)(f \times f\times f)$};
\node (11) at (1,1){$\tens(\transid \times \tens)(f \times f\times f)$};
\node (21) at (2,1){$\tens(f \times f\tens) $};
\node (31) at (3,1){$f \tens ( \transid \times \tens)$};
\node (00) at (0,0){$\tens(\tens \times \transid)(g \times g \times g \times g$};
\node (10) at (1,0){$\tens(\transid \times \tens)(g \times g\times g)$};
\node (20) at (2,0){$\tens(g \times g\tens) $};
\node (30) at (3,0){$g \tens ( \transid \times \tens)$};
\draw[doubleloose] (02) to node[above]{$\looseid_{\tens} (\chi \times \looseid_f)$} (12);
\draw[doubleloose] (12) to node[above]{$\chi \looseid_{(\tens \times \transid)}$} (22);
\draw[doubleloose] (22) to node[above]{$\looseid_{f} \alpha$} (32);
\draw[doubleloose] (01) to node[above]{$\alpha \looseid_{f\times f \times f}$} (11);
\draw[doubleloose] (11) to node[above]{$\looseid_{\tens} (\transid_f \times \chi)$} (21);
\draw[doubleloose] (21) to node[above]{$\chi \looseid_{\transid \times \tens}$} (31);
\draw[doubleloose] (00) to node[above]{$\alpha \looseid_{g\times g \times g} $} (10);
\draw[doubleloose] (10) to node[above]{$\looseid_{\tens} (\transid_g \times \chi)$} (20);
\draw[doubleloose] (20) to node[above]{$\chi \looseid_{\transid \times \tens}$} (30);
\draw[=] (02) to node[left]{} (01);
\draw[=] (32) to node[left]{} (31);
\draw[doubletight] (01) to node[left]{$\tightid (\beta\times \beta \times \beta)$} (00);
\draw[doubletight] (11) to node {$\tightid (\beta\times \beta \times \beta)$} (10);
\draw[doubletight] (21) to node {$\tightid (\beta\times \beta \tightid)$} (20);
\draw[doubletight] (31) to node[left]{$\beta \tightid$} (30);
\node at (1.5,1.5){$\DDownarrow \omega^f$};
\node at (0.5,0.5){$\DDownarrow \tightid_{\alpha} \looseid_{\beta \times \beta \times \beta}$};
\node at (1.5,0.5){$\DDownarrow \tightid (\tightid \times \Sigma^{\beta})$};
\node at (2.5,0.5){$\DDownarrow \Sigma^{\beta} \tightid$};
\end{tikzpicture}
\end{aligned}
\end{equation}
\[=\]
\begin{equation*}
\begin{aligned}
\begin{tikzpicture}[xscale=4,yscale=2]
\node (02) at (0,2){$\tens(\tens \times \transid)(f \times f\times f)$};
\node (12) at (1,2){$\tens(f\tens \times f) $};
\node (22) at (2,2){$f\tens(\tens \times \transid)$};
\node (32) at (3,2){$f\tens(\transid \times \tens)$};
\node (01) at (0,1){$\tens(\tens \times \transid)(g \times g\times g)$};
\node (11) at (1,1){$\tens(g\tens \times g)$};
\node (21) at (2,1){$g\tens(\tens \times \transid) $};
\node (31) at (3,1){$g \tens ( \transid \times \tens)$};
\node (00) at (0,0){$\tens(\tens \times \transid)(g \times g \times g \times g$};
\node (10) at (1,0){$\tens(\transid \times \tens)(g \times g\times g)$};
\node (20) at (2,0){$\tens(g \times g\tens) $};
\node (30) at (3,0){$g \tens ( \transid \times \tens)$};
\draw[doubleloose] (02) to node[above]{$\looseid_{\tens} (\chi \times \looseid_f)$} (12);
\draw[doubleloose] (12) to node[above]{$\chi \looseid_{(\tens \times \transid)}$} (22);
\draw[doubleloose] (22) to node[above]{$\looseid_{f} \alpha$} (32);
\draw[doubleloose] (01) to node[above]{$\looseid_{\tens} (\chi \times \looseid_g)$} (11);
\draw[doubleloose] (11) to node[above]{$\chi \looseid_{(\tens \times \transid)}$} (21);
\draw[doubleloose] (21) to node[above]{$\looseid_{g} \alpha$} (31);
\draw[doubleloose] (00) to node[above]{$\alpha \looseid_{g\times g \times g} $} (10);
\draw[doubleloose] (10) to node[above]{$\looseid_{\tens} (\transid_g \times \chi)$} (20);
\draw[doubleloose] (20) to node[above]{$\chi \looseid_{\transid \times \tens}$} (30);
\draw[=] (01) to node[left]{} (00);
\draw[doubletight] (12) to node {$\tightid (\beta \tightid \times \beta)$} (11);
\draw[doubletight] (22) to node[left] {$\beta \tightid $} (21);
\draw[=] (31) to node[left]{} (30);
\draw[doubletight] (02) to node[left]{$\tightid (\beta\times \beta \times \beta)$} (01);
\draw[doubletight] (32) to node[left]{$\beta \tightid$} (31);
\node at (1.5,0.5){$\DDownarrow \omega^g$};
\node at (0.5,1.5){$\DDownarrow \tightid (\Sigma^{\beta} \times \tightid)$};
\node at (1.5,1.5){$\DDownarrow \Sigma^{\beta} \tightid$};
\node at (2.5,1.5){$\DDownarrow \looseid_{\beta} \tightid_{\alpha}$};
\end{tikzpicture}
\end{aligned}
\end{equation*}

\begin{equation}\label{eq:bricon}
\begin{aligned}
\begin{tikzpicture}[xscale=4, yscale=2]
\node (02) at (0,2){$\tens(f \times f)$};
\node (12) at (1,2){$\tens \tau (f \times f)$};
\node (22) at (2,2){$f \tens \tau$};
\node (01) at (0,1){$\tens(f \times f)$};
\node (11) at (1,1){$f \tens $};
\node (21) at (2,1){$f \tens \tau$};
\node (00) at (0,0){$\tens (g \times g)$};
\node (10) at (1,0){$g\tens $};
\node (20) at (2,0){$g \tens \tau$};
%%%%%%%%%%
\draw[doubleloose] (02) to node[above]{$\sigma \looseid_{f\times f}$} (12);
\draw[doubleloose] (12) to node[above]{$\chi \looseid_{\tau}$} (22);
\draw[doubleloose] (01) to node[above]{$\chi$} (11);
\draw[doubleloose] (11) to node[above]{$\looseid_f \sigma$} (21);
\draw[doubleloose] (00) to node[above]{$\chi$} (10);
\draw[doubleloose] (10) to node[above]{$\looseid_g \sigma$} (20);
%%%%%%%%%%%%
\draw[=] (02) to (01);
\draw[doubletight] (01) to node[left]{$\looseid_{\tens} (\beta \times \beta)$}(00);
\draw[doubletight] (11) to node[left]{$\beta \looseid_{\tens}$}(10);
\draw[=] (22) to (21);
\draw[doubletight] (21) to node[left]{$\beta \looseid_{\tens \tau}$}(20);
%%%%%%%%%%%%%
\node at (1,1.5){$\DDownarrow u$};
\node at (.5,.5){$\DDownarrow\Pi^{\beta}$};
\node at (1.5,.5) {$\DDownarrow \looseid_{\beta} \tightid_{\sigma}$};
\end{tikzpicture}
\end{aligned}
\end{equation}
\[=\]
\begin{equation*}
\begin{aligned}
\begin{tikzpicture}[xscale=4, yscale=2]
\node (02) at (0,2){$\tens(f \times f)$};
\node (12) at (1,2){$\tens \tau (f \times f)$};
\node (22) at (2,2){$f \tens \tau$};
\node (01) at (0,1){$\tens(g \times g)$};
\node (11) at (1,1){$ \tens \tau (g\times g)$};
\node (21) at (2,1){$g \tens \tau$};
\node (00) at (0,0){$\tens (g \times g)$};
\node (10) at (1,0){$g\tens $};
\node (20) at (2,0){$g \tens \tau$};
%%%%%%%%%%
\draw[doubleloose] (02) to node[above]{$\sigma \looseid_{f\times f}$} (12);
\draw[doubleloose] (12) to node[above]{$\chi \looseid_{\tau}$} (22);
\draw[doubleloose] (01) to node[above]{$\sigma \looseid_{g \times g}$} (11);
\draw[doubleloose] (11) to node[above]{$\chi \looseid_{\tau}$} (21);
\draw[doubleloose] (00) to node[above]{$\chi$} (10);
\draw[doubleloose] (10) to node[above]{$\looseid_g \sigma$} (20);
%%%%%%%%%%%%%
\draw[doubletight] (02) to node[left]{$\looseid_{\tens} (\beta \times \beta)$}(01);
\draw[=] (01) to (00);
\draw[doubletight] (12) to node[xshift=5pt]{$\looseid_{\tens \tau} \hspace{.1cm} (\beta \times \beta) $}(11);
\draw[doubletight] (22) to node[left]{$\beta \looseid_{\tens \tau}$}(21);
\draw[=] (21) to (20);
%%%%%%%%%%%%%
\node at (1,0.5){$\DDownarrow u$};
\node at (.5,1.5){$\DDownarrow \tightid_{\sigma} \looseid_{\beta \times \beta} $};
\node at (1.5,1.5) {$\DDownarrow \Pi^{\beta}$};
\end{tikzpicture}
\end{aligned}
\end{equation*}
\end{defn}

\begin{defn}
  Let $f,g,f',g': A \rightarrow B$ be lax monoidal 1-cells, let $\alpha: f \looseRightarrow{} g$, $\beta: f' \looseRightarrow{} g'$ be lax monoidal 2-cells, and let $\gamma: f \Rightarrow f'$, $\delta: g \Rightarrow g'$ be lax monoidal icons. A \textbf{lax monoidal 3-cell} is a 3-cell 
  
   \[
 \begin{tikzpicture}[scale=2]
 \node (tl) at (0,1) {$f$};
 \node (tr) at (1,1) {$g$};
 \node (bl) at (0,0) {$f'$};
 \node (br) at (01,0) {$g'$}; 
 \draw[doubleloose] (tl)  to node[above]{$\alpha$} (tr);
 \draw[doubletight] (tl) to node[left]{$\gamma$} (bl);
 \draw[doubleloose] (bl) to node[below] {$\beta$}(br);
  \draw[doubletight] (tr) to node[right] {$\delta$}(br);
 \node at (0.5,0.5) {\footnotesize $\DDownarrow \Gamma$}; 
 \end{tikzpicture}
 \]
 such that the two equalities below hold.
 
 \begin{equation}\label{eq:mon3cell1}
\begin{aligned}
 \begin{tikzpicture}[scale=2]
 \node (tm) at (0,1) {$f  I_A$};
 \node (tr) at (1,1) {$g  I_A$};
 \node (bm) at (0,0) {$f' I_A$};
 \node (br) at (01,0) {$g' I_A$}; 
 \draw[doubleloose] (tm)  to node[above]{$\alpha \looseid_I$} (tr);
 \draw[doubletight] (tm) to node[right, yshift=8] {$\gamma \tightid_I$} (bm);
 \draw[doubleloose] (bm) to node[above] {$\beta \looseid_I$}(br);
  \draw[-implies, double equal sign distance] (tr) to node[right] {$\delta \tightid_I$}(br);
 \node at (0.5,0.5) {\footnotesize $\DDownarrow \Gamma \tightid_{\looseid}$}; 
 \node (tl) at (-1,1) {$I_B$};
 \node (bl) at (-1,0) {$I_B$};
 \draw[doubleloose] (tl)  to node[above]{$\iota_f$} (tm);
 \draw[doubleeq] (tl) to (bl);
 \draw[doubleloose] (bl) to node[above]{$\iota_{f'}$}(bm);
 \node at (-0.5,.5) {\footnotesize $\DDownarrow N^{\gamma}$};
\node (bl1) at (-1,-.7){$I_B$};  
 \node (bm1) at (0,-.7) {$I_B$};
  \node (br1) at (1,-.7) {$g' I_A$}; 
 \draw[doubleloose] (bl1)  to node[above]{$\looseid_{I}$} (bm1);
 \draw[doubleloose] (bm1) to  node[above]{$\iota_{g'}$}(br1);
  \draw[doubleeq] (bl)  to (bl1);
    \draw[doubleeq] (br)  to (br1);
 \node at (0,-0.35) {\footnotesize $\DDownarrow M^{\beta}$}; 
 \end{tikzpicture}
\end{aligned}
 =
 \begin{aligned}
  \begin{tikzpicture}[scale=2]
 \node (ml) at (0,1) {$I_B$};
 \node (mm) at (1,1) {$I_B$};
 \node (bl) at (0,0) {$I_B$};
 \node (bm) at (01,0) {$I_B$}; 
 \draw[doubleloose] (ml)  to node[above]{$ \looseid_{I}$}(mm);
 \draw[doubleeq] (ml) to  (bl);
 \draw[doubleloose] (bl) to  node[above]{$ \looseid_{I}$}(bm);
 \draw[doubleeq] (mm) to (bm);
 \node at (0.5,0.5) {\footnotesize $=$}; 
 \node (tl) at (0,1.7) {$I_B$};
 \node (tm) at (1,1.7) {$f I_A$};
 \node (tr) at (2,1.7) {$g I_A$};
 \node (mr) at (2,1) {$g I_A$};
 \draw[doubleloose] (tl)  to node[above]{$\iota_f$} (tm);
 \draw[doubleloose] (tm) to node[above]{$\alpha \looseid_I$} (tr);
 \draw[doubleloose] (mm) to node[above]{$\iota_{g}$}(mr);
 \node at (1,1.35) {\footnotesize $\DDownarrow M^{\alpha}$};
  \node (br) at (2,0) {$g' I$};
 \draw[doubleloose] (bm)  to node[above]{$\iota_{g'}$} (br);
 \draw[doubletight] (mr) to  node[right]{$\delta \tightid_I$}(br);
 \draw[doubleeq] (tr) to (mr);
  \draw[doubleeq] (tl) to (ml);
 \node at (1.5,.5) {\footnotesize $\DDownarrow N^{\delta}$}; 
 \end{tikzpicture}
 \end{aligned}
\end{equation}

 \begin{equation}\label{eq:mon3cell2}
\begin{aligned}
 \begin{tikzpicture}[yscale=2, xscale=2.5]
 \node (tm) at (0,1) {$f\ten$};
 \node (tr) at (1,1) {$g \ten$};
 \node (mm) at (0,0) {$f' \ten$};
 \node (mr) at (01,0) {$g' \ten$}; 
 \draw[doubleloose] (tm)  to node[above]{$\alpha  \looseid_{\ten}$} (tr);
 \draw[doubletight] (tm) to node[right, yshift=8]{$\gamma \tightid_{\ten}$} (mm);
 \draw[doubleloose] (mm) to node[above, xshift=1pt, yshift=-1pt] {$\beta \looseid_{\ten}$}(mr);
  \draw[doubletight] (tr) to node[right] {$\delta \tightid_{\ten}$}(mr);
 \node at (0.5,0.5) {\footnotesize $\DDownarrow \Gamma \tightid$}; 
 \node (tl) at (-1,1) {$\ten  (f\times f)$};
 \node (ml) at (-1,0) {$\ten  (f'\times f')$};
 \draw[doubleloose] (tl)  to node[above]{$\chi^f$} (tm);
 \draw[doubletight] (tl) to node[left]{$\tightid_{\ten} (\gamma \times \gamma)$} (ml);
 \draw[doubleloose] (ml) to node[above]{$\chi^{f'}$}(mm);
 \node at (-0.5,0.5) {\footnotesize $\DDownarrow \Sigma^{\gamma}$};
 \node (bl) at (-1,-.7) {$\ten (f'\times f')$};
  \node (bm) at (0,-.7) {$\ten (g'\times g')$};
  \node (br) at (1,-.7) {$g' \ten$};
  \draw[doubleeq] (ml) to (bl);
 \draw[doubleloose] (bl)  to node[above]{$\looseid_{\ten} (\beta \times \beta)$} (bm);
 \draw[doubleloose] (bm) to  node[above]{$\chi^{g'}$}(br);
   \draw[doubleeq] (mr) to (br);
 \node at (0,-0.35) {\footnotesize $\DDownarrow \Pi^{\beta}$}; 
 \end{tikzpicture}
\end{aligned}
 =
 \begin{aligned}
  \begin{tikzpicture}[yscale=2, xscale=2.5]
 \node (ml) at (0,1) {$\ten (f\times f)$};
 \node (mm) at (1,1) {$\ten (g\times g)$};
 \node (bl) at (0,0) {$\ten (f'\times f')$};
 \node (bm) at (01,0) {$\ten (g'\times g')$}; 
 \draw[doubleloose] (ml)  to node[above]{$\looseid_{\ten} (\alpha \times \alpha)$} (mm);
 \draw[doubletight] (ml) to node[left]{$\tightid_{\ten} (\gamma \times \gamma)$}  (bl);
 \draw[doubleloose] (bl) to node [below] {$\looseid_{\ten} (\beta \times \beta)$} (bm);
  \draw[doubletight] (mm) to node[above] {$\tightid_{\ten} (\delta \times \delta)$} (bm);
 \node at (0.5,0.5) {\footnotesize $\DDownarrow \tightid (\Gamma \times \Gamma)$}; 
 \node (tl) at (0,1.7) {$ \ten (f \times f$)};
 \node (tm) at (1,1.7) {$f \ten$};
 \node (tr) at (2,1.7) {$g \ten$};
   \node (mr) at (2,1) {$g \ten$};
   \node(br) at (2,0) {$g' \ten$};
 \draw[doubleloose] (tl)  to node[above]{$\chi^f$} (tm);
 \draw[doubleloose] (tm) to node[above]{$\alpha \looseid_{\ten}$} (tr);
 \draw[doubletick] (mm) to node[above]{$\chi^{g}$}(mr);
 \node at (1,1.35) {\footnotesize $\DDownarrow \Pi^{\alpha}$};
 \draw[doubleloose] (bm)  to node[below]{$\chi^{g'}$} (br);
 \draw[doubletight] (mr) to  node[right]{$\delta \tightid_{\ten}$}(br);
 \draw[doubleeq] (tr) to (mr);
  \draw[doubleeq] (tl) to (ml);
 \node at (1.5,.5) {\footnotesize $\DDownarrow \Sigma^{\delta}$}; 
 \end{tikzpicture}
 \end{aligned}
\end{equation}

  Let $f,g,f',g', \alpha, \beta, \gamma$ and $\delta: g \Rightarrow g'$ be oplax monoidal 1-cells, 2-cells, and icons, respectively. An \textbf{oplax monoidal 3-cell} is a 3-cell  $\Gamma$ as depicted above, such that two equalities analogous to~\ref{eq:mon3cell1} and~\ref{eq"mon3cell2} hold. $\Gamma$ is {\bf strong monoidal} if it is lax monoidal and oplax monoidal.
Let $f,g,f',g', \alpha, \beta, \gamma$ and $\delta: g \Rightarrow g'$ be braided, sylleptic, or symmetric monoidal 1-cells, 2-cells, and icons, respectively. A \textbf{braided, sylleptic, or symmetric monoidal 3-cell} $\Gamma$ as depicted above, is simply a monoidal 3-cell. 
\end{defn}

We will show that monoidal objects, lax monoidal 1-cells, lax monoidal 2-cells, lax monoidal icons, and lax monoidal 3-cells in a locally cubical bicategory \fB\ form a locally cubical bicategory $\cM on_l\cB$. If we consider oplax or strong cells instead, we obtain the locally cubical bicategory $\cM on_o\cB$ or $\cM on_s\cB$, respectively. Braided, sylleptic and symmetric cells form  locally cubical bicategories $\cB r \cM on_w\cB$, $\cS yl \cM on_w\cB$, and $\cS ym \cM on_w\cB$. Here, $w \in \{l,o,s\}$ denotes whether the cells are lax, oplax, or strong.

\begin{prop}\label{prop:dc}
Let $A,B$ be monoidal objects in a locally cubical bicategory. The hom-spaces $\cM on_w\cB (A,B)$, $\cB r \cM on_w\cB(A,B)$, $\cS yl \cM on_w\cB(A,B)$, and $\cS ym \cM on_w\cB(A,B)$ are double categories.
\end{prop}

\begin{proof}
First we show that 1-cells and icons in the respective hom-spaces form a category. For each lax monoidal 1-cell $f:A \rightarrow B$, the identity icon $\tightid_f$ is a lax monoidal icon with the 3-cells $N^{\tightid_f} := \tightid_{\iota_f}$ and $\Sigma^{\tightid_f} := \tightid_{\chi_f}$. This is well-defined, because the functor ``$\comp$" preserves tight identities. The coherence equations are trivially satisfied.  For each two lax monoidal 1-cells $f,g$ and lax monoidal icons $\alpha, \beta: f \Rightarrow g$, the composite icon $\beta \circ \alpha$ can be equiped with the lax monoidal structure given by the composites $N^{\beta \verc \alpha} := N^{\beta} \verc N^{\alpha}$ and $\Sigma^{\beta \verc \alpha} : = \Sigma^{\beta} \verc \Sigma^{\alpha}$.  We have a strict interchange law between $\verc$ and $\comp$, induced by functoriality of $\comp$, so these 3-cells are well-defined. The coherence conditions~\eqref{eq:monicon1}--\eqref{eq:monicon3} hold by componentwise application of the coherence equalities for $N^{\beta \verc \alpha}$ and $\Sigma^{\beta \verc \alpha}$. The same argument holds for oplax 1-cells and icons. For strong 1-cells and icons we need to verify that the lax structure cells $N^{\tightid_f}$, $N^{\beta \verc \alpha}, \Sigma^{\tightid_f}$, and $\Sigma^{\beta \verc \alpha} $ are inverse in the loose direction to their oplax counterparts. For $N^{\tightid_f}$ and $\Sigma^{\tightid_f}$, this follows from functoriality of ``$\horc$". For $N^{\beta \verc \alpha}$ and $\Sigma^{\beta \verc \alpha}$, this follows from the the fact that the statement is true for their components combined with the exchange law between ``$\horc$" and ``$\verc$" and strictness of ``$\verc$".
When $f$ and $g$ are braided, sylleptic or symmetric, the same data satisfies the coherence equation for braided monoidal icons.

We also need to show that lax monoidal 2-cells and monoidal 3-cells form a category. For every lax monoidal 2-cell $\alpha: f \looseRightarrow{} g$, the identity 3-cell $\tightid_{\alpha}$ in $\cB$  is lax monoidal. The required two equations~\ref{eq:mon3cell1},~\ref{eq:mon3cell2} are trivially satisfied.
For any two monoidal 3-cells $L: \alpha \Rightthreecell \beta$, $K:\beta \Rightthreecell \gamma$, the composition $K \circ L$ in $\cB$ is a monoidal 3-cell. The equations for monoidal 3-cells hold by sequential application of the respective equations for $L$ and $K$. The same is true for oplax monoidal 2-cells and 3-cells, and hence for strong monoidal 2-cells and 3-cells. As braided, sylleptic, and symmetric monoidal 3-cells are not required to satisfy additional data, it follows that braided, sylleptic, or symmetric monoidal 2-cells and 3-cells form a category.

Now we describe the loose structure; we need to show that $\horc$ and $\looseid$ are well-defined as the functors which give the loose structure in the new double category. To see this, recall that $\horc$ and $\looseid$ correspond to the functors $\odot$ and $U$, respectively, given  in Definition~\ref{defn:dblcat}.
Let $f$ be a lax monoidal 1-cell. The loose identity 2-cell $\looseid_f$ is a lax monoidal 2-cell with monoidal structure given by the composition of coherence cells $\horl$, $\horr$.

\begin{equation}
M^{\looseid_f}:=
\begin{aligned}
 \begin{tikzpicture}[yscale=1.5, xscale=3]
 \node (tl) at (0,1) {$I_B$};
\node (tr) at (1,1) {$f   I_A$};
 \node (tm) at (.5,1) {$f  I_A$};
 \node (bl) at (0,0) {$I_B$};
 \node (bm) at (.5,0) {$I_B$};
 \node (br) at (1,0) {$f I_A$}; 
 \draw[doubleloose] (tl)  to node[above]{$\iota_f$} (tm);
  \draw[doubleloose] (tm)  to node[above]{$\looseid_f \looseid_I$} (tr);
 \draw[doubleeq] (tl) to (bl);
  \draw[doubleloose] (bl) to node[below] {$\looseid_I$}(bm);
 \draw[doubleloose] (bm) to node[below] {$\iota_f$}(br);
  \draw[=] (tr) to (br);
 \node at (0.5,0.5) {\footnotesize $\DDownarrow$ $\iso $}; 
 \end{tikzpicture}
 \end{aligned}
 \hspace{.5cm}
 \Pi^{\looseid_f}:=
 \begin{aligned}
  \begin{tikzpicture}[yscale=1.5, xscale=5]
 \node (tl) at (0,1) {$\ten  (f \times f)$};
 \node (tr) at (1,1) {$f  \ten$};
 \node (bl) at (0,0) {$\ten  (f \times f)$};
 \node (br) at (01,0) {$f \ten$}; 
 \node(tm) at (.5,1) {$f \ten$};
 \node (bm) at (.5,0) {$\ten (f\times f)$};
 \draw[doubleloose] (tl)  to node[above]{$\chi_f $} (tm);
  \draw[doubleloose] (tm)  to node[above]{$\looseid_f \looseid{\ten}$} (tr);
 \draw[=] (tl) to (bl);
  \draw[doubleloose] (bl) to node[below] {$\looseid_{\ten}(\looseid_f \times \looseid_f)$}(bm);
 \draw[doubleloose] (bm) to node[below] {$\chi_f$}(br);
  \draw[=] (tr) to (br);
 \node at (0.5,0.5) {\footnotesize $ \DDownarrow$ $\iso$}; 
 \end{tikzpicture}
\end{aligned}
\end{equation}

The conditions for monoidal 3-cells follow from the naturality conditions of the coherence cells $\horl$ and $\horr$. 
For every monoidal icon $\gamma$, the loose identity 3-cell $\looseid_{\gamma}$ is a monoidal 3-cell. The loose source and target 2-cells of $\Gamma$, $\alpha$ and $\beta$, are loose identities, hence $M^{\alpha}$ and $M^{\beta}$ are the coherence cells defined above. Therefore, the coherence condition holds by naturality of $\horl$ and $\horr$.

Let $\alpha:f \looseRightarrow{} g$ and $\beta: g \looseRightarrow{} h$ be two lax monoidal 2-cells. Their composition $\alpha \horc \beta$ is lax monoidal with the following structure 3-cells.

\begin{equation}
M^{\alpha \horc \beta} := 
\begin{aligned}
 \begin{tikzpicture}[yscale=1.5, xscale=3]
 \node (tl) at (0,1) {$I_B$};
\node (tr) at (1,1) {$g   I_A$};
 \node (tm) at (.5,1) {$f  I_A$};
 \node (bl) at (0,0) {$I_B$};
 \node (bm) at (0.5,0) {$I_B$};
 \node (br) at (1,0) {$g I_A$}; 
 \node (trr) at (1.5,1) {$h I_A$};
 \node (brr) at (1.5,0) {$h I_A$};
 \node (bbr) at (1.5,-1) {$hI_A$};
  \node (bbm1) at (.5,-1) {$I_B$};
 \node (bbm) at (1,-1) {$I_B$};
 \node(bbl) at (0,-1) {$I_B$};
    \draw[doubleloose] (tm) to[in=120, out=60] node[above]{$(\alpha \horc \beta)\looseid_{I}$} (trr);
 \draw[doubletight] (brr) to node[right] {} (bbr);
 \draw[doubleeq] (bl) to (bbl);
  \draw[doubleloose] (bbl) to node [above]{$\looseid_{I}$} (bbm1);
    \draw[doubleloose] (bbm1) to node [above]{$\looseid_{I}$} (bbm);
 \draw[doubleloose] (bbm) to node [above]{$\iota_{h}$} (bbr);
 \draw[doubleloose] (tr) to node[above]{$\beta \looseid_I$} (trr);
  \draw[doubleloose] (br) to node[above]{$\beta \looseid_I$}(brr);
  \draw[doubleeq] (trr) to (brr);
 \draw[doubleloose] (tl)  to node[above]{$\iota_f$} (tm);
  \draw[doubleloose] (tm)  to node[above]{$\alpha \looseid_I$} (tr);
 \draw[doubleeq] (tl) to (bl);
  \draw[doubleloose] (bl) to node[below] {$\looseid_I$}(bm);
 \draw[doubleloose] (bm) to node[below] {$\iota_g$}(br);
 \draw[doubleloose] (bbl) to[in=220, out=-60] node[below]{$\looseid_I$} (bbm);
  \draw[doubleeq] (tr) to (br);
   \draw[doubleeq] (bm) to (bbm1);
 \node at (0.5,0.5) {\footnotesize $M^{\alpha} \DDownarrow  $}; 
  \node at (1,-.5) {\footnotesize $M^{\beta} \DDownarrow $}; 
 \node at (1.25,.5) {\footnotesize $=$}; 
 \node at (1,1.25) {$\iso$};
 \node at (0.5,-1.25) {$\iso$};
 \end{tikzpicture}
 \end{aligned}
\end{equation}
\begin{equation}
 \Pi^{\alpha \horc \beta}:=
 \begin{aligned}
  \begin{tikzpicture}[yscale=1.5, xscale=5]
 \node (tl) at (0,1) {$\ten  (f \times f)$};
 \node (tr) at (1,1) {$g \ten$};
 \node (bl) at (0,0) {$\ten  (f \times f)$};
 \node (br) at (01,0) {$g \ten$}; 
 \node(tm) at (.5,1) {$f \ten$};
 \node (bm) at (.5,0) {$\ten (g\times g)$};
 \node (trr) at (1.5,1) {$h \ten$};
  \node (brr) at (1.5,0) {$h \ten$};
  \node (bbl) at (0,-1) {$\ten (f \times f)$};
  \node (bbm) at (.5,-1) {$\ten (g \times g)$}; 
  \node (bbr) at (1,-1) {$\ten (h \times h)$};
  \node (bbrr) at (1.5,-1) {$h \ten $};
 \draw[doubleloose] (tl)  to node[above]{$\chi_f $} (tm);
  \draw[doubleloose] (tm)  to node[above]{$\alpha \looseid_{\ten}$} (tr);
 \draw[doubleeq] (tl) to (bl);
  \draw[doubleloose] (bl) to node[below] {$\looseid_{\ten} (\alpha \times \alpha)$}(bm);
 \draw[doubleloose] (bm) to node[below] {$\chi_g$}(br);
  \draw[doubleeq] (tr) to (br); 
 \draw[doubleeq] (trr) to (brr);
 \draw[doubleloose] (tr) to node[above]{$\beta \looseid_{\ten}$} (trr);
 \draw[doubleloose] (br) to node[above]{$\beta \looseid_{\ten}$} (brr);
 \draw[doubleloose] (bbr) to node[above]{$\chi_h$} (bbrr);
 \draw[doubleeq] (bl) to (bbl);
 \draw[doubleeq] (bm) to (bbm);
 \draw[doubleeq] (brr) to (bbrr);
 \draw[doubleloose] (bbl) to node[above]{$\looseid_{\ten} (\alpha \times \alpha)$} (bbm);
  \draw[doubleloose] (bbm) to node[above]{$\looseid_{\ten} (\beta \times \beta)$} (bbr);
   \draw[doubleloose] (tm) to[in=120, out=60] node[above]{$(\alpha \horc \beta)\looseid_{\ten}$} (trr);
   \draw[doubleloose] (bbl) to[in=220, out=-60] node[below]{$\looseid_{\ten} \comp (\alpha \horc \beta)\times (\alpha \horc \beta)$} (bbr);
    \node at (0.5,0.5) {\footnotesize $\DDownarrow  \Pi^{\alpha}$};
  \node at (1.25,0.5) {\footnotesize $=$};
  \node at (0.25,-.5) {\footnotesize $=$};
  \node at (1,-.5) {\footnotesize $\DDownarrow  \Pi^{\beta}$};
  \node at (1,1.2) {$\iso$};
 \node at (.5,-1.2) {$\iso$};
 \end{tikzpicture}
\end{aligned}
\end{equation}


The coherence equations are satisfied by sequential application of the respective equation for $\alpha$ and $\beta$, applications of the exchange law between loose and tight composition, together with simple manipulations of coherence cells.

Let $\Gamma$ and $\Delta$ be lax monoidal 3-cells. Their composite $\Gamma \horc \Delta$ is again lax monoidal. Again, the conditions for monoidal 3-cells follow directly from the conditions on the monoidal 3-cells $\Gamma$ and $\Delta$, applications of the exchange law between loose and tight composition, and simple manipulations of coherence cells.
By analogous arguments, one can show that the images of $\horc$ and $\looseid$ on oplax cells are again well-defined as oplax monoidal cells. For  the strong monoidal cells, we need to prove that $M^{\looseid_f}, \Pi^{\looseid_f}, M^{\alpha \horc \beta}$, and $\Pi^{\alpha \horc \beta}$ correspond as mates to their respective oplax counterparts under the adjoint equivalence structure on $\chi$ and $\iota$. For $M^{\looseid_f}$ and$ \Pi^{\looseid_f}$, this follows from manipulation of coherence cells, together with the adjoint equivalences $\iota \dashv \bar{\iota}$ and $\chi \dashv \bar{\chi}$. For $M^{\alpha \horc \beta}$, and $\Pi^{\alpha \horc \beta}$ one needs to insert an extra instance of $(\epsilon \horc \looseid) \verc (\looseid \horc \eta) = \verc$ in between $M_{\alpha}$ and $M^{\beta}$, and in between $\Pi^{\alpha}$ and $\Pi^{\beta}$, respectively. The result then follows from manipulations with coherence cells, $\epsilon$ and $\eta$.

Let $f$ be a braided, sylleptic or symmetric monoidal 1-cell. The loose identity $\looseid_f$ is a braided, sylleptic or symmetric monoidal 2-cell, respectively, as the coherence equation~\ref{eq:br2cell} merely states that the 3-cell $u$ pasted with coherence 3-cells equals itself. Let $\alpha, \beta$ be braided, sylleptic, or symmetric monoidal 2-cells, the loose composition $\alpha \horc \beta$ is braided, sylleptic, or symmetric monoidal, respectively. One can verify that~\ref{eq:br2cell} holds by applying the exchange law between loose and tight composition, manipulation of coherence cells, and sequential application of the respective equations for $\alpha$ and $\beta$.  Braided, sylleptic and symmetric monoidal 3-cells are simply monoidal 3-cells; therefore, it follows that the images of $\horc$ and $\looseid$ of braided, sylleptic, or symmetric monoidal cells are well-defined in $\cB r \cM on_w\cB(A,B)$, $\cS yl \cM on_w\cB(A,B)$, and $\cS ym \cM on_w\cB(A,B)$, respectively.

Functoriality of $\horc$ and $\looseid$ in $\cM on \cB(A,B)$ follow from their functoriality in $\cB(A,B)$. 
The unitality and associativity constraints $\hora$, $\horl$, and $\horr$ are lax and oplax monoidal 3-cells, depending on their source and target cells. The conditions for lax and oplax monoidal 3-cells in this case amount to 3-cells pasted together with coherence cells being equal to themselves. This follows directly from coherence of the functor $\horc$. It follows that $\cM on_l\cB(A,B)$ and $\cM on_o\cB(A,B)$ are double categories. As a direct consequence, $\cM on_s\cB(A,B)$ is also a double category.
Since braided, symmetric, or sylleptic monoidal 3-cells require no extra data, it follows that $\cB r \cM on_w\cB(A,B)$, $\cS yl \cM on_w\cB(A,B)$, and $\cS ym \cM on_w\cB(A,B)$ are double categories for $w \in \{l,o,s\}$.
\end{proof}



\begin{thm}\label{thm:lcbc}
  Monoidal objects, lax monoidal 1-cells, lax monoidal 2-cells, lax monoidal icons, and lax monoidal 3-cells in a locally cubical bicategory \fB\ form a locally cubical bicategory $\cM on_l\cB$ if the unit $\compI_A$ is a lax monoidal 1-cell for each object $A \in$ \fB.
  \begin{anfxnote}[author=MS]{Condition}
    This condition doesn't even parse to me; it only makes sense to say that $\compI_A$ is lax monoidal once $A$ is given a monoidal strurcture, and in that case it ought to be automatic.
  \end{anfxnote}
  If we consider oplax or strong cells instead, we obtain the locally cubical bicategory $\cM on_o\cB$ or $\cM on_s\cB$, respectively. When the objects and cells are braided, sylleptic or symmetric,  we obtain the locally cubical bicategories $\cB r \cM on_w\cB$, $\cS yl \cM on_w\cB$, and $\cS ym \cM on_w\cB$, where $w \in \{l,o,s\}$ denotes whether the cells are lax, oplax, or strong.
\end{thm}

\begin{proof}
We have established that the respective homsets $\cM on_wB(A,B)$, $\cB r \cM on_w\cB$, $\cS yl \cM on_w\cB$, and $\cS ym \cM on_w\cB$ for $w \in \{l,o,s\}$ form double categories in Proposition \ref{prop:dc}. 

We need to check that the unit $I^{\comp}_A$ is a well-defined functor from the trivial double category to the respective hom-categories of lax, oplax and strong monoidal cells, as well as braided, sylleptic and symmetric ones. 
By assumption, the unit 1-cells $\compI_A$ are monoidal for all objects $A \in$ \fB. By functoriality of $\compI$, the loose 2-cell $\compI_{\compI_A}$ is isomorphic to the loose identity $\looseid_{\compI_A}$. This isomorphism gives rise to the following lax monoidal structure on $\compI_A$: 

\begin{equation}
M^{\compI_{\compI_A}}:=
\begin{aligned}
 \begin{tikzpicture}[yscale=1.5, xscale=3]
 \node (tl) at (0,1) {$I_A$};
\node (tr) at (1,1) {$\compI_A   I_A$};
 \node (tm) at (.5,1) {$\compI_A  I_A$};
 \node (bl) at (0,0) {$I_A$};
 \node (bm) at (.5,0) {$I_A$};
 \node (br) at (1,0) {$\compI_A I_A$}; 
 \draw[doubleloose] (tl)  to node[above]{$\iota_{\compI_A}$} (tm);
  \draw[doubleloose] (tm) to[in=120, out=60] node[above] {$\compI_{\compI_A} \looseid_I$} (tr);
 \draw[doubleloose] (tm)  to node[below]{$\looseid_{\compI_A} \looseid_I$} (tr);
 \draw[doubleeq] (tl) to (bl);
  \draw[doubleloose] (bl) to node[below] {$\looseid_I$}(bm);
 \draw[doubleloose] (bm) to node[below] {$\iota_{\compI_A}$}(br);
  \draw[=] (tr) to (br);
 \node at (0.5,0.5) {\footnotesize $\DDownarrow$ $\iso $}; 
   \node at (0.75,1.2) {\footnotesize $ \DDownarrow$ $\iso$}; 
 \end{tikzpicture}
 \end{aligned}
\hspace{.5cm}
 \Pi^{\compI_{\compI_A}}:=
 \begin{aligned}
  \begin{tikzpicture}[yscale=1.5, xscale=5]
 \node (tl) at (0,1) {$\ten  (\compI_A \times \compI_A)$};
 \node (tr) at (1,1) {$\compI_A  \ten$};
 \node (bl) at (0,0) {$\ten  (\compI_A \times \compI_A)$};
 \node (br) at (01,0) {$\compI_A \ten$}; 
 \node(tm) at (.5,1) {$\compI_A \ten$};
 \node (bm) at (.5,0) {$\ten (\compI_A \times \compI_A)$};
 \draw[doubleloose] (tl)  to node[above]{$\chi_{\compI_A} $} (tm);
 \draw[doubleloose] (tm) to[in=120, out=60] node[above]{$\compI_{\compI_A} \looseid_{\ten}$} (tr);
  \draw[doubleloose] (tm)  to node[below]{$\looseid_{\compI_A} \looseid{\ten}$} (tr);
 \draw[=] (tl) to (bl);
  \draw[doubleloose] (bl) to node[above] {$\looseid_{\ten}(\looseid_{\compI_A} \times \looseid_{\compI_A})$}(bm);
          \draw[doubleloose] (bl) to[in=220, out=-60] node[below]{\small $\looseid_{\ten}(\compI_{\compI_A} \times \compI_{\compI_A})$}(bm); 
 \draw[doubleloose] (bm) to node[above] {$\chi_{\compI_A}$}(br);    
  \draw[=] (tr) to (br);
 \node at (0.5,0.5) {\footnotesize $ \DDownarrow$ $\iso$}; 
  \node at (0.75,1.2) {\footnotesize $ \DDownarrow$ $\iso$}; 
    \node at (0.25,-.1) {\footnotesize $ \DDownarrow$ $\iso$}; 
 \end{tikzpicture}
\end{aligned}
\end{equation}

Coherence equations~\ref{eq:mon2cell1}, \ref{eq:mon2cell2}, \ref{eq:mon2cell3}, and \ref{eq:br2cell} hold by simple manipulations of the isomorphisms. Note that this makes $\compI_{\compI_A}$ a braided, sylleptic, or symmetric monoidal 2-cells if $\compI_A$ is braided, sylleptic, or symmetric, respectively.
An analogous construction makes $\compI_{\compI_A}$ into an oplax monoidal 2-cell. It follows directly from the fact that $M^{\looseid_{\compI_A}}$ and $\overline{M^{\looseid_{\compI_A}}}$ are mates under the adjoint equivalence structure on $\iota$, that the same is true for $M^{\compI_{\compI_A}}$ and $\overline{M^{\compI_{\compI_A}}}$. Likewise, $\Pi^{\compI_{\compI_A}}$ and $\overline{\Pi^{\compI_{\compI_A}}}$ are mates under the adjoint equivalence structure on $\chi$.
By functoriality of $\compI$, the image of $\compI$ on the tight 2-cell and 3-cell are $\tightid_{\compI_A}$ and $\tightid_{\compI_{\compI_A}}$, respectively. These cells are braided, symmetric or sylleptic lax, oplax, or strong monoidal, depending on $A$. It follows that $I^{\comp}_A$ is a well-defined functor from the trivial double category to the respective hom-categories of lax, oplax and strong monoidal cells, as well as braided, sylleptic and symmetric ones.  

Next, we need to show that monoidal structure is preserved by the composition along a 0-cell boundary.
For any two lax monoidal 1-cells $f:A \rightarrow B$, $g:B \rightarrow C$, the composite $g \comp f$ is monoidal with $\chi^{g \comp f}$ and $\iota^{g \comp f}$ defined below. 
\begin{align}
\chi_{g \comp f} &: \hspace{.5cm} &\otimes (gf \times gf) \xlooseRightarrow{\chi_g \looseid_{f \times f}} g \otimes (f \times f) \xlooseRightarrow{\looseid_g \chi_f} gf \tens \\
\iota_{g \comp f} & : \hspace{.5cm} &I_C \xlooseRightarrow{\iota_g} g I_B \xlooseRightarrow{\looseid_g \iota_f} gfI_A
\end{align}

The structure 3-cell $\gamma$ is defined as

\begin{equation}
\gamma^{g \comp f} := 
\begin{aligned}
 \begin{tikzpicture}[yscale=1.5, xscale=5]
 \node (t0) at (0,2) {\small $\tens(I_C \times gf)i_2$};
 \node (t1) at (.5,2) {\small $\tens(gI_B \times gf)i_2$};
\node (t2) at (1,2) {\small $g \tens (I_B \times f)i_2$};
 \node (t3) at (1.5,2) {\small $g \tens (fI_A \times f)i_2$};
  \node (t4) at (2,2) {\small $gf \tens (I_A \times \transid)i_2$};
 \node (t5) at (2.5,2) {\small $gf$};
  \node (m0) at (0,1) {\small $\tens(I_C \times g)i_2f$};
 \node (m1) at (.5,1) {\small $\tens(gI_B \times g)i_2f$};
\node (m2) at (1,1) {\small $g \tens (I_B \times \transid)i_2f$};
 \node (m5) at (2.5,1) {\small $gf$};
  \node (b0) at (0,0) {\small $\tens(I_C \times \transid)i_2 gf$};
 \node (b5) at (2.5,0) {\small $gf$};
 %%%%%%%%%%%%%%%%
  \draw[doubleloose] (t0) to[in=120, out=60] node[above]{$\looseid_{\tens} (\iota_{gf} \times \looseid_{gf})\looseid_{(I_A \times \transid)i_2} \horc \chi_{gf}\looseid_{i_2}$} (t4);
  %%%%%%%%%%%%%%%%
 \draw[doubleloose] (t0)  to node[above]{\small $\looseid_{\tens}(\iota_g \times \looseid_{gf})\looseid_{i_2}$} (t1);
  \draw[doubleloose] (t1)  to node[above]{\small $\chi_g\looseid_{I_A \times f}\looseid_{i_2}$} (t2);
\draw[doubleloose] (t2) to node[above]{\small $\looseid_{g\tens }(\iota_f \times \looseid_{f})\looseid_{i_2}$} (t3);
  \draw[doubleloose] (t3) to node[above]{\small $\looseid_g \chi_f \looseid_{(I_A \times \transid)i_2}$}(t4);
  \draw[doubleloose] (t4) to node[above]{\small $\looseid_{gf}l_I$}(t5);
  %%%%%%%%%%%%%%%%%%
  \draw[doubleloose] (m0)  to node[above]{\small $\looseid_{\tens}(\iota_g \times \looseid_{g})\looseid_f$} (m1);
  \draw[doubleloose] (m1)  to node[above]{\small $\chi_g\looseid_{(I_B \times \transid)i_2 f}$} (m2);
   \draw[doubleloose] (m2) to node[below]{\small $ \looseid_g l \looseid_f$}(m5); 
   %%%%%%%%%%%%%%%%%
    \draw[doubleloose] (b0) to node[above]{\small $ l \looseid_g \looseid_f$}(b5); 
       \draw[doubleloose] (b0) to[in=220, out=-60] node[above]{\small $l \looseid_{gf}$}(b5); 
    %%%%%%%%
  \draw[doubleeq] (t0) to (m0);
    \draw[doubleeq] (t2) to (m2);
  \draw[doubleeq] (t5) to (m5);
  \draw[doubleeq] (m0) to (b0);
    \draw[doubleeq] (m5) to (b5);
    \node at (.5,1.5) {\footnotesize $=$}; 
   \node at (1.75,1.5) {\footnotesize $\overline{ \tightid_g \gamma^f}$}; 
   \node at (1.25,.5) {\footnotesize $\overline{  \gamma^g \tightid_{\looseid}}$}; 
      \node at (1,2.35) {\footnotesize $\iso$}; 
  \node at (1.25,-.35) {\footnotesize $\iso$}; 
 \end{tikzpicture}
 \end{aligned}
\end{equation}

The 3-cells $\delta^{g \comp f}$ and $\omega^{g \comp f}$ are defined similarly, and so is $u^{g \comp f}$ when $g, f$ are braided monoidal 1-cells. 

Let $f,h: A \rightarrow B $ and $g,i: B \rightarrow C$ be lax monoidal 1-cells and let $\alpha: f \rightarrow h$, $\beta: g \rightarrow i$ be lax monoidal 2-cells, the composite $\beta \comp \alpha$ is lax monoidal with the following structure 3-cells

\begin{equation}
M^{\beta \comp \alpha} := 
\begin{aligned}
 \begin{tikzpicture}[yscale=1.5, xscale=4]
 \node (t0) at (0,1) {$I_C$};
\node (t2) at (1,1) {$g f  I_A$};
 \node (t4) at (2,1) {$i h I_A$};
 \node (m0) at (0,0) {$I_C$};
 \node (m1) at (.5,0) {$g I_B$}; 
\node (m2) at (1,0) {$h I_B$};
\node (m3) at (1.5,0) {$h f I_A$};
\node (m4) at (2,0) {$h k I_A$};
 \node (b0) at (0,-1) {$I_C$};
 \node (b1) at (.5,-1) {$I_C$}; 
\node (b2) at (1,-1) {$h I_B$};
\node (b3) at (1.5,-1) {$h I_B$};
\node (b4) at (2,-1) {$h k I_A$};
\node (bb0) at (0,-2) {$I_C$};
 \node(bb2) at (1,-2) {$I_C$};
   \node(bb4) at (2,-2) {$hk I_A$};
 \draw[doubleloose] (t0)  to node[above]{$\iota_{g \comp f}$} (t2);
  \draw[doubleloose] (t2)  to node[above]{$\beta \comp \alpha$} (t4);
\draw[doubleloose] (m0) to node[above]{$\iota_g $} (m1);
  \draw[doubleloose] (m1) to node[above]{$\beta \looseid_{I}$}(m2);
  \draw[doubleloose] (m2) to node[above]{$\looseid_h \iota_f $}(m3);
  \draw[doubleloose] (m3) to node[above]{$\looseid_h \alpha \looseid_{I}$}(m4);
  %%%%%%%%%%%%
  \draw[doubleloose] (b0) to node[above]{$\looseid$} (b1);
  \draw[doubleloose] (b1) to node[above]{$\iota_h$} (b2);
  \draw[doubleloose] (b2) to node[above]{$\looseid_h \looseid_{I}$}(b3);
  \draw[doubleloose] (b3) to node[above]{$\looseid_h \iota_k$}(b4);
 %%%%%%%%%
  \draw[doubleloose] (bb0)  to node[above]{$\looseid_{I}$} (bb2);
  \draw[doubleloose] (bb2)  to node[above]{$\iota_{hk}$} (bb4); 
   %%%%%%%%%% 
  \draw[doubleeq] (t0) to (m0);  
   \draw[doubleeq] (m0) to (b0);
      \draw[doubleeq] (b0) to (bb0);
    \draw[doubleeq] (t4) to (m4);  
   \draw[doubleeq] (m4) to (b4);
      \draw[doubleeq] (b4) to (bb4);
   \draw[doubleeq] (m2) to (b2);
 \node at (1,-1.5) {\footnotesize $\iso$}; 
  \node at (.5,-.5) {\footnotesize $\DDownarrow \Pi^{\beta} $}; 
    \node at (1.5,-.5) {\footnotesize $\DDownarrow \overline{\tightid_{I} \Pi^{\alpha}} $}; 
   \node at (1,.5) {\footnotesize $\iso$}; 
 \end{tikzpicture}
 \end{aligned}
\end{equation}


\begin{equation}
\Pi^{\beta \comp \alpha} := 
\begin{aligned}
  \begin{tikzpicture}[yscale=1.5, xscale=5]
 \node (t0) at (0,1) {$\tens (gf \times gf)$};
\node (t2) at (1,1) {$gf \tens $};
 \node (t4) at (2,1) {$hk \tens $};
 \node (m0) at (0,0) {$\tens (gf \times gf)$};
 \node (m1) at (.5,0) {$g \tens (f\times f)$}; 
\node (m2) at (1,0) {$h \tens (f \times f)$};
\node (m3) at (1.5,0) {$hf \tens $};
\node (m4) at (2,0) {$hk \tens $};
 \node (b0) at (0,-1) {$\tens (gf  \times gf)$};
 \node (b1) at (.5,-1) {$\tens (hf \times hf)$}; 
\node (b2) at (1,-1) {$h \tens (f \times f)$};
\node (b3) at (1.5,-1) {$h \tens (k \times k)$};
\node (b4) at (2,-1) {$hk \tens $};
\node (bb0) at (0,-2) {$\tens (gf \times gf)$};
 \node(bb2) at (1,-2) {$\tens (hk \times hk)$};
   \node(bb4) at (2,-2) {$hk \tens $};
 \draw[doubleloose] (t0)  to node[above]{$\chi_{gf} $} (t2);
  \draw[doubleloose] (t2)  to node[above]{$\beta \alpha \looseid_{\tens}$} (t4);
\draw[doubleloose] (m0) to node[above]{$\chi_g \looseid_{f \times f}$} (m1);
  \draw[doubleloose] (m1) to node[above]{$\beta \looseid_{\tens (f \times f)}$}(m2);
  \draw[doubleloose] (m2) to node[above]{$\looseid_{h} \chi_f$}(m3);
  \draw[doubleloose] (m3) to node[above]{$\looseid_h \alpha \looseid_{\tens}$}(m4);
  %%%%%%%%%%%%
  \draw[doubleloose] (b0) to node[above]{$\looseid_{\tens} (\beta \times \beta) \looseid_{f \times f}$} (b1);
  \draw[doubleloose] (b1) to node[above]{$\chi_h \looseid_{f \times f}$} (b2);
  \draw[doubleloose] (b2) to node[above]{$\looseid_{h \tens} (\alpha \times \alpha) $}(b3);
  \draw[doubleloose] (b3) to node[above]{$\looseid_h \chi_k $}(b4);
 %%%%%%%%%
  \draw[doubleloose] (bb0)  to node[above]{$\looseid_{\tens} (\beta \alpha \times \beta \alpha) $} (bb2);
  \draw[doubleloose] (bb2)  to node[above]{$\chi_{hk} $} (bb4); 
   %%%%%%%%%% 
  \draw[doubleeq] (t0) to (m0);  
   \draw[doubleeq] (m0) to (b0);
      \draw[doubleeq] (b0) to (bb0);
    \draw[doubleeq] (t4) to (m4);  
   \draw[doubleeq] (m4) to (b4);
      \draw[doubleeq] (b4) to (bb4);
   \draw[doubleeq] (m2) to (b2);
 \node at (1,-1.5) {\footnotesize $\iso$}; 
  \node at (.5,-.5) {\footnotesize $\DDownarrow \overline{\Pi^{\beta} \tightid_{\looseid_{f \times f}}}$}; 
    \node at (1.5,-.5) {\footnotesize $\DDownarrow \overline{\tightid_{\looseid_{h}} \Pi^{\alpha} }$}; 
   \node at (1,.5) {\footnotesize $\iso$}; 
 \end{tikzpicture}
 \end{aligned}
\end{equation}

Let $f,h: A \rightarrow B $ and $g,i: B \rightarrow C$ be lax monoidal 1-cells and let $\alpha: f \rightarrow h$, $\beta: g \rightarrow i$ be lax monoidal icons, the composite $\beta \comp \alpha$ is lax monoidal with $N^{\beta \comp \alpha}:= N^{\beta} \horc \looseid_{\beta} N^{\alpha}$ and $\Sigma^{\beta \comp \alpha}:= \Sigma^{\beta}\looseid_{\alpha \times \alpha} \horc \looseid_{\beta} \Sigma^{\alpha}$.

Oplax structure 2-cells and 3-cells are obtained in a similar way. When $g,f$ are strong monoidal, the maps $\chi_{g \comp f}$ and $\overline{\chi_{g \comp f}}$ are an adjoint equivalence, constructed from the adjoint equivalence of the pairs $\chi_g, \overline{\chi_g}$ and $\chi_f, \overline{\chi_f}$. Similarly, $\iota_{g \comp f}$ and $\overline{\iota_{g \comp f}}$ form an adjoint equivalence. One can check that the required pairs of 3-cells correspond to eachother as mates by componentwise application of the adjoint equivalences for the composites of $\iota_{g \comp f}$ and $\chi_{g \comp f}$.

In all coherence equations between 3-cells for the monoidal and braided, sylleptic and symmetric structure of transversal composition above, each 3-cell consists of a conponent for the first composite  composed with the identity on the second composite, and a component for the second composite composed with the identity on the product of the first composite with itself. This means that the coherence equations for $g \comp f$  can be established by componentwise application of the equations for $g$ and $f$. Some 3-cells also contain coherence cells, but these equally break up in a part concerning the first, and a part concerning the second component. Manipulation of these coherence cells results in the required equalities. Note that rewriting the 1-cells and composites of loose 2-cells is necessary in several of the steps. A similar argument holds for coherence equations for braided, sylleptic and symmetric cells.

Let $\Gamma$ and $\Delta$ be two composable monoidal 3-cells. It is easy to see that the composition $\Gamma \comp \Delta$ satisfies the two equations for monoidal 3-cells. This is a matter of applying the equations for $\Gamma$ and $\Delta$ sequentially.
\end{proof}

% Local Variables:
% TeX-master: "smbicat"
% End:


\section{Symmetric Monoidal Bicategories}
\label{sec:constr-symm-mono}

In this section we will discuss how the iconic functor $\mathcal{H}$ from section~\ref{sec:1x1-to-bicat} extends to a functor of locally cubical bicategories. Furthermore, we will show that it preserves the monoidal structure functorially. We will do this by proving that the functor preserves products, and that every product preserving functor defines a functor between the respective locally cubical bicategories of monoidal cells. 

%Note that there are several related notions of these locally cubical bicategories, depending on whether we want the functors to be lax monoidal, oplax monoidal, or strongly monoidal. We will discuss the lax case in this section. In the next section we will generalise this result. 
%t was shown in~\cite{gg:ldstr-tricat} that tricategories, lax homomorphisms, ico-icons and pseudo-icons form a locally cubical bicategory. As a monoidal bicategory is a tricategory with one object, it follows that monoidal bicategories, lax monoidal functors, lax icons and lax transformations form a locally cubical bicategory as well. 
Double categories, pseudo double functors and tight transformations form a locally cubical bicategory with identity tight 2-morphisms and identity 3-cells. The functor $\comp$ is defined on tight transformations as the Godement product. The pseudo double functor $\transid_A: * \rightarrow \cD bl(A,A)$ maps the cells in the trivial double category $*$ to the identity cells and morphisms in $\cD bl(A,A)$. 

As described in~\cite{gg:ldstr-tricat}, bicategories, pseudo functors, pseudo transformations, icons and cubical modifications also form a locally cubical bicategory. We will recall the definition of a cubical modification.

\begin{defn}
Let $F,G,H,K: \cD \rightarrow \cE$ be pseudo functors; let $\alpha: F \Rightarrow G$, $\beta: H \Rightarrow K$ be pseudo transformations; let $\gamma: F \Rightarrow H$, $\delta: G \Rightarrow K$ be icons. A cubical modification
\[
\begin{tikzpicture}
\node (tl) at (0,1) {$F$};
\node (tr) at (1,1) {$G$};
\node (bl) at (0,0) {$H$};
\node (br) at (1,0) {$K$};
\draw[doubletight] (tl) to node[above]{$\alpha$} (tr);
\draw[doubletight] (bl) to node[below]{$\beta$} (br);
\draw[doubletight] (tl) to node[left]{$\gamma$} (bl);
\draw[doubletight] (tr) to node[right]{$\delta$} (br);
\node at (.5,.5) {$\DDownarrow \Gamma$};
\end{tikzpicture}
\]
is given by a family of 2-cells $\Gamma_A: \alpha_A \RRightarrow \beta_A$ such that for every 1-cell $f:A \rightarrow B$ of $\cD$, the following equality holds.

 \begin{equation}
 \begin{aligned}
 \begin{tikzpicture}[scale=1.5]
 \node (tl) at (-1,1) {$FA$};
 \node (tm) at (0,1) {$FB$};
 \node (tr) at (1,1) {$GB$};
 \node (bl) at (-1,0) {$FA$};
 \node (bm) at (0,0) {$GA$};
 \node (br) at (01,0) {$GB$};
 \node (bl1) at (-1,-.7){$HA$};  
 \node (bm1) at (0,-.7) {$KA$};
 \node (br1) at (1,-.7) {$KB$}; 
 \draw[doubletight] (tm)  to node[above]{$\alpha_B$} (tr);
 \draw[doubleeq] (bm) to (bm1);
 \draw[doubletight] (bm) to node[above] {$Gf$}(br);
 \draw[doubleeq] (tr) to (br);
 \draw[doubleeq] (tl)  to  (tm);
 \draw[doubleeq] (tl) to (bl);
 \draw[doubletight] (tl) to node[above]{$Ff$}(tm);
 \draw[doubletight] (bl) to node[above]{$\alpha_A$}(bm);
 \node at (0,.5) {\footnotesize $\Downarrow \alpha_f$}; 
 \node at (0.5,-.3) {\footnotesize $\Downarrow \delta_f$}; 
  \node at (-0.5,-.3) {\footnotesize $\Downarrow \Gamma_A$};
 \draw[doubletight] (bl1)  to node[above]{$\beta_A$} (bm1);
 \draw[doubletight] (bm1) to  node[above]{$Kf$}(br1);
 \draw[doubleeq] (bl)  to (bl1);
 \draw[doubleeq] (br)  to (br1);
 \end{tikzpicture}
 \end{aligned}
 =
\begin{aligned}
 \begin{tikzpicture}[scale=1.5]
 \node (tl) at (-1,1) {$FA$};
 \node (tm) at (0,1) {$FB$};
 \node (tr) at (1,1) {$GB$};
 \node (bl) at (-1,0) {$HA$};
 \node (bm) at (0,0) {$HB$};
 \node (br) at (01,0) {$KB$};
 \node (bl1) at (-1,-.7){$HA$};  
 \node (bm1) at (0,-.7) {$KA$};
 \node (br1) at (1,-.7) {$KB$}; 
 \draw[doubletight] (tm)  to node[above]{$\alpha_B$} (tr);
 \draw[doubleeq] (tm) to (bm);
 \draw[doubletight] (bm) to node[above] {$\beta_B$}(br);
 \draw[doubleeq] (tr) to (br);
 \draw[doubleeq] (tl)  to  (tm);
 \draw[doubleeq] (tl) to (bl);
 \draw[doubletight] (tl) to node[above]{$Ff$}(tm);
 \draw[doubletight] (bl) to node[above]{$Hf$}(bm);
 \node at (-0.5,.5) {\footnotesize $\Downarrow \gamma_f$}; 
 \node at (0.5,.5) {\footnotesize $\Downarrow \Gamma_B$}; 
 \draw[doubletight] (bl1)  to node[above]{$\beta_A$} (bm1);
 \draw[doubletight] (bm1) to  node[above]{$Kf$}(br1);
 \draw[doubleeq] (bl)  to (bl1);
 \draw[doubleeq] (br)  to (br1);
 \node at (0,-0.3) {\footnotesize $\DDownarrow \beta_f$}; 
 \end{tikzpicture}
 \end{aligned}
\end{equation}

\end{defn}

The pseudo functor $\transid_A: * \rightarrow \cB icat(A,A)$ maps the cells in the trivial bicategory $*$ to the identity cells and morphisms of $\cB icat(A,A)$. 
The pseudofunctor $\comp$ is defined on functors as composition in the iconic tricategory $\cB icat$. On pseudo transformations and icons it is given by the Godement product. On cubical modifications it is defined below:

\begin{equation*}
\begin{aligned}
 \begin{tikzpicture}[scale=2]
 \node (tl) at (-1,1) {$FF'A$};
 \node (tm) at (0,1) {$GF'A$};
 \node (tr) at (1,1) {$GG'A$};
 \node (bl) at (-1,0) {$HF'A$};
 \node (bm) at (0,0) {$KF'A$};
 \node (br) at (01,0) {$KG'A$};
 \node (bl1) at (-1,-1){$HH'A$};  
 \node (bm1) at (0,-1) {$KH'A$};
 \node (br1) at (1,-1) {$KK'A$}; 
 \draw[doubletight] (tm)  to node[above]{$G(\alpha'_A)$} (tr);
 \draw[doubleeq] (tm) to (bm);
 \draw[doubletight] (bm) to node[above] {$K(\alpha'_A)$}(br);
 \draw[doubleeq] (tr) to (br);
 \draw[doubleeq] (tl)  to  (tm);
 \draw[doubleeq] (tl) to (bl);
  \draw[doubleeq] (bm) to (bm1);
 \draw[doubletight] (tl) to node[above]{$\alpha_{F'A}$}(tm);
 \draw[doubletight] (bl) to node[above]{$\beta_{F'A}$}(bm);
 \node at (-0.5,.5) {\footnotesize $\Downarrow \Gamma_{F'A}$}; 
 \node at (0.5,.5) {\footnotesize $\Downarrow \delta_{\alpha'_A}$}; 
 \draw[doubletight] (bl1)  to node[above]{$\beta_{H'A}$} (bm1);
 \draw[doubletight] (bm1) to  node[above]{$K(\beta'A)$}(br1);
 \draw[doubleeq] (bl)  to (bl1);
 \draw[doubleeq] (br)  to (br1);
 \node at (-.5,-0.5) {\footnotesize $=$}; 
\node at (.5,-0.5) {\footnotesize $\DDownarrow K\Gamma'_A$}; 
\end{tikzpicture}
\end{aligned}
\end{equation*}
%%%%%%%%%

Functoriality follows from naturality of the icons. Note that there are several canonical ways to define this composition on cubical modifications, by choosing different versions of the Godement product.  

We upgrade the functor from theorem 4.11 to a functor of locally cubical bicategories.


\begin{defn}\label{def:lcbcfunc}
Let ${\bf S,T}$ be locally cubical bicategories. A functor $\cT: {\bf T} \rightarrow {\bf S}$ consists of the following data:
\begin{enumerate}
\item An assignment on objects that sends each object $A$ of ${\bf T}$ to an object $\cT A$ of ${\bf S}$.
\item For each two objects $A,B$, a pseudo double functor (1-cell in \cDbl) ${\bf T}(A,B) \rightarrow {\bf S}(\cT(A),\cT(B))$
\item For every triple of objects $A,B,C$ of ${\bf T}$, a tight transformation (2-cell in $\cD bl$) 
\begin{align} 
\begin{tikzpicture}
\node(1) at (0,0) {${\bf T}(A,B) \times {\bf T}(B,C)$};
\node(2) at (5,0) {${\bf S}(\cT(A),\cT(C))$};
\draw[->] (1) to[in=155, out=25] node[above]{$\cT \comp $} (2); 
\draw[->] (1) to[in=-155, out=-25] node[below]{$ \comp (\cT,\cT)$} (2); 
\node at (2.5,0) {$\Downarrow \phi \iso$};
\end{tikzpicture}
\end{align}
\item For every object $A$ of {\bf T} a tight transformation
\begin{align}
\begin{tikzpicture}[xscale=.5, yscale=.3]
\node(1) at (0,0) {$*$};
\node(2) at (5,0) {${\bf S}(A,A)$};
\node(3) at (5,-5) {${\bf T}(\cT(A),\cT(A))$};
\draw[->] (1) to node[above]{$\looseid_{A}$} (2); 
\draw[->] (1) to node[below]{$\looseid_{\cT(A)}$} (3);
\draw[->] (2) to node[right]{$\cT$} (3); 
\node at (3.5,-1.5) {$\Downarrow \phi_u \iso$};
\end{tikzpicture}
\end{align}
\item The usual coherence diagrams, Definition 10 of~\cite{nick:tricatsbook} commute.
\end{enumerate}
\end{defn}



\begin{prop}
The iconic functor $\cH: \cD bl_{\bf f} \rightarrow \cB icat$ extends to a functor of locally cubical bicategories $\fH: \fDbl_{\bf f} \rightarrow \fBicat$ that preserves products.
\end{prop}

\begin{proof}
For any $\D, \E$, the double category $\fDblf(\D, \E)$ has only globular 2-cells and is identical to the bicategory in section~\ref{sec:1x1-to-bicat}. Consequently, each pseudo functor $\cH: \cDbl(\D,\E) \rightarrow \cBicat(\cH\D, \cH \E)$ corresponds to a pseudo double functor $\fH: \fDblf(\D,\E) \rightarrow \fBicat(\fH\D, \fH \E)$. The icons $\phi$ and $\phi_u$ then define tight transformations consisting of globular 2-cells and the coherence diagrams are automatically satisfied. The proof that $\cH$ preserves product carries directly over to the case of $\fH$.
\end{proof}


 \begin{lem}\label{lem:funcmonob}
  Let ${\bf S,T}$ be locally cubical bicategories with products. Let $\mathcal{T}: T \rightarrow S$ be a functor such that the tight transformations $\phi$, $\phi_u$ are globular. If $F$ preserves products, it preserves  monoidal objects, 1-cells, 2-cells, icons and 3-cells as well as any braided, sylleptic or symmetric structure on the objects, 1-cells,2-cells, icons and 3-cells.
 \end{lem}
 
 \begin{proof}
Let $A$ be a monoidal object. As the functor $\mathcal{T}$ preserves products, we have a product $\cT(A) \times_{\cT} \cT(A) = \cT(A \times A)$. As a consequence $\ten\maps
  A\times A\to A$ induces 1-cells $\ten_{\mathcal{T}} \maps
 \mathcal{T}A\times_{\cT} \mathcal{T} A\to\mathcal{T}A$ and $I_{\cT}:= \mathcal{T}(I_A)$. 
 
Since $\phi$ and $\phi_u$ are globular, we have an equality $\cT(f \comp g) = \cT(f) \comp \cT(g)$ for all $f$ and $g$ and for the identity 1-cell we have an equality $\cT(\transid_A) = \transid_{\cT(A)}$. The loose associativity 2-cell of $A$ gives rise to a loose 2-cell
  \[\vcenter{\xymatrix@C=6pc{\cT(A)\times\cT(A)\times\cT(A) \rtwocell^{\ten_{\cT}
        (\Id\times\ten_{\cT})}_{\ten_{\cT}(\ten_{\cT}\times\Id)}{\hspace{.2cm}\fa_{\cT}\eqv} &\cT(A) }}\]
  which simply equals $\cT(\alpha)$ together with the invertible 2-cells.
  
  Likewise, the unit constraints $l, r$ as well as the constraints for (braided) monoidal 1-cells $\sigma$ induce 1-cells $l_{\cT}, r_{\cT}$, and $\sigma_{\cT}$, respectively. Note that the swap functor $\tau$ is mapped by $\cT$ to the swap functor for the product $\times_{\cT}$, so $\sigma_{\cT}$ is well-defined.
  
 Furthermore, the invertible 3-cell filling the Mac Lane pentagon lifts to the invertible 3-cell of the Mac Lane Pentagon for $\cT(A)$. Which is simply its image under $\cT$, composed the natural transformations $\cT_{\odot}$, $\phi$,and $\phi_u$ ensuring that it has the right type.
%   \[\xy
%  (-10,0)*{\ten_{\cT}(\ten_{\cT},\Id)(\ten_{\cT}, \Id, %\Id)}="A";
%  (20,10)*{\ten_{\cT}(\ten_{\cT},\Id)(\Id, \ten_{\cT},\Id)}="B";
%  (50,0)*{\ten_{\cT}(\Id,\ten_{\cT})(\Id, \ten_{\cT},\Id)}="C";
%  (0,-15)*{\ten_{\cT}(\Id, \ten_{\cT})(\ten, \Id, \Id)}="D";
%  (40,-15)*{\ten_{\cT}(\Id, \ten_{\cT})(\Id, \Id, \ten, _{\cT})}="E";
%  (20,-5)*{\scriptstyle\Downarrow \pi \iso};
%  \ar "B";"A";^{\fa_{\cT} \ten_{\cT} \id}
%  \ar "C";"B";^{\fa_{\cT}}
%  \ar "D";"A";_{\fa_{\cT}}
%  \ar "E";"D";_{\fa_{\cT}}
%  \ar "E";"C";^{\id\ten_{\cT} \fa_{\cT}}
%  \endxy
%  \]
  
\begin{tikzpicture}[yscale=1.5, xscale=3]
\node(tl) at (0,1) {$\ten (\ten \times \transid)(\ten \times \transid \times \transid)$};
\node(t) at (1.5,2) {$\ten (\ten \times \transid)(\transid \times \ten \times \transid)$};
\node(tr) at (3,1) {$\ten (\transid \times \ten )(\transid \times \ten \times \transid)$};
\node(br) at (3,0) {$\ten (\transid \times \ten )(\transid \times \transid \times \ten )$};
\node(b) at (1.5,-1) {$\ten (\ten \times \transid)(\transid \times \transid \times \ten )$};
\node(bl) at (0,0) {$\ten (\transid \times \ten )(\ten \times \transid \times \transid)$};
\draw[->] (tl) to node[left, yshift=1pt] {$\looseid (\alpha \times \looseid)$} (t);
\draw[->] (t) to node[right, yshift=1pt] {$\alpha \looseid$} (tr);
\draw[->] (tr) to node[right] {$\looseid (\looseid \times \alpha)$} (br);
\draw[->] (tl) to node[left] {$\alpha \looseid$} (bl);
\draw[->] (bl) to node[left,yshift=-1pt] {$\looseid$} (b);
\draw[->] (b) to node[right,yshift=-1pt] {$\alpha \looseid$} (br);
\draw[->] (tl) to [in=155, out=5] (br);
\draw[->] (tl) to [in=180, out=-10] (br);
%\draw[->] (tl) to [in=180, out=10](tr);
%\draw[->] (bl) to [in=185, out=0](br);
\node at (1.5,.6) {$\Downarrow \cT(\pi) \iso$};
%\node at (2.5,.6) {$\Downarrow \phi^{-1} \iso$};
%\node at (1.5,1.5) {$\Downarrow \phi^{-1} \iso$};
%\node at (.5,.4) {$\Downarrow \phi \iso$};
%\node at (1.5,-.5) {$\Downarrow \phi \iso $};
\node at (2,1) {$\iso$};
\node at (1,0) {$\iso $};
\end{tikzpicture}  

 Note that there may be several way to past these 3-cells, but by coherence of enriched pseudo functors, the result is the same. Likewise, the isomorphic 3-cells $\mu, \lambda,\rho$ and the isomorphic 3-cells $R,S$, and $v$ witnessing the braiding and syllepsis, lift to the appropriate 3-cell for $\cT(A)$. This is also the case for the isomorphic 3-cells $R,S$ that characterise the braiding. 
   Finally,  the three equations between pasting composites of $\pi_{\cT}, \mu_{\cT}, \lambda_{\cT}, \rho_{\cT}$ hold by coherence of enriched pseudo functors.

%Similarly, we prove that $\cT$ preserves braided, sylleptic and symmetric structure. The swap functor $\tau$ is mapped by $\cT$ to the swap functor for the product $\times_{\cT}$. It follows that the braiding 2-cell $\sigma: \tens \looseRightarrow \tens \tau$  gives rise to a braiding 

%\[\vcenter{\xymatrix@C=6pc{\cT(A)\times\cT(A) \rtwocell^{\ten_{\cT}
  %      }_{\ten_{\cT}\tau}{\hspace{.2cm}\sigma_{\cT}\eqv} &\cT(A) }}\]
        
%Analogously to $\alpha$, the invertible globular 3-cells $S, T$, and $\upsilon$ are lifted to their images under $\cT$, augmented with instances of $\phi$ to ensure that the 3-cells have the right type. 

Similarly, one can show that for a monoidal transformation $f$, $\cT(f)$ is monoidal with structure loose 2-cells $\cT(\iota_f)$ and $\cT(\chi_f)$ and 3-cells obtained from $\cT(\omega),  \cT(\gamma)$, and $\cT(\delta)$ analogously to $\alpha$. If $f$ is braided, the braiding of $\cT(f)$ is witnessed by the 3-cell obtained from $\cT(u)$. Likewise, $\cT$ preserves any monoidal, braided, sylleptic and symmetric structure of tight 2-morphisms, loose 2-cells and 3-cells. 
 \end{proof}


\begin{thm}\label{thm:lcbcfunctor}
Let $F: \cC  \rightarrow \cD$ be a functor of locally cubical bicategories, such that the tight transformations $\phi$, $\phi_u$ are globular. If $F$ preserves products, then it lifts to the functors below between locally cubical bicategories, for $w \in \{l,o,s\}$.
\begin{align*}
&\cM on_wF: \cM on_w \cC  \rightarrow \cM on_w\cD\\ 
&\cBr \cM on_wF: \cBr \cM on_w \cC  \rightarrow \cBr \cM on_w\cD\\
& \cSyl \cM on_wF: \cSyl \cM on_w \cC  \rightarrow \cSyl \cM on_w\cD\\
&\cSym \cM on_wF: \cSym \cM on_w \cC  \rightarrow \cSym \cM on_w\cD
\end{align*}
\end{thm}

\begin{proof}
By Lemma~\ref{lem:funcmonob}, the assignment of $F$ on objects and higher cells is well-defined in the respective categories.
We wil show that functor $F$ gives rise to a pseudo double functor $\cM on_lF: \cM on_l \cC(A, B)  \rightarrow \cM on_l\cD(\cM on_lF(A), \cM on_lF(B))$. The other functors are derived in a similar way. We need to verify that $F(N^{\alpha \verc \beta}) = F(N^{\alpha}) \verc F(N^{\beta})$ and $F(N^{\tightid_f}) = N^{\tightid_{Ff}}$ in $\cM on_l\cD$ and likewise for $\Sigma$. This follows from functoriality of $F$, the fact that $N^{\alpha \verc \beta} = N^{\alpha \verc \beta}$ and $N^{\tightid_f} = \tightid_{\iota_f}$, $F(\iota_f) = \iota_{Ff}$.
The natural transformations $F_{\odot}$ and $F_U$ are well-defined 3-cells in $\cM onD$; the respective equations hold by coherence of the pseudo double functor $F$.
Finally, we need to prove that $\phi$ and $\phi_u$ are well-defined tight transformations in $\cM on(\cT A, \cT B)$. Since their components are globular 3-cells, we only need to check that these are monoidal. The equations hold by coherence of enriched functors after expanding the definitions of $M^{\cT\alpha \comp \cT \beta}, M^{\cT(\alpha \comp \beta)}, M^{\cT(\looseid_f)}$, and $M^{\looseid_{\cT f}}$ and similarly for $\Pi$.
\end{proof}

\begin{thm}
The functor $\fH: \fDbl_f \rightarrow \fBicat$ lifts to the functors $\cM on_w \fH: \cMon_w\fDbl_f \rightarrow \cMon_w \fBicat$, $\cBr_w \fH: \cBr_w\fDbl_f \rightarrow \cBr_w \fBicat$,  $\cSyl_w\cH: \cSyl_w\fDbl_f \rightarrow \cSyl_w\fBicat$, and $\cSym_w \fH: \cSym_w\fDbl_f \rightarrow \cSym_w \fBicat$ for $w\in \{l,o,s\}$.
\end{thm}

\begin{proof}
Note that in the case of $\cH$, $\phi$ and $\phi_u$ have globular components. The result then follows from Theorem~\ref{thm:lcbcfunctor}
\end{proof}


% Local Variables:
% TeX-master: "smbicat"
% End:


\section{An example}\label{sec:Alg} 
As a demonstration of our method, we apply Theorem~\ref{thm:lcbcfunctor} to prove that the bicategory of algebras, bimodules and bimodule homomorphisms in a braided monoidal category ${\cat C}$ is braided monoidal. It is symmetric when ${\bf C}$ is symmetric. 
Frobenius algebras and modules play an important role in quantum theory. Recently, the bicategory $2(CP({\bf C}))$ of dagger Frobenius algebras, dagger bimodules and bimodule homomorphisms has been defined in~\cite{heunenvicarywester} as the mathematical foundation of a diagrammatic language for quantum protocols. The existence of monoidal structures is an essential feature which allows for the spatial composition of information flows. We will explore this in generality for the bicategory $\mathcal{A}lg({\cat C})$, defined below.

\begin{defn}
Let ${\bf C}$ be a monoidal category. We can define a  bicategory $\mathcal{A}lg({\cat C})$ consisting of the elements listed below.

\begin{itemize}
\item 0-cells are monoids in {\cat C}: pairs $(a,A)$ of a 0-cell $a \in {\cat C}$ and an algebra $A \in {\cat C}(a,a)$.
\item 1-cells $(a,A) \rightarrow (b,B)$ are 1-cells $X \in {\cat C}(a,b)$ with the structure of an $A-B-$bimodule.
\item 2-cells are 2-morphisms in ${\cat C}$ that are bimodule homomorphisms.
\end{itemize}
Structural data regarding this construction, such as horizontal composition, is described in ~\cite{heunenvicarywester}. 
\end{defn}

To obtain the result, we define the double category $\mathbb{A}lg({\cat C})$, whose horizontal bicategory corresponds to $\mathcal{A}lg({\cat C})$and we show that $\mathbb{A}lg({\cat B})$ is fibrant and symmetric monoidal.
We will often refer to these categories as $\mathbb{A}lg$ and $\mathcal{A}lg$, when the choice of the original monoidal category ${\bf C}$ is not important.

\subsection{Notation}
We use the language of string diagrams for braided monoidal categories, first introduced by M. Kelly and M.L. Laplaza in ~\cite{kellylaplaza}. In this diagrammatic language, we depict objects (and identity morphisms) as lines, and morphisms as boxes between the source object in the bottom and the target objectl on top. We write composition $(\ver)$ in the vertical direction and the tensor product $(\ten)$ as juxtaposition in the horizontal direction. 

\begin{align}
 \mbox{Composition}  \hspace{60pt}
 & \mbox{Tensor product} \\
       \begin{tikzpicture}[scale=1.75]  
\node at (-1,0){};
\node at (1,-2){};  
      \node[morphism, minimum width=10mm] (g) at (0,-.6) {$g$};
       \node[morphism, minimum width=10mm] (f) at (0,-1.4) {$f$};
       \draw (0,0) to (g.north);
      \draw (f.north) to (g.south);
      \draw (f.south) to (0,-2) node[below] {};
    \end{tikzpicture}
 \hspace{40pt}
 &\begin{tikzpicture}[scale=1.75]   
      \node[morphism, minimum width=10mm] (l) at (-.4,-1) {$f$};
      \draw (-.4,0) to (l.north);
      \draw (l.south) to (-.4,-2) node[below] {};
            \node[morphism, minimum width=10mm] (r) at (.4,-1) {$h$};
      \draw (.4,0) to (r.north);
      \draw (r.south) to (.4,-2) node[below] {};
    \end{tikzpicture}
\end{align}

\subsection{The double category $\Alg$}\label{sec:Dalg}
We will consider the abstract notion of an {\bf algebra} $(A, \tinymult[white dot], \tinyunit[white dot])$ in a monoidal category {\cat C}. This is an object $a$, together with a multiplication morphism $\tinymult[white dot]: A \ten A \rightarrow A$, as well as a unit morphism, $\tinyunit[white dot]: I \rightarrow A$, satisfying the associativity and unit law below.

\begin{calign}
\label{eq:frobenius}
  \begin{pic}[yscale=0.85]
    \node[white dot] (l) at (0,0) {};
    \node[white dot] (r) at (.5,.5) {};
    \draw (l.center) to[out=90,in=180] (r.center);
    \draw (r.center) to (.5,1)node[above] {$A$};
    \draw (r.center) to[out=0,in=90] (1,-.5)node[below] {$A$};
    \draw (l.center) to[out=180,in=90] (-.5,-.5)node[below] {$A$};
    \draw (l.center) to[out=0,in=90] (.5,-.5)node [below] {$A$};
  \end{pic}
  \hspace{1pt}=\hspace{1pt}
  \begin{pic}[yscale=0.85]
    \node[white dot] (r) at (0,0) {};
    \node[white dot] (l) at (-.5,.5) {};
    \draw (r.center) to[in=0,out=90] (l);
    \draw (l.center) to (-.5,1)node[above] {$A$};
    \draw (l.center) to[in=90,out=180] (-1,-.5)node[below] {$A$};
    \draw (r.center) to[in=90,out=0] (.5,-.5)node[below] {$A$};
    \draw (r.center) to[in=90,out=180] (-.5,-.5)node[below] {$A$};
  \end{pic}
  \hspace{30pt}
  \begin{pic}[yscale=0.85]
    \node[white dot] (m) at (0,.5) {};
    \node[white dot] (u) at (-.5,0) {};
    \draw (m.center) to (0,1)node[above] {$A$};
    \draw (u.center) to[out=90,in=180] (m.center);
    \draw (m.center) to[out=0,in=90] (.5,-.5)node[below] {$A$};
  \end{pic}
  \hspace{1pt}=\hspace{1pt}
  \begin{pic}[yscale=0.85]
    \draw (0,-.5) node[below] {$A$} to (0,1) node[above]{$A$};
  \end{pic}
  \hspace{1pt}=\hspace{1pt}
  \begin{pic}[yscale=0.85]
    \node[white dot] (m) at (0,.5) {};
    \node[white dot] (u) at (.5,0) {};
    \draw (m.center) to (0,1)node[above] {$A$};
    \draw (u.center) to[out=90,in=0] (m.center);
    \draw (m.center) to[out=180,in=90] (-.5,-.5)node[below] {$A$};
  \end{pic}
\end{calign}

An {\bf algebra homomorphism} $f: (A, \tinymult[white dot], \tinyunit[white dot]) \rightarrow (B, \tinymult[gray dot], \tinyunit[gray dot])$ between two algebras is a morphism $f:A \rightarrow B$ in ${\cat C}$ that respects the multiplication: $f \ver \tinymult[white dot] = \tinymult[gray dot] \ver (f \tens f)$, and the unit: $\tinyunit[gray dot] = f \ver \tinyunit[white dot]$. If ${\bf C}$ is braided or symmetric monoidal, then the category of algebras and algebra homomorphisms is also braided or symmetric monoidal, respectively. The tensor product for algebras in inherited from the original 2-category. This is depicted below. The unit object is the trivial algebra on the unit for the original monoidal structure with multiplication given by $\lambda_I = \rho_I$. The original structure isomorphisms are all algebra homomorphisms. In principle this can be shown directly, but in the diagrammatic calculus this holds trivially.  
\begin{calign}
  \begin{pic}[xscale=3, yscale=2.25]
    \draw (-.5,0.5) to (-.5,1)node[above] {};
    \draw (.2,-.4) node[below]{} to[out=90,in=-90] (-.3,0);
    \draw (-.3,0) node[below]{} to[out=90,in=0] (-.5,0.5);
    \draw (-.5,0.5) to[out=180,in=90] (-.75,.1)node[below] {};
    \draw (-.75,.1) to[out=-90,in=90] (-.8,-.4)node[below] {};
    \node [white dot] (m) at (-.5,.5) {};
    %%%
    \node[white dot] (m) at (0,.5) {};
    \draw (m.center) to (0,1);
    \draw (.3,-.4) node[below]{} to[out=90,in=-90, looseness=1] (.25,.1);
    \draw (.25,.1) node[below]{} to[out=90,in=0, looseness=1] (m.center);
    \draw (m.center) to[out=180,in=0, looseness=1] (-.4,-.2)node[below] {};
    \draw (-.4,-.2) to[out=180,in=90, looseness=1] (-.7,-.4)node[below] {};
  \end{pic}
\end{calign}



An A,B-{\bf bimodule} in {\bf C} is an object $M$ together with a morphism ${\bf M}: A \ten M \ten  B \rightarrow M$ in ${\bf C}$, satisfying the composition and unit law below.

\begin{equation}\label{eq:bimodule}
    \begin{pic}[yscale=0.85]
      \node[morphism, minimum width=10mm] (u) at (0,0) {$\mathbf{M}$};
      \node[morphism, minimum width=10mm] (l) at (0,-1.1) {$\mathbf{M}$};
      \draw (u.north) to (0,.6) node[right] {$M$};
      \draw (u.south) to (l.north);
      \draw (l.south) to (0,-2) node[right] {$M$};
      \draw (l.-45) to[out=-90,in=90] (.6,-2) node[right] {$B$};
      \draw (l.-135) to[out=-90,in=90] (-.6,-2) node[right] {$A$};
      \draw (u.-45) to[out=-90,in=90] (1.1,-1) to (1.1,-2) node[right] {$B$};
      \draw (u.-135) to[out=-90,in=90] (-1.1,-1) to (-1.1,-2) node[right] {$A$};
    \end{pic}
    =
    \begin{pic}[yscale=0.85]
      \node[morphism, minimum width=10mm] (m) at (0,0) {$\mathbf{M}$};
      \node[white dot] (l) at (-.85,-1) {};
      \node[black dot] (r) at (.85,-1) {};
      \draw (m.north) to (0,.6) node[right] {$M$};
      \draw (m.south) to (0,-2) node[right, xshift=-.2] {$M$};
      \draw (m.-45) to[out=-90,in=90] (r);
      \draw (m.-135) to[out=-90,in=90] (l);
      \draw (l) to[out=-150,in=90] (-1.1,-2) node[left] {$A$};
      \draw (l) to[out=-30,in=90] (-.6,-2) node[left] {$A$};
      \draw (r) to[out=-150,in=90] (.6,-2) node[right] {$B$};
      \draw (r) to[out=-30,in=90] (1.1,-2) node[right] {$B$};
    \end{pic}
    \qquad
    \begin{pic}[yscale=0.85]
      \draw (0,.6) node[right] {$M$} to (0,-2) node[right] {$M$};
    \end{pic}
    \!\!\!=\,\,\,
    \begin{pic}[yscale=0.85]
      \node[morphism, minimum width=10mm] (m) at (0,0) {$\mathbf{M}$};
      \draw (m.north) to (0,.6) node[right] {$M$};
      \draw (m.south) to (0,-2) node[right] {$M$};
      \draw (m.-135) to[out=-90,in=90] +(0,-0.75) node[white dot] {};
      \draw (m.-45) to[out=-90,in=90] +(0,-0.75) node[black dot] {};
    \end{pic}
  \end{equation}
For readability, we sometimes write A-B-bimodules ${\bf M}$ as $_A{\bf M}_B$ and we abbreviate the following compositions: ${}_{\tinydot[white dot]}{\bf M} := {\bf M} \ver (\tinyunit[white dot] \tens \mathid)$, ${\bf M}_{\tinydot[gray dot]}:= {\bf M} \ver (\mathid \tens \tinyunit[gray dot])$, and $_{\tinydot[white dot]}{\bf M}_{\tinydot[gray dot]}:= {\bf M} \ver (\tinyunit[white dot] \tens \tinyunit[gray dot])$. An {\bf extended bimodule homomorphism} ${\bf f}: {\bf M} \rightarrow {\bf N}$ is a triple of 2-cells $(A \xrightarrow{f_l} C, M \xrightarrow{f} N, B \xrightarrow{f_r} D)$ such that $f \ver {\bf M} = {\bf N} \ver (f_l \tens f \tens f_r)$. When $f_l$ and $f_r$ are both identities, this corresponds to the standard definition of a {\bf bimodule homomorphism}.
Bimodules and extended bimodule homomorphisms in a braided or symmetric monoidal category {\bf C} form a braided or symmetric monoidal category, respectively, where the tensor product on objects is inherited from {\cat C}. On morphisms, it is defined as $(f_l, f, f_r) \tens (g_l, g, g_r) := (f_l \tens g_l, f \tens g, f_r \tens g_r)$. The unit object corresponds to the bimodule given below on the unit object of {\cat C}. 

\begin{equation}
{\bf I}:= \quad
\smash{{\hspace{-5pt}\ensuremath{\begin{pic}[scale=1.5,string,yscale=-1.5]
      \node[gray dot, inner sep=1.5pt] (1) at (0,0.55) {};
      \node (2) at (-0.5,1) {};
      \node (3) at (0.5,1) {};
      \draw (0.center)node[above]{$I$} to (1.center);
      \draw (1.center) to [out=left, in=down, out looseness=1.5] (2.center)node[below]{$I$};
      \draw (1.center) to [out=right, in=down, out looseness=1.5] (3.center)node[below]{$I$};
      \draw (1.center) to (0,1)node[below]{$I$};
      \end{pic}}\hspace{-3pt}}}
\hspace{30pt}
_A{\bf M}_B \tens {}_C{\bf N}_D :=
   \ensuremath{\begin{pic}[scale=1.5]
      \node[morphism, minimum width=10mm] (l) at (0,-.8) {$\mathbf{M}$};
      \draw (0,0) to (l.north);
      \draw (l.south) to (0,-2) node[below] {};
      \draw (l.-45) to[out=-90,in=90] (1.5,-2) node[below] {};
      \draw (l.-135) to[out=-90,in=90] (-.6,-2) node[below] {};
      %%%%%%%%%%%%%%%%%%%%%%%    
      \node[morphism, minimum width=10mm] (l) at (1.1,-.8) {$\mathbf{N}$};
      \draw (1.1,0) to (l.north);
      \draw (l.south) to (1.1,-2.1) node[below] {};
      \draw (l.-45) to[out=-90,in=90] (1.7,-2) node[below] {};
      \draw (l.-135) to[out=-90,in=90] (-.4,-2) node[below] {};
    \end{pic}}
\end{equation}
We may draw the multiplication on the unit object symmetrically because of the associativity condition for algebras. 
The associator, unitor, and swap natural isomorphisms are given by the maps with the following components, where $\alpha, \lambda$, and $\rho$ are the structure isomorphisms of the original category. 

\begin{align}
 {\bf \alpha}_{_A{\bf M}_B,_C{\bf N}_D,_E{\bf P}_F} &:= (\alpha_{A,C,E}, \alpha_{M,N,P}, \alpha_{B,D,F}) 
\\{\bf \lambda}_{_A{\bf M}_B}&:= (\lambda_A, \lambda_M, \lambda_B)
\\{\bf \rho}_{_A{\bf M}_B} &:= (\rho_A, \rho_M, \rho_B)
\\{\bf \sigma}_{\bf {}_AM_B,{}_CN_D}&: = (\sigma_{A,C}, \sigma_{M,N}, \sigma_{B,D})
\end{align}

For clarity, we write algebras and algebra homomorphisms in normal font and we write bimodules and (extended) bimodule homomorphisms in bold. This will help distinguishing between the two categories, which together form a double category.

\begin{lem}\label{lem:algdouble}
Let ${\cat C}$ be a monoidal category where all colimits exist, and where the tensor product preserves all colimits. The category $\D_0$ of algebras and algebra homomorphisms of ${\bf C}$ and the category $\D_1$ of bimodules and bimodule homomorphisms of ${\bf C}$ form a double category, which we call $\Alg$
\end{lem}

\begin{proof}
The {\bf unit functor} $\lD_0 \xrightarrow{U} \lD_1$ is defined on objects as $U(A,\tinymult[gray dot],\tinyunit[gray dot]) := 
      \smash{{\hspace{-5pt}\ensuremath{\begin{pic}[scale=0.4,string,yscale=-1]
      \node (0) at (0,0) {};
      \node[gray dot, inner sep=1.5pt] (1) at (0,0.55) {};
      \node (2) at (-0.5,1) {};
      \node (3) at (0.5,1) {};
      \draw (0.center) to (1.center);
      \draw (1.center) to [out=left, in=down, out looseness=1.5] (2.center);
      \draw (1.center) to [out=right, in=down, out looseness=1.5] (3.center);
      \draw (1.center) to (0,1);
      \end{pic}}\hspace{-3pt}}}$; 
      and on morphisms as $U(f) := (f,f,f)$. The {\bf source and target functors} $S,T: \lD_1 \rightarrow \lD_0$ are defined on objects as $S({}_A {\bf M}_B) = A$, and  $T({}_A {\bf M}_B) = B$, and on morphisms as  $S(f_l,f,f_r)=f_l$, and $T(f_l, f, f_r) = f_r$.
The functor $\odot: {\lD_1} \times_{\lD_0} {\lD_1} \rightarrow {\lD_1}$, maps objects $(_A{\bf M}_B$, $_B{\bf N}_C)$ to $_A{\bf M} \tinydot[gray dot] {\bf N}_C$, and morphisms $((f_l, f, h), (h, g, g_r))$ to $(f_l, f \tinydot[gray dot] g, g_r)$, defined as follows. The pairs $({M} \tinydot[gray dot] {N}, c)$ and $({M'} \tinydot[gray dot] {N'}, c')$ depicted below are coequalisers.  By commutativity of the left square, $c' \ver (f \ten g)$ is another coequaliser of the upper two parallel maps, hence there exists a unique map $f \tinydot[gray dot] g$.  

\begin{equation}\label{eq:coequaliser}
  \begin{pic}[xscale=4,yscale=1.66, thin]
    \node (tl) at (-.3,1) {$M \ten B \ten N$};
    \node (bl) at (-.3,0) {$M' \ten B' \ten N'$};
    \node (t) at (1,1) {$M \ten N$};
    \node (b) at (1,0) {$M' \ten N'$};
    \node (tr) at (1.7,1) {$M \tinydot[gray dot] N$};
    \node (br) at (1.7,0) {$M' \tinydot[gray dot] N'$};
    \draw[->] (tl) to node[left] {$(f \ten h \ten g)$} (bl);
    \draw[->] (t) to node[right] {$f \ten g$} (b);
    \draw[->, dashed] (tr) to node[right] {$f \tinydot[gray dot] g$} (br);
    \draw[->] (t) to node[above] {$c$} (tr);
    \draw[->] (b) to node[below] {$c'$} (br);
    \draw[->] ([yshift=1.5pt]tl.east) to node[above] {${}_{\tinydot[white dot]}\mathbf{M} \ten \mathid_{N}$} ([yshift=1.5pt]t.west);
    \draw[->] ([yshift=-1.5pt]tl.east) to node[below] {$\mathid_{M} \ten \mathbf{N}_{\tinydot[black dot]}$} ([yshift=-1.5pt]t.west);
    \draw[->] ([yshift=1.5pt]bl.east) to node[above] {${}_{\tinydot[white dot]}\mathbf{M'} \ten \mathid_{N'}$} ([yshift=1.5pt]b.west);
    \draw[->] ([yshift=-1.5pt]bl.east) to node[below] {$\mathid_{M'} \ten \mathbf{N'}_{\tinydot[black dot]}$} ([yshift=-1.5pt]b.west);
  \end{pic}
 \end{equation}

  The module ${\bf M} \tinydot[gray dot] {\bf N}$ is defined as the unique map that makes the diagram of coequalisers below commute.  

\begin{equation}\label{eq:coequaliser2}
  \begin{pic}[xscale=4,yscale=1.66, thin]
    \node (tl) at (-.3,1) {$A \ten M \ten B \ten N \ten C$};
    \node (bl) at (-.3,0) {$M \ten B \ten N$};
    \node (t) at (1,1) {$A \ten M \ten N \ten C$};
    \node (b) at (1,0) {$M \ten N$};
    \node (tr) at (1.7,1) {$A \ten M \tinydot[gray dot] N \ten C$};
    \node (br) at (1.7,0) {$M \tinydot[gray dot] N$};
    \draw[->] (tl) to node[left] {${\bf M}_{\tinydot[gray dot]} \ten \mathid_B \ten {}_{\tinydot[gray dot]}{\bf N}$} (bl);
    \draw[->] (t) to node[right] {${\bf M}_{\tinydot[gray dot]} \ten  {}_{\tinydot[gray dot]}{\bf N}$} (b);
    \draw[->, dashed] (tr) to node[right] {${\bf M} \tinydot[gray dot] {\bf N}$} (br);
    \draw[->] (t) to node[above] {$\mathid_A \ten c \ten \mathid_C$} (tr);
    \draw[->] (b) to node[below] {$c$} (br);
    \draw[->] ([yshift=1.5pt]tl.east) to node[above] {$\mathid_A \ten {}_{\tinydot[white dot]}\mathbf{M} \ten \mathid_N \ten \mathid_C$} ([yshift=1.5pt]t.west);
    \draw[->] ([yshift=-1.5pt]tl.east) to node[below] {$\mathid_A \ten \mathid_M \ten \mathbf{N}_{\tinydot[black dot]} \ten \mathid_C$} ([yshift=-1.5pt]t.west);
    \draw[->] ([yshift=1.5pt]bl.east) to node[above] {${}_{\tinydot[white dot]}\mathbf{M'} \ten \mathid_{N'}$} ([yshift=1.5pt]b.west);
    \draw[->] ([yshift=-1.5pt]bl.east) to node[below] {$\mathid_{M'} \ten \mathbf{N'}_{\tinydot[black dot]}$} ([yshift=-1.5pt]b.west);
  \end{pic}  
 \end{equation}

Note that the pair $(A \tens M \tinydot[gray dot] N \tens C, \id_A \tens c \tens \id_C)$ is a coequaliser, as $\ten$ preserves coequalisers. Furthermore, ${\bf M}_{\tinydot[gray dot]} \ten {}_{\tinydot[gray dot]}{\bf N}$ is a module. It is easy to check that ${\bf M}_{\tinydot[gray dot]}{\bf N}$ satisfies the module axioms as well, using commutativity of the diagram above and the fact that $c$ is epic. Pasting the diagrams~\ref{eq:coequaliser},\ref{eq:coequaliser2} together into a cube proves that $(f \tinydot[gray dot] g)$ is a bimodule homomorphism.
 
It is left to verify functoriality of $\odot$. This comes down to checking that $(f\tinydot[gray dot]f') \ver ( g \tinydot[gray dot] g') = (f \ver f')\tinydot[gray dot] (g \ver g')$, and $\mathid_{\bf M} \tinydot[gray dot] \mathid_{\bf M'} = \mathid_{\bf M \tinydot[gray dot] M'}$. The two maps that form the first equality are the unique morphisms that make the two diagrams below commute. The outer left squares of the two diagrams are equal by the exchange law for $\ten$ and $\ver$. As a consequence, the dashed arrows must be equal too. A similar argument shows that $\mathid_{\bf M} \tinydot[gray dot] \mathid_{\bf M'} = \mathid_{\bf M \tinydot[gray dot] M'}$.
 
 \begin{equation}
  \begin{pic}[xscale=2.3, yscale=1.5, thin]
    \node (tl) at (-.3,1) {$M \ten B \ten N$};
    \node (bl) at (-.3,0) {$L \ten C \ten K$};
    \node (ld) at (-.3,-1) {$P \ten D \ten Q$};
    \node (t) at (1,1) {$M \ten N$};
    \node (b) at (1,0) {$L \ten K$};
    \node (md) at (1,-1) {$P \ten Q$};
    \node (tr) at (1.7,1) {${M}_{\tinydot[gray dot]} {N}$};
    \node (br) at (1.7,0) {${K}_{\tinydot[gray dot]} {L}$};
    \node (rd) at (1.7,-1) {${M}_{\tinydot[gray dot]} {N}$};
    \draw (tl) to  (-.3,.65);
    \node at (-.3, 0.5) {$f \ten i \ten g$};
    \draw[->] (-.3,.35) to  (bl);
    \draw[->] (t) to node[right] {$f \ten g$} (b);
    \draw[->, dashed] (tr) to node[right] {$f_{\tinydot[gray dot]}g$} (br);
    \draw[->, dashed] (br) to node[right] {$h_{\tinydot[gray dot]}k$} (rd);
    \draw[->] (t) to node[above] {$c_B$} (tr);
    \draw[->] (b) to node[below] {$c_C$} (br);
    \draw[->] ([yshift=1.5pt]tl.east) to node[above] {${}_{\tinydot[white dot]}\mathbf{M} \ten \mathid_{N}$} ([yshift=1.5pt]t.west);
    \draw[->] ([yshift=-1.5pt]tl.east) to node[below] {$\mathid_{M} \ten \mathbf{N}_{\tinydot[black dot]}$} ([yshift=-1.5pt]t.west);
    \draw[->] ([yshift=1.5pt]bl.east) to node[above] {${}_{\tinydot[white dot]}\mathbf{K} \ten \mathid_{L}$} ([yshift=1.5pt]b.west);
    \draw[->] ([yshift=-1.5pt]bl.east) to node[below] {$\mathid_{K} \ten \mathbf{L}_{\tinydot[black dot]}$} ([yshift=-1.5pt]b.west);
    \draw (bl) to  (-.3,-.35);
    \node at (-.3,-.5) {$h \ten j \ten k$};
    \draw[->] (-.3,-.65) to  (ld);
    \draw[->] (b) to node[right] {$h \ten k$} (md);
   % \draw[->, dashed] (tr) to node[right] {$\rho_M \tinydot[gray dot] \rho_N$} (rd);
    \draw[->] (md) to node[above] {$c_D$} (rd);
    \draw[->] ([yshift=1.5pt]ld.east) to node[above] {${}_{\tinydot[white dot]}\mathbf{P} \ten \mathid_{Q}$} ([yshift=1.5pt]md.west);
    \draw[->] ([yshift=-1.5pt]ld.east) to node[below] {$\mathid_{P}  \ten \mathbf{Q}_{\tinydot[black dot]}$} ([yshift=-1.5pt]md.west);
%    \draw[->, dashed] (br) to [in=90, out=90] {$\hat{\xi}^{-1}$} (rd);
  \end{pic}  
   \begin{pic}[xscale=2.3, yscale=1.5, thin]
    \node (tl) at (-.3,1) {$M \ten B \ten N$};
    \node (ld) at (-.3,-1) {$P \ten D \ten Q$};
    \node (t) at (1,1) {$M \ten N$};
    \node (md) at (1,-1) {$P \ten Q$};
    \node (tr) at (1.7,1) {${M}_{\tinydot[gray dot]} {N}$};
    \node (rd) at (1.7,-1) {${M}_{\tinydot[gray dot]} {N}$};
    \draw (tl) to  (-.3,0.25);
    \draw[->] (-.3,-0.25) to  (ld);
    \node at (-.3,0.125) {$(k \ver f) \ten (j \ver i)$};
    \node at (-.3,-.125) {$\ten (k \ver g)$};
    \draw (t) to  (1,0.3);
    \draw[->] (1,0.1) to  (md);
    \node at (1,0.2) {$(h \ver f) \ten (k \ver g)$};
    \draw[dashed] (tr) to (1.7,-0.1);
    \node at (1.7,-.2) {$(h \ver f)_{\tinydot[gray dot]}(k \ver g)$};
    \draw[->, dashed] (1.7,-0.3) to (rd); 
    \draw[->] (t) to node[above] {$c_B$} (tr);
    \draw[->] ([yshift=1.5pt]tl.east) to node[above] {${}_{\tinydot[white dot]}\mathbf{M} \ten \mathid_{N}$} ([yshift=1.5pt]t.west);
    \draw[->] ([yshift=-1.5pt]tl.east) to node[below] {$\mathid_{M} \ten \mathbf{N}_{\tinydot[black dot]}$} ([yshift=-1.5pt]t.west);
    \draw[->] (md) to node[above] {$c_D$} (rd);
    \draw[->] ([yshift=1.5pt]ld.east) to node[above] {${}_{\tinydot[white dot]}\mathbf{P} \ten \mathid_{Q}$} ([yshift=1.5pt]md.west);
    \draw[->] ([yshift=-1.5pt]ld.east) to node[below] {$\mathid_{P}  \ten \mathbf{Q}_{\tinydot[black dot]}$} ([yshift=-1.5pt]md.west);
  \end{pic}
 \end{equation}

The {\bf associator} ${\bf a}: ({\bf M_1} \odot {\bf M_2}) \odot {\bf M_3} \rightarrow {\bf M_1} \odot ({\bf M_2} \odot {\bf M_3})$ is an the extended algebra homomorphism of the form $(\mathid, a, \mathid)$. The morphism $a$ is defined by the commuting diagrams below. Let $\pi_1$ and $\pi_2$ be two different coequalisers of the maps depicted below. Then $a_{M_1,M_2,M_3}$ is the unique isomorphism between their images.
Similarly we derive the component $a_{N_1,N_2,N_3}$. Naturality of $a$ follows from commutativity of the rightmost square of the diagram, which commutes because all other sub diagrams commute. 

\begin{tikzpicture}[xscale=5,yscale=2.5]
 \node (tl) at (0,1) {$M_1 \ten B \ten  M_2 \ten C \ten M_3$};
 \node (bl) at (0,0) {$N_1 \ten D \ten N_2 \ten E \ten N_3$};
  \node (tm) at (1,1) {$M_1 \ten M_2 \ten M_3$};
 \node (bm) at (1,0) {$N_1 \ten N_2 \ten N_3$};
\draw[->] ([yshift=1.5pt] tl.east) to node[above] {$_{\tinydot [gray dot]}{\bf M_1} \ten {}_{\tinydot[gray dot]}{\bf M_2} \ten \mathid_{M_3}$} ([yshift=1.5pt] tm.west);
\draw[->] ([yshift=-1.5pt] tl.east) to node[below] {$\mathid_{M_1} \ten {\bf M_2}_{\tinydot[gray dot]} \ten  {\bf M_3}_{\tinydot[gray dot]}$} ([yshift=-1.5pt] tm.west);
\draw[->] ([yshift=1.5pt] bl.east) to node[above] {$_{\tinydot [gray dot]}{\bf N_1} \ten {}_{\tinydot[gray dot]}{\bf N_2} \ten \mathid_{N_3}$} ([yshift=1.5pt] bm.west);
\draw[->] ([yshift=-1.5pt] bl.east) to node[below] {$\mathid_{N_1} \ten {\bf N_2}_{\tinydot[gray dot]} \ten  {\bf N_3}_{\tinydot[gray dot]}$} ([yshift=-1.5pt] bm.west);
 \node (r1) at (2,1.25) {$({M_1}_{\tinydot[gray dot]} {M_2})_{\tinydot[gray dot]} {M_3}$};
 \node (r2) at (1.6,0.75) {${M_1}_{\tinydot[gray dot]} ({M_2}_{\tinydot[gray dot]} M_3)$};
  \node (r3) at (2,0.25) {$({N_1}_{\tinydot[gray dot]} {N_2})_{\tinydot[gray dot]} N_3$};
 \node (r4) at (1.6,-0.25) {${N_1}_{\tinydot[gray dot]} ({N_2}_{\tinydot[gray dot]} N_3)$};
 \draw[->] (tm) to node[above] {$\pi_1$} (r1.west);
 \draw[->] (tm) to node[below] {$\pi_2$} (r2);
  \draw[->] (bm) to node[above] {$\phi_1$} (r3.west);
 \draw[->] (bm) to node[below] {$\phi_2$} (r4);
  \draw[->, dashed] (r1) to node[left] {$a_{M_1,M_2,M_3}$} (r2);
 \draw[->, dashed] (r3) to node[right] {$a_{N_1,N_2,N_3}$} (r4);
  \draw[->, dashed] (r1) to node[right] {$(f_1 \tinydot[gray dot] f_2) \tinydot[gray dot] f_3$} (r3);
 \draw[->, dashed, cross] (r2) to node[left, yshift=14pt] {$f_1 \tinydot[gray dot] (f_2 \tinydot[gray dot] f_3)$} (r4);
 \draw (tl) to (0,0.6) node[below]{$f_1 \ten f_B \ten f_2 \ten f_C \ten f_3$};
  \draw[->] (0,0.4) to (bl);
 \draw[->] (tm) to node[left] {$f_1 \ten f_2 \ten f_3$} (bm); 
\end{tikzpicture}

The coequalisers $\pi_1$ and $\pi_2$ correspond to the compositions of coequalisers below. 

\begin{equation}
\pi_1 :=
\begin{aligned}
\begin{tikzpicture}[scale=0.75]
\node at (0.75,1.25) {$({M_1}_{\tinydot[gray dot]} {M_2})_{\tinydot[gray dot]} {M_3}$};
\draw (0,0) -- (0,0.5) -- (1.5, 0.5) -- (1.5,0) -- (0,0);
\node at (0.75,0.25) [] {$\tilde{c}_B$};
\draw (0.75,0.5) to  (0.75,1);
\draw (0.3,0) -- (0.3,-1.5)node[left]{$M_1$};
\draw  (1.1,0) to node[right]{${M_2}_{\tinydot[gray dot]}M_3$} (1.1,-0.5);
\draw (0.6,-1) -- (0.6,-0.5) -- (1.65, -0.5) -- (1.65,-1) -- (0.6,-1);
\node at (1,-0.75) [] {$c_C$};
\draw (0.8,-1) to (0.8,-1.5)node[right,xshift=-3pt]{$M_2$}; 
\draw (1.45,-1) -- (1.45,-1.5)node[right]{$M_3$};
\end{tikzpicture}
\end{aligned}
\hspace{.5cm}
\pi_2:=
\begin{aligned}
\begin{tikzpicture}[xscale=-0.75, yscale=0.75]
\node at (0.75,1.25) {${M_1}_{\tinydot[gray dot]} ({M_2}_{\tinydot[gray dot]} {M_3})$};
\draw (0,0) -- (0,0.5) -- (1.5, 0.5) -- (1.5,0) -- (0,0);
\node at (0.75,0.25) [] {$\tilde{c}_C$};
\draw (0.75,0.5) -- (0.75,1);
\draw (0.3,0) -- (0.3,-1.5)node [right]{$M_3$};
\draw  (1.1,0) to node[left]{${M_1}_{\tinydot[gray dot]}M_2$} (1.1,-0.5);
\draw (0.6,-1) -- (0.6,-0.5) -- (1.65, -0.5) -- (1.65,-1) -- (0.6,-1);
\node at (1,-0.75) [] {$c_B$};
\draw (0.8,-1) to (0.8,-1.5)node [left,xshift=3pt]{$M_2$}; 
\draw (1.45,-1) to (1.45,-1.5)node[left]{$M_1$};
\end{tikzpicture}
\end{aligned}
\end{equation}

To verify that $\pi_1$ is a coequalised, we first check that it is a cocone of $_{\tinydot [gray dot]}{\bf M_1} \ten {}_{\tinydot[gray dot]}{\bf M_2} \ten \mathid_{M_3}$ and $\mathid_{M_1} \ten {\bf M_2}_{\tinydot[gray dot]} \ten  {\bf M_3}_{\tinydot[gray dot]}$. This follows from the equalities below.

\begin{equation}
\begin{aligned}
\begin{tikzpicture}[xscale=0.8,yscale=0.7]
\node at (0.75,1.25) {$({M_1}_{\tinydot[gray dot]} {M_2})_{\tinydot[gray dot]} {M_3}$};
\draw (0,0) -- (0,0.5) -- (1.5, 0.5) -- (1.5,0) -- (0,0);
\node at (0.75,0.25) [] {$\tilde{c}_B$};
\draw (0.75,0.5) -- (0.75,1);
\draw (0.3,0) to[in=90,out=-90] (-0.1,-1.5);
\draw  (1.1,0) -- (1.1,-0.5);
\draw (0.6,-1) -- (0.6,-0.5) -- (1.65, -0.5) -- (1.65,-1) -- (0.6,-1);
\node at (1.1,-0.75) [] {$c_C$};
\draw (0.8,-1) to (0.8,-1.5); 
\draw (1.45,-1) to[in=90,out=-90] (1.85,-3);
\draw (-0.5,-2) -- (0.3, -2) -- (0.3, -1.5) -- (-0.5,-1.5) -- (-0.5,-2);
\node at (-0.1, -1.75) {$\bf M_1$};
\draw (0.4,-2) -- (1.2,-2) -- (1.2,-1.5) -- (0.4,-1.5) -- (0.4,-2);
\draw (-0.4,-2) to (-0.4,-2.25) node[gray dot]{};
\draw (-0.1,-2) -- (-0.1,-3);
\draw (0.2,-2) -- (0.2,-3);
\draw (0.5,-2) to (0.5,-2.25) node[gray dot]{};
\draw (0.8,-2) to (0.8,-3);
\draw (1.1,-2) to[in=90,out=-90] (1.3,-3);
\node at (0.8,-1.75) {$\bf M_2$};
\node at (-0.1,-3.25) [] {$M_1$};
\node at (0.3,-3.25) [] {$B$};
\node at (0.8,-3.25) [] {$M_2$};
\node at (1.3,-3.25) [] {$C$};
\node at (1.85,-3.25) [] {$M_3$};
\end{tikzpicture}
\end{aligned}
=
\begin{aligned}
\begin{tikzpicture}[xscale=1.2,yscale=0.7]
\node at (0.75,1.25) {$({M_1}_{\tinydot[gray dot]} {M_2})_{\tinydot[gray dot]} {M_3}$};
\draw (0,0) -- (0,0.5) -- (1.5, 0.5) -- (1.5,0) -- (0,0);
\node at (0.75,0.25) [] {$\tilde{c}_B$};
\draw (0.75,0.5) -- (0.75,1);
\draw (0.3,0) to[in=90, out=-90] (0.1,-3.5);
\draw  (1.1,0) -- (1.1,-0.5);
\draw (0.5,-1) -- (0.5,-0.5) -- (1.7, -0.5) -- (1.7,-1) -- (0.5,-1);
\node at (1.1,-0.75) [] {$\bf {M_2}_{\tinydot[gray dot]}M_3$};
\draw (0.6,-1) to[in=90,out=-90] (0.5,-3.5); 
\draw (1.1,-1) -- (1.1,-1.5);
\draw (1.6,-1) to (1.6,-1.25) node[gray dot] {};
\draw (0.7,-2) -- (1.7,-2) -- (1.7,-1.5) -- (0.7,-1.5) -- (0.7,-2);
\node at (1.1,-1.75) {$c_C$};
\draw (0.9,-2) -- (0.9,-3.5);
\draw (1.5,-2) -- (1.5,-2.5);
\draw (1.1,-3) -- (1.9,-3) -- (1.9,-2.5) -- (1.1,-2.5) -- (1.1,-3);
\node at (1.5,-2.75) {$\bf M_3$};
\draw (1.2,-3) -- (1.2,-3.5);
\draw (1.5,-3) -- (1.5,-3.5);
\draw (1.8,-3) to (1.8,-3.25) node[gray dot]{};
\node at (0.1,-3.75) [] {$M_1$};
\node at (0.5,-3.75) [] {$B$};
\node at (0.9,-3.75) [] {$M_2$};
\node at (1.2,-3.75) [] {$C$};
\node at (1.5,-3.75) [] {$M_3$};
\end{tikzpicture}
\end{aligned}
=
\begin{aligned}
\begin{tikzpicture}[xscale=1.2,yscale=0.7]
\node at (0.75,1.25) {$({M_1}_{\tinydot[gray dot]} {M_2})_{\tinydot[gray dot]} {M_3}$};
\draw (0,0) -- (0,0.5) -- (1.5, 0.5) -- (1.5,0) -- (0,0);
\node at (0.75,0.25) [] {$\tilde{c}_B$};
\draw (0.75,0.5) -- (0.75,1);
\draw (0.2,0) to[in=90, out=-90] (-0.2,-3.5);
\draw  (1.1,0) -- (1.1,-0.5);
\draw (0.5,-1) -- (0.5,-0.5) -- (1.7, -0.5) -- (1.7,-1) -- (0.5,-1);
\node at (1.1,-0.75) [] {$c_C$};
\draw (0.6,-1) -- (0.6,-1.5); 
\draw (1.6,-1) -- (1.6,-1.5);
\draw (0.2,-2) -- (1,-2) -- (1,-1.5) -- (0.2,-1.5) -- (0.2,-2);
\node at (0.6,-1.75) {$\bf M_2$};
\draw (0.3,-2) to [in=90,out=-90] (0.2,-3.5);
\draw (0.6,-2) -- (0.6,-3.5);
\draw (0.9,-2) to (0.9,-2.25) node[gray dot] {};
\draw (1.2,-2) -- (2,-2) -- (2,-1.5) -- (1.2,-1.5) -- (1.2,-2);
\node at (1.6,-1.75) {$\bf M_3$};
\draw (1.3,-2) to (1.3,-2.25) node[gray dot] {};
\draw (1.6,-2) -- (1.6,-2.5);
\draw (1.9,-2) to (1.9,-2.25) node[gray dot] {};
\draw (1.1,-3) -- (1.9,-3) -- (1.9,-2.5) -- (1.1,-2.5) -- (1.1,-3);
\node at (1.5,-2.75) {$\bf M_3$};
\draw (1.2,-3) to[in=90,out=-90] (1.1,-3.5);
\draw (1.5,-3) -- (1.5,-3.5);
\draw (1.8,-3) to (1.8,-3.25) node[gray dot]{};
\node at (-0.2,-3.75) [] {$M_1$};
\node at (0.2,-3.75) [] {$B$};
\node at (0.6,-3.75) [] {$M_2$};
\node at (1.1,-3.75) [] {$C$};
\node at (1.5,-3.75) [] {$M_3$};
\end{tikzpicture}
\end{aligned}
=
\begin{aligned}
\begin{tikzpicture}[xscale=1.2,yscale=0.7]
\node at (0.75,1.25) {$({M_1}_{\tinydot[gray dot]} {M_2})_{\tinydot[gray dot]} {M_3}$};
\draw (0,0) -- (0,0.5) -- (1.5, 0.5) -- (1.5,0) -- (0,0);
\node at (0.75,0.25) [] {$\tilde{c}_B$};
\draw (0.75,0.5) -- (0.75,1);
\draw (0.2,0) to[in=90, out=-90] (-0.2,-3);
\draw  (1.1,0) -- (1.1,-0.5);
\draw (0.5,-1) -- (0.5,-0.5) -- (1.7, -0.5) -- (1.7,-1) -- (0.5,-1);
\node at (1.1,-0.75) [] {$c_C$};
\draw (0.6,-1) -- (0.6,-1.5); 
\draw (1.6,-1) -- (1.6,-1.5);
\draw (0.2,-2) -- (1,-2) -- (1,-1.5) -- (0.2,-1.5) -- (0.2,-2);
\node at (0.6,-1.75) {$\bf M_2$};
\draw (0.3,-2) to [in=90,out=-90] (0.2,-3);
\draw (0.6,-2) -- (0.6,-3);
\draw (0.9,-2) to (0.9,-2.25) node[gray dot] {};
\draw (1.2,-2) -- (2,-2) -- (2,-1.5) -- (1.2,-1.5) -- (1.2,-2);
\node at (1.6,-1.75) {$\bf M_3$};
\draw (1.3,-2) to [in=90, out=-90] (1.2,-3) {};
\draw (1.6,-2) -- (1.6,-3);
\draw (1.9,-2) to (1.9,-2.25) node[gray dot] {};
\node at (-0.2,-3.25) [] {$M_1$};
\node at (0.2,-3.25) [] {$B$};
\node at (0.6,-3.25) [] {$M_2$};
\node at (1.2,-3.25) [] {$C$};
\node at (1.6,-3.25) [] {$M_3$};
\end{tikzpicture}
\end{aligned}
\end{equation}

The last step for showing that $\pi_1$ is a coequaliser, is to prove that if we have another cocone $f$, then $f$ factorises through $\pi_1$. Let $f$ be such a cocone. First of all, it is easy to show that the two equalities below hold, expressing that $f$ is a cocone of the respective pairs of morphisms. One obtains this by plugging  the unit on $B$ or $C$, respectively, in the cocone equality for $f$ and then applying the unit laws for modules.

\begin{equation}
\begin{aligned}
\begin{tikzpicture}[xscale=0.8,yscale=0.7]
\node at (0.75,1.25) {$({M_1}_{\tinydot[gray dot]} {M_2})_{\tinydot[gray dot]} {M_3}$};
\draw (0,0) -- (0,0.5) -- (1.5, 0.5) -- (1.5,0) -- (0,0);
\node at (0.75,0.25) [] {$f$};
\draw (0.75,0.5) -- (0.75,1);
\draw (0.3,0) to[in=90,out=-90] (-0.1,-3);
\draw (0.8,0) to (0.8,-1.5); 
\draw (1.35,0) to[in=90,out=-90] (1.85,-3);
\draw (0.4,-2) -- (1.2,-2) -- (1.2,-1.5) -- (0.4,-1.5) -- (0.4,-2);
\draw (0.5,-2) to (0.5,-2.25) node[gray dot]{};
\draw (0.8,-2) to (0.8,-3);
\draw (1.1,-2) to[in=90,out=-90] (1.3,-3);
\node at (0.8,-1.75) {$\bf M_2$};
\node at (-0.1,-3.25) [] {$M_1$};
\node at (0.8,-3.25) [] {$M_2$};
\node at (1.3,-3.25) [] {$C$};
\node at (1.85,-3.25) [] {$M_3$};
\end{tikzpicture}
\end{aligned}
=
\begin{aligned}
\begin{tikzpicture}[xscale=0.8, yscale=0.7]
\node at (0.75,1.25) {$({M_1}_{\tinydot[gray dot]}{M_2})_{\tinydot[gray dot]} {M_3}$};
\draw (0,0) -- (0,0.5) -- (1.5, 0.5) -- (1.5,0) -- (0,0);
\node at (0.75,0.25) [] {$f$};
\draw (0.75,0.5) -- (0.75,1);
\draw (0.2,0) to[in=90, out=-90] (-0.2,-3);
\draw (0.6,0) -- (0.6,-3); 
\draw (1.35,0) to [in=90,out=-90] (1.6,-1.5);
\draw (0.6,-2) -- (0.6,-3);
\draw (1.2,-2) -- (2,-2) -- (2,-1.5) -- (1.2,-1.5) -- (1.2,-2);
\node at (1.6,-1.75) {$\bf M_3$};
\draw (1.3,-2) to [in=90, out=-90] (1.2,-3) {};
\draw (1.6,-2) -- (1.6,-3);
\draw (1.9,-2) to (1.9,-2.25) node[gray dot] {};
\node at (-0.2,-3.25) [] {$M_1$};
\node at (0.6,-3.25) [] {$M_2$};
\node at (1.2,-3.25) [] {$C$};
\node at (1.6,-3.25) [] {$M_3$};
\end{tikzpicture}
\end{aligned}
\hspace{1cm}
\begin{aligned}
\begin{tikzpicture}[xscale=0.8,yscale=0.7]
\node at (0.75,1.25) {$({M_1}_{\tinydot[gray dot]} {M_2})_{\tinydot[gray dot]} {M_3}$};
\draw (0,0) -- (0,0.5) -- (1.5, 0.5) -- (1.5,0) -- (0,0);
\node at (0.75,0.25) [] {$f$};
\draw (0.75,0.5) -- (0.75,1);
\draw (0.3,0) to[in=90,out=-90] (-0.1,-1.5);
\draw (0.8,0) to (0.8,-3); 
\draw (1.35,0) to[in=90,out=-90] (1.85,-3);
\draw (-0.5,-2) -- (0.3, -2) -- (0.3, -1.5) -- (-0.5,-1.5) -- (-0.5,-2);
\node at (-0.1, -1.75) {$\bf M_1$};
\draw (-0.4,-2) to (-0.4,-2.25) node[gray dot]{};
\draw (-0.1,-2) -- (-0.1,-3);
\draw (0.2,-2) -- (0.2,-3);
\node at (-0.1,-3.25) [] {$M_1$};
\node at (0.3,-3.25) [] {$B$};
\node at (0.8,-3.25) [] {$M_2$};
\node at (1.85,-3.25) [] {$M_3$};
\end{tikzpicture}
\end{aligned}
=
\begin{aligned}
\begin{tikzpicture}[xscale=0.8,yscale=0.7]
\node at (0.75,1.25) {$({M_1}_{\tinydot[gray dot]} {M_2})_{\tinydot[gray dot]} {M_3}$};
\draw (0,0) -- (0,0.5) -- (1.5, 0.5) -- (1.5,0) -- (0,0);
\node at (0.75,0.25) [] {$f$};
\draw (0.75,0.5) -- (0.75,1);
\draw (0.2,0) to[in=90, out=-90] (-0.2,-3);
\draw (0.6,0) -- (0.6,-1.5); 
\draw (1.35,0) to [in=90,out=-90] (1.6,-3);
\draw (0.2,-2) -- (1,-2) -- (1,-1.5) -- (0.2,-1.5) -- (0.2,-2);
\node at (0.6,-1.75) {$\bf M_2$};
\draw (0.3,-2) to [in=90,out=-90] (0.2,-3);
\draw (0.6,-2) -- (0.6,-3);
\draw (0.9,-2) to (0.9,-2.25) node[gray dot] {};
\node at (-0.2,-3.25) [] {$M_1$};
\node at (0.2,-3.25) [] {$B$};
\node at (0.6,-3.25) [] {$M_2$};
\node at (1.6,-3.25) [] {$M_3$};
\end{tikzpicture}
\end{aligned}
\end{equation}



It follows from the first equality that $f$ factorises through $\id_{M_1} \ten c_{C}$. This means that there exists a morphism $f'$ such that $f = f' \verc (\id_{M_1} \ten c_{C})$. 
From the second statement we deduce that $f'$ factorises through $\tilde{c}_B$, which finishes our proof. The equations below and the fact that $\mathid \ten c_C$ is epic imply that $f'$ is a cocone of $_{\tinydot[gray dot]}{\bf M_1} \otimes \id_{M_2} \otimes \id_{M_3}$ and $\id_{M_1} \otimes {\bf M_{2\tinydot[gray dot]}M_3}_{\tinydot[gray dot]}$. 
As a result, $f'$ factorises through  $\id_{M_1} \ten \tilde{c}_B$. Hence, there exists a morphism $f''$ such that $f = f'' \verc \tilde{c}_B \verc  (\id_{M_1} \ten c_C) = f'' \verc \pi_1$. One can use a similar argument to show that $\pi_2$ is a coequaliser.  
 
 
\begin{equation}
\begin{aligned}
\begin{tikzpicture}[yscale=0.78, xscale=0.7]
\draw (2,2.5) -- (2,3);
\draw (0.8,2) -- (3.2,2) -- (3.2,2.5) -- (0.8,2.5) -- (0.8,2);
\node at (2,2.25) {$f'$};
\draw (1,2) -- (1,1.5);
\draw (2.5,2) -- (2.5,0.5);
\draw (0.6,1) -- (1.4,1) -- (1.4,1.5) -- (0.6,1.5) -- (0.6,1);
\node at (1,1.25) {$\bf M_1$};
\draw (1.8,0) -- (3.2,0) -- (3.2,0.5) -- (1.8,0.5) -- (1.8,0);
\node at (2.5,0.25) {$c_C$};
\draw (0.7,1) to (0.7,0.75) node[gray dot] {};
\draw (1,1) to (1,-0.5);
\draw (1.3,1) to [in=90,out=-90] (1.5,-0.5);
\node at (1,-0.75) {$M_1$};
\node at (1.5,-0.75) {$B$};
\draw (2,0) -- (2,-0.5);
\draw (3,0) -- (3,-0.5);
\node at (2,-0.75) {$M_2$};
\node at (3,-0.75) {$M_3$};
\end{tikzpicture}
\end{aligned}
=
\begin{aligned}
\begin{tikzpicture}[yscale=0.78, xscale=0.7]
\draw (2,2.5) -- (2,3);
\draw (0.8,2) -- (3.2,2) -- (3.2,2.5) -- (0.8,2.5) -- (0.8,2);
\node at (2,2.25) {$f$};
\draw (1,2) -- (1,1.5);
\draw (2,2) -- (2,-0.5);
\draw (3,2) -- (3,-0.5);
\draw (0.6,1) -- (1.4,1) -- (1.4,1.5) -- (0.6,1.5) -- (0.6,1);
\node at (1,1.25) {$\bf M_1$};
\draw (0.7,1) to (0.7,0.75) node[gray dot] {};
\draw (1,1) to (1,-0.5);
\draw (1.3,1) to [in=90,out=-90](1.5,-0.5);
\node at (1,-0.75) {$M_1$};
\node at (1.5,-0.75) {$B$};
\node at (2,-0.75) {$M_2$};
\node at (3,-0.75) {$M_3$};
\end{tikzpicture}
\end{aligned}
=
\begin{aligned}
\begin{tikzpicture}[yscale=0.78, xscale=0.7]
\draw (2,2.5) -- (2,3);
\draw (0.8,2) -- (3.2,2) -- (3.2,2.5) -- (0.8,2.5) -- (0.8,2);
\node at (2,2.25) {$f$};
\draw (1,2) -- (1,-0.5);
\draw (2,2) -- (2,1.5);
\draw (3,2) -- (3,1.5);
\draw (1.6,1) -- (2.4,1) -- (2.4,1.5) -- (1.6,1.5) -- (1.6,1);
\node at (2,1.25) {$\bf M_2$};
\draw (2.6,1) -- (3.4,1) -- (3.4,1.5) -- (2.6,1.5) -- (2.6,1);
\node at (3,1.25) {$\bf M_3$};
\draw (1.7,1) to [in=90,out=-90](1.5,-0.5);
\draw (2,1) -- (2,-0.5);
\draw (2.3,1) to (2.3,0.75) node[gray dot] {};
\node at (1,-0.75) {$M_1$};
\draw (2.7,1) to (2.7,0.75)node[gray dot]{};
\draw (3,1) -- (3,-0.5);
\draw (3.3,1) to (3.3,0.75) node[gray dot] {};
\node at (1.5,-0.75) {$B$};
\node at (2,-0.75) {$M_2$};
\node at (3,-0.75) {$M_3$};
\end{tikzpicture}
\end{aligned}
=
\begin{aligned}
\begin{tikzpicture}[yscale=0.78, xscale=0.7]
\draw (2,2.5) -- (2,3);
\draw (0.8,2) -- (3.2,2) -- (3.2,2.5) -- (0.8,2.5) -- (0.8,2);
\node at (2,2.25) {$f'$};
\draw (1,2) -- (1,-0.5);
\draw (2.5,2) -- (2.5,1.5);
\draw (1.8,1.5) -- (3.2,1.5) -- (3.2,1) -- (1.8,1) -- (1.8,1.5);
\node at (2.5,1.25){$c_C$};
\draw (2,1) -- (2,0.5);
\draw (3,1) -- (3,0.5);
\draw (1.6,0) -- (2.4,0) -- (2.4,0.5) -- (1.6,0.5) -- (1.6,0);
\node at (2,0.25) {$\bf M_2$};
\draw (2.6,0) -- (3.4,0) -- (3.4,0.5) -- (2.6,0.5) -- (2.6,0);
\node at (3,0.25) {$\bf M_3$};
\draw (1.7,0) to [in=90,out=-90] (1.5,-0.5);
\draw (2,0) -- (2,-0.5);
\draw (2.3,0) to (2.3,-0.25) node[gray dot] {};
\node at (1,-0.75) {$M_1$};
\draw (2.7,0) to (2.7,-0.25)node[gray dot]{};
\draw (3,1) -- (3,-0.5);
\draw (3.3,0) to (3.3,-0.25) node[gray dot] {};
\draw (2,0) -- (2,-0.5);
\draw (3,0) -- (3,-0.5);
\node at (1.5,-0.75) {$B$};
\node at (2,-0.75) {$M_2$};
\node at (3,-0.75) {$M_3$};
\end{tikzpicture}
\end{aligned}
=
\begin{aligned}
\begin{tikzpicture}[yscale=0.78, xscale=0.7]
\draw (2,2.5) -- (2,3);
\draw (0.8,2) -- (3.2,2) -- (3.2,2.5) -- (0.8,2.5) -- (0.8,2);
\node at (2,2.25) {$f'$};
\draw (1,2) -- (1,-0.5);
\draw (2.5,2) -- (2.5,1.5);
\draw (1.5,1.5) -- (3.5,1.5) -- (3.5,1) -- (1.5,1) -- (1.5,1.5);
\node at (2.5,1.25) {${\bf M_2}_{\tinydot[gray dot]}{\bf M_3}$};
\draw (2.5,1) -- (2.5,0.5);
\draw (1.6,1) to [in=90,out=-90] (1.5,-0.5);
\draw (3.4,1) to (3.4,0.5); 
\node at (3.4,0.5) {$\tinydot[gray dot]$};
\draw (2,0) to (2,-0.5);
\draw (3,0) to (3,-0.5);
\draw (1.8, 0.5) -- (3.2,0.5) -- (3.2,0) -- (1.8,0) -- (1.8,0.5);
\node at (2.5,0.25) {$c_C$};
\node at (1.5,-0.75) {$B$};
\node at (1,-0.75) {$M_1$};
\node at (2,-0.75) {$M_2$};
\node at (3,-0.75) {$M_3$};
\end{tikzpicture}
\end{aligned}
\end{equation}



The components of the {\bf left unitor} $l_{\bf M}: U_A \odot {\bf M} \rightarrow {\bf M}$ are given by $(\mathid, l_{M}, \mathid)$, where $ l_{M}$ is defined as follows: By the module laws, ${\bf M}_{\tinydot[gray dot]}: A \ten M \rightarrow M$ is a coequaliser of the maps $_{\tinydot[gray dot]}{\bf A} \ten \mathid_{M}$ and $\mathid_{A} \ {\bf M}_{\tinydot[gray dot]}$. The canonical isomorphisms between ${\bf M}$ and ${\bf A}_{\tinydot[gray dot]}{\bf M}$ is the middle component of $l_{\bf M}$, and naturality of $l$ follows from commutativity of the diagram below. We can define $r$, and prove its naturality, in a similar way.
    
\begin{tikzpicture}[xscale=4,yscale=2]
 \node (tl) at (0,1) {$A \ten A \ten M$};
 \node (bl) at (0,0) {$B \ten B \ten N$};
  \node (tm) at (1,1) {$A \ten M$};
 \node (bm) at (1,0) {$B \ten N$};
\draw[->] ([yshift=1.5pt] tl.east) to node[above] {${}_{\tinydot[gray dot]}U_{A} \ten \mathid_M$} ([yshift=1.5pt] tm.west);
\draw[->] ([yshift=-1.5pt] tl.east) to node[below] {$\mathid_{A} \ten {\bf M}_{\tinydot[gray dot]}$} ([yshift=-1.5pt] tm.west);
\draw[->] ([yshift=1.5pt] bl.east) to node[above] {$_{\tinydot[gray dot]}U_A \ten \mathid_N$} ([yshift=1.5pt] bm.west);
\draw[->] ([yshift=-1.5pt] bl.east) to node[below] {$\mathid_{B} \ten {N}_{\tinydot[gray dot]}$} ([yshift=-1.5pt] bm.west);
 \node (r1) at (2,1.25) {$A_{\tinydot[gray dot]} M$};
 \node (r2) at (1.6,0.75) {$M$};
  \node (r3) at (2,0.25) {$B_{\tinydot[gray dot]} N$};
 \node (r4) at (1.6,-0.25) {$N$};
 \draw[->] (tm) to node[above] {$c_M$} (r1.west);
 \draw[->] (tm) to node[below] {${\bf M}_{\tinydot[gray dot]}$} (r2);
  \draw[->] (bm) to node[above] {$c_N$} (r3.west);
 \draw[->] (bm) to node[below] {${\bf N}_{\tinydot[gray dot]}$} (r4);
  \draw[->, dashed] (r1) to node[left] {$l_M$} (r2);
 \draw[->, dashed] (r3) to node[right] {$l_N$} (r4);
  \draw[->, dashed] (r1) to node[right] {$f \tinydot[gray dot] g$} (r3);
 \draw[->, dashed, cross] (r2) to node[left, yshift=10pt] {$g$} (r4);
 \draw[->] (tl) to node[left] {$f \ten f \ten g$} (bl);
 \draw[->] (tm) to node[left] {$f \ten g$} (bm);
\end{tikzpicture}
\end{proof}


\subsection{The braided monoidal structure of $\Alg$}\label{sec:brmonDalg}

In this section we will prove that $\Alg({\bf C})$ is a braided monoidal double category and that it is symmetric whenever ${\bf C}$ is symmetric.

\begin{prop}\label{lem:algsymmon}
Let ${\bf C}$ be a braided or symmetric monoidal category, the double category $\Alg({\cat C})$ of algebras, bimodules and bimodule homomorphisms in {\bf C} is braided or symmetric monoidal, respectively.
\end{prop}

\begin{proof}

We have seen that $\D_0$ and $\D_1$ are braided or symmetric monoidal when ${\bf C}$ is braided or symmetric. Clearly, $S$ and $T$ are strict monoidal functors preserving the associativity and unit constraints, and $U_I$ is the monoidal unit of $D_1$, when $I$ is the monoidal unit of $D_0$.

Unfolding definitions shows that the globular isomorphism $\fu$ from $U \circ \tens$ to $\tens \circ (U \times U)$ is the identity. We define the exchange law, ${\bf \fx}$, between the functors $\odot$ and $\tens$ as $(\mathid, \xi, \mathid)$, where $\xi$ is the unique morphism completing the commuting diagram below.
In this diagram, $c$, $c_M$, and $c_N$ are the coequalisers defining  $({M_1} \ten {N_1})_{\tinydot[gray dot]}({M_2} \ten {N_2})$, ${M_1}_{\tinydot[gray dot]} M_2$, and ${N_1}_{\tinydot[gray dot]} N_2$, respectively. By a similar argument given while defining the associator ${\bf a}$ for the double category, one can show that $c_M \ten c_N$ is the coequaliser of the lower two parallel maps in the diagram. The two leftmost vertical morphisms are natural isomorphisms constructed from swam maps. For later use, we will denote the middle morphism by $\hat{\sigma}$.

\begin{equation}\label{def:xi}
  \begin{tikzpicture}[xscale=5, yscale=2, thin]
    \node (tl) at (-.3,1) {$(M_1 \tens M_2) \ten (B_1 \tens B_2) \ten (N_1 \tens N_2)$};
    \node (bl) at (-.3,0) {$(M_1 \ten B_1 \ten N_1) \tens (M_2 \ten B_2 \ten N_2)$};
    \node (t) at (1,1) {$(M_1 \tens M_2) \ten (N_1 \tens N_2)$};
    \node (b) at (1,0) {$(M_1 \ten N_1) \tens (M_2 \ten N_2)$};
    \node (tr) at (1.7,1) {$({M_1} \tens {N_1}) \tinydot[gray dot] ({M_2} \tens {N_2})$};
    \node (br) at (1.7,0) {$({M_1} \tinydot[gray dot] {M_2}) \tens ({N_1} \tinydot[gray dot] {N_2})$};
    \draw[->] (tl) to node[left] {$\iso$} (bl);
    \draw[->] (t) to node[right] {$\iso \hat{\sigma} $} (b);
    \draw[->, dashed] (tr) to node[right] {$\iso {\xi} $} (br);
    \draw[->] (t) to node[above] {c} (tr);
    \draw[->] (b) to node[below] {$c_M \tens c_N$} (br);
    \draw[->] ([yshift=1.5pt]tl.east) to node[above] {${}_{\tinydot[white dot]}(\mathbf{M_1} \tens \mathbf{M_2}) \ten \mathid_{N_1 \tens N_2}$} ([yshift=1.5pt]t.west);
    \draw[->] ([yshift=-1.5pt]tl.east) to node[below] {$\mathid_{M_1 \tens M_2} \ten (\mathbf{N_1} \tens \mathbf{N_2})_{\tinydot[black dot]}$} ([yshift=-1.5pt]t.west);
    \draw[->] ([yshift=1.5pt]bl.east) to node[above] {$
    ({}_{\tinydot[white dot]}\mathbf{M_1} \ten \mathid_{N_1}) \tens ({}_{\tinydot[white dot]}\mathbf{M_2} \ten \mathid_{N_2})$} ([yshift=1.5pt]b.west);
    \draw[->] ([yshift=-1.5pt]bl.east) to node[below] {$(\mathid_{M_1} \ten \mathbf{N_1}_{\tinydot[black dot]}) \tens (\mathid_{M_2} \ten \mathbf{N_2}_{\tinydot[black dot]})$} ([yshift=-1.5pt]b.west);
  \end{tikzpicture}
 \end{equation}


It is left to prove is that the diagrams given in points (iv),(v),(vi), and (ix) of Definition ~\ref{def:symmondoub} commute. Because $\fu$ and $\fx$ are globular, it suffices to verify commutativity of the diagrams for the middle component of the extended module homomorphisms.

The first equation of (iv) corresponds to the front square of the diagram below. The unlabelled arrows represent the coequalisers defining their target object. The back square commutes by coherence of the braided monoidal category {\cat C}. The squares on the sides commute by definition of  $\xi$, ${\bf a}$ and $\tinydot[gray dot]$. It follows that the front square must commute as well, which is what we wanted to prove.
\begin{equation}
\begin{tikzpicture}[xscale=3.2, yscale=2]
%lowersquare
\node (b1) at (0,0) [] {$(M_1 \tens N_1) \hor ((M_2 \hor M_3) \tens (N_2 \hor N_3))$};
\node (b2) at (2,0) [] {$(M_1 \hor (M_2 \hor M_3) \tens (N_1 \hor (N_2 \hor N_3))$};
\node (b3) at (1,-0.5) [] {$(M_1 \tens {N_1})_{\tinydot[gray dot]} (({M_2}_{\tinydot[gray dot]} M_3) \tens ({N_2}_{\tinydot[gray dot]} N_3))$};
\node (b4) at (3,-0.5) [] {$({M_1}_{\tinydot[gray dot]} ({M_2}_{\tinydot[gray dot]} M_3 )) \tens ({N_1}_{\tinydot[gray dot]} ({N_2}_{\tinydot[gray dot]} N_3))$}; 
\draw[->] (b1) to node[above, xshift=7pt] {$\hat{\sigma}$} (b2);
\draw[->] (b3) to node[above] {$\xi$} (b4);
\draw[->] (b1) to node[above] {} (b3); 
\draw[->] (b2) to node[above] {} (b4);
%middlesquare
\node (m1) at (0,1.5) [] {$(M_1 \tens N_1) \hor ((M_2 \tens N_2)) \hor (M_3 \tens N_3))$};
\node (m2) at (2,1.5) [] {$((M_1 \hor M_2) \hor M_3) \tens ((N_1 \hor N_2) \hor N_3)$};
\node (m3) at (1,1) [] {$(M_1 \tens {N_1})_{\tinydot[gray dot]} ((M_2 \tens {N_2})_{\tinydot[gray dot]} (M_3 \tens N_3))$};
\node (m4) at (3,1) [] {$({M_1}_{\tinydot[gray dot]} {M_2})_{\tinydot[gray dot]} {M_3}) \tens (({N_1}_{\tinydot[gray dot]}N_2)_{\tinydot[gray dot]} N_3)$};
\draw[->] (m1) to node[above] {} (m3); 
\draw[->] (m2) to node[above] {} (m4);
%lowervertical
\draw[->] (m1) to node[left] {$\mathid \hor \hat{\sigma}$} (b1); 
\draw[->] (m2) to node[left] {$\alpha \tens \alpha$} (b2);
\draw[->, cross] (m3) to node[left, yshift=8pt] {$\mathid \tinydot[gray dot] \xi$} (b3); 
\draw[->] (m4) to node[left] {$a \tens a$} (b4);
%uppersquare
\node (u1) at (0,3) [] {$((M_1 \tens N_1) \hor (M_2 \tens N_2)) \hor (M_3 \tens N_3)$};
\node (u2) at (2,3) [] {$((M_1 \hor M_2) \tens (N_1 \hor N_2)) \hor (M_3 \tens N_3)$};
\node (u3) at (1,2.5) [] {$((M_1 \tens N_1)_{\tinydot[gray dot]}(M_2 \tens N_2))_{\tinydot[gray dot]}(M_3 \tens N_3)$};
\node (u4) at (3,2.5) [] {$(({M_1}_{\tinydot[gray dot]} M_2) \tens ({N_1}_{\tinydot[gray dot]} N_2))_{\tinydot[gray dot]} (M_3 \tens N_3)$};
\draw[->] (u1) to node[left] {$\alpha$} (m1); 
\draw[->] (u2) to node[left] {$\hat{\sigma}$} (m2);
\draw[->, cross] (u3) to node[left] {$a$} (m3); 
\draw[->] (u4) to node[left] {$\xi$} (m4);
\draw[->, cross] (u3) to node[above, xshift=8pt] {$\xi_{\tinydot[gray dot]} \mathid$} (u4); 
\draw[->] (u1) to node[above] {$\hat{\sigma} \hor \mathid$} (u2);
\draw[->] (u1) to node[below] {} (u3); 
\draw[->, cross] (u2) to node[above] {} (u4);
\end{tikzpicture}
\end{equation}

All other equations concerning $\xi$ are proven in the same way. These are the second and third equation of (iv), the first equation of (v), the first and the third equation of (vi), and the first equation of (ix).
It is easy to see that the other equations, which concern $\fu$ hold, keeping in mind that $\fu$ is the identity, and $U_{\alpha_{A,B,C}} = \alpha_{U_A,U_B,U_C}$, $U(\rho_A) = \rho_{U_A}$, $U(\lambda_A) = \lambda_{U_A}$, and $\sigma_{U_A,U_B} = U(\sigma_{A,B})$.
\end{proof}

\subsection{Companions and conjoints in $\Alg$}\label{sec:fibDalg}
Finally, we will prove that $\Alg(\cat C$) is fibrant. this is the last step we need to take to show that we can lift $\Alg({\cat C})$ to obtain the bicategory ${\cA}lg({\cat C})$, which preserves any braided monoidal structure.

\begin{lem}\label{lem:algfib}
Let ${\bf C}$ be a braided monoidal category. The double category $\Alg({\cat C})$ is fibrant.
\end{lem}

\begin{proof}
Let $f$ be a morphism of ${\lD_0}$. The companion of $f$ is defined by the following data: 

\begin{equation}\label{eq:companion}
\hat{\bf f}:= 
\begin{aligned}
 \begin{tikzpicture}[scale=0.6] 
\draw (0.5,-0.5)node[below]{$A$} -- (0.5, -0.2);
\draw (0.2,-0.2) -- (0.8,-0.2) -- (0.8,0.7) -- (0.2,0.7) -- (0.2,-0.2);
\draw (0.5,0.7) to[out=90,in=160] (1.2,1.2) node[gray dot] {};
\draw (1.2,-0.5) node[below]{$B$} to (1.2,1.2);
\draw (1.9,0.7) to[out=90, in=20] (1.2,1.2){};
\draw (1.9,0.7)  to (1.9,-0.5)node [below]{$B$};
\draw (1.2,1.9) -- (1.2,1.2){};
\node at (0.5,0.25)[] {$f$};
\node at (1.2,2.1)[]{$B$};
\end{tikzpicture}
\end{aligned}
\quad \quad {\bf \eta_{\hat{f}}} := (id_A, f, f)
\quad \quad {\bf \epsilon_{\hat{f}}} := (f, \mathid_B, \mathid_B)
\end{equation}

The first equation of ~\ref{eq:compeqn} holds because ${\bf \epsilon_{\hat{f}} \circ \eta_{\hat{f}}} = (f,f,f) = {\bf U}_f$. The left-hand-side of the second equation corresponds to ${\bf \eta_{\hat{f}} \odot \epsilon_{\hat{f}}}= (id_A, f\tinydot[gray dot] \mathid_B, \mathid_B): {{\bf U}_A}_{\tinydot[gray dot]} \hat{\bf f} \rightarrow \hat{\bf f}_{\tinydot[gray dot]}{\bf U}_B$, where $f \tinydot[gray dot] \mathid_B$: is the unique map that makes back of the diagram below commute.  

\begin{tikzpicture}[xscale=4,yscale=2]
 \node (tl) at (0,1) {$A \ten A \ten B$};
 \node (bl) at (0,0) {$B \ten B \ten B$};
  \node (tm) at (1,1) {$A \ten B$};
 \node (bm) at (1,0) {$B \ten B$};
\draw[->] ([yshift=1.5pt] tl.east) to node[above] {${}_{\tinydot[white dot]}U_A \ten \mathid_B$} ([yshift=1.5pt] tm.west);
\draw[->] ([yshift=-1.5pt] tl.east) to node[below] {$\mathid_{A} \ten \hat{f}_{\tinydot[gray dot]}$} ([yshift=-1.5pt] tm.west);
\draw[->] ([yshift=1.5pt] bl.east) to node[above] {$_{\tinydot[white dot]}\hat{f} \ten \mathid_B$} ([yshift=1.5pt] bm.west);
\draw[->] ([yshift=-1.5pt] bl.east) to node[below] {$\mathid_{B} \ten {U_B}_{\tinydot[gray dot]}$} ([yshift=-1.5pt] bm.west);
 \node (r1) at (2,1.25) {$A_{\tinydot[gray dot]} B$};
 \node (r2) at (1.6,0.75) {$B$};
  \node (r3) at (2,0.25) {$B_{\tinydot[gray dot]} B$};
 \node (r4) at (1.6,-0.25) {$B$};
 \draw[->] (tm) to node[above] {$c_A$} (r1.west);
 \draw[->] (tm) to node[below] {$\hat{f}_{\tinydot[gray dot]}$} (r2);
  \draw[->] (bm) to node[above,xshift=-4pt] {$c_B$} (r3.west);
 \draw[->] (bm) to node[below] {$_{\tinydot[white dot]}\hat{f}$} (r4);
  \draw[->, dashed] (r1) to node[left] {$l_{\hat{f}}$} (r2);
 \draw[->, dashed] (r3) to node[right] {$r_{\hat{f}}$} (r4);
  \draw[->, dashed] (r1) to node[right] {$f \tinydot[gray dot] \mathid_B$} (r3);
 \draw[->, dashed, cross] (r2) to node[right,yshift=7pt] {$\mathid_B$} (r4);
 \draw[->] (tl) to node[left] {$f \ten f \ten \mathid_B$} (bl);
 \draw[->] (tm) to node[left] {$f \ten \mathid_B$} (bm);
\end{tikzpicture}

We need to show that $f\tinydot[gray dot] \id_B \iso  \mathid_B$. In this diagram, $c_A$ and $c_B$ are the coequalisers defining ${{\bf U}_A}_{\tinydot[gray dot]} \hat{\bf f}$ and $\hat{\bf f}_{\tinydot[gray dot]}{\bf U}_B$, respectively.
It is easy to check that $\hat{f}_{\tinydot[gray dot]}$ and $_{\tinydot[white dot]}{\hat{f}}$ are coequalisers as well. The morphisms $\id_B$, $l_{\hat{f}}$ and $r_{\hat{f}}$ are the unique morphisms that make all diagrams commute. It follows from coherence of double categories that $f \tinydot[gray dot] \mathid_B \iso \mathid_B$, and thus $\eta_{\hat{f}} \odot \epsilon_{\hat{f}} \iso 1_{\hat{f}}$.

This simultaneously proves that $f$ has a conjoint, because $\Alg({\bf C}^{h* op}) \cong$ $\Alg({\cat C})$.
\end{proof}

\begin{eg}
When we take the category $\mathcal{A}B$ of Abelian groups and group homomorphisms as our symmetric monoidal category, we obtain the double category $\cMod$ of rings, ring homomorphisms, bimodules and bimodule homomorphisms. As a consequence of Lemma's ~\ref{lem:algdouble}, ~\ref{lem:algsymmon}, and ~\ref{lem:algfib}, and Theorem ~\ref{thm:lcbcfunctor}, the horizontal bicategory $\cMod$ of $\lMod$ is symmetric monoidal.
\end{eg}

\begin{thm}\label{thm:eqcomp}
Let {\bf C} be a braided monoidal category. The horizontal bicategory $\mathcal{A}lg({\cat C})$ is braided monoidal; it is symmetric whenever {\bf C} is symmetric.
\end{thm}

\begin{proof}
This follows directly from Lemmas \ref{lem:algdouble}, \ref{lem:algsymmon}, \ref{lem:algfib}, and from Theorem ~\ref{thm:lcbcfunctor}.
\end{proof}

\subsection{$\cat 2(CP(C))$}
We will extend the proof to the bicategory $2(CP({\cat C}))$ of dagger Frobenius algebras, dagger bimodules and bimodule homomorphisms in a braided monoidal category and demonstrate that it is braided monoidal and symmetric whenever ${\bf C}$ is symmetric.
\begin{defn}
A {\bf Frobenius algebra} is a monoidal category ${\cat C}$ is a monoid $(A, \tinymult, \tinyunit)$ together with a comonoid 
$(A, \tinycomult, \tinycounit)$ that satisfies the equations below.
\begin{align}
  \begin{pic}[]
    \node[gray dot] (t) at (.5,.5) {};
    \node[gray dot] (b) at (-.5,0) {};
    \draw (b.center) to[out=0,in=180] (t.center);
    \draw (-.5,-.5) to (b.center);
    \draw (t.center) to (.5,1);
    \draw (t.center) to[out=0,in=90] (1,-.5);
    \draw (-1,1) to[out=-90,in=180] (b.center);
  \end{pic}
\,=\,
  \begin{pic}[]
    \node[gray dot] (t) at (0,.5) {};
    \node[gray dot] (b) at (0,0) {};
    \draw (b.center) to (t.center);
    \draw (t.center) to[out=180,in=-90] (-.5,1);
    \draw (t.center) to[out=0,in=-90] (.5,1);
    \draw (b.center) to[out=180,in=90] (-.5,-.5);
    \draw (b.center) to[out=0,in=90] (.5,-.5);
  \end{pic}
 \, =\,
  \begin{pic}[]
    \node[gray dot] (t) at (-.5,.5) {};
    \node[gray dot] (b) at (.5,0) {};
    \draw (b.center) to[out=180,in=0] (t.center);
    \draw (.5,-.5) to (b.center);
    \draw (t.center) to (-.5,1);
    \draw (t.center) to[out=180,in=90] (-1,-.5);
    \draw (1,1) to[out=-90,in=0] (b.center);
  \end{pic}
\end{align}
We call a Frobenius algebra {\bf special} whenever the following equation holds.
 \begin{align}
  \begin{pic}[]
        \node (0) at (0,-0.5) {};
        \node[gray dot] (1) at (0,-.1) {};
        \node[gray dot] (2) at (0,0.6) {};
        \node (3) at (0,1) {};
        \draw (0.center) to (1.center);
        \draw (2.center) to (3.center);
        \draw[in=180, out=180, looseness=2] (1.center) to (2.center);
        \draw[in=0, out=0, looseness=2] (1.center) to (2.center);
\end{pic}
\,=\,
  \begin{pic}[]
    \draw (0,-.5) to (0,1);
  \end{pic}
\end{align}
\end{defn}

\begin{rmk}\label{comspecfrob}
The bicategory of Frobenius algebras, bimodules and bimodule homomorphisms is symmetric monoidal. This follows from Theorem ~\ref{thm:eqcomp} since Frobenius algebras and algebra homomorphisms form a full symmetric monoidal subcategory of the category of algebras and algebra homomorphisms. The same statement holds for special and commutative Frobenius algebras. 
\end{rmk}

\begin{defn}
Let {\bf C} be a braided monoidal category equiped with a dagger functor. A Frobenius algebra in {\cat C} is a {\bf dagger Frobenius algebra} when the coalgebra is the dagger of the algebra. 
A bimodule is called a {\bf dagger bimodule} when the equation below holds.
\begin{align}
    \begin{pic}[yscale=.85]
      \node[morphism, minimum width=10mm] (m) at (0,-.7) {$\textbf{M}^\dag$};
      \draw (m.south) to (0,-2) node[right]{$M$};
      \draw (m.north) to  (0,.6) node[right]{$M$};
      \draw (m.45) to[out=90,in=-90] (.5,.6) node[right]{$D$};
      \draw (m.135) to[out=90,in=-90] (-.5,.6) node[right]{$C$};
    \end{pic}
    \!\!\!=\,\,\,
    \begin{pic}[yscale=.85]
      \node[morphism, minimum width=10mm] (m) at (0,-.7) {$\textbf{M}$};
      \draw (m.south) to (0,-2) node[right]{$M$};
      \draw (m.north) to (0,.6) node[right]{$M$};
      \node[white dot] (l) at (-.6,-1.3) {};
      \node[black dot] (r) at (.6,-1.3) {};
      \draw (m.-45) to[out=-90,in=150] (r);
      \draw (m.-135) to[out=-90,in=30] (l);
      \draw (r) to (.6,-1.6) node[black dot] {};
      \draw (l) to (-.6,-1.6) node[white dot] {};
      \draw (l) to[out=150, in=-90, looseness=.5] (-1,.6) node[right] {$C$};
      \draw (r) to[out=30, in=-90, looseness=.5] (1,.6) node[right] {$D$};
    \end{pic}
    \end{align}
\end{defn}
When a braided monoidal category {\bf C} is equiped with a dagger functor, dagger Frobenius algebras and dagger bimodules and bimodule homomorphisms do not form a fibrant double category by the method introduced in this paper. The reason is that the bimodule defined in equation ~\ref{eq:companion} as part of the data that forms a companion is not a dagger bimodule. To solve this, we restrict the categories $\mathbb{D}_0$ and $\mathbb{D}_1$ in an appropriate way to ensure that the result still holds.

\begin{prop}\label{prop:dagger}
Let ${\cat C}$ be a braided monoidal dagger category. The bicategory $2(CP({\cat C}))$ of dagger Frobenius algebras, dagger bimodules and bimodule homomorphisms of ${\cat C}$ is braided monoidal; it is symmetric whenever {\bf C} is symmetric.
\end{prop}

\begin{proof}
The bicategory is the horizontal bicategory of the double category of dagger Frobenius algebras and self-conjugate algebra homomorphisms, dagger bimodules and extended bimodule homomorphisms $(f_1,f,f_2)$ in {\cat C}, of which $f_1$ and $f_2$ are self-conjugate. This ensures that ~\ref{eq:companion} is a dagger bimodule.  This is a symmetric monoidal subcategories of our original double category, since self-conjugate morphisms are closed under the tensor product and all structure isomorphisms of the monoidal double category are self-conjugate. Applying Theorem~\ref{thm:lcbcfunctor} on this double category give the required result.
\end{proof}

\begin{cor}
Let ${\bf C}$ be a symmetric monoidal category. The bicategory $2(CP({\bf C}))$ of ~\cite{heunenvicarywester} is symmetric monoidal.
\end{cor}

\begin{proof}
This follows directly from proposition \ref{prop:dagger}.
\end{proof}


  

% Local Variables:
% TeX-master: "smbicat"
% End:

 
\bibliographystyle{alpha}
\bibliography{smbicat}

\end{document}
