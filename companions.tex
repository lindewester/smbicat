\section{Companions and conjoints}
\label{sec:comp-conj}

Suppose that \lD\ is a monoidal double category; when does
$\cH(\lD)$ become a monoidal bicategory?  It clearly has a
unit object $I$, and the pseudo double functor $\ten\maps
\lD\times\lD\to\lD$ clearly induces a functor $\ten\maps
\cH(\lD)\times\cH(\lD)\to\cH(\lD)$.  However, the problem is that the
constraint isomorphisms such as $A\ten (B\ten C)\iso (A\ten B)\ten C$
are \emph{tight} 1-morphisms, which get discarded when we pass to
$\cH(\lD)$.  Thus, in order for $\cH(\lD)$ to inherit a symmetric
monoidal structure, we must have a way to make tight 1-morphisms
into loose 1-cells.  Thus is the purpose of the following
definition.


\begin{defn}\label{def:companion}
  Let \lD\ be a double category and $f\maps A\to B$ a tight
  1-morphism.  A \textbf{companion} of $f$ is a loose 1-cell
  $\fhat\maps A\hto B$ together with 2-morphisms
  \begin{equation*}
    \begin{array}{c}
      \xymatrix@-.5pc{
        \ar[r]|-@{|}^-{\fhat} \ar[d]_f \ar@{}[dr]|{\Downarrow \epsilon_{\hat{f}} }
        & \ar@{=}[d]\\
        \ar[r]|-@{|}_-{U_B} & }
    \end{array}\quad\text{and}\quad
    \begin{array}{c}
      \xymatrix@-.5pc{
        \ar[r]|-@{|}^-{U_A} \ar@{=}[d] \ar@{}[dr]|{\Downarrow \eta_{\hat{f}}}
        & \ar[d]^f\\
        \ar[r]|-@{|}_-{\fhat} & }
    \end{array}
  \end{equation*}
  such that the following equations hold.
  \begin{align}\label{eq:compeqn}
    \begin{array}{c}
      \xymatrix@-.5pc{
        \ar[r]|-@{|}^-{U_A} \ar@{=}[d] \ar@{}[dr]|{\Downarrow \eta_{\hat{f}}}
        & \ar[d]^f\\
        \ar[r]|-{\fhat} \ar[d]_f \ar@{}[dr]|
        {\Downarrow  \epsilon_{\hat{f}} }
        & \ar@{=}[d]\\
        \ar[r]|-@{|}_-{U_B} & }
    \end{array} &= 
    \begin{array}{c}
      \xymatrix@-.5pc{ \ar[r]|-@{|}^-{U_A} \ar[d]_f
        \ar@{}[dr]|{\Downarrow U_f} &  \ar[d]^f\\
        \ar[r]|-@{|}_-{U_B} & }
    \end{array}
    &
    \begin{array}{c}
      \xymatrix@-.5pc{
        \ar[r]|-@{|}^-{U_A} \ar@{=}[d] \ar@{}[dr]|{ \Downarrow \eta_{\hat{f}}}&
        \ar[r]|-@{|}^{\fhat}\ar[d]|f \ar@{}[dr]|{\Downarrow  \epsilon_{\hat{f}} }
        & \ar@{=}[d]\\
        \ar[r]|-@{|}_-{\fhat} &
        \ar[r]|-@{|}_-{U_B} &}
%       \xymatrix@-.5pc{
%         \ar[rr]|-@{|}^-{\fhat} \ar@{}[drr]|\iso \ar@{=}[d] &&
%         \ar@{=}[d] \\
%         \ar[r]|-@{|}^-{U_A} \ar@{=}[d] \ar@{}[dr]|\Downarrow &
%         \ar[r]|-@{|}^-{\fhat} \ar[d]_f \ar@{}[dr]|\Downarrow
%         & \ar@{=}[d]\\
%         \ar[r]|-@{|}_-{\fhat} &
%         \ar[r]|-@{|}_-{U_B} &\\
%         \ar[rr]|-@{|}_-{\fhat} \ar@{}[urr]|\iso \ar@{=}[u] &&
%         \ar@{=}[u]}
    \end{array} &=
    \begin{array}{c}
      \xymatrix@-.5pc{
        \ar[r]|-@{|}^-{\fhat} \ar@{=}[d] \ar@{}[dr]|{\Downarrow 1_{\fhat}}
        & \ar@{=}[d]\\
        \ar[r]|-@{|}_-{\fhat} & }
    \end{array}
  \end{align}
  A \textbf{conjoint} of $f$, denoted $\fchk\maps B\hto A$, is a
  companion of $f$ in the double category $\lD^{l\cdot\mathrm{op}}$
  obtained by reversing the loose 1-cells, but not the tight
  1-morphisms, of \lD.
\end{defn}

\begin{rmk}
  We momentarily suspend our convention of pretending that our double
  categories are strict to mention that the second
  equation in~\eqref{eq:compeqn} actually requires an insertion of unit
  isomorphisms to make sense.
\end{rmk}

The form of this definition is due
to~\cite{gp:double-adjoints,dpp:spans}, but the ideas date back
to~\cite{bs:dblgpd-xedmod}; see
also~\cite{bm:dbl-thin-conn,fiore:pscat}.  In the terminology of these
references, a \emph{connection} on a double category is equivalent to
a strictly functorial choice of a companion for each tight arrow.

% a loose 1-cell $\fchk\maps B\hto
%   A$ together with 2-morphisms
%   \[\begin{array}{c}
%     \xymatrix@-.5pc{
%       \ar[r]|-@{|}^-{\fchk} \ar@{=}[d] \ar@{}[dr]|\Downarrow
%       & \ar[d]^f\\
%       \ar[r]|-@{|}_-{U_B} & }
%   \end{array}\quad\text{and}\quad
%   \begin{array}{c}
%     \xymatrix@-.5pc{
%       \ar[r]|-@{|}^-{U_A} \ar[d]_f \ar@{}[dr]|\Downarrow
%       & \ar@{=}[d]\\
%       \ar[r]|-@{|}_-{\fchk} & }
%   \end{array}\]
%   such that the following equations hold.
%   \begin{align*}
%     \begin{array}{c}
%       \xymatrix@-.5pc{
%         \ar[r]|-@{|}^-{U_A} \ar[d]_f \ar@{}[dr]|\Downarrow
%         & \ar@{=}[d]\\
%         \ar[r]|-{\fchk} \ar@{=}[d] \ar@{}[dr]|\Downarrow
%         & \ar[d]^f\\
%         \ar[r]|-@{|}_-{U_B} & }
%     \end{array} &= 
%     \begin{array}{c}
%       \xymatrix@-.5pc{ \ar[r]|-@{|}^-{U_A} \ar[d]_f
%         \ar@{}[dr]|{\Downarrow U_f} &  \ar[d]^f\\
%         \ar[r]|-@{|}_-{U_B} & }
%     \end{array}
%     &
%     \begin{array}{c}
%       \xymatrix@-.5pc{
%         \ar[rr]|-@{|}^-{\fchk} \ar@{}[drr]|\iso \ar@{=}[d] &&
%         \ar@{=}[d] \\
%         \ar[r]|-@{|}^-{\fchk} \ar@{=}[d] \ar@{}[dr]|\Downarrow &
%         \ar[r]|-@{|}^-{U_A} \ar[d]_f \ar@{}[dr]|\Downarrow
%         & \ar@{=}[d]\\
%         \ar[r]|-@{|}_-{U_B} &
%         \ar[r]|-@{|}_-{\fchk} &\\
%         \ar[rr]|-@{|}_-{\fchk} \ar@{}[urr]|\iso \ar@{=}[u] &&
%         \ar@{=}[u]}
%     \end{array} &=
%     \begin{array}{c}
%       \xymatrix@-.5pc{
%         \ar[r]|-@{|}^-{\fchk} \ar@{=}[d]
%         & \ar@{=}[d]\\
%         \ar[r]|-@{|}_-{\fchk} & }
%     \end{array}
%   \end{align*}

\begin{egs}
  \lMod, \lnCob, and \lProf\ have companions and conjoints for all tight morphisms.  In \lMod, the companion
  of a ring homomorphism $f\maps A\to B$ is $B$ regarded as an
  $A$-$B$-bimodule via $f$ on the left, and dually for its conjoint.
  In \lnCob, companions and conjoints are obtained by regarding a
  diffeomorphism as a cobordism.  And in \lProf, companions and
  conjoints are obtained by regarding a functor $f\maps A\to B$ as a
  `representable' profunctor $B(f-,-)$ or $B(-,f-)$.
\end{egs}

% \begin{rmk}
%   For an $(n\times 1)$-category (recall \autoref{rmk:monglob}), the
%   lifting condition we should require is simply that each double
%   category $\lD_{i+1} \toto \lD_i$, for $0\le i < n$, is fibrant.
% \end{rmk}

The existence of companions and conjoints gives us a way to `lift'
tight 1-morphisms to loose 1-cells.  What is even more crucial
for our applications, however, is that these liftings are unique up to
isomorphism, and that these isomorphisms are canonical and coherent.
This is the content of the following lemmas.  We state most of them
only for companions, but all have dual versions for conjoints. Later we will use them to prove the existence of structure isomorphisms and the commutativity of diagrams needed for $\cH$ to preserve monoidal structures.

\begin{lem}\label{thm:theta}
  Let $\fhat\maps A\hto B$ and $\fhat'\maps A\hto B$ be companions of
  $f$ (that is, each comes \emph{equipped with} 2-morphisms as in
  \autoref{def:companion}).  Then there is a unique globular isomorphism
  $\theta_{\fhat,\fhat'}\maps \fhat\too[\iso]\fhat'$ such that
  \begin{equation}\label{eq:comp-iso}
    \vcenter{\xymatrix@R=1.5pc{
        \ar[r]|-@{|}^-{U_A} \ar@{=}[d] \ar@{}[dr]|{\Downarrow \eta_{\hat{f}}} &  \ar[d]^f\\
        \ar[r]|-{\fhat} \ar@{=}[d] \ar@{}[dr]|{\Downarrow \theta_{\fhat,\fhat'}} &  \ar@{=}[d]\\
        \ar[r]|-{\fhat'} \ar[d]_f \ar@{}[dr]|{\Downarrow \epsilon_{\hat{f}'}} & \ar@{=}[d]\\
        \ar[r]|-@{|}_-{U_B} & }} \quad = \quad
    \vcenter{\xymatrix@-.5pc{ \ar[r]|-@{|}^-{U_A} \ar[d]_f
        \ar@{}[dr]|{\Downarrow U_f} &  \ar[d]^f\\
        \ar[r]|-@{|}_-{U_B} & .}}
  \end{equation}
\end{lem}
\begin{proof}
  Composing~\eqref{eq:comp-iso} on the left with
  $\vcenter{\xymatrix@-.5pc{ \ar[r]|-@{|}^-{U_A} \ar@{=}[d]
      \ar@{}[dr]|{\Downarrow \eta_{\hat{f}}} & \ar[d]^f\\ \ar[r]|-@{|}_-{\fhat'} & }}$
  and on the right with $\vcenter{\xymatrix@-.5pc{
      \ar[r]|-@{|}^-{\fhat} \ar[d]_f \ar@{}[dr]|{\Downarrow \epsilon_{\hat{f}}}&
      \ar@{=}[d]\\ \ar[r]|-@{|}_-{U_B} & }}$, and using the second
  equation~\eqref{eq:compeqn}, we see that if~\eqref{eq:comp-iso} is
  satisfied then $\theta_{\fhat,\fhat'}$ must be the composite
  \begin{equation}
    \vcenter{\xymatrix@-.5pc{
        \ar[r]|-@{|}^-{U_A} \ar@{=}[d] \ar@{}[dr]|{\Downarrow \eta_{\hat{f}'}}&
        \ar[r]|-@{|}^-{\fhat} \ar[d]|f \ar@{}[dr]|{\Downarrow \epsilon_{\hat{f}}}
        & \ar@{=}[d]\\
        \ar[r]|-@{|}_-{\fhat'} &
        \ar[r]|-@{|}_-{U_B} &}}\label{eq:theta}
%     \vcenter{\xymatrix@-.5pc{
%         \ar[rr]|-@{|}^-{\fhat} \ar@{}[drr]|\iso \ar@{=}[d] &&
%         \ar@{=}[d] \\
%         \ar[r]|-@{|}^-{U_A} \ar@{=}[d] \ar@{}[dr]|\Downarrow &
%         \ar[r]|-@{|}^-{\fhat} \ar[d]_f \ar@{}[dr]|\Downarrow
%         & \ar@{=}[d]\\
%         \ar[r]|-@{|}_-{\fhat'} &
%         \ar[r]|-@{|}_-{U_B} &\\
%         \ar[rr]|-@{|}_-{\fhat'} \ar@{}[urr]|\iso \ar@{=}[u] &&
%         \ar@{=}[u]}}\label{eq:theta}
  \end{equation}
  Two applications of the first equation~\eqref{eq:compeqn} shows that
  this indeed satisfies~\eqref{eq:comp-iso}.  As for its being an
  isomorphism, we have the dual composite $\theta_{\fhat',\fhat}$:
  \[\vcenter{\xymatrix@-.5pc{
      \ar[r]|-@{|}^-{U_A} \ar@{=}[d] \ar@{}[dr]|{\Downarrow \eta_{\hat{f}}} &
      \ar[r]|-@{|}^{\fhat'}\ar[d]|f \ar@{}[dr]|
{\Downarrow \epsilon_{\hat{f}'}}
      & \ar@{=}[d]\\
      \ar[r]|-@{|}_-{\fhat} &
      \ar[r]|-@{|}_-{U_B} &}}\]
  which we verify is an inverse using~\eqref{eq:compeqn}:
  \[\vcenter{\xymatrix@-.5pc{
      \ar[r]|-@{|}^{U_A}\ar@{=}[d] \ar@{}[dr]|{=} &
      \ar[r]|-@{|}^{U_A}\ar@{=}[d] \ar@{}[dr]|{\Downarrow \eta_{\hat{f}'}} &
      \ar[r]|-@{|}^{\fhat}\ar[d]|f \ar@{}[dr]|{\Downarrow \epsilon_{\hat{f}}} &
      \ar@{=}[d]\\
      \ar[r]|{U_A}\ar@{=}[d] \ar@{}[dr]|{\Downarrow \eta_{\hat{f}}} &
      \ar[r]|{\fhat'}\ar[d]|f \ar@{}[dr]|{\Downarrow \epsilon_{\hat{f}'}} &
      \ar[r]|{U_B}\ar@{=}[d] \ar@{}[dr]|{=} &
      \ar@{=}[d]\\
      \ar[r]|-@{|}_{\fhat} &
      \ar[r]|-@{|}_{U_B} &
      \ar[r]|-@{|}_{U_B} &
    }} \;=\;
  \vcenter{\xymatrix@-.5pc{
      \ar[r]|-@{|}^-{U_A} \ar@{=}[d] \ar@{}[dr]|{\Downarrow  \eta_{\hat{f}}}&
      \ar[r]|-@{|}^{\fhat}\ar[d]|f \ar@{}[dr]|{\Downarrow \epsilon_{\hat{f}}}
      & \ar@{=}[d]\\
      \ar[r]|-@{|}_-{\fhat} &
      \ar[r]|-@{|}_-{U_B} &}} \;=\;
  \vcenter{\xymatrix@-.5pc{
      \ar[r]|-@{|}^-{\fhat} \ar@{=}[d] \ar@{}[dr]|{\Downarrow 1_{\fhat}}
      & \ar@{=}[d]\\
      \ar[r]|-@{|}_-{\fhat} & }}\]
  (and dually).
\end{proof}

\begin{lem}\label{thm:theta-id}
  For any companion \fhat\ of $f$ we have $\theta_{\fhat,\fhat}=1_{\fhat}$.
\end{lem}
\begin{proof}
  This is the second equation~\eqref{eq:compeqn}.
\end{proof}

\begin{lem}\label{thm:theta-compose-vert}
  Suppose that $f$ has three companions $\fhat$, $\fhat'$, and
  $\fhat''$.  Then $\theta_{\fhat,\fhat''} = \theta_{\fhat',\fhat''}
  \circ\theta_{\fhat,\fhat'}$.
\end{lem}
\begin{proof}
  By definition, we have
  \[\theta_{\fhat',\fhat''} \circ\theta_{\fhat,\fhat'} =\;
  \vcenter{\xymatrix@-.5pc{
      \ar[r]|-@{|}^{U_A}\ar@{=}[d] \ar@{}[dr]|{=} &
      \ar[r]|-@{|}^{U_A}\ar@{=}[d] \ar@{}[dr]|{\Downarrow \eta_{\hat{f}'}} &
      \ar[r]|-@{|}^{\fhat}\ar[d]|f \ar@{}[dr]|{\Downarrow \epsilon_{\hat{f}}} &
      \ar@{=}[d]\\
      \ar[r]|{U_A}\ar@{=}[d] \ar@{}[dr]|{\Downarrow \eta_{\hat{f}''}} &
      \ar[r]|{\fhat'}\ar[d]|f \ar@{}[dr]|{\Downarrow \epsilon_{\hat{f}'}} &
      \ar[r]|{U_B}\ar@{=}[d] \ar@{}[dr]|{=} &
      \ar@{=}[d]\\
      \ar[r]|-@{|}_{\fhat''} &
      \ar[r]|-@{|}_{U_B} &
      \ar[r]|-@{|}_{U_B} &
    }} \;=\;
  \vcenter{\xymatrix@-.5pc{
      \ar[r]|-@{|}^-{U_A} \ar@{=}[d] \ar@{}[dr]|{\Downarrow \eta_{\hat{f}''}}&
      \ar[r]|-@{|}^{\fhat}\ar[d]|f \ar@{}[dr]|{\Downarrow \epsilon_{\hat{f}}}
      & \ar@{=}[d]\\
      \ar[r]|-@{|}_-{\fhat''} &
      \ar[r]|-@{|}_-{U_B} &}} \;=
  \theta_{\fhat,\fhat''}\]
  as desired.
\end{proof}

\begin{lem}\label{thm:comp-unit}
  $U_A\maps A\hto A$ is always a companion of $1_A\maps A\to A$ in a
  canonical way.
\end{lem}
\begin{proof}
  We take both defining 2-morphisms to be
  $1_{U_A}$; the truth of~\eqref{eq:compeqn} is evident.
\end{proof}

\begin{lem}\label{thm:comp-compose}
  Suppose that $f\maps A\to B$ has a companion \fhat\ and $g\maps B\to
  C$ has a companion \ghat.  Then $\ghat\odot\fhat$ is a companion of
  $gf$.
\end{lem}
\begin{proof}
  We take the defining 2-morphisms to be the composites
  \[\vcenter{\xymatrix@-.5pc{
      \ar[r]|-@{|}^-{\fhat} \ar[d]_f \ar@{}[dr]|{\Downarrow \epsilon_{\hat{f}}}&
      \ar[r]|-@{|}^-{\ghat} \ar@{=}[d] \ar@{}[dr]|{1_{\ghat}} &
      \ar@{=}[d]\\
      \ar[r]|-{U_B} \ar[d]_g \ar@{}[dr]|{U_g} &
      \ar[r]|-{\ghat} \ar[d]|g \ar@{}[dr]|{\Downarrow \epsilon_{\hat{g}}}&
      \ar@{=}[d]\\
      \ar[r]|-@{|}_-{U_C} &
      \ar[r]|-@{|}_-{U_C} &
    }}\quad\text{and}\quad
  \vcenter{\xymatrix@-.5pc{
      \ar[r]|-@{|}^-{U_A} \ar@{=}[d] \ar@{}[dr]|{\Downarrow \eta_{\hat{f}}} &
      \ar[r]|-@{|}^-{U_A} \ar[d]|f \ar@{}[dr]|{U_f} &
      \ar[d]^f\\
      \ar[r]|-{\fhat} \ar@{=}[d] \ar@{}[dr]|{1_{\fhat}} &
      \ar[r]|-{U_B} \ar@{=}[d] \ar@{}[dr]|{\Downarrow \eta_{\hat{g}}}&
      \ar[d]^g\\
      \ar[r]|-@{|}_-{\fhat} &
      \ar[r]|-@{|}_-{\ghat} &
    }}
  \]
  It is easy to verify that these satisfy~\eqref{eq:compeqn}, using
  the interchange law for $\odot$ and $\circ$ in a double category.
\end{proof}

\begin{lem}\label{thm:theta-compose-horiz}
  Suppose that $f\maps A\to B$ has companions $\fhat$ and $\fhat'$,
  and that $g\maps B\to C$ has companions $\ghat$ and $\ghat'$.  Then
  $\theta_{\ghat,\ghat'}\odot \theta_{\fhat,\fhat'}  =
    \theta_{\ghat\odot\fhat, \ghat'\odot\fhat'}$.
\end{lem}
\begin{proof}
  Using the interchange law for $\odot$ and $\circ$, we have:
  \begin{align}
    \theta_{\ghat\odot\fhat, \ghat'\odot\fhat'} &=\;
    \vcenter{\xymatrix@-.5pc{
        \ar[r]|-@{|}^-{U_A} \ar@{=}[d] \ar@{}[dr]|{\Downarrow \eta_{\hat{f}'}}&
        \ar[r]|-@{|}^-{U_A} \ar[d]|f \ar@{}[dr]|{U_f} &
        \ar[r]|-@{|}^-{\fhat} \ar[d]|f \ar@{}[dr]|{\Downarrow \epsilon_{\hat{f}}}&
        \ar[r]|-@{|}^-{\ghat} \ar@{=}[d] \ar@{}[dr]|{1_{\fhat}} &
        \ar@{=}[d]\\
        \ar[r]|-{\fhat'} \ar@{=}[d] \ar@{}[dr]|{1_{\ghat}} &
        \ar[r]|-{U_B} \ar@{=}[d] \ar@{}[dr]|{\Downarrow \eta_{\hat{g}'}} &
        \ar[r]|-{U_B} \ar[d]|g \ar@{}[dr]|{U_g} &
        \ar[r]|-{\ghat} \ar[d]|g \ar@{}[dr]|{\Downarrow \epsilon_{\hat{g}}} &
        \ar@{=}[d]\\
        \ar[r]|-@{|}_-{\fhat'} &
        \ar[r]|-@{|}_-{\ghat'} &
        \ar[r]|-@{|}_-{U_C} &
        \ar[r]|-@{|}_-{U_C} &
      }}
    \;=\;
    \vcenter{\xymatrix@-.5pc{
        \ar[r]|-@{|}^-{U_A} \ar@{=}[d] \ar@{}[dr]|{\Downarrow \eta_{\hat{f}'}}&
        \ar[r]|-@{|}^-{\fhat} \ar[d]|f \ar@{}[dr]|{\Downarrow \epsilon_{\hat{f}}}&
        \ar[r]|-@{|}^-{\ghat} \ar@{=}[d] \ar@{}[dr]|{1_{\fhat}} &
        \ar@{=}[d]\\
        \ar[r]|-{\fhat'} \ar@{=}[d] \ar@{}[dr]|{1_{\ghat}} &
        \ar[r]|-{U_B} \ar@{=}[d] \ar@{}[dr]|{\Downarrow \eta_{\hat{g}'}} &
        \ar[r]|-{\ghat} \ar[d]|g \ar@{}[dr]|{\Downarrow \epsilon_{\hat{g}}} &
        \ar@{=}[d]\\
        \ar[r]|-@{|}_-{\fhat'} &
        \ar[r]|-@{|}_-{\ghat'} &
        \ar[r]|-@{|}_-{U_C} &
      }}\\
    &=\;
    \vcenter{\xymatrix@-.5pc{
        \ar[r]|-@{|}^-{U_A} \ar@{=}[d] \ar@{}[dr]|{\Downarrow \eta_{\hat{f}'}}&
        \ar[r]|-@{|}^-{\fhat} \ar[d]|f \ar@{}[dr]|{\Downarrow \epsilon_{\hat{f}}}&
        \ar[r]|-@{|}^-{U_B} \ar@{=}[d] \ar@{}[dr]|{1_{U_B}} &
        \ar[r]|-@{|}^-{\ghat} \ar@{=}[d] \ar@{}[dr]|{1_{\fhat}} &
        \ar@{=}[d]\\
        \ar[r]|-{\fhat'} \ar@{=}[d] \ar@{}[dr]|{1_{\ghat}} &
        \ar[r]|-{U_B} \ar@{=}[d] \ar@{}[dr]|{1_{U_B}} &
        \ar[r]|-{U_B} \ar@{=}[d] \ar@{}[dr]|{\Downarrow \eta_{\hat{g}'}} &
        \ar[r]|-{\ghat} \ar[d]|g \ar@{}[dr]|{\Downarrow \epsilon_{\hat{g}}} &
        \ar@{=}[d]\\
        \ar[r]|-@{|}_-{\fhat'} &
        \ar[r]|-@{|}_-{U_B} &
        \ar[r]|-@{|}_-{\ghat'} &
        \ar[r]|-@{|}_-{U_C} &
      }}\;=\;
    \vcenter{\xymatrix@-.5pc{
        \ar[r]|-@{|}^-{U_A} \ar@{=}[d] \ar@{}[dr]|{\Downarrow \eta_{\hat{f}'}}&
        \ar[r]|-@{|}^-{\fhat} \ar[d]|f \ar@{}[dr]|{\Downarrow \epsilon_{\hat{f}}}&
        \ar[r]|-@{|}^-{U_B} \ar@{=}[d] \ar@{}[dr]|{\Downarrow \eta_{\hat{g}'}}&
        \ar[r]|-@{|}^-{\ghat} \ar[d]|g \ar@{}[dr]|{\Downarrow \epsilon_{\hat{g}}}& \ar@{=}[d]\\
        \ar[r]|-@{|}_-{\fhat'} &
        \ar[r]|-@{|}_-{U_B} &
        \ar[r]|-@{|}_-{\ghat'} &
        \ar[r]|-@{|}_-{U_C} &
      }}\\
    &=\;
    \theta_{\ghat,\ghat'}\odot \theta_{\fhat,\fhat'} 
  \end{align}
  as desired.
\end{proof}

\begin{lem}\label{thm:theta-unit}
  If $f\maps A\to B$ has a companion \fhat, then
  $\theta_{\fhat,\fhat\odot U_A}$ and $\theta_{\fhat,U_B\odot \fhat}$
  are equal to the unit constraints $\fhat \iso \fhat\odot U_A$ and
  $\fhat\iso U_B\odot \fhat$.
\end{lem}
\begin{proof}
  By definition, we have
  \[\theta_{\fhat,\fhat\odot U_A} =\;
  \vcenter{\xymatrix@-.5pc{
      \ar[r]|-@{|}^-{U_A} \ar@{=}[d] \ar@{}[dr]|{\Downarrow 1_{U_A}} &
      \ar[r]|-@{|}^-{U_A} \ar@{=}[d] \ar@{}[dr]|{1_{U_A}} &
      \ar@{=}[d] \ar[rr]|-@{|}^-{\fhat} \ar@{}[ddrr]|{\Downarrow \epsilon_{\hat{f}}}&& \ar@{=}[dd]\\
      \ar[r]|-{U_A} \ar@{=}[d] \ar@{}[dr]|{1_{U_A}} &
      \ar[r]|-{U_A} \ar@{=}[d] \ar@{}[dr]|{\Downarrow \eta_{\hat{f}}}&
      \ar[d]^f\\
      \ar[r]|-@{|}_-{U_A} &
      \ar[r]|-@{|}_-{\fhat} & \ar[rr]|-@{|}^-{U_B} &&
    }}\;=\;
  \vcenter{\xymatrix{ \ar[r]|-@{|}^-{U_A} \ar@{=}[d]
      \ar@{}[dr]|{\Downarrow 1_{U_A}} &  \ar@{=}[d]\\
      \ar[r]|-@{|}_-{U_A} & }}
  \]
  which, bearing in mind our suppression of unit and associativity
  constraints, means that in actuality it is the unit constraint
  $\fhat \iso \fhat\odot U_A$.  The other case is dual.
\end{proof}

\begin{lem}\label{thm:comp-func}
  Let $F\maps \lD\to\lE$ be a functor between double categories and
  let $f\maps A\to B$ have a companion \fhat\ in \lD.  Then $F(\fhat)$
  is a companion of $F(f)$ in \lE.
\end{lem}
\begin{proof}
  We take the defining 2-morphisms to be
  \[\vcenter{\xymatrix@R=1.5pc@C=3pc{
      \ar[r]|-@{|}^-{F(\fhat)} \ar[d]_{F(f)}
      \ar@{}[dr]|{F(\Downarrow \epsilon_{\hat{f}})} &  \ar@{=}[d]\\
      \ar[r]|-{F(U_B)} \ar@{=}[d] \ar@{}[dr]|\iso &  \ar@{=}[d]\\
      \ar[r]|-@{|}_-{U_{F(B)}} & }}
  \quad\text{and}\quad
  \vcenter{\xymatrix@R=1.5pc@C=3pc{
      \ar[r]|-@{|}^-{U_{FA}} \ar@{=}[d] \ar@{}[dr]|\iso & \ar@{=}[d]\\
      \ar[r]|-{F(U_{A})} \ar@{=}[d] \ar@{}[dr]|{F(\Downarrow \eta_{\hat{f}})} & 
      \ar[d]^{F(f)}\\
      \ar[r]|-@{|}_-{F(\fhat)} & .}}\]
  The axioms~\eqref{eq:compeqn} follow directly from those for \fhat.
\end{proof}

% \begin{lem}\label{thm:comp-ten}
%   Suppose that \lD\ is a monoidal double category and that $f\maps
%   A\to B$ and $g\maps C\to D$ have companions \fhat\ and \ghat\
%   respectively.  Then $\fhat\ten\ghat$ is a companion of $f\ten g$.
% \end{lem}
% \begin{proof}
%   This follows from \autoref{thm:comp-func}, since $\ten\maps
%   \lD\times\lD\to\lD$ is a functor, and a companion in $\lD\times\lD$
%   is simply a pair of companions in \lD.
% %   We take the defining 2-morphisms to be
% %   \[\vcenter{\xymatrix@R=1.5pc@C=3pc{
% %       \ar[r]|-@{|}^-{\fhat\ten\ghat} \ar[d]_{f\ten g}
% %       \ar@{}[dr]|{\Downarrow\ten\Downarrow} &  \ar@{=}[d]\\
% %       \ar[r]|-{U_B\ten U_D} \ar@{=}[d] \ar@{}[dr]|\iso &  \ar@{=}[d]\\
% %       \ar[r]|-@{|}_-{U_{B\ten D}} & }}
% %   \quad\text{and}\quad
% %   \vcenter{\xymatrix@R=1.5pc@C=3pc{
% %       \ar[r]|-@{|}^-{U_{A\ten C}} \ar@{=}[d] \ar@{}[dr]|\iso & \ar@{=}[d]\\
% %       \ar[r]|-{U_{A}\ten U_C} \ar@{=}[d] \ar@{}[dr]|{\Downarrow\ten\Downarrow} & 
% %       \ar[d]^{f\ten g}\\
% %       \ar[r]|-@{|}_-{\fhat\ten\ghat} & .}}\]
% \end{proof}

\begin{lem}\label{thm:theta-func}
  Suppose that $F\maps \lD\to\lE$ is a functor and that $f\maps A\to
  B$ has companions \fhat\ and $\fhat'$ in \lD.  Then
  $\theta_{F(\fhat),F(\fhat')} = F(\theta_{\fhat,\fhat'})$.
\end{lem}
\begin{proof}
  Using the axioms of a pseudo double functor and the definition of
  the 2-morphisms in \autoref{thm:comp-func}, we have
  \begin{equation}
    F(\theta_{\fhat,\fhat'})
    =\;
    \vcenter{\xymatrix@C=3pc{
        \ar[r]|-@{|}^-{F(\fhat)}
        \ar[d] \ar@{}[dr]|{\Downarrow F(\eta_{\hat{f}'} \odot\epsilon_{\hat{f}})} &  \ar[d]\\
        \ar[r]|-@{|}_-{F(\fhat')} &}}
    \;=\;
    \vcenter{\xymatrix@C=3pc{
        \ar[rr]|-@{|}^-{F(\fhat)}
        \ar@{=}[d] \ar@{}[drr]|\iso &&  \ar@{=}[d]\\
        \ar[r]|-@{|}^-{F(U_{A})} \ar@{=}[d]
        \ar@{}[dr]|{\Downarrow F(\eta_{\hat{f}'})} &
        \ar[r]|-@{|}^-{F(\fhat)} \ar[d]|{F(f)}
        \ar@{}[dr]|{\Downarrow F(\epsilon_{\hat{f}})}
        & \ar@{=}[d]\\
        \ar[r]|-@{|}_-{F(\fhat')} \ar@{}[drr]|\iso\ar@{=}[d] &
        \ar[r]|-@{|}_-{U_{F(B)}} & \ar@{=}[d]\\
        \ar[rr]|-@{|}_-{F(\fhat')} && }}
    \;=\;
    \vcenter{\xymatrix@R=1.5pc@C=3pc{
        \ar[r]|-@{|}^-{U_{F(A)}} \ar@{=}[d] \ar@{}[dr]|\iso &
        \ar[r]|-@{|}^-{F(\fhat)} \ar@{=}[d] \ar@{}[dr]|=
        & \ar@{=}[d]\\
        \ar[r]|-{F(U_{A})} \ar@{=}[d] \ar@{}[dr]|{\Downarrow F(\eta_{\hat{f}'})} &
        \ar[r]|-{F(\fhat)} \ar[d]|{F(f)} \ar@{}[dr]|{\Downarrow F(\epsilon_{\hat{f}})}
        & \ar@{=}[d]\\
        \ar[r]|-{F(\fhat')}  \ar@{=}[d] \ar@{}[dr]|= &
        \ar[r]|-{F(U_{B})} \ar@{}[dr]|\iso  \ar@{=}[d] & \ar@{=}[d]\\
        \ar[r]|-@{|}_-{F(\fhat')} &
        \ar[r]|-@{|}_-{U_{F(B)}} &}}
    \;=
    \theta_{F(\fhat),\,F(\fhat')}
  \end{equation}
  as desired.
\end{proof}

% \begin{lem}\label{thm:theta-ten}
%   Suppose that \lD\ is a monoidal double category, that $f\maps A\to
%   B$ has companions \fhat\ and $\fhat'$, and that $g\maps C\to D$ has
%   companions \ghat\ and $\ghat'$.  Then $\theta_{\fhat,\fhat'} \ten
%   \theta_{\ghat,\ghat'} = \theta_{\fhat\ten \ghat, \fhat'\ten\ghat'}.$
% \end{lem}
% \begin{proof}
%   This follows from \autoref{thm:theta-func} in the same way that
%   \autoref{thm:comp-ten} follows from \autoref{thm:comp-func}.
% %   Using the naturality and functoriality axioms spelled out in
% %   \S\ref{sec:symm-mono-double}, we have
% %   \begin{equation}
% %     \theta_{\fhat,\fhat'} \ten \theta_{\ghat,\ghat'}
% %     =\;
% %     \vcenter{\xymatrix@C=4.5pc{
% %         \ar[r]|-@{|}^-{\fhat\ten\ghat} \ar@{=}[d] \ar@{}[dr]|\iso &  \ar@{=}[d]\\
% %         \ar[r]|-@{|}^-{(U_A\odot \fhat)\ten(U_C\odot \ghat)}
% %         \ar[d] \ar@{}[dr]|{(\Downarrow\odot\Downarrow)\ten(\Downarrow\odot\Downarrow)} &  \ar[d]\\
% %         \ar[r]|-@{|}_-{(\fhat'\odot U_C) \ten (\ghat'\odot U_D)} \ar@{=}[d] \ar@{}[dr]|\iso &  \ar@{=}[d]\\
% %         \ar[r]|-@{|}_-{\fhat'\ten\ghat'} & }}
% %     \;=\;
% %     \vcenter{\xymatrix@C=2pc{
% %         \ar[rr]|-@{|}^-{\fhat\ten\ghat} \ar@{=}[d] \ar@{}[drr]|\iso &&  \ar@{=}[d]\\
% %         \ar[rr]|-@{|}^-{(U_A\odot \fhat)\ten(U_C\odot \ghat)}
% %         \ar@{=}[d] \ar@{}[drr]|\iso &&  \ar@{=}[d]\\
% %         \ar[r]|-@{|}^-{U_{A}\ten U_C} \ar@{=}[d]
% %         \ar@{}[dr]|{\Downarrow\ten \Downarrow} &
% %         \ar[r]|-@{|}^-{\fhat\ten\ghat} \ar[d]|{f\ten g}
% %         \ar@{}[dr]|{\Downarrow\ten\Downarrow}
% %         & \ar@{=}[d]\\
% %         \ar[r]|-@{|}_-{\fhat'\ten\ghat'} \ar@{}[drr]|\iso\ar@{=}[d] &
% %         \ar[r]|-@{|}_-{U_{B\ten D}} & \ar@{=}[d]\\
% %         \ar[rr]|-@{|}_-{(\fhat'\odot U_C) \ten (\ghat'\odot U_D)}
% %         \ar@{=}[d] \ar@{}[drr]|\iso &&  \ar@{=}[d]\\
% %         \ar[rr]|-@{|}_-{\fhat'\ten\ghat'} && }}
% %     \;=\;
% %     \vcenter{\xymatrix@R=1.5pc@C=2.5pc{
% %         \ar[rr]|-@{|}^-{\fhat\ten\ghat} \ar@{}[drr]|\iso \ar@{=}[d] &&
% %         \ar@{=}[d] \\
% %         \ar[r]|-@{|}^-{U_{A\ten C}} \ar@{=}[d] \ar@{}[dr]|\iso &
% %         \ar[r]|-@{|}^-{\fhat\ten\ghat} \ar@{=}[d] \ar@{}[dr]|=
% %         & \ar@{=}[d]\\
% %         \ar[r]|-@{|}^-{U_{A}\ten U_C} \ar@{=}[d] \ar@{}[dr]|{\Downarrow\ten\Downarrow} &
% %         \ar[r]|-@{|}^-{\fhat\ten\ghat} \ar[d]|{f\ten g} \ar@{}[dr]|{\Downarrow\ten\Downarrow}
% %         & \ar@{=}[d]\\
% %         \ar[r]|-@{|}_-{\fhat'\ten\ghat'}  \ar@{=}[d] \ar@{}[dr]|= &
% %         \ar[r]|-@{|}_-{U_{B}\ten U_D} \ar@{}[dr]|\iso  \ar@{=}[d] & \ar@{=}[d]\\
% %         \ar[r]|-@{|}_-{\fhat'\ten\ghat'} &
% %         \ar[r]|-@{|}_-{U_{B\ten D}} &\\
% %         \ar[rr]|-@{|}_-{\fhat'\ten\ghat'} \ar@{}[urr]|\iso \ar@{=}[u] &&
% %         \ar@{=}[u]}}
% %     \;=
% %     \theta_{\fhat\ten\ghat}
% %   \end{equation}
% %   as desired.
% \end{proof}


\begin{lem}\label{thm:comp-iso}
  If $f\maps A\to B$ is a tight isomorphism with a companion \fhat,
  then \fhat\ is a conjoint of its inverse $f\inv$.
\end{lem}
\begin{proof}
  The composites
  \[\vcenter{\xymatrix@-.5pc{
      \ar[r]|-@{|}^{\fhat}\ar[d]_f \ar@{}[dr]|{\Downarrow} &
      \ar@{=}[d]\\
      \ar[r]|{U_B}\ar[d]_{f\inv} \ar@{}[dr]|{\Downarrow U_{f\inv}} &
      \ar[d]^{f\inv}\\
      \ar[r]|-@{|}_{U_A} &
    }}\quad\text{and}\quad
  \vcenter{\xymatrix@-.5pc{
      \ar[r]|-@{|}^{U_B}\ar[d]_{f\inv} \ar@{}[dr]|{\Downarrow U_{f\inv}} &
      \ar[d]^{f\inv}\\
      \ar[r]|{U_A}\ar@{=}[d] \ar@{}[dr]|{\Downarrow} &
      \ar[d]^f\\
      \ar[r]|-@{|}_{\fhat} &
    }}
  \]
  exhibit \fhat\ as a conjoint of $f\inv$.
\end{proof}

\begin{lem}\label{thm:compconj-adj}
  If $f\maps A\to B$ has both a companion \fhat\ and a conjoint \fchk,
  then we have an adjunction $\fhat\adj\fchk$ in $\cH\lD$.  If $f$ is
  an isomorphism, then this is an adjoint equivalence.
\end{lem}
\begin{proof}
  The unit and counit of the adjunction $\fhat\adj\fchk$ are the
  composites
  \[\vcenter{\xymatrix@-.5pc{
      \ar[r]|-@{|}^{U_A}\ar@{=}[d] \ar@{}[dr]|{\Downarrow \eta_{\hat{f}}} &
      \ar[r]|-@{|}^{U_A}\ar[d]|{f} \ar@{}[dr]|{\Downarrow \eta_{\check{f}}} &
      \ar@{=}[d]\\
      \ar[r]|-@{|}_{\fhat} &
      \ar[r]|-@{|}_{\fchk} &
    }}\quad\text{and}\quad
  \vcenter{\xymatrix@-.5pc{
      \ar[r]|-@{|}^{\fchk}\ar@{=}[d] \ar@{}[dr]|{\Downarrow \epsilon_{\check{f}}} &
      \ar[r]|-@{|}^{\fhat}\ar[d]|{f} \ar@{}[dr]|{\Downarrow \epsilon_{\hat{f}}} &
      \ar@{=}[d]\\
      \ar[r]|-@{|}_{U_B} &
      \ar[r]|-@{|}_{U_B} &
    }}
  \]
%   \[\vcenter{\xymatrix@-.5pc{
%       \ar[rr]|-@{|}^{U_A}\ar@{=}[d] \ar@{}[drr]|{\iso} &&
%       \ar@{=}[d]\\
%       \ar[r]|-@{|}^{U_A}\ar@{=}[d] \ar@{}[dr]|{\Downarrow} &
%       \ar[r]|-@{|}^{U_A}\ar[d]|{f} \ar@{}[dr]|{\Downarrow} &
%       \ar@{=}[d]\\
%       \ar[r]|-@{|}_{\fhat} &
%       \ar[r]|-@{|}_{\fchk} &
%     }}\quad\text{and}\quad
%   \vcenter{\xymatrix@-.5pc{
%       \ar[r]|-@{|}^{\fchk}\ar@{=}[d] \ar@{}[dr]|{\Downarrow} &
%       \ar[r]|-@{|}^{\fhat}\ar[d]|{f} \ar@{}[dr]|{\Downarrow} &
%       \ar@{=}[d]\\
%       \ar[r]|-@{|}_{U_B}\ar@{=}[d] \ar@{}[drr]|{\iso} &
%       \ar[r]|-@{|}_{U_B} &
%       \ar@{=}[d]\\
%       \ar[rr]|-@{|}_{U_B} &&.
%     }}
%   \]
  The triangle identities follow from~\eqref{eq:compeqn}.  If $f$ is
  an isomorphism, then by the dual of \autoref{thm:comp-iso}, \fchk\
  is a companion of $f\inv$.  But then by \autoref{thm:comp-compose}
  $\fchk\odot \fhat$ is a companion of $1_A=f\inv \circ f$ and
  $\fhat\odot\fchk$ is a companion of $1_B = f\circ f\inv$, and hence
  \fhat\ and \fchk\ are equivalences.  We can then check that in this
  case the above unit and counit actually are the isomorphisms
  $\theta$, or appeal to the general fact that any adjunction
  involving an equivalence is an adjoint equivalence.
\end{proof}

% To conclude this section, we combine some of the Lemmas above to derive a more general statement which will play a central role in section~\ref{sec:constr-symm-mono}.
% \begin{lem}\label{lem:equal}  % FALSE as stated
% Any two composites of $\theta$-isomorphisms that have the same source and target loose 1-cells are equal.
% \end{lem}
% \begin{proof}
% By Lemmas~\ref{thm:theta-compose-vert} and~\ref{thm:theta-compose-horiz}, $\theta$-isomorphisms are closed under composition. By uniqueness of $\theta$-isomorphisms, any two compositions of $\theta$-isomorphisms that have the same source and target loose 1-cells must be equal.
% \end{proof}

\begin{lem}\label{lem:FUtheta}
Suppose $F:\lD \rightarrow \lE$ is a functor of double categories. The 2-cell $F_U$ is equal to $\theta_{\id_{FA}, F\id_A}$.
\end{lem}

\begin{proof}
We show that equation~\eqref{eq:comp-iso} holds when we substitute $\theta_{\id_{FA}, F\id_A}$ by $\hat{F}_U$.  Unfolding the definitions of $\eta_{U_{FA}}$, $\epsilon_{FU_A}$, and $\hat{F}_U$, and applying functoriality of $F$, we obtain an expression that can be rewritten to $U_{\id_{FA}}$. It follows that $F_U$ is a $\theta$-isomorphism, by the uniqueness of $\theta s$ in this expression.
\end{proof}

\begin{rmk}
  It is tempting to want to state a general coherence theorem along the lines of ``any two composites of $\theta$-isomorphisms having the same source and target are equal.''
  However, like statements such as ``any two composites of constraints in a monoidal category are equal'', this statement is actually literally false, because to determine a $\theta$-isomorphism requires not only a source and target but also the choice of companion data.
  If the same 1-cell is a companion of the same 1-morphism in more than one way (which is the case as soon as it has any nontrivial automorphisms), then there will be different $\theta$-isomorphisms with the same source and target.
  This is analogous to how in a particular monoidal category there can be ``accidental'' composites of constraints that are not covered by the coherence theorem.
  It is probably possible to state a general coherence theorem for $\theta$-isomorphisms that is sufficiently careful to be true, but we will not need this.
\end{rmk}


\begin{rmk}
  Since all the tight constraints of a monoidal double category are invertible, to construct its underlying monoidal bicategory we only need it to have companions (and hence, by \cref{thm:comp-iso}, conjoints) for all tight \emph{isomorphisms}.
  In~\cite{gg:ldstr-tricat} double categories of this sort were called ``fibrant'', but we prefer to call them \textbf{isofibrant} to emphasize the restriction to isomorphisms.
  
  To lift lax or colax monoidal \emph{functors}, and noninvertible transformations between monoidal functors, to the bicategorical level, we require our double categories to have companions (or conjoints, depending on the directions) for noninvertible tight morphisms as well.
  In~\cite{shulman:frbi} double categories with companions and conjoints for \emph{all} tight morphisms were called \emph{framed bicategories}; we might also call them \textbf{bifibrant} since in this case the (source, target) functor $\lD_1 \to \lD_0\times\lD_0$ is a bifibration (i.e.\ both a fibration and an opfibration; though in fact, it is sufficient to assume it to be one \emph{or} the other).

  However, even this assumption is not enough to lift monoidal functors and transformations, since a ``componentwise'' companion of a tight transformation will in general only be an \emph{oplax} natural transformation on bicategories.
  Pseudo-ness of this is really a condition on the particular transformation rather than on the double categories involved; in~\cite{cs:multicats} such transformations were called \emph{horizontally strong}, so we call them \textbf{loosely strong} instead.
  We will incorporate this condition by restricting the domain of our functor $\cH$ to contain only loosely strong transformations.
\end{rmk}


% Local Variables:
% TeX-master: "smbicat"
% End:
