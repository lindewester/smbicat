\section{Introduction}
\label{sec:introduction}

Symmetric monoidal bicategories are important in many contexts.
However, the definition of even a monoidal bicategory
(see~\cite{gps:tricats,nick:tricats}), let alone a symmetric monoidal
one
(see~\cite{kv:2cat-zam,kv:bm2cat,bn:hda-i,ds:monbi-hopfagbd,crans:centers,mccrudden:bal-coalgb,gurski:brmonbicat}),
% TODO: update references
is quite imposing, and time-consuming to verify in any example.  In
this paper we describe a method for constructing symmetric monoidal
bicategories which is hardly more difficult than constructing a pair
of ordinary symmetric monoidal categories.  While not universally
applicable, this method applies in many cases of interest.  This idea
has often been implicitly used in particular cases, such as
bicategories of enriched profunctors, but to our knowledge the first
general statement was claimed in~\cite[Appendix B]{shulman:frbi}.  Our
purpose here is to work out the details, independently
of~\cite{shulman:frbi}.\footnote{These details have also been worked out
in~\cite[\S5]{gg:ldstr-tricat} from a different perspective, using
``locally-double bicategories''.  Our goal is to be more
comprehensive, considering also the case of symmetric monoidal
structure and the construction of monoidal functors and
transformations.}

The method relies on the fact that in many bicategories, the 1-cells
are not the most fundamental notion of `morphism' between the objects.
For instance, in the bicategory \cMod\ of rings, bimodules, and
bimodule maps, the more fundamental notion of morphism between objects
is a ring homomorphism. The addition of these extra morphisms promotes
a bicategory to a \emph{double category}, or a category internal to
\cCat.  The extra morphisms are usually stricter than the 1-cells in
the bicategory and easier to deal with for coherence questions; in
many cases it is quite easy to show that we have a \emph{symmetric
  monoidal double category}.  The central observation is that in most
cases (when the double category is `fibrant') we can then `lift' this
symmetric monoidal structure to the original bicategory.  That is, we
prove the following theorem:

\begin{thm}\label{thm:mondbl-monbi-intro}
  If \lD\ is a fibrant monoidal double category, then its underlying
  bicategory $\cH(\lD)$ is a monoidal bicategory.  If \lD\ is braided
  or symmetric, so is $\cH(\lD)$. Furthermore, the assignment $\cH$
  extends to a trifunctor, so we can use it to construct functors and
  transformations between monoidal bicategories as well.
\end{thm}

We expect similar theorems to be true in higher dimensions.  For
instance, Chris Douglas~\cite{douglas:tfttalk} has suggested that many
apparent tricategories are more naturally bicategories internal to
\cCat\ or categories internal to \cTwocat; and in most such cases
arising in practice, we can again `lift' the coherence to give a
tricategory after discarding the additional structure.

We propose the generic term \textbf{$(n\times k)$-category}
(pronounced ``$n$-by-$k$-category'') for an $n$-category internal to
$k$-categories, a structure which has $(n+1)(k+1)$ different types of
cells or morphisms arranged in an $(n+1)$ by $(k+1)$ grid.  Thus
double categories may be called \textbf{1x1-categories}, while in
place of tricategories we may consider 2x1-categories and
1x2-categories --- or even 1x1x1-categories, i.e.\ triple categories,
as in~\cite{gp:intercategories-i,gp:intercategories-ii}.  Any
$(n\times k)$-category which satisfies a suitable lifting property
should have an underlying $(n+k)$-category, but clearly as $n$ and $k$
grow an increasing amount of structure is discarded in this process.

% , such as duality and
% trace~\cite{ps:traces} or autonomous
% structure~\cite{ds:monbi-hopfagbd,street:funct-calc,dms:antipodes},

There is a good case to be made that often the extra morphisms should
\emph{not} be discarded; indeed this is the perspective of much work
on double categories.  However, even from this perspective, sometimes
it really is the underlying $(n+k)$-category that one cares about; for
instance, the Baez-Dolan cobordism hypothesis asserts a universal
property of the $(n+1)$-category of cobordisms which is not shared by
the $(n\times 1)$-category from which it is naturally constructed
(see~\cite{lurie:tft}).  Thus, regardless of one's philosophical bent,
results such as \autoref{thm:mondbl-monbi-intro} are of interest.

Proceeding to the contents of this paper, in
\S\ref{sec:symm-mono-double} we review the definition of symmetric
monoidal double categories, and in \S\ref{sec:comp-conj} we recall the
notions of `companion' and `conjoint' whose presence supplies the
necessary lifting property, which we call being \emph{fibrant}.  Then
in \S\ref{sec:1x1-to-bicat} we describe a functor from fibrant double
categories to bicategories, and in \S\ref{sec:constr-symm-mono} we
show that it preserves monoidal, braided, and symmetric structures. Finally, in \S\ref{sec:Alg} we illustrate our method by constructing the symmetric monoidal bicategory ${\cA}lg$ of algebras, bimodules and bimodule homomorphisms.

We would like to thank Peter May, Tom Fiore, Stephan Stolz, Chris
Douglas, Nick Gurski, and Jamie Vicary for helpful discussions and comments.
