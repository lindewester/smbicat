\section{Intercategories of monoidal objects}
\label{sec:intercat}

In the previous section we constructed, from a product-preserving functor of locally cubical bicategories, \emph{twelve} new functors between locally cubical bicategories of monoidal objects according to whether the objects are braided, sylleptic, symmetric, or none, and whether the morphisms are lax, colax, or strong.
We now show that this menagerie can be reduced to only \emph{four} functors (according to the choice of objects), by incorporating lax, colax, and strong monoidal morphisms into a single structure.

One dimension down, the relevant structure is a \emph{strict} double category, with lax and colax morphisms as its horizontal and vertical arrows; see e.g.~\cite{gp:double-adjoints,shulman:dblderived}.
In our categorified case, we will use the \emph{intercategories} of~\cite{gp:intercategories-i,gp:intercategories-ii}.

[Give definition of intergategories]

[Insert picture]

Any locally cubical bicategory $\cB$ forms an intercategory, as shown in [reference]. The horizontal and vertical arrows are the 1-cells of $\cB$, transversal arrows are identities; horizontal and vertical cells are the vertical 2-cells of $\cB$, basic cells are the horizontal 2-cells of $\cB$ and the 3-dimensional cubes are the 3-cells of $\cB$, between the compositions of a vertical and horizontal arrow and between the compositions of a vertical and horizontal cell. [example from picture]

We will show that we can construct a new intercategory with the monoidal objects and cells defined in section \ref{sec:mono-objects}, that captures lax and colax monoidal cells. In order for this to work, we need to define (horizontal) 2-cells between composites of lax and colax 2-cells.

\begin{defn}
Let $A,B,C,D$ be monoidal objects in a locally cubical bicategory $\cB$; let $A \xrightarrow{f} B$, $C \xrightarrow{k}$ be lax monoidal 1-cells; and let $A \xrightarrow{g} C$, $B \xrightarrow{h} D$ be oplax monoidal 1-cells.
We define a {\bf quintet monoidal 2-cel} to be a monoidal 2-cell $\alpha: f \odot h \Rightarrow g \odot k$, together with invertible coherence  3-cells $H, K$
\begin{equation}
\begin{aligned}
\begin{tikzpicture}[scale=2]
\node (t) at (0,3){$h \odot \otimes \odot (f,f)$};
\node (tl) at (-1,2){$f \odot f \odot \otimes$};
\node (bl) at (-1,1){$k \odot g \odot \otimes$};
\node (b) at (0,0){$k \odot \otimes \odot (g,g)$};
\node (br) at (1,1){$\otimes \odot (k,k) \odot (g,g)$};
\node (tr) at (1,2){$\otimes \odot (h,h) \odot (f,f)$};
\draw[darrow] (t) to node[left]{$\id_h \odot \chi_f$} (tl);
\draw[darrow] (tl) to node[left]{$\alpha$} (bl);
\draw[darrow] (bl) to node[left]{$\id_k \odot \chi_g$} (b);
\draw[darrow] (t) to node[right]{$\chi_h \odot \id_{(f,f)}$} (tr);
\draw[darrow] (tr) to node[right]{$\alpha \otimes \alpha$} (br);
\draw[darrow] (br) to node [right]{$\chi_k \odot \id_{(g,g)}$}(b);
\node at (0,1.5){$\Downarrow H^{\alpha}$};
\end{tikzpicture}
\end{aligned}
\hspace{0.5cm}
\begin{aligned}
\begin{tikzpicture}[scale=2]
\node (t) at (0,3){$h \odot I$};
\node (tl) at (-1,2){$h \odot f \odot I$};
\node (bl) at (-1,1){$k \odot g \odot I$};
\node (b) at (0,0){$k \odot I$};
\node (br) at (1,1){$I$};
\node (tr) at (1,2){$I$};
\draw[darrow] (t) to node[left]{$\id_h \odot \iota_f$} (tl);
\draw[darrow] (tl) to node[left]{$\alpha \odot \id_I$} (bl);
\draw[darrow] (bl) to node[left]{$\id_k \odot \iota_g^{-1}$} (b);
\draw[darrow] (t) to node[right]{$\iota_h^{-1}$} (tr);
\draw[doubleeq] (tr) to (br);
\draw[darrow] (br) to node [right]{$\iota_k$}(b);
\node at (0,1.5){$\Downarrow K^{\alpha}$};
\end{tikzpicture}
\end{aligned}
\end{equation}

The 3-cells $H,K$ need to satisfy the following three equations:

[I will add these later, when I am sure they are not wrong]

\end{defn} 
\begin{thm}
Let $\cB$ be a locally cubical bicategory with products. There exist intercategories $\cM on\cI$, $\cB r\cI$, and $\cS ym\cI$ of the monoidal, braided and symmetric objects and cells of $\cB$. The intercategory $Mon\cI$ has the following structure:
\begin{itemize}
\item objects are the monoidal objects 
\item horizontal arrows are the lax 1-cells
\item vertical arrows are the oplax 1-cells
\item transversal arrows are identities
\item horizontal cells are the lax icons
\item vertical cells are the oplax icons
\item basic cells are the quintet monoidal 2-cells
\item 3-dimensional cubes are the monoidal 3-cells
\end{itemize}
The intercategories $\cB r\cI$ and $\cS ym\cI$ are defined analogously, for braided and symmetric monoidal objects and cells of $\cB$
\end{thm}

[Do we need conditions on the monoidal 3-cells?]
\begin{proof}
By Theorem {thm:lcbc} we know that the arrows in the horizontal and transversal direction between two given objects form a week bicategory, and so do the arrows in the vertical and transversal direction.
What we need to prove is that the arrows in the lax and oplax cells are related by interchangers.
\end{proof}



% Local Variables:
% TeX-master: "smbicat"
% End:
