\section{Appendix}
\label{ap:coherence}

We give the coherence diagrams for  monoidal objects, 1-cells, 2-cells and icons in a locally cubical bicategory, defined in Section~\ref{sec:mono-objects}. For readability, we omit certain cells, arising as coherence constraints for double categories. Where we do so, we write $\hat{\alpha}$ for the composition of a 3-cell ${\alpha}$ with coherence cells that ensure that the source and target 2-cells are of the right type.

\subsubsection*{Monoidal Object}

%
\documentclass[12pt]{ociamthesis}
\usepackage{tikz}
\usepackage{amsmath}
\usepackage{rotating}

\usepackage{amssymb,amsmath,stmaryrd,txfonts,mathrsfs,amsthm}
\usepackage[all,2cell]{xy}
\usepackage[neveradjust]{paralist}
\usepackage{hyperref}
\usepackage{mathtools}
\usepackage{tikz}
\usetikzlibrary{trees}
\usetikzlibrary{topaths}
\usetikzlibrary{decorations}
\usetikzlibrary{decorations.pathreplacing}
\usetikzlibrary{decorations.pathmorphing}
\usetikzlibrary{decorations.markings}
\usetikzlibrary{matrix,backgrounds,folding}
\usetikzlibrary{chains,scopes,positioning,fit}
\usetikzlibrary{arrows,shadows}
\usetikzlibrary{calc} 
\usetikzlibrary{chains}
\usetikzlibrary{shapes,shapes.geometric,shapes.misc}
\usepackage{smbicat}


\makeatletter
\let\ea\expandafter

%% Defining commands that are always in math mode.
\def\mdef#1#2{\ea\ea\ea\gdef\ea\ea\noexpand#1\ea{\ea\ensuremath\ea{#2}}}
\def\alwaysmath#1{\ea\ea\ea\global\ea\ea\ea\let\ea\ea\csname your@#1\endcsname\csname #1\endcsname
  \ea\def\csname #1\endcsname{\ensuremath{\csname your@#1\endcsname}}}

% Script letters
\newcommand{\sA}{\ensuremath{\mathscr{A}}}
\newcommand{\sB}{\ensuremath{\mathscr{B}}}
\newcommand{\sC}{\ensuremath{\mathscr{C}}}
\newcommand{\sD}{\ensuremath{\mathscr{D}}}
\newcommand{\sE}{\ensuremath{\mathscr{E}}}
\newcommand{\sF}{\ensuremath{\mathscr{F}}}
\newcommand{\sG}{\ensuremath{\mathscr{G}}}
\newcommand{\sH}{\ensuremath{\mathscr{H}}}
\newcommand{\sI}{\ensuremath{\mathscr{I}}}
\newcommand{\sJ}{\ensuremath{\mathscr{J}}}
\newcommand{\sK}{\ensuremath{\mathscr{K}}}
\newcommand{\sL}{\ensuremath{\mathscr{L}}}
\newcommand{\sM}{\ensuremath{\mathscr{M}}}
\newcommand{\sN}{\ensuremath{\mathscr{N}}}
\newcommand{\sO}{\ensuremath{\mathscr{O}}}
\newcommand{\sP}{\ensuremath{\mathscr{P}}}
\newcommand{\sQ}{\ensuremath{\mathscr{Q}}}
\newcommand{\sR}{\ensuremath{\mathscr{R}}}
\newcommand{\sS}{\ensuremath{\mathscr{S}}}
\newcommand{\sT}{\ensuremath{\mathscr{T}}}
\newcommand{\sU}{\ensuremath{\mathscr{U}}}
\newcommand{\sV}{\ensuremath{\mathscr{V}}}
\newcommand{\sW}{\ensuremath{\mathscr{W}}}
\newcommand{\sX}{\ensuremath{\mathscr{X}}}
\newcommand{\sY}{\ensuremath{\mathscr{Y}}}
\newcommand{\sZ}{\ensuremath{\mathscr{Z}}}

% Calligraphic letters
\newcommand{\cA}{\ensuremath{\mathcal{A}}}
\newcommand{\cB}{\ensuremath{\mathcal{B}}}
\newcommand{\cC}{\ensuremath{\mathcal{C}}}
\newcommand{\cD}{\ensuremath{\mathcal{D}}}
\newcommand{\cE}{\ensuremath{\mathcal{E}}}
\newcommand{\cF}{\ensuremath{\mathcal{F}}}
\newcommand{\cG}{\ensuremath{\mathcal{G}}}
\newcommand{\cH}{\ensuremath{\mathcal{H}}}
\newcommand{\cI}{\ensuremath{\mathcal{I}}}
\newcommand{\cJ}{\ensuremath{\mathcal{J}}}
\newcommand{\cK}{\ensuremath{\mathcal{K}}}
\newcommand{\cL}{\ensuremath{\mathcal{L}}}
\newcommand{\cM}{\ensuremath{\mathcal{M}}}
\newcommand{\cN}{\ensuremath{\mathcal{N}}}
\newcommand{\cO}{\ensuremath{\mathcal{O}}}
\newcommand{\cP}{\ensuremath{\mathcal{P}}}
\newcommand{\cQ}{\ensuremath{\mathcal{Q}}}
\newcommand{\cR}{\ensuremath{\mathcal{R}}}
\newcommand{\cS}{\ensuremath{\mathcal{S}}}
\newcommand{\cT}{\ensuremath{\mathcal{T}}}
\newcommand{\cU}{\ensuremath{\mathcal{U}}}
\newcommand{\cV}{\ensuremath{\mathcal{V}}}
\newcommand{\cW}{\ensuremath{\mathcal{W}}}
\newcommand{\cX}{\ensuremath{\mathcal{X}}}
\newcommand{\cY}{\ensuremath{\mathcal{Y}}}
\newcommand{\cZ}{\ensuremath{\mathcal{Z}}}

% blackboard bold letters
\newcommand{\lA}{\ensuremath{\mathbb{A}}}
\newcommand{\lB}{\ensuremath{\mathbb{B}}}
\newcommand{\lC}{\ensuremath{\mathbb{C}}}
\newcommand{\lD}{\ensuremath{\mathbb{D}}}
\newcommand{\lE}{\ensuremath{\mathbb{E}}}
\newcommand{\lF}{\ensuremath{\mathbb{F}}}
\newcommand{\lG}{\ensuremath{\mathbb{G}}}
\newcommand{\lH}{\ensuremath{\mathbb{H}}}
\newcommand{\lI}{\ensuremath{\mathbb{I}}}
\newcommand{\lJ}{\ensuremath{\mathbb{J}}}
\newcommand{\lK}{\ensuremath{\mathbb{K}}}
\newcommand{\lL}{\ensuremath{\mathbb{L}}}
\newcommand{\lM}{\ensuremath{\mathbb{M}}}
\newcommand{\lN}{\ensuremath{\mathbb{N}}}
\newcommand{\lO}{\ensuremath{\mathbb{O}}}
\newcommand{\lP}{\ensuremath{\mathbb{P}}}
\newcommand{\lQ}{\ensuremath{\mathbb{Q}}}
\newcommand{\lR}{\ensuremath{\mathbb{R}}}
\newcommand{\lS}{\ensuremath{\mathbb{S}}}
\newcommand{\lT}{\ensuremath{\mathbb{T}}}
\newcommand{\lU}{\ensuremath{\mathbb{U}}}
\newcommand{\lV}{\ensuremath{\mathbb{V}}}
\newcommand{\lW}{\ensuremath{\mathbb{W}}}
\newcommand{\lX}{\ensuremath{\mathbb{X}}}
\newcommand{\lY}{\ensuremath{\mathbb{Y}}}
\newcommand{\lZ}{\ensuremath{\mathbb{Z}}}

% bold letters
\newcommand{\bA}{\ensuremath{\mathbf{A}}}
\newcommand{\bB}{\ensuremath{\mathbf{B}}}
\newcommand{\bC}{\ensuremath{\mathbf{C}}}
\newcommand{\bD}{\ensuremath{\mathbf{D}}}
\newcommand{\bE}{\ensuremath{\mathbf{E}}}
\newcommand{\bF}{\ensuremath{\mathbf{F}}}
\newcommand{\bG}{\ensuremath{\mathbf{G}}}
\newcommand{\bH}{\ensuremath{\mathbf{H}}}
\newcommand{\bI}{\ensuremath{\mathbf{I}}}
\newcommand{\bJ}{\ensuremath{\mathbf{J}}}
\newcommand{\bK}{\ensuremath{\mathbf{K}}}
\newcommand{\bL}{\ensuremath{\mathbf{L}}}
\newcommand{\bM}{\ensuremath{\mathbf{M}}}
\newcommand{\bN}{\ensuremath{\mathbf{N}}}
\newcommand{\bO}{\ensuremath{\mathbf{O}}}
\newcommand{\bP}{\ensuremath{\mathbf{P}}}
\newcommand{\bQ}{\ensuremath{\mathbf{Q}}}
\newcommand{\bR}{\ensuremath{\mathbf{R}}}
\newcommand{\bS}{\ensuremath{\mathbf{S}}}
\newcommand{\bT}{\ensuremath{\mathbf{T}}}
\newcommand{\bU}{\ensuremath{\mathbf{U}}}
\newcommand{\bV}{\ensuremath{\mathbf{V}}}
\newcommand{\bW}{\ensuremath{\mathbf{W}}}
\newcommand{\bX}{\ensuremath{\mathbf{X}}}
\newcommand{\bY}{\ensuremath{\mathbf{Y}}}
\newcommand{\bZ}{\ensuremath{\mathbf{Z}}}

% fraktur letters
\newcommand{\fa}{\ensuremath{\mathfrak{a}}}
\newcommand{\fb}{\ensuremath{\mathfrak{b}}}
\newcommand{\fc}{\ensuremath{\mathfrak{c}}}
\newcommand{\fd}{\ensuremath{\mathfrak{d}}}
\newcommand{\fe}{\ensuremath{\mathfrak{e}}}
\newcommand{\ff}{\ensuremath{\mathfrak{f}}}
\newcommand{\fg}{\ensuremath{\mathfrak{g}}}
\newcommand{\fh}{\ensuremath{\mathfrak{h}}}
\newcommand{\fj}{\ensuremath{\mathfrak{j}}}
\newcommand{\fk}{\ensuremath{\mathfrak{k}}}
\newcommand{\fl}{\ensuremath{\mathfrak{l}}}
\newcommand{\fm}{\ensuremath{\mathfrak{m}}}
\newcommand{\fn}{\ensuremath{\mathfrak{n}}}
\newcommand{\fo}{\ensuremath{\mathfrak{o}}}
\newcommand{\fp}{\ensuremath{\mathfrak{p}}}
\newcommand{\fq}{\ensuremath{\mathfrak{q}}}
\newcommand{\fr}{\ensuremath{\mathfrak{r}}}
\newcommand{\fs}{\ensuremath{\mathfrak{s}}}
\newcommand{\ft}{\ensuremath{\mathfrak{t}}}
\newcommand{\fu}{\ensuremath{\mathfrak{u}}}
\newcommand{\fv}{\ensuremath{\mathfrak{v}}}
\newcommand{\fw}{\ensuremath{\mathfrak{w}}}
\newcommand{\fx}{\ensuremath{\mathfrak{x}}}
\newcommand{\fy}{\ensuremath{\mathfrak{y}}}
\newcommand{\fz}{\ensuremath{\mathfrak{z}}}

% fraktur letters
\newcommand{\fA}{\ensuremath{\mathfrak{A}}}
\newcommand{\fB}{\ensuremath{\mathfrak{B}}}
\newcommand{\fC}{\ensuremath{\mathfrak{C}}}

\mdef\fahat{\hat{\fa}}

% Underline letters
\newcommand{\uA}{\ensuremath{\underline{A}}}
\newcommand{\uB}{\ensuremath{\underline{B}}}
\newcommand{\uC}{\ensuremath{\underline{C}}}
\newcommand{\uD}{\ensuremath{\underline{D}}}
\newcommand{\uE}{\ensuremath{\underline{E}}}
\newcommand{\uF}{\ensuremath{\underline{F}}}
\newcommand{\uG}{\ensuremath{\underline{G}}}
\newcommand{\uH}{\ensuremath{\underline{H}}}
\newcommand{\uI}{\ensuremath{\underline{I}}}
\newcommand{\uJ}{\ensuremath{\underline{J}}}
\newcommand{\uK}{\ensuremath{\underline{K}}}
\newcommand{\uL}{\ensuremath{\underline{L}}}
\newcommand{\uM}{\ensuremath{\underline{M}}}
\newcommand{\uN}{\ensuremath{\underline{N}}}
\newcommand{\uO}{\ensuremath{\underline{O}}}
\newcommand{\uP}{\ensuremath{\underline{P}}}
\newcommand{\uQ}{\ensuremath{\underline{Q}}}
\newcommand{\uR}{\ensuremath{\underline{R}}}
\newcommand{\uS}{\ensuremath{\underline{S}}}
\newcommand{\uT}{\ensuremath{\underline{T}}}
\newcommand{\uU}{\ensuremath{\underline{U}}}
\newcommand{\uV}{\ensuremath{\underline{V}}}
\newcommand{\uW}{\ensuremath{\underline{W}}}
\newcommand{\uX}{\ensuremath{\underline{X}}}
\newcommand{\uY}{\ensuremath{\underline{Y}}}
\newcommand{\uZ}{\ensuremath{\underline{Z}}}

% bars
\newcommand{\Abar}{\ensuremath{\overline{A}}}
\newcommand{\Bbar}{\ensuremath{\overline{B}}}
\newcommand{\Cbar}{\ensuremath{\overline{C}}}
\newcommand{\Dbar}{\ensuremath{\overline{D}}}
\newcommand{\Ebar}{\ensuremath{\overline{E}}}
\newcommand{\Fbar}{\ensuremath{\overline{F}}}
\newcommand{\Gbar}{\ensuremath{\overline{G}}}
\newcommand{\Hbar}{\ensuremath{\overline{H}}}
\newcommand{\Ibar}{\ensuremath{\overline{I}}}
\newcommand{\Jbar}{\ensuremath{\overline{J}}}
\newcommand{\Kbar}{\ensuremath{\overline{K}}}
\newcommand{\Lbar}{\ensuremath{\overline{L}}}
\newcommand{\Mbar}{\ensuremath{\overline{M}}}
\newcommand{\Nbar}{\ensuremath{\overline{N}}}
\newcommand{\Obar}{\ensuremath{\overline{O}}}
\newcommand{\Pbar}{\ensuremath{\overline{P}}}
\newcommand{\Qbar}{\ensuremath{\overline{Q}}}
\newcommand{\Rbar}{\ensuremath{\overline{R}}}
\newcommand{\Sbar}{\ensuremath{\overline{S}}}
\newcommand{\Tbar}{\ensuremath{\overline{T}}}
\newcommand{\Ubar}{\ensuremath{\overline{U}}}
\newcommand{\Vbar}{\ensuremath{\overline{V}}}
\newcommand{\Wbar}{\ensuremath{\overline{W}}}
\newcommand{\Xbar}{\ensuremath{\overline{X}}}
\newcommand{\Ybar}{\ensuremath{\overline{Y}}}
\newcommand{\Zbar}{\ensuremath{\overline{Z}}}
\newcommand{\abar}{\ensuremath{\overline{a}}}
\newcommand{\bbar}{\ensuremath{\overline{b}}}
\newcommand{\cbar}{\ensuremath{\overline{c}}}
\newcommand{\dbar}{\ensuremath{\overline{d}}}
\newcommand{\ebar}{\ensuremath{\overline{e}}}
\newcommand{\fbar}{\ensuremath{\overline{f}}}
\newcommand{\gbar}{\ensuremath{\overline{g}}}
%\newcommand{\hbar}{\ensuremath{\overline{h}}} % whoops, \hbar means something else!
\newcommand{\ibar}{\ensuremath{\overline{\imath}}}
\newcommand{\jbar}{\ensuremath{\overline{\jmath}}}
\newcommand{\kbar}{\ensuremath{\overline{k}}}
\newcommand{\lbar}{\ensuremath{\overline{l}}}
\newcommand{\mbar}{\ensuremath{\overline{m}}}
\newcommand{\nbar}{\ensuremath{\overline{n}}}
%\newcommand{\obar}{\ensuremath{\overline{o}}}
\newcommand{\pbar}{\ensuremath{\overline{p}}}
\newcommand{\qbar}{\ensuremath{\overline{q}}}
\newcommand{\rbar}{\ensuremath{\overline{r}}}
\newcommand{\sbar}{\ensuremath{\overline{s}}}
\newcommand{\tbar}{\ensuremath{\overline{t}}}
\newcommand{\ubar}{\ensuremath{\overline{u}}}
\newcommand{\vbar}{\ensuremath{\overline{v}}}
\newcommand{\wbar}{\ensuremath{\overline{w}}}
\newcommand{\xbar}{\ensuremath{\overline{x}}}
\newcommand{\ybar}{\ensuremath{\overline{y}}}
\newcommand{\zbar}{\ensuremath{\overline{z}}}

% tildes
\newcommand{\Atil}{\ensuremath{\widetilde{A}}}
\newcommand{\Btil}{\ensuremath{\widetilde{B}}}
\newcommand{\Ctil}{\ensuremath{\widetilde{C}}}
\newcommand{\Dtil}{\ensuremath{\widetilde{D}}}
\newcommand{\Etil}{\ensuremath{\widetilde{E}}}
\newcommand{\Ftil}{\ensuremath{\widetilde{F}}}
\newcommand{\Gtil}{\ensuremath{\widetilde{G}}}
\newcommand{\Htil}{\ensuremath{\widetilde{H}}}
\newcommand{\Itil}{\ensuremath{\widetilde{I}}}
\newcommand{\Jtil}{\ensuremath{\widetilde{J}}}
\newcommand{\Ktil}{\ensuremath{\widetilde{K}}}
\newcommand{\Ltil}{\ensuremath{\widetilde{L}}}
\newcommand{\Mtil}{\ensuremath{\widetilde{M}}}
\newcommand{\Ntil}{\ensuremath{\widetilde{N}}}
\newcommand{\Otil}{\ensuremath{\widetilde{O}}}
\newcommand{\Ptil}{\ensuremath{\widetilde{P}}}
\newcommand{\Qtil}{\ensuremath{\widetilde{Q}}}
\newcommand{\Rtil}{\ensuremath{\widetilde{R}}}
\newcommand{\Stil}{\ensuremath{\widetilde{S}}}
\newcommand{\Ttil}{\ensuremath{\widetilde{T}}}
\newcommand{\Util}{\ensuremath{\widetilde{U}}}
\newcommand{\Vtil}{\ensuremath{\widetilde{V}}}
\newcommand{\Wtil}{\ensuremath{\widetilde{W}}}
\newcommand{\Xtil}{\ensuremath{\widetilde{X}}}
\newcommand{\Ytil}{\ensuremath{\widetilde{Y}}}
\newcommand{\Ztil}{\ensuremath{\widetilde{Z}}}
\newcommand{\atil}{\ensuremath{\widetilde{a}}}
\newcommand{\btil}{\ensuremath{\widetilde{b}}}
\newcommand{\ctil}{\ensuremath{\widetilde{c}}}
\newcommand{\dtil}{\ensuremath{\widetilde{d}}}
\newcommand{\etil}{\ensuremath{\widetilde{e}}}
\newcommand{\ftil}{\ensuremath{\widetilde{f}}}
\newcommand{\gtil}{\ensuremath{\widetilde{g}}}
\newcommand{\htil}{\ensuremath{\widetilde{h}}}
\newcommand{\itil}{\ensuremath{\widetilde{\imath}}}
\newcommand{\jtil}{\ensuremath{\widetilde{\jmath}}}
\newcommand{\ktil}{\ensuremath{\widetilde{k}}}
\newcommand{\ltil}{\ensuremath{\widetilde{l}}}
\newcommand{\mtil}{\ensuremath{\widetilde{m}}}
\newcommand{\ntil}{\ensuremath{\widetilde{n}}}
\newcommand{\otil}{\ensuremath{\widetilde{o}}}
\newcommand{\ptil}{\ensuremath{\widetilde{p}}}
\newcommand{\qtil}{\ensuremath{\widetilde{q}}}
\newcommand{\rtil}{\ensuremath{\widetilde{r}}}
\newcommand{\stil}{\ensuremath{\widetilde{s}}}
\newcommand{\ttil}{\ensuremath{\widetilde{t}}}
\newcommand{\util}{\ensuremath{\widetilde{u}}}
\newcommand{\vtil}{\ensuremath{\widetilde{v}}}
\newcommand{\wtil}{\ensuremath{\widetilde{w}}}
\newcommand{\xtil}{\ensuremath{\widetilde{x}}}
\newcommand{\ytil}{\ensuremath{\widetilde{y}}}
\newcommand{\ztil}{\ensuremath{\widetilde{z}}}

% Hats
\newcommand{\Ahat}{\ensuremath{\widehat{A}}}
\newcommand{\Bhat}{\ensuremath{\widehat{B}}}
\newcommand{\Chat}{\ensuremath{\widehat{C}}}
\newcommand{\Dhat}{\ensuremath{\widehat{D}}}
\newcommand{\Ehat}{\ensuremath{\widehat{E}}}
\newcommand{\Fhat}{\ensuremath{\widehat{F}}}
\newcommand{\Ghat}{\ensuremath{\widehat{G}}}
\newcommand{\Hhat}{\ensuremath{\widehat{H}}}
\newcommand{\Ihat}{\ensuremath{\widehat{I}}}
\newcommand{\Jhat}{\ensuremath{\widehat{J}}}
\newcommand{\Khat}{\ensuremath{\widehat{K}}}
\newcommand{\Lhat}{\ensuremath{\widehat{L}}}
\newcommand{\Mhat}{\ensuremath{\widehat{M}}}
\newcommand{\Nhat}{\ensuremath{\widehat{N}}}
\newcommand{\Ohat}{\ensuremath{\widehat{O}}}
\newcommand{\Phat}{\ensuremath{\widehat{P}}}
\newcommand{\Qhat}{\ensuremath{\widehat{Q}}}
\newcommand{\Rhat}{\ensuremath{\widehat{R}}}
\newcommand{\Shat}{\ensuremath{\widehat{S}}}
\newcommand{\That}{\ensuremath{\widehat{T}}}
\newcommand{\Uhat}{\ensuremath{\widehat{U}}}
\newcommand{\Vhat}{\ensuremath{\widehat{V}}}
\newcommand{\What}{\ensuremath{\widehat{W}}}
\newcommand{\Xhat}{\ensuremath{\widehat{X}}}
\newcommand{\Yhat}{\ensuremath{\widehat{Y}}}
\newcommand{\Zhat}{\ensuremath{\widehat{Z}}}
\newcommand{\ahat}{\ensuremath{\hat{a}}}
\newcommand{\bhat}{\ensuremath{\hat{b}}}
\newcommand{\chat}{\ensuremath{\hat{c}}}
\newcommand{\dhat}{\ensuremath{\hat{d}}}
\newcommand{\ehat}{\ensuremath{\hat{e}}}
\newcommand{\fhat}{\ensuremath{\hat{f}}}
\newcommand{\ghat}{\ensuremath{\hat{g}}}
\newcommand{\hhat}{\ensuremath{\hat{h}}}
\newcommand{\ihat}{\ensuremath{\hat{\imath}}}
\newcommand{\jhat}{\ensuremath{\hat{\jmath}}}
\newcommand{\khat}{\ensuremath{\hat{k}}}
\newcommand{\lhat}{\ensuremath{\hat{l}}}
\newcommand{\mhat}{\ensuremath{\hat{m}}}
\newcommand{\nhat}{\ensuremath{\hat{n}}}
\newcommand{\ohat}{\ensuremath{\hat{o}}}
\newcommand{\phat}{\ensuremath{\hat{p}}}
\newcommand{\qhat}{\ensuremath{\hat{q}}}
\newcommand{\rhat}{\ensuremath{\hat{r}}}
\newcommand{\shat}{\ensuremath{\hat{s}}}
\newcommand{\that}{\ensuremath{\hat{t}}}
\newcommand{\uhat}{\ensuremath{\hat{u}}}
\newcommand{\vhat}{\ensuremath{\hat{v}}}
\newcommand{\what}{\ensuremath{\hat{w}}}
\newcommand{\xhat}{\ensuremath{\hat{x}}}
\newcommand{\yhat}{\ensuremath{\hat{y}}}
\newcommand{\zhat}{\ensuremath{\hat{z}}}

%% FONTS AND DECORATION FOR GREEK LETTERS

%% the package `mathbbol' gives us blackboard bold greek and numbers,
%% but it does it by redefining \mathbb to use a different font, so that
%% all the other \mathbb letters look different too.  Here we import the
%% font with bb greek and numbers, but assign it a different name,
%% \mathbbb, so as not to replace the usual one.
\DeclareSymbolFont{bbold}{U}{bbold}{m}{n}
\DeclareSymbolFontAlphabet{\mathbbb}{bbold}
\newcommand{\bbDelta}{\ensuremath{\mathbbb{\Delta}}}
\newcommand{\bbone}{\ensuremath{\mathbbb{1}}}
\newcommand{\bbtwo}{\ensuremath{\mathbbb{2}}}
\newcommand{\bbthree}{\ensuremath{\mathbbb{3}}}

% greek with bars
\newcommand{\albar}{\ensuremath{\overline{\alpha}}}
\newcommand{\bebar}{\ensuremath{\overline{\beta}}}
\newcommand{\gmbar}{\ensuremath{\overline{\gamma}}}
\newcommand{\debar}{\ensuremath{\overline{\delta}}}
\newcommand{\phibar}{\ensuremath{\overline{\varphi}}}
\newcommand{\psibar}{\ensuremath{\overline{\psi}}}
\newcommand{\xibar}{\ensuremath{\overline{\xi}}}
\newcommand{\ombar}{\ensuremath{\overline{\omega}}}

% greek with hats
\newcommand{\alhat}{\ensuremath{\hat{\alpha}}}
\newcommand{\behat}{\ensuremath{\hat{\beta}}}
\newcommand{\gmhat}{\ensuremath{\hat{\gamma}}}
\newcommand{\dehat}{\ensuremath{\hat{\delta}}}

% greek with checks
\newcommand{\alchk}{\ensuremath{\check{\alpha}}}
\newcommand{\bechk}{\ensuremath{\check{\beta}}}
\newcommand{\gmchk}{\ensuremath{\check{\gamma}}}
\newcommand{\dechk}{\ensuremath{\check{\delta}}}

% greek with tildes
\newcommand{\altil}{\ensuremath{\widetilde{\alpha}}}
\newcommand{\betil}{\ensuremath{\widetilde{\beta}}}
\newcommand{\gmtil}{\ensuremath{\widetilde{\gamma}}}
\newcommand{\phitil}{\ensuremath{\widetilde{\varphi}}}
\newcommand{\psitil}{\ensuremath{\widetilde{\psi}}}
\newcommand{\xitil}{\ensuremath{\widetilde{\xi}}}
\newcommand{\omtil}{\ensuremath{\widetilde{\omega}}}

% MISCELLANEOUS SYMBOLS
\mdef\del{\partial}
\mdef\delbar{\overline{\partial}}
\let\sm\wedge
\newcommand{\dd}[1]{\ensuremath{\frac{\partial}{\partial {#1}}}}
\newcommand{\inv}{^{-1}}
\newcommand{\dual}{^{\vee}}
\mdef\hf{\textstyle\frac{1}{2}}
\mdef\thrd{\textstyle\frac{1}{3}}
\mdef\qtr{\textstyle\frac{1}{4}}
\let\meet\wedge
\let\join\vee
\let\dn\downarrow
\newcommand{\op}{^{\mathit{op}}}
\newcommand{\co}{^{\mathit{co}}}
\newcommand{\coop}{^{\mathit{coop}}}
\let\adj\dashv
\SelectTips{cm}{}
\newdir{ >}{{}*!/-10pt/@{>}}    % extra spacing for tail arrows in XYpic
\newcommand{\pushoutcorner}[1][dr]{\save*!/#1+1.2pc/#1:(1,-1)@^{|-}\restore}
\newcommand{\pullbackcorner}[1][dr]{\save*!/#1-1.2pc/#1:(-1,1)@^{|-}\restore}
\let\iso\cong
\let\eqv\simeq
\let\cng\equiv
\mdef\Id{\mathrm{Id}}
\mdef\id{\mathrm{id}}
\alwaysmath{ell}
\alwaysmath{infty}
\alwaysmath{odot}
\def\frc#1/#2.{\frac{#1}{#2}}   % \frc x^2+1 / x^2-1 .
\mdef\ten{\mathrel{\otimes}}
\mdef\bigten{\bigotimes}
\mdef\sqten{\mathrel{\boxtimes}}
\def\pow(#1,#2){\mathop{\pitchfork}(#1,#2)} % powers and
\def\cpw{\mathop{\odot}}                    % copowers
\newcommand{\mathid}{\mbox{id}}
\newcommand{\cat}[1]{\ensuremath{\mathbf{#1}}}


%% OPERATORS
\DeclareMathOperator\lan{Lan}
\DeclareMathOperator\ran{Ran}
\DeclareMathOperator\colim{colim}
\DeclareMathOperator\coeq{coeq}
\DeclareMathOperator\eq{eq}
\DeclareMathOperator\Tot{Tot}
\DeclareMathOperator\cosk{cosk}
\DeclareMathOperator\sk{sk}
\DeclareMathOperator\im{im}
\DeclareMathOperator\Spec{Spec}
\DeclareMathOperator\Ho{Ho}
\DeclareMathOperator\Aut{Aut}
\DeclareMathOperator\End{End}
\DeclareMathOperator\Hom{Hom}
\DeclareMathOperator\Map{Map}

%% TIKZ ARROWS AND HIGHER CELLS
\makeatletter
\def\slashedarrowfill@#1#2#3#4#5{%
  $\m@th\thickmuskip0mu\medmuskip\thickmuskip\thinmuskip\thickmuskip
   \relax#5#1\mkern-7mu%
   \cleaders\hbox{$#5\mkern-2mu#2\mkern-2mu$}\hfill
   \mathclap{#3}\mathclap{#2}%
   \cleaders\hbox{$#5\mkern-2mu#2\mkern-2mu$}\hfill
   \mkern-7mu#4$%
}

\def\Rightslashedarrowfill@{%
  \slashedarrowfill@\Relbar\Relbar\Mapstochar\Rightarrow}
\newcommand\xslashedRightarrow[2][]{%
  \ext@arrow 0055{\Rightslashedarrowfill@}{#1}{#2}}
\def\hTo{\xslashedRightarrow{}}
\def\hToo{\xslashedRightarrow{\quad}}
\let\xhTo\xslashedRightarrow

\pagestyle{empty}

\newcommand{\Rightthreecell}{\RRightarrow}
\newcommand{\Rtwocell}{\Rightarrow}

\tikzstyle{doubletick}=[-implies, double equal sign distance, postaction={decorate},decoration={markings,mark=at position .5 with {\draw[-] (0,-0.1) -- (0,0.1);}}]

\tikzstyle{darrow}=[-implies, double equal sign distance]

\tikzstyle{doubleeq}=[double equal sign distance]


%% ARROWS
% \to already exists
\newcommand{\too}[1][]{\ensuremath{\overset{#1}{\longrightarrow}}}
\newcommand{\ot}{\ensuremath{\leftarrow}}
\newcommand{\oot}[1][]{\ensuremath{\overset{#1}{\longleftarrow}}}
\let\toot\rightleftarrows
\let\otto\leftrightarrows
\let\Impl\Rightarrow
\let\imp\Rightarrow
\let\toto\rightrightarrows
\let\into\hookrightarrow
\let\xinto\xhookrightarrow
\mdef\we{\overset{\sim}{\longrightarrow}}
\mdef\leftwe{\overset{\sim}{\longleftarrow}}
\let\mono\rightarrowtail
\let\leftmono\leftarrowtail
\let\cof\rightarrowtail
\let\leftcof\leftarrowtail
\let\epi\twoheadrightarrow
\let\leftepi\twoheadleftarrow
\let\fib\twoheadrightarrow
\let\leftfib\twoheadleftarrow
\let\cohto\rightsquigarrow
\let\maps\colon
\newcommand{\spam}{\,:\!}       % \maps for left arrows

\newsavebox{\DDownarrowbox}
\savebox{\DDownarrowbox}{\tikz[scale=1.5]{\draw[-implies,double equal
sign distance] (0,.1) -- (0,-.1); \draw (0,.1) -- (0,-.1);}}
\newcommand{\DDownarrow}{\mathrel{\raisebox{-.2em}{\usebox{\DDownarrowbox}}}}

\newsavebox{\RRightarrowbox}
\savebox{\RRightarrowbox}{\tikz[scale=1.5]{\draw[-implies,double equal
sign distance] (-.1,0) -- (.1,0); \draw (-.1,0) -- (.1,0);}}
\newcommand{\RRightarrow}{\mathrel{\raisebox{.2em}{\usebox{\RRightarrowbox}}}}

%\newsavebox{\Rightslashedarrowbox}
%\savebox{\Rightslashedarrowbox}{\tikz[scale=1.5]{\draw[Rightslashedarrow{}] (-.1,0) -- (1,0);}}
%\newcommand{\Rightslashedarrow}{\mathrel{\raisebox{-.2em}%{\usebox{\Rightslashedarrowbox}}}}


%% EXTENSIBLE ARROWS
\let\xto\xrightarrow
\let\xot\xleftarrow
% See Voss' Mathmode.tex for instructions on how to create new
% extensible arrows.
\def\rightarrowtailfill@{\arrowfill@{\Yright\joinrel\relbar}\relbar\rightarrow}
\newcommand\xrightarrowtail[2][]{\ext@arrow 0055{\rightarrowtailfill@}{#1}{#2}}
\let\xmono\xrightarrowtail
\let\xcof\xrightarrowtail
\def\twoheadrightarrowfill@{\arrowfill@{\relbar\joinrel\relbar}\relbar\twoheadrightarrow}
\newcommand\xtwoheadrightarrow[2][]{\ext@arrow 0055{\twoheadrightarrowfill@}{#1}{#2}}
\let\xepi\xtwoheadrightarrow
\let\xfib\xtwoheadrightarrow
% Let's leave the left-going ones until I need them.

%% EXTENSIBLE SLASHED ARROWS
% Making extensible slashed arrows, by modifying the underlying AMS code.
% Arguments are:
% 1 = arrowhead on the left (\relbar or \Relbar if none)
% 2 = fill character (usually \relbar or \Relbar)
% 3 = slash character (such as \mapstochar or \Mapstochar)
% 4 = arrowhead on the left (\relbar or \Relbar if none)
% 5 = display mode (\displaystyle etc)
\def\slashedarrowfill@#1#2#3#4#5{%
  $\m@th\thickmuskip0mu\medmuskip\thickmuskip\thinmuskip\thickmuskip
   \relax#5#1\mkern-7mu%
   \cleaders\hbox{$#5\mkern-2mu#2\mkern-2mu$}\hfill
   \mathclap{#3}\mathclap{#2}%
   \cleaders\hbox{$#5\mkern-2mu#2\mkern-2mu$}\hfill
   \mkern-7mu#4$%
}
% Here's the idea: \<slashed>arrowfill@ should be a box containing
% some stretchable space that is the "middle of the arrow".  This
% space is created as a "leader" using \cleader<thing>\hfill, which
% fills an \hfill of space with copies of <thing>.  Here instead of
% just one \cleader, we use two, with the slash in between (and an
% extra copy of the filler, to avoid extra space around the slash).
\def\rightslashedarrowfill@{%
  \slashedarrowfill@\relbar\relbar\mapstochar\rightarrow}
\newcommand\xslashedrightarrow[2][]{%
  \ext@arrow 0055{\rightslashedarrowfill@}{#1}{#2}}
\mdef\hto{\xslashedrightarrow{}}
\mdef\htoo{\xslashedrightarrow{\quad}}
\let\xhto\xslashedrightarrow

%% To get a slashed arrow in XYpic, do
% \[\xymatrix{A \ar[r]|-@{|} & B}\]

% ISOMORPHISMS
\def\xiso#1{\mathrel{\mathrlap{\smash{\xto[\smash{\raisebox{1.3mm}{$\scriptstyle\sim$}}]{#1}}}\hphantom{\xto{#1}}}}
\def\toiso{\xto{\smash{\raisebox{-.5mm}{$\scriptstyle\sim$}}}}

% SHADOWS
\def\shvar#1#2{{\ensuremath{%
  \hspace{1mm}\makebox[-1mm]{$#1\langle$}\makebox[0mm]{$#1\langle$}\hspace{1mm}%
  {#2}%
  \makebox[1mm]{$#1\rangle$}\makebox[0mm]{$#1\rangle$}%
}}}
\def\sh{\shvar{}}
\def\scriptsh{\shvar{\scriptstyle}}
\def\bigsh{\shvar{\big}}
\def\Bigsh{\shvar{\Big}}
\def\biggsh{\shvar{\bigg}}
\def\Biggsh{\shvar{\Bigg}}

%HIGHER CELLS



% THEOREM-TYPE ENVIRONMENTS, hacked to
%% (a) number all with the same numbers, and
%% (b) have the right names for autoref
\def\defthm#1#2{%
  \newtheorem{#1}{#2}[section]%
  \expandafter\def\csname #1autorefname\endcsname{#2}%
  \expandafter\let\csname c@#1\endcsname\c@thm}
\newtheorem{thm}{Theorem}[section]
\newcommand{\thmautorefname}{Theorem}
\defthm{cor}{Corollary}
\defthm{prop}{Proposition}
\defthm{lem}{Lemma}
\defthm{sch}{Scholium}
\defthm{assume}{Assumption}
\defthm{claim}{Claim}
\defthm{conj}{Conjecture}
\defthm{hyp}{Hypothesis}
\defthm{fact}{Fact}
\theoremstyle{definition}
\defthm{defn}{Definition}
\defthm{notn}{Notation}
\theoremstyle{remark}
\defthm{rmk}{Remark}
\defthm{eg}{Example}
\defthm{egs}{Examples}
\defthm{ex}{Exercise}
\defthm{ceg}{Counterexample}

% How to get QED symbols inside equations at the end of the statements
% of theorems.  AMS LaTeX knows how to do this inside equations at the
% end of *proofs* with \qedhere, and at the end of the statement of a
% theorem with \qed (meaning no proof will be given), but it can't
% seem to combine the two.  Use this instead.
\def\thmqedhere{\expandafter\csname\csname @currenvir\endcsname @qed\endcsname}

% Number numbered lists as (i), (ii), ...
\renewcommand{\theenumi}{(\roman{enumi})}
\renewcommand{\labelenumi}{\theenumi}

%% Labeling that keeps track of theorem-type names.  Superseded by
%% autoref from hyperref, as above, but we keep this in case we are
%% using a journal style file that is incompatible with hyperref.
% 
% \ifx\SK@label\undefined\let\SK@label\label\fi
% \let\your@thm\@thm
% \def\@thm#1#2#3{\gdef\currthmtype{#3}\your@thm{#1}{#2}{#3}}
% \def\xlabel#1{{\let\your@currentlabel\@currentlabel\def\@currentlabel
% {\currthmtype~\your@currentlabel}
% \SK@label{#1@}}\label{#1}}
% \def\xref#1{\ref{#1@}}

% Also number formulas with the theorem counter
\let\c@equation\c@thm
\numberwithin{equation}{section}

% Only show numbers for equations that are actually referenced (or
% whose tags are specified manually) - uses mathtools.
\mathtoolsset{showonlyrefs,showmanualtags}

%% Fix enumerate spacing with paralist.  This has two parts:
%%   1. enable mixing of "old spacing" lists with those adjusted by paralist
%%   2. allow us to specify a number based on which to adjust the spacing
%% For the first, use \killspacingtrue when you want the spacing
%% adjusted, then \killspacingfalse to turn adjustment off.  For the
%% second, use \maxenum=14 to set the widest number you want the
%% spacing to be calculated with.
\newlength\oldleftmargini       % save old spacing
\newlength\oldleftmarginii
\newlength\oldleftmarginiii
\newlength\oldleftmarginiv
\newlength\oldleftmarginv
\newlength\oldleftmarginvi
\newcount\maxenum
\maxenum=7
\newif\ifkillspacing
\def\@adjust@enum@labelwidth{%
  \advance\@listdepth by 1\relax
  \ifkillspacing                % do the paralist thing
    \csname c@\@enumctr\endcsname\maxenum
    \settowidth{\@tempdima}{%
      \csname label\@enumctr\endcsname\hspace{\labelsep}}%
    \csname leftmargin\romannumeral\@listdepth\endcsname
      \@tempdima
  \else                         % otherwise, reset it
    \csname fixspacing\romannumeral\@listdepth\endcsname
  \fi
  \advance\@listdepth by -1\relax}
% these commands, one for each enum level (I couldn't get a generic
% one to work), test whether oldleftmargin has been set yet, and if
% not, set it from leftmargin; otherwise, they reset leftmargin to
% it.  Just setting oldleftmargin to leftmargin in the preamble
% doesn't seem to work.
\def\fixspacingi{\ifnum\oldleftmargini=0\setlength\oldleftmargini\leftmargini\else\setlength\leftmargini\oldleftmargini\fi}
\def\fixspacingii{\ifnum\oldleftmarginii=0\setlength\oldleftmarginii\leftmarginii\else\setlength\leftmarginii\oldleftmarginii\fi}
\def\fixspacingiii{\ifnum\oldleftmarginiii=0\setlength\oldleftmarginiii\leftmarginiii\else\setlength\leftmarginiii\oldleftmarginiii\fi}
\def\fixspacingiv{\ifnum\oldleftmarginiv=0\setlength\oldleftmarginiv\leftmarginiv\else\setlength\leftmarginiv\oldleftmarginiv\fi}
\def\fixspacingv{\ifnum\oldleftmarginv=0\setlength\oldleftmarginv\leftmarginv\else\setlength\leftmarginv\oldleftmarginv\fi}
\def\fixspacingvi{\ifnum\oldleftmarginvi=0\setlength\oldleftmarginvi\leftmarginvi\else\setlength\leftmarginvi\oldleftmarginvi\fi}

%% Fix paralist references, so that we can refer to (1) instead of
%% just 1.
\def\pl@label#1#2{%
  \edef\pl@the{\noexpand#1{\@enumctr}}%
  \pl@lab\expandafter{\the\pl@lab\csname yourthe\@enumctr\endcsname}%
  \advance\@tempcnta1
  \pl@loop}
\def\@enumlabel@#1[#2]{%
  \@plmylabeltrue
  \@tempcnta0
  \pl@lab{}%
  \let\pl@the\pl@qmark
  \expandafter\pl@loop\@gobble#2\@@@
  \ifnum\@tempcnta=1\else
    \PackageWarning{paralist}{Incorrect label; no or multiple
      counters.\MessageBreak The label is: \@gobble#2}%
  \fi
  \expandafter\edef\csname label\@enumctr\endcsname{\the\pl@lab}%
  \expandafter\edef\csname the\@enumctr\endcsname{\the\pl@lab}%
  \expandafter\let\csname yourthe\@enumctr\endcsname\pl@the
  #1}


% GREEK LETTERS, ETC.
\alwaysmath{alpha}
\alwaysmath{beta}
\alwaysmath{gamma}
\alwaysmath{Gamma}
\alwaysmath{delta}
\alwaysmath{Delta}
\alwaysmath{epsilon}
\mdef\ep{\varepsilon}
\alwaysmath{zeta}
\alwaysmath{eta}
\alwaysmath{theta}
\alwaysmath{Theta}
\alwaysmath{iota}
\alwaysmath{kappa}
\alwaysmath{lambda}
\alwaysmath{Lambda}
\alwaysmath{mu}
\alwaysmath{nu}
\alwaysmath{xi}
\alwaysmath{pi}
\alwaysmath{rho}
\alwaysmath{sigma}
\alwaysmath{Sigma}
\alwaysmath{tau}
\alwaysmath{upsilon}
\alwaysmath{Upsilon}
\alwaysmath{phi}
\alwaysmath{Pi}
\alwaysmath{Phi}
\mdef\ph{\varphi}
\alwaysmath{chi}
\alwaysmath{psi}
\alwaysmath{Psi}
\alwaysmath{omega}
\alwaysmath{Omega}
\let\al\alpha
\let\be\beta
\let\gm\gamma
\let\Gm\Gamma
\let\de\delta
\let\De\Delta
\let\si\sigma
\let\Si\Sigma
\let\om\omega
\let\ka\kappa
\let\la\lambda
\let\La\Lambda
\let\ze\zeta
\let\th\theta
\let\Th\Theta
\let\vth\vartheta

\makeatother

% Tikz styles
\tikzstyle{tickarrow}=[->,postaction={decorate},decoration={markings,mark=at position .5 with {\draw[-] (0,-0.1) -- (0,0.1);}},line width=0.50]

% Local Variables:
% mode: latex
% TeX-master: ""
% End:

\begin{document}

{\small
\begin{equation*}\hspace{-2cm}
\begin{tikzpicture}[xscale=2.5, yscale=3]
%%%%Row A
\node (A0) at (0,5) {$\substack{\tens(\tens \times \transid)\\(\tens \times \transid \times \transid)\\(\tens \times \transid \times \transid \times \transid)}$};
\node (A1) at (0,6.5) {$\substack{\tens(\tens \times \transid)\\(\tens \times \transid \times \transid)\\(\transid \times \tens \times \transid \times \transid)}$};
\node (A2) at (1.5,7) {$\substack{\tens(\tens \times \transid)\\(\transid \times \tens \times \transid)\\(\transid \times \tens \times \transid \times \transid)}$};
\node (A3) at (3,7.5) {$\substack{\tens(\tens \times \transid)\\(\transid \times \tens \times \transid)\\(\transid \times \transid \times \tens \times \transid)}$};
\node (A4) at (4.5,7) {$\substack{\tens(\transid \times \tens)\\(\transid \times \tens \times \transid)\\(\transid \times \transid \times \tens \times \transid)}$};
\node (A5) at (6,6.5) {$\substack{\tens(\transid \times \tens)\\(\transid  \times \transid \times \tens)\\(\transid \times \transid \times \tens \times \transid)}$};
\node (A6) at (6,5) {$\substack{\tens(\transid \times \tens)\\(\transid  \times \transid \times \tens)\\(\transid \times \transid \times \transid \times \tens )}$};
%%%%%%
\draw[doubleloose] (A0) to node[above, xshift=-20pt]{$\substack{\looseid \looseid \\(\alpha \times \looseid \times \looseid)}$} (A1);
\draw[doubleloose] (A1) to node[above, xshift=-16pt]{$\substack{\looseid (\alpha \times \looseid) \looseid}$}
(A2);
\draw[doubleloose] (A2) to node[above, xshift=-16pt]{$\substack{\looseid \looseid \\ (\looseid \times \alpha \times \looseid)}$} (A3);
\draw[doubleloose] (A3) to node[above, xshift=16pt]{$\substack{ \alpha \looseid \looseid}$} (A4);
\draw[doubleloose] (A4) to node[above, xshift=16pt]{$\substack{\looseid (\looseid \times \alpha) \looseid}$} (A5);
\draw[doubleloose] (A5) to node[above, xshift=16pt]{$\substack{ \looseid \looseid \\ (\looseid \times \looseid \times \alpha)}$} (A6);
%%%%row B
\node (B3) at (3,6.5) {$\substack{\tens(\transid \times \tens)\\(\transid \times \tens \times \transid)\\(\transid \times \tens  \times \transid \times \transid)}$};
%%%%%%
\draw[doubleloose] (A2) to node[above, xshift=16pt]{$\substack{ \alpha \looseid \looseid }$} (B3);
\draw[doubleloose] (B3) to node[above, xshift=-16pt]{$\substack{ \looseid \looseid \\ (\looseid \times \alpha \times \looseid)}$} (A4);
%%%%RowC
\node (C4) at (3,5.5) {$\substack{\tens(\transid \times \tens)\\(\transid  \times \transid \times \tens) \\ (\transid \times \tens \times \transid \times \transid)}$};
\node (C5) at (4,5) {$\substack{\tens(\transid \times \tens)\\(\transid  \times \tens\times \transid) \\ (\transid \times \tens \times \transid \times \transid)}$};
%%%%%% Extra cells
\draw[doubleloose] (A1) to[out=0, in=125] node[above]{$\substack{S(\pi)\looseid}$} (C4);
\draw[doubleloose] (A1) to[out=-35, in=180] node[above, xshift=5pt]{$\substack{T(\pi) \looseid}$}(C4);
%%%%%%%Extra cells
\draw[doubleloose] (B3) to[out=0, in=115] node[above, xshift=15pt]{$\substack{\looseid \\( \looseid \times S(\pi))}$} (A6);
\draw[doubleloose] (B3) to[out=-35, in=160] node[above, xshift=15pt]{$\substack{\looseid \\( \looseid \times T(\pi))}$}(A6);
%%%%%%%%%
\draw[doubleloose] (B3) to node[right]{$\substack{ \looseid \\ (\looseid \times \alpha) \\ \looseid}$} (C4);
\draw[doubletighteq] (C4) to (C5);
\draw[doubleloose] (C5) to node[above]{$\substack{\looseid (\looseid \times \alpha) \looseid}$} (A6);
%%%%RowD
\node (D2) at (1,5) {$\substack{\tens(\transid \times \tens)\\(\tens \times \transid \times \transid)\\(\transid \times \tens \times \transid \times \transid)}$};
\node (D3) at (2,5) {$\substack{\tens(\tens \times \transid)\\(\transid \times \transid \times \tens)\\(\transid \times \tens \times \transid \times \transid)}$};
%%%%%%
\draw[doubleloose] (A1) to node[above, xshift=10pt] {$\substack{\alpha \looseid \looseid}$}   (D2);
\draw[doubletighteq] (D2) to  (D3);
\draw[doubleloose] (D3) to node[above, xshift=-12pt, yshift=-3pt] {$\substack{ \alpha \looseid \looseid}$} (C4);
%%%%RowE
\node (E1) at (1 ,3.5) {$\substack{\tens( \transid \times \tens)\\(\tens \times \transid \times \transid)\\(\tens \times \transid \times  \transid \times \transid)}$};
\node (E3) at (2.5 ,4.5) {$\substack{\tens( \tens \times \transid)\\(\transid \times \tens \times \transid)\\(\transid \times  \transid \times \transid \times \tens )}$};
%%%%%%
\draw[doubleloose] (A0) to node[above,xshift=-23]{$\substack{\alpha \looseid \looseid}$} (E1);
\draw[doubleloose] (E1) to node[above,xshift=20] {$\substack{\looseid \looseid \\ (\alpha \times \looseid \times \looseid)}$}   (D2);
\draw[doubletighteq] (D3) to  (E3);
\draw[doubleloose] (E3) to node[above,xshift=-10pt, yshift=2pt] {$\substack{\alpha \looseid \looseid}$}   (C5);
%%%%Row G
\node (G3) at (2.5,3) {$\substack{\tens (\tens \times \transid) \\(\transid \times \transid \times \tens) \\ (\tens \times \transid \times \transid \times \transid)}$};
\node (G4) at (4,3) {$\substack{\tens (\transid \times \tens) \\ ( \transid \times \transid \times \tens) \\ (\tens \times \transid \times \transid \times \transid)}$};
\node (G5) at (5,3) {$\substack{\tens(\tens\times \transid)\\(\transid  \times \transid \times \tens)\\(\transid \times \transid  \times \transid \times \tens)}$};
%%%%%%
\draw[doubletighteq] (E1) to  (G3);
\draw[doubleloose] (G3) to  node[above]{$\substack{ \alpha \looseid \looseid}$}(G4);
\draw[doubletighteq] (G4) to  (G5);
\draw[doubleloose] (G5) to  node[above, xshift=20pt]{$\substack{ \alpha \looseid  \looseid}$}(A6);
\draw[doubleloose] (G3) to  node[right]{$\substack{ \looseid \\ (\alpha \times \looseid) \\ \looseid}$}(E3);
%%%%%Extra cells
\draw[doubleloose] (G3) to[out=60, in=200] node[above, xshift=-15pt]{$\substack{S(\pi) \looseid}$} (A6);
\draw[doubleloose] (G3) to[out=15, in=230] node[below, xshift=15pt]{$\substack{T(\pi) \looseid }$}(A6);
%%%%%3-cells
\node at (3,7) {$\substack{\DDownarrow \iso }$};
\node at (1.6,6.7) {$\substack{\DDownarrow \iso }$};
\node at (4.5,6.7) {$\substack{\DDownarrow \iso }$};
\node at (4.25,6) {$\substack{\DDownarrow \tightid (\tightid \times \pi)}$};
\node at (.5,5) {$\substack{\DDownarrow \iso}$};
\node at (1.5,6) {$\substack{\DDownarrow \pi \tightid}$};
\node at (1.5,5.3) {$\substack{\DDownarrow \iso }$};
\node at (4,5.5) {$\substack{\DDownarrow \iso }$};
\node at (3,5) {$\substack{=}$};
\node at (1.5,4) {$\substack{\DDownarrow \iso}$};
\node at (3.5,4.5) {$\substack{\DDownarrow \iso}$};
\node at (4,4) {$\substack{\DDownarrow \pi \tightid}$};
\node at (4.5,3.5) {$\substack{\DDownarrow \iso}$};
\end{tikzpicture} \hspace{-2cm}
\end{equation*}
\begin{equation}\label{eq:monobjeq1}
=
\end{equation}
\begin{equation*}\hspace{-2cm}
\begin{tikzpicture}[xscale=2.5, yscale=3]
%%%%Row A
\node (A0) at (0,6) {$\substack{\tens(\tens \times \transid)\\(\tens \times \transid \times \transid)\\(\tens \times \transid \times \transid \times \transid)}$};
\node (A1) at (0,7) {$\substack{\tens(\tens \times \transid)\\(\tens \times \transid \times \transid)\\(\transid \times \tens \times \transid \times \transid)}$};
\node (A2) at (1.5,7.5) {$\substack{\tens(\tens \times \transid)\\(\transid \times \tens \times \transid)\\(\transid \times \tens \times \transid \times \transid)}$};
\node (A3) at (3,7) {$\substack{\tens(\tens \times \transid)\\(\transid \times \tens \times \transid)\\(\transid \times \transid \times \tens \times \transid)}$};
\node (A4) at (4,7) {$\substack{\tens(\transid \times \tens)\\(\transid \times \tens \times \transid)\\(\transid \times \transid \times \tens \times \transid)}$};
\node (A5) at (5,6) {$\substack{\tens(\transid \times \tens)\\(\transid  \times \transid \times \tens)\\(\transid \times \transid \times \tens \times \transid)}$};
\node (A6) at (6,5) {$\substack{\tens(\transid \times \tens)\\(\transid  \times \transid \times \tens)\\(\transid \times \transid \times \transid \times \tens )}$};
%%%%%%
\draw[doubleloose] (A0) to node[left, xshift=-5pt]{$\substack{\looseid \looseid \\(\alpha \times \looseid \times \looseid)}$} (A1);
\draw[doubleloose] (A1) to node[above, xshift=-16pt]{$\substack{\looseid (\alpha \times \looseid) \looseid}$}
(A2);
\draw[doubleloose] (A2) to node[above, xshift=16pt]{$\substack{\looseid \looseid \\ (\looseid \times \alpha \times \looseid)}$} (A3);
\draw[doubleloose] (A3) to node[above]{$\substack{ \alpha \looseid \looseid}$} (A4);
\draw[doubleloose] (A4) to node[above, xshift=16pt]{$\substack{\looseid (\looseid \times \alpha) \looseid}$} (A5);
\draw[doubleloose] (A5) to node[above, xshift=16pt]{$\substack{ \looseid \looseid \\ (\looseid \times \looseid \times \alpha)}$} (A6);
%%%%row B
\node (B1) at (1.5,5.5) {$\substack{\tens (\tens \times \transid) \\ (\transid \times \tens \times \transid) \\ (\tens \times \transid \times \transid \times \transid)}$};
\node (B2) at (3,6) {$\substack{\tens (\tens \times \transid) \\ (\tens \times \transid \times \transid) \\ (\transid \times \transid \times \tens \times \transid)}$};
%%%%%%
\draw[doubleloose] (A0) to node[above, xshift=16pt]{$\substack{\looseid (\alpha \times \looseid)  \looseid}$} (B1);
\draw[doubletighteq] (B1) to  (B2);
\draw[doubleloose] (B2) to node[below, xshift=16pt]{$\substack{ \looseid (\alpha \times \looseid) }$} (A3);
%%%%row C
\node (C1) at (2.5,4.5) {$\substack{\tens( \tens \times \transid)\\( \transid \times \tens \times  \transid)\\(\tens \times \transid \times \transid \times  \transid)}$};
\node (C3) at (3.5,5) {$\substack{\tens( \transid \times \tens)\\( \tens \times \transid \times \transid)\\(\transid \times \transid \times \tens \times \transid)}$};
\node (C4) at (4.5,5) {$\substack{\tens(\tens \times \transid)\\(\transid  \times \transid \times \tens)\\(\transid \times \transid \times \tens \times \transid)}$};
%%%%%%
\draw[doubleloose] (B1) to node[right, xshift=5pt]{$\substack{ \alpha \looseid  \looseid}$} (C1);
\draw[doubletighteq] (C1) to  (C3);
\draw[doubleloose] (B2) to node[left, xshift=-3pt]{$\substack{ \alpha \looseid  \looseid}$} (C3);
\draw[doubletighteq] (C3) to (C4);
\draw[doubleloose] (C4) to node[right, xshift=2pt]{$\substack{ \alpha \looseid  \looseid}$} (A5);
%%%%row D
\node (D5) at (5,4) {$\substack{\tens(\tens \times \transid)\\(\transid  \times \transid \times \tens)\\(\transid \times \transid  \times \transid \times \tens)}$};
%%%%%%
\draw[doubleloose] (C4) to node[below, xshift=-16pt]{$\substack{  \looseid  \looseid \\ (\looseid \times \looseid \times \alpha)}$} (D5);
\draw[doubleloose] (D5) to  node[right, xshift=2pt]{$\substack{ \alpha \looseid  \looseid}$}(A6);
%%%%row E
\node (E2) at (0,4.5) {$\substack{\tens (\transid \times \tens)\\(\tens \times \transid \times \transid) \\ (\tens \times \transid \times \transid \times \transid)}$};
\node (E3) at (1,3.5) {$\substack{\tens (\tens \times \transid) \\(\transid \times \transid \times \tens) \\ (\tens \times \transid \times \transid \times \transid)}$};
\node (E4) at (2.5,3.5) {$\substack{\tens (\transid \times \tens) \\ ( \transid \times \transid \times \tens) \\ (\tens \times \transid \times \transid \times \transid)}$};
%%%%%%
\draw[doubleloose] (A0) to node[left, xshift=-2pt]{$\substack{ \alpha \looseid \looseid}$} (E2);
\draw[doubletighteq] (E2) to  (E3);
\draw[doubleloose] (E3) to  node[above]{$\substack{ \alpha \looseid \looseid}$}(E4);
\draw[doubletighteq] (E4) to  (D5);
\draw[doubleloose] (C1) to  node[right]{$\substack{ \looseid\\ (\looseid \times \alpha) \\ \looseid}$}(E4);
%%%%% 3cells
\node at (1.5,6.5) {$\substack{ \DDownarrow \tightid (\pi \times \tightid)}$};
\node at (4,6) {$\substack{\DDownarrow \pi \tightid} $};
\node at (2,4.75) {$\substack{\DDownarrow \iso} $};
\node at (1.25,4.75) {$\substack{\DDownarrow \pi \tightid} $};
\node at (.5,4.25) {$\substack{\DDownarrow \iso} $};
\node at (2.5,5.25) {$\substack{=}$};
\node at (5.25,5) {$\substack{\DDownarrow \iso}$};
\node at (3.5,4.5) {$\substack{\DDownarrow \iso} $};
%%%%%Extra cells
\draw[doubleloose] (A0) to[out=60, in=180] node[above, xshift=-15pt]{$\substack{\looseid (S(\pi) \times \looseid) }$} (A3);
\draw[doubleloose] (A0) to[out=0, in=230] node[below, xshift=15pt]{$\substack{\looseid (T(\pi) \times \looseid)  }$}(A3);
%%%%%Extra cells
\draw[doubleloose] (B2) to[out=35, in=155] node[above]{$\substack{S(\pi) \looseid }$} (A5);
\draw[doubleloose] (B2) to[out=-35, in=205] node[below]{$\substack{T(\pi)  \looseid  }$}(A5);
%%%%%Extra cells
\draw[doubleloose] (A0) to[out=-30, in=120] node[right]{$\substack{S(\pi) \looseid }$} (E4);
\draw[doubleloose] (A0) to[out=-70, in=160] node[left]{$\substack{T(\pi)  \looseid  }$}(E4);
\end{tikzpicture}\hspace{-2cm}
\end{equation*}}
\end{document}  \newpage
%
\documentclass[12pt]{ociamthesis}
\usepackage{tikz}
\usepackage{amsmath}
\usepackage{rotating}

\usepackage{amssymb,amsmath,stmaryrd,txfonts,mathrsfs,amsthm}
\usepackage[all,2cell]{xy}
\usepackage[neveradjust]{paralist}
\usepackage{hyperref}
\usepackage{mathtools}
\usepackage{tikz}
\usetikzlibrary{trees}
\usetikzlibrary{topaths}
\usetikzlibrary{decorations}
\usetikzlibrary{decorations.pathreplacing}
\usetikzlibrary{decorations.pathmorphing}
\usetikzlibrary{decorations.markings}
\usetikzlibrary{matrix,backgrounds,folding}
\usetikzlibrary{chains,scopes,positioning,fit}
\usetikzlibrary{arrows,shadows}
\usetikzlibrary{calc} 
\usetikzlibrary{chains}
\usetikzlibrary{shapes,shapes.geometric,shapes.misc}
\usepackage{smbicat}


\makeatletter
\let\ea\expandafter

%% Defining commands that are always in math mode.
\def\mdef#1#2{\ea\ea\ea\gdef\ea\ea\noexpand#1\ea{\ea\ensuremath\ea{#2}}}
\def\alwaysmath#1{\ea\ea\ea\global\ea\ea\ea\let\ea\ea\csname your@#1\endcsname\csname #1\endcsname
  \ea\def\csname #1\endcsname{\ensuremath{\csname your@#1\endcsname}}}

% Script letters
\newcommand{\sA}{\ensuremath{\mathscr{A}}}
\newcommand{\sB}{\ensuremath{\mathscr{B}}}
\newcommand{\sC}{\ensuremath{\mathscr{C}}}
\newcommand{\sD}{\ensuremath{\mathscr{D}}}
\newcommand{\sE}{\ensuremath{\mathscr{E}}}
\newcommand{\sF}{\ensuremath{\mathscr{F}}}
\newcommand{\sG}{\ensuremath{\mathscr{G}}}
\newcommand{\sH}{\ensuremath{\mathscr{H}}}
\newcommand{\sI}{\ensuremath{\mathscr{I}}}
\newcommand{\sJ}{\ensuremath{\mathscr{J}}}
\newcommand{\sK}{\ensuremath{\mathscr{K}}}
\newcommand{\sL}{\ensuremath{\mathscr{L}}}
\newcommand{\sM}{\ensuremath{\mathscr{M}}}
\newcommand{\sN}{\ensuremath{\mathscr{N}}}
\newcommand{\sO}{\ensuremath{\mathscr{O}}}
\newcommand{\sP}{\ensuremath{\mathscr{P}}}
\newcommand{\sQ}{\ensuremath{\mathscr{Q}}}
\newcommand{\sR}{\ensuremath{\mathscr{R}}}
\newcommand{\sS}{\ensuremath{\mathscr{S}}}
\newcommand{\sT}{\ensuremath{\mathscr{T}}}
\newcommand{\sU}{\ensuremath{\mathscr{U}}}
\newcommand{\sV}{\ensuremath{\mathscr{V}}}
\newcommand{\sW}{\ensuremath{\mathscr{W}}}
\newcommand{\sX}{\ensuremath{\mathscr{X}}}
\newcommand{\sY}{\ensuremath{\mathscr{Y}}}
\newcommand{\sZ}{\ensuremath{\mathscr{Z}}}

% Calligraphic letters
\newcommand{\cA}{\ensuremath{\mathcal{A}}}
\newcommand{\cB}{\ensuremath{\mathcal{B}}}
\newcommand{\cC}{\ensuremath{\mathcal{C}}}
\newcommand{\cD}{\ensuremath{\mathcal{D}}}
\newcommand{\cE}{\ensuremath{\mathcal{E}}}
\newcommand{\cF}{\ensuremath{\mathcal{F}}}
\newcommand{\cG}{\ensuremath{\mathcal{G}}}
\newcommand{\cH}{\ensuremath{\mathcal{H}}}
\newcommand{\cI}{\ensuremath{\mathcal{I}}}
\newcommand{\cJ}{\ensuremath{\mathcal{J}}}
\newcommand{\cK}{\ensuremath{\mathcal{K}}}
\newcommand{\cL}{\ensuremath{\mathcal{L}}}
\newcommand{\cM}{\ensuremath{\mathcal{M}}}
\newcommand{\cN}{\ensuremath{\mathcal{N}}}
\newcommand{\cO}{\ensuremath{\mathcal{O}}}
\newcommand{\cP}{\ensuremath{\mathcal{P}}}
\newcommand{\cQ}{\ensuremath{\mathcal{Q}}}
\newcommand{\cR}{\ensuremath{\mathcal{R}}}
\newcommand{\cS}{\ensuremath{\mathcal{S}}}
\newcommand{\cT}{\ensuremath{\mathcal{T}}}
\newcommand{\cU}{\ensuremath{\mathcal{U}}}
\newcommand{\cV}{\ensuremath{\mathcal{V}}}
\newcommand{\cW}{\ensuremath{\mathcal{W}}}
\newcommand{\cX}{\ensuremath{\mathcal{X}}}
\newcommand{\cY}{\ensuremath{\mathcal{Y}}}
\newcommand{\cZ}{\ensuremath{\mathcal{Z}}}

% blackboard bold letters
\newcommand{\lA}{\ensuremath{\mathbb{A}}}
\newcommand{\lB}{\ensuremath{\mathbb{B}}}
\newcommand{\lC}{\ensuremath{\mathbb{C}}}
\newcommand{\lD}{\ensuremath{\mathbb{D}}}
\newcommand{\lE}{\ensuremath{\mathbb{E}}}
\newcommand{\lF}{\ensuremath{\mathbb{F}}}
\newcommand{\lG}{\ensuremath{\mathbb{G}}}
\newcommand{\lH}{\ensuremath{\mathbb{H}}}
\newcommand{\lI}{\ensuremath{\mathbb{I}}}
\newcommand{\lJ}{\ensuremath{\mathbb{J}}}
\newcommand{\lK}{\ensuremath{\mathbb{K}}}
\newcommand{\lL}{\ensuremath{\mathbb{L}}}
\newcommand{\lM}{\ensuremath{\mathbb{M}}}
\newcommand{\lN}{\ensuremath{\mathbb{N}}}
\newcommand{\lO}{\ensuremath{\mathbb{O}}}
\newcommand{\lP}{\ensuremath{\mathbb{P}}}
\newcommand{\lQ}{\ensuremath{\mathbb{Q}}}
\newcommand{\lR}{\ensuremath{\mathbb{R}}}
\newcommand{\lS}{\ensuremath{\mathbb{S}}}
\newcommand{\lT}{\ensuremath{\mathbb{T}}}
\newcommand{\lU}{\ensuremath{\mathbb{U}}}
\newcommand{\lV}{\ensuremath{\mathbb{V}}}
\newcommand{\lW}{\ensuremath{\mathbb{W}}}
\newcommand{\lX}{\ensuremath{\mathbb{X}}}
\newcommand{\lY}{\ensuremath{\mathbb{Y}}}
\newcommand{\lZ}{\ensuremath{\mathbb{Z}}}

% bold letters
\newcommand{\bA}{\ensuremath{\mathbf{A}}}
\newcommand{\bB}{\ensuremath{\mathbf{B}}}
\newcommand{\bC}{\ensuremath{\mathbf{C}}}
\newcommand{\bD}{\ensuremath{\mathbf{D}}}
\newcommand{\bE}{\ensuremath{\mathbf{E}}}
\newcommand{\bF}{\ensuremath{\mathbf{F}}}
\newcommand{\bG}{\ensuremath{\mathbf{G}}}
\newcommand{\bH}{\ensuremath{\mathbf{H}}}
\newcommand{\bI}{\ensuremath{\mathbf{I}}}
\newcommand{\bJ}{\ensuremath{\mathbf{J}}}
\newcommand{\bK}{\ensuremath{\mathbf{K}}}
\newcommand{\bL}{\ensuremath{\mathbf{L}}}
\newcommand{\bM}{\ensuremath{\mathbf{M}}}
\newcommand{\bN}{\ensuremath{\mathbf{N}}}
\newcommand{\bO}{\ensuremath{\mathbf{O}}}
\newcommand{\bP}{\ensuremath{\mathbf{P}}}
\newcommand{\bQ}{\ensuremath{\mathbf{Q}}}
\newcommand{\bR}{\ensuremath{\mathbf{R}}}
\newcommand{\bS}{\ensuremath{\mathbf{S}}}
\newcommand{\bT}{\ensuremath{\mathbf{T}}}
\newcommand{\bU}{\ensuremath{\mathbf{U}}}
\newcommand{\bV}{\ensuremath{\mathbf{V}}}
\newcommand{\bW}{\ensuremath{\mathbf{W}}}
\newcommand{\bX}{\ensuremath{\mathbf{X}}}
\newcommand{\bY}{\ensuremath{\mathbf{Y}}}
\newcommand{\bZ}{\ensuremath{\mathbf{Z}}}

% fraktur letters
\newcommand{\fa}{\ensuremath{\mathfrak{a}}}
\newcommand{\fb}{\ensuremath{\mathfrak{b}}}
\newcommand{\fc}{\ensuremath{\mathfrak{c}}}
\newcommand{\fd}{\ensuremath{\mathfrak{d}}}
\newcommand{\fe}{\ensuremath{\mathfrak{e}}}
\newcommand{\ff}{\ensuremath{\mathfrak{f}}}
\newcommand{\fg}{\ensuremath{\mathfrak{g}}}
\newcommand{\fh}{\ensuremath{\mathfrak{h}}}
\newcommand{\fj}{\ensuremath{\mathfrak{j}}}
\newcommand{\fk}{\ensuremath{\mathfrak{k}}}
\newcommand{\fl}{\ensuremath{\mathfrak{l}}}
\newcommand{\fm}{\ensuremath{\mathfrak{m}}}
\newcommand{\fn}{\ensuremath{\mathfrak{n}}}
\newcommand{\fo}{\ensuremath{\mathfrak{o}}}
\newcommand{\fp}{\ensuremath{\mathfrak{p}}}
\newcommand{\fq}{\ensuremath{\mathfrak{q}}}
\newcommand{\fr}{\ensuremath{\mathfrak{r}}}
\newcommand{\fs}{\ensuremath{\mathfrak{s}}}
\newcommand{\ft}{\ensuremath{\mathfrak{t}}}
\newcommand{\fu}{\ensuremath{\mathfrak{u}}}
\newcommand{\fv}{\ensuremath{\mathfrak{v}}}
\newcommand{\fw}{\ensuremath{\mathfrak{w}}}
\newcommand{\fx}{\ensuremath{\mathfrak{x}}}
\newcommand{\fy}{\ensuremath{\mathfrak{y}}}
\newcommand{\fz}{\ensuremath{\mathfrak{z}}}

% fraktur letters
\newcommand{\fA}{\ensuremath{\mathfrak{A}}}
\newcommand{\fB}{\ensuremath{\mathfrak{B}}}
\newcommand{\fC}{\ensuremath{\mathfrak{C}}}

\mdef\fahat{\hat{\fa}}

% Underline letters
\newcommand{\uA}{\ensuremath{\underline{A}}}
\newcommand{\uB}{\ensuremath{\underline{B}}}
\newcommand{\uC}{\ensuremath{\underline{C}}}
\newcommand{\uD}{\ensuremath{\underline{D}}}
\newcommand{\uE}{\ensuremath{\underline{E}}}
\newcommand{\uF}{\ensuremath{\underline{F}}}
\newcommand{\uG}{\ensuremath{\underline{G}}}
\newcommand{\uH}{\ensuremath{\underline{H}}}
\newcommand{\uI}{\ensuremath{\underline{I}}}
\newcommand{\uJ}{\ensuremath{\underline{J}}}
\newcommand{\uK}{\ensuremath{\underline{K}}}
\newcommand{\uL}{\ensuremath{\underline{L}}}
\newcommand{\uM}{\ensuremath{\underline{M}}}
\newcommand{\uN}{\ensuremath{\underline{N}}}
\newcommand{\uO}{\ensuremath{\underline{O}}}
\newcommand{\uP}{\ensuremath{\underline{P}}}
\newcommand{\uQ}{\ensuremath{\underline{Q}}}
\newcommand{\uR}{\ensuremath{\underline{R}}}
\newcommand{\uS}{\ensuremath{\underline{S}}}
\newcommand{\uT}{\ensuremath{\underline{T}}}
\newcommand{\uU}{\ensuremath{\underline{U}}}
\newcommand{\uV}{\ensuremath{\underline{V}}}
\newcommand{\uW}{\ensuremath{\underline{W}}}
\newcommand{\uX}{\ensuremath{\underline{X}}}
\newcommand{\uY}{\ensuremath{\underline{Y}}}
\newcommand{\uZ}{\ensuremath{\underline{Z}}}

% bars
\newcommand{\Abar}{\ensuremath{\overline{A}}}
\newcommand{\Bbar}{\ensuremath{\overline{B}}}
\newcommand{\Cbar}{\ensuremath{\overline{C}}}
\newcommand{\Dbar}{\ensuremath{\overline{D}}}
\newcommand{\Ebar}{\ensuremath{\overline{E}}}
\newcommand{\Fbar}{\ensuremath{\overline{F}}}
\newcommand{\Gbar}{\ensuremath{\overline{G}}}
\newcommand{\Hbar}{\ensuremath{\overline{H}}}
\newcommand{\Ibar}{\ensuremath{\overline{I}}}
\newcommand{\Jbar}{\ensuremath{\overline{J}}}
\newcommand{\Kbar}{\ensuremath{\overline{K}}}
\newcommand{\Lbar}{\ensuremath{\overline{L}}}
\newcommand{\Mbar}{\ensuremath{\overline{M}}}
\newcommand{\Nbar}{\ensuremath{\overline{N}}}
\newcommand{\Obar}{\ensuremath{\overline{O}}}
\newcommand{\Pbar}{\ensuremath{\overline{P}}}
\newcommand{\Qbar}{\ensuremath{\overline{Q}}}
\newcommand{\Rbar}{\ensuremath{\overline{R}}}
\newcommand{\Sbar}{\ensuremath{\overline{S}}}
\newcommand{\Tbar}{\ensuremath{\overline{T}}}
\newcommand{\Ubar}{\ensuremath{\overline{U}}}
\newcommand{\Vbar}{\ensuremath{\overline{V}}}
\newcommand{\Wbar}{\ensuremath{\overline{W}}}
\newcommand{\Xbar}{\ensuremath{\overline{X}}}
\newcommand{\Ybar}{\ensuremath{\overline{Y}}}
\newcommand{\Zbar}{\ensuremath{\overline{Z}}}
\newcommand{\abar}{\ensuremath{\overline{a}}}
\newcommand{\bbar}{\ensuremath{\overline{b}}}
\newcommand{\cbar}{\ensuremath{\overline{c}}}
\newcommand{\dbar}{\ensuremath{\overline{d}}}
\newcommand{\ebar}{\ensuremath{\overline{e}}}
\newcommand{\fbar}{\ensuremath{\overline{f}}}
\newcommand{\gbar}{\ensuremath{\overline{g}}}
%\newcommand{\hbar}{\ensuremath{\overline{h}}} % whoops, \hbar means something else!
\newcommand{\ibar}{\ensuremath{\overline{\imath}}}
\newcommand{\jbar}{\ensuremath{\overline{\jmath}}}
\newcommand{\kbar}{\ensuremath{\overline{k}}}
\newcommand{\lbar}{\ensuremath{\overline{l}}}
\newcommand{\mbar}{\ensuremath{\overline{m}}}
\newcommand{\nbar}{\ensuremath{\overline{n}}}
%\newcommand{\obar}{\ensuremath{\overline{o}}}
\newcommand{\pbar}{\ensuremath{\overline{p}}}
\newcommand{\qbar}{\ensuremath{\overline{q}}}
\newcommand{\rbar}{\ensuremath{\overline{r}}}
\newcommand{\sbar}{\ensuremath{\overline{s}}}
\newcommand{\tbar}{\ensuremath{\overline{t}}}
\newcommand{\ubar}{\ensuremath{\overline{u}}}
\newcommand{\vbar}{\ensuremath{\overline{v}}}
\newcommand{\wbar}{\ensuremath{\overline{w}}}
\newcommand{\xbar}{\ensuremath{\overline{x}}}
\newcommand{\ybar}{\ensuremath{\overline{y}}}
\newcommand{\zbar}{\ensuremath{\overline{z}}}

% tildes
\newcommand{\Atil}{\ensuremath{\widetilde{A}}}
\newcommand{\Btil}{\ensuremath{\widetilde{B}}}
\newcommand{\Ctil}{\ensuremath{\widetilde{C}}}
\newcommand{\Dtil}{\ensuremath{\widetilde{D}}}
\newcommand{\Etil}{\ensuremath{\widetilde{E}}}
\newcommand{\Ftil}{\ensuremath{\widetilde{F}}}
\newcommand{\Gtil}{\ensuremath{\widetilde{G}}}
\newcommand{\Htil}{\ensuremath{\widetilde{H}}}
\newcommand{\Itil}{\ensuremath{\widetilde{I}}}
\newcommand{\Jtil}{\ensuremath{\widetilde{J}}}
\newcommand{\Ktil}{\ensuremath{\widetilde{K}}}
\newcommand{\Ltil}{\ensuremath{\widetilde{L}}}
\newcommand{\Mtil}{\ensuremath{\widetilde{M}}}
\newcommand{\Ntil}{\ensuremath{\widetilde{N}}}
\newcommand{\Otil}{\ensuremath{\widetilde{O}}}
\newcommand{\Ptil}{\ensuremath{\widetilde{P}}}
\newcommand{\Qtil}{\ensuremath{\widetilde{Q}}}
\newcommand{\Rtil}{\ensuremath{\widetilde{R}}}
\newcommand{\Stil}{\ensuremath{\widetilde{S}}}
\newcommand{\Ttil}{\ensuremath{\widetilde{T}}}
\newcommand{\Util}{\ensuremath{\widetilde{U}}}
\newcommand{\Vtil}{\ensuremath{\widetilde{V}}}
\newcommand{\Wtil}{\ensuremath{\widetilde{W}}}
\newcommand{\Xtil}{\ensuremath{\widetilde{X}}}
\newcommand{\Ytil}{\ensuremath{\widetilde{Y}}}
\newcommand{\Ztil}{\ensuremath{\widetilde{Z}}}
\newcommand{\atil}{\ensuremath{\widetilde{a}}}
\newcommand{\btil}{\ensuremath{\widetilde{b}}}
\newcommand{\ctil}{\ensuremath{\widetilde{c}}}
\newcommand{\dtil}{\ensuremath{\widetilde{d}}}
\newcommand{\etil}{\ensuremath{\widetilde{e}}}
\newcommand{\ftil}{\ensuremath{\widetilde{f}}}
\newcommand{\gtil}{\ensuremath{\widetilde{g}}}
\newcommand{\htil}{\ensuremath{\widetilde{h}}}
\newcommand{\itil}{\ensuremath{\widetilde{\imath}}}
\newcommand{\jtil}{\ensuremath{\widetilde{\jmath}}}
\newcommand{\ktil}{\ensuremath{\widetilde{k}}}
\newcommand{\ltil}{\ensuremath{\widetilde{l}}}
\newcommand{\mtil}{\ensuremath{\widetilde{m}}}
\newcommand{\ntil}{\ensuremath{\widetilde{n}}}
\newcommand{\otil}{\ensuremath{\widetilde{o}}}
\newcommand{\ptil}{\ensuremath{\widetilde{p}}}
\newcommand{\qtil}{\ensuremath{\widetilde{q}}}
\newcommand{\rtil}{\ensuremath{\widetilde{r}}}
\newcommand{\stil}{\ensuremath{\widetilde{s}}}
\newcommand{\ttil}{\ensuremath{\widetilde{t}}}
\newcommand{\util}{\ensuremath{\widetilde{u}}}
\newcommand{\vtil}{\ensuremath{\widetilde{v}}}
\newcommand{\wtil}{\ensuremath{\widetilde{w}}}
\newcommand{\xtil}{\ensuremath{\widetilde{x}}}
\newcommand{\ytil}{\ensuremath{\widetilde{y}}}
\newcommand{\ztil}{\ensuremath{\widetilde{z}}}

% Hats
\newcommand{\Ahat}{\ensuremath{\widehat{A}}}
\newcommand{\Bhat}{\ensuremath{\widehat{B}}}
\newcommand{\Chat}{\ensuremath{\widehat{C}}}
\newcommand{\Dhat}{\ensuremath{\widehat{D}}}
\newcommand{\Ehat}{\ensuremath{\widehat{E}}}
\newcommand{\Fhat}{\ensuremath{\widehat{F}}}
\newcommand{\Ghat}{\ensuremath{\widehat{G}}}
\newcommand{\Hhat}{\ensuremath{\widehat{H}}}
\newcommand{\Ihat}{\ensuremath{\widehat{I}}}
\newcommand{\Jhat}{\ensuremath{\widehat{J}}}
\newcommand{\Khat}{\ensuremath{\widehat{K}}}
\newcommand{\Lhat}{\ensuremath{\widehat{L}}}
\newcommand{\Mhat}{\ensuremath{\widehat{M}}}
\newcommand{\Nhat}{\ensuremath{\widehat{N}}}
\newcommand{\Ohat}{\ensuremath{\widehat{O}}}
\newcommand{\Phat}{\ensuremath{\widehat{P}}}
\newcommand{\Qhat}{\ensuremath{\widehat{Q}}}
\newcommand{\Rhat}{\ensuremath{\widehat{R}}}
\newcommand{\Shat}{\ensuremath{\widehat{S}}}
\newcommand{\That}{\ensuremath{\widehat{T}}}
\newcommand{\Uhat}{\ensuremath{\widehat{U}}}
\newcommand{\Vhat}{\ensuremath{\widehat{V}}}
\newcommand{\What}{\ensuremath{\widehat{W}}}
\newcommand{\Xhat}{\ensuremath{\widehat{X}}}
\newcommand{\Yhat}{\ensuremath{\widehat{Y}}}
\newcommand{\Zhat}{\ensuremath{\widehat{Z}}}
\newcommand{\ahat}{\ensuremath{\hat{a}}}
\newcommand{\bhat}{\ensuremath{\hat{b}}}
\newcommand{\chat}{\ensuremath{\hat{c}}}
\newcommand{\dhat}{\ensuremath{\hat{d}}}
\newcommand{\ehat}{\ensuremath{\hat{e}}}
\newcommand{\fhat}{\ensuremath{\hat{f}}}
\newcommand{\ghat}{\ensuremath{\hat{g}}}
\newcommand{\hhat}{\ensuremath{\hat{h}}}
\newcommand{\ihat}{\ensuremath{\hat{\imath}}}
\newcommand{\jhat}{\ensuremath{\hat{\jmath}}}
\newcommand{\khat}{\ensuremath{\hat{k}}}
\newcommand{\lhat}{\ensuremath{\hat{l}}}
\newcommand{\mhat}{\ensuremath{\hat{m}}}
\newcommand{\nhat}{\ensuremath{\hat{n}}}
\newcommand{\ohat}{\ensuremath{\hat{o}}}
\newcommand{\phat}{\ensuremath{\hat{p}}}
\newcommand{\qhat}{\ensuremath{\hat{q}}}
\newcommand{\rhat}{\ensuremath{\hat{r}}}
\newcommand{\shat}{\ensuremath{\hat{s}}}
\newcommand{\that}{\ensuremath{\hat{t}}}
\newcommand{\uhat}{\ensuremath{\hat{u}}}
\newcommand{\vhat}{\ensuremath{\hat{v}}}
\newcommand{\what}{\ensuremath{\hat{w}}}
\newcommand{\xhat}{\ensuremath{\hat{x}}}
\newcommand{\yhat}{\ensuremath{\hat{y}}}
\newcommand{\zhat}{\ensuremath{\hat{z}}}

%% FONTS AND DECORATION FOR GREEK LETTERS

%% the package `mathbbol' gives us blackboard bold greek and numbers,
%% but it does it by redefining \mathbb to use a different font, so that
%% all the other \mathbb letters look different too.  Here we import the
%% font with bb greek and numbers, but assign it a different name,
%% \mathbbb, so as not to replace the usual one.
\DeclareSymbolFont{bbold}{U}{bbold}{m}{n}
\DeclareSymbolFontAlphabet{\mathbbb}{bbold}
\newcommand{\bbDelta}{\ensuremath{\mathbbb{\Delta}}}
\newcommand{\bbone}{\ensuremath{\mathbbb{1}}}
\newcommand{\bbtwo}{\ensuremath{\mathbbb{2}}}
\newcommand{\bbthree}{\ensuremath{\mathbbb{3}}}

% greek with bars
\newcommand{\albar}{\ensuremath{\overline{\alpha}}}
\newcommand{\bebar}{\ensuremath{\overline{\beta}}}
\newcommand{\gmbar}{\ensuremath{\overline{\gamma}}}
\newcommand{\debar}{\ensuremath{\overline{\delta}}}
\newcommand{\phibar}{\ensuremath{\overline{\varphi}}}
\newcommand{\psibar}{\ensuremath{\overline{\psi}}}
\newcommand{\xibar}{\ensuremath{\overline{\xi}}}
\newcommand{\ombar}{\ensuremath{\overline{\omega}}}

% greek with hats
\newcommand{\alhat}{\ensuremath{\hat{\alpha}}}
\newcommand{\behat}{\ensuremath{\hat{\beta}}}
\newcommand{\gmhat}{\ensuremath{\hat{\gamma}}}
\newcommand{\dehat}{\ensuremath{\hat{\delta}}}

% greek with checks
\newcommand{\alchk}{\ensuremath{\check{\alpha}}}
\newcommand{\bechk}{\ensuremath{\check{\beta}}}
\newcommand{\gmchk}{\ensuremath{\check{\gamma}}}
\newcommand{\dechk}{\ensuremath{\check{\delta}}}

% greek with tildes
\newcommand{\altil}{\ensuremath{\widetilde{\alpha}}}
\newcommand{\betil}{\ensuremath{\widetilde{\beta}}}
\newcommand{\gmtil}{\ensuremath{\widetilde{\gamma}}}
\newcommand{\phitil}{\ensuremath{\widetilde{\varphi}}}
\newcommand{\psitil}{\ensuremath{\widetilde{\psi}}}
\newcommand{\xitil}{\ensuremath{\widetilde{\xi}}}
\newcommand{\omtil}{\ensuremath{\widetilde{\omega}}}

% MISCELLANEOUS SYMBOLS
\mdef\del{\partial}
\mdef\delbar{\overline{\partial}}
\let\sm\wedge
\newcommand{\dd}[1]{\ensuremath{\frac{\partial}{\partial {#1}}}}
\newcommand{\inv}{^{-1}}
\newcommand{\dual}{^{\vee}}
\mdef\hf{\textstyle\frac{1}{2}}
\mdef\thrd{\textstyle\frac{1}{3}}
\mdef\qtr{\textstyle\frac{1}{4}}
\let\meet\wedge
\let\join\vee
\let\dn\downarrow
\newcommand{\op}{^{\mathit{op}}}
\newcommand{\co}{^{\mathit{co}}}
\newcommand{\coop}{^{\mathit{coop}}}
\let\adj\dashv
\SelectTips{cm}{}
\newdir{ >}{{}*!/-10pt/@{>}}    % extra spacing for tail arrows in XYpic
\newcommand{\pushoutcorner}[1][dr]{\save*!/#1+1.2pc/#1:(1,-1)@^{|-}\restore}
\newcommand{\pullbackcorner}[1][dr]{\save*!/#1-1.2pc/#1:(-1,1)@^{|-}\restore}
\let\iso\cong
\let\eqv\simeq
\let\cng\equiv
\mdef\Id{\mathrm{Id}}
\mdef\id{\mathrm{id}}
\alwaysmath{ell}
\alwaysmath{infty}
\alwaysmath{odot}
\def\frc#1/#2.{\frac{#1}{#2}}   % \frc x^2+1 / x^2-1 .
\mdef\ten{\mathrel{\otimes}}
\mdef\bigten{\bigotimes}
\mdef\sqten{\mathrel{\boxtimes}}
\def\pow(#1,#2){\mathop{\pitchfork}(#1,#2)} % powers and
\def\cpw{\mathop{\odot}}                    % copowers
\newcommand{\mathid}{\mbox{id}}
\newcommand{\cat}[1]{\ensuremath{\mathbf{#1}}}


%% OPERATORS
\DeclareMathOperator\lan{Lan}
\DeclareMathOperator\ran{Ran}
\DeclareMathOperator\colim{colim}
\DeclareMathOperator\coeq{coeq}
\DeclareMathOperator\eq{eq}
\DeclareMathOperator\Tot{Tot}
\DeclareMathOperator\cosk{cosk}
\DeclareMathOperator\sk{sk}
\DeclareMathOperator\im{im}
\DeclareMathOperator\Spec{Spec}
\DeclareMathOperator\Ho{Ho}
\DeclareMathOperator\Aut{Aut}
\DeclareMathOperator\End{End}
\DeclareMathOperator\Hom{Hom}
\DeclareMathOperator\Map{Map}

%% TIKZ ARROWS AND HIGHER CELLS
\makeatletter
\def\slashedarrowfill@#1#2#3#4#5{%
  $\m@th\thickmuskip0mu\medmuskip\thickmuskip\thinmuskip\thickmuskip
   \relax#5#1\mkern-7mu%
   \cleaders\hbox{$#5\mkern-2mu#2\mkern-2mu$}\hfill
   \mathclap{#3}\mathclap{#2}%
   \cleaders\hbox{$#5\mkern-2mu#2\mkern-2mu$}\hfill
   \mkern-7mu#4$%
}

\def\Rightslashedarrowfill@{%
  \slashedarrowfill@\Relbar\Relbar\Mapstochar\Rightarrow}
\newcommand\xslashedRightarrow[2][]{%
  \ext@arrow 0055{\Rightslashedarrowfill@}{#1}{#2}}
\def\hTo{\xslashedRightarrow{}}
\def\hToo{\xslashedRightarrow{\quad}}
\let\xhTo\xslashedRightarrow

\pagestyle{empty}

\newcommand{\Rightthreecell}{\RRightarrow}
\newcommand{\Rtwocell}{\Rightarrow}

\tikzstyle{doubletick}=[-implies, double equal sign distance, postaction={decorate},decoration={markings,mark=at position .5 with {\draw[-] (0,-0.1) -- (0,0.1);}}]

\tikzstyle{darrow}=[-implies, double equal sign distance]

\tikzstyle{doubleeq}=[double equal sign distance]


%% ARROWS
% \to already exists
\newcommand{\too}[1][]{\ensuremath{\overset{#1}{\longrightarrow}}}
\newcommand{\ot}{\ensuremath{\leftarrow}}
\newcommand{\oot}[1][]{\ensuremath{\overset{#1}{\longleftarrow}}}
\let\toot\rightleftarrows
\let\otto\leftrightarrows
\let\Impl\Rightarrow
\let\imp\Rightarrow
\let\toto\rightrightarrows
\let\into\hookrightarrow
\let\xinto\xhookrightarrow
\mdef\we{\overset{\sim}{\longrightarrow}}
\mdef\leftwe{\overset{\sim}{\longleftarrow}}
\let\mono\rightarrowtail
\let\leftmono\leftarrowtail
\let\cof\rightarrowtail
\let\leftcof\leftarrowtail
\let\epi\twoheadrightarrow
\let\leftepi\twoheadleftarrow
\let\fib\twoheadrightarrow
\let\leftfib\twoheadleftarrow
\let\cohto\rightsquigarrow
\let\maps\colon
\newcommand{\spam}{\,:\!}       % \maps for left arrows

\newsavebox{\DDownarrowbox}
\savebox{\DDownarrowbox}{\tikz[scale=1.5]{\draw[-implies,double equal
sign distance] (0,.1) -- (0,-.1); \draw (0,.1) -- (0,-.1);}}
\newcommand{\DDownarrow}{\mathrel{\raisebox{-.2em}{\usebox{\DDownarrowbox}}}}

\newsavebox{\RRightarrowbox}
\savebox{\RRightarrowbox}{\tikz[scale=1.5]{\draw[-implies,double equal
sign distance] (-.1,0) -- (.1,0); \draw (-.1,0) -- (.1,0);}}
\newcommand{\RRightarrow}{\mathrel{\raisebox{.2em}{\usebox{\RRightarrowbox}}}}

%\newsavebox{\Rightslashedarrowbox}
%\savebox{\Rightslashedarrowbox}{\tikz[scale=1.5]{\draw[Rightslashedarrow{}] (-.1,0) -- (1,0);}}
%\newcommand{\Rightslashedarrow}{\mathrel{\raisebox{-.2em}%{\usebox{\Rightslashedarrowbox}}}}


%% EXTENSIBLE ARROWS
\let\xto\xrightarrow
\let\xot\xleftarrow
% See Voss' Mathmode.tex for instructions on how to create new
% extensible arrows.
\def\rightarrowtailfill@{\arrowfill@{\Yright\joinrel\relbar}\relbar\rightarrow}
\newcommand\xrightarrowtail[2][]{\ext@arrow 0055{\rightarrowtailfill@}{#1}{#2}}
\let\xmono\xrightarrowtail
\let\xcof\xrightarrowtail
\def\twoheadrightarrowfill@{\arrowfill@{\relbar\joinrel\relbar}\relbar\twoheadrightarrow}
\newcommand\xtwoheadrightarrow[2][]{\ext@arrow 0055{\twoheadrightarrowfill@}{#1}{#2}}
\let\xepi\xtwoheadrightarrow
\let\xfib\xtwoheadrightarrow
% Let's leave the left-going ones until I need them.

%% EXTENSIBLE SLASHED ARROWS
% Making extensible slashed arrows, by modifying the underlying AMS code.
% Arguments are:
% 1 = arrowhead on the left (\relbar or \Relbar if none)
% 2 = fill character (usually \relbar or \Relbar)
% 3 = slash character (such as \mapstochar or \Mapstochar)
% 4 = arrowhead on the left (\relbar or \Relbar if none)
% 5 = display mode (\displaystyle etc)
\def\slashedarrowfill@#1#2#3#4#5{%
  $\m@th\thickmuskip0mu\medmuskip\thickmuskip\thinmuskip\thickmuskip
   \relax#5#1\mkern-7mu%
   \cleaders\hbox{$#5\mkern-2mu#2\mkern-2mu$}\hfill
   \mathclap{#3}\mathclap{#2}%
   \cleaders\hbox{$#5\mkern-2mu#2\mkern-2mu$}\hfill
   \mkern-7mu#4$%
}
% Here's the idea: \<slashed>arrowfill@ should be a box containing
% some stretchable space that is the "middle of the arrow".  This
% space is created as a "leader" using \cleader<thing>\hfill, which
% fills an \hfill of space with copies of <thing>.  Here instead of
% just one \cleader, we use two, with the slash in between (and an
% extra copy of the filler, to avoid extra space around the slash).
\def\rightslashedarrowfill@{%
  \slashedarrowfill@\relbar\relbar\mapstochar\rightarrow}
\newcommand\xslashedrightarrow[2][]{%
  \ext@arrow 0055{\rightslashedarrowfill@}{#1}{#2}}
\mdef\hto{\xslashedrightarrow{}}
\mdef\htoo{\xslashedrightarrow{\quad}}
\let\xhto\xslashedrightarrow

%% To get a slashed arrow in XYpic, do
% \[\xymatrix{A \ar[r]|-@{|} & B}\]

% ISOMORPHISMS
\def\xiso#1{\mathrel{\mathrlap{\smash{\xto[\smash{\raisebox{1.3mm}{$\scriptstyle\sim$}}]{#1}}}\hphantom{\xto{#1}}}}
\def\toiso{\xto{\smash{\raisebox{-.5mm}{$\scriptstyle\sim$}}}}

% SHADOWS
\def\shvar#1#2{{\ensuremath{%
  \hspace{1mm}\makebox[-1mm]{$#1\langle$}\makebox[0mm]{$#1\langle$}\hspace{1mm}%
  {#2}%
  \makebox[1mm]{$#1\rangle$}\makebox[0mm]{$#1\rangle$}%
}}}
\def\sh{\shvar{}}
\def\scriptsh{\shvar{\scriptstyle}}
\def\bigsh{\shvar{\big}}
\def\Bigsh{\shvar{\Big}}
\def\biggsh{\shvar{\bigg}}
\def\Biggsh{\shvar{\Bigg}}

%HIGHER CELLS



% THEOREM-TYPE ENVIRONMENTS, hacked to
%% (a) number all with the same numbers, and
%% (b) have the right names for autoref
\def\defthm#1#2{%
  \newtheorem{#1}{#2}[section]%
  \expandafter\def\csname #1autorefname\endcsname{#2}%
  \expandafter\let\csname c@#1\endcsname\c@thm}
\newtheorem{thm}{Theorem}[section]
\newcommand{\thmautorefname}{Theorem}
\defthm{cor}{Corollary}
\defthm{prop}{Proposition}
\defthm{lem}{Lemma}
\defthm{sch}{Scholium}
\defthm{assume}{Assumption}
\defthm{claim}{Claim}
\defthm{conj}{Conjecture}
\defthm{hyp}{Hypothesis}
\defthm{fact}{Fact}
\theoremstyle{definition}
\defthm{defn}{Definition}
\defthm{notn}{Notation}
\theoremstyle{remark}
\defthm{rmk}{Remark}
\defthm{eg}{Example}
\defthm{egs}{Examples}
\defthm{ex}{Exercise}
\defthm{ceg}{Counterexample}

% How to get QED symbols inside equations at the end of the statements
% of theorems.  AMS LaTeX knows how to do this inside equations at the
% end of *proofs* with \qedhere, and at the end of the statement of a
% theorem with \qed (meaning no proof will be given), but it can't
% seem to combine the two.  Use this instead.
\def\thmqedhere{\expandafter\csname\csname @currenvir\endcsname @qed\endcsname}

% Number numbered lists as (i), (ii), ...
\renewcommand{\theenumi}{(\roman{enumi})}
\renewcommand{\labelenumi}{\theenumi}

%% Labeling that keeps track of theorem-type names.  Superseded by
%% autoref from hyperref, as above, but we keep this in case we are
%% using a journal style file that is incompatible with hyperref.
% 
% \ifx\SK@label\undefined\let\SK@label\label\fi
% \let\your@thm\@thm
% \def\@thm#1#2#3{\gdef\currthmtype{#3}\your@thm{#1}{#2}{#3}}
% \def\xlabel#1{{\let\your@currentlabel\@currentlabel\def\@currentlabel
% {\currthmtype~\your@currentlabel}
% \SK@label{#1@}}\label{#1}}
% \def\xref#1{\ref{#1@}}

% Also number formulas with the theorem counter
\let\c@equation\c@thm
\numberwithin{equation}{section}

% Only show numbers for equations that are actually referenced (or
% whose tags are specified manually) - uses mathtools.
\mathtoolsset{showonlyrefs,showmanualtags}

%% Fix enumerate spacing with paralist.  This has two parts:
%%   1. enable mixing of "old spacing" lists with those adjusted by paralist
%%   2. allow us to specify a number based on which to adjust the spacing
%% For the first, use \killspacingtrue when you want the spacing
%% adjusted, then \killspacingfalse to turn adjustment off.  For the
%% second, use \maxenum=14 to set the widest number you want the
%% spacing to be calculated with.
\newlength\oldleftmargini       % save old spacing
\newlength\oldleftmarginii
\newlength\oldleftmarginiii
\newlength\oldleftmarginiv
\newlength\oldleftmarginv
\newlength\oldleftmarginvi
\newcount\maxenum
\maxenum=7
\newif\ifkillspacing
\def\@adjust@enum@labelwidth{%
  \advance\@listdepth by 1\relax
  \ifkillspacing                % do the paralist thing
    \csname c@\@enumctr\endcsname\maxenum
    \settowidth{\@tempdima}{%
      \csname label\@enumctr\endcsname\hspace{\labelsep}}%
    \csname leftmargin\romannumeral\@listdepth\endcsname
      \@tempdima
  \else                         % otherwise, reset it
    \csname fixspacing\romannumeral\@listdepth\endcsname
  \fi
  \advance\@listdepth by -1\relax}
% these commands, one for each enum level (I couldn't get a generic
% one to work), test whether oldleftmargin has been set yet, and if
% not, set it from leftmargin; otherwise, they reset leftmargin to
% it.  Just setting oldleftmargin to leftmargin in the preamble
% doesn't seem to work.
\def\fixspacingi{\ifnum\oldleftmargini=0\setlength\oldleftmargini\leftmargini\else\setlength\leftmargini\oldleftmargini\fi}
\def\fixspacingii{\ifnum\oldleftmarginii=0\setlength\oldleftmarginii\leftmarginii\else\setlength\leftmarginii\oldleftmarginii\fi}
\def\fixspacingiii{\ifnum\oldleftmarginiii=0\setlength\oldleftmarginiii\leftmarginiii\else\setlength\leftmarginiii\oldleftmarginiii\fi}
\def\fixspacingiv{\ifnum\oldleftmarginiv=0\setlength\oldleftmarginiv\leftmarginiv\else\setlength\leftmarginiv\oldleftmarginiv\fi}
\def\fixspacingv{\ifnum\oldleftmarginv=0\setlength\oldleftmarginv\leftmarginv\else\setlength\leftmarginv\oldleftmarginv\fi}
\def\fixspacingvi{\ifnum\oldleftmarginvi=0\setlength\oldleftmarginvi\leftmarginvi\else\setlength\leftmarginvi\oldleftmarginvi\fi}

%% Fix paralist references, so that we can refer to (1) instead of
%% just 1.
\def\pl@label#1#2{%
  \edef\pl@the{\noexpand#1{\@enumctr}}%
  \pl@lab\expandafter{\the\pl@lab\csname yourthe\@enumctr\endcsname}%
  \advance\@tempcnta1
  \pl@loop}
\def\@enumlabel@#1[#2]{%
  \@plmylabeltrue
  \@tempcnta0
  \pl@lab{}%
  \let\pl@the\pl@qmark
  \expandafter\pl@loop\@gobble#2\@@@
  \ifnum\@tempcnta=1\else
    \PackageWarning{paralist}{Incorrect label; no or multiple
      counters.\MessageBreak The label is: \@gobble#2}%
  \fi
  \expandafter\edef\csname label\@enumctr\endcsname{\the\pl@lab}%
  \expandafter\edef\csname the\@enumctr\endcsname{\the\pl@lab}%
  \expandafter\let\csname yourthe\@enumctr\endcsname\pl@the
  #1}


% GREEK LETTERS, ETC.
\alwaysmath{alpha}
\alwaysmath{beta}
\alwaysmath{gamma}
\alwaysmath{Gamma}
\alwaysmath{delta}
\alwaysmath{Delta}
\alwaysmath{epsilon}
\mdef\ep{\varepsilon}
\alwaysmath{zeta}
\alwaysmath{eta}
\alwaysmath{theta}
\alwaysmath{Theta}
\alwaysmath{iota}
\alwaysmath{kappa}
\alwaysmath{lambda}
\alwaysmath{Lambda}
\alwaysmath{mu}
\alwaysmath{nu}
\alwaysmath{xi}
\alwaysmath{pi}
\alwaysmath{rho}
\alwaysmath{sigma}
\alwaysmath{Sigma}
\alwaysmath{tau}
\alwaysmath{upsilon}
\alwaysmath{Upsilon}
\alwaysmath{phi}
\alwaysmath{Pi}
\alwaysmath{Phi}
\mdef\ph{\varphi}
\alwaysmath{chi}
\alwaysmath{psi}
\alwaysmath{Psi}
\alwaysmath{omega}
\alwaysmath{Omega}
\let\al\alpha
\let\be\beta
\let\gm\gamma
\let\Gm\Gamma
\let\de\delta
\let\De\Delta
\let\si\sigma
\let\Si\Sigma
\let\om\omega
\let\ka\kappa
\let\la\lambda
\let\La\Lambda
\let\ze\zeta
\let\th\theta
\let\Th\Theta
\let\vth\vartheta

\makeatother

% Tikz styles
\tikzstyle{tickarrow}=[->,postaction={decorate},decoration={markings,mark=at position .5 with {\draw[-] (0,-0.1) -- (0,0.1);}},line width=0.50]

% Local Variables:
% mode: latex
% TeX-master: ""
% End:

\begin{document}

{\small
\begin{equation*}
\begin{tikzpicture}[xscale=2.25, yscale=1.5]
%%%% Row A
\node (A1) at (-.5,7){$\substack{\tens (\tens \times \transid)\\ (\transid \times \transid \times \transid)}$};
\node (A2) at (0,9){$\substack{\tens (\tens \times \transid)\\ ([\tens(\transid \times I] \times \transid \times \transid )}$};
\node (A3) at (1,9){$\substack{\tens (\tens \times \transid) \\  (\tens \times \transid \times \transid) \\ (\transid \times I \times \transid \times \transid)}$};
\node (A5) at (2.5,10){$\substack{\tens (\tens \times \transid) \\ ( \transid \times \tens \times \transid) \\ (\transid \times I \times \transid \times \transid)}$};
\node (A6) at (3.5,10){$\substack{\tens (\tens \times \transid) \\  (\transid \times [\tens (I \times \transid)] \times \transid)}$};
\node (A7) at (5,9.5){$\substack{\tens (\tens \times \transid) \\ (\transid \times \transid \times \transid)}$};
\node (A75) at (6,8.5){$\substack{\tens (\transid \times \tens) \\ (\transid \times \transid \times \transid)}$};
\node (A8) at (6,5){$\substack{\tens (\transid \times \tens)}$};
%%%
\draw[doubleloose] (A1) to node[above, xshift=-20]{$\substack{\looseid \looseid \\ (r^{-1} \times \looseid \times \looseid) }$} (A2);
\draw[doubletighteq] (A2) to  (A3);
\draw[doubleloose] (A3) to node[above, yshift=5pt]{$\substack{\looseid (\alpha \times \looseid) \looseid }$} (A5);
\draw[doubletighteq] (A5) to (A6);
\draw[doubleloose] (A6) to node[above, xshift=8pt]{$\substack{\looseid \looseid \\ (\looseid \times l \times \looseid) }$} (A7);
\draw[doubleloose] (A7) to node[above]{$\substack{\alpha}$} (A75);
\draw[doubletighteq] (A75) to (A8);
%%%% Row B
\node (B6) at (3,8){$\substack{\tens (\transid \times \tens) \\ ( \transid \times \tens \times \transid) \\ (\transid \times I \times \transid \times \transid)}$};
\node (B7) at (4,8){$\substack{\tens (\transid \times \tens) \\ ( \transid \times [\tens (I \times \transid)] \times \transid)}$};
%%%
\draw[doubleloose] (A5) to node[right]{$\substack{\alpha \looseid \looseid  }$} (B6);
\draw[doubletighteq] (B6) to  (B7); 
\draw[doubleloose] (B7) to node[below, xshift=10pt, yshift=-5pt]{$\substack{\looseid \looseid (\looseid \times l \times \looseid) }$} (A75); 
%%%% Row C
\node (C7) at (3,6){$\substack{\tens (\transid \times \tens) \\ ( \transid \times \transid \times \tens) \\ (\transid \times I \times \transid  \times \transid)}$};
\node (C75) at (4,5){$\substack{\tens (\transid \times [\tens (I \times \transid)]) \\ (\transid \times \transid \times \tens) }$};
\node (C8) at (5,5){$\substack{\tens (\transid \times \transid) \\ (\transid \times \transid \times \tens) }$};
%%%
\draw[doubleloose] (B6) to node[right]{$\substack{\looseid (\looseid \times \alpha) \\ \looseid}$} (C7); 
\draw[doubletighteq] (C7) to (C75);
\draw[doubleloose] (C75) to node[above]{$\substack{ \looseid (\looseid \times l ) \\ \looseid}$} (C8); 
\draw[doubletighteq] (C8) to  (A8); 
%%%% Row D
\node (D4) at (0,4){$\substack{\tens ([\tens  (\transid \times I)] \times \tens) \\ ( \transid \times \transid \times \transid) }$};
\node (D5) at (1,5){$\substack{\tens (\transid \times \tens) \\ ( \tens \times \transid \times \transid) \\ (\transid \times I \times \transid \times \transid) }$};
\node (D6) at (2,5){$\substack{\tens (\tens \times \transid) \\ ( \transid \times \transid \times \tens) \\ (\transid \times I \times \transid \times \transid)}$};
%%%
\draw[doubleloose] (A3) to node[left]{$\substack{\alpha \looseid \looseid }$} (D5);
\draw[doubletighteq] (D4) to (D5);
\draw[doubletighteq] (D5) to (D6);
\draw[doubleloose] (D6) to node[above,xshift=-10pt]{$\substack{\alpha \looseid    \looseid}$} (C7); 
%%%% Row E
\node (E1) at (-.5,3){$\substack{\tens (\transid \times \tens)\\ (\transid \times \transid \times \transid)}$};
\node (E3) at (0,2){$\substack{\tens (\transid \times \tens)}$};
%%%
\draw[doubleloose] (A1) to node[left]{$\substack{\alpha}$} (E1);
\draw[doubleloose] (E1) to node[right]{$\substack{ \looseid ( r^{-1} \times \looseid ) \looseid}$}  (D4);
\draw[doubletighteq] (E1) to (E3);
\draw[doubletighteq] (E3) to[out= 0, in=270] (A8); 
%%%% 3-cells
\node at (0.5,6.5) {$\substack{\DDownarrow \iso } $};
\node at (5.5,7.5) {$\substack{\DDownarrow \iso } $};
\node at (4.5,6.7) {$\substack{\DDownarrow \tightid \lambda}$};
\node at (4,5.8) {$\substack{\DDownarrow \iso } $};
\node at (2.5,8.5){$\substack{\DDownarrow \iso } $};
\node at (2,7){$\substack{\DDownarrow \pi \tightid}$};
\node at (1.5,6){$\substack{\DDownarrow \iso } $};
\node at (4,8.5){$\substack{\DDownarrow \iso } $};
\node at (2.5,4.5){$\substack{\DDownarrow  \iso}$};
\node at (3.5,3){$\substack{\DDownarrow  \looseid (\looseid \times \mu)}$};
%%%%%
\draw[doubleloose] (A3) to[out= -35, in=115] node[right]{$\substack{S(\pi) \looseid}$} (C7); 
\draw[doubleloose] (A3) to[out= -75, in=180] node[left]{$\substack{T(\pi) \\ \looseid}$} (C7);
%%%%%
\draw[doubleloose] (B6) to[out= -25, in=115] node[above, xshift=10pt]{$\substack{\looseid S(\lambda)}$} (A8); 
\draw[doubleloose] (B6) to[out= -55, in=160] node[below, xshift=-10pt]{$\substack{\looseid T(\lambda) }$} (A8);
%%%%%
\draw[doubleloose] (E1) to[out= 25, in=225] node[above]{$\substack{\looseid (\looseid \times S(\mu))}$} (A8); 
\end{tikzpicture}
\end{equation*}
\begin{equation}\label{eq:monobjeq2}
=
\end{equation}
\begin{equation*}
\begin{tikzpicture}[xscale=3, yscale=1]
%%%% Row A
\node (A1) at (-.5,7){$\substack{\tens (\tens \times \transid)\\ (\transid \times \transid \times \transid)}$};
\node (A2) at (0,9){$\substack{\tens (\tens \times \transid)\\ ([\tens(\transid \times I] \times \transid \times \transid )}$};
\node (A3) at (1,10){$\substack{\tens (\tens \times \transid) \\  (\tens \times \transid \times \transid) \\ (\transid \times I \times \transid \times \transid)}$};
\node (A5) at (2.5,10){$\substack{\tens (\tens \times \transid) \\ ( \transid \times \tens \times \transid) \\ (\transid \times I \times \transid \times \transid)}$};
\node (A6) at (3.5,9){$\substack{\tens (\tens \times \transid) \\  (\transid \times [\tens (I \times \transid)] \times \transid)}$};
\node (A7) at (4,7){$\substack{\tens (\tens \times \transid) \\ (\transid \times \transid \times \transid)}$};
\node (A75) at (4,6){$\substack{\tens (\transid \times \tens) \\ (\transid \times \transid \times \transid)}$};
\node (A8) at (3.5,5){$\substack{\tens (\transid \times \tens)}$};
%%%
\draw[doubleloose] (A1) to node[above, xshift=-20]{$\substack{\looseid \looseid \\ (r^{-1} \times \looseid \times \looseid) }$} (A2);
\draw[doubletighteq] (A2) to  (A3);
\draw[doubleloose] (A3) to node[above, yshift=5pt]{$\substack{\looseid (\alpha \times \looseid) \looseid }$} (A5);
\draw[doubletighteq] (A5) to (A6);
\draw[doubleloose] (A6) to node[right, xshift=8pt]{$\substack{\looseid \looseid \\ (\looseid \times l \times \looseid) }$} (A7);
\draw[doubleloose] (A7) to node[right]{$\substack{\alpha}$} (A75);
\draw[doubletighteq] (A75) to (A8);
%%%% Row E
\node (E1) at (-.5,6){$\substack{\tens (\transid \times \tens)\\ (\transid \times \transid \times \transid)}$};
\node (E3) at (0,5){$\substack{\tens (\transid \times \tens)}$};
%%%
\draw[doubleloose] (A1) to node[left]{$\substack{\alpha}$} (E1);
\draw[doubletighteq] (E1) to (E3);
\draw[doubletighteq] (E3) to (A8); 
%%%% 3-cells
\node at (1.75,9){$\substack{\DDownarrow  \iso}$};
\node at (1.75,7.5){$\substack{\DDownarrow  \tightid (\mu \times \tightid)}$};
\node at (1.75,6){$\substack{\DDownarrow  \iso}$};
%%%%%
\draw[doubleloose] (A1) to node[above]{$\substack{\looseid (\looseid \times \looseid)}$} (A7); 
\draw[doubleloose] (A1) to[out= 55, in=125] node[above]{$\substack{\looseid (S(\mu) \times \looseid)}$} (A7); 
\end{tikzpicture}
\end{equation*}}
\end{document}  \newpage
%
\documentclass[12pt]{ociamthesis}
\usepackage{tikz}
\usepackage{amsmath}
\usepackage{rotating}

\usepackage{amssymb,amsmath,stmaryrd,txfonts,mathrsfs,amsthm}
\usepackage[all,2cell]{xy}
\usepackage[neveradjust]{paralist}
\usepackage{hyperref}
\usepackage{mathtools}
\usepackage{tikz}
\usetikzlibrary{trees}
\usetikzlibrary{topaths}
\usetikzlibrary{decorations}
\usetikzlibrary{decorations.pathreplacing}
\usetikzlibrary{decorations.pathmorphing}
\usetikzlibrary{decorations.markings}
\usetikzlibrary{matrix,backgrounds,folding}
\usetikzlibrary{chains,scopes,positioning,fit}
\usetikzlibrary{arrows,shadows}
\usetikzlibrary{calc} 
\usetikzlibrary{chains}
\usetikzlibrary{shapes,shapes.geometric,shapes.misc}
\usepackage{smbicat}


\makeatletter
\let\ea\expandafter

%% Defining commands that are always in math mode.
\def\mdef#1#2{\ea\ea\ea\gdef\ea\ea\noexpand#1\ea{\ea\ensuremath\ea{#2}}}
\def\alwaysmath#1{\ea\ea\ea\global\ea\ea\ea\let\ea\ea\csname your@#1\endcsname\csname #1\endcsname
  \ea\def\csname #1\endcsname{\ensuremath{\csname your@#1\endcsname}}}

% Script letters
\newcommand{\sA}{\ensuremath{\mathscr{A}}}
\newcommand{\sB}{\ensuremath{\mathscr{B}}}
\newcommand{\sC}{\ensuremath{\mathscr{C}}}
\newcommand{\sD}{\ensuremath{\mathscr{D}}}
\newcommand{\sE}{\ensuremath{\mathscr{E}}}
\newcommand{\sF}{\ensuremath{\mathscr{F}}}
\newcommand{\sG}{\ensuremath{\mathscr{G}}}
\newcommand{\sH}{\ensuremath{\mathscr{H}}}
\newcommand{\sI}{\ensuremath{\mathscr{I}}}
\newcommand{\sJ}{\ensuremath{\mathscr{J}}}
\newcommand{\sK}{\ensuremath{\mathscr{K}}}
\newcommand{\sL}{\ensuremath{\mathscr{L}}}
\newcommand{\sM}{\ensuremath{\mathscr{M}}}
\newcommand{\sN}{\ensuremath{\mathscr{N}}}
\newcommand{\sO}{\ensuremath{\mathscr{O}}}
\newcommand{\sP}{\ensuremath{\mathscr{P}}}
\newcommand{\sQ}{\ensuremath{\mathscr{Q}}}
\newcommand{\sR}{\ensuremath{\mathscr{R}}}
\newcommand{\sS}{\ensuremath{\mathscr{S}}}
\newcommand{\sT}{\ensuremath{\mathscr{T}}}
\newcommand{\sU}{\ensuremath{\mathscr{U}}}
\newcommand{\sV}{\ensuremath{\mathscr{V}}}
\newcommand{\sW}{\ensuremath{\mathscr{W}}}
\newcommand{\sX}{\ensuremath{\mathscr{X}}}
\newcommand{\sY}{\ensuremath{\mathscr{Y}}}
\newcommand{\sZ}{\ensuremath{\mathscr{Z}}}

% Calligraphic letters
\newcommand{\cA}{\ensuremath{\mathcal{A}}}
\newcommand{\cB}{\ensuremath{\mathcal{B}}}
\newcommand{\cC}{\ensuremath{\mathcal{C}}}
\newcommand{\cD}{\ensuremath{\mathcal{D}}}
\newcommand{\cE}{\ensuremath{\mathcal{E}}}
\newcommand{\cF}{\ensuremath{\mathcal{F}}}
\newcommand{\cG}{\ensuremath{\mathcal{G}}}
\newcommand{\cH}{\ensuremath{\mathcal{H}}}
\newcommand{\cI}{\ensuremath{\mathcal{I}}}
\newcommand{\cJ}{\ensuremath{\mathcal{J}}}
\newcommand{\cK}{\ensuremath{\mathcal{K}}}
\newcommand{\cL}{\ensuremath{\mathcal{L}}}
\newcommand{\cM}{\ensuremath{\mathcal{M}}}
\newcommand{\cN}{\ensuremath{\mathcal{N}}}
\newcommand{\cO}{\ensuremath{\mathcal{O}}}
\newcommand{\cP}{\ensuremath{\mathcal{P}}}
\newcommand{\cQ}{\ensuremath{\mathcal{Q}}}
\newcommand{\cR}{\ensuremath{\mathcal{R}}}
\newcommand{\cS}{\ensuremath{\mathcal{S}}}
\newcommand{\cT}{\ensuremath{\mathcal{T}}}
\newcommand{\cU}{\ensuremath{\mathcal{U}}}
\newcommand{\cV}{\ensuremath{\mathcal{V}}}
\newcommand{\cW}{\ensuremath{\mathcal{W}}}
\newcommand{\cX}{\ensuremath{\mathcal{X}}}
\newcommand{\cY}{\ensuremath{\mathcal{Y}}}
\newcommand{\cZ}{\ensuremath{\mathcal{Z}}}

% blackboard bold letters
\newcommand{\lA}{\ensuremath{\mathbb{A}}}
\newcommand{\lB}{\ensuremath{\mathbb{B}}}
\newcommand{\lC}{\ensuremath{\mathbb{C}}}
\newcommand{\lD}{\ensuremath{\mathbb{D}}}
\newcommand{\lE}{\ensuremath{\mathbb{E}}}
\newcommand{\lF}{\ensuremath{\mathbb{F}}}
\newcommand{\lG}{\ensuremath{\mathbb{G}}}
\newcommand{\lH}{\ensuremath{\mathbb{H}}}
\newcommand{\lI}{\ensuremath{\mathbb{I}}}
\newcommand{\lJ}{\ensuremath{\mathbb{J}}}
\newcommand{\lK}{\ensuremath{\mathbb{K}}}
\newcommand{\lL}{\ensuremath{\mathbb{L}}}
\newcommand{\lM}{\ensuremath{\mathbb{M}}}
\newcommand{\lN}{\ensuremath{\mathbb{N}}}
\newcommand{\lO}{\ensuremath{\mathbb{O}}}
\newcommand{\lP}{\ensuremath{\mathbb{P}}}
\newcommand{\lQ}{\ensuremath{\mathbb{Q}}}
\newcommand{\lR}{\ensuremath{\mathbb{R}}}
\newcommand{\lS}{\ensuremath{\mathbb{S}}}
\newcommand{\lT}{\ensuremath{\mathbb{T}}}
\newcommand{\lU}{\ensuremath{\mathbb{U}}}
\newcommand{\lV}{\ensuremath{\mathbb{V}}}
\newcommand{\lW}{\ensuremath{\mathbb{W}}}
\newcommand{\lX}{\ensuremath{\mathbb{X}}}
\newcommand{\lY}{\ensuremath{\mathbb{Y}}}
\newcommand{\lZ}{\ensuremath{\mathbb{Z}}}

% bold letters
\newcommand{\bA}{\ensuremath{\mathbf{A}}}
\newcommand{\bB}{\ensuremath{\mathbf{B}}}
\newcommand{\bC}{\ensuremath{\mathbf{C}}}
\newcommand{\bD}{\ensuremath{\mathbf{D}}}
\newcommand{\bE}{\ensuremath{\mathbf{E}}}
\newcommand{\bF}{\ensuremath{\mathbf{F}}}
\newcommand{\bG}{\ensuremath{\mathbf{G}}}
\newcommand{\bH}{\ensuremath{\mathbf{H}}}
\newcommand{\bI}{\ensuremath{\mathbf{I}}}
\newcommand{\bJ}{\ensuremath{\mathbf{J}}}
\newcommand{\bK}{\ensuremath{\mathbf{K}}}
\newcommand{\bL}{\ensuremath{\mathbf{L}}}
\newcommand{\bM}{\ensuremath{\mathbf{M}}}
\newcommand{\bN}{\ensuremath{\mathbf{N}}}
\newcommand{\bO}{\ensuremath{\mathbf{O}}}
\newcommand{\bP}{\ensuremath{\mathbf{P}}}
\newcommand{\bQ}{\ensuremath{\mathbf{Q}}}
\newcommand{\bR}{\ensuremath{\mathbf{R}}}
\newcommand{\bS}{\ensuremath{\mathbf{S}}}
\newcommand{\bT}{\ensuremath{\mathbf{T}}}
\newcommand{\bU}{\ensuremath{\mathbf{U}}}
\newcommand{\bV}{\ensuremath{\mathbf{V}}}
\newcommand{\bW}{\ensuremath{\mathbf{W}}}
\newcommand{\bX}{\ensuremath{\mathbf{X}}}
\newcommand{\bY}{\ensuremath{\mathbf{Y}}}
\newcommand{\bZ}{\ensuremath{\mathbf{Z}}}

% fraktur letters
\newcommand{\fa}{\ensuremath{\mathfrak{a}}}
\newcommand{\fb}{\ensuremath{\mathfrak{b}}}
\newcommand{\fc}{\ensuremath{\mathfrak{c}}}
\newcommand{\fd}{\ensuremath{\mathfrak{d}}}
\newcommand{\fe}{\ensuremath{\mathfrak{e}}}
\newcommand{\ff}{\ensuremath{\mathfrak{f}}}
\newcommand{\fg}{\ensuremath{\mathfrak{g}}}
\newcommand{\fh}{\ensuremath{\mathfrak{h}}}
\newcommand{\fj}{\ensuremath{\mathfrak{j}}}
\newcommand{\fk}{\ensuremath{\mathfrak{k}}}
\newcommand{\fl}{\ensuremath{\mathfrak{l}}}
\newcommand{\fm}{\ensuremath{\mathfrak{m}}}
\newcommand{\fn}{\ensuremath{\mathfrak{n}}}
\newcommand{\fo}{\ensuremath{\mathfrak{o}}}
\newcommand{\fp}{\ensuremath{\mathfrak{p}}}
\newcommand{\fq}{\ensuremath{\mathfrak{q}}}
\newcommand{\fr}{\ensuremath{\mathfrak{r}}}
\newcommand{\fs}{\ensuremath{\mathfrak{s}}}
\newcommand{\ft}{\ensuremath{\mathfrak{t}}}
\newcommand{\fu}{\ensuremath{\mathfrak{u}}}
\newcommand{\fv}{\ensuremath{\mathfrak{v}}}
\newcommand{\fw}{\ensuremath{\mathfrak{w}}}
\newcommand{\fx}{\ensuremath{\mathfrak{x}}}
\newcommand{\fy}{\ensuremath{\mathfrak{y}}}
\newcommand{\fz}{\ensuremath{\mathfrak{z}}}

% fraktur letters
\newcommand{\fA}{\ensuremath{\mathfrak{A}}}
\newcommand{\fB}{\ensuremath{\mathfrak{B}}}
\newcommand{\fC}{\ensuremath{\mathfrak{C}}}

\mdef\fahat{\hat{\fa}}

% Underline letters
\newcommand{\uA}{\ensuremath{\underline{A}}}
\newcommand{\uB}{\ensuremath{\underline{B}}}
\newcommand{\uC}{\ensuremath{\underline{C}}}
\newcommand{\uD}{\ensuremath{\underline{D}}}
\newcommand{\uE}{\ensuremath{\underline{E}}}
\newcommand{\uF}{\ensuremath{\underline{F}}}
\newcommand{\uG}{\ensuremath{\underline{G}}}
\newcommand{\uH}{\ensuremath{\underline{H}}}
\newcommand{\uI}{\ensuremath{\underline{I}}}
\newcommand{\uJ}{\ensuremath{\underline{J}}}
\newcommand{\uK}{\ensuremath{\underline{K}}}
\newcommand{\uL}{\ensuremath{\underline{L}}}
\newcommand{\uM}{\ensuremath{\underline{M}}}
\newcommand{\uN}{\ensuremath{\underline{N}}}
\newcommand{\uO}{\ensuremath{\underline{O}}}
\newcommand{\uP}{\ensuremath{\underline{P}}}
\newcommand{\uQ}{\ensuremath{\underline{Q}}}
\newcommand{\uR}{\ensuremath{\underline{R}}}
\newcommand{\uS}{\ensuremath{\underline{S}}}
\newcommand{\uT}{\ensuremath{\underline{T}}}
\newcommand{\uU}{\ensuremath{\underline{U}}}
\newcommand{\uV}{\ensuremath{\underline{V}}}
\newcommand{\uW}{\ensuremath{\underline{W}}}
\newcommand{\uX}{\ensuremath{\underline{X}}}
\newcommand{\uY}{\ensuremath{\underline{Y}}}
\newcommand{\uZ}{\ensuremath{\underline{Z}}}

% bars
\newcommand{\Abar}{\ensuremath{\overline{A}}}
\newcommand{\Bbar}{\ensuremath{\overline{B}}}
\newcommand{\Cbar}{\ensuremath{\overline{C}}}
\newcommand{\Dbar}{\ensuremath{\overline{D}}}
\newcommand{\Ebar}{\ensuremath{\overline{E}}}
\newcommand{\Fbar}{\ensuremath{\overline{F}}}
\newcommand{\Gbar}{\ensuremath{\overline{G}}}
\newcommand{\Hbar}{\ensuremath{\overline{H}}}
\newcommand{\Ibar}{\ensuremath{\overline{I}}}
\newcommand{\Jbar}{\ensuremath{\overline{J}}}
\newcommand{\Kbar}{\ensuremath{\overline{K}}}
\newcommand{\Lbar}{\ensuremath{\overline{L}}}
\newcommand{\Mbar}{\ensuremath{\overline{M}}}
\newcommand{\Nbar}{\ensuremath{\overline{N}}}
\newcommand{\Obar}{\ensuremath{\overline{O}}}
\newcommand{\Pbar}{\ensuremath{\overline{P}}}
\newcommand{\Qbar}{\ensuremath{\overline{Q}}}
\newcommand{\Rbar}{\ensuremath{\overline{R}}}
\newcommand{\Sbar}{\ensuremath{\overline{S}}}
\newcommand{\Tbar}{\ensuremath{\overline{T}}}
\newcommand{\Ubar}{\ensuremath{\overline{U}}}
\newcommand{\Vbar}{\ensuremath{\overline{V}}}
\newcommand{\Wbar}{\ensuremath{\overline{W}}}
\newcommand{\Xbar}{\ensuremath{\overline{X}}}
\newcommand{\Ybar}{\ensuremath{\overline{Y}}}
\newcommand{\Zbar}{\ensuremath{\overline{Z}}}
\newcommand{\abar}{\ensuremath{\overline{a}}}
\newcommand{\bbar}{\ensuremath{\overline{b}}}
\newcommand{\cbar}{\ensuremath{\overline{c}}}
\newcommand{\dbar}{\ensuremath{\overline{d}}}
\newcommand{\ebar}{\ensuremath{\overline{e}}}
\newcommand{\fbar}{\ensuremath{\overline{f}}}
\newcommand{\gbar}{\ensuremath{\overline{g}}}
%\newcommand{\hbar}{\ensuremath{\overline{h}}} % whoops, \hbar means something else!
\newcommand{\ibar}{\ensuremath{\overline{\imath}}}
\newcommand{\jbar}{\ensuremath{\overline{\jmath}}}
\newcommand{\kbar}{\ensuremath{\overline{k}}}
\newcommand{\lbar}{\ensuremath{\overline{l}}}
\newcommand{\mbar}{\ensuremath{\overline{m}}}
\newcommand{\nbar}{\ensuremath{\overline{n}}}
%\newcommand{\obar}{\ensuremath{\overline{o}}}
\newcommand{\pbar}{\ensuremath{\overline{p}}}
\newcommand{\qbar}{\ensuremath{\overline{q}}}
\newcommand{\rbar}{\ensuremath{\overline{r}}}
\newcommand{\sbar}{\ensuremath{\overline{s}}}
\newcommand{\tbar}{\ensuremath{\overline{t}}}
\newcommand{\ubar}{\ensuremath{\overline{u}}}
\newcommand{\vbar}{\ensuremath{\overline{v}}}
\newcommand{\wbar}{\ensuremath{\overline{w}}}
\newcommand{\xbar}{\ensuremath{\overline{x}}}
\newcommand{\ybar}{\ensuremath{\overline{y}}}
\newcommand{\zbar}{\ensuremath{\overline{z}}}

% tildes
\newcommand{\Atil}{\ensuremath{\widetilde{A}}}
\newcommand{\Btil}{\ensuremath{\widetilde{B}}}
\newcommand{\Ctil}{\ensuremath{\widetilde{C}}}
\newcommand{\Dtil}{\ensuremath{\widetilde{D}}}
\newcommand{\Etil}{\ensuremath{\widetilde{E}}}
\newcommand{\Ftil}{\ensuremath{\widetilde{F}}}
\newcommand{\Gtil}{\ensuremath{\widetilde{G}}}
\newcommand{\Htil}{\ensuremath{\widetilde{H}}}
\newcommand{\Itil}{\ensuremath{\widetilde{I}}}
\newcommand{\Jtil}{\ensuremath{\widetilde{J}}}
\newcommand{\Ktil}{\ensuremath{\widetilde{K}}}
\newcommand{\Ltil}{\ensuremath{\widetilde{L}}}
\newcommand{\Mtil}{\ensuremath{\widetilde{M}}}
\newcommand{\Ntil}{\ensuremath{\widetilde{N}}}
\newcommand{\Otil}{\ensuremath{\widetilde{O}}}
\newcommand{\Ptil}{\ensuremath{\widetilde{P}}}
\newcommand{\Qtil}{\ensuremath{\widetilde{Q}}}
\newcommand{\Rtil}{\ensuremath{\widetilde{R}}}
\newcommand{\Stil}{\ensuremath{\widetilde{S}}}
\newcommand{\Ttil}{\ensuremath{\widetilde{T}}}
\newcommand{\Util}{\ensuremath{\widetilde{U}}}
\newcommand{\Vtil}{\ensuremath{\widetilde{V}}}
\newcommand{\Wtil}{\ensuremath{\widetilde{W}}}
\newcommand{\Xtil}{\ensuremath{\widetilde{X}}}
\newcommand{\Ytil}{\ensuremath{\widetilde{Y}}}
\newcommand{\Ztil}{\ensuremath{\widetilde{Z}}}
\newcommand{\atil}{\ensuremath{\widetilde{a}}}
\newcommand{\btil}{\ensuremath{\widetilde{b}}}
\newcommand{\ctil}{\ensuremath{\widetilde{c}}}
\newcommand{\dtil}{\ensuremath{\widetilde{d}}}
\newcommand{\etil}{\ensuremath{\widetilde{e}}}
\newcommand{\ftil}{\ensuremath{\widetilde{f}}}
\newcommand{\gtil}{\ensuremath{\widetilde{g}}}
\newcommand{\htil}{\ensuremath{\widetilde{h}}}
\newcommand{\itil}{\ensuremath{\widetilde{\imath}}}
\newcommand{\jtil}{\ensuremath{\widetilde{\jmath}}}
\newcommand{\ktil}{\ensuremath{\widetilde{k}}}
\newcommand{\ltil}{\ensuremath{\widetilde{l}}}
\newcommand{\mtil}{\ensuremath{\widetilde{m}}}
\newcommand{\ntil}{\ensuremath{\widetilde{n}}}
\newcommand{\otil}{\ensuremath{\widetilde{o}}}
\newcommand{\ptil}{\ensuremath{\widetilde{p}}}
\newcommand{\qtil}{\ensuremath{\widetilde{q}}}
\newcommand{\rtil}{\ensuremath{\widetilde{r}}}
\newcommand{\stil}{\ensuremath{\widetilde{s}}}
\newcommand{\ttil}{\ensuremath{\widetilde{t}}}
\newcommand{\util}{\ensuremath{\widetilde{u}}}
\newcommand{\vtil}{\ensuremath{\widetilde{v}}}
\newcommand{\wtil}{\ensuremath{\widetilde{w}}}
\newcommand{\xtil}{\ensuremath{\widetilde{x}}}
\newcommand{\ytil}{\ensuremath{\widetilde{y}}}
\newcommand{\ztil}{\ensuremath{\widetilde{z}}}

% Hats
\newcommand{\Ahat}{\ensuremath{\widehat{A}}}
\newcommand{\Bhat}{\ensuremath{\widehat{B}}}
\newcommand{\Chat}{\ensuremath{\widehat{C}}}
\newcommand{\Dhat}{\ensuremath{\widehat{D}}}
\newcommand{\Ehat}{\ensuremath{\widehat{E}}}
\newcommand{\Fhat}{\ensuremath{\widehat{F}}}
\newcommand{\Ghat}{\ensuremath{\widehat{G}}}
\newcommand{\Hhat}{\ensuremath{\widehat{H}}}
\newcommand{\Ihat}{\ensuremath{\widehat{I}}}
\newcommand{\Jhat}{\ensuremath{\widehat{J}}}
\newcommand{\Khat}{\ensuremath{\widehat{K}}}
\newcommand{\Lhat}{\ensuremath{\widehat{L}}}
\newcommand{\Mhat}{\ensuremath{\widehat{M}}}
\newcommand{\Nhat}{\ensuremath{\widehat{N}}}
\newcommand{\Ohat}{\ensuremath{\widehat{O}}}
\newcommand{\Phat}{\ensuremath{\widehat{P}}}
\newcommand{\Qhat}{\ensuremath{\widehat{Q}}}
\newcommand{\Rhat}{\ensuremath{\widehat{R}}}
\newcommand{\Shat}{\ensuremath{\widehat{S}}}
\newcommand{\That}{\ensuremath{\widehat{T}}}
\newcommand{\Uhat}{\ensuremath{\widehat{U}}}
\newcommand{\Vhat}{\ensuremath{\widehat{V}}}
\newcommand{\What}{\ensuremath{\widehat{W}}}
\newcommand{\Xhat}{\ensuremath{\widehat{X}}}
\newcommand{\Yhat}{\ensuremath{\widehat{Y}}}
\newcommand{\Zhat}{\ensuremath{\widehat{Z}}}
\newcommand{\ahat}{\ensuremath{\hat{a}}}
\newcommand{\bhat}{\ensuremath{\hat{b}}}
\newcommand{\chat}{\ensuremath{\hat{c}}}
\newcommand{\dhat}{\ensuremath{\hat{d}}}
\newcommand{\ehat}{\ensuremath{\hat{e}}}
\newcommand{\fhat}{\ensuremath{\hat{f}}}
\newcommand{\ghat}{\ensuremath{\hat{g}}}
\newcommand{\hhat}{\ensuremath{\hat{h}}}
\newcommand{\ihat}{\ensuremath{\hat{\imath}}}
\newcommand{\jhat}{\ensuremath{\hat{\jmath}}}
\newcommand{\khat}{\ensuremath{\hat{k}}}
\newcommand{\lhat}{\ensuremath{\hat{l}}}
\newcommand{\mhat}{\ensuremath{\hat{m}}}
\newcommand{\nhat}{\ensuremath{\hat{n}}}
\newcommand{\ohat}{\ensuremath{\hat{o}}}
\newcommand{\phat}{\ensuremath{\hat{p}}}
\newcommand{\qhat}{\ensuremath{\hat{q}}}
\newcommand{\rhat}{\ensuremath{\hat{r}}}
\newcommand{\shat}{\ensuremath{\hat{s}}}
\newcommand{\that}{\ensuremath{\hat{t}}}
\newcommand{\uhat}{\ensuremath{\hat{u}}}
\newcommand{\vhat}{\ensuremath{\hat{v}}}
\newcommand{\what}{\ensuremath{\hat{w}}}
\newcommand{\xhat}{\ensuremath{\hat{x}}}
\newcommand{\yhat}{\ensuremath{\hat{y}}}
\newcommand{\zhat}{\ensuremath{\hat{z}}}

%% FONTS AND DECORATION FOR GREEK LETTERS

%% the package `mathbbol' gives us blackboard bold greek and numbers,
%% but it does it by redefining \mathbb to use a different font, so that
%% all the other \mathbb letters look different too.  Here we import the
%% font with bb greek and numbers, but assign it a different name,
%% \mathbbb, so as not to replace the usual one.
\DeclareSymbolFont{bbold}{U}{bbold}{m}{n}
\DeclareSymbolFontAlphabet{\mathbbb}{bbold}
\newcommand{\bbDelta}{\ensuremath{\mathbbb{\Delta}}}
\newcommand{\bbone}{\ensuremath{\mathbbb{1}}}
\newcommand{\bbtwo}{\ensuremath{\mathbbb{2}}}
\newcommand{\bbthree}{\ensuremath{\mathbbb{3}}}

% greek with bars
\newcommand{\albar}{\ensuremath{\overline{\alpha}}}
\newcommand{\bebar}{\ensuremath{\overline{\beta}}}
\newcommand{\gmbar}{\ensuremath{\overline{\gamma}}}
\newcommand{\debar}{\ensuremath{\overline{\delta}}}
\newcommand{\phibar}{\ensuremath{\overline{\varphi}}}
\newcommand{\psibar}{\ensuremath{\overline{\psi}}}
\newcommand{\xibar}{\ensuremath{\overline{\xi}}}
\newcommand{\ombar}{\ensuremath{\overline{\omega}}}

% greek with hats
\newcommand{\alhat}{\ensuremath{\hat{\alpha}}}
\newcommand{\behat}{\ensuremath{\hat{\beta}}}
\newcommand{\gmhat}{\ensuremath{\hat{\gamma}}}
\newcommand{\dehat}{\ensuremath{\hat{\delta}}}

% greek with checks
\newcommand{\alchk}{\ensuremath{\check{\alpha}}}
\newcommand{\bechk}{\ensuremath{\check{\beta}}}
\newcommand{\gmchk}{\ensuremath{\check{\gamma}}}
\newcommand{\dechk}{\ensuremath{\check{\delta}}}

% greek with tildes
\newcommand{\altil}{\ensuremath{\widetilde{\alpha}}}
\newcommand{\betil}{\ensuremath{\widetilde{\beta}}}
\newcommand{\gmtil}{\ensuremath{\widetilde{\gamma}}}
\newcommand{\phitil}{\ensuremath{\widetilde{\varphi}}}
\newcommand{\psitil}{\ensuremath{\widetilde{\psi}}}
\newcommand{\xitil}{\ensuremath{\widetilde{\xi}}}
\newcommand{\omtil}{\ensuremath{\widetilde{\omega}}}

% MISCELLANEOUS SYMBOLS
\mdef\del{\partial}
\mdef\delbar{\overline{\partial}}
\let\sm\wedge
\newcommand{\dd}[1]{\ensuremath{\frac{\partial}{\partial {#1}}}}
\newcommand{\inv}{^{-1}}
\newcommand{\dual}{^{\vee}}
\mdef\hf{\textstyle\frac{1}{2}}
\mdef\thrd{\textstyle\frac{1}{3}}
\mdef\qtr{\textstyle\frac{1}{4}}
\let\meet\wedge
\let\join\vee
\let\dn\downarrow
\newcommand{\op}{^{\mathit{op}}}
\newcommand{\co}{^{\mathit{co}}}
\newcommand{\coop}{^{\mathit{coop}}}
\let\adj\dashv
\SelectTips{cm}{}
\newdir{ >}{{}*!/-10pt/@{>}}    % extra spacing for tail arrows in XYpic
\newcommand{\pushoutcorner}[1][dr]{\save*!/#1+1.2pc/#1:(1,-1)@^{|-}\restore}
\newcommand{\pullbackcorner}[1][dr]{\save*!/#1-1.2pc/#1:(-1,1)@^{|-}\restore}
\let\iso\cong
\let\eqv\simeq
\let\cng\equiv
\mdef\Id{\mathrm{Id}}
\mdef\id{\mathrm{id}}
\alwaysmath{ell}
\alwaysmath{infty}
\alwaysmath{odot}
\def\frc#1/#2.{\frac{#1}{#2}}   % \frc x^2+1 / x^2-1 .
\mdef\ten{\mathrel{\otimes}}
\mdef\bigten{\bigotimes}
\mdef\sqten{\mathrel{\boxtimes}}
\def\pow(#1,#2){\mathop{\pitchfork}(#1,#2)} % powers and
\def\cpw{\mathop{\odot}}                    % copowers
\newcommand{\mathid}{\mbox{id}}
\newcommand{\cat}[1]{\ensuremath{\mathbf{#1}}}


%% OPERATORS
\DeclareMathOperator\lan{Lan}
\DeclareMathOperator\ran{Ran}
\DeclareMathOperator\colim{colim}
\DeclareMathOperator\coeq{coeq}
\DeclareMathOperator\eq{eq}
\DeclareMathOperator\Tot{Tot}
\DeclareMathOperator\cosk{cosk}
\DeclareMathOperator\sk{sk}
\DeclareMathOperator\im{im}
\DeclareMathOperator\Spec{Spec}
\DeclareMathOperator\Ho{Ho}
\DeclareMathOperator\Aut{Aut}
\DeclareMathOperator\End{End}
\DeclareMathOperator\Hom{Hom}
\DeclareMathOperator\Map{Map}

%% TIKZ ARROWS AND HIGHER CELLS
\makeatletter
\def\slashedarrowfill@#1#2#3#4#5{%
  $\m@th\thickmuskip0mu\medmuskip\thickmuskip\thinmuskip\thickmuskip
   \relax#5#1\mkern-7mu%
   \cleaders\hbox{$#5\mkern-2mu#2\mkern-2mu$}\hfill
   \mathclap{#3}\mathclap{#2}%
   \cleaders\hbox{$#5\mkern-2mu#2\mkern-2mu$}\hfill
   \mkern-7mu#4$%
}

\def\Rightslashedarrowfill@{%
  \slashedarrowfill@\Relbar\Relbar\Mapstochar\Rightarrow}
\newcommand\xslashedRightarrow[2][]{%
  \ext@arrow 0055{\Rightslashedarrowfill@}{#1}{#2}}
\def\hTo{\xslashedRightarrow{}}
\def\hToo{\xslashedRightarrow{\quad}}
\let\xhTo\xslashedRightarrow

\pagestyle{empty}

\newcommand{\Rightthreecell}{\RRightarrow}
\newcommand{\Rtwocell}{\Rightarrow}

\tikzstyle{doubletick}=[-implies, double equal sign distance, postaction={decorate},decoration={markings,mark=at position .5 with {\draw[-] (0,-0.1) -- (0,0.1);}}]

\tikzstyle{darrow}=[-implies, double equal sign distance]

\tikzstyle{doubleeq}=[double equal sign distance]


%% ARROWS
% \to already exists
\newcommand{\too}[1][]{\ensuremath{\overset{#1}{\longrightarrow}}}
\newcommand{\ot}{\ensuremath{\leftarrow}}
\newcommand{\oot}[1][]{\ensuremath{\overset{#1}{\longleftarrow}}}
\let\toot\rightleftarrows
\let\otto\leftrightarrows
\let\Impl\Rightarrow
\let\imp\Rightarrow
\let\toto\rightrightarrows
\let\into\hookrightarrow
\let\xinto\xhookrightarrow
\mdef\we{\overset{\sim}{\longrightarrow}}
\mdef\leftwe{\overset{\sim}{\longleftarrow}}
\let\mono\rightarrowtail
\let\leftmono\leftarrowtail
\let\cof\rightarrowtail
\let\leftcof\leftarrowtail
\let\epi\twoheadrightarrow
\let\leftepi\twoheadleftarrow
\let\fib\twoheadrightarrow
\let\leftfib\twoheadleftarrow
\let\cohto\rightsquigarrow
\let\maps\colon
\newcommand{\spam}{\,:\!}       % \maps for left arrows

\newsavebox{\DDownarrowbox}
\savebox{\DDownarrowbox}{\tikz[scale=1.5]{\draw[-implies,double equal
sign distance] (0,.1) -- (0,-.1); \draw (0,.1) -- (0,-.1);}}
\newcommand{\DDownarrow}{\mathrel{\raisebox{-.2em}{\usebox{\DDownarrowbox}}}}

\newsavebox{\RRightarrowbox}
\savebox{\RRightarrowbox}{\tikz[scale=1.5]{\draw[-implies,double equal
sign distance] (-.1,0) -- (.1,0); \draw (-.1,0) -- (.1,0);}}
\newcommand{\RRightarrow}{\mathrel{\raisebox{.2em}{\usebox{\RRightarrowbox}}}}

%\newsavebox{\Rightslashedarrowbox}
%\savebox{\Rightslashedarrowbox}{\tikz[scale=1.5]{\draw[Rightslashedarrow{}] (-.1,0) -- (1,0);}}
%\newcommand{\Rightslashedarrow}{\mathrel{\raisebox{-.2em}%{\usebox{\Rightslashedarrowbox}}}}


%% EXTENSIBLE ARROWS
\let\xto\xrightarrow
\let\xot\xleftarrow
% See Voss' Mathmode.tex for instructions on how to create new
% extensible arrows.
\def\rightarrowtailfill@{\arrowfill@{\Yright\joinrel\relbar}\relbar\rightarrow}
\newcommand\xrightarrowtail[2][]{\ext@arrow 0055{\rightarrowtailfill@}{#1}{#2}}
\let\xmono\xrightarrowtail
\let\xcof\xrightarrowtail
\def\twoheadrightarrowfill@{\arrowfill@{\relbar\joinrel\relbar}\relbar\twoheadrightarrow}
\newcommand\xtwoheadrightarrow[2][]{\ext@arrow 0055{\twoheadrightarrowfill@}{#1}{#2}}
\let\xepi\xtwoheadrightarrow
\let\xfib\xtwoheadrightarrow
% Let's leave the left-going ones until I need them.

%% EXTENSIBLE SLASHED ARROWS
% Making extensible slashed arrows, by modifying the underlying AMS code.
% Arguments are:
% 1 = arrowhead on the left (\relbar or \Relbar if none)
% 2 = fill character (usually \relbar or \Relbar)
% 3 = slash character (such as \mapstochar or \Mapstochar)
% 4 = arrowhead on the left (\relbar or \Relbar if none)
% 5 = display mode (\displaystyle etc)
\def\slashedarrowfill@#1#2#3#4#5{%
  $\m@th\thickmuskip0mu\medmuskip\thickmuskip\thinmuskip\thickmuskip
   \relax#5#1\mkern-7mu%
   \cleaders\hbox{$#5\mkern-2mu#2\mkern-2mu$}\hfill
   \mathclap{#3}\mathclap{#2}%
   \cleaders\hbox{$#5\mkern-2mu#2\mkern-2mu$}\hfill
   \mkern-7mu#4$%
}
% Here's the idea: \<slashed>arrowfill@ should be a box containing
% some stretchable space that is the "middle of the arrow".  This
% space is created as a "leader" using \cleader<thing>\hfill, which
% fills an \hfill of space with copies of <thing>.  Here instead of
% just one \cleader, we use two, with the slash in between (and an
% extra copy of the filler, to avoid extra space around the slash).
\def\rightslashedarrowfill@{%
  \slashedarrowfill@\relbar\relbar\mapstochar\rightarrow}
\newcommand\xslashedrightarrow[2][]{%
  \ext@arrow 0055{\rightslashedarrowfill@}{#1}{#2}}
\mdef\hto{\xslashedrightarrow{}}
\mdef\htoo{\xslashedrightarrow{\quad}}
\let\xhto\xslashedrightarrow

%% To get a slashed arrow in XYpic, do
% \[\xymatrix{A \ar[r]|-@{|} & B}\]

% ISOMORPHISMS
\def\xiso#1{\mathrel{\mathrlap{\smash{\xto[\smash{\raisebox{1.3mm}{$\scriptstyle\sim$}}]{#1}}}\hphantom{\xto{#1}}}}
\def\toiso{\xto{\smash{\raisebox{-.5mm}{$\scriptstyle\sim$}}}}

% SHADOWS
\def\shvar#1#2{{\ensuremath{%
  \hspace{1mm}\makebox[-1mm]{$#1\langle$}\makebox[0mm]{$#1\langle$}\hspace{1mm}%
  {#2}%
  \makebox[1mm]{$#1\rangle$}\makebox[0mm]{$#1\rangle$}%
}}}
\def\sh{\shvar{}}
\def\scriptsh{\shvar{\scriptstyle}}
\def\bigsh{\shvar{\big}}
\def\Bigsh{\shvar{\Big}}
\def\biggsh{\shvar{\bigg}}
\def\Biggsh{\shvar{\Bigg}}

%HIGHER CELLS



% THEOREM-TYPE ENVIRONMENTS, hacked to
%% (a) number all with the same numbers, and
%% (b) have the right names for autoref
\def\defthm#1#2{%
  \newtheorem{#1}{#2}[section]%
  \expandafter\def\csname #1autorefname\endcsname{#2}%
  \expandafter\let\csname c@#1\endcsname\c@thm}
\newtheorem{thm}{Theorem}[section]
\newcommand{\thmautorefname}{Theorem}
\defthm{cor}{Corollary}
\defthm{prop}{Proposition}
\defthm{lem}{Lemma}
\defthm{sch}{Scholium}
\defthm{assume}{Assumption}
\defthm{claim}{Claim}
\defthm{conj}{Conjecture}
\defthm{hyp}{Hypothesis}
\defthm{fact}{Fact}
\theoremstyle{definition}
\defthm{defn}{Definition}
\defthm{notn}{Notation}
\theoremstyle{remark}
\defthm{rmk}{Remark}
\defthm{eg}{Example}
\defthm{egs}{Examples}
\defthm{ex}{Exercise}
\defthm{ceg}{Counterexample}

% How to get QED symbols inside equations at the end of the statements
% of theorems.  AMS LaTeX knows how to do this inside equations at the
% end of *proofs* with \qedhere, and at the end of the statement of a
% theorem with \qed (meaning no proof will be given), but it can't
% seem to combine the two.  Use this instead.
\def\thmqedhere{\expandafter\csname\csname @currenvir\endcsname @qed\endcsname}

% Number numbered lists as (i), (ii), ...
\renewcommand{\theenumi}{(\roman{enumi})}
\renewcommand{\labelenumi}{\theenumi}

%% Labeling that keeps track of theorem-type names.  Superseded by
%% autoref from hyperref, as above, but we keep this in case we are
%% using a journal style file that is incompatible with hyperref.
% 
% \ifx\SK@label\undefined\let\SK@label\label\fi
% \let\your@thm\@thm
% \def\@thm#1#2#3{\gdef\currthmtype{#3}\your@thm{#1}{#2}{#3}}
% \def\xlabel#1{{\let\your@currentlabel\@currentlabel\def\@currentlabel
% {\currthmtype~\your@currentlabel}
% \SK@label{#1@}}\label{#1}}
% \def\xref#1{\ref{#1@}}

% Also number formulas with the theorem counter
\let\c@equation\c@thm
\numberwithin{equation}{section}

% Only show numbers for equations that are actually referenced (or
% whose tags are specified manually) - uses mathtools.
\mathtoolsset{showonlyrefs,showmanualtags}

%% Fix enumerate spacing with paralist.  This has two parts:
%%   1. enable mixing of "old spacing" lists with those adjusted by paralist
%%   2. allow us to specify a number based on which to adjust the spacing
%% For the first, use \killspacingtrue when you want the spacing
%% adjusted, then \killspacingfalse to turn adjustment off.  For the
%% second, use \maxenum=14 to set the widest number you want the
%% spacing to be calculated with.
\newlength\oldleftmargini       % save old spacing
\newlength\oldleftmarginii
\newlength\oldleftmarginiii
\newlength\oldleftmarginiv
\newlength\oldleftmarginv
\newlength\oldleftmarginvi
\newcount\maxenum
\maxenum=7
\newif\ifkillspacing
\def\@adjust@enum@labelwidth{%
  \advance\@listdepth by 1\relax
  \ifkillspacing                % do the paralist thing
    \csname c@\@enumctr\endcsname\maxenum
    \settowidth{\@tempdima}{%
      \csname label\@enumctr\endcsname\hspace{\labelsep}}%
    \csname leftmargin\romannumeral\@listdepth\endcsname
      \@tempdima
  \else                         % otherwise, reset it
    \csname fixspacing\romannumeral\@listdepth\endcsname
  \fi
  \advance\@listdepth by -1\relax}
% these commands, one for each enum level (I couldn't get a generic
% one to work), test whether oldleftmargin has been set yet, and if
% not, set it from leftmargin; otherwise, they reset leftmargin to
% it.  Just setting oldleftmargin to leftmargin in the preamble
% doesn't seem to work.
\def\fixspacingi{\ifnum\oldleftmargini=0\setlength\oldleftmargini\leftmargini\else\setlength\leftmargini\oldleftmargini\fi}
\def\fixspacingii{\ifnum\oldleftmarginii=0\setlength\oldleftmarginii\leftmarginii\else\setlength\leftmarginii\oldleftmarginii\fi}
\def\fixspacingiii{\ifnum\oldleftmarginiii=0\setlength\oldleftmarginiii\leftmarginiii\else\setlength\leftmarginiii\oldleftmarginiii\fi}
\def\fixspacingiv{\ifnum\oldleftmarginiv=0\setlength\oldleftmarginiv\leftmarginiv\else\setlength\leftmarginiv\oldleftmarginiv\fi}
\def\fixspacingv{\ifnum\oldleftmarginv=0\setlength\oldleftmarginv\leftmarginv\else\setlength\leftmarginv\oldleftmarginv\fi}
\def\fixspacingvi{\ifnum\oldleftmarginvi=0\setlength\oldleftmarginvi\leftmarginvi\else\setlength\leftmarginvi\oldleftmarginvi\fi}

%% Fix paralist references, so that we can refer to (1) instead of
%% just 1.
\def\pl@label#1#2{%
  \edef\pl@the{\noexpand#1{\@enumctr}}%
  \pl@lab\expandafter{\the\pl@lab\csname yourthe\@enumctr\endcsname}%
  \advance\@tempcnta1
  \pl@loop}
\def\@enumlabel@#1[#2]{%
  \@plmylabeltrue
  \@tempcnta0
  \pl@lab{}%
  \let\pl@the\pl@qmark
  \expandafter\pl@loop\@gobble#2\@@@
  \ifnum\@tempcnta=1\else
    \PackageWarning{paralist}{Incorrect label; no or multiple
      counters.\MessageBreak The label is: \@gobble#2}%
  \fi
  \expandafter\edef\csname label\@enumctr\endcsname{\the\pl@lab}%
  \expandafter\edef\csname the\@enumctr\endcsname{\the\pl@lab}%
  \expandafter\let\csname yourthe\@enumctr\endcsname\pl@the
  #1}


% GREEK LETTERS, ETC.
\alwaysmath{alpha}
\alwaysmath{beta}
\alwaysmath{gamma}
\alwaysmath{Gamma}
\alwaysmath{delta}
\alwaysmath{Delta}
\alwaysmath{epsilon}
\mdef\ep{\varepsilon}
\alwaysmath{zeta}
\alwaysmath{eta}
\alwaysmath{theta}
\alwaysmath{Theta}
\alwaysmath{iota}
\alwaysmath{kappa}
\alwaysmath{lambda}
\alwaysmath{Lambda}
\alwaysmath{mu}
\alwaysmath{nu}
\alwaysmath{xi}
\alwaysmath{pi}
\alwaysmath{rho}
\alwaysmath{sigma}
\alwaysmath{Sigma}
\alwaysmath{tau}
\alwaysmath{upsilon}
\alwaysmath{Upsilon}
\alwaysmath{phi}
\alwaysmath{Pi}
\alwaysmath{Phi}
\mdef\ph{\varphi}
\alwaysmath{chi}
\alwaysmath{psi}
\alwaysmath{Psi}
\alwaysmath{omega}
\alwaysmath{Omega}
\let\al\alpha
\let\be\beta
\let\gm\gamma
\let\Gm\Gamma
\let\de\delta
\let\De\Delta
\let\si\sigma
\let\Si\Sigma
\let\om\omega
\let\ka\kappa
\let\la\lambda
\let\La\Lambda
\let\ze\zeta
\let\th\theta
\let\Th\Theta
\let\vth\vartheta

\makeatother

% Tikz styles
\tikzstyle{tickarrow}=[->,postaction={decorate},decoration={markings,mark=at position .5 with {\draw[-] (0,-0.1) -- (0,0.1);}},line width=0.50]

% Local Variables:
% mode: latex
% TeX-master: ""
% End:

\begin{document}

{\small
\begin{equation*}
\begin{aligned}
\begin{tikzpicture}[xscale=2.25, yscale=1.5]
%%%% Row A
\node (A1) at (0,7){$\substack{\tens (\tens \times \transid)}$};
\node (A3) at (3,7){$\substack{\tens (\transid \times \tens) }$};
\node (A5) at (4.5,7){$\substack{\tens (\transid \times \tens) \\ ( \transid \times \tens \times \transid) \\ (\transid \times \transid  \times I \times \transid)}$};
\node (A7) at (6,7){$\substack{\tens (\transid \times \tens)\\(\transid \times \transid \times \tens)\\ (\transid \times \transid  \times I \times \transid)}$};
\node (A8) at (7,7){$\substack{\tens (\transid \times \tens)}$};
%%%
\draw[doubleloose] (A1) to node[above]{$\substack{\alpha }$} (A3);
\draw[doubleloose] (A3) to node[above]{$\substack{\looseid \looseid \\(\looseid \times r^{-1} \times  \looseid) }$} (A5);
\draw[doubleloose] (A5) to node[above]{$\substack{\looseid (\looseid \times \alpha) \looseid }$} (A7);
\draw[doubleloose] (A7) to node[above]{$\substack{\looseid \looseid \\ (\looseid \times \looseid \times l)}$} (A8);
%%%% Row B
\node (B1) at (0,6){$\substack{\tens (\tens \times \transid)}$};
\node (B3) at (3,6){$\substack{\tens (\tens \times \transid) \\ (\transid \times \tens \times \transid) \\ (\transid \times \transid \times I \times \transid)}$};
\node (B5) at (4.5,6){$\substack{\tens (\transid \times \tens) \\ ( \transid \times \tens \times \transid) \\ (\transid \times \transid  \times I \times \transid)}$};
\node (B7) at (6,6){$\substack{\tens (\transid \times \tens)\\(\transid \times \transid \times \tens)\\ (\transid \times \transid  \times I \times \transid)}$};
\node (B8) at (7,6){$\substack{\tens (\transid \times \tens)}$};
%%%
\draw[doubleloose] (B1) to node[above]{$\substack{\looseid \looseid \\(\looseid \times r^{-1} \times  \looseid)}$} (B3);
\draw[doubleloose] (B3) to node[above]{$\substack{\alpha \looseid \looseid  }$} (B5);
\draw[doubleloose] (B5) to node[above]{$\substack{\looseid (\looseid \times \alpha) \looseid }$} (B7);
\draw[doubleloose] (B7) to node[above]{$\substack{\looseid \looseid \\ (\looseid \times \looseid \times l)}$} (B8);
%%% AB
\draw[doubletighteq] (A1) to (B1);
\draw[doubletighteq] (A5) to (B5);
\draw[doubletighteq] (A8) to (B8);
%%%% Row C
\node (C1) at (0,5){$\substack{\tens (\tens \times \transid)}$};
\node (C15) at (.9,5){$\substack{\tens (\tens \times \transid) \\(\transid \times I \times \transid) \\(\tens \times \transid \times \transid)}$};
\node (C2) at (1.9,5){$\substack{\tens (\tens \times \transid) \\(\tens \times \transid \times \transid) \\(\transid \times \transid \times I \times \transid) \\}$};
\node (C3) at (3,5){$\substack{\tens (\tens \times \transid) \\ (\transid \times \tens \times \transid) \\ (\transid \times \transid \times I \times \transid)}$};
\node (C5) at (4.5,5){$\substack{\tens (\transid \times \tens) \\ ( \transid \times \tens \times \transid) \\ (\transid \times \transid  \times I \times \transid)}$};
\node (C7) at (6,5){$\substack{\tens (\transid \times \tens)\\(\transid \times \transid \times \tens)\\ (\transid \times \transid  \times I \times \transid)}$};
\node (C8) at (7,5){$\substack{\tens (\transid \times \tens)}$};
%%%
\draw[doubleloose] (C1) to node[above]{$\substack{\looseid (r^{-1} \times \looseid)\\ \looseid}$} (C15);
\draw[doubletighteq] (C15) to  (C2);
\draw[doubleloose] (C2) to node[above]{$\substack{\looseid (\alpha \times \looseid) \\ \looseid }$} (C3);
\draw[doubleloose] (C3) to node[above]{$\substack{\substack{\alpha } \looseid \looseid  }$} (C5);
\draw[doubleloose] (C5) to node[above]{$\substack{\looseid (\looseid \times \alpha) \looseid }$} (C7);
\draw[doubleloose] (C7) to node[above]{$\substack{\looseid \looseid \\ (\looseid \times \looseid \times l)}$} (C8);
%%% BC
\draw[doubletighteq] (B1) to (C1);
\draw[doubletighteq] (B3) to (C3);
\draw[doubletighteq] (B8) to (C8);
%%%% Row D
\node (D1) at (0,4){$\substack{\tens (\tens \times \transid)}$};
\node (D15) at (.9,4){$\substack{\tens (\tens \times \transid) \\(\transid \times I \times \transid) \\(\tens \times \transid \times \transid)}$};
\node (D2) at (1.9,4){$\substack{\tens (\tens \times \transid) \\(\tens \times \transid \times \transid) \\(\transid \times \transid \times I \times \transid) \\}$};
\node (D3) at (3,4){$\substack{\tens (\transid \times \tens) \\ (\tens \times \transid \times  \transid) \\ (\transid \times \transid \times I \times \transid)}$};
\node (D5) at (4.5,4){$\substack{\tens (\tens \times \transid) \\ (\transid \times \transid \times \tens) \\ (\transid \times \transid \times I \times \transid)}$};
\node (D7) at (6,4){$\substack{\tens (\transid \times \tens) \\ (\transid \times \transid \times \tens) \\ (\transid \times \transid \times I \times \transid)}$};
\node (D8) at (7,4){$\substack{\tens (\transid \times \tens)}$};
%%%
\draw[doubleloose] (D1) to node[above]{$\substack{\looseid (r^{-1} \times \looseid)\\ \looseid}$} (D15);
\draw[doubletighteq] (D15) to  (D2);
\draw[doubleloose] (D2) to node[above]{$\substack{\alpha \looseid \looseid }$} (D3);
\draw[doubletighteq] (D3) to (D5);
\draw[doubleloose] (D5) to node[above]{$\substack{\alpha  \looseid \looseid  }$} (D7);
\draw[doubleloose] (D7) to node[above]{$\substack{\looseid \looseid \\ (\looseid \times \looseid \times l)}$} (D8);
%%% CD
\draw[doubletighteq] (C1) to (D1);
\draw[doubletighteq] (C2) to (D2);
\draw[doubletighteq] (C7) to (D7);
\draw[doubletighteq] (C8) to (D8);
%%%% Row E
\node (E1) at (0,3){$\substack{\tens (\tens \times \transid)}$};
\node (E15) at (.9,3){$\substack{\tens (\tens \times \transid) \\(\transid \times I \times \transid) \\(\tens \times \transid \times \transid)}$};
\node (E2) at (1.9,3){$\substack{\tens (\tens \times \transid) \\(\tens \times \transid \times \transid) \\(\transid \times \transid \times I \times \transid) \\}$};
\node (E3) at (3,3){$\substack{\tens (\transid \times \tens) \\ (\tens \times \transid \times  \transid) \\ (\transid \times \transid \times I \times \transid)}$};
\node (E5) at (4.5,3){$\substack{\tens (\transid \times \tens) \\ (\transid \times \transid \times \tens) \\ (\transid \times \transid \times I \times \transid)}$};
\node (E7) at (6,3){$\substack{\tens (\tens \times \transid)}$};
\node (E8) at (7,3){$\substack{\tens (\transid \times \tens)}$};
%%%
\draw[doubleloose] (E1) to node[above]{$\substack{\looseid (r^{-1} \times \looseid)\\ \looseid}$} (E15);
\draw[doubletighteq] (E15) to  (E2);
\draw[doubleloose] (E2) to node[above]{$\substack{\alpha \looseid \looseid }$} (E3);
\draw[doubletighteq] (E3) to (E5);
\draw[doubleloose] (E5) to node[above]{$\substack{\looseid \looseid (\looseid \times \looseid \times l)  }$} (E7);
\draw[doubleloose] (E7) to node[above]{$\substack{\alpha }$} (E8);
%%% DE
\draw[doubletighteq] (D1) to (E1);
\draw[doubletighteq] (D5) to (E5);
\draw[doubletighteq] (D8) to (E8);
%%%% Row F
\node (F1) at (0,2){$\substack{\tens (\tens \times \transid)}$};
\node (F7) at (6,2){$\substack{\tens (\tens \times \transid)}$};
\node (F8) at (7,2){$\substack{\tens (\transid \times \tens)}$};
%%%
\draw[doubleloose] (F1) to node[above]{$\substack{\looseid}$} (F7);
\draw[doubleloose] (F7) to node[above]{$\substack{\alpha }$} (F8);
%%% EF
\draw[doubletighteq] (E1) to (F1);
\draw[doubletighteq] (E7) to (F7);
\draw[doubletighteq] (E8) to (F8);
%%% 3-cells
\node at (2.5,6.5) {$\substack{\iso \horl \verc {\horr}^{-1} }$};
\node at (6,6.5) {$\substack{= }$};
\node at (1.5,5.5) {$\substack{\Downarrow \tightid \tightid (\tightid \times \rho \times \tightid) }$};
\node at (5,5.5) {$\substack{= }$};
\node at (1,4.5) {$\substack{=}$};
\node at (4,4.5) {$\substack{\DDownarrow \pi \tightid }$};
\node at (6.5,4.5) {$\substack{=}$};
\node at (2.5,3.5) {$\substack{=}$};
\node at (6,3.5) {$\substack{\iso \horl \verc {\horr}^{-1} }$};
\node at (3,2.5) {$\substack{\DDownarrow \tightid (\mu \times \tightid) }$};
\node at (6.5,2.5) {$\substack{=}$};
\end{tikzpicture} 
\end{aligned}
\end{equation*}
\begin{equation}\label{eq:monobjeq3}
=
\end{equation}
\begin{equation*}
\begin{tikzpicture}[xscale=2.25, yscale=1.5]
%%%% Row A
\node (A1) at (0,7){$\substack{\tens (\tens \times \transid)}$};
\node (A3) at (2,7){$\substack{\tens (\transid \times \tens) }$};
\node (A5) at (4,7){$\substack{\tens (\transid \times \tens) \\ ( \transid \times \tens \times \transid) \\ (\transid \times \transid  \times I \times \transid)}$};
\node (A7) at (6,7){$\substack{\tens (\transid \times \tens)\\(\transid \times \transid \times \tens)\\ (\transid \times \transid  \times I \times \transid)}$};
\node (A8) at (7,7){$\substack{\tens (\transid \times \tens)}$};
%%%
\draw[doubleloose] (A1) to node[above]{$\substack{\alpha }$} (A3);
\draw[doubleloose] (A3) to node[above]{$\substack{\looseid \looseid \\(\looseid \times r^{-1} \times  \looseid) }$} (A5);
\draw[doubleloose] (A5) to node[above]{$\substack{\looseid (\looseid \times \alpha) \looseid }$} (A7);
\draw[doubleloose] (A7) to node[above]{$\substack{\looseid \looseid \\ (\looseid \times l \times \looseid)}$} (A8);
%%%% Row B
\node (B1) at (0,6){$\substack{\tens (\tens \times \transid)}$};
\node (B3) at (2,6){$\substack{\tens ( \transid \times \tens)}$};
\node (B8) at (7,6){$\substack{\tens (\transid \times \tens)}$};
%%%
\draw[doubleloose] (B1) to node[above]{$\substack{\alpha }$} (B3);
\draw[doubleloose] (B3) to node[above]{$\substack{\looseid}$}(B8);
%%% AB
\draw[doubletighteq] (A1) to (B1);
\draw[doubletighteq] (A3) to (B3);
\draw[doubletighteq] (A8) to (B8);
%%%% Row C
\node (C1) at (0,5){$\substack{\tens (\tens \times \transid)}$};
\node (C7) at (6,5){$\substack{\tens (\tens \times \transid)}$};\node (C8) at (7,5){$\substack{\tens (\transid \times \tens)}$};
%%%
\draw[doubleloose] (C1) to node[above]{$\looseid$} (C7);
\draw[doubleloose] (C7) to node[above]{$\substack{\alpha }$}(C8);
%%% BC
\draw[doubletighteq] (B1) to  (C1);
\draw[doubletighteq] (B8) to  (C8);
%%%
\node at (5,6.5) {$\substack{ \DDownarrow \tightid (\tightid \times \mu)}$};
\node at (6.5,6.5) {$\substack{=}$};
\node at (3.5,5.5) {$\substack{\iso \horl \verc {\horr}^{-1} }$};
\end{tikzpicture} 
\end{equation*}}
\end{document}  \newpage


\subsubsection*{Lax Monoidal 1-cell}

[To be added]

\subsubsection*{Strong Monoidal 1-cell}

The structure cells $\chi$ and $\bar{\chi}$ form an adjoint equivalence, if  there exist globular 3-cells $\eta$ and $\epsilon$, such that the equations below hold. 

%
\documentclass[12pt]{ociamthesis}
\usepackage{tikz}
\newcommand{\id}{\mathrm{id}}
\begin{document}

\begin{equation}\label{eq:strong2mates}
\begin{pic}[scale=1.75]
\draw[fill=blue, opacity = 0.5, draw=blue] (-1,0) -- (-1,-2) -- (0.6,-2) -- (0.6, -.6) -- (0,-.6) -- (0,-1.4) --  (-.6,-1.4) -- (-.6, 0) -- (-1,0);
\draw[fill=red, opacity = 0.5, draw=red] (-.6,0) -- (-.6,-1.4) -- (0,-1.4) -- (0,-.6) -- (.6,-.6) -- (.6,-2) -- (1,-2) -- (1,0) -- (-.6,0);   
   %  \draw (.3,-.6) to (.3,0);
     \draw (0,-1.4) to node[left]{$\bar{\chi}$} (0,-.6);
     \draw (-.6,-1.4) to (-.6,0);
       \node[morphism, minimum width=20mm] (l) at (-.3,-1.4) {$\eta$};
       \node[morphism, minimum width=20mm] (r) at (.3,-.6) {$\epsilon$};
%\draw (-.3, -2) to (l.south);
\node at (-.6,.2) {$\chi$};
\node at (-.3,-1.8) {$\otimes(f\times f)$};
\node at (.3,-.2) {$f\otimes$};
\node at (.6,-2.2) {$\chi$};
    \end{pic}
    =
    \begin{pic}[scale=1.75]
\draw[fill=red, opacity = 0.5, draw=red] (-1,0) -- (-1,-2) -- (-0.6,-2) -- (-.6, -.6) -- (0,-.6) -- (0,-1.4) --  (.6,-1.4) -- (.6, 0) -- (-1,0);
\draw[fill=blue, opacity = 0.5, draw=blue] (-.6,-2) -- (-.6,-.6) -- (0,-.6) -- (0,-1.4) -- (.6,-1.4) -- (.6,0) -- (1,0) -- (1,-2) -- (-.6,-2);   
 %    \draw (.3,-1.4) to (.3,-2);
     \draw (0,-1.4) to node[left]{$\chi$} (0,-.6);
     \draw (-.6,-.6) to (-.6,-2);
       \node[morphism, minimum width=20mm] (l) at (-.3,-.6) {$\epsilon$};
       \node[morphism, minimum width=20mm] (r) at (.3,-1.4) {$\eta $};
%\draw (-.3, 0) to (l.north);
\node at (-.6,-2.2) {$\bar{\chi}$};
\node at (-.3,-.2) {$f\otimes $};
\node at (.3,-1.8) {$\otimes(f\times f)$};
\node at (.6,.2) {$\bar{\chi}$};
    \end{pic}
\end{equation}
\end{document} 


The 3-cells $\omega$ and $\bar{\omega}$ correspond to each other as mates, if the equation below holds. Here, one needs to note that for two adjoint equivalences $\alpha \dashv \bar{\alpha}$ and $\beta \dashv \bar{\beta}$, where $\alpha: f \rightarrow g$ and $\beta: h \rightarrow k$, there is an adjoint equivalence $\alpha \beta \dashv \bar{\alpha}\bar{\beta}$ witnessed by $\eta_{\alpha \beta} := \eta_{\alpha} \eta_{\beta}$ and $\epsilon_{\alpha \beta} :=  \epsilon_{\alpha} \epsilon_{\beta}$.  

%
\documentclass[12pt]{ociamthesis}
\usepackage{tikz}
\newcommand{\id}{\mathrm{id}}
\begin{document}

\begin{align}\label{eq:strong2mates2}
\begin{pic}[xscale=-.8]
\draw[fill=blue, opacity = 0.5, draw=black] (0,8) -- (0,0) -- (7,0) -- (7,5) -- (6,5) -- (6,1) -- (1,1) -- (1,8) -- (0,8);
\draw[fill=purple, opacity = 0.5, draw=black] (1,8) -- (1,1) -- (6,1) -- (6,4) -- (5,4) -- (5,2) -- (2,2) -- (2,8) -- (1,8); 
\draw[fill=red, opacity = 0.5, draw=black] (2,8) -- (2,2) -- (5,2) -- (5,4) -- (4,4) -- (4,3) -- (3,3) -- (3,8) -- (2,8); 
\draw[fill=orange, opacity = 0.5, draw=black] (3,8) -- (3,3) -- (4,3) -- (4,7) -- (9,7) -- (9,0) -- (10,0) -- (10,8) -- (3,8); 
\draw[fill=yellow, opacity = 0.5, draw=black] (9,0) -- (8,0) -- (8,6) -- (5,6) -- (5,4) -- (4,4) -- (4,7) -- (9,7) -- (9,0);
\draw[fill=green, opacity = 0.5, draw=black] (7,0) -- (8,0) -- (8,6) -- (5,6) -- (5,4) -- (6,4) -- (6,5) -- (7,5) -- (7,0) ;
\node[morphism, minimum width=20mm] (l) at (5,4) {$\bar{\omega}$};
\node[morphism, minimum width=10mm] (l) at (3.5,3) {$\eta_{\chi \looseid}$};
\node[morphism, minimum width=28mm] (l) at (3.5,2) {$\eta_{\looseid (\looseid\times\chi)}$};
\node[morphism, minimum width=42mm] (l) at (3.5,1) {$\eta_{\alpha\looseid}$};
\node[morphism, minimum width=10mm] (l) at (6.5,5) {$\epsilon_{\looseid(\chi \times \looseid)}$};
\node[morphism, minimum width=28mm] (l) at (6.5,6) {$\epsilon_{\chi\looseid}$};
\node[morphism, minimum width=42mm] (l) at (6.5,7) {$\epsilon_{\looseid\alpha}$};
    \end{pic}
    =
    \begin{pic}[xscale=-0.8]
\draw[fill=blue, opacity = 0.5, draw=black] (0,8) -- (0,0) -- (1,0) -- (1,8) -- (0,8);
\draw[fill=purple, opacity = 0.5, draw=black] (1,8) -- (1,4) -- (2,4) -- (2,8) -- (1,8); 
\draw[fill=red, opacity = 0.5, draw=black] (2,8) -- (2,4) -- (3,4) -- (3,8) --  (2,8); 
\draw[fill=orange, opacity = 0.5, draw=black] (3,8) -- (3,0) -- (4,0) -- (4,8) -- (3,8); 
\draw[fill=green, opacity = 0.5, draw=black] (1,4) -- (2,4) -- (2,0) -- (1,0) -- (1,4);
\draw[fill=yellow, opacity = 0.5, draw=black] (2,4) -- (3,4) -- (3,0) -- (2,0) -- (2,4);
\node[morphism, minimum width=20mm] (l) at (2,4) {$\omega$};
    \end{pic}
\end{align}
\end{document} 


%
\documentclass[12pt]{ociamthesis}
\usepackage{tikz}
\newcommand{\id}{\mathrm{id}}
\begin{document}

\begin{align}\label{eq:strong2mates3}
 \begin{pic}[yscale=.8, xscale=-.5]
\draw[fill=blue, opacity = 0.5, draw=black] (0,8) -- (0,0) -- (7,0) -- (7,5) -- (6,5) -- (6,4) -- (5,4) -- (5,2) -- (2,2) -- (2,8) -- (0,8);
\draw[fill=orange, opacity = 0.5, draw=black] (2,8) -- (2,2) -- (5,2) -- (5,4) -- (4,4) -- (4,7) -- (9,7) -- (9,0) -- (10,0) -- (10,8) -- (2,8); 
\draw[fill=yellow, opacity = 0.5, draw=black] (9,0) -- (8,0) -- (8,6) -- (5,6) -- (5,4) -- (4,4) -- (4,7) -- (9,7) -- (9,0);
\draw[fill=green, opacity = 0.5, draw=black] (7,0) -- (8,0) -- (8,6) -- (5,6) -- (5,4) -- (6,4) -- (6,5) -- (7,5) -- (7,0) ;
\node[morphism, minimum width=15mm] (l) at (5,4) {$\bar{\gamma}$};
\node[morphism, minimum width=20mm] (l) at (3.5,2) {$\eta_{l \looseid}$};
\node[morphism, minimum width=10mm] (l) at (6.5,5) {$\epsilon_{\looseid(\iota \times \looseid)\looseid}$};
\node[morphism, minimum width=20mm] (l) at (6.5,6) {$\epsilon_{\chi\looseid \looseid}$};
\node[morphism, minimum width=32mm] (l) at (6.5,7) {$\epsilon_{\looseid l}$};
    \end{pic}
    =
    \begin{pic}[yscale=0.8, xscale=-.5]
\draw[fill=blue, opacity = 0.5, draw=black] (0,8) -- (0,0) -- (1,0) -- (1,4) -- (2,4) -- (2,8) -- (0,8);
\draw[fill=orange, opacity = 0.5, draw=black] (2,8) -- (2,4) -- (3,4) -- (3,0) -- (4,0) -- (4,8) -- (2,8); 
\draw[fill=green, opacity = 0.5, draw=black] (1,4) -- (2,4) -- (2,0) -- (1,0) -- (1,4);
\draw[fill=yellow, opacity = 0.5, draw=black] (2,4) -- (3,4) -- (3,0) -- (2,0) -- (2,4);
\node[morphism, minimum width=15mm] (l) at (2,4) {$\gamma$};
    \end{pic}
    \end{align}
\end{document} 


%
\documentclass[12pt]{ociamthesis}
\usepackage{tikz}
\newcommand{\id}{\mathrm{id}}
\begin{document}

\begin{align}\label{eq:strong2mates4}
\begin{pic}[yscale=.8,xscale=-.5]
\draw[fill=blue, opacity = 0.5, draw=black] (0,8) -- (0,0) -- (8,0) -- (8,6) -- (5,6) -- (5,4) -- (6,4) -- (6,1) -- (1,1) -- (1,8) -- (0,8);
\draw[fill=purple, opacity = 0.5, draw=black] (1,8) -- (1,1) -- (6,1) -- (6,4) -- (5,4) -- (5,2) -- (2,2) -- (2,8) -- (1,8); 
\draw[fill=red, opacity = 0.5, draw=black] (2,8) -- (2,2) -- (5,2) -- (5,4) -- (4,4) -- (4,3) -- (3,3) -- (3,8) -- (2,8); 
\draw[fill=orange, opacity = 0.5, draw=black] (3,8) -- (3,3) -- (4,3) -- (4,4) -- (5,4) -- (5,6) -- (8,6) -- (8,0) -- (10,0) -- (10,8) -- (3,8); 
\node[morphism, minimum width=15mm] (l) at (5,4) {$\bar{\delta}$};
\node[morphism, minimum width=10mm] (l) at (3.5,3) {$\eta_{\chi \looseid}$};
\node[morphism, minimum width=20mm] (l) at (3.5,2) {$\eta_{\looseid (\looseid\times\iota)\looseid}$};
\node[morphism, minimum width=32mm] (l) at (3.5,1) {$\eta_{r \looseid}$};
\node[morphism, minimum width=20mm] (l) at (6.5,6) {$\epsilon_{\looseid r}$};
    \end{pic}
    =
   \begin{pic}[yscale=0.8, xscale=-.5]
\draw[fill=blue, opacity = 0.5, draw=black] (0,8) -- (0,0) -- (2,0) --(2,4) -- (1,4) -- (1,8) -- (0,8);
\draw[fill=purple, opacity = 0.5, draw=black] (1,8) -- (1,4) -- (2,4) -- (2,8) -- (1,8); 
\draw[fill=red, opacity = 0.5, draw=black] (2,8) -- (2,4) -- (3,4) -- (3,8) --  (2,8); 
\draw[fill=orange, opacity = 0.5, draw=black] (3,8) -- (3,4) -- (2,4) -- (2,0) -- (4,0) -- (4,8) -- (3,8); 
\node[morphism, minimum width=15mm] (l) at (2,4) {$\delta$};
    \end{pic}
\end{align}

\end{document} 
\newpage

\subsubsection*{Lax Monoidal 2-cell}

 %
\documentclass[12pt]{ociamthesis}
\usepackage{tikz}
\newcommand{\id}{\mathrm{id}}
\begin{document}


\begin{equation*}
\begin{aligned}
\begin{tikzpicture}[xscale=3, yscale=1.5]
\node (t0) at (0,2) {\scriptsize $\tens(I_B \times f)$};
\node (t1) at (1,2) {\scriptsize $\tens(f I_A \times f)$};
\node (t2) at (2,2) {\scriptsize $f \tens(I_A \times \transid)$};
\node (t3) at (3,2) {\scriptsize $f $};
\node (t4) at (4,2) {\scriptsize $g $};
\node (m0) at (0,1) {\scriptsize $\tens(I_B \times \transid)f$};
\node (m3) at (3,1) {\scriptsize $f $};
\node (m4) at (4,1) {\scriptsize $g $};
\node (b0) at (0,0) {\scriptsize $\tens(I_B \times \transid)f$};
\node (b3) at (3,0) {\scriptsize $\tens (I_B \times \transid)g$};
\node (b4) at (4,0) {\scriptsize $g $};
\draw[doubleloose] (t0) to node[above]{\scriptsize $\looseid_{\tens}(\iota_f \times \looseid_f)$} (t1);
\draw[doubleloose] (t1) to node[above]{\scriptsize $\chi (\looseid_{I \times \transid})$} (t2);
\draw[doubleloose] (t2) to node[above]{\scriptsize $\looseid_f l$} (t3);
\draw[doubleloose] (t3) to node[above]{\scriptsize $\beta$} (t4);
\draw[doubleloose] (m0) to node[above]{\scriptsize $l \looseid_f$} (m3);
\draw[doubleloose] (m3) to node[above]{\scriptsize $\beta$} (m4);
\draw[doubleloose] (b0) to node[above]{\scriptsize $\looseid_{\tens}(\beta \times \looseid_I)$} (b3);
\draw[doubleloose] (b3) to node[above]{\scriptsize $l \looseid_g$} (b4);
\draw[doubletighteq] (t0) to (m0);
\draw[doubletighteq] (m0) to (b0);
\draw[doubletighteq] (t3) to (m3);
\draw[doubletighteq] (t4) to (m4);
\draw[doubletighteq] (m4) to (b4);
\node at (1.5,1.5) {\scriptsize $\DDownarrow \gamma^f$};
\node at (3.5,1.5) {\scriptsize $\DDownarrow \tightid_{\beta}$};
\node at (2,0.5) {\scriptsize $\iso$};
\end{tikzpicture}
\end{aligned}
\end{equation*}
\begin{equation}\label{eq:mon2cell1}
=
\end{equation}
\begin{equation*}
\begin{aligned}
\begin{tikzpicture}[xscale=3, yscale=1.5]
\node (04) at (0,4) {\scriptsize $\tens(I_B \times f)$};
\node (14) at (1,4) {\scriptsize $\tens(f I_A\times f)$};
\node (24) at (2,4) {\scriptsize $f \tens(I_A \times \transid_A)$};
\node (34) at (3,4) {\scriptsize $f $};
\node (44) at (4,4) {\scriptsize $g $};
%%%%%%
\node (03) at (0,3) {\scriptsize $\tens(I_B \times f)$};
\node (13) at (1,3) {\scriptsize $\tens(f I_A\times f)$};
\node (23) at (2,3) {\scriptsize $f \tens(I_A \times \transid_A)$};
\node (33) at (3,3) {\scriptsize $g \tens(I_A \times \transid_A)$};
\node (43) at (4,3) {\scriptsize $g $};
%%%%%%
\node (02) at (0,2) {\scriptsize $\tens(I_B \times f)$};
\node (12) at (1,2) {\scriptsize $\tens(f I_A \times f)$};
\node (22) at (2,2) {\scriptsize $\tens(g I_A \times g)$};
\node (32) at (3,2) {\scriptsize $g \tens (I_A\times \transid_A)$};
\node (42) at (4,2) {\scriptsize $g $};
%%%%%% 
\node (01) at (0,1) {\scriptsize $\tens(I_B \times f)$};
\node (11) at (1,1) {\scriptsize $\tens(I_B \times g)$};
\node (21) at (2,1) {\scriptsize $\tens(g I_A \times g)$};
\node (31) at (3,1) {\scriptsize $g \tens (I_A \times \transid_A)$};
\node (41) at (4,1) {\scriptsize $g $};
%%%%%%%
\node (00) at (0,0) {\scriptsize $\tens(I_B \times \transid) f$};
\node (10) at (1,0) {\scriptsize $\tens(I_B \times \transid) g$};
\node (40) at (4,0) {\scriptsize $g $};
%%%%%%%
\draw[doubleloose] (04) to node[above]{\scriptsize $\looseid_{\tens}(\looseid_f \times \iota_f) $} (14);
\draw[doubleloose] (14) to node[above]{\scriptsize $\chi (\looseid_{I \times \transid})$} (24);
\draw[doubleloose] (24) to node[above]{\scriptsize $\looseid_f l$} (34);
\draw[doubleloose] (34) to node[above]{\scriptsize$\beta$} (44);
%%%%%%%
\draw[doubleloose] (03) to node[above]{\scriptsize $\looseid_{\tens}(\looseid_f \times \iota_f) $} (13);
\draw[doubleloose] (13) to node[above]{\scriptsize $\chi (\looseid_{I \times \transid})$} (23);
\draw[doubleloose] (23) to node[above]{\scriptsize $\beta \looseid_{\tens}(\looseid_{I\times \transid})$} (33);
\draw[doubleloose] (33) to node[above]{\scriptsize$\looseid_g l$} (43);
%%%%
\draw[doubleloose] (02) to node[above]{\scriptsize $\looseid_{\tens} (\iota_f  \times \looseid_f)$} (12);
\draw[doubleloose] (12) to node[above]{\scriptsize $\looseid_{\tens} (\beta \looseid_I \times \beta) $} (22);
\draw[doubleloose] (22) to node[above]{\scriptsize $\chi_g \looseid_{I \times \id}$} (32);
\draw[doubleloose] (32) to node[above]{\scriptsize $\looseid_g l$} (42);
%%%%%%
\draw[doubleloose] (01) to node[above]{\scriptsize $\looseid_{\tens} (\looseid_I \times \beta)$} (11);
\draw[doubleloose] (11) to node[above]{\scriptsize $\looseid_{\tens} (\iota_g \times \looseid_g) $} (21);
\draw[doubleloose] (21) to node[above]{\scriptsize $\chi \looseid_{I_A \times \transid}$} (31);
\draw[doubleloose] (31) to node[above]{\scriptsize $\looseid_g l$} (41);
%%%%%%
\draw[doubleloose] (00) to node[above]{\scriptsize $\looseid_{\tens} (\looseid_I  \times \looseid) \beta$} (10);
\draw[doubleloose] (10) to node[above]{\scriptsize $l \looseid_g $} (40);
%%%%%%
\draw[doubletighteq] (04) to (03);
\draw[=] (24) to (23);
\draw[doubletighteq] (44) to (43);
%%%%%%
\draw[doubletighteq] (03) to (02);
\draw[doubletighteq] (13) to (12);
\draw[doubletighteq] (33) to (32);
\draw[doubletighteq] (43) to (42);
%%%%%%
\draw[doubletighteq] (02) to (01);
\draw[doubletighteq] (22) to (21);
\draw[doubletighteq] (42) to (41);
%%%%%%
\draw[doubletighteq] (01) to (00);
\draw[doubletighteq] (11) to (10);
\draw[doubletighteq] (41) to (40);
%%%%%%%%
\node at (1,3.5) {\scriptsize $=$};
\node at (3,3.5) {\scriptsize $\DDownarrow \cong $};
\node at (.5,2.5) {\scriptsize $=$};
\node at (2,2.5) {\scriptsize $\DDownarrow \overline{\Pi^{\beta}\tightid_{\looseid}}$};
\node at (3.5,2.5) {\scriptsize $=$};
\node at (1,1.5) {\scriptsize $\DDownarrow \overline{\tightid_{\looseid} (M^{\beta} \times (\horl \verc {\horr}^{-1}) }$};
\node at (3,1.5) {\scriptsize $=$};
\node at (.5,.5) {\scriptsize $=$};
\node at (2.5,0.5) {\scriptsize $\DDownarrow \gamma^g$};
\end{tikzpicture}
\end{aligned}
\end{equation*}


\end{document} 
 \newpage

%
\documentclass[12pt]{ociamthesis}
\usepackage{tikz}
\usepackage{amsmath}
\usepackage{rotating}

\usepackage{amssymb,amsmath,stmaryrd,txfonts,mathrsfs,amsthm}
\usepackage[all,2cell]{xy}
\usepackage[neveradjust]{paralist}
\usepackage{hyperref}
\usepackage{mathtools}
\usepackage{tikz}
\usetikzlibrary{trees}
\usetikzlibrary{topaths}
\usetikzlibrary{decorations}
\usetikzlibrary{decorations.pathreplacing}
\usetikzlibrary{decorations.pathmorphing}
\usetikzlibrary{decorations.markings}
\usetikzlibrary{matrix,backgrounds,folding}
\usetikzlibrary{chains,scopes,positioning,fit}
\usetikzlibrary{arrows,shadows}
\usetikzlibrary{calc} 
\usetikzlibrary{chains}
\usetikzlibrary{shapes,shapes.geometric,shapes.misc}
\usepackage{smbicat}


\makeatletter
\let\ea\expandafter

%% Defining commands that are always in math mode.
\def\mdef#1#2{\ea\ea\ea\gdef\ea\ea\noexpand#1\ea{\ea\ensuremath\ea{#2}}}
\def\alwaysmath#1{\ea\ea\ea\global\ea\ea\ea\let\ea\ea\csname your@#1\endcsname\csname #1\endcsname
  \ea\def\csname #1\endcsname{\ensuremath{\csname your@#1\endcsname}}}

% Script letters
\newcommand{\sA}{\ensuremath{\mathscr{A}}}
\newcommand{\sB}{\ensuremath{\mathscr{B}}}
\newcommand{\sC}{\ensuremath{\mathscr{C}}}
\newcommand{\sD}{\ensuremath{\mathscr{D}}}
\newcommand{\sE}{\ensuremath{\mathscr{E}}}
\newcommand{\sF}{\ensuremath{\mathscr{F}}}
\newcommand{\sG}{\ensuremath{\mathscr{G}}}
\newcommand{\sH}{\ensuremath{\mathscr{H}}}
\newcommand{\sI}{\ensuremath{\mathscr{I}}}
\newcommand{\sJ}{\ensuremath{\mathscr{J}}}
\newcommand{\sK}{\ensuremath{\mathscr{K}}}
\newcommand{\sL}{\ensuremath{\mathscr{L}}}
\newcommand{\sM}{\ensuremath{\mathscr{M}}}
\newcommand{\sN}{\ensuremath{\mathscr{N}}}
\newcommand{\sO}{\ensuremath{\mathscr{O}}}
\newcommand{\sP}{\ensuremath{\mathscr{P}}}
\newcommand{\sQ}{\ensuremath{\mathscr{Q}}}
\newcommand{\sR}{\ensuremath{\mathscr{R}}}
\newcommand{\sS}{\ensuremath{\mathscr{S}}}
\newcommand{\sT}{\ensuremath{\mathscr{T}}}
\newcommand{\sU}{\ensuremath{\mathscr{U}}}
\newcommand{\sV}{\ensuremath{\mathscr{V}}}
\newcommand{\sW}{\ensuremath{\mathscr{W}}}
\newcommand{\sX}{\ensuremath{\mathscr{X}}}
\newcommand{\sY}{\ensuremath{\mathscr{Y}}}
\newcommand{\sZ}{\ensuremath{\mathscr{Z}}}

% Calligraphic letters
\newcommand{\cA}{\ensuremath{\mathcal{A}}}
\newcommand{\cB}{\ensuremath{\mathcal{B}}}
\newcommand{\cC}{\ensuremath{\mathcal{C}}}
\newcommand{\cD}{\ensuremath{\mathcal{D}}}
\newcommand{\cE}{\ensuremath{\mathcal{E}}}
\newcommand{\cF}{\ensuremath{\mathcal{F}}}
\newcommand{\cG}{\ensuremath{\mathcal{G}}}
\newcommand{\cH}{\ensuremath{\mathcal{H}}}
\newcommand{\cI}{\ensuremath{\mathcal{I}}}
\newcommand{\cJ}{\ensuremath{\mathcal{J}}}
\newcommand{\cK}{\ensuremath{\mathcal{K}}}
\newcommand{\cL}{\ensuremath{\mathcal{L}}}
\newcommand{\cM}{\ensuremath{\mathcal{M}}}
\newcommand{\cN}{\ensuremath{\mathcal{N}}}
\newcommand{\cO}{\ensuremath{\mathcal{O}}}
\newcommand{\cP}{\ensuremath{\mathcal{P}}}
\newcommand{\cQ}{\ensuremath{\mathcal{Q}}}
\newcommand{\cR}{\ensuremath{\mathcal{R}}}
\newcommand{\cS}{\ensuremath{\mathcal{S}}}
\newcommand{\cT}{\ensuremath{\mathcal{T}}}
\newcommand{\cU}{\ensuremath{\mathcal{U}}}
\newcommand{\cV}{\ensuremath{\mathcal{V}}}
\newcommand{\cW}{\ensuremath{\mathcal{W}}}
\newcommand{\cX}{\ensuremath{\mathcal{X}}}
\newcommand{\cY}{\ensuremath{\mathcal{Y}}}
\newcommand{\cZ}{\ensuremath{\mathcal{Z}}}

% blackboard bold letters
\newcommand{\lA}{\ensuremath{\mathbb{A}}}
\newcommand{\lB}{\ensuremath{\mathbb{B}}}
\newcommand{\lC}{\ensuremath{\mathbb{C}}}
\newcommand{\lD}{\ensuremath{\mathbb{D}}}
\newcommand{\lE}{\ensuremath{\mathbb{E}}}
\newcommand{\lF}{\ensuremath{\mathbb{F}}}
\newcommand{\lG}{\ensuremath{\mathbb{G}}}
\newcommand{\lH}{\ensuremath{\mathbb{H}}}
\newcommand{\lI}{\ensuremath{\mathbb{I}}}
\newcommand{\lJ}{\ensuremath{\mathbb{J}}}
\newcommand{\lK}{\ensuremath{\mathbb{K}}}
\newcommand{\lL}{\ensuremath{\mathbb{L}}}
\newcommand{\lM}{\ensuremath{\mathbb{M}}}
\newcommand{\lN}{\ensuremath{\mathbb{N}}}
\newcommand{\lO}{\ensuremath{\mathbb{O}}}
\newcommand{\lP}{\ensuremath{\mathbb{P}}}
\newcommand{\lQ}{\ensuremath{\mathbb{Q}}}
\newcommand{\lR}{\ensuremath{\mathbb{R}}}
\newcommand{\lS}{\ensuremath{\mathbb{S}}}
\newcommand{\lT}{\ensuremath{\mathbb{T}}}
\newcommand{\lU}{\ensuremath{\mathbb{U}}}
\newcommand{\lV}{\ensuremath{\mathbb{V}}}
\newcommand{\lW}{\ensuremath{\mathbb{W}}}
\newcommand{\lX}{\ensuremath{\mathbb{X}}}
\newcommand{\lY}{\ensuremath{\mathbb{Y}}}
\newcommand{\lZ}{\ensuremath{\mathbb{Z}}}

% bold letters
\newcommand{\bA}{\ensuremath{\mathbf{A}}}
\newcommand{\bB}{\ensuremath{\mathbf{B}}}
\newcommand{\bC}{\ensuremath{\mathbf{C}}}
\newcommand{\bD}{\ensuremath{\mathbf{D}}}
\newcommand{\bE}{\ensuremath{\mathbf{E}}}
\newcommand{\bF}{\ensuremath{\mathbf{F}}}
\newcommand{\bG}{\ensuremath{\mathbf{G}}}
\newcommand{\bH}{\ensuremath{\mathbf{H}}}
\newcommand{\bI}{\ensuremath{\mathbf{I}}}
\newcommand{\bJ}{\ensuremath{\mathbf{J}}}
\newcommand{\bK}{\ensuremath{\mathbf{K}}}
\newcommand{\bL}{\ensuremath{\mathbf{L}}}
\newcommand{\bM}{\ensuremath{\mathbf{M}}}
\newcommand{\bN}{\ensuremath{\mathbf{N}}}
\newcommand{\bO}{\ensuremath{\mathbf{O}}}
\newcommand{\bP}{\ensuremath{\mathbf{P}}}
\newcommand{\bQ}{\ensuremath{\mathbf{Q}}}
\newcommand{\bR}{\ensuremath{\mathbf{R}}}
\newcommand{\bS}{\ensuremath{\mathbf{S}}}
\newcommand{\bT}{\ensuremath{\mathbf{T}}}
\newcommand{\bU}{\ensuremath{\mathbf{U}}}
\newcommand{\bV}{\ensuremath{\mathbf{V}}}
\newcommand{\bW}{\ensuremath{\mathbf{W}}}
\newcommand{\bX}{\ensuremath{\mathbf{X}}}
\newcommand{\bY}{\ensuremath{\mathbf{Y}}}
\newcommand{\bZ}{\ensuremath{\mathbf{Z}}}

% fraktur letters
\newcommand{\fa}{\ensuremath{\mathfrak{a}}}
\newcommand{\fb}{\ensuremath{\mathfrak{b}}}
\newcommand{\fc}{\ensuremath{\mathfrak{c}}}
\newcommand{\fd}{\ensuremath{\mathfrak{d}}}
\newcommand{\fe}{\ensuremath{\mathfrak{e}}}
\newcommand{\ff}{\ensuremath{\mathfrak{f}}}
\newcommand{\fg}{\ensuremath{\mathfrak{g}}}
\newcommand{\fh}{\ensuremath{\mathfrak{h}}}
\newcommand{\fj}{\ensuremath{\mathfrak{j}}}
\newcommand{\fk}{\ensuremath{\mathfrak{k}}}
\newcommand{\fl}{\ensuremath{\mathfrak{l}}}
\newcommand{\fm}{\ensuremath{\mathfrak{m}}}
\newcommand{\fn}{\ensuremath{\mathfrak{n}}}
\newcommand{\fo}{\ensuremath{\mathfrak{o}}}
\newcommand{\fp}{\ensuremath{\mathfrak{p}}}
\newcommand{\fq}{\ensuremath{\mathfrak{q}}}
\newcommand{\fr}{\ensuremath{\mathfrak{r}}}
\newcommand{\fs}{\ensuremath{\mathfrak{s}}}
\newcommand{\ft}{\ensuremath{\mathfrak{t}}}
\newcommand{\fu}{\ensuremath{\mathfrak{u}}}
\newcommand{\fv}{\ensuremath{\mathfrak{v}}}
\newcommand{\fw}{\ensuremath{\mathfrak{w}}}
\newcommand{\fx}{\ensuremath{\mathfrak{x}}}
\newcommand{\fy}{\ensuremath{\mathfrak{y}}}
\newcommand{\fz}{\ensuremath{\mathfrak{z}}}

% fraktur letters
\newcommand{\fA}{\ensuremath{\mathfrak{A}}}
\newcommand{\fB}{\ensuremath{\mathfrak{B}}}
\newcommand{\fC}{\ensuremath{\mathfrak{C}}}

\mdef\fahat{\hat{\fa}}

% Underline letters
\newcommand{\uA}{\ensuremath{\underline{A}}}
\newcommand{\uB}{\ensuremath{\underline{B}}}
\newcommand{\uC}{\ensuremath{\underline{C}}}
\newcommand{\uD}{\ensuremath{\underline{D}}}
\newcommand{\uE}{\ensuremath{\underline{E}}}
\newcommand{\uF}{\ensuremath{\underline{F}}}
\newcommand{\uG}{\ensuremath{\underline{G}}}
\newcommand{\uH}{\ensuremath{\underline{H}}}
\newcommand{\uI}{\ensuremath{\underline{I}}}
\newcommand{\uJ}{\ensuremath{\underline{J}}}
\newcommand{\uK}{\ensuremath{\underline{K}}}
\newcommand{\uL}{\ensuremath{\underline{L}}}
\newcommand{\uM}{\ensuremath{\underline{M}}}
\newcommand{\uN}{\ensuremath{\underline{N}}}
\newcommand{\uO}{\ensuremath{\underline{O}}}
\newcommand{\uP}{\ensuremath{\underline{P}}}
\newcommand{\uQ}{\ensuremath{\underline{Q}}}
\newcommand{\uR}{\ensuremath{\underline{R}}}
\newcommand{\uS}{\ensuremath{\underline{S}}}
\newcommand{\uT}{\ensuremath{\underline{T}}}
\newcommand{\uU}{\ensuremath{\underline{U}}}
\newcommand{\uV}{\ensuremath{\underline{V}}}
\newcommand{\uW}{\ensuremath{\underline{W}}}
\newcommand{\uX}{\ensuremath{\underline{X}}}
\newcommand{\uY}{\ensuremath{\underline{Y}}}
\newcommand{\uZ}{\ensuremath{\underline{Z}}}

% bars
\newcommand{\Abar}{\ensuremath{\overline{A}}}
\newcommand{\Bbar}{\ensuremath{\overline{B}}}
\newcommand{\Cbar}{\ensuremath{\overline{C}}}
\newcommand{\Dbar}{\ensuremath{\overline{D}}}
\newcommand{\Ebar}{\ensuremath{\overline{E}}}
\newcommand{\Fbar}{\ensuremath{\overline{F}}}
\newcommand{\Gbar}{\ensuremath{\overline{G}}}
\newcommand{\Hbar}{\ensuremath{\overline{H}}}
\newcommand{\Ibar}{\ensuremath{\overline{I}}}
\newcommand{\Jbar}{\ensuremath{\overline{J}}}
\newcommand{\Kbar}{\ensuremath{\overline{K}}}
\newcommand{\Lbar}{\ensuremath{\overline{L}}}
\newcommand{\Mbar}{\ensuremath{\overline{M}}}
\newcommand{\Nbar}{\ensuremath{\overline{N}}}
\newcommand{\Obar}{\ensuremath{\overline{O}}}
\newcommand{\Pbar}{\ensuremath{\overline{P}}}
\newcommand{\Qbar}{\ensuremath{\overline{Q}}}
\newcommand{\Rbar}{\ensuremath{\overline{R}}}
\newcommand{\Sbar}{\ensuremath{\overline{S}}}
\newcommand{\Tbar}{\ensuremath{\overline{T}}}
\newcommand{\Ubar}{\ensuremath{\overline{U}}}
\newcommand{\Vbar}{\ensuremath{\overline{V}}}
\newcommand{\Wbar}{\ensuremath{\overline{W}}}
\newcommand{\Xbar}{\ensuremath{\overline{X}}}
\newcommand{\Ybar}{\ensuremath{\overline{Y}}}
\newcommand{\Zbar}{\ensuremath{\overline{Z}}}
\newcommand{\abar}{\ensuremath{\overline{a}}}
\newcommand{\bbar}{\ensuremath{\overline{b}}}
\newcommand{\cbar}{\ensuremath{\overline{c}}}
\newcommand{\dbar}{\ensuremath{\overline{d}}}
\newcommand{\ebar}{\ensuremath{\overline{e}}}
\newcommand{\fbar}{\ensuremath{\overline{f}}}
\newcommand{\gbar}{\ensuremath{\overline{g}}}
%\newcommand{\hbar}{\ensuremath{\overline{h}}} % whoops, \hbar means something else!
\newcommand{\ibar}{\ensuremath{\overline{\imath}}}
\newcommand{\jbar}{\ensuremath{\overline{\jmath}}}
\newcommand{\kbar}{\ensuremath{\overline{k}}}
\newcommand{\lbar}{\ensuremath{\overline{l}}}
\newcommand{\mbar}{\ensuremath{\overline{m}}}
\newcommand{\nbar}{\ensuremath{\overline{n}}}
%\newcommand{\obar}{\ensuremath{\overline{o}}}
\newcommand{\pbar}{\ensuremath{\overline{p}}}
\newcommand{\qbar}{\ensuremath{\overline{q}}}
\newcommand{\rbar}{\ensuremath{\overline{r}}}
\newcommand{\sbar}{\ensuremath{\overline{s}}}
\newcommand{\tbar}{\ensuremath{\overline{t}}}
\newcommand{\ubar}{\ensuremath{\overline{u}}}
\newcommand{\vbar}{\ensuremath{\overline{v}}}
\newcommand{\wbar}{\ensuremath{\overline{w}}}
\newcommand{\xbar}{\ensuremath{\overline{x}}}
\newcommand{\ybar}{\ensuremath{\overline{y}}}
\newcommand{\zbar}{\ensuremath{\overline{z}}}

% tildes
\newcommand{\Atil}{\ensuremath{\widetilde{A}}}
\newcommand{\Btil}{\ensuremath{\widetilde{B}}}
\newcommand{\Ctil}{\ensuremath{\widetilde{C}}}
\newcommand{\Dtil}{\ensuremath{\widetilde{D}}}
\newcommand{\Etil}{\ensuremath{\widetilde{E}}}
\newcommand{\Ftil}{\ensuremath{\widetilde{F}}}
\newcommand{\Gtil}{\ensuremath{\widetilde{G}}}
\newcommand{\Htil}{\ensuremath{\widetilde{H}}}
\newcommand{\Itil}{\ensuremath{\widetilde{I}}}
\newcommand{\Jtil}{\ensuremath{\widetilde{J}}}
\newcommand{\Ktil}{\ensuremath{\widetilde{K}}}
\newcommand{\Ltil}{\ensuremath{\widetilde{L}}}
\newcommand{\Mtil}{\ensuremath{\widetilde{M}}}
\newcommand{\Ntil}{\ensuremath{\widetilde{N}}}
\newcommand{\Otil}{\ensuremath{\widetilde{O}}}
\newcommand{\Ptil}{\ensuremath{\widetilde{P}}}
\newcommand{\Qtil}{\ensuremath{\widetilde{Q}}}
\newcommand{\Rtil}{\ensuremath{\widetilde{R}}}
\newcommand{\Stil}{\ensuremath{\widetilde{S}}}
\newcommand{\Ttil}{\ensuremath{\widetilde{T}}}
\newcommand{\Util}{\ensuremath{\widetilde{U}}}
\newcommand{\Vtil}{\ensuremath{\widetilde{V}}}
\newcommand{\Wtil}{\ensuremath{\widetilde{W}}}
\newcommand{\Xtil}{\ensuremath{\widetilde{X}}}
\newcommand{\Ytil}{\ensuremath{\widetilde{Y}}}
\newcommand{\Ztil}{\ensuremath{\widetilde{Z}}}
\newcommand{\atil}{\ensuremath{\widetilde{a}}}
\newcommand{\btil}{\ensuremath{\widetilde{b}}}
\newcommand{\ctil}{\ensuremath{\widetilde{c}}}
\newcommand{\dtil}{\ensuremath{\widetilde{d}}}
\newcommand{\etil}{\ensuremath{\widetilde{e}}}
\newcommand{\ftil}{\ensuremath{\widetilde{f}}}
\newcommand{\gtil}{\ensuremath{\widetilde{g}}}
\newcommand{\htil}{\ensuremath{\widetilde{h}}}
\newcommand{\itil}{\ensuremath{\widetilde{\imath}}}
\newcommand{\jtil}{\ensuremath{\widetilde{\jmath}}}
\newcommand{\ktil}{\ensuremath{\widetilde{k}}}
\newcommand{\ltil}{\ensuremath{\widetilde{l}}}
\newcommand{\mtil}{\ensuremath{\widetilde{m}}}
\newcommand{\ntil}{\ensuremath{\widetilde{n}}}
\newcommand{\otil}{\ensuremath{\widetilde{o}}}
\newcommand{\ptil}{\ensuremath{\widetilde{p}}}
\newcommand{\qtil}{\ensuremath{\widetilde{q}}}
\newcommand{\rtil}{\ensuremath{\widetilde{r}}}
\newcommand{\stil}{\ensuremath{\widetilde{s}}}
\newcommand{\ttil}{\ensuremath{\widetilde{t}}}
\newcommand{\util}{\ensuremath{\widetilde{u}}}
\newcommand{\vtil}{\ensuremath{\widetilde{v}}}
\newcommand{\wtil}{\ensuremath{\widetilde{w}}}
\newcommand{\xtil}{\ensuremath{\widetilde{x}}}
\newcommand{\ytil}{\ensuremath{\widetilde{y}}}
\newcommand{\ztil}{\ensuremath{\widetilde{z}}}

% Hats
\newcommand{\Ahat}{\ensuremath{\widehat{A}}}
\newcommand{\Bhat}{\ensuremath{\widehat{B}}}
\newcommand{\Chat}{\ensuremath{\widehat{C}}}
\newcommand{\Dhat}{\ensuremath{\widehat{D}}}
\newcommand{\Ehat}{\ensuremath{\widehat{E}}}
\newcommand{\Fhat}{\ensuremath{\widehat{F}}}
\newcommand{\Ghat}{\ensuremath{\widehat{G}}}
\newcommand{\Hhat}{\ensuremath{\widehat{H}}}
\newcommand{\Ihat}{\ensuremath{\widehat{I}}}
\newcommand{\Jhat}{\ensuremath{\widehat{J}}}
\newcommand{\Khat}{\ensuremath{\widehat{K}}}
\newcommand{\Lhat}{\ensuremath{\widehat{L}}}
\newcommand{\Mhat}{\ensuremath{\widehat{M}}}
\newcommand{\Nhat}{\ensuremath{\widehat{N}}}
\newcommand{\Ohat}{\ensuremath{\widehat{O}}}
\newcommand{\Phat}{\ensuremath{\widehat{P}}}
\newcommand{\Qhat}{\ensuremath{\widehat{Q}}}
\newcommand{\Rhat}{\ensuremath{\widehat{R}}}
\newcommand{\Shat}{\ensuremath{\widehat{S}}}
\newcommand{\That}{\ensuremath{\widehat{T}}}
\newcommand{\Uhat}{\ensuremath{\widehat{U}}}
\newcommand{\Vhat}{\ensuremath{\widehat{V}}}
\newcommand{\What}{\ensuremath{\widehat{W}}}
\newcommand{\Xhat}{\ensuremath{\widehat{X}}}
\newcommand{\Yhat}{\ensuremath{\widehat{Y}}}
\newcommand{\Zhat}{\ensuremath{\widehat{Z}}}
\newcommand{\ahat}{\ensuremath{\hat{a}}}
\newcommand{\bhat}{\ensuremath{\hat{b}}}
\newcommand{\chat}{\ensuremath{\hat{c}}}
\newcommand{\dhat}{\ensuremath{\hat{d}}}
\newcommand{\ehat}{\ensuremath{\hat{e}}}
\newcommand{\fhat}{\ensuremath{\hat{f}}}
\newcommand{\ghat}{\ensuremath{\hat{g}}}
\newcommand{\hhat}{\ensuremath{\hat{h}}}
\newcommand{\ihat}{\ensuremath{\hat{\imath}}}
\newcommand{\jhat}{\ensuremath{\hat{\jmath}}}
\newcommand{\khat}{\ensuremath{\hat{k}}}
\newcommand{\lhat}{\ensuremath{\hat{l}}}
\newcommand{\mhat}{\ensuremath{\hat{m}}}
\newcommand{\nhat}{\ensuremath{\hat{n}}}
\newcommand{\ohat}{\ensuremath{\hat{o}}}
\newcommand{\phat}{\ensuremath{\hat{p}}}
\newcommand{\qhat}{\ensuremath{\hat{q}}}
\newcommand{\rhat}{\ensuremath{\hat{r}}}
\newcommand{\shat}{\ensuremath{\hat{s}}}
\newcommand{\that}{\ensuremath{\hat{t}}}
\newcommand{\uhat}{\ensuremath{\hat{u}}}
\newcommand{\vhat}{\ensuremath{\hat{v}}}
\newcommand{\what}{\ensuremath{\hat{w}}}
\newcommand{\xhat}{\ensuremath{\hat{x}}}
\newcommand{\yhat}{\ensuremath{\hat{y}}}
\newcommand{\zhat}{\ensuremath{\hat{z}}}

%% FONTS AND DECORATION FOR GREEK LETTERS

%% the package `mathbbol' gives us blackboard bold greek and numbers,
%% but it does it by redefining \mathbb to use a different font, so that
%% all the other \mathbb letters look different too.  Here we import the
%% font with bb greek and numbers, but assign it a different name,
%% \mathbbb, so as not to replace the usual one.
\DeclareSymbolFont{bbold}{U}{bbold}{m}{n}
\DeclareSymbolFontAlphabet{\mathbbb}{bbold}
\newcommand{\bbDelta}{\ensuremath{\mathbbb{\Delta}}}
\newcommand{\bbone}{\ensuremath{\mathbbb{1}}}
\newcommand{\bbtwo}{\ensuremath{\mathbbb{2}}}
\newcommand{\bbthree}{\ensuremath{\mathbbb{3}}}

% greek with bars
\newcommand{\albar}{\ensuremath{\overline{\alpha}}}
\newcommand{\bebar}{\ensuremath{\overline{\beta}}}
\newcommand{\gmbar}{\ensuremath{\overline{\gamma}}}
\newcommand{\debar}{\ensuremath{\overline{\delta}}}
\newcommand{\phibar}{\ensuremath{\overline{\varphi}}}
\newcommand{\psibar}{\ensuremath{\overline{\psi}}}
\newcommand{\xibar}{\ensuremath{\overline{\xi}}}
\newcommand{\ombar}{\ensuremath{\overline{\omega}}}

% greek with hats
\newcommand{\alhat}{\ensuremath{\hat{\alpha}}}
\newcommand{\behat}{\ensuremath{\hat{\beta}}}
\newcommand{\gmhat}{\ensuremath{\hat{\gamma}}}
\newcommand{\dehat}{\ensuremath{\hat{\delta}}}

% greek with checks
\newcommand{\alchk}{\ensuremath{\check{\alpha}}}
\newcommand{\bechk}{\ensuremath{\check{\beta}}}
\newcommand{\gmchk}{\ensuremath{\check{\gamma}}}
\newcommand{\dechk}{\ensuremath{\check{\delta}}}

% greek with tildes
\newcommand{\altil}{\ensuremath{\widetilde{\alpha}}}
\newcommand{\betil}{\ensuremath{\widetilde{\beta}}}
\newcommand{\gmtil}{\ensuremath{\widetilde{\gamma}}}
\newcommand{\phitil}{\ensuremath{\widetilde{\varphi}}}
\newcommand{\psitil}{\ensuremath{\widetilde{\psi}}}
\newcommand{\xitil}{\ensuremath{\widetilde{\xi}}}
\newcommand{\omtil}{\ensuremath{\widetilde{\omega}}}

% MISCELLANEOUS SYMBOLS
\mdef\del{\partial}
\mdef\delbar{\overline{\partial}}
\let\sm\wedge
\newcommand{\dd}[1]{\ensuremath{\frac{\partial}{\partial {#1}}}}
\newcommand{\inv}{^{-1}}
\newcommand{\dual}{^{\vee}}
\mdef\hf{\textstyle\frac{1}{2}}
\mdef\thrd{\textstyle\frac{1}{3}}
\mdef\qtr{\textstyle\frac{1}{4}}
\let\meet\wedge
\let\join\vee
\let\dn\downarrow
\newcommand{\op}{^{\mathit{op}}}
\newcommand{\co}{^{\mathit{co}}}
\newcommand{\coop}{^{\mathit{coop}}}
\let\adj\dashv
\SelectTips{cm}{}
\newdir{ >}{{}*!/-10pt/@{>}}    % extra spacing for tail arrows in XYpic
\newcommand{\pushoutcorner}[1][dr]{\save*!/#1+1.2pc/#1:(1,-1)@^{|-}\restore}
\newcommand{\pullbackcorner}[1][dr]{\save*!/#1-1.2pc/#1:(-1,1)@^{|-}\restore}
\let\iso\cong
\let\eqv\simeq
\let\cng\equiv
\mdef\Id{\mathrm{Id}}
\mdef\id{\mathrm{id}}
\alwaysmath{ell}
\alwaysmath{infty}
\alwaysmath{odot}
\def\frc#1/#2.{\frac{#1}{#2}}   % \frc x^2+1 / x^2-1 .
\mdef\ten{\mathrel{\otimes}}
\mdef\bigten{\bigotimes}
\mdef\sqten{\mathrel{\boxtimes}}
\def\pow(#1,#2){\mathop{\pitchfork}(#1,#2)} % powers and
\def\cpw{\mathop{\odot}}                    % copowers
\newcommand{\mathid}{\mbox{id}}
\newcommand{\cat}[1]{\ensuremath{\mathbf{#1}}}


%% OPERATORS
\DeclareMathOperator\lan{Lan}
\DeclareMathOperator\ran{Ran}
\DeclareMathOperator\colim{colim}
\DeclareMathOperator\coeq{coeq}
\DeclareMathOperator\eq{eq}
\DeclareMathOperator\Tot{Tot}
\DeclareMathOperator\cosk{cosk}
\DeclareMathOperator\sk{sk}
\DeclareMathOperator\im{im}
\DeclareMathOperator\Spec{Spec}
\DeclareMathOperator\Ho{Ho}
\DeclareMathOperator\Aut{Aut}
\DeclareMathOperator\End{End}
\DeclareMathOperator\Hom{Hom}
\DeclareMathOperator\Map{Map}

%% TIKZ ARROWS AND HIGHER CELLS
\makeatletter
\def\slashedarrowfill@#1#2#3#4#5{%
  $\m@th\thickmuskip0mu\medmuskip\thickmuskip\thinmuskip\thickmuskip
   \relax#5#1\mkern-7mu%
   \cleaders\hbox{$#5\mkern-2mu#2\mkern-2mu$}\hfill
   \mathclap{#3}\mathclap{#2}%
   \cleaders\hbox{$#5\mkern-2mu#2\mkern-2mu$}\hfill
   \mkern-7mu#4$%
}

\def\Rightslashedarrowfill@{%
  \slashedarrowfill@\Relbar\Relbar\Mapstochar\Rightarrow}
\newcommand\xslashedRightarrow[2][]{%
  \ext@arrow 0055{\Rightslashedarrowfill@}{#1}{#2}}
\def\hTo{\xslashedRightarrow{}}
\def\hToo{\xslashedRightarrow{\quad}}
\let\xhTo\xslashedRightarrow

\pagestyle{empty}

\newcommand{\Rightthreecell}{\RRightarrow}
\newcommand{\Rtwocell}{\Rightarrow}

\tikzstyle{doubletick}=[-implies, double equal sign distance, postaction={decorate},decoration={markings,mark=at position .5 with {\draw[-] (0,-0.1) -- (0,0.1);}}]

\tikzstyle{darrow}=[-implies, double equal sign distance]

\tikzstyle{doubleeq}=[double equal sign distance]


%% ARROWS
% \to already exists
\newcommand{\too}[1][]{\ensuremath{\overset{#1}{\longrightarrow}}}
\newcommand{\ot}{\ensuremath{\leftarrow}}
\newcommand{\oot}[1][]{\ensuremath{\overset{#1}{\longleftarrow}}}
\let\toot\rightleftarrows
\let\otto\leftrightarrows
\let\Impl\Rightarrow
\let\imp\Rightarrow
\let\toto\rightrightarrows
\let\into\hookrightarrow
\let\xinto\xhookrightarrow
\mdef\we{\overset{\sim}{\longrightarrow}}
\mdef\leftwe{\overset{\sim}{\longleftarrow}}
\let\mono\rightarrowtail
\let\leftmono\leftarrowtail
\let\cof\rightarrowtail
\let\leftcof\leftarrowtail
\let\epi\twoheadrightarrow
\let\leftepi\twoheadleftarrow
\let\fib\twoheadrightarrow
\let\leftfib\twoheadleftarrow
\let\cohto\rightsquigarrow
\let\maps\colon
\newcommand{\spam}{\,:\!}       % \maps for left arrows

\newsavebox{\DDownarrowbox}
\savebox{\DDownarrowbox}{\tikz[scale=1.5]{\draw[-implies,double equal
sign distance] (0,.1) -- (0,-.1); \draw (0,.1) -- (0,-.1);}}
\newcommand{\DDownarrow}{\mathrel{\raisebox{-.2em}{\usebox{\DDownarrowbox}}}}

\newsavebox{\RRightarrowbox}
\savebox{\RRightarrowbox}{\tikz[scale=1.5]{\draw[-implies,double equal
sign distance] (-.1,0) -- (.1,0); \draw (-.1,0) -- (.1,0);}}
\newcommand{\RRightarrow}{\mathrel{\raisebox{.2em}{\usebox{\RRightarrowbox}}}}

%\newsavebox{\Rightslashedarrowbox}
%\savebox{\Rightslashedarrowbox}{\tikz[scale=1.5]{\draw[Rightslashedarrow{}] (-.1,0) -- (1,0);}}
%\newcommand{\Rightslashedarrow}{\mathrel{\raisebox{-.2em}%{\usebox{\Rightslashedarrowbox}}}}


%% EXTENSIBLE ARROWS
\let\xto\xrightarrow
\let\xot\xleftarrow
% See Voss' Mathmode.tex for instructions on how to create new
% extensible arrows.
\def\rightarrowtailfill@{\arrowfill@{\Yright\joinrel\relbar}\relbar\rightarrow}
\newcommand\xrightarrowtail[2][]{\ext@arrow 0055{\rightarrowtailfill@}{#1}{#2}}
\let\xmono\xrightarrowtail
\let\xcof\xrightarrowtail
\def\twoheadrightarrowfill@{\arrowfill@{\relbar\joinrel\relbar}\relbar\twoheadrightarrow}
\newcommand\xtwoheadrightarrow[2][]{\ext@arrow 0055{\twoheadrightarrowfill@}{#1}{#2}}
\let\xepi\xtwoheadrightarrow
\let\xfib\xtwoheadrightarrow
% Let's leave the left-going ones until I need them.

%% EXTENSIBLE SLASHED ARROWS
% Making extensible slashed arrows, by modifying the underlying AMS code.
% Arguments are:
% 1 = arrowhead on the left (\relbar or \Relbar if none)
% 2 = fill character (usually \relbar or \Relbar)
% 3 = slash character (such as \mapstochar or \Mapstochar)
% 4 = arrowhead on the left (\relbar or \Relbar if none)
% 5 = display mode (\displaystyle etc)
\def\slashedarrowfill@#1#2#3#4#5{%
  $\m@th\thickmuskip0mu\medmuskip\thickmuskip\thinmuskip\thickmuskip
   \relax#5#1\mkern-7mu%
   \cleaders\hbox{$#5\mkern-2mu#2\mkern-2mu$}\hfill
   \mathclap{#3}\mathclap{#2}%
   \cleaders\hbox{$#5\mkern-2mu#2\mkern-2mu$}\hfill
   \mkern-7mu#4$%
}
% Here's the idea: \<slashed>arrowfill@ should be a box containing
% some stretchable space that is the "middle of the arrow".  This
% space is created as a "leader" using \cleader<thing>\hfill, which
% fills an \hfill of space with copies of <thing>.  Here instead of
% just one \cleader, we use two, with the slash in between (and an
% extra copy of the filler, to avoid extra space around the slash).
\def\rightslashedarrowfill@{%
  \slashedarrowfill@\relbar\relbar\mapstochar\rightarrow}
\newcommand\xslashedrightarrow[2][]{%
  \ext@arrow 0055{\rightslashedarrowfill@}{#1}{#2}}
\mdef\hto{\xslashedrightarrow{}}
\mdef\htoo{\xslashedrightarrow{\quad}}
\let\xhto\xslashedrightarrow

%% To get a slashed arrow in XYpic, do
% \[\xymatrix{A \ar[r]|-@{|} & B}\]

% ISOMORPHISMS
\def\xiso#1{\mathrel{\mathrlap{\smash{\xto[\smash{\raisebox{1.3mm}{$\scriptstyle\sim$}}]{#1}}}\hphantom{\xto{#1}}}}
\def\toiso{\xto{\smash{\raisebox{-.5mm}{$\scriptstyle\sim$}}}}

% SHADOWS
\def\shvar#1#2{{\ensuremath{%
  \hspace{1mm}\makebox[-1mm]{$#1\langle$}\makebox[0mm]{$#1\langle$}\hspace{1mm}%
  {#2}%
  \makebox[1mm]{$#1\rangle$}\makebox[0mm]{$#1\rangle$}%
}}}
\def\sh{\shvar{}}
\def\scriptsh{\shvar{\scriptstyle}}
\def\bigsh{\shvar{\big}}
\def\Bigsh{\shvar{\Big}}
\def\biggsh{\shvar{\bigg}}
\def\Biggsh{\shvar{\Bigg}}

%HIGHER CELLS



% THEOREM-TYPE ENVIRONMENTS, hacked to
%% (a) number all with the same numbers, and
%% (b) have the right names for autoref
\def\defthm#1#2{%
  \newtheorem{#1}{#2}[section]%
  \expandafter\def\csname #1autorefname\endcsname{#2}%
  \expandafter\let\csname c@#1\endcsname\c@thm}
\newtheorem{thm}{Theorem}[section]
\newcommand{\thmautorefname}{Theorem}
\defthm{cor}{Corollary}
\defthm{prop}{Proposition}
\defthm{lem}{Lemma}
\defthm{sch}{Scholium}
\defthm{assume}{Assumption}
\defthm{claim}{Claim}
\defthm{conj}{Conjecture}
\defthm{hyp}{Hypothesis}
\defthm{fact}{Fact}
\theoremstyle{definition}
\defthm{defn}{Definition}
\defthm{notn}{Notation}
\theoremstyle{remark}
\defthm{rmk}{Remark}
\defthm{eg}{Example}
\defthm{egs}{Examples}
\defthm{ex}{Exercise}
\defthm{ceg}{Counterexample}

% How to get QED symbols inside equations at the end of the statements
% of theorems.  AMS LaTeX knows how to do this inside equations at the
% end of *proofs* with \qedhere, and at the end of the statement of a
% theorem with \qed (meaning no proof will be given), but it can't
% seem to combine the two.  Use this instead.
\def\thmqedhere{\expandafter\csname\csname @currenvir\endcsname @qed\endcsname}

% Number numbered lists as (i), (ii), ...
\renewcommand{\theenumi}{(\roman{enumi})}
\renewcommand{\labelenumi}{\theenumi}

%% Labeling that keeps track of theorem-type names.  Superseded by
%% autoref from hyperref, as above, but we keep this in case we are
%% using a journal style file that is incompatible with hyperref.
% 
% \ifx\SK@label\undefined\let\SK@label\label\fi
% \let\your@thm\@thm
% \def\@thm#1#2#3{\gdef\currthmtype{#3}\your@thm{#1}{#2}{#3}}
% \def\xlabel#1{{\let\your@currentlabel\@currentlabel\def\@currentlabel
% {\currthmtype~\your@currentlabel}
% \SK@label{#1@}}\label{#1}}
% \def\xref#1{\ref{#1@}}

% Also number formulas with the theorem counter
\let\c@equation\c@thm
\numberwithin{equation}{section}

% Only show numbers for equations that are actually referenced (or
% whose tags are specified manually) - uses mathtools.
\mathtoolsset{showonlyrefs,showmanualtags}

%% Fix enumerate spacing with paralist.  This has two parts:
%%   1. enable mixing of "old spacing" lists with those adjusted by paralist
%%   2. allow us to specify a number based on which to adjust the spacing
%% For the first, use \killspacingtrue when you want the spacing
%% adjusted, then \killspacingfalse to turn adjustment off.  For the
%% second, use \maxenum=14 to set the widest number you want the
%% spacing to be calculated with.
\newlength\oldleftmargini       % save old spacing
\newlength\oldleftmarginii
\newlength\oldleftmarginiii
\newlength\oldleftmarginiv
\newlength\oldleftmarginv
\newlength\oldleftmarginvi
\newcount\maxenum
\maxenum=7
\newif\ifkillspacing
\def\@adjust@enum@labelwidth{%
  \advance\@listdepth by 1\relax
  \ifkillspacing                % do the paralist thing
    \csname c@\@enumctr\endcsname\maxenum
    \settowidth{\@tempdima}{%
      \csname label\@enumctr\endcsname\hspace{\labelsep}}%
    \csname leftmargin\romannumeral\@listdepth\endcsname
      \@tempdima
  \else                         % otherwise, reset it
    \csname fixspacing\romannumeral\@listdepth\endcsname
  \fi
  \advance\@listdepth by -1\relax}
% these commands, one for each enum level (I couldn't get a generic
% one to work), test whether oldleftmargin has been set yet, and if
% not, set it from leftmargin; otherwise, they reset leftmargin to
% it.  Just setting oldleftmargin to leftmargin in the preamble
% doesn't seem to work.
\def\fixspacingi{\ifnum\oldleftmargini=0\setlength\oldleftmargini\leftmargini\else\setlength\leftmargini\oldleftmargini\fi}
\def\fixspacingii{\ifnum\oldleftmarginii=0\setlength\oldleftmarginii\leftmarginii\else\setlength\leftmarginii\oldleftmarginii\fi}
\def\fixspacingiii{\ifnum\oldleftmarginiii=0\setlength\oldleftmarginiii\leftmarginiii\else\setlength\leftmarginiii\oldleftmarginiii\fi}
\def\fixspacingiv{\ifnum\oldleftmarginiv=0\setlength\oldleftmarginiv\leftmarginiv\else\setlength\leftmarginiv\oldleftmarginiv\fi}
\def\fixspacingv{\ifnum\oldleftmarginv=0\setlength\oldleftmarginv\leftmarginv\else\setlength\leftmarginv\oldleftmarginv\fi}
\def\fixspacingvi{\ifnum\oldleftmarginvi=0\setlength\oldleftmarginvi\leftmarginvi\else\setlength\leftmarginvi\oldleftmarginvi\fi}

%% Fix paralist references, so that we can refer to (1) instead of
%% just 1.
\def\pl@label#1#2{%
  \edef\pl@the{\noexpand#1{\@enumctr}}%
  \pl@lab\expandafter{\the\pl@lab\csname yourthe\@enumctr\endcsname}%
  \advance\@tempcnta1
  \pl@loop}
\def\@enumlabel@#1[#2]{%
  \@plmylabeltrue
  \@tempcnta0
  \pl@lab{}%
  \let\pl@the\pl@qmark
  \expandafter\pl@loop\@gobble#2\@@@
  \ifnum\@tempcnta=1\else
    \PackageWarning{paralist}{Incorrect label; no or multiple
      counters.\MessageBreak The label is: \@gobble#2}%
  \fi
  \expandafter\edef\csname label\@enumctr\endcsname{\the\pl@lab}%
  \expandafter\edef\csname the\@enumctr\endcsname{\the\pl@lab}%
  \expandafter\let\csname yourthe\@enumctr\endcsname\pl@the
  #1}


% GREEK LETTERS, ETC.
\alwaysmath{alpha}
\alwaysmath{beta}
\alwaysmath{gamma}
\alwaysmath{Gamma}
\alwaysmath{delta}
\alwaysmath{Delta}
\alwaysmath{epsilon}
\mdef\ep{\varepsilon}
\alwaysmath{zeta}
\alwaysmath{eta}
\alwaysmath{theta}
\alwaysmath{Theta}
\alwaysmath{iota}
\alwaysmath{kappa}
\alwaysmath{lambda}
\alwaysmath{Lambda}
\alwaysmath{mu}
\alwaysmath{nu}
\alwaysmath{xi}
\alwaysmath{pi}
\alwaysmath{rho}
\alwaysmath{sigma}
\alwaysmath{Sigma}
\alwaysmath{tau}
\alwaysmath{upsilon}
\alwaysmath{Upsilon}
\alwaysmath{phi}
\alwaysmath{Pi}
\alwaysmath{Phi}
\mdef\ph{\varphi}
\alwaysmath{chi}
\alwaysmath{psi}
\alwaysmath{Psi}
\alwaysmath{omega}
\alwaysmath{Omega}
\let\al\alpha
\let\be\beta
\let\gm\gamma
\let\Gm\Gamma
\let\de\delta
\let\De\Delta
\let\si\sigma
\let\Si\Sigma
\let\om\omega
\let\ka\kappa
\let\la\lambda
\let\La\Lambda
\let\ze\zeta
\let\th\theta
\let\Th\Theta
\let\vth\vartheta

\makeatother

% Tikz styles
\tikzstyle{tickarrow}=[->,postaction={decorate},decoration={markings,mark=at position .5 with {\draw[-] (0,-0.1) -- (0,0.1);}},line width=0.50]

% Local Variables:
% mode: latex
% TeX-master: ""
% End:

\begin{document}


\begin{equation*}\hspace{-2cm}
\begin{aligned}
\begin{tikzpicture}[xscale=3, yscale=1.5]
\node (t0) at (0,2) {\small $\tens(f\times I_B)$};
\node (t1) at (1,2) {\small $\tens(f\times fI_A)$};
\node (t15) at (2,2) {\small $\tens (f \times f) (\transid \times I)$};
\node (t2) at (3,2) {\small $f\tens(\id \times I_A)$};
\node (t3) at (4,2) {\small $f $};
\node (t4) at (5,2) {\small $g $};
\node (m0) at (0,1) {\small $\tens(\transid \times I_B)f$};
\node (b3) at (5,1) {\small $\tens (\transid \times I_B)g$};
%%%%%%%%%%%%%
\draw[doubleloose] (t0) to node[above]{$\substack{\looseid(\looseid \times \iota)}$} (t1);
\draw[doubleeq] (t1) to  (t15);
\draw[doubleloose] (t15) to node[above]{$\substack{\chi (\looseid \times \looseid)}$} (t2);
\draw[doubleloose] (t2) to node[above]{$\substack{\looseid r}$} (t3);
\draw[doubleloose] (t3) to node[above]{$\substack{\beta}$} (t4);
\draw[doubleloose] (m0) to node[above]{$\substack{r \looseid}$} (t3);
\draw[doubleloose] (m0) to node[above]{$\substack{\looseid_{\tens}(\beta \times \looseid_I)}$} (b3);
\draw[doubleloose] (b3) to node[right]{$\substack{r \looseid}$} (t4);
\draw[doubletighteq] (t0) to (m0);
\node at (1,1.5) {$\substack{\DDownarrow \delta^f}$};
\node at (4,1.5) {$\substack{\iso}$};
\end{tikzpicture}
\end{aligned}\hspace{-2cm}
\end{equation*}
\begin{equation}\label{eq:mon2cell2}
=
\end{equation}
\begin{equation*}\hspace{-2cm}
\begin{aligned}
\begin{tikzpicture}[xscale=3, yscale=1.5]
\node (04) at (0,4) {\small $\tens(f\times I_B)$};
\node (14) at (1,5.5) {\small $\tens(f\times f I_A)$};
\node (154) at (1.5,6) {\small $\tens(f \times f) (\transid \times I_A)$};
\node (24) at (3.5,6.5) {\small $f \tens(\transid \times I_A)$};
\node (34) at (4.5,5.5) {\small $f $};
\node (44) at (5,4) {\small $g $};
%%%%%%
\node (11) at (.5,2) {\small $\tens(g\times  I_B)$};
\node (22) at (2,2.5) {\small $\tens(g\times g I_A)$};
\node (32) at (3,3) {\small $\tens(g\times g) (\transid \times  I_A)$};
\node (33) at (4,4) {\small $g \tens(\transid \times I_A)$};
%%%%%%%
\node (00) at (0,1) {\small $\tens(\transid \times I_B)f$};
\node (10) at (5,1) {\small $\tens(\transid \times  I_B)g$};
%%%%%%%
\draw[doubleloose] (04) to node[above, xshift=-10pt]{$\substack{\looseid (\looseid \times \iota_f)}$} (14);
\draw[doubleeq] (14) to (154);
\draw[doubleloose] (154) to node[above]{$\substack{\chi \looseid}$} (24);
\draw[doubleloose] (24) to node[above]{$\substack{\looseid r}$} (34);
\draw[doubleloose] (34) to node[above, xshift=3pt]{$\substack{\beta}$} (44);
%%%%%%%
\draw[doubleloose] (24) to node[right]{$\substack{\beta \looseid \looseid}$} (33);
\draw[doubleloose] (33) to node[above]{$\substack{\looseid r}$} (44);
%%%%
\draw[doubleloose] (154) to node[below, xshift=-20pt]{$\substack{\looseid (\beta \times \beta)\looseid }$} (32);
\draw[doubleeq] (22) to (32);
\draw[doubleloose] (32) to node[above, xshift=-10pt]{$\substack{\chi \looseid}$} (33);
%%%%%%
\draw[doubleloose] (04) to node[right]{$\substack{\looseid (\beta \times \looseid)}$} (11);
\draw[doubleloose] (11) to node[above, xshift=-10pt]{$\substack{\looseid (\looseid \times \iota_g) }$} (22);
%%%%%%
\draw[doubleloose] (00) to node[above]{$\substack{\looseid  \looseid \beta}$} (10);
\draw[doubleloose] (10) to node[left]{$\substack{r \looseid }$} (44);
\draw[doubleloose] (14) to node[right]{$\substack{ \looseid (\beta \times \beta \looseid)}$} (22);
%%%%%%
\draw[doubletighteq] (04) to (00);
\draw[doubletighteq] (11) to (10);
%%%%%%%%
\node at (4.25,5) {$\substack{\DDownarrow \iso }$};
\node at (3.25,5.75) {$\substack{\DDownarrow\iso }$};
\node at (3,3.75) {$\substack{\DDownarrow\iso }$};
\node at (2,3.75) {$\substack{\DDownarrow\iso }$};
\node at (3,5) {$\substack{\DDownarrow \Pi^{\beta}\tightid}$};
\node at (1,4.75) {$\substack {\DDownarrow \iso }$};
\node at (0.8,3.75) {$\substack {\DDownarrow \tightid_{\looseid} (\tightid \times M^{\beta} ) }$};
\node at (.75,2.75) {$\substack {\DDownarrow \iso }$};
\node at (1,1.5) {$\substack{\DDownarrow \iso}$};
\node at (4,2.5) {$\substack{\DDownarrow \delta^g}$};
%%%%%
%\draw[doubleloose] (11) to node[below, xshift=3pt]{$\substack{S(\delta)\tightid}$} (44);
%\draw[doubleloose] (11) to[in=240, out=-20] node[below, xshift=-3pt]{$\substack{T(\delta)\tightid}$} (44);
%%%%%
\draw[doubleloose] (154) to[in=115, out=0]  node[below, xshift=3pt]{$\substack{S(\Pi)\tightid}$} (33);
\draw[doubleloose] (154) to[in=180
, out=-45] node[below, xshift=-3pt]{$\substack{T(\Pi)\tightid}$} (33);
%%%%%%
\draw[doubleloose] (04) to[in=135, out=20]  node[above, xshift=3pt, yshift=5pt]{$\substack{\looseid (\beta \times S(M)) }$} (22);
\draw[doubleloose] (04) to[in=180
, out=-45] node[above, xshift=3pt,yshift=5pt]{$\substack{\looseid (\beta \times T(M))}$} (22);
\end{tikzpicture}
\end{aligned}\hspace{-2cm}
\end{equation*}

\end{document} 
 \newpage

%
\documentclass[12pt]{ociamthesis}
\usepackage{tikz}
\newcommand{\id}{\mathrm{id}}
\begin{document}

\begin{equation*}
\begin{aligned}
\begin{tikzpicture}[xscale=3.5, yscale=1.5]
\node (04) at (0,4) {\scriptsize$ \tens( \tens \times \transid)(f \times f \times f)$};
\node (14) at (1,4) {\scriptsize $ \tens(f \tens \times f)$};
\node (24) at (2,4) {\scriptsize $f \tens(\tens \times \transid)$};
\node (34) at (3,4) {\scriptsize $f\tens (\transid \times \tens)$};
\node (44) at (4,4) {\scriptsize $g \tens (\transid \times \tens)$};
\node (03) at (0,3) {\scriptsize $\tens( \tens \times \transid)(f \times f \times f)$};
\node (13) at (1,3) {\scriptsize $\tens( \transid \times \tens)(f \times f \times f)$};
\node (23) at (2,3) {\scriptsize $\tens (f \times f \tens)$};
\node (33) at (3,3) {\scriptsize $f \tens (\transid \times  \tens)$};
\node (43) at (4,3) {\scriptsize $g \tens (\transid \times  \tens)$};
\node (02) at (0,2) {\scriptsize $\tens( \tens \times \transid)(f \times f \times f)$};
\node (12) at (1,2) {\scriptsize $\tens( \transid \times \tens)(f \times f \times f)$};
\node (22) at (2,2) {\scriptsize $\tens (f \times f \tens)$};
\node (32) at (3,2) {\scriptsize $\tens (g \times g \tens)$};
\node (42) at (4,2) {\scriptsize $g \tens (\transid \times  \tens)$};
%%%%%%%
\node (01) at (0,1) {\scriptsize $\tens( \tens \times \transid)(f \times f \times f)$};
\node (11) at (1,1) {\scriptsize $\tens( \transid \times \tens)(f \times f \times f)$};
\node (21) at (2,1) {\scriptsize $\tens (\transid \times \tens) (g \times g \times g)$};
\node (31) at (3,1) {\scriptsize $\tens (g \times g \tens)$};
\node (41) at (4,1) {\scriptsize $g \tens (\transid \times  \tens)$};
%%%%%%%
\node (00) at (0,0) {\scriptsize $\tens( \tens \times \transid)(f \times f \times f)$};
\node (10) at (1,0) {\scriptsize $\tens( \transid \times \tens)(g \times g \times g)$};
\node (20) at (2,0) {\scriptsize $\tens (\transid \times \tens) (g \times g \times g)$};
\node (30) at (3,0) {\scriptsize $\tens (g \times g \tens)$};
\node (40) at (4,0) {\scriptsize $g \tens (\transid \times  \tens)$};
%%%%%%%
\draw[doubleloose] (04) to node[above]{\scriptsize $\looseid_{\tens}(\chi_f \times \looseid_f)$} (14);
\draw[doubleloose] (14) to node[above]{\scriptsize $\chi_f \looseid_{\tens \times \transid}$} (24);
\draw[doubleloose] (24) to node[above]{\scriptsize $\looseid_{f}\alpha$} (34);
\draw[doubleloose] (34) to node[above]{\scriptsize $\beta \looseid_{\tens} \looseid_{\transid \times \tens}$} (44);
%%%%%%%%
\draw[doubleloose] (03) to node[above]{\scriptsize $\alpha \looseid_{f \times f \times f}$} (13);
\draw[doubleloose] (13) to node[above]{\scriptsize $\looseid_{\tens} (\looseid_{\transid} \times \chi_f)$} (23);
\draw[doubleloose] (23) to node[above]{\scriptsize $\chi_f \looseid_{\transid \times \tens}$} (33);
\draw[doubleloose] (33) to node[above]{\scriptsize $\beta \looseid_{\tens} \looseid_{\transid \times \tens}$} (43);
%%%%%%%%
\draw[doubleloose] (12) to node[above]{\scriptsize $\looseid_{\tens} (\looseid_{\transid} \times \chi_f)$} (22);
\draw[doubleloose] (22) to node[above]{\scriptsize $\looseid_{\tens} (\beta \times \beta) \looseid_{\transid \times \tens}$} (32);
\draw[doubleloose] (32) to node[above]{\scriptsize $\chi_g \looseid_{\transid \times \tens}$} (42);
%%%%%%%%
\draw[doubleloose] (01) to node[above]{\scriptsize $\alpha \looseid_{f \times f \times f}$} (11);
\draw[doubleloose] (11) to node[above]{\scriptsize $\looseid_{\tens} \looseid_{\transid \times \tens} (\beta \times \beta \times \beta)$} (21);
\draw[doubleloose] (21) to node[above]{\scriptsize $\looseid_{\tens} (\looseid_{\transid} \times \chi_g)$} (31);
%%%%%%%%
\draw[doubleloose] (00) to node[above]{\scriptsize $\looseid_{\tens} (\looseid_{\tens} \times \looseid_{\transid})(\beta \times \beta \times \beta)$} (10);
\draw[doubleloose] (10) to node[above]{\scriptsize $\alpha \looseid_{g \times g \times g}$} (20);
\draw[doubleloose] (20) to node[above]{\scriptsize $\looseid_{\tens} (\looseid_{\transid} \times \chi_g)$} (30);
\draw[doubleloose] (30) to node[above]{\scriptsize $\chi_g \looseid_{\transid \times \tens}$} (40);
%%%%%%%%
\draw[doubletighteq] (04) to (03);
\draw[doubletighteq] (34) to (33);
\draw[doubletighteq] (44) to (43);
%%%%%%%%%
\draw[doubletighteq] (03) to (02);
\draw[doubletighteq] (13) to (12);
\draw[doubletighteq] (23) to (22);
\draw[doubletighteq] (43) to (42);
%%%%%%%%%
\draw[doubletighteq] (02) to (01);
\draw[doubletighteq] (12) to (11);
\draw[doubletighteq] (32) to (31);
\draw[doubletighteq] (42) to (41);
%%%%%%%%%
\draw[doubletighteq] (01) to (00);
\draw[doubletighteq] (21) to (20);
\draw[doubletighteq] (31) to (30);
\draw[doubletighteq] (41) to (40);
%%%%%%%%%
\node at (1.5,3.5) {\scriptsize $\DDownarrow \omega^f$};
\node at (3.5,3.5) {\scriptsize $=$};
\node at (0.5,2) {\scriptsize $=$};
\node at (1.5,2.5) {\scriptsize $=$};
\node at (3,2.5) {\scriptsize $\DDownarrow \overline{\Pi^{\beta} \tightid_{\looseid_{\transid \times \tens}}}$};
\node at (2,1.5) {\scriptsize $\DDownarrow \overline{\tightid_{\looseid} ({\horl}^{-1} \horr \times \Pi^{\beta})}$};
\node at (3.5,1) {\scriptsize $=$};
\node at (1,0.5) {\scriptsize $\DDownarrow ({\horl}^{-1} \verc \horr) ({\horr}^{-1} \verc \horl)$};
\node at (2.5,0.5) {\scriptsize $=$};
\end{tikzpicture}
\end{aligned}
\end{equation*}
\begin{equation}\label{eq:mon2cell3}
=
\end{equation}
\begin{equation*}
\begin{aligned}
\begin{tikzpicture}[xscale=3.5, yscale=1.5]
\node (04) at (0,4) {\scriptsize $\tens( \tens \times \transid)(f \times f \times f)$};
\node (14) at (1,4) {\scriptsize $\tens(f \tens \times f)$};
\node (24) at (2,4) {\scriptsize $f \tens(\tens \times \transid)$};
\node (34) at (3,4) {\scriptsize $f\tens (\transid \times \tens)$};
\node (44) at (4,4) {\scriptsize $g \tens (\transid \times \tens)$};
%%%%%%%%%
\node (03) at (0,3) {\scriptsize $\tens( \tens \times \transid)(f \times f \times f)$};
\node (13) at (1,3) {\scriptsize $\tens( f \tens \times f)$};
\node (23) at (2,3) {\scriptsize $f \tens ( \tens \times \transid)$};
\node (33) at (3,3) {\scriptsize $g \tens (\tens \times  \transid)$};
\node (43) at (4,3) {\scriptsize $g \tens (\transid \times \tens)$};
\node (02) at (0,2) {\scriptsize $\tens( \tens \times \transid)(f \times f \times f)$};
\node (12) at (1,2) {\scriptsize $\tens(f \tens \times f )$};
\node (22) at (2,2) {\scriptsize $\tens (g \tens \times g)$};
\node (32) at (3,2) {\scriptsize $g \tens (\tens \times \transid)$};
\node (42) at (4,2) {\scriptsize $g \tens (\transid \times  \tens)$};
%%%%%%%
\node (01) at (0,1) {\scriptsize $\tens( \tens \times \transid)(f \times f \times f)$};
\node (11) at (1,1) {\scriptsize $\tens( \tens \times \transid)(g \times g \times g)$};
\node (21) at (2,1) {\scriptsize $\tens (g \tens \times g)$};
\node (31) at (3,1) {\scriptsize $g \tens ( \tens \times \transid )$};
\node (41) at (4,1) {\scriptsize $g \tens (\transid \times  \tens)$};
%%%%%%%
\node (00) at (0,0) {\scriptsize $\tens( \tens \times \transid)(f \times f \times f)$};
\node (10) at (1,0) {\scriptsize $\tens( \transid \times \tens)(g \times g \times g)$};
\node (20) at (2,0) {\scriptsize $\tens (\transid \times \tens) (g \times g \times g)$};
\node (30) at (3,0) {\scriptsize $\tens (g \times g \tens)$};
\node (40) at (4,0) {\scriptsize $g \tens (\transid \times  \tens)$};
%%%%%%%
\draw[doubleloose] (04) to node[above]{\scriptsize $\looseid_{\tens}(\chi_f \times \looseid_f)$} (14);
\draw[doubleloose] (14) to node[above]{\scriptsize $\chi_f (\looseid_{\tens \times \transid})$} (24);
\draw[doubleloose] (24) to node[above]{\scriptsize $\looseid_{f}\alpha$} (34);
\draw[doubleloose] (34) to node[above]{\scriptsize $\beta \looseid_{\tens} \looseid_{\transid \times \tens}$} (44);
%%%%%%%%
\draw[doubleloose] (13) to node[above]{\scriptsize $\chi_f \looseid_{\tens \times \transid}$} (23);
\draw[doubleloose] (23) to node[above]{\scriptsize $\beta \looseid_{\tens} \looseid_{\tens \times \transid}$} (33);
\draw[doubleloose] (33) to node[above]{\scriptsize $ \looseid_{g} \alpha$} (43);
%%%%%%%%
\draw[doubleloose] (02) to node[above]{\scriptsize $\looseid_{\tens}(\chi_f \times \looseid_f)$} (12);
\draw[doubleloose] (12) to node[above]{\scriptsize $\looseid_{\tens} (\beta \times \beta) \looseid_{\tens \times \transid}$} (22);
\draw[doubleloose] (22) to node[above]{\scriptsize $\chi_g \looseid_{\tens \times \transid}$} (32);
%%%%%%%%
\draw[doubleloose] (01) to node[above]{\scriptsize $\looseid_{\tens} \looseid_{\tens \times \transid}(\beta \times \beta \times \beta)$} (11);
\draw[doubleloose] (11) to node[above]{\scriptsize $\looseid_{\tens} (\chi_g \times \looseid_g)$} (21);
\draw[doubleloose] (21) to node[above]{\scriptsize $\chi_g \looseid_{\tens \times \transid}$} (31);
\draw[doubleloose] (31) to node[above]{\scriptsize $\looseid_g \alpha$} (41);
%%%%%%%%
\draw[doubleloose] (00) to node[above]{\scriptsize $\looseid_{\tens} (\looseid_{\tens \times \transid})(\beta \times \beta \times \beta)$} (10);
\draw[doubleloose] (10) to node[above]{\scriptsize $\alpha \looseid_{g \times g \times g}$} (20);
\draw[doubleloose] (20) to node[above]{\scriptsize $\looseid_{\tens} (\looseid_{\transid} \times \chi_g)$} (30);
\draw[doubleloose] (30) to node[above]{\scriptsize $\chi_g \looseid_{\transid \times \tens}$} (40);
%%%%%%%%
\draw[doubletighteq] (04) to (03);
\draw[doubletighteq] (14) to (13);
\draw[doubletighteq] (24) to (23);
\draw[doubletighteq] (44) to (43);
%%%%%%%%%
\draw[doubletighteq] (03) to (02);
\draw[doubletighteq] (13) to (12);
\draw[doubletighteq] (33) to (32);
\draw[doubletighteq] (43) to (42);
%%%%%%%%%
\draw[doubletighteq] (02) to (01);
\draw[doubletighteq] (22) to (21);
\draw[doubletighteq] (32) to (31);
\draw[doubletighteq] (42) to (41);
%%%%%%%%%
\draw[doubletighteq] (01) to (00);
\draw[doubletighteq] (11) to (10);
\draw[doubletighteq] (41) to (40);
%%%%%%%%%
\node at (.5,3) {\scriptsize $=$};
\node at (1.5,3.5) {\scriptsize $=$};
\node at (3,3.5) {\scriptsize $\DDownarrow ({\horl}^{-1} \verc \horr) ({\horr}^{-1} \verc \horl)$};
\node at (2,2.5) {\scriptsize $\DDownarrow \overline{\Pi^{\beta} \tightid_{\looseid_{\tens \times \transid}}}$};
\node at (2.5,1.5) {\scriptsize $=$};
\node at (.5,.5) {\scriptsize $=$};
\node at (2.5,0.5) {\scriptsize $\DDownarrow \omega^g$};
\node at (1,1.5) {\scriptsize $\DDownarrow \overline{\tightid_{\looseid_{\tens}} (\Pi^{\beta} \times  {\horr}^{-1} \verc \horl)}$};
\node at (3.5,2) {\scriptsize $=$};
\end{tikzpicture}
\end{aligned}
\end{equation*}

\end{document} 
 \newpage


%%
\documentclass[12pt]{ociamthesis}
\usepackage{tikz}
\newcommand{\id}{\mathrm{id}}
\begin{document}


\begin{equation}\label{eq:br2cell}
\begin{aligned}
\begin{tikzpicture}[xscale=3, yscale=1.5]
\node (03) at (0,3) {\small $\tens(f \times f)$};
\node (13) at (1,3) {\small $\tens \tau (f \times f)$};
\node (23) at (2,3) {\small $\tens(f \times f) \tau$};
\node (33) at (3,3) {\small $f \tens \tau$};
\node (43) at (4,3) {\small $g \tens \tau$};
%%%%%%
\node (02) at (0,2) {\small $\tens(f \times f)$};
\node (12) at (1,2) {\small $f \tens$};
\node (32) at (3,2) {\small $f \tens \tau$};
\node (42) at (4,2) {\small $g \tens \tau$};
%%%%%% 
\node (01) at (0,1) {\small $\tens(f \times f)$};
\node (11) at (1,1) {\small $f \tens$};
\node (31) at (3,1) {\small $g \tens $};
\node (41) at (4,1) {\small $g \tens \tau$};
%%%%%%%
\node (00) at (0,0) {\small $\tens(f \times f)$};
\node (10) at (1,0) {\small $\tens(g \times g)$};
\node (30) at (3,0) {\small $g \tens$};
\node (40) at (4,0) {\small $g \tens \tau$};
%%%%%%%
\draw[doubleloose] (03) to node[above]{\small $\sigma \looseid_{f \times f}$} (13);
\draw[double] (13) to (23);
\draw[doubleloose] (23) to node[above]{\small $\chi \looseid_{\tau}$} (33);
\draw[doubleloose] (33) to node[above]{\small $\beta \looseid_{\tens \tau}$} (43);
%%%%
\draw[doubleloose] (02) to node[above]{\small $\chi$} (12);
\draw[doubleloose] (12) to node[above]{\small $\looseid_f \sigma$} (32);
\draw[doubleloose] (32) to node[above]{\small $\beta \looseid_{\tens \tau}$} (42);
%%%%%%
\draw[doubleloose] (01) to node[above]{\small $\chi$} (11);
\draw[doubleloose] (11) to node[above]{\small $\beta \looseid_{\tens}$} (31);
\draw[doubleloose] (31) to node[above]{\small $ \looseid_g \sigma$} (41);
%%%%%%
\draw[doubleloose] (00) to node[above]{\small $\looseid_{\tens} (\beta \times \beta)$} (10);
\draw[doubleloose] (10) to node[above]{\small $\chi $} (30);
\draw[doubleloose] (30) to node[above]{\small $\looseid_g \sigma $} (40);
%%%%%%
\draw[doubletighteq] (03) to (02);
\draw[doubletighteq] (33) to (32);
\draw[doubletighteq] (43) to (42);
%%%%%%
\draw[doubletighteq] (02) to (01);
\draw[doubletighteq] (12) to (11);
\draw[doubletighteq] (42) to (41);
%%%%%%
\draw[doubletighteq] (01) to (00);
\draw[doubletighteq] (31) to (30);
\draw[doubletighteq] (41) to (40);
%%%%%%%%
\node at (1.5,2.5) {\small $\DDownarrow u$};
\node at (3.5,2.5) {\small $=$};
\node at (.5,1.5) {\small $=$};
\node at (2.5,1.5) {\small $\iso$};
\node at (1.5,.5) {\small $\DDownarrow \Pi^{\beta}$};
\node at (3.5,.5) {\small $=$};
\end{tikzpicture}
\end{aligned}
\end{equation}
\[=\]
\begin{equation*}
\begin{aligned}
\begin{tikzpicture}[xscale=3, yscale=1.5]
\node (03) at (0,3) {\small $\tens(f \times f)$};
\node (13) at (1,3) {\small $\tens \tau (f \times f)$};
\node (23) at (2,3) {\small $\tens(f \times f) \tau$};
\node (33) at (3,3) {\small $f \tens \tau$};
\node (43) at (4,3) {\small $g \tens \tau$};
%%%%%%
\node (02) at (0,2) {\small $\tens(f \times f)$};
\node (12) at (1,2) {\small $\tens \tau (f \times f)$};
\node (22) at (2,2) {\small $\tens (f \times f) \tau $};
\node (32) at (3,2) {\small $\tens (g \times g) \tau$};
\node (42) at (4,2) {\small $g \tens \tau$};
%%%%%% 
\node (01) at (0,1) {\small $\tens(f \times f)$};
\node (11) at (1,1) {\small $\tens (g \times g)$};
\node (21) at (2,1) {\small $\tens \tau (g \times g)$};
\node (31) at (3,1) {\small $\tens (g \times g) \tau $};
\node (41) at (4,1) {\small $g \tens \tau$};
%%%%%%%
\node (00) at (0,0) {\small $\tens(f \times f)$};
\node (10) at (1,0) {\small $\tens(g \times g)$};
\node (30) at (3,0) {\small $g \tens$};
\node (40) at (4,0) {\small $g \tens \tau$};
%%%%%%%
\draw[doubleloose] (03) to node[above]{\small $\sigma \looseid_{f \times f}$} (13);
\draw[double] (13) to (23);
\draw[doubleloose] (23) to node[above]{\small $\chi \looseid_{\tau}$} (33);
\draw[doubleloose] (33) to node[above]{\small $\beta \looseid_{\tens \tau}$} (43);
%%%%
\draw[doubleloose] (02) to node[above]{\small $\sigma \looseid_{f \times f}$} (12);
\draw[double] (12) to  (22);
\draw[doubleloose] (22) to node[above]{\small $\looseid_{\tens} (\beta \times \beta) \looseid_{\tau}$} (32);
\draw[doubleloose] (32) to node[above]{\small $\chi \looseid_{\tau}$} (42);
%%%%%%
\draw[doubleloose] (01) to node[above]{\small $\looseid_{\tens} (\beta \times \beta)$} (11);
\draw[doubleloose] (11) to node[above]{\small $\sigma \looseid_{g \times g}$} (21);
\draw[double] (21) to (31);
\draw[doubleloose] (31) to node[above]{\small $ \chi \looseid_{\tau}$} (41);
%%%%%%
\draw[doubleloose] (00) to node[above]{\small $\looseid_{\tens} (\beta \times \beta)$} (10);
\draw[doubleloose] (10) to node[above]{\small $\chi $} (30);
\draw[doubleloose] (30) to node[above]{\small $\looseid_g \sigma $} (40);
%%%%%%
\draw[doubletighteq] (03) to (02);
\draw[doubletighteq] (13) to (12);
\draw[doubletighteq] (43) to (42);
%%%%%%
\draw[doubletighteq] (02) to (01);
\draw[doubletighteq] (32) to (31);
\draw[doubletighteq] (42) to (41);
%%%%%%
\draw[doubletighteq] (01) to (00);
\draw[doubletighteq] (11) to (10);
\draw[doubletighteq] (41) to (40);
%%%%%%%%
\node at (.5,2.5) {\small $=$};
\node at (2.5,2.5) {\small $\DDownarrow \overline{\Pi^{\beta} \looseid_{\tau}}$};
\node at (1.5,1.5) {\small $\iso$};
\node at (3.5,1.5) {\small $=$};
\node at (.5,.5) {\small $=$};
\node at (2.5,.5) {\small $\DDownarrow u$};
\end{tikzpicture}
\end{aligned}
\end{equation*}

\end{document} 
 \newpage


%\subsubsection*{Strong Monoidal 2-cells}

%%
\documentclass[12pt]{ociamthesis}
\usepackage{tikz}
\newcommand{\id}{\mathrm{id}}
\begin{document}

\begin{equation}\label{eq:StrongMon2cell1}
    \begin{pic}[yscale=0.8, xscale=.5]
\draw[fill=blue, opacity = 0.5, draw=black] (0,4) -- (0,0) -- (3,0) -- (3,2) -- (2,2) -- (2,1) -- (1,1) -- (1,4) -- (0,4);
\draw[fill=orange, opacity = 0.5, draw=black] (1,4) -- (1,1) -- (2,1) -- (2,4) -- (1,4);
\draw[fill=green, opacity = 0.5, draw=black] (2,4) -- (2,2) -- (3,2) -- (3,3) -- (4,3) -- (4,0) -- (5,0) -- (5,4) -- (0,4);
\draw[fill=yellow, opacity = 0.5, draw=black] (3,0) -- (4,0) -- (4,3) -- (3,3) -- (3,0);
\node[morphism, minimum width=10mm] at (2.5,2) {$\Pi_{oplax}$};
\node[morphism, minimum width=10mm] at (1.5,1) {$\eta_{\chi}$};
\node[morphism, minimum width=10mm] at (3.5,3) {$\epsilon_{\chi}$};
\node at (2.6,-.2) {$\looseid(\beta \times \beta)  $};
\node at (4.4,-.2) {$\bar{\chi}$};
\node at (0.8,4.2) {$\bar{\chi} $};
\node at (2.2,4.2) {$ \beta \looseid$};
    \end{pic}
=
    \begin{pic}[yscale=0.8, xscale=.5]
\draw[fill=blue, opacity = 0.5, draw=black] (0,4) -- (0,0) -- (2,0) -- (2,4) -- (0,4);
\draw[fill=orange, opacity = 0.5, draw=black] (2,4) -- (2,2) -- (3,2) -- (3,4) -- (2,4);
\draw[fill=green, opacity = 0.5, draw=black] (3,4) -- (3,0) -- (5,0) -- (5,4) -- (3,4);
\draw[fill=yellow, opacity = 0.5, draw=black] (2,0) -- (3,0) -- (3,2) -- (2,2) -- (2,0);
\node[morphism, minimum width=10mm] at (2.5,2) {$\overline{\Pi_{lax}}$};
\node at (1.6,-.2) {$\looseid(\beta \times \beta)$};
\node at (3.4,-.2) {$\bar{\chi}$};
\node at (1.8,4.2) {$\beta \looseid$};
\node at (3.2,4.2) {$\bar{\chi}$};
    \end{pic}
    \end{equation}
\end{document} 

%%
\documentclass[12pt]{ociamthesis}
\usepackage{tikz}
\newcommand{\id}{\mathrm{id}}
\begin{document}

\begin{equation}\label{eq:StrongMon2cell2} 
        \begin{pic}[yscale=0.8, xscale=.5]
\draw[fill=yellow, opacity = 0.5, draw=black] (0,0) -- (1,0) -- (1,3) -- (2,3) -- (2,2) -- (3,2) -- (3,4) -- (0,4) -- (0,0);
\draw[fill=blue, opacity = 0.5, draw=black] (1,0) -- (2,0) -- (2,3) -- (1,3) -- (1,0);
\draw[fill=orange, opacity = 0.5, draw=black] (2,0) -- (5,0) -- (5,4) -- (4,4) -- (4,1) -- (3,1) -- (3,2) -- (2,2) -- (2,0);
\draw[fill=green, opacity = 0.5, draw=black] (3,4) -- (4,4) -- (4,1) -- (3,1) -- (3,4);
\node[morphism, minimum width=10mm] at (2.5,2) {$\overline{\Pi_{oplax}}$};
\node[morphism, minimum width=10mm] at (3.5,1) {$\eta_{\chi}$};
\node[morphism, minimum width=10mm] at (1.5,3) {$\epsilon_{\chi}$};
\node at (0.8,-.2) {$\chi$};
\node at (2.2,-.2) {$\beta \looseid $};
\node at (2.6,4.2) {$\chi$};
\node at (4.4,4.2) {$\looseid(\beta \times \beta)$};
    \end{pic}
=
    \begin{pic}[yscale=0.8, xscale=.5]
\draw[fill=yellow, opacity = 0.5, draw=black] (0,4) -- (0,0) -- (2,0) -- (2,4) -- (0,4);
\draw[fill=green, opacity = 0.5, draw=black] (2,4) -- (2,2) -- (3,2) -- (3,4) -- (2,4);
\draw[fill=orange, opacity = 0.5, draw=black] (3,4) -- (3,0) -- (5,0) -- (5,4) -- (3,4);
\draw[fill=blue, opacity = 0.5, draw=black] (2,0) -- (3,0) -- (3,2) -- (2,2) -- (2,0);
\node[morphism, minimum width=10mm] at (2.5,2) {\bf $\Pi_{lax}$};
\node at (1.8,-.2) {$\chi$};
\node at (3.2,-.2) {$\beta \looseid $};
\node at (1.6,4.2) {$\looseid(\beta \times \beta)$};
\node at (3.4,4.2) {$\chi$};
    \end{pic}
    \end{equation}
\end{document} 


\newpage
\subsubsection*{Monoidal Icon}

%
\documentclass[12pt]{ociamthesis}
\usepackage{tikz}
\newcommand{\id}{\mathrm{id}}
\begin{document}

\begin{equation}\label{eq:monicon1}
\begin{aligned}
\begin{tikzpicture}[xscale=4,yscale=2]
\node (02) at (0,2){$\tens(I \times f )i_2 $};
\node (12) at (1,2){$\tens(fI_A \times f)i_2 $};
\node (22) at (2,2){$f\tens(I_A \times \transid)i_2 $};
\node (32) at (3,2){$f $};
\node (01) at (0,1){$\tens(I_B \times \transid )i_2 f$};
\node (31) at (3,1){$f $};
\node (00) at (0,0){$\tens(I_B \times \transid )i_2 g$};
\node (30) at (3,0){$g$};
\draw[doubleloose] (02) to node[above]{$\looseid_{\tens} (\iota_f \times \looseid_f) \looseid_{i_2}$} (12);
\draw[doubleloose] (12) to node[above]{$\chi \looseid_{(I_A \times \transid)i_2}$} (22);
\draw[doubleloose] (22) to node[above]{$\looseid_f l$} (32);
\draw[doubleloose] (01) to node[above]{$l \looseid_f $} (31);
\draw[doubleloose] (00) to node[above]{$l \looseid_g $} (30);
\draw[=] (02) to node[left]{} (01);
\draw[=] (32) to node[left]{} (31);
\draw[doubletight] (01) to node[left]{$\tightid_{\tens} (\tightid_I\times \beta)\tightid_{i_2}$} (00);
\draw[doubletight] (31) to node[left]{$\beta$} (30);
\node at (1.5,1.5){$\DDownarrow \gamma^f$};
\node at (1.5,0.5){$\DDownarrow \looseid_{l}\tightid_{\beta}$};
\end{tikzpicture}
\end{aligned}
\end{equation}
\[=\]
\begin{equation*}
\begin{aligned}
\begin{tikzpicture}[xscale=4,yscale=2]
\node (02) at (0,2){$\tens(I_B \times f )i_2 $};
\node (12) at (1,2){$\tens(fI_A \times f)i_2 $};
\node (22) at (2,2){$f\tens(I_A \times \transid)i_2 $};
\node (32) at (3,2){$f $};
\node (01) at (0,1){$\tens(I_B \times g )i_2 $};
\node (11) at (1,1){$\tens(gI_A \times g)i_2 $};
\node (21) at (2,1){$g\tens(I_A \times \transid)i_2 $};
\node (31) at (3,1){$g $};
\node (00) at (0,0){$\tens(I_B \times \transid )i_2 g$};
\node (30) at (3,0){$g $};
\draw[doubleloose] (02) to node[above]{$\looseid_{\tens} (\iota_f \times \looseid_f) \looseid_{i_2}$} (12);
\draw[doubleloose] (12) to node[above]{$\chi \looseid_{(I_A \times \transid)i_2}$} (22);
\draw[doubleloose] (22) to node[above]{$\looseid_f l$} (32);
\draw[doubleloose] (01) to node[above]{$\looseid_{\tens} (\iota_g \times \looseid_g \looseid_{i_2}$} (11);
\draw[doubleloose] (11) to node[above]{$\chi \looseid_{(I_A \times \transid)i_2}$} (21);
\draw[doubleloose] (21) to node[above]{$\looseid_g l$} (31);
\draw[doubleloose] (00) to node[above]{$l \looseid_g $} (30);
\draw[doubletight] (02) to node[left]{$\tightid_{\tens(I\times \transid)}\beta$} (01);
\draw[doubletight] (12) to node[right]{$\tightid (\beta \times \beta) \tightid$} (11);
\draw[doubletight] (22) to node[left]{$\beta \tightid$} (21);
\draw[doubletight] (32) to node[left]{$\beta$} (31);
\draw[doubletight] (01) to node[left]{$\tightid_{\tens} (\tightid_I\times \beta)\tightid_{i_2}$} (00);
\draw[doubletight] (31) to node[left]{$\beta$} (30);
\node at (0.5,1.5){$\DDownarrow \tightid (N^{\beta} \times \tightid_{\looseid}) \tightid$};
\node at (1.5,1.5){$\DDownarrow \Sigma^{\beta} \tightid$};
\node at (2.5,1.5){$\DDownarrow \looseid_{\beta} \tightid_{l}$};
\node at (1.5,0.5){$\DDownarrow \gamma^g$};
\end{tikzpicture}
\end{aligned}
\end{equation*}


\end{document} 
\newpage

%
\documentclass[12pt]{ociamthesis}
\usepackage{tikz}
\newcommand{\id}{\mathrm{id}}
\begin{document}

\begin{equation*}\hspace{-2cm}
\begin{aligned}
\begin{tikzpicture}[xscale=4,yscale=2]
\node (02) at (0,2){$\tens(f \times I_B ) $};
\node (12) at (1,2){$\tens(f \times fI_A) $};
\node (22) at (2,2){$f\tens(\transid \times I_A) $};
\node (32) at (3,2){$f $};
\node (01) at (0,1){$\tens(\transid \times I_B) f$};
\node (31) at (3,1){$f $};
\node (00) at (0,0){$\tens(\transid \times I_B ) g$};
\node (30) at (3,0){$g $};
\draw[doubleloose] (02) to node[above]{$\looseid_{\tens} (\looseid_f \times \iota_f) $} (12);
\draw[doubleloose] (12) to node[above]{$\chi \looseid_{(\transid \times I_A)}$} (22);
\draw[doubleloose] (22) to node[above]{$\looseid_f r$} (32);
\draw[doubleloose] (01) to node[above]{$r \looseid_f $} (31);
\draw[doubleloose] (00) to node[above]{$r \looseid_g $} (30);
\draw[=] (02) to node[left]{} (01);
\draw[=] (32) to node[left]{} (31);
\draw[doubletight] (01) to node[left]{$\tightid_{\tens (\transid \times I)} \beta$} (00);
\draw[doubletight] (31) to node[left]{$\beta$} (30);
\node at (1.5,1.5){$\DDownarrow \delta^f$};
\node at (1.5,0.5){$\DDownarrow \looseid_{r}\tightid_{\beta}$};
\end{tikzpicture}
\end{aligned}\hspace{-2cm}
\end{equation*}
\begin{equation}\label{eq:monicon2}
  =
\end{equation}
\begin{equation*}\hspace{-2cm}
\begin{aligned}
\begin{tikzpicture}[xscale=4,yscale=2]
\node (02) at (0,2){$\tens(f \times I_B ) $};
\node (12) at (1,2){$\tens(f \times fI_A) $};
\node (22) at (2,2){$f\tens(\transid \times I_A) $};
\node (32) at (3,2){$f$};
\node (01) at (0,1){$\tens(g \times I_B) $};
\node (11) at (1,1){$\tens(g\times gI_A ) $};
\node (21) at (2,1){$g\tens(\transid \times I_A)  $};
\node (31) at (3,1){$g $};
\node (00) at (0,0){$\tens(\transid \times I_B ) g$};
\node (30) at (3,0){$g $};
\draw[doubleloose] (02) to node[above]{$\looseid_{\tens} (\looseid \times\iota_f) $} (12);
\draw[doubleloose] (12) to node[above]{$\chi \looseid_{(\transid \times I_A)}$} (22);
\draw[doubleloose] (22) to node[above]{$\looseid_f r$} (32);
\draw[doubleloose] (01) to node[above]{$\looseid_{\tens} (\looseid_g \times \iota_g) $} (11);
\draw[doubleloose] (11) to node[above]{$\chi \looseid_{(\transid \times I_A)}$} (21);
\draw[doubleloose] (21) to node[above]{$\looseid_g r$} (31);
\draw[doubleloose] (00) to node[above]{$r \looseid_g $} (30);
\draw[doubletight] (02) to node[left]{$\tightid_{\tens(\transid\times I)}\beta$} (01);
\draw[doubletight] (12) to node[right]{$\tightid (\beta \times \beta) \tightid$} (11);
\draw[doubletight] (22) to node[left]{$\beta \tightid$} (21);
\draw[doubletight] (32) to node[left]{$\beta$} (31);
\draw[doubletight] (01) to node[left]{$\tightid_{\tens} (\beta\times \tightid_I)$} (00);
\draw[doubletight] (31) to node[left]{$\beta$} (30);
\node at (0.5,1.5){$\DDownarrow \tightid (\tightid \times N^{\beta}) \tightid$};
\node at (1.5,1.5){$\DDownarrow \Sigma^{\beta} \tightid$};
\node at (2.5,1.5){$\DDownarrow \looseid_{\beta} \tightid_{r}$};
\node at (1.5,0.5){$\DDownarrow \delta^g$};
\end{tikzpicture}
\end{aligned}\hspace{-2cm}
\end{equation*}

\end{document} 
\newpage

%
\documentclass[12pt]{ociamthesis}
\usepackage{tikz}
\newcommand{\id}{\mathrm{id}}
\begin{document}
{\small
\begin{equation*}
\begin{aligned}
\begin{tikzpicture}[xscale=4,yscale=2]
\node (02) at (0,2){\scriptsize$\tens(\tens \times \transid)(f \times f\times f)$};
\node (12) at (1,2){$\scriptsize\tens(f\tens \times f) $};
\node (22) at (2,2){\scriptsize$f\tens(\tens \times \transid)$};
\node (32) at (3,2){\scriptsize$f\tens(\transid \times \tens)$};
\node (01) at (0,1){\scriptsize$\tens(\tens \times \transid)(f \times f\times f)$};
\node (11) at (1,1){\scriptsize$\tens(\transid \times \tens)(f \times f\times f)$};
\node (21) at (2,1){\scriptsize$\tens(f \times f\tens) $};
\node (31) at (3,1){\scriptsize$f \tens ( \transid \times \tens)$};
\node (00) at (0,0){\scriptsize$\tens(\tens \times \transid)(g \times g \times g \times g$};
\node (10) at (1,0){\scriptsize$\tens(\transid \times \tens)(g \times g\times g)$};
\node (20) at (2,0){\scriptsize$\tens(g \times g\tens) $};
\node (30) at (3,0){\scriptsize$g \tens ( \transid \times \tens)$};
\draw[doubleloose] (02) to node[above]{\scriptsize$\looseid_{\tens} (\chi \times \looseid_f)$} (12);
\draw[doubleloose] (12) to node[above]{\scriptsize$\chi \looseid_{(\tens \times \transid)}$} (22);
\draw[doubleloose] (22) to node[above]{\scriptsize$\looseid_{f} \alpha$} (32);
\draw[doubleloose] (01) to node[above]{\scriptsize$\alpha \looseid_{f\times f \times f}$} (11);
\draw[doubleloose] (11) to node[above]{\scriptsize$\looseid_{\tens} (\transid_f \times \chi)$} (21);
\draw[doubleloose] (21) to node[above]{\scriptsize$\chi \looseid_{\transid \times \tens}$} (31);
\draw[doubleloose] (00) to node[above]{\scriptsize$\alpha \looseid_{g\times g \times g} $} (10);
\draw[doubleloose] (10) to node[above]{\scriptsize$\looseid_{\tens} (\transid_g \times \chi)$} (20);
\draw[doubleloose] (20) to node[above]{\scriptsize$\chi \looseid_{\transid \times \tens}$} (30);
\draw[=] (02) to node[left]{} (01);
\draw[=] (32) to node[left]{} (31);
\draw[doubletight] (01) to node[left]{\scriptsize$\tightid (\beta\times \beta \times \beta)$} (00);
\draw[doubletight] (11) to node {\scriptsize$\tightid (\beta\times \beta \times \beta)$} (10);
\draw[doubletight] (21) to node {\scriptsize$\tightid (\beta\times \beta \tightid)$} (20);
\draw[doubletight] (31) to node[left]{\scriptsize$\beta \tightid$} (30);
\node at (1.5,1.5){\scriptsize$\DDownarrow \omega^f$};
\node at (0.5,0.5){\scriptsize$\DDownarrow \tightid_{\alpha} \looseid_{\beta \times \beta \times \beta}$};
\node at (1.5,0.5){\scriptsize$\DDownarrow \tightid (\tightid \times \Sigma^{\beta})$};
\node at (2.5,0.5){\scriptsize$\DDownarrow \Sigma^{\beta} \tightid$};
\end{tikzpicture}
\end{aligned}
\end{equation*}}
\begin{equation}\label{eq:monicon3}
=
\end{equation}
{\small
\begin{equation*}
\begin{aligned}
\begin{tikzpicture}[xscale=4,yscale=2]
\node (02) at (0,2){\scriptsize$\tens(\tens \times \transid)(f \times f\times f)$};
\node (12) at (1,2){\scriptsize$\tens(f\tens \times f) $};
\node (22) at (2,2){\scriptsize$f\tens(\tens \times \transid)$};
\node (32) at (3,2){\scriptsize$f\tens(\transid \times \tens)$};
\node (01) at (0,1){\scriptsize$\tens(\tens \times \transid)(g \times g\times g)$};
\node (11) at (1,1){\scriptsize$\tens(g\tens \times g)$};
\node (21) at (2,1){\scriptsize$g\tens(\tens \times \transid) $};
\node (31) at (3,1){\scriptsize$g \tens ( \transid \times \tens)$};
\node (00) at (0,0){\scriptsize$\tens(\tens \times \transid)(g \times g \times g \times g)$};
\node (10) at (1,0){\scriptsize$\tens(\transid \times \tens)(g \times g\times g)$};
\node (20) at (2,0){\scriptsize$\tens(g \times g\tens) $};
\node (30) at (3,0){\scriptsize$g \tens ( \transid \times \tens)$};
\draw[doubleloose] (02) to node[above]{\scriptsize$\looseid_{\tens} (\chi \times \looseid_f)$} (12);
\draw[doubleloose] (12) to node[above]{\scriptsize$\chi \looseid_{(\tens \times \transid)}$} (22);
\draw[doubleloose] (22) to node[above]{\scriptsize$\looseid_{f} \alpha$} (32);
\draw[doubleloose] (01) to node[above]{\scriptsize$\looseid_{\tens} (\chi \times \looseid_g)$} (11);
\draw[doubleloose] (11) to node[above]{\scriptsize$\chi \looseid_{(\tens \times \transid)}$} (21);
\draw[doubleloose] (21) to node[above]{\scriptsize$\looseid_{g} \alpha$} (31);
\draw[doubleloose] (00) to node[above]{\scriptsize$\alpha \looseid_{g\times g \times g} $} (10);
\draw[doubleloose] (10) to node[above]{\scriptsize$\looseid_{\tens} (\transid_g \times \chi)$} (20);
\draw[doubleloose] (20) to node[above]{\scriptsize$\chi \looseid_{\transid \times \tens}$} (30);
\draw[=] (01) to node[left]{} (00);
\draw[doubletight] (12) to node {\scriptsize$\tightid (\beta \tightid \times \beta)$} (11);
\draw[doubletight] (22) to node[left] {\scriptsize$\beta \tightid $} (21);
\draw[=] (31) to node[left]{} (30);
\draw[doubletight] (02) to node[left]{\scriptsize$\tightid (\beta\times \beta \times \beta)$} (01);
\draw[doubletight] (32) to node[left]{\scriptsize$\beta \tightid$} (31);
\node at (1.5,0.5){\scriptsize$\DDownarrow \omega^g$};
\node at (0.5,1.5){\scriptsize$\DDownarrow \tightid (\Sigma^{\beta} \times \tightid)$};
\node at (1.5,1.5){\scriptsize$\DDownarrow \Sigma^{\beta} \tightid$};
\node at (2.5,1.5){\scriptsize$\DDownarrow \looseid_{\beta} \tightid_{\alpha}$};
\end{tikzpicture}
\end{aligned}
\end{equation*}}

\end{document} 
\newpage

